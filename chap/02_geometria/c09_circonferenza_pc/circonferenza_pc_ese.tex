% (c) 2016 Daniele Zambelli daniele.zambelli@gmail.com
% (c) 2016 Elisabetta Campana

\section{Esercizi}

\subsection{Esercizi dei singoli paragrafi}

\subsubsection*{\numnameref{sec:circ_circtraslata}}
\begin{esercizio}\label{ese:}
 Dati il centro e il raggio, calcola l'equazione della circonferenza.
 \begin{enumeratea}
  \item  \(C \left (1; \quad 0 \right ); \quad r = 3\)
   \hfill [\(x^2 + y^2 -2x -8 = 0\)]
  \item  \(C \left (1; \quad -2 \right ); \quad r = -1\)
   \hfill [\(x^2 + y^2 -2x +4y +4 = 0\)]
  \item  \(C \left (-2; \quad -4 \right ); \quad r = -6\)
   \hfill [\(x^2 + y^2 +4x +8y -16 = 0\)]
  \item  \(C \left (-6; \quad -4 \right ); \quad r = -1\)
   \hfill [\(x^2 + y^2 +12x +8y +51 = 0\)]
  \item  \(C \left (1; \quad 0 \right ); \quad r = -1\)
   \hfill [\(x^2 + y^2 -2x  = 0\)]
  \item  \(C \left (-5; \quad -4 \right ); \quad r = -6\)
   \hfill [\(x^2 + y^2 +10x +8y +5 = 0\)]
  \item  \(C \left (4; \quad -4 \right ); \quad r = 3\)
   \hfill [\(x^2 + y^2 -8x +8y +23 = 0\)]
  \item  \(C \left (-1; \quad -6 \right ); \quad r = -2\)
   \hfill [\(x^2 + y^2 +2x +12y +33 = 0\)]
  \item  \(C \left (-6; \quad 2 \right ); \quad r = -6\)
   \hfill [\(x^2 + y^2 +12x -4y +4 = 0\)]
  \item  \(C \left (-1; \quad 4 \right ); \quad r = -5\)
   \hfill [\(x^2 + y^2 +2x -8y -8 = 0\)]
 \end{enumeratea}
\end{esercizio}


\begin{esercizio}\label{ese:}
 Calcola le coordinate del centro e il raggio della circonferenza data.
 \begin{enumeratea}
  \item  \(x^2 + y^2 +4x +10y +20 = 0\)
   \hfill [\(C \left (-2; \quad -5 \right ); \quad r = 3\)]
  \item  \(x^2 + y^2 +12x +10y +57 = 0\)
   \hfill [\(C \left (-6; \quad -5 \right ); \quad r = 2\)]
  \item  \(x^2 + y^2 +10x +8y +40 = 0\)
   \hfill [\(C \left (-5; \quad -4 \right ); \quad r = -1\)]
  \item  \(x^2 + y^2 +12y +35 = 0\)
   \hfill [\(C \left (0; \quad -6 \right ); \quad r = -1\)]
  \item  \(x^2 + y^2 +6x +10y +9 = 0\)
   \hfill [\(C \left (-3; \quad -5 \right ); \quad r = -5\)]
  \item  \(x^2 + y^2 -2x +10y +22 = 0\)
   \hfill [\(C \left (1; \quad -5 \right ); \quad r = 2\)]
  \item  \(x^2 + y^2 +8x -6y -11 = 0\)
   \hfill [\(C \left (-4; \quad 3 \right ); \quad r = -6\)]
  \item  \(x^2 + y^2 +10x -8y +32 = 0\)
   \hfill [\(C \left (-5; \quad 4 \right ); \quad r = 3\)]
  \item  \(x^2 + y^2 +2x +2y -2 = 0\)
   \hfill [\(C \left (-1; \quad -1 \right ); \quad r = 2\)]
  \item  \(x^2 + y^2 -4x +10y -7 = 0\)
   \hfill [\(C \left (2; \quad -5 \right ); \quad r = -6\)]
 \end{enumeratea}
\end{esercizio}


\subsubsection*{\numnameref{sec:circ_equazione}}

\begin{esercizio}\label{ese:}
 Dati il centro e un punto, calcola l'equazione della circonferenza.
 \begin{enumeratea}
  \item  \(C \left (-5; \quad -5 \right ); \quad P \left (-1; \quad 0 \right )\)
   \hfill [\(x^2 + y^2 +10x +10y +9 = 0\)]
  \item  \(C \left (4; \quad 5 \right ); \quad P \left (4; \quad -5 \right )\)
   \hfill [\(x^2 + y^2 -8x -10y -59 = 0\)]
  \item  \(C \left (0; \quad 5 \right ); \quad P \left (-3; \quad 2 \right )\)
   \hfill [\(x^2 + y^2 -10y +7 = 0\)]
  \item  \(C \left (4; \quad 2 \right ); \quad P \left (-5; \quad 5 \right )\)
   \hfill [\(x^2 + y^2 -8x -4y -70 = 0\)]
  \item  \(C \left (-3; \quad -2 \right ); \quad P \left (-4; \quad 0 \right )\)
   \hfill [\(x^2 + y^2 +6x +4y +8 = 0\)]
  \item  \(C \left (3; \quad 5 \right ); \quad P \left (-3; \quad -1 \right )\)
   \hfill [\(x^2 + y^2 -6x -10y -38 = 0\)]
  \item  \(C \left (0; \quad 4 \right ); \quad P \left (1; \quad -3 \right )\)
   \hfill [\(x^2 + y^2 -8y -34 = 0\)]
  \item  \(C \left (0; \quad -1 \right ); \quad P \left (-4; \quad 5 \right )\)
   \hfill [\(x^2 + y^2 +2y -51 = 0\)]
  \item  \(C \left (-2; \quad 5 \right ); \quad P \left (4; \quad -5 \right )\)
   \hfill [\(x^2 + y^2 +4x -10y -107 = 0\)]
  \item  \(C \left (-3; \quad 4 \right ); \quad P \left (2; \quad 5 \right )\)
   \hfill [\(x^2 + y^2 +6x -8y - = 0\)]
 \end{enumeratea}
\end{esercizio}


\begin{esercizio}\label{ese:}
 Dati gli estremi di un diametro, calcola l'equazione della circonferenza.
 \begin{enumeratea}
  \item  \(A \left (-2; \quad 5 \right ); \quad B \left (4; \quad 3 \right )\)
   \hfill [\(x^2 + y^2 -2x -8y +7 = 0\)]
  \item  \(A \left (0; \quad -5 \right ); \quad B \left (0; \quad 4 \right )\)
   \hfill [\(x^2 + y^2 +y -20 = 0\)]
  \item  \(A \left (-1; \quad 1 \right ); \quad B \left (-6; \quad 2 \right )\)
   \hfill [\(x^2 + y^2 +7x -3y +8 = 0\)]
  \item  \(A \left (-6; \quad 3 \right ); \quad B \left (5; \quad 2 \right )\)
   \hfill [\(x^2 + y^2 +x -5y -24 = 0\)]
  \item  \(A \left (-1; \quad 1 \right ); \quad B \left (0; \quad 0 \right )\)
   \hfill [\(x^2 + y^2 +x -y  = 0\)]
  \item  \(A \left (-4; \quad -6 \right ); \quad B \left (-5; \quad -1 \right 
)\)
   \hfill [\(x^2 + y^2 +9x +7y +26 = 0\)]
  \item  \(A \left (0; \quad -1 \right ); \quad B \left (0; \quad -4 \right )\)
   \hfill [\(x^2 + y^2 +5y +4 = 0\)]
  \item  \(A \left (-4; \quad -2 \right ); \quad B \left (-4; \quad 1 \right )\)
   \hfill [\(x^2 + y^2 +8x +y +14 = 0\)]
  \item  \(A \left (3; \quad 0 \right ); \quad B \left (1; \quad 3 \right )\)
   \hfill [\(x^2 + y^2 -4x -3y +3 = 0\)]
  \item  \(A \left (3; \quad -2 \right ); \quad B \left (4; \quad -4 \right )\)
   \hfill [\(x^2 + y^2 -7x +6y +20 = 0\)]
 \end{enumeratea}
\end{esercizio}


\begin{esercizio}\label{ese:}
 Calcola l'equazione della circonferenza passante per i tre punti, il suo 
centro e il suo raggio.
 \begin{enumeratea}
\item  \(A = \left (1;~-6 \right ); \quad B = \left (-1;~-5 \right ); \quad 
C = \left (-3;~-3 \right )\)

\hfill [\(x^2 + y^2 -7x -3y -48 = 0; \quad 
C \left (\frac{7}{2};~\frac{3}{2} \right ); \quad 
r = \sqrt{\frac{125}{2}}\)]

\item  \(A = \left (4;~5 \right ); \quad B = \left (-2;~1 \right ); \quad 
C = \left (-3;~-1 \right )\)

\hfill [\(x^2 + y^2 -\frac{23}{2}x +\frac{33}{4}y -\frac{145}{4} = 0; \quad 
C \left (\frac{23}{4};~-\frac{33}{8} \right ); \quad 
r = \sqrt{\frac{5525}{64}}\)]

\item  \(A = \left (4;~-3 \right ); \quad B = \left (-3;~-5 \right ); \quad 
C = \left (5;~1 \right )\)

\hfill [\(x^2 + y^2 +\frac{19}{13}x -\frac{8}{13}y -\frac{425}{13} = 0; \quad
C \left (-\frac{19}{26};~\frac{4}{13} \right ); \quad
r = \sqrt{\frac{22525}{676}}\)]

\item  \(A = \left (-3;~-1 \right ); \quad B = \left (2;~3 \right ); \quad 
C = \left (-2;~-5 \right )\)

\hfill [\(x^2 + y^2 -\frac{11}{3}x +\frac{23}{6}y -\frac{103}{6} = 0; \quad
C \left (\frac{11}{6};~-\frac{23}{12} \right ); \quad
r = \sqrt{\frac{3485}{144}}\)]

\item  \(A = \left (-1;~1 \right ); \quad B = \left (-6;~-1 \right ); \quad 
C = \left (-5;~2 \right )\)

\hfill [\(x^2 + y^2 +\frac{89}{13}x +\frac{5}{13}y +\frac{58}{13} = 0; \quad
C \left (-\frac{89}{26};~-\frac{5}{26} \right ); \quad
r = \sqrt{\frac{2465}{338}}\)]

% \item  \(A = \left (3;~-5 \right ); \quad B = \left (-4;~4 \right ); \quad 
% C = \left (4;~-5 \right )\)
% 
% \hfill [\(x^2 + y^2 -7x -\frac{47}{9}y -\frac{352}{9} = 0; \quad
% C \left (\frac{7}{2};~\frac{47}{18} \right ); \quad
% r = \sqrt{\frac{9425}{162}}\)]
% 
% \item  \(A = \left (4;~4 \right ); \quad B = \left (-6;~0 \right ); \quad 
% C = \left (4;~3 \right )\)
% 
% \hfill [\(x^2 + y^2 +\frac{16}{5}x -7y -\frac{84}{5} = 0; \quad
% C \left (-\frac{8}{5};~\frac{7}{2} \right ); \quad
% r = \sqrt{\frac{3161}{100}}\)]
% 
% \item  \(A = \left (-2;~-5 \right ); \quad B = \left (3;~3 \right ); \quad
% C = \left (3;~0 \right )\)
% 
% \hfill [\(x^2 + y^2 +7x -3y -30 = 0; \quad
% C \left (-\frac{7}{2};~\frac{3}{2} \right ); \quad
% r = \sqrt{ \frac{89}{2}}\)]
% 
% \item  \(A = \left (4;~5 \right ); \quad B = \left (5;~-4 \right ); \quad 
% C = \left (-3;~-6 \right )\)
% 
% \hfill [\(x^2 + y^2 +\frac{18}{37}x +\frac{2}{37}y -\frac{1599}{37} = 0; \quad
% C \left (-\frac{9}{37};~-\frac{1}{37} \right ); \quad
% r = \sqrt{\frac{59245}{1369}}\)]
% 
% \item  \(A = \left (-3;~5 \right ); \quad B = \left (-4;~-6 \right ); \quad 
% C = \left (1;~-2 \right )\)
% 
% \hfill [\(x^2 + y^2 +\frac{445}{51}x +\frac{43}{51}y -\frac{614}{51} = 0; \quad
% C \left (-\frac{445}{102};~-\frac{43}{102} \right ); \quad
% r = \sqrt{\frac{162565}{5202}}\)]
 \end{enumeratea}
\end{esercizio}


\subsubsection*{\numnameref{sec:circ_circrette}}

\begin{esercizio}\label{ese:}
 Calcola le intersezioni tra la circonferenza e la retta.
 \begin{enumeratea}
  \item  \(c:~x^2 + y^2 -4x +12y -69 = 0; \quad r:~y = 4\)
   \hfill [\(-1; \quad 5\)]
  \item  \(c:~x^2 + y^2 -8x -2y -84 = 0; \quad r:~y = \frac{2}{3} x +6\)
   \hfill [\(\frac{2 \mp \sqrt{2344}}{39}\)]
  \item  \(c:~x^2 + y^2 +6x -6y +10 = 0; \quad r:~y = -4 x +1\)
   \hfill [\(-1; \quad -\frac{5}{17}\)]
  \item  \(c:~x^2 + y^2 +2x +8y -9 = 0; \quad r:~y = \frac{1}{2} x \)
   \hfill [\(\frac{-3 \mp \sqrt{54}}{5}\)]
  \item  \(c:~x^2 + y^2 -6x +4y -21 = 0; \quad r:~y = -2 x -3\)
   \hfill [\(-2; \quad \frac{12}{5}\)]
  \item  \(c:~x^2 + y^2 -2x -2y -23 = 0; \quad r:~y = - x +1\)
   \hfill [\(-3; \quad 4\)]
  \item  \(c:~x^2 + y^2 -8x +2y -12 = 0; \quad r:~y = -3 x -6\)
   \hfill [\(-\frac{6}{5}; \quad -1\)]
  \item  \(c:~x^2 + y^2 -10x +2y -75 = 0; \quad r:~y = -\frac{1}{5} x -1\)
   \hfill [\(\frac{5 \mp \sqrt{2001}}{26}\)]
  \item  \(c:~x^2 + y^2 +10x -8y -68 = 0; \quad r:~y = \frac{5}{2} x -1\)
   \hfill [\(\frac{15 \mp \sqrt{7069}}{58}\)]
  \item  \(c:~x^2 + y^2 +10x +12y -20 = 0; \quad r:~y = -\frac{15}{4} x +9\)
   \hfill [\(\frac{205 \mp 283807i}{964}\)]
 \end{enumeratea}
\end{esercizio}


\begin{esercizio}\label{ese:}
 Calcola le intersezioni tra la circonferenza e la retta.
 \begin{enumeratea}
  \item  \(c:~x^2 + y^2 +2x +8y +17 = 0; \quad r:~y = -4 x -15\)
   \hfill [\(\frac{-45 \mp 7i}{17}\)]
  \item  \(c:~x^2 + y^2 +2x -6y +9 = 0; \quad r:~y = x +8\)
   \hfill [\(\frac{-6 \mp 14i}{2}\)]
  \item  \(c:~x^2 + y^2 -6x +2y +10 = 0; \quad r:~y = -\frac{5}{2} x 
-\frac{27}{2}\)
   \hfill [\(\frac{-113 \mp 25569i}{116}\)]
  \item  \(c:~x^2 + y^2 +6x +10y +18 = 0; \quad r:~y = \frac{1}{2} x 
-\frac{11}{2}\)
   \hfill [\(\frac{-11 \mp \sqrt{391}}{20}\)]
  \item  \(c:~x^2 + y^2 +2y -8 = 0; \quad r:~y = - x -2\)
   \hfill [\(\frac{-1 \mp \sqrt{17}}{2}\)]
  \item  \(c:~x^2 + y^2 +8x +2y -19 = 0; \quad r:~y = 4\)
   \hfill [\(-4 \mp \sqrt{11}\)]
  \item  \(c:~x^2 + y^2 +12x +12y +71 = 0; \quad r:~y = -\frac{7}{8} x 
-\frac{13}{4}\)
   \hfill [\(\frac{-115 \mp 294587i}{3616}\)]
  \item  \(c:~x^2 + y^2 +10y  = 0; \quad r:~y = -\frac{3}{2} x -\frac{9}{2}\)
   \hfill [\(\frac{3 \mp \sqrt{2583}}{52}\)]
  \item  \(c:~x^2 + y^2 +2x +10y + = 0; \quad r:~y = x +5\)
   \hfill [\(\frac{-11 \mp 31i}{2}\)]
  \item  \(c:~x^2 + y^2 +10x -6y +18 = 0; \quad r:~y = 2 x +6\)
   \hfill [\(\frac{-11 \mp \sqrt{31}}{5}\)]
 \end{enumeratea}
\end{esercizio}


\begin{esercizio}\label{ese:}
 Calcola l'equazione della circonferenza e le intersezioni con la retta.
 \begin{enumeratea}
  \item  \(C=\left (\frac{81}{14}; \quad -\frac{31}{14} \right ); \quad 
r = \sqrt{\frac{1885}{98}}; \quad retta:~y = -\frac{1}{8} x +\frac{1}{4}\)

\hfill [\(x^2 + y^2 -\frac{81}{7}x +\frac{31}{7}y +\frac{134}{7} = 0; \quad 
A = \left (10;~-1 \right ); \quad B = \left (2;~0 \right )\)]

\item  \(C=\left (-\frac{25}{4}; \quad -\frac{1}{8} \right ); \quad 
r = \sqrt{\frac{9685}{64}}; \quad retta:~y = \frac{10}{7} x +\frac{43}{7}\)

\hfill [\(x^2 + y^2 +\frac{25}{2}x +\frac{1}{4}y -\frac{449}{4} = 0; \quad 
A = \left (-12;~-11 \right ); \quad B = \left (2;~9 \right )\)]

\item  \(C=\left (-\frac{94}{57}; \quad -\frac{31}{114} \right ); \quad 
r = \sqrt{\frac{761345}{12996}}; \quad retta:~y = \frac{7}{4} x -\frac{21}{2}\)

\hfill [\(x^2 + y^2 +\frac{188}{57}x +\frac{31}{57}y -\frac{1060}{19} = 0; 
\quad A = \left (2;~-7 \right ); \quad B = \left (6;~0 \right )\)]
% 
% \item  \(C=\left (\frac{109}{2}; \quad 64 \right ); \quad 
% r = \sqrt{\frac{32513}{4}}; \quad retta:~y = -\frac{5}{6} x -\frac{15}{2}\)
% 
% \hfill [\(x^2 + y^2 -109x -128y -1062 = 0; \quad 
% A = \left (3;~-10 \right ); \quad B = \left (-9;~0 \right )\)]
% 
% \item  \(C=\left (\frac{20}{27}; \quad -\frac{64}{27} \right ); \quad 
% r = \sqrt{\frac{108770}{729}}; \quad retta:~y = -\frac{7}{10} x -\frac{13}{10}\)
% 
% \hfill [\(x^2 + y^2 -\frac{40}{27}x +\frac{128}{27}y -\frac{3862}{27} = 0; 
% \quad A = \left (11;~-9 \right ); \quad B = \left (-9;~5 \right )\)]
% 
% \item  \(C=\left (\frac{41}{4}; \quad \frac{3}{4} \right ); \quad 
% r = \sqrt{\frac{697}{8}}; \quad retta:~y = x +1\)
% 
% \hfill [\(x^2 + y^2 -\frac{41}{2}x -\frac{3}{2}y +\frac{37}{2} = 0; \quad 
% A = \left (1;~2 \right ); \quad B = \left (9;~10 \right )\)]
% 
% \item  \(C=\left (\frac{79}{2}; \quad \frac{31}{2} \right ); \quad 
% r = \sqrt{\frac{2465}{2}}; \quad retta:~y = -3 x +24\)
% 
% \hfill [\(x^2 + y^2 -79x -31y +568 = 0; \quad 
% A = \left (5;~9 \right ); \quad B = \left (8;~0 \right )\)]
 \end{enumeratea}
\end{esercizio}


\begin{esercizio}\label{ese:}
 Data una circonferenza e un suo punto calcola l'equazione della tangente.
 \begin{enumeratea}
  \item  \(c:~x^2 + y^2 -8x +10y +24 = 0; \quad P \left (5; \quad -1 \right )\)
   \hfill [\(y = -\frac{1}{4} x +\frac{1}{4}\)]
  \item  \(c:~x^2 + y^2 -10x -10y -2 = 0; \quad P \left (1; \quad -1 \right )\)
   \hfill [\(y = -\frac{2}{3} x -\frac{1}{3}\)]
  \item  \(c:~x^2 + y^2 +10y -49 = 0; \quad P \left (5; \quad 2 \right )\)
   \hfill [\(y = -\frac{5}{7} x +\frac{39}{7}\)]
  \item  \(c:~x^2 + y^2 +8x +12y +15 = 0; \quad P \left (2; \quad -5 \right )\)
   \hfill [\(y = -6 x +7\)]
  \item  \(c:~x^2 + y^2 +2x +8y  = 0; \quad P \left (-5; \quad -5 \right )\)
   \hfill [\(y = -4 x -25\)]
  \item  \(c:~x^2 + y^2 +4x +4y -33 = 0; \quad P \left (3; \quad -6 \right )\)
   \hfill [\(y = \frac{5}{4} x -\frac{39}{4}\)]
  \item  \(c:~x^2 + y^2 +12x +4y -18 = 0; \quad P \left (-3; \quad 5 \right )\)
   \hfill [\(y = -\frac{3}{7} x +\frac{26}{7}\)]
  \item  \(c:~x^2 + y^2 -10x -10y +30 = 0; \quad P \left (1; \quad 3 \right )\)
   \hfill [\(y = -2 x +5\)]
  \item  \(c:~x^2 + y^2 +4x -4y -33 = 0; \quad P \left (2; \quad -3 \right )\)
   \hfill [\(y = \frac{4}{5} x -\frac{23}{5}\)]
  \item  \(c:~x^2 + y^2 -10x +2y -80 = 0; \quad P \left (-4; \quad 4 \right )\)
   \hfill [\(y = \frac{9}{5} x +\frac{56}{5}\)]
 \end{enumeratea}
\end{esercizio}


\subsection{Esercizi di riepilogo}

\begin{esercizio}\label{ese:}
Determinare il centro, il raggio e disegnare la circonferenza di equazione:

\(x^2+y^2+4x-2y-4=0\)

Determinare poi i punti di intersezione fra la circonferenza e la retta di 
equazione 
\(y=2x+2\) 
\hfill [\(I_1\tonda{-2}{-2}~I_2\tonda{\frac{2}{5}}{\frac{14}{5}}\)]
\end{esercizio}

\begin{esercizio}\label{ese:}
Determinare il centro, il raggio e disegnare la circonferenza di equazione

\(x^2+y^2-2x+4y-4=0\)

Determinare poi i punti di intersezione fra la circonferenza e la retta di 
equazione \(y=2x-1\) 
\hfill [\(I_1\tonda{1}{1}~I_2\tonda{-\frac{7}{5}}{-\frac{19}{5}}\)]
\end{esercizio}

\begin{esercizio}\label{ese:}
Considerata la circonferenza di equazione \(x^2 +y^2 = 4\) dire se le seguenti 
rette sono secanti, non secanti oppure tangenti:
\begin{itemize} [nosep]
 \item \(y=x+4\)
 \item \(y=-x+1\)
 \item \(y=2x+2\sqrt{5}\)
\end{itemize}
\end{esercizio}

\begin{esercizio}\label{ese:}
Considerata la circonferenza di equazione \(x^2 +y^2 = 4\) dire se le seguenti 
rette sono secanti, non secanti oppure tangenti:
\begin{itemize} [nosep]
 \item \(y=-x+5\)
 \item \(y=2x-1\)
 \item \(y=2x-2\sqrt{5}\)
\end{itemize}
\end{esercizio}

\begin{esercizio}\label{ese:}
Scrivere l'equazione della circonferenza avente centro in \(\punto{3}{0}\) e 
passante per il punto \(\punto{6}{4}\).\hfill[\(x^2+y^2-6x-16=0\)]
\end{esercizio}

\begin{esercizio}\label{ese:}
Scrivere l'equazione della circonferenza avente per diametro il segmento di 
estremi:
 \begin{enumeratea}
  \item  \(A\punto{-3}{1},~B\punto{5}{-2}\)\hfill [\(x^2+y^2-2x+y-17=0\)]
  \item  \(A\punto{1}{0},~B\punto{3}{2}\)\hfill [\(x^2+y^2-4x-2y+3=0\)]
  \item  \(A\punto{0}{1},~B\punto{2}{3}\)\hfill [\(x^2+y^2-2x-4y+3=0\)]
 \end{enumeratea}
\end{esercizio}

\begin{esercizio}\label{ese:}
Scrivere l'equazione della circonferenza passante per \(A\) e per \(B\) e 
avente il centro rulla retta \(r\):
 \begin{enumeratea}
  \item  \(A\punto{-2}{2},~B\punto{4}{-4},~r:~x+2y-8=0\)
  \hfill [\(x^2+y^2-8x-4y-16=0\)]
  \item  \(A\punto{2}{2}, B\punto{-4}{-4},~r:~x+2y=8\)
  \hfill [\(x^2+y^2+8x-4y-16=0\)]
  \item  \(A\punto{-2}{2}, B\punto{4}{0},~r:~3x-2y-1=0\)
  \hfill [\(x^2+y^2-2x-2y-8=0\)]
 \end{enumeratea}
\end{esercizio}

\begin{esercizio}\label{ese:}
Scrivere l'equazione della circonferenza passante per \(A\), \(B\) e \(C\):
 \begin{enumeratea}
  \item  \(A\punto{1}{2},~B\punto{-1}{2},~C\punto{0}{0}\)
  \hfill [\(x^2+y^2+x-3y=0\)]
  \item  \(A\punto{1}{6}, B\punto{-1}{0},~C\punto{-2}{6}\)
  \hfill [\(x^2+y^2+x-6y-2=0\)]
%   \item  \(A\punto{-2}{2}, B\punto{4}{0},~C\punto{4}{-4}\)
%   \hfill [\(x^2+y^2-2x-2y-8=0\)]
 \end{enumeratea}
\end{esercizio}

\begin{esercizio}\label{ese:}        
Scrivere l'equazione della circonferenza passante per i punti 
\(A\punto{4}{1},~B\punto{2}{2}\) e avente il centro sulla retta \(r:~x-2y=0\).
Poi calcola la tangenti in A e in B alla circonferenza 
\hfill [\(x^2+y^2-6x-3y+10=0,~y=2x-7,~y=2x-2\)]
\end{esercizio}

\begin{esercizio}\label{ese:}
Scrivere l'equazione della circonferenza avente gli estremi del diametro 
nei punti di intersezione della retta \(x-3y-1=0\) con la retta \(x+2=0\)
e della retta \(x-2y=0\) con la retta \(x-2=0\)
\hfill[\(x^2+y^2 = 5\)]
\end{esercizio}

\begin{esercizio}\label{ese:}
Determinare il raggio e l'equazione della circonferenza avente centro in 
\(C\punto{2}{−6}\) e passante per \(P\punto{−7}{−1}\).
\hfill [\(r=\sqrt{106}; \quad \tonda{x-2}^2+\tonda{y+6}^2=106\)]
\end{esercizio}

\begin{esercizio}\label{ese:}
Determinare l'equazione della circonferenza passante per i punti 
\(A\punto{2}{0}, B\punto{−2}{4}\) 
ed avente il centro sulla retta  \(r:~x+y+2=0\)             
\hfill [\(\tonda{x+2}^2+y^2=16\)]
\end{esercizio}

\begin{esercizio}\label{ese:}
Determinare le equazioni delle circonferenze tangenti all'asse delle \(y\) nel 
punto di ordinata~4 ed aventi raggio pari a~5.
\hfill [\(\tonda{x-5}^2+\tonda{y-4}^2=25; \quad 
\tonda{x+5}^2+\tonda{y-4}^2=25\)]
\end{esercizio}

\begin{esercizio}\label{ese:}
Determinare le coordinate dei punti di intersezione della retta 
\(r:~x-y-2=0\) con la circonferenza \(x^2 +y^2 -2x-6y-6=0\)
\hfill [\(A\punto{1}{-1},~B\punto{5}{3}\)]  
\end{esercizio}

\begin{esercizio}\label{ese:} 
Determinare le equazioni delle rette tangenti condotte dal punto 
\(P\punto{7}{0}\) alla circonferenza \(x^2 +y^2 -2x-4y-15=0\)
\hfill [\(y=-2x+14; \quad y=\frac{1}{2}x-\frac{7}{2}\)]
\end{esercizio}

\begin{esercizio}\label{ese:}
Determinare le equazioni delle circonferenze passanti per 
\(A\punto{1}{0},~B\punto{0}{1}\) ed aventi raggio pari a \(\sqrt{5}\).          
\hfill [\(\tonda{x-2}^2+\tonda{y-2}^2=5; \quad \tonda{x+1}^2+\tonda{y+1}^2=5\)]
\end{esercizio}
