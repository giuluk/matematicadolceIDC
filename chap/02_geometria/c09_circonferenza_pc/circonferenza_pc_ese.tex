% (c) 2016 Daniele Zambelli daniele.zambelli@gmail.com
% (c) 2016 Elisabetta Campana

\section{TODO}

\section{Esercizi}

% \subsection{Esercizi dei singoli paragrafi}
% 
% \subsubsection*{\numnameref{sec:01_}}

\begin{esercizio}\label{ese:}
Determinare il centro, il raggio e disegnare la circonferenza di equazione:

\(x^2+y^2+4x-2y-4=0\)

Determinare poi i punti di intersezione fra la circonferenza e la retta di 
equazione 
\(y=2x+2\) 
\hfill [\(I_1\tonda{-2}{-2}~I_2\tonda{\frac{2}{5}}{\frac{14}{5}}\)]
\end{esercizio}

\begin{esercizio}\label{ese:}
Determinare il centro, il raggio e disegnare la circonferenza di equazione

\(x^2+y^2-2x+4y-4=0\)

Determinare poi i punti di intersezione fra la circonferenza e la retta di 
equazione \(y=2x-1\) 
\hfill [\(I_1\tonda{1}{1}~I_2\tonda{-\frac{7}{5}}{-\frac{19}{5}}\)]
\end{esercizio}

\begin{esercizio}\label{ese:}
Considerata la circonferenza di equazione \(x^2 +y^2 = 4\) dire se le seguenti 
rette sono secanti, non secanti oppure tangenti:
\begin{itemize} [nosep]
 \item \(y=x+4\)
 \item \(y=-x+1\)
 \item \(y=2x+2\sqrt{5}\)
\end{itemize}
\end{esercizio}

\begin{esercizio}\label{ese:}
Considerata la circonferenza di equazione \(x^2 +y^2 = 4\) dire se le seguenti 
rette sono secanti, non secanti oppure tangenti:
\begin{itemize} [nosep]
 \item \(y=-x+5\)
 \item \(y=2x-1\)
 \item \(y=2x-2\sqrt{5}\)
\end{itemize}
\end{esercizio}

\begin{esercizio}\label{ese:}
Scrivere l'equazione della circonferenza avente centro in \(\punto{3}{0}\) e 
passante per il punto \(\punto{6}{4}\).\hfill[\(x^2+y^2-6x-16=0\)]
\end{esercizio}

\begin{esercizio}\label{ese:}
Scrivere l'equazione della circonferenza avente per diametro il segmento di 
estremi:
 \begin{enumeratea}
  \item  \(A\punto{-3}{1},~B\punto{5}{-2}\)\hfill [\(x^2+y^2-2x+y-17=0\)]
  \item  \(A\punto{1}{0},~B\punto{3}{2}\)\hfill [\(x^2+y^2-4x-2y+3=0\)]
  \item  \(A\punto{0}{1},~B\punto{2}{3}\)\hfill [\(x^2+y^2-2x-4y+3=0\)]
 \end{enumeratea}
\end{esercizio}

\begin{esercizio}\label{ese:}
Scrivere l'equazione della circonferenza passante per \(A\) e per \(B\) e 
avente il centro rulla retta \(r\):
 \begin{enumeratea}
  \item  \(A\punto{-2}{2},~B\punto{4}{-4},~r:~x+2y-8=0\)
  \hfill [\(x^2+y^2-8x-4y-16=0\)]
  \item  \(A\punto{2}{2}, B\punto{-4}{-4},~r:~x+2y=8\)
  \hfill [\(x^2+y^2+8x-4y-16=0\)]
  \item  \(A\punto{-2}{2}, B\punto{4}{0},~r:~3x-2y-1=0\)
  \hfill [\(x^2+y^2-2x-2y-8=0\)]
 \end{enumeratea}
\end{esercizio}

\begin{esercizio}\label{ese:}
Scrivere l'equazione della circonferenza passante per \(A\), \(B\) e \(C\):
 \begin{enumeratea}
  \item  \(A\punto{1}{2},~B\punto{-1}{2},~C\punto{0}{0}\)
  \hfill [\(x^2+y^2+x-3y=0\)]
  \item  \(A\punto{1}{6}, B\punto{-1}{0},~C\punto{-2}{6}\)
  \hfill [\(x^2+y^2+x-6y-2=0\)]
%   \item  \(A\punto{-2}{2}, B\punto{4}{0},~C\punto{4}{-4}\)
%   \hfill [\(x^2+y^2-2x-2y-8=0\)]
 \end{enumeratea}
\end{esercizio}

\begin{esercizio}\label{ese:}        
Scrivere l'equazione della circonferenza passante per i punti 
\(A\punto{4}{1},~B\punto{2}{2}\) e avente il centro sulla retta \(r:~x-2y=0\).
Poi calcola la tangenti in A e in B alla circonferenza 
\hfill [\(x^2+y^2-6x-3y+10=0,~y=2x-7,~y=2x-2\)]
\end{esercizio}

\begin{esercizio}\label{ese:}
Scrivere l'equazione della circonferenza avente gli estremi del diametro 
nei punti di intersezione della retta \(x-3y-1=0\) con la retta \(x+2=0\)
e della retta \(x-2y=0\) con la retta \(x-2=0\)
\hfill[\(x^2+y^2 = 5\)]
\end{esercizio}

\begin{esercizio}\label{ese:}
Determinare il raggio e l'equazione della circonferenza avente centro in 
\(C\punto{2}{−6}\) e passante per \(P\punto{−7}{−1}\).
\hfill [\(r=\sqrt{106};~\tonda{x-2}^2+\tonda{y+6}^2=106\)]
\end{esercizio}

\begin{esercizio}\label{ese:}
Determinare l'equazione della circonferenza passante per i punti 
\(A\punto{2}{0}, B\punto{−2}{4}\) 
ed avente il centro sulla retta  \(r:~x+y+2=0\)             
\hfill [\(\tonda{x+2}^2+y^2=16\)]
\end{esercizio}

\begin{esercizio}\label{ese:}
Determinare le equazioni delle circonferenze tangenti all'asse delle \(y\) nel 
punto di ordinata~4 ed aventi raggio pari a~5.
\hfill [\(\tonda{x-5}^2+\tonda{y-4}^2=25;~\tonda{x+5}^2+\tonda{y-4}^2=25\)]
\end{esercizio}

\begin{esercizio}\label{ese:}
Determinare le coordinate dei punti di intersezione della retta 
\(r:~x-y-2=0\) con la circonferenza \(x^2 +y^2 -2x-6y-6=0\)
\hfill [\(A\punto{1}{-1},~B\punto{5}{3}\)]  
\end{esercizio}

\begin{esercizio}\label{ese:} 
Determinare le equazioni delle rette tangenti condotte dal punto 
\(P\punto{7}{0}\) alla circonferenza \(x^2 +y^2 -2x-4y-15=0\)
\hfill [\(y=-2x+14;~y=\frac{1}{2}x-\frac{7}{2}\)]
\end{esercizio}

\begin{esercizio}\label{ese:}
Determinare le equazioni delle circonferenze passanti per 
\(A\punto{1}{0},~B\punto{0}{1}\) ed aventi raggio pari a \(\sqrt{5}\).          
\hfill [\(\tonda{x-2}^2+\tonda{y-2}^2=5;~\tonda{x+1}^2+\tonda{y+1}^2=5\)]
\end{esercizio}
