% (c) 2015 Daniele Zambelli daniele.zambelli@gmail.com

\chapter{Luoghi geometrici}

\section{TODO}

\section{Il cono e le sue sezioni}
\label{sec:01_sezioniconiche}

% \begin{inaccessibleblock}[Parabola di equazione $y=x^2$.]
% \centering
%   % (c) 2014 Daniele Zambelli - daniele.zambelli@gmail.com

%%%
% Varie prove per il disegno di una parabola nel piano cartesiano.
%%%%
%  
\begin{tikzpicture}[x=5mm, y=5mm, smooth, color=Blue!50!black]]

\tkzInit[xmin=-10.5,xmax=+10.5,ymin=-10.5,ymax=+15.5]

\clip (-10.3, -6.3) rectangle (10.7, 10.7);

% (c) 2014 Daniele Zambelli - daniele.zambelli@gmail.com

%%%
% Piano cartesiano: da (-10; -10) a (+10; +10)
%%%%

% Griglia
\draw[gray!50, very thin, step=1] (-10.2, -10.2) grid (10.2, 10.2);

%Asse x
\draw [-{Stealth[length=2mm, open, round]}] (-10.3,0) -- (10.5,0) node [below]  {$x$};

%Asse y
\draw [-{Stealth[length=2mm, open, round]}] (0, -10.3) -- (0, 10.5) node [left]  {$y$};


\tkzFct[domain=-10:+10, ultra thick]{-1./4.*x*x+2*x+1}

\tkzFct[domain=-10:+10, ultra thick, color=Green!50!black]{x+2}

\tkzFct[domain=-10:+10, ultra thick, color=Green!50!black]{3*x+2}

\tkzFct[domain=-10:+10, ultra thick, color=Red!50!black]{-2.*x+17}

\begin{scope}[color=Green!50!black]
\coordinate (a) at (0, +2);
\filldraw  (a) circle (1.5pt); 
\node at (a) [xshift=-9pt] {$A$};
\end{scope}

\begin{scope}[color=Red!50!black]
\coordinate (b) at (+8, +1);
\filldraw (b) circle (1.5pt); 
\node at (b) [xshift=+7pt] {$B$};
\end{scope}

\begin{scope}[color=Blue!50!black]
\coordinate (b) at (+5, -2);
\filldraw (b) circle (1.5pt); 
\node at (b) [xshift=+7pt] {$C$};

\filldraw (+2, +4) circle (1.5pt); 
\filldraw (-2, -4) circle (1.5pt); 
\end{scope}

\end{tikzpicture}

%   \caption{Tangenti ad una parabola.} \label{fig:parabola_tangenti}
% \end{inaccessibleblock}

% \begin{figure}[h]
% \begin{minipage}{.40\textwidth}
% \begin{enumerate}
%  \item il punto è esterno alla parabola: 2 tangenti reali distinte;
%  \item il punto appartiene alla parabola: due tangenti reali coincidenti
%   (una tangente?);
%  \item il punto è interno alla parabola: nessuna tangente reale.
% \end{enumerate}
% Vediamo i tre casi con un esempio \ref{fig:parabola_tangenti}.
% \end{minipage}
% \begin{minipage}{.60\textwidth}
% \begin{inaccessibleblock}[Parabola di equazione $y=x^2$.]
% \centering
%   % (c) 2014 Daniele Zambelli - daniele.zambelli@gmail.com

%%%
% Varie prove per il disegno di una parabola nel piano cartesiano.
%%%%
%  
\begin{tikzpicture}[x=5mm, y=5mm, smooth, color=Blue!50!black]]

\tkzInit[xmin=-10.5,xmax=+10.5,ymin=-10.5,ymax=+15.5]

\clip (-10.3, -6.3) rectangle (10.7, 10.7);

% (c) 2014 Daniele Zambelli - daniele.zambelli@gmail.com

%%%
% Piano cartesiano: da (-10; -10) a (+10; +10)
%%%%

% Griglia
\draw[gray!50, very thin, step=1] (-10.2, -10.2) grid (10.2, 10.2);

%Asse x
\draw [-{Stealth[length=2mm, open, round]}] (-10.3,0) -- (10.5,0) node [below]  {$x$};

%Asse y
\draw [-{Stealth[length=2mm, open, round]}] (0, -10.3) -- (0, 10.5) node [left]  {$y$};


\tkzFct[domain=-10:+10, ultra thick]{-1./4.*x*x+2*x+1}

\tkzFct[domain=-10:+10, ultra thick, color=Green!50!black]{x+2}

\tkzFct[domain=-10:+10, ultra thick, color=Green!50!black]{3*x+2}

\tkzFct[domain=-10:+10, ultra thick, color=Red!50!black]{-2.*x+17}

\begin{scope}[color=Green!50!black]
\coordinate (a) at (0, +2);
\filldraw  (a) circle (1.5pt); 
\node at (a) [xshift=-9pt] {$A$};
\end{scope}

\begin{scope}[color=Red!50!black]
\coordinate (b) at (+8, +1);
\filldraw (b) circle (1.5pt); 
\node at (b) [xshift=+7pt] {$B$};
\end{scope}

\begin{scope}[color=Blue!50!black]
\coordinate (b) at (+5, -2);
\filldraw (b) circle (1.5pt); 
\node at (b) [xshift=+7pt] {$C$};

\filldraw (+2, +4) circle (1.5pt); 
\filldraw (-2, -4) circle (1.5pt); 
\end{scope}

\end{tikzpicture}

%   \caption{Tangenti ad una parabola.} \label{fig:parabola_tangenti}
% \end{inaccessibleblock}
% \end{minipage}
% \end{figure}

% \begin{wrapfloat}{figure}{r}{0pt}
% \includegraphics[scale=0.35]{img/fig000_.png}
% \caption{...}
% \label{fig:...}
% \end{wrapfloat}
% 
% \begin{center} \input{\folder lbr/fig000_.pgf} \end{center}

\section{La circonferenza}
\label{sec:02_circonferenza}

\section{L'ellisse}
\label{sec:03_ellisse}

\section{La parabola}
\label{sec:04_parabola}

\section{L'iperbole}
\label{sec:05_iperbole}


