% (c) 2015 Daniele Zambelli daniele.zambelli@gmail.com

\section{Esercizi}

\subsection{Esercizi dei singoli paragrafi}

\begin{esercizio}\label{ese:}
 Determina il valore del polinomio per i diversi valori di x.
 \begin{enumeratea}
  \item  $P(x)=- x^2 +5 x +9;~x_0=-4;~x_1=3;~x_2=8$
   \hfill [$P(-4)=-27;~P(3)=15;~P(8)=-15$]
  \item  $P(x)=-2 x^2 -6 x -5;~x_0=8;~x_1=4;~x_2=5$
   \hfill [$P(8)=-181;~P(4)=-61;~P(5)=-85$]
  \item  $P(x)=4 x^2 -8 x +4;~x_0=-10;~x_1=3;~x_2=-4$
   \hfill [$P(-10)=484;~P(3)=16;~P(-4)=100$]
  \item  $P(x)=-3 x^2 -10 x -4;~x_0=5;~x_1=4;~x_2=-2$
   \hfill [$P(5)=-129;~P(4)=-92;~P(-2)=4$]
  \item  $P(x)=-2 x^2 +7 x +6;~x_0=7;~x_1=-5;~x_2=2$
   \hfill [$P(7)=-43;~P(-5)=-79;~P(2)=12$]
  \item  $P(x)=2 x^2 -9 x -9;~x_0=-4;~x_1=-9;~x_2=7$
   \hfill [$P(-4)=59;~P(-9)=234;~P(7)=26$]
  \item  $P(x)=3 x^2 -8 x -4;~x_0=-8;~x_1=2;~x_2=6$
   \hfill [$P(-8)=252;~P(2)=-8;~P(6)=56$]
  \item  $P(x)=3 x^2 +7 x -1;~x_0=-7;~x_1=-10;~x_2=3$
   \hfill [$P(-7)=97;~P(-10)=229;~P(3)=47$]
%   \item  $P(x)=x^2 + x +5;~x_0=1;~x_1=-1;~x_2=-8$
%    \hfill [$P(1)=7;~P(-1)=5;~P(-8)=61$]
%   \item  $P(x)=x^2 -10 x +8;~x_0=-6;~x_1=-1;~x_2=5$
%    \hfill [$P(-6)=104;~P(-1)=19;~P(5)=-17$]
%   \item  $P(x)=x^2 +4 x +8;~x_0=-4;~x_1=-1;~x_2=-2$
%    \hfill [$P(-4)=8;~P(-1)=5;~P(-2)=4$]
 \end{enumeratea}
\end{esercizio}
% \smallsmile \smallfrown
\begin{esercizio}\label{ese:}
 Determina la concavità e il vertice della parabola senza disegnarne il grafico.
\begin{multicols}{2}
  \begin{enumeratea}
  \item  $y=- x^2 -2 x +6$
   \hfill [$\smallfrown;~V \left (-1;~7 \right )$]
  \item  $y=\frac{1}{2} x^2 +2 x +9$
   \hfill [$\smallsmile;~V \left (-2;~7 \right )$]
  \item  $y=\frac{3}{2} x^2 +3 x -8$
   \hfill [$\smallsmile;~V \left (-1;~-\frac{19}{2} \right )$]
%   \item  $y=-\frac{1}{3} x^2 -3 x -3$
%    \hfill [$\smallfrown;~V \left (-\frac{9}{2};~\frac{15}{4} \right 
% )$]
%   \item  $y=4 x^2 +3 x +3$
%    \hfill [$\smallsmile;~V \left (-\frac{3}{8};~\frac{39}{16} 
% \right % )$]
%   \item  $y=x^2 +3 x -5$
%    \hfill [$\smallsmile;~V \left (-\frac{3}{2};~-\frac{29}{4} \right % )$]
  \item  $y=-\frac{4}{5} x^2 -4 x +1$
   \hfill [$\smallfrown;~V \left (-\frac{5}{2};~6 \right )$]
  \item  $y=\frac{2}{5} x^2 -3 x -8$
   \hfill [$\smallsmile;~V \left (\frac{15}{4};~-\frac{109}{8} \right )$]
  \item  $y=-\frac{3}{4} x^2 -3 x -7$
   \hfill [$\smallfrown;~V \left (-2;~-4 \right )$]
  \item  $y=\frac{3}{4} x^2 -3 x -3$
   \hfill [$\smallsmile;~V \left (2;~-6 \right )$]
%   \item  $y=2 x^2 -5 x -7$
%    \hfill [$\smallsmile;~V \left (\frac{5}{4};~-\frac{81}{8} \right )$]
  \item  $y=-\frac{5}{3} x^2 +4 x -1$
   \hfill [$\smallfrown;~V \left (\frac{6}{5};~\frac{7}{5} \right )$]
%   \item  $y=5 x^2 -5 x +9$
%    \hfill [$\smallsmile;~V \left (\frac{1}{2};~\frac{31}{4} \right 
% )$]
%   \item  $y=\frac{5}{4} x^2 -5 x +5$
%    \hfill [$\smallsmile;~V \left (2;~0 \right )$]
%   \item  $y=\frac{3}{2} x^2 +3 x +7$
%    \hfill [$\smallsmile;~V \left (-1;~\frac{11}{2} \right )$]
%   \item  $y=- x^2 -1$
%    \hfill [$\smallfrown;~V \left (0;~-1 \right )$]
%   \item  $y=3 x^2 -4 x -2$
%    \hfill [$\smallsmile;~V \left (\frac{2}{3};~-\frac{10}{3} \right 
% )$]
%   \item  $y=x^2 +2 x +3$
%    \hfill [$\smallsmile;~V \left (-1;~2 \right )$]
%   \item  $y=\frac{1}{2} x^2 -3 x +3$
%    \hfill [$\smallsmile;~V \left (3;~-\frac{3}{2} \right )$]
%   \item  $y=\frac{1}{3} x^2 -3 x +9$
%    \hfill [$\smallsmile;~V \left (\frac{9}{2};~\frac{9}{4} \right 
% )$]
 \end{enumeratea}
\end{multicols}
\end{esercizio}

\begin{esercizio}\label{ese:}
 Determina la concavità, calcola il vertice della parabola e disegnala.
\begin{multicols}{2}
 \begin{enumeratea}
  \item  $y=x^2 -2 x -4$
   \hfill [$\smallsmile;~V \left (1;~-5 \right )$]
  \item  $y=x^2 +3$
   \hfill [$\smallsmile;~V \left (0;~3 \right )$]
%   \item  $y=2 x^2 +2 x -3$
%    \hfill [$\smallsmile;~V \left (-\frac{1}{2};~-\frac{7}{2} \right 
% )$]
  \item  $y=-2 x^2 +4 x -2$
   \hfill [$\smallfrown;~V \left (1;~0 \right )$]
  \item  $y=\frac{1}{2} x^2 +4 x -4$
   \hfill [$\smallsmile;~V \left (-4;~-12 \right )$]
  \item  $y=- x^2 -1$
   \hfill [$\smallfrown;~V \left (0;~-1 \right )$]
  \item  $y=\frac{1}{2} x^2 +2 x -1$
   \hfill [$\smallsmile;~V \left (-2;~-3 \right )$]
  \item  $y=x^2 -2 x -4$
   \hfill [$\smallsmile;~V \left (1;~-5 \right )$]
  \item  $y=-2 x^2 +4 x +3$
   \hfill [$\smallfrown;~V \left (1;~5 \right )$]
  \item  $y=x^2 +2 x -5$
   \hfill [$\smallsmile;~V \left (-1;~-6 \right )$]
  \item  $y=-\frac{1}{2} x^2 +2 x -5$
   \hfill [$\smallfrown;~V \left (2;~-3 \right )$]
  \item  $y=-\frac{1}{2} x^2 -1$
   \hfill [$\smallfrown;~V \left (0;~-1 \right )$]
%   \item  $y=-2 x^2 -2 x -4$
%    \hfill [$\smallfrown;~V \left (-\frac{1}{2};~-\frac{7}{2} \right )$]
%   \item  $y=2 x^2 +4 x +4$
%    \hfill [$\smallsmile;~V \left (-1;~2 \right )$]
%   \item  $y=2 x^2 -2 x +5$
%    \hfill [$\smallsmile;~V \left (\frac{1}{2};~\frac{9}{2} \right )$]
%   \item  $y=x^2 +2$
%    \hfill [$\smallsmile;~V \left (0;~2 \right )$]
%   \item  $y=-2 x^2 +4 x -2$
%    \hfill [$\smallfrown;~V \left (1;~0 \right )$]
%   \item  $y=x^2 -4 x +5$
%    \hfill [$\smallsmile;~V \left (2;~1 \right )$]
%   \item  $y=2 x^2 -1$
%    \hfill [$\smallsmile;~V \left (0;~-1 \right )$]
  \item  $y=-\frac{1}{2} x^2 $
   \hfill [$\smallfrown;~V \left (0;~0 \right )$]
 \end{enumeratea}
\end{multicols}
\end{esercizio}

\begin{esercizio}\label{ese:}
 Disegna la parabola e calcola le intersezioni con gli assi.
 \begin{enumeratea}
  \item  $y=-2 x^2 +4 x -2$
   \hfill [$\left (0;~-2 \right ), \left(1;~0 \right), \left(1;~0 \right)$]
  \item  $y=2 x^2 $
   \hfill [$\left (0;~0 \right ), \left(0;~0 \right), \left(0;~0 \right)$]
  \item  $y=-2 x^2 +4 x -3$
   \hfill [$\left (0;~-3 \right )$]
  \item  $y=x^2 -2 x $
   \hfill [$\left (0;~0 \right ), \left(0;~0 \right), \left(2;~0 \right)$]
  \item  $y=x^2 +4 x -5$
   \hfill [$\left (0;~-5 \right ), \left(-5;~0 \right), \left(1;~0 \right)$]
  \item  $y=-\frac{1}{2} x^2 +4 x -6$
   \hfill [$\left (0;~-6 \right ), \left(6;~0 \right), \left(2;~0 \right)$]
  \item  $y=x^2 -2 x +5$
   \hfill [$\left (0;~5 \right )$]
  \item  $y=-2 x^2 +2 x -3$
   \hfill [$\left (0;~-3 \right )$]
%   \item  $y=2 x^2 -4 x -3$
%    \hfill [$\left (0;~-3 \right ), \left(-0.5811;~0 \right), \left(2.5811;~0 
% \right)$]
%   \item  $y=x^2 +4 x +3$
%    \hfill [$\left (0;~3 \right ), \left(-3;~0 \right), \left(-1;~0 \right)$]
%   \item  $y=-\frac{1}{2} x^2 +4 x +4$
%    \hfill [$\left (0;~4 \right ), \left(8.899;~0 \right), \left(-0.899;~0 
% \right)$]
%   \item  $y=2 x^2 -2 x +5$
%    \hfill [$\left (0;~5 \right )$]
%   \item  $y=- x^2 +2 x -5$
%    \hfill [$\left (0;~-5 \right )$]
%   \item  $y=-\frac{1}{2} x^2 -4 x -6$
%    \hfill [$\left (0;~-6 \right ), \left(-2;~0 \right), \left(-6;~0 \right)$]
%   \item  $y=x^2 +4 x -5$
%    \hfill [$\left (0;~-5 \right ), \left(-5;~0 \right), \left(1;~0 \right)$]
%   \item  $y=- x^2 +2 x +5$
%    \hfill [$\left (0;~5 \right ), \left(3.4495;~0 \right), \left(-1.4495;~0 
% \right)$]
%   \item  $y=- x^2 $
%    \hfill [$\left (0;~0 \right ), \left(0;~0 \right), \left(0;~0 \right)$]
  \item  $y=x^2 +4 x -3$
   \hfill [$\left (0;~-3 \right ), 
            \left(-4.6458;~0 \right), \left(0.6458;~0 \right)$]
  \item  $y=- x^2 -2 x -5$
   \hfill [$\left (0;~-5 \right )$]
  \item  $y=-\frac{1}{2} x^2 -2 x -1$
   \hfill [$\left (0;~-1 \right ), 
            \left(-0.5858;~0 \right), \left(-3.4142;~0 \right)$]
  \item  $y=2 x^2 -6$
   \hfill [$\left (0;~-6 \right ), 
            \left(-1,7321;~0 \right), \left(1,7321;~0 \right)$]
  \item  $y=-4 x^2 + x +6$
   \hfill [$\left (0;~6 \right ), 
            \left(1,3561;~0 \right), \left(-1,1061;~0 \right)$]
  \item  $y=- x^2 -5$
   \hfill [$\left (0;~-5 \right )$]
  \item  $y=-\frac{1}{4} x^2 -3 x +4$
   \hfill [$\left (0;~4 \right ), 
            \left(1,2111;~0 \right), \left(-13,2111;~0 \right)$]
%   \item  $y=\frac{1}{2} x^2 - x $
%    \hfill [$\left (0;~0 \right ), \left(0;~0 \right), \left(2;~0 \right)$]
 \end{enumeratea}
\end{esercizio}

% \begin{esercizio}\label{ese:}
%  Calcola le intersezioni con gli assi.
%  \begin{enumeratea}
%   \item  $y=\frac{4}{3} x^2 -2 x +9$
%    \hfill [$\left (0;~9 \right )$]
%   \item  $y=-\frac{5}{4} x^2 + x -3$
%    \hfill [$\left (0;~-3 \right )$]
%   \item  $y=-\frac{1}{5} x^2 +2 x +7$
%    \hfill [$\left (0;~7 \right ), \left(12,746;~0 \right), \left(-2,746;~0 
% \right)$]
%   \item  $y=-\frac{4}{3} x^2 - x -6$
%    \hfill [$\left (0;~-6 \right )$]
%   \item  $y=-\frac{3}{2} x^2 -3 x +4$
%    \hfill [$\left (0;~4 \right ), \left(0,9149;~0 \right), \left(-2,9149;~0 
% \right)$]
%   \item  $y=\frac{4}{3} x^2 - x -3$
%    \hfill [$\left (0;~-3 \right ), \left(-1,1712;~0 \right), \left(1,9212;~0 
% \right)$]
%   \item  $y=4 x^2 - x -4$
%    \hfill [$\left (0;~-4 \right ), \left(-0,8828;~0 \right), \left(1,1328;~0 
% \right)$]
%   \item  $y=-\frac{5}{2} x^2 - x +4$
%    \hfill [$\left (0;~4 \right ), \left(1,0806;~0 \right), \left(-1,4806;~0 
% \right)$]
%   \item  $y=-4 x^2 +2 x +7$
%    \hfill [$\left (0;~7 \right ), \left(1,5963;~0 \right), \left(-1,0963;~0 
% \right)$]
%   \item  $y=-\frac{3}{2} x^2 +2 x -7$
%    \hfill [$\left (0;~-7 \right )$]
%   \item  $y=-\frac{1}{5} x^2 + x -6$
%    \hfill [$\left (0;~-6 \right )$]
%   \item  $y=-\frac{3}{4} x^2 +3 x +8$
%    \hfill [$\left (0;~8 \right ), \left(5,8297;~0 \right), \left(-1,8297;~0 
% \right)$]
%   \item  $y=-\frac{5}{3} x^2 -3 x +5$
%    \hfill [$\left (0;~5 \right ), \left(1,0519;~0 \right), \left(-2,8519;~0 
% \right)$]
%   \item  $y=\frac{1}{4} x^2 -2 x -2$
%    \hfill [$\left (0;~-2 \right ), \left(-0,899;~0 \right), \left(8,899;~0 
% \right)$]
%   \item  $y=-\frac{4}{3} x^2 +4 x -1$
%    \hfill [$\left (0;~-1 \right ), \left(2,7247;~0 \right), \left(0,2753;~0 \right)$]
%  \end{enumeratea}
% \end{esercizio}

\begin{esercizio}\label{ese:}
 Calcola le intersezioni tra la retta e la parabola e disegnale.
 \begin{enumeratea}
  \item  $r:~y = 2 x +10,~p:~y=- x^2 -4 x +5$
   \hfill [$\left (-1;~8 \right ),~\left (-5;~0 \right )$]
  \item  $r:~y = 2 x -6,~p:~y=-2 x^2 -6$
   \hfill [$\left (0;~-6 \right ),~\left (-1;~-8 \right )$]
%   \item  $r:~x = -3,~p:~y=-\frac{1}{2} x^2 +2 x +5$
%    \hfill [$\left (-3;~-\frac{11}{2} \right ),~\left (-3;~-\frac{11}{2} 
% \right 
% )$]
%   \item  $r:~y = -2 x +4,~p:~y=2 x^2 -2 x +2$
%    \hfill [$\left (1;~2 \right ),~\left (-1;~6 \right )$]
%   \item  $r:~x = -3,~p:~y=-2 x^2 -4 x $
%    \hfill [$\left (-3;~-6 \right ),~\left (-3;~-6 \right )$]
%   \item  $r:~y = -7,~p:~y=-2 x^2 +4 x -1$
%    \hfill [$\left (-1;~-7 \right ),~\left (3;~-7 \right )$]
  \item  $r:~y = 6 x -13,~p:~y=x^2 -5$
   \hfill [$\left (2;~-1 \right ),~\left (4;~11 \right )$]
  \item  $r:~y = -6 x +3,~p:~y=-2 x^2 -1$
   \hfill [$\left (2;~-9 \right ),~\left (1;~-3 \right )$]
  \item  $r:~y = -2 x -5,~p:~y=-2 x^2 -1$
   \hfill [$\left (2;~-9 \right ),~\left (-1;~-3 \right )$]
  \item  $r:~y = -2 x -4,~p:~y=2 x^2 -4 x -4$
   \hfill [$\left (1;~-6 \right ),~\left (0;~-4 \right )$]
  \item  $r:~y = 11,~p:~y=2 x^2 -2 x -1$
   \hfill [$\left (-2;~11 \right ),~\left (3;~11 \right )$]
  \item  $r:~y = 5 x +5,~p:~y=x^2 +4 x +5$
   \hfill [$\left (0;~5 \right ),~\left (1;~10 \right )$]
%   \item  $r:~y = - x +1,~p:~y=x^2 +4 x +5$
%    \hfill [$\left (-1;~2 \right ),~\left (-4;~5 \right )$]
%   \item  $r:~y = \frac{1}{2} x -1,~p:~y=-\frac{1}{2} x^2 +2 x +4$
%    \hfill [$\left (5;~\frac{3}{2} \right ),~\left (-2;~-2 \right )$]
%   \item  $r:~x = -2,~p:~y=- x^2 -2 x -3$
%    \hfill [$\left (-2;~-3 \right ),~\left (-2;~-3 \right )$]
%   \item  $r:~y = 3 x -5,~p:~y=- x^2 +2 x -3$
%    \hfill [$\left (-2;~-11 \right ),~\left (1;~-2 \right )$]
%   \item  $r:~y = 2,~p:~y=\frac{1}{2} x^2 -6$
%    \hfill [$\left (4;~2 \right ),~\left (-4;~2 \right )$]
%   \item  $r:~y = -2 x -2,~p:~y=\frac{1}{2} x^2 +2 x +4$
%    \hfill [$\left (-6;~10 \right ),~\left (-2;~2 \right )$]
%   \item  $r:~x = -1,~p:~y=-2 x^2 +4 x +4$
%    \hfill [$\left (-1;~-2 \right ),~\left (-1;~-2 \right )$]
%   \item  $r:~y = \frac{1}{2} x +1,~p:~y=\frac{1}{2} x^2 -2 x -6$
%    \hfill [$\left (7;~\frac{9}{2} \right ),~\left (-2;~0 \right )$]
 \end{enumeratea}
\end{esercizio}

\begin{esercizio}\label{ese:}
 Calcola le intersezioni tra la parabola e la retta.
 \begin{enumeratea}
%   \item  $r:~y = -\frac{4}{3} x +6,~p:~y=-2 x^2 -\frac{52}{3} x -8$
%    \hfill [$\left (-1;~\frac{22}{3} \right ),~\left (-7;~\frac{46}{3} \right )$]
  \item  $r:~y = 3 x -3,~p:~y=\frac{9}{10} x^2 -\frac{3}{10} x $
   \hfill [$\left (2;~3 \right ),~\left (\frac{5}{3};~2 \right )$]
%   \item  $r:~y = \frac{3}{4} x +1,~p:~y=\frac{7}{72} x^2 +\frac{47}{72} x -6$
%    \hfill [$\left (-8;~-5 \right ),~\left (9;~\frac{31}{4} \right )$]
  \item  $r:~y = -2 x +4,~p:~y=-\frac{3}{8} x^2 -\frac{13}{2} x -8$
   \hfill [$\left (-4;~12 \right ),~\left (-8;~20 \right )$]
%   \item  $r:~y = -4 x -3,~p:~y=-\frac{1}{24} x^2 -\frac{41}{12} x -5$
%    \hfill [$\left (8;~-35 \right ),~\left (6;~-27 \right )$]
  \item  $r:~y = x -5,~p:~y=\frac{1}{12} x^2 -\frac{5}{12} x +1$
   \hfill [$\left (8;~3 \right ),~\left (9;~4 \right )$]
%   \item  $r:~y = 5 x -2,~p:~y=\frac{25}{7} x^2 -\frac{10}{7} x $
%    \hfill [$\left (\frac{2}{5};~0 \right ),~\left (\frac{7}{5};~5 \right )$]
  \item  $r:~y = \frac{1}{2} x -8,~p:~y=-\frac{13}{12} x^2 -\frac{7}{12} x +5$
   \hfill [$\left (3;~-\frac{13}{2} \right ),~\left (-4;~-10 \right )$]
%   \item  $r:~y = \frac{4}{3} x -9,~p:~y=-\frac{112}{27} x^2 +\frac{152}{9} x +5$
%    \hfill [$\left (-\frac{3}{4};~-10 \right ),~\left (\frac{9}{2};~-3 \right )$]
%   \item  $r:~y = -\frac{4}{3} x -1,~p:~y=-\frac{9}{8} x^2 -\frac{97}{12} x -10$
%    \hfill [$\left (-4;~\frac{13}{3} \right ),~\left (-2;~\frac{5}{3} \right )$]
  \item  $r:~y = \frac{2}{5} x -5,~p:~y=\frac{1}{9} x^2 +\frac{7}{5} x -3$
   \hfill [$\left (-6;~-\frac{37}{5} \right ),~\left (-3;~-\frac{31}{5} \right )$]
%   \item  $r:~y = -2 x +1,~p:~y=\frac{3}{25} x^2 -2 x -2$
%    \hfill [$\left (-5;~11 \right ),~\left (5;~-9 \right )$]
%   \item  $r:~y = \frac{4}{5} x -1,~p:~y=\frac{4}{63} x^2 -\frac{68}{315} x +3$
%    \hfill [$\left (7;~\frac{23}{5} \right ),~\left (9;~\frac{31}{5} \right )$]
%   \item  $r:~y = - x +3,~p:~y=-\frac{6}{7} x^2 -\frac{43}{7} x +9$
%    \hfill [$\left (1;~2 \right ),~\left (-7;~10 \right )$]
%  \end{enumeratea}
% \end{esercizio}
% 
% 
% \begin{esercizio}\label{ese:}
%  Calcola le intersezioni tra la parabola e la retta (2).
%  \begin{enumeratea}
%   \item  $r:~y = -\frac{2}{7} x +2,~p:~y=-\frac{3}{224} x^2 -\frac{2}{7} x +\frac{85}{32}$
%    \hfill [$\left (-7;~4 \right ),~\left (7;~0 \right )$]
  \item  $r:~y = -17 x -10,~p:~y=-11 x^2 -28 x -10$
   \hfill [$\left (-1;~7 \right ),~\left (0;~-10 \right )$]
%   \item  $r:~y = 14 x -80,~p:~y=\frac{37}{10} x^2 -\frac{267}{10} x +31$
%    \hfill [$\left (5;~-10 \right ),~\left (6;~4 \right )$]
  \item  $r:~y = x +7,~p:~y=- x^2 -7 x -8$
   \hfill [$\left (-3;~4 \right ),~\left (-5;~2 \right )$]
%   \item  $r:~y = -\frac{5}{2} x -9,~p:~y=x^2 -\frac{1}{2} x -9$
%    \hfill [$\left (0;~-9 \right ),~\left (-2;~-4 \right )$]
%   \item  $r:~y = \frac{1}{4} x -\frac{39}{4},~p:~y=\frac{1}{12} x^2 +\frac{1}{12} x -10$
%    \hfill [$\left (3;~-9 \right ),~\left (-1;~-10 \right )$]
%   \item  $r:~y = -\frac{5}{4} x -1,~p:~y=\frac{3}{128} x^2 -\frac{17}{16} x -1$
%    \hfill [$\left (0;~-1 \right ),~\left (-8;~9 \right )$]
%   \item  $r:~y = -\frac{1}{6} x -\frac{22}{3},~p:~y=\frac{85}{42} x^2 -\frac{59}{14} x -\frac{494}{21}$
%    \hfill [$\left (4;~-8 \right ),~\left (-2;~-7 \right )$]
%   \item  $r:~y = 1,~p:~y=- x^2 +9 x -13$
%    \hfill [$\left (2;~1 \right ),~\left (7;~1 \right )$]
%   \item  $r:~y = -\frac{9}{2} x +11,~p:~y=\frac{49}{30} x^2 -\frac{143}{10} x +\frac{361}{15}$
%    \hfill [$\left (2;~2 \right ),~\left (4;~-7 \right )$]
%   \item  $r:~y = -\frac{7}{10} x -\frac{43}{5},~p:~y=\frac{7}{170} x^2 -\frac{77}{170} x -\frac{787}{85}$
%    \hfill [$\left (-8;~-3 \right ),~\left (2;~-10 \right )$]
%   \item  $r:~y = \frac{3}{4} x -\frac{11}{4},~p:~y=-\frac{13}{44} x^2 +\frac{7}{44} x +\frac{167}{22}$
%    \hfill [$\left (-7;~-8 \right ),~\left (5;~1 \right )$]
%   \item  $r:~y = 10 x -78,~p:~y=\frac{13}{4} x^2 -\frac{155}{4} x +104$
%    \hfill [$\left (8;~2 \right ),~\left (7;~-8 \right )$]
%   \item  $r:~y = x ,~p:~y=\frac{3}{14} x^2 -\frac{1}{14} x -\frac{36}{7}$
%    \hfill [$\left (-3;~-3 \right ),~\left (8;~8 \right )$]
%   \item  $r:~y = \frac{3}{11} x +\frac{12}{11},~p:~y=-\frac{5}{154} x^2 +\frac{57}{154} x +2$
%    \hfill [$\left (-4;~0 \right ),~\left (7;~3 \right )$]
%   \item  $r:~y = -3 x -4,~p:~y=\frac{1}{4} x^2 -\frac{15}{4} x -\frac{7}{2}$
%    \hfill [$\left (2;~-10 \right ),~\left (1;~-7 \right )$]
%   \item  $r:~y = \frac{7}{17} x -\frac{56}{17},~p:~y=-\frac{57}{884} x^2 +\frac{307}{884} x +\frac{298}{221}$
%    \hfill [$\left (8;~0 \right ),~\left (-9;~-7 \right )$]
%   \item  $r:~y = \frac{1}{6} x +\frac{1}{6},~p:~y=-\frac{1}{72} x^2 +\frac{1}{18} x +\frac{5}{72}$
%    \hfill [$\left (-1;~0 \right ),~\left (-7;~-1 \right )$]
%   \item  $r:~y = \frac{3}{5} x -\frac{3}{5},~p:~y=\frac{29}{70} x^2 +\frac{129}{70} x -\frac{79}{35}$
%    \hfill [$\left (-4;~-3 \right ),~\left (1;~0 \right )$]
  \item  $r:~y = -\frac{2}{3} x -\frac{17}{3},~p:~y=-\frac{1}{24} x^2 -\frac{3}{4} x -\frac{16}{3}$
   \hfill [$\left (2;~-7 \right ),~\left (-4;~-3 \right )$]
 \end{enumeratea}
\end{esercizio}

\begin{esercizio}\label{ese:}
 Disegna le due parabole e calcola le loro intersezioni
 \begin{enumeratea}
%   \item  $p_0:~y=- x^2 ;~p_1:~y=-\frac{1}{2} x^2 +4 x +1$
%    \hfill [$\left (-7.7417;~-59.9333 \right ),~\left (-0.2583;~-0.0667 \right 
% )$]
  \item  $p_0:~y=2 x^2 +2 x -2;~p_1:~y=-\frac{1}{2} x^2 -2 x -1$
   \hfill [$\left (0.2198;~-1.4638 \right ),~\left (-1.8198;~0.9838 \right )$]
  \item  $p_0:~y=- x^2 +2 x -3;~p_1:~y=-2 x^2 -2 x +2$
   \hfill [$\left (1;~-2 \right ),~\left (-5;~-38 \right )$]
%   \item  $p_0:~y=x^2 +4 x -4;~p_1:~y=-2 x^2 -2 x -3$
%    \hfill [$\left (0.1547;~-3.3573 \right ),~\left (-2.1547;~-7.9761 \right )$]
%   \item  $p_0:~y=x^2 -4 x -4;~p_1:~y=-2 x^2 +4 x $
%    \hfill [$\left (3.0972;~-6.7962 \right ),~\left (-0.4305;~-2.0927 \right )$]
  \item  $p_0:~y=2 x^2 -2 x +1;~p_1:~y=x^2 -4$
   \hfill [No inters. reali]
%   \item  $p_0:~y=\frac{1}{3} x^2 +4 x -1;~p_1:~y=-\frac{1}{3} x^2 -4 x -3$
%    \hfill [$\left (-0.2554;~-2.0 \right ),~\left (-11.7446;~-2.0 \right )$]
  \item  $p_0:~y=-\frac{1}{3} x^2 +5;~p_1:~y=\frac{1}{2} x^2 +2 x +2$
   \hfill [$\left (-3.445;~1.044 \right ),~\left (1.045;~4.636 \right )$]
  \item  $p_0:~y=2 x^2 -4 x +1;~p_1:~y=x^2 -1$
   \hfill [$\left (3.4142;~10.6569 \right ),~\left (0.5858;~-0.6569 \right )$]
%   \item  $p_0:~y=-\frac{1}{3} x^2 -5;~p_1:~y=-\frac{1}{2} x^2 +2 x -2$
%    \hfill [$\left (13.3485;~-64.3939 \right ),~\left (-1.3485;~-5.6061 \right 
% )$]
  \item  $p_0:~y=\frac{1}{2} x^2 +4 x -3;~p_1:~y=\frac{1}{2} x^2 $
   \hfill [$\left (\frac{3}{4};~\frac{9}{32} \right )$]
  \item  $p_0:~y=- x^2 +2 x +1;~p_1:~y=\frac{1}{3} x^2 -2 x -2$
   \hfill [$\left (-0.6213;~-0.6287 \right ),~\left (3.6213;~-4.8713 \right )$]
  \item  $p_0:~y=-\frac{1}{2} x^2 +2 x -3;~p_1:~y=-\frac{1}{2} x^2 +5$
   \hfill [$\left (4;~-3 \right )$]
  \item  $p_0:~y=x^2 +4 x -4;~p_1:~y=- x^2 +4 x +3$
   \hfill [$\left (1.8708;~6.9833 \right ),~\left (-1.8708;~-7.9833 \right )$]
  \item  $p_0:~y=x^2 +2 x -1;~p_1:~y=\frac{1}{2} x^2 +4 x +5$
   \hfill [$\left (6;~47 \right ),~\left (-2;~-1 \right )$]
  \item  $p_0:~y=2 x^2 -4 x +5;~p_1:~y=\frac{1}{2} x^2 +1$
   \hfill [No inters. reali]
  \item  $p_0:~y=-2 x^2 +2 x -3;~p_1:~y=x^2 -4 x +4$
   \hfill [No inters. reali]
%   \item  $p_0:~y=2 x^2 +2 x +5;~p_1:~y=-\frac{1}{3} x^2 +2 x $
%    \hfill [No inters. reali]
  \item  $p_0:~y=-2 x^2 -2 x -1;~p_1:~y=-2 x^2 -4 x +2$
   \hfill [$\left (\frac{3}{2};~-\frac{17}{2} \right )$]
%   \item  $p_0:~y=-\frac{1}{2} x^2 +4 x -3;~p_1:~y=- x^2 -2 x -6$
%    \hfill [$\left (-0.5228;~-5.2277 \right ),~\left (-11.4772;~-114.7723 \right 
% )$]
 \end{enumeratea}
\end{esercizio}

\begin{esercizio}\label{ese:}
 Calcola le intersezioni tra le due parabole.
 \begin{enumeratea}
  \item  $p_0:~y=2 x^2 - x +6,~p_1:~y=- x^2 - x +9$
   \hfill [$\left (1;~7 \right ),~\left (-1;~9 \right )$]
  \item  $p_0:~y=- x^2 -2 x -7,~p_1:~y=\frac{4}{3} x^2 +\frac{22}{3} x $
   \hfill [$\left (-1;~-6 \right ),~\left (-3;~-10 \right )$]
  \item  $p_0:~y=\frac{3}{4} x^2 + x +6,~p_1:~y=-\frac{27}{4} x^2 -\frac{43}{2} x -9$
   \hfill [$\left (-2;~7 \right ),~\left (-1;~\frac{23}{4} \right )$]
  \item  $p_0:~y=-3 x^2 -8,~p_1:~y=-8 x^2 -3$
   \hfill [$\left (-1;~-11 \right ),~\left (1;~-11 \right )$]
  \item  $p_0:~y=- x^2 +3,~p_1:~y=-\frac{1}{2} x^2 +1$
   \hfill [$\left (-2;~-1 \right ),~\left (2;~-1 \right )$]
  \item  $p_0:~y=x^2 +2 x +2,~p_1:~y=- x^2 +6$
   \hfill [$\left (-2;~2 \right ),~\left (1;~5 \right )$]
  \item  $p_0:~y=x^2 - x -8,~p_1:~y=-\frac{3}{2} x^2 -\frac{7}{2} x -3$
   \hfill [$\left (-2;~-2 \right ),~\left (1;~-8 \right )$]
  \item  $p_0:~y=\frac{1}{2} x^2 -4 x -9,~p_1:~y=\frac{55}{42} x^2 -\frac{254}{21} x +8$
   \hfill [$\left (3;~-\frac{33}{2} \right ),~\left (7;~-\frac{25}{2} \right )$]
%   \item  $p_0:~y=4 x^2 +3 x -8,~p_1:~y=3 x -4$
%    \hfill [$\left (1;~-1 \right ),~\left (-1;~-7 \right )$]
%   \item  $p_0:~y=-\frac{4}{3} x^2 +2 x -10,~p_1:~y=-\frac{31}{3} x^2 +2 x -1$
%    \hfill [$\left (1;~-\frac{28}{3} \right ),~\left (-1;~-\frac{40}{3} \right )$]
%   \item  $p_0:~y=-\frac{3}{4} x^2 + x -6,~p_1:~y=\frac{9}{4} x^2 +10 x $
%    \hfill [$\left (-2;~-11 \right ),~\left (-1;~-\frac{31}{4} \right )$]
%   \item  $p_0:~y=-\frac{2}{3} x^2 -3 x -10,~p_1:~y=-\frac{7}{6} x^2 -\frac{9}{2} x -8$
%    \hfill [$\left (-4;~-\frac{26}{3} \right ),~\left (1;~-\frac{41}{3} \right )$]
%   \item  $p_0:~y=x^2 +2 x -5,~p_1:~y=-11 x^2 +2 x +7$
%    \hfill [$\left (1;~-2 \right ),~\left (-1;~-6 \right )$]
%   \item  $p_0:~y=\frac{3}{2} x^2 -5 x -4,~p_1:~y=5 x^2 -\frac{31}{2} x +3$
%    \hfill [$\left (2;~-8 \right ),~\left (1;~-\frac{15}{2} \right )$]
%   \item  $p_0:~y=\frac{3}{2} x^2 -2 x +3,~p_1:~y=\frac{19}{2} x^2 -2 x -5$
%    \hfill [$\left (1;~\frac{5}{2} \right ),~\left (-1;~\frac{13}{2} \right )$]
%   \item  $p_0:~y=\frac{1}{2} x^2 +2 x -3,~p_1:~y=\frac{5}{4} x^2 +\frac{23}{4} x $
%    \hfill [$\left (-4;~-3 \right ),~\left (-1;~-\frac{9}{2} \right )$]
%   \item  $p_0:~y=-\frac{5}{2} x^2 -4,~p_1:~y=-\frac{21}{2} x^2 +4$
%    \hfill [$\left (1;~-\frac{13}{2} \right ),~\left (-1;~-\frac{13}{2} \right )$]
%   \item  $p_0:~y=-\frac{4}{3} x^2 +2 x +2,~p_1:~y=\frac{32}{3} x^2 +2 x -10$
%    \hfill [$\left (1;~\frac{8}{3} \right ),~\left (-1;~-\frac{4}{3} \right )$]
%   \item  $p_0:~y=-\frac{1}{2} x^2 + x -8,~p_1:~y=-\frac{1}{10} x^2 -\frac{7}{5} x -6$
%    \hfill [$\left (5;~-\frac{31}{2} \right ),~\left (1;~-\frac{15}{2} \right )$]
%   \item  $p_0:~y=- x^2 -4 x +8,~p_1:~y=-\frac{29}{12} x^2 -\frac{167}{12} x -9$
%    \hfill [$\left (-3;~11 \right ),~\left (-4;~8 \right )$]
 \end{enumeratea}
\end{esercizio}


% \begin{esercizio}\label{ese:}
%  Calcola le intersezioni tra le due parabole (2)
%  \begin{enumeratea}
%   \item  $p_0:~y=-\frac{13}{280} x^2 -\frac{1}{10} x +\frac{317}{56},~p_1:~y=-\frac{27}{280} x^2 +\frac{3}{5} x +\frac{191}{56}$
%    \hfill [$\left (5;~4 \right ),~\left (-5;~5 \right )$]
%   \item  $p_0:~y=-\frac{17}{40} x^2 -\frac{63}{20} x +3,~p_1:~y=-\frac{23}{40} x^2 -\frac{87}{20} x +6$
%    \hfill [$\left (-10;~-8 \right ),~\left (-8;~1 \right )$]
%   \item  $p_0:~y=-\frac{7}{12} x^2 +\frac{1}{4} x +\frac{16}{3},~p_1:~y=-\frac{5}{18} x^2 +\frac{7}{6} x +\frac{37}{9}$
%    \hfill [$\left (-4;~-5 \right ),~\left (5;~-8 \right )$]
%   \item  $p_0:~y=\frac{7}{171} x^2 +\frac{70}{171} x -5,~p_1:~y=\frac{151}{1710} x^2 -\frac{29}{1710} x -5$
%    \hfill [$\left (9;~2 \right ),~\left (-10;~-5 \right )$]
%   \item  $p_0:~y=-\frac{1}{44} x^2 +\frac{57}{44} x -\frac{21}{11},~p_1:~y=-\frac{31}{132} x^2 +\frac{227}{132} x +\frac{14}{11}$
%    \hfill [$\left (-3;~-6 \right ),~\left (8;~7 \right )$]
%   \item  $p_0:~y=-\frac{11}{36} x^2 +\frac{13}{12} x +\frac{15}{2},~p_1:~y=-\frac{7}{36} x^2 +\frac{1}{12} x +\frac{19}{2}$
%    \hfill [$\left (6;~3 \right ),~\left (-6;~-10 \right )$]
%   \item  $p_0:~y=-\frac{1}{14} x^2 +\frac{11}{14} x +\frac{48}{7},~p_1:~y=-\frac{153}{182} x^2 -\frac{417}{182} x +\frac{3774}{91}$
%    \hfill [$\left (5;~9 \right ),~\left (-8;~-4 \right )$]
%   \item  $p_0:~y=2 x^2 +9 x -1,~p_1:~y=\frac{23}{8} x^2 +16 x +\frac{97}{8}$
%    \hfill [$\left (-3;~-10 \right ),~\left (-1;~-8 \right )$]
%   \item  $p_0:~y=-\frac{1}{9} x^2 -\frac{7}{9} x +\frac{44}{9},~p_1:~y=\frac{1}{3} x^2 -\frac{1}{3} x -20$
%    \hfill [$\left (-8;~4 \right ),~\left (-5;~6 \right )$]
%   \item  $p_0:~y=\frac{59}{480} x^2 -\frac{41}{60} x -\frac{1409}{160},~p_1:~y=\frac{83}{480} x^2 -\frac{17}{60} x -\frac{1353}{160}$
%    \hfill [$\left (-7;~2 \right ),~\left (9;~-5 \right )$]
%   \item  $p_0:~y=\frac{1}{3} x^2 -\frac{1}{3} x -8,~p_1:~y=\frac{7}{24} x^2 -\frac{13}{24} x -\frac{33}{4}$
%    \hfill [$\left (-3;~-4 \right ),~\left (6;~2 \right )$]
%   \item  $p_0:~y=-\frac{19}{60} x^2 +\frac{67}{30} x +\frac{31}{20},~p_1:~y=-\frac{2}{5} x^2 +\frac{29}{10} x +\frac{23}{10}$
%    \hfill [$\left (-1;~-1 \right ),~\left (-3;~-8 \right )$]
%   \item  $p_0:~y=\frac{29}{280} x^2 -\frac{61}{70} x -\frac{461}{56},~p_1:~y=\frac{17}{140} x^2 -\frac{51}{70} x -\frac{235}{28}$
%    \hfill [$\left (-9;~8 \right ),~\left (5;~-10 \right )$]
%   \item  $p_0:~y=-\frac{1}{21} x^2 +\frac{64}{21} x -10,~p_1:~y=\frac{20}{21} x^2 -\frac{104}{21} x -3$
%    \hfill [$\left (1;~-7 \right ),~\left (0;~-10 \right )$]
%   \item  $p_0:~y=-\frac{9}{220} x^2 +\frac{291}{220} x +\frac{9}{22},~p_1:~y=-\frac{1}{44} x^2 +\frac{47}{44} x +\frac{27}{22}$
%    \hfill [$\left (9;~9 \right ),~\left (-6;~-9 \right )$]
%   \item  $p_0:~y=\frac{31}{15} x^2 -19 x +\frac{326}{15},~p_1:~y=\frac{11}{15} x^2 -7 x +\frac{46}{15}$
%    \hfill [$\left (7;~-10 \right ),~\left (8;~2 \right )$]
%   \item  $p_0:~y=\frac{37}{63} x^2 -\frac{11}{21} x -\frac{712}{63},~p_1:~y=\frac{8}{21} x^2 +\frac{5}{7} x -\frac{272}{21}$
%    \hfill [$\left (4;~-4 \right ),~\left (-5;~6 \right )$]
%   \item  $p_0:~y=\frac{5}{128} x^2 -\frac{9}{64} x +\frac{397}{128},~p_1:~y=-\frac{1}{128} x^2 +\frac{21}{64} x +\frac{343}{128}$
%    \hfill [$\left (1;~3 \right ),~\left (-7;~6 \right )$]
%   \item  $p_0:~y=\frac{3}{2} x^2 +\frac{23}{2} x +19,~p_1:~y=\frac{11}{2} x^2 +\frac{63}{2} x +35$
%    \hfill [$\left (-4;~-3 \right ),~\left (-3;~-2 \right )$]
%   \item  $p_0:~y=-\frac{5}{12} x^2 -\frac{35}{12} x +\frac{15}{2},~p_1:~y=-\frac{113}{156} x^2 -\frac{839}{156} x +\frac{355}{26}$
%    \hfill [$\left (-10;~-5 \right ),~\left (3;~-5 \right )$]
%  \end{enumeratea}
% \end{esercizio}

\begin{esercizio}\label{ese:}
 Calcola l'equazione della parabola passante per i tre punti.
 \begin{enumeratea}
  \item  $p_0 \left (0;~0 \right ),~p_1 \left (2;~0 \right ),~p_2 \left (1;~-2 \right )$
   \hfill [$y=2 x^2 -4 x $]
  \item  $p_0 \left (0;~-1 \right ),~p_1 \left (-1;~3 \right ),~p_2 \left (1;~-1 \right )$
   \hfill [$y=2 x^2 -2 x -1$]
  \item  $p_0 \left (-1;~0 \right ),~p_1 \left (0;~-2 \right ),~p_2 \left (1;~0 \right )$
   \hfill [$y=2 x^2 -2$]
  \item  $p_0 \left (2;~1 \right ),~p_1 \left (0;~1 \right ),~p_2 \left (1;~3 \right )$
   \hfill [$y=-2 x^2 +4 x +1$]
  \item  $p_0 \left (6;~14 \right ),~p_1 \left (14;~-\frac{22}{3} \right ),~p_2 \left (0;~2 \right )$
   \hfill [$y=-\frac{1}{3} x^2 +4 x +2$]
  \item  $p_0 \left (0;~-3 \right ),~p_1 \left (-1;~1 \right ),~p_2 \left (1;~-3 \right )$
   \hfill [$y=2 x^2 -2 x -3$]
  \item  $p_0 \left (-1;~-8 \right ),~p_1 \left (1;~0 \right ),~p_2 \left (2;~1 \right )$
   \hfill [$y=- x^2 +4 x -3$]
  \item  $p_0 \left (7;~\frac{49}{3} \right ),~p_1 \left (2;~\frac{4}{3} \right ),~p_2 \left (3;~3 \right )$
   \hfill [$y=\frac{1}{3} x^2 $]
  \item  $p_0 \left (0;~-1 \right ),~p_1 \left (1;~-5 \right ),~p_2 \left (-1;~-1 \right )$
   \hfill [$y=-2 x^2 -2 x -1$]
  \item  $p_0 \left (4;~3 \right ),~p_1 \left (0;~-5 \right ),~p_2 \left (-1;~-2 \right )$
   \hfill [$y=x^2 -2 x -5$]
%   \item  $p_0 \left (3;~-\frac{13}{2} \right ),~p_1 \left (2;~-5 \right ),~p_2 \left (-2;~11 \right )$
%    \hfill [$y=\frac{1}{2} x^2 -4 x +1$]
%   \item  $p_0 \left (-5;~-\frac{28}{3} \right ),~p_1 \left (-8;~-\frac{67}{3} \right ),~p_2 \left (-6;~-13 \right )$
%    \hfill [$y=-\frac{1}{3} x^2 -1$]
%   \item  $p_0 \left (11;~-\frac{1}{3} \right ),~p_1 \left (14;~-\frac{40}{3} \right ),~p_2 \left (10;~\frac{8}{3} \right )$
%    \hfill [$y=-\frac{1}{3} x^2 +4 x -4$]
%   \item  $p_0 \left (0;~-1 \right ),~p_1 \left (-1;~-5 \right ),~p_2 \left (1;~-1 \right )$
%    \hfill [$y=-2 x^2 +2 x -1$]
%   \item  $p_0 \left (2;~-\frac{11}{3} \right ),~p_1 \left (6;~-1 \right ),~p_2 \left (7;~\frac{4}{3} \right )$
%    \hfill [$y=\frac{1}{3} x^2 -2 x -1$]
%   \item  $p_0 \left (-7;~-\frac{31}{2} \right ),~p_1 \left (-1;~-\frac{7}{2} \right ),~p_2 \left (2;~-11 \right )$
%    \hfill [$y=-\frac{1}{2} x^2 -2 x -5$]
%   \item  $p_0 \left (-2;~-1 \right ),~p_1 \left (0;~-1 \right ),~p_2 \left (-1;~1 \right )$
%    \hfill [$y=-2 x^2 -4 x -1$]
%   \item  $p_0 \left (-10;~-\frac{43}{3} \right ),~p_1 \left (-6;~-1 \right ),~p_2 \left (-1;~\frac{2}{3} \right )$
%    \hfill [$y=-\frac{1}{3} x^2 -2 x -1$]
%   \item  $p_0 \left (0;~-3 \right ),~p_1 \left (2;~-5 \right ),~p_2 \left (-3;~-\frac{15}{2} \right )$
%    \hfill [$y=-\frac{1}{2} x^2 -3$]
%   \item  $p_0 \left (-15;~14 \right ),~p_1 \left (-4;~-\frac{35}{3} \right ),~p_2 \left (-7;~-\frac{38}{3} \right )$
%    \hfill [$y=\frac{1}{3} x^2 +4 x -1$]
 \end{enumeratea}
\end{esercizio}


\begin{esercizio}\label{ese:}
 Calcola l'equazione della parabola dati il vertice e un punto.
 \begin{enumeratea}
%   \item  $v \left (0;~3 \right ),~p \left (-9;~30 \right )$
%    \hfill [$y=\frac{1}{3} x^2 +3$]
  \item  $v \left (2;~6 \right ),~p \left (4;~2 \right )$
   \hfill [$y=- x^2 +4 x +2$]
%   \item  $v \left (-3;~-7 \right ),~p \left (-2;~-\frac{20}{3} \right )$
%    \hfill [$y=\frac{1}{3} x^2 +2 x -4$]
  \item  $v \left (1;~-4 \right ),~p \left (0;~-2 \right )$
   \hfill [$y=2 x^2 -4 x -2$]
  \item  $v \left (-2;~-4 \right ),~p \left (-1;~-3 \right )$
   \hfill [$y=x^2 +4 x $]
  \item  $v \left (-2;~-3 \right ),~p \left (1;~\frac{3}{2} \right )$
   \hfill [$y=\frac{1}{2} x^2 +2 x -1$]
  \item  $v \left (-1;~2 \right ),~p \left (-2;~0 \right )$
   \hfill [$y=-2 x^2 -4 x $]
%   \item  $v \left (-4;~3 \right ),~p \left (1;~-\frac{19}{2} \right )$
%    \hfill [$y=-\frac{1}{2} x^2 -4 x -5$]
%   \item  $v \left (-3;~2 \right ),~p \left (2;~-\frac{19}{3} \right )$
%    \hfill [$y=-\frac{1}{3} x^2 -2 x -1$]
  \item  $v \left (-2;~4 \right ),~p \left (1;~-\frac{1}{2} \right )$
   \hfill [$y=-\frac{1}{2} x^2 -2 x +2$]
  \item  $v \left (0;~2 \right ),~p \left (8;~-\frac{58}{3} \right )$
   \hfill [$y=-\frac{1}{3} x^2 +2$]
  \item  $v \left (6;~16 \right ),~p \left (2;~\frac{32}{3} \right )$
   \hfill [$y=-\frac{1}{3} x^2 +4 x +4$]
%   \item  $v \left (2;~1 \right ),~p \left (7;~-\frac{23}{2} \right )$
%    \hfill [$y=-\frac{1}{2} x^2 +2 x -1$]
  \item  $v \left (-6;~-16 \right ),~p \left (3;~11 \right )$
   \hfill [$y=\frac{1}{3} x^2 +4 x -4$]
%   \item  $v \left (-6;~13 \right ),~p \left (-7;~\frac{38}{3} \right )$
%    \hfill [$y=-\frac{1}{3} x^2 -4 x +1$]
%   \item  $v \left (-4;~-5 \right ),~p \left (-1;~-\frac{1}{2} \right )$
%    \hfill [$y=\frac{1}{2} x^2 +4 x +3$]
  \item  $v \left (0;~-5 \right ),~p \left (2;~-3 \right )$
   \hfill [$y=\frac{1}{2} x^2 -5$]
%   \item  $v \left (-1;~-1 \right ),~p \left (1;~-5 \right )$
%    \hfill [$y=- x^2 -2 x -2$]
%   \item  $v \left (0;~-3 \right ),~p \left (1;~-\frac{5}{2} \right )$
%    \hfill [$y=\frac{1}{2} x^2 -3$]
%   \item  $v \left (-2;~1 \right ),~p \left (0;~3 \right )$
%    \hfill [$y=\frac{1}{2} x^2 +2 x +3$]
 \end{enumeratea}
\end{esercizio}

\begin{esercizio}\label{ese:}
 Calcola l'equazione della tangente in un punto della parabola di ascissa data.
 \begin{enumeratea}
  \item  $y=-\frac{1}{2} x^2 - x +3,~x_P=1$
   \hfill [$y = -2 x +\frac{7}{2}$]
  \item  $y=-2 x^2 -2 x +2,~x_P=1$
   \hfill [$y = -6 x +4$]
  \item  $y=x^2 -2 x ,~x_P=-4$
   \hfill [$y = -10 x -16$]
  \item  $y=x^2 -2 x -1,~x_P=-5$
   \hfill [$y = -12 x -26$]
  \item  $y=-\frac{1}{2} x^2 + x -2,~x_P=4$
   \hfill [$y = -3 x +6$]
  \item  $y=- x^2 +2 x -4,~x_P=3$
   \hfill [$y = -4 x +5$]
  \item  $y=2 x^2 - x -1,~x_P=-4$
   \hfill [$y = -17 x -33$]
  \item  $y=2 x^2 + x ,~x_P=-4$
   \hfill [$y = -15 x -32$]
  \item  $y=- x^2 + x -4,~x_P=-1$
   \hfill [$y = 3 x -3$]
  \item  $y=\frac{1}{2} x^2 - x +4,~x_P=-5$
   \hfill [$y = -6 x -\frac{17}{2}$]
%   \item  $y=\frac{1}{2} x^2 + x +4,~x_P=-2$
%    \hfill [$y = - x +2$]
%   \item  $y=\frac{1}{2} x^2 - x -4,~x_P=-4$
%    \hfill [$y = -5 x -12$]
%   \item  $y=- x^2 + x -3,~x_P=-5$
%    \hfill [$y = 11 x +22$]
%   \item  $y=-2 x^2 + x -5,~x_P=3$
%    \hfill [$y = -11 x +13$]
%   \item  $y=-\frac{1}{2} x^2 +2 x +4,~x_P=-1$
%    \hfill [$y = 3 x +\frac{9}{2}$]
%   \item  $y=- x^2 - x -3,~x_P=-5$
%    \hfill [$y = 9 x +22$]
%   \item  $y=\frac{1}{2} x^2 - x +1,~x_P=0$
%    \hfill [$y = - x +1$]
%   \item  $y=- x^2 - x +1,~x_P=4$
%    \hfill [$y = -9 x +17$]
%   \item  $y=\frac{1}{2} x^2 + x +1,~x_P=1$
%    \hfill [$y = 2 x +\frac{1}{2}$]
%   \item  $y=x^2 -2 x +4,~x_P=-5$
%    \hfill [$y = -12 x -21$]
 \end{enumeratea}
\end{esercizio}

\begin{esercizio}\label{ese:}
 Calcola le equazioni delle tangente alla parabola passanti per un punto 
 dato e i punti di tangenza.
 \begin{enumeratea}
%   \item  $y=2 x^2 - x ,~P \left (\frac{19}{2};~\frac{341}{2} \right )$
%    \hfill [$y = 35 x -162;~y = 39 x -200;~P_0 \left (9;~153 \right );~P_1 \left (10;~190 \right )$]
%   \item  $y=- x^2 +2 x -2,~P \left (-6;~-49 \right )$
%    \hfill [$y = 12 x +23;~y = 16 x +47;~P_0 \left (-5;~-37 \right );~P_1 \left (-7;~-65 \right )$]
%   \item  $y=2 x^2 +2 x -5,~P \left (-\frac{13}{2};~66 \right )$
%    \hfill [$y = -22 x -77;~y = -26 x -103;~P_0 \left (-6;~55 \right );~P_1 \left (-7;~79 \right )$]
  \item  $y=\frac{1}{2} x^2 - x -3,~P \left (5;~4 \right )$
   \hfill [$y = 3 x -11;~y = 5 x -21;~P_0 \left (4;~1 \right );~P_1 \left (6;~9 \right )$]
  \item  $y=\frac{1}{2} x^2 - x +3,~P \left (-\frac{5}{2};~\frac{17}{2} \right )$
   \hfill [$y = -3 x +1;~y = -4 x -\frac{3}{2};~P_0 \left (-2;~7 \right );~P_1 \left (-3;~\frac{21}{2} \right )$]
%   \item  $y=-\frac{1}{2} x^2 + x +4,~P \left (-\frac{15}{2};~-\frac{63}{2} \right )$
%    \hfill [$y = 9 x +36;~y = 8 x +\frac{57}{2};~P_0 \left (-8;~-36 \right );~P_1 \left (-7;~-\frac{55}{2} \right )$]
%   \item  $y=- x^2 - x -5,~P \left (-\frac{13}{2};~-\frac{81}{2} \right )$
%    \hfill [$y = 11 x +31;~y = 13 x +44;~P_0 \left (-6;~-35 \right );~P_1 \left (-7;~-47 \right )$]
%   \item  $y=- x^2 +2 x +3,~P \left (10;~-76 \right )$
%    \hfill [$y = -16 x +84;~y = -20 x +124;~P_0 \left (9;~-60 \right );~P_1 \left (11;~-96 \right )$]
  \item  $y=\frac{1}{2} x^2 - x +2,~P \left (-\frac{3}{2};~\frac{9}{2} \right )$
   \hfill [$y = -3 x ;~y = -2 x +\frac{3}{2};~P_0 \left (-2;~6 \right );~P_1 \left (-1;~\frac{7}{2} \right )$]
  \item  $y=- x^2 -2 x +2,~P \left (-\frac{5}{2};~1 \right )$
   \hfill [$y = 2 x +6;~y = 4 x +11;~P_0 \left (-2;~2 \right );~P_1 \left (-3;~-1 \right )$]
%   \item  $y=2 x^2 +2 x -4,~P \left (7;~106 \right )$
%    \hfill [$y = 34 x -132;~y = 26 x -76;~P_0 \left (8;~140 \right );~P_1 \left (6;~80 \right )$]
%   \item  $y=-\frac{1}{2} x^2 +2 x -4,~P \left (\frac{13}{2};~-12 \right )$
%    \hfill [$y = -5 x +\frac{41}{2};~y = -4 x +14;~P_0 \left (7;~-\frac{29}{2} \right );~P_1 \left (6;~-10 \right )$]
  \item  $y=\frac{1}{2} x^2 -2 x +1,~P \left (\frac{3}{2};~-1 \right )$
   \hfill [$y = -1;~y = - x +\frac{1}{2};~P_0 \left (2;~-1 \right );~P_1 \left (1;~-\frac{1}{2} \right )$]
  \item  $y=- x^2 -2 x -4,~P \left (-\frac{1}{2};~-3 \right )$
   \hfill [$y = -2 x -4;~y = -3;~P_0 \left (0;~-4 \right );~P_1 \left (-1;~-3 \right )$]
  \item  $y=- x^2 -2 x +3,~P \left (-3;~1 \right )$
   \hfill [$y = 2 x +7;~y = 6 x +19;~P_0 \left (-2;~3 \right );~P_1 \left (-4;~-5 \right )$]
%   \item  $y=- x^2 +2 x -4,~P \left (-6;~-51 \right )$
%    \hfill [$y = 12 x +21;~y = 16 x +45;~P_0 \left (-5;~-39 \right );~P_1 \left (-7;~-67 \right )$]
 \end{enumeratea}
\end{esercizio}


