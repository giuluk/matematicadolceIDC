% (c) 2014 Daniele Zambelli - daniele.zambelli@gmail.com

\section{Esercizi}

\subsection{Esercizi dei singoli paragrafi}

\subsubsection*{\numnameref{sec:retta_equazionilineari}}

\begin{esercizio}\label{ese:}
 Individua quale tra i seguenti punti appartiene alla retta.
 \begin{enumeratea}
  \item  $y = \frac{3}{2} x +\frac{33}{2};\quad \quad (-5;~9),~(4;~-10),~(4;~-9),~(5;~-9),~(-6;~9),~(5;~-10)$
  \item  $y = -\frac{13}{2} x -37;\quad \quad (3;~10),~(-4;~-11),~(-5;~-12),~(4;~10),~(-5;~-11),~(-4;~-12)$
  \item  $y = \frac{3}{7} x +\frac{72}{7};\quad \quad (-11;~6),~(10;~-7),~(-10;~5),~(-11;~5),~(-10;~6),~(10;~-6)$
  \item  $y = \frac{2}{15} x -\frac{2}{5};\quad \quad (-12;~-2),~(11;~1),~(-13;~-3),~(11;~2),~(12;~1),~(12;~2)$
  \item  $y = -\frac{13}{5} x +\frac{32}{5};\quad \quad (-1;~9),~(1;~-9),~(-2;~9),~(0;~-9),~(0;~-10),~(1;~-10)$
  \item  $y = -\frac{3}{5} x +\frac{4}{5};\quad \quad (-9;~3), (-9;~4), (-8;~4), (8;~-5), (7;~-5), (8;~-4)$
  \item  $y = \frac{1}{6} x +\frac{7}{6};\quad \quad (5;~2), (-6;~-2), (-5;~-3), (4;~2), (-5;~-2), (-6;~-3)$
  \item  $y = \frac{4}{3} x -3;\quad \quad (5;~5), (-7;~-5), (6;~5), (-7;~-6), (5;~4), (-6;~-6)$
  \item  $y = -5 x +49;\quad \quad (-9;~-10), (-8;~-10), (7;~8), (8;~8), (-9;~-9), (8;~9)$
  \item  $y = \frac{2}{9} x +\frac{49}{9};\quad \quad (-11;~3), (10;~-3), (-11;~2), (11;~-3), (-12;~3), (10;~-4)$
  \item  $y = \frac{8}{3} x +\frac{56}{3};\quad \quad (4;~-8), (3;~-8), (-4;~7), (-5;~7), (4;~-9), (-4;~8)$
  \item  $x = -9;\quad \quad (-10;~-9), (-10;~-8), (-9;~-8), (8;~7), (8;~8), (9;~8)$
  \item  $y = \frac{3}{2} x +\frac{19}{2};\quad \quad (-2;~8), (0;~-9), (-1;~7), (-2;~7), (-1;~8), (0;~-8)$
  \item  $y = 3 x +10;\quad \quad (4;~4), (5;~4), (-5;~-6), (-6;~-5), (4;~5), (-5;~-5)$
  \item  $y = \frac{3}{17} x -\frac{135}{17};\quad \quad (-12;~5), (-12;~6), (10;~-6), (11;~-6), (-11;~5), (11;~-7)$
  \item  $y = \frac{10}{3} x +\frac{4}{3};\quad \quad (-2;~-2), (-1;~-2), (0;~1), (1;~2), (-1;~-3), (0;~2)$
  \item  $y = \frac{3}{10} x +\frac{19}{5};\quad \quad (-7;~2), (-6;~2), (-7;~1), (5;~-2), (-6;~1), (6;~-2)$
  \item  $y = \frac{4}{7} x -\frac{20}{7};\quad \quad (-2;~-5), (2;~4), (1;~3), (-2;~-4), (2;~3), (-3;~-4)$
  \item  $y = \frac{1}{12} x -\frac{109}{12};\quad \quad (-2;~8), (0;~-10), (1;~-10), (-1;~9), (-1;~8), (1;~-9)$
  \item  $y = -\frac{17}{11} x +\frac{76}{11};\quad \quad (-3;~9), (1;~-11), (-2;~9), (2;~-10), (-3;~10), (-2;~10)$
 \end{enumeratea}
\end{esercizio}

\subsubsection*{\numnameref{sec:retta_equazioni}}

\begin{esercizio}\label{ese:02_01.} %TODO
Riconosci quali delle seguenti è l'equazione di una retta:
 \begin{enumeratea}
  \item  $ y = -3 x +4$
  \item  $ y^2 = x + 3$
  \item  $ y^3 = -x + y^3 -2$
  \item  $ (x +2)(x-2)=(x +3)^2$
  \item  $ (x +y)(x-y) + (y-5)^2 = (x +4)^2$
  \item  $ 0,1 x + 0,2 y = 0,3$
  \item  $ x^2 - y^2 = 0$
  \item  $ y +4 x = 5$
  \item  $ 5 x -4 y+3 = 0$
  \item  $ (x+3)^2 - (y+2)^2 = 2 x - 2y$
  \item  $ y = 0$
  \item  $ y = 2$
  \item  $ x = y$
  \item  $ 0 x + 0 y = 7$
  \item  $ x^2 -(x+2)^2 = 7 (x -y)$
 \end{enumeratea}
\end{esercizio}

\begin{esercizio}\label{ese:}
 Trasforma le equazioni implicite in equazioni esplicite.
 \begin{multicols}{3}
 \begin{enumeratea}
  \item  $-1 x - 11 y - 110 = 0$
  \item  $-8 x - 2 y - 20 = 0$
  \item  $-9 x  = 0$
  \item  $8 x + y - 7 = 0$
  \item  $-2 x - 6 y - 54 = 0$
  \item  $-6 x - 9 y - 27 = 0$
  \item  $-7 x - 8 y - 64 = 0$
  \item  $4 x - 5 y - 5 = 0$
  \item  $7 x - y - 9 = 0$
  \item  $4 x + 7 y + 14 = 0$
  \item  $-7 x + 6 y + 0 = 0$
  \item  $-5 x + 10 y - 50 = 0$
  \item  $x = 0$
  \item  $6 x + 4 y - 4 = 0$
  \item  $6 x - 11 y + 99 = 0$
  \item  $11 x - 8 y - 40 = 0$
  \item  $x = -7$
  \item  $-4 x - 9 y + 36 = 0$
  \item  $-9 x - 6 y - 6 = 0$
  \item  $-7 x + 4 y - 40 = 0$
  \item  $8 x + 4 y - 12 = 0$
 \end{enumeratea}
 \end{multicols}
\end{esercizio}


\begin{esercizio}\label{ese:}
 Trasforma le equazioni esplicite in equazioni implicite.
 \begin{multicols}{3}
 \begin{enumeratea}
  \item  $y = \frac{2}{5} x +2$
  \item  $y = \frac{10}{11} x -6$
  \item  $y = -\frac{5}{3} x +10$
  \item  $y = \frac{1}{3} x -3$
  \item  $y = - x -5$
  \item  $x = 0$
  \item  $y = \frac{8}{7} x +8$
  \item  $y = \frac{1}{11} x +9$
  \item  $y = -\frac{3}{10} x -7$
  \item  $y = -\frac{1}{3} x -7$
  \item  $y = -11 x +8$
  \item  $y = -\frac{5}{11} x +10$
  \item  $y = -8$
  \item  $y = -\frac{3}{5} x -6$
  \item  $y = 1$
  \item  $y = -\frac{9}{5} x -9$
  \item  $y = 8 x $
  \item  $y = \frac{2}{5} x +7$
  \item  $y = -\frac{11}{10} x +2$
  \item  $y = -\frac{1}{2} x +8$
  \item  $y = -\frac{3}{5} x +1$
 \end{enumeratea}
 \end{multicols}
\end{esercizio}

\subsubsection*{\numnameref{sec:retta_disegno}}

Disegna i seguenti gruppi di rette in diversi piani cartesiani 
calcolandone prima, in una tabella, tre punti.

\begin{esercizio}\label{ese:}
 \begin{multicols}{3}
 \begin{enumeratea}
  \item  $y = \frac{6}{5} x +7$
  \item  $y = \frac{11}{10} x -6$
  \item  $y = -\frac{6}{11} x +11$
 \end{enumeratea}
 \end{multicols}
\end{esercizio}

\begin{esercizio}\label{ese:}
 \begin{multicols}{3}
 \begin{enumeratea}
  \item  $y = 3 x +12$
  \item  $y = -3$
  \item  $y = \frac{9}{5} x -2$
 \end{enumeratea}
 \end{multicols}
\end{esercizio}

\begin{esercizio}\label{ese:}
 \begin{multicols}{3}
 \begin{enumeratea}
  \item  $y = -\frac{5}{9} x +6$
  \item  $y = 2 x -9$
  \item  $y = \frac{9}{7} x +12$
 \end{enumeratea}
 \end{multicols}
\end{esercizio}

\begin{esercizio}\label{ese:}
 \begin{multicols}{3}
 \begin{enumeratea}
  \item  $x = 0$
  \item  $y = -\frac{10}{3} x -7$
  \item  $y = -\frac{12}{11} x -11$
 \end{enumeratea}
 \end{multicols}
\end{esercizio}

\begin{esercizio}\label{ese:}
 \begin{multicols}{3}
 \begin{enumeratea}
  \item  $y = \frac{8}{3} x -6$
  \item  $y = 3 x +10$
  \item  $y = \frac{3}{5} x +3$
 \end{enumeratea}
 \end{multicols}
\end{esercizio}

\begin{esercizio}\label{ese:}
 \begin{multicols}{3}
 \begin{enumeratea}
  \item  $y = 7 x +5$
  \item  $y = 5 x +4$
  \item  $y = 0$
 \end{enumeratea}
 \end{multicols}
\end{esercizio}

\begin{esercizio}\label{ese:}
 \begin{multicols}{3}
 \begin{enumeratea}
  \item  $y = \frac{10}{7} x -10$
  \item  $y = -\frac{2}{5} x +4$
  \item  $x = 5$
 \end{enumeratea}
 \end{multicols}
\end{esercizio}

\newpage
Disegna i seguenti gruppi di rette in diversi piani cartesiani 
calcolandone prima, in una tabella, tre punti.

\begin{esercizio}\label{ese:}
 \begin{multicols}{3}
 \begin{enumeratea}
  \item  $2 x - 10 y - 30 = 0$
  \item  $4 x + 10 y - 40 = 0$
  \item  $-3 x + y + 0 = 0$
 \end{enumeratea}
 \end{multicols}
\end{esercizio}

\begin{esercizio}\label{ese:}
 \begin{multicols}{3}
 \begin{enumeratea}
  \item  $11 x - 3 y - 12 = 0$
  \item  $7 x = 0$
  \item  $-10 x - 2 y - 16 = 0$
 \end{enumeratea}
 \end{multicols}
\end{esercizio}

\begin{esercizio}\label{ese:}
 \begin{multicols}{3}
 \begin{enumeratea}
  \item  $-7 x - 4 y - 4 = 0$
  \item  $9 x + 7 y + 42 = 0$
  \item  $-8 x + y + 9 = 0$
 \end{enumeratea}
 \end{multicols}
\end{esercizio}

\begin{esercizio}\label{ese:}
 \begin{multicols}{3}
 \begin{enumeratea}
  \item  $10 x - y + 9 = 0$
  \item  $6 x - 8 y - 48 = 0$
  \item  $-7 x - y - 11 = 0$
 \end{enumeratea}
 \end{multicols}
\end{esercizio}

\begin{esercizio}\label{ese:}
 \begin{multicols}{3}
 \begin{enumeratea}
  \item  $4 x + 4 y + 36 = 0$
  \item  $-5 x - 8 y - 48 = 0$
  \item  $-7 x = 0$
 \end{enumeratea}
 \end{multicols}
\end{esercizio}

\begin{esercizio}\label{ese:}
 \begin{multicols}{3}
 \begin{enumeratea}
  \item  $-7 x + 7 y + 63 = 0$
  \item  $7 x + 6 y + 30 = 0$
  \item  $-11 x - y + 3 = 0$
 \end{enumeratea}
 \end{multicols}
\end{esercizio}

\begin{esercizio}\label{ese:}
 \begin{multicols}{3}
 \begin{enumeratea}
  \item  $-5 x + 5 y - 45 = 0$
  \item  $8 x - y + 11 = 0$
  \item  $5 x + 6 y - 24 = 0$
 \end{enumeratea}
 \end{multicols}
\end{esercizio}

Disegna i seguenti gruppi di rette in diversi 
piani cartesiani usando il metodo rapido.

\begin{esercizio}\label{ese:}
 \begin{multicols}{3}
 \begin{enumeratea}
  \item  $y = -\frac{1}{3} x -4$
  \item  $y = x -8$
  \item  $y = -\frac{2}{5} x -2$
 \end{enumeratea}
 \end{multicols}
\end{esercizio}

\begin{esercizio}\label{ese:}
 \begin{multicols}{3}
 \begin{enumeratea}
  \item  $y = -\frac{1}{2} x +11$
  \item  $y = -\frac{9}{11} x -6$
  \item  $y = 7 x -7$
 \end{enumeratea}
 \end{multicols}
\end{esercizio}

\begin{esercizio}\label{ese:}
 \begin{multicols}{3}
 \begin{enumeratea}
  \item  $y = \frac{5}{6} x +3$
  \item  $y = 2 x -6$
  \item  $y = - x +1$
 \end{enumeratea}
 \end{multicols}
\end{esercizio}

\begin{esercizio}\label{ese:}
 \begin{multicols}{3}
 \begin{enumeratea}
  \item  $y = \frac{9}{2} x -4$
  \item  $y = \frac{11}{4} x -3$
  \item  $y = -\frac{2}{5} x -5$
 \end{enumeratea}
 \end{multicols}
\end{esercizio}

\begin{esercizio}\label{ese:}
 \begin{multicols}{3}
 \begin{enumeratea}
  \item  $y = -\frac{9}{10} x -11$
  \item  $y = -3 x +12$
  \item  $y = -\frac{2}{3} x +12$
 \end{enumeratea}
 \end{multicols}
\end{esercizio}

\begin{esercizio}\label{ese:}
 \begin{multicols}{3}
 \begin{enumeratea}
  \item  $y = 1$
  \item  $y = \frac{1}{8} x +3$
  \item  $y = \frac{10}{3} x -4$
 \end{enumeratea}
 \end{multicols}
\end{esercizio}

\begin{esercizio}\label{ese:}
 \begin{multicols}{3}
 \begin{enumeratea}
  \item  $y = -2$
  \item  $y = x +12$
  \item  $y = 2x -7$
 \end{enumeratea}
 \end{multicols}
\end{esercizio}

\newpage
\subsubsection*{\numnameref{sec:retta_coefficienti}}

Disegna i seguenti gruppi di rette in diversi 
piani cartesiani usando il metodo rapido.

\begin{esercizio}\label{ese:}
 \begin{multicols}{3}
 \begin{enumeratea}
  \item  $8 x - 2 y - 18 = 0$
  \item  $-5 x + 6 y - 18 = 0$
  \item  $9 x - 45 = 0$
 \end{enumeratea}
 \end{multicols}
\end{esercizio}

\begin{esercizio}\label{ese:}
 \begin{multicols}{3}
 \begin{enumeratea}
  \item  $-7 x + 8 y + 80 = 0$
  \item  $2 y + 18 = 0$
  \item  $-4 x + 6 y + 12 = 0$
 \end{enumeratea}
 \end{multicols}
\end{esercizio}

\begin{esercizio}\label{ese:}
 \begin{multicols}{3}
 \begin{enumeratea}
  \item  $-7 x - 6 y + 12 = 0$
  \item  $-6 x - 4 y + 20 = 0$
  \item  $-4 x + y + 6 = 0$
 \end{enumeratea}
 \end{multicols}
\end{esercizio}

\begin{esercizio}\label{ese:}
 \begin{multicols}{3}
 \begin{enumeratea}
  \item  $10 x - 11 y = 0$
  \item  $- 5 y + 15 = 0$
  \item  $3 x + 11 y + 0 = 0$
 \end{enumeratea}
 \end{multicols}
\end{esercizio}

\begin{esercizio}\label{ese:}
 \begin{multicols}{3}
 \begin{enumeratea}
  \item  $-5 x + 5 y + 50 = 0$
  \item  $-2 x + 7 = 0$
  \item  $6 x - 5 y + 30 = 0$
 \end{enumeratea}
 \end{multicols}
\end{esercizio}

\begin{esercizio}\label{ese:}
 \begin{multicols}{3}
 \begin{enumeratea}
  \item  $-8 x - y + 12 = 0$
  \item  $-4 x + 11 y - 11 = 0$
  \item  $-2 x + 7 y - 84 = 0$
 \end{enumeratea}
 \end{multicols}
\end{esercizio}

\begin{esercizio}\label{ese:}
 \begin{multicols}{3}
 \begin{enumeratea}
  \item  $-12 x - 7 y = 0$
  \item  $8 x - 10 y - 50 = 0$
  \item  $5 x - 10 y - 30 = 0$
 \end{enumeratea}
 \end{multicols}
\end{esercizio}

\subsubsection*{\numnameref{sec:retta_rettaperduepunti}}

\begin{esercizio}\label{ese:}
 Calcola l'equazione della retta: AB.
 \begin{enumeratea}
  \item  $A(3;~2),~B(8;~8)$ \hfill 
   [$y = \frac{6}{5} x -\frac{8}{5}$]
  \item  $A(-6;~7),~B(-11;~6)$ \hfill 
   [$y = \frac{1}{5} x +\frac{41}{5}$]
  \item  $A(-9;~1),~B(9;~4)$ \hfill 
   [$y = \frac{1}{6} x +\frac{5}{2}$]
  \item  $A(0;~-12),~B(-10;~11)$ \hfill 
   [$y = -\frac{23}{10} x -12$]
  \item  $A(-5;~1),~B(4;~-2)$ \hfill 
   [$y = -\frac{1}{3} x -\frac{2}{3}$]
  \item  $A(-3;~-4),~B(4;~-7)$ \hfill 
   [$y = -\frac{3}{7} x -\frac{37}{7}$]
  \item  $A(6;~-7),~B(-1;~-9)$ \hfill 
   [$y = \frac{2}{7} x -\frac{61}{7}$]
  \item  $A(-1;~3),~B(-7;~-4)$ \hfill 
   [$y = \frac{7}{6} x +\frac{25}{6}$]
  \item  $A(10;~1),~B(-11;~-10)$ \hfill 
   [$y = \frac{11}{21} x -\frac{89}{21}$]
  \item  $A(-8;~-6),~B(-1;~-11)$ \hfill 
   [$y = -\frac{5}{7} x -\frac{82}{7}$]
  \item  $A(-4;~9),~B(3;~6)$ \hfill 
   [$y = -\frac{3}{7} x +\frac{51}{7}$]
  \item  $A(1;~8),~B(-1;~-11)$ \hfill 
   [$y = \frac{19}{2} x -\frac{3}{2}$]
  \item  $A(-6;~1),~B(-12;~6)$ \hfill 
   [$y = -\frac{5}{6} x -4$]
  \item  $A(4;~11),~B(2;~-9)$ \hfill 
   [$y = 10 x -29$]
  \item  $A(-10;~-5),~B(4;~-10)$ \hfill 
   [$y = -\frac{5}{14} x -\frac{60}{7}$]
  \item  $A(10;~-6),~B(-12;~7)$ \hfill 
   [$y = -\frac{13}{22} x -\frac{1}{11}$]
  \item  $A(-6;~-5),~B(-4;~-3)$ \hfill 
   [$y = x +1$]
  \item  $A(-9;~9),~B(9;~10)$ \hfill 
   [$y = \frac{1}{18} x +\frac{19}{2}$]
  \item  $A(4;~-5),~B(-10;~11)$ \hfill 
   [$y = -\frac{8}{7} x -\frac{3}{7}$]
  \item  $A(-4;~8),~B(-6;~2)$ \hfill 
   [$y = 3 x +20$]
 \end{enumeratea}
\end{esercizio}

\subsubsection*{\numnameref{sec:retta_paralleleleperpendicolari}}

\begin{esercizio}\label{ese:}
 Per ciascuna delle seguenti terne di 
punti disegna la retta~AB e le rette parallela e perpendicolare passanti 
per~C.
 \begin{enumeratea}
  \item  $A(10;~7),~B(-9;~-10),~C(3;~-12)$
  \item  $A(-1;~6),~B(-5;~6),~C(-4;~-5)$
  \item  $A(-7;~-2),~B(-9;~-6),~C(5;~-12)$
  \item  $A(-3;~0),~B(-4;~-4),~C(-9;~-9)$
  \item  $A(4;~-3),~B(-10;~9),~C(8;~6)$
  \item  $A(4;~11),~B(-12;~-11),~C(9;~5)$
  \item  $A(6;~-2),~B(-12;~-7),~C(10;~-8)$
  \item  $A(-4;~4),~B(10;~-10),~C(11;~-1)$
  \item  $A(-3;~-10),~B(9;~8),~C(8;~-9)$
  \item  $A(7;~-12),~B(6;~-4),~C(-11;~-3)$
  \item  $A(0;~0),~B(-8;~-3),~C(4;~11)$
  \item  $A(-2;~-2),~B(7;~-7),~C(4;~8)$
  \item  $A(-7;~-9),~B(-4;~8),~C(4;~10)$
  \item  $A(-8;~-5),~B(11;~11),~C(9;~5)$
  \item  $A(11;~-7),~B(-12;~5),~C(-4;~-7)$
  \item  $A(11;~3),~B(-1;~-4),~C(-10;~-1)$
  \item  $A(5;~0),~B(6;~11),~C(3;~-1)$
  \item  $A(-7;~8),~B(-7;~4),~C(8;~-8)$
  \item  $A(7;~5),~B(-4;~2),~C(-6;~-5)$
  \item  $A(7;~-5),~B(2;~-12),~C(-7;~0)$
 \end{enumeratea}
\end{esercizio}

\subsubsection*{\numnameref{sec:retta_fasci}}


\begin{esercizio}\label{ese:}
 Per ciascuna delle seguenti terne di 
punti disegna la retta~AB e le rette parallela e perpendicolare passanti 
per~C. poi calcolane le equazioni.
 \begin{enumeratea}
  \item  $A(3;~-3),~B(-10;~3),~C(-4;~9)$ \hfill 
   [$y = -\frac{6}{13} x -\frac{21}{13},~y = -\frac{6}{13} x +\frac{93}{13},~y = \frac{13}{6} x +\frac{53}{3}$]
  \item  $A(4;~-12),~B(9;~-6),~C(6;~-9)$ \hfill 
   [$y = \frac{6}{5} x -\frac{84}{5},~y = \frac{6}{5} x -\frac{81}{5},~y = -\frac{5}{6} x -4$]
  \item  $A(4;~-9),~B(-11;~-5),~C(4;~-10)$ \hfill 
   [$y = -\frac{4}{15} x -\frac{119}{15},~y = -\frac{4}{15} x -\frac{134}{15},~y = \frac{15}{4} x -25$]
  \item  $A(-3;~3),~B(-10;~-7),~C(0;~-3)$ \hfill 
   [$y = \frac{10}{7} x +\frac{51}{7},~y = \frac{10}{7} x -3,~y = -\frac{7}{10} x -3$]
  \item  $A(6;~-3),~B(9;~-12),~C(10;~8)$ \hfill 
   [$y = -3 x +15,~y = -3 x +38,~y = \frac{1}{3} x +\frac{14}{3}$]
  \item  $A(-4;~-8),~B(4;~2),~C(-12;~-11)$ \hfill 
   [$y = \frac{5}{4} x -3,~y = \frac{5}{4} x +4,~y = -\frac{4}{5} x -\frac{103}{5}$]
  \item  $A(10;~-6),~B(9;~7),~C(0;~-5)$ \hfill 
   [$y = -13 x +124,~y = -13 x -5,~y = \frac{1}{13} x -5$]
  \item  $A(-2;~0),~B(1;~4),~C(2;~0)$ \hfill 
   [$y = \frac{4}{3} x +\frac{8}{3},~y = \frac{4}{3} x -\frac{8}{3},~y = -\frac{3}{4} x +\frac{3}{2}$]
  \item  $A(-10;~7),~B(-6;~-3),~C(11;~5)$ \hfill 
   [$y = -\frac{5}{2} x -18,~y = -\frac{5}{2} x +\frac{65}{2},~y = \frac{2}{5} x +\frac{3}{5}$]
  \item  $A(-6;~8),~B(2;~-5),~C(-11;~-3)$ \hfill 
   [$y = -\frac{13}{8} x -\frac{7}{4},~y = -\frac{13}{8} x -\frac{167}{8},~y = \frac{8}{13} x +\frac{49}{13}$]
  \item  $A(-4;~-4),~B(1;~2),~C(-7;~4)$ \hfill 
   [$y = \frac{6}{5} x +\frac{4}{5},~y = \frac{6}{5} x +\frac{62}{5},~y = -\frac{5}{6} x -\frac{11}{6}$]
  \item  $A(-1;~-6),~B(8;~-9),~C(10;~9)$ \hfill 
   [$y = -\frac{1}{3} x -\frac{19}{3},~y = -\frac{1}{3} x +\frac{37}{3},~y = 3 x -21$]
  \item  $A(10;~-10),~B(-12;~10),~C(-12;~5)$ \hfill 
   [$y = -\frac{10}{11} x -\frac{10}{11},~y = -\frac{10}{11} x -\frac{65}{11},~y = \frac{11}{10} x +\frac{91}{5}$]
  \item  $A(-1;~-9),~B(2;~-11),~C(-9;~11)$ \hfill 
   [$y = -\frac{2}{3} x -\frac{29}{3},~y = -\frac{2}{3} x +5,~y = \frac{3}{2} x +\frac{49}{2}$]
  \item  $A(11;~2),~B(-12;~11),~C(7;~9)$ \hfill 
   [$y = -\frac{9}{23} x +\frac{145}{23},~y = -\frac{9}{23} x +\frac{270}{23},~y = \frac{23}{9} x -\frac{80}{9}$]
  \item  $A(-8;~-4),~B(5;~-10),~C(-2;~-7)$ \hfill 
   [$y = -\frac{6}{13} x -\frac{100}{13},~y = -\frac{6}{13} x -\frac{103}{13},~y = \frac{13}{6} x -\frac{8}{3}$]
  \item  $A(10;~7),~B(1;~5),~C(-10;~1)$ \hfill 
   [$y = \frac{2}{9} x +\frac{43}{9},~y = \frac{2}{9} x +\frac{29}{9},~y = -\frac{9}{2} x -44$]
  \item  $A(0;~11),~B(-1;~9),~C(-4;~-1)$ \hfill 
   [$y = 2 x +11,~y = 2 x +7,~y = -\frac{1}{2} x -3$]
  \item  $A(8;~-1),~B(2;~-10),~C(-6;~-5)$ \hfill 
   [$y = \frac{3}{2} x -13,~y = \frac{3}{2} x +4,~y = -\frac{2}{3} x -9$]
  \item  $A(11;~-8),~B(-11;~-10),~C(4;~-11)$ \hfill 
   [$y = \frac{1}{11} x -9,~y = \frac{1}{11} x -\frac{125}{11},~y = -11 x +33$]
 \end{enumeratea}
\end{esercizio}

\subsubsection*{\numnameref{sec:retta_distanzapuntoretta}}

\begin{esercizio}\label{ese:}
 Calcola la distanza tra il punto~$P$ e la retta~$r$
 \begin{enumeratea}
  \item  $P(11;~-7),~r:~-6 x + 7 y + 21 = 0$ \hfill 
   [$\frac{94}{\sqrt{85}}\approx  10.2$]
  \item  $P(-10;~10),~r:~3 x + 10 y + 10 = 0$ \hfill 
   [$\frac{80}{\sqrt{109}}\approx 7.663$]
  \item  $P(8;~-1),~r:~-12 x - 10 y + 40 = 0$ \hfill 
   [$\frac{46}{\sqrt{244}}\approx 2.945$]
  \item  $P(-5;~-11),~r:~-6 x + 0 = 0$ \hfill 
   [$\frac{30}{\sqrt{36}}\approx   5.0$]
  \item  $P(-1;~-4),~r:~-3 x + 9 y - 81 = 0$ \hfill 
   [$\frac{114}{\sqrt{90}}\approx 12.02$]
  \item  $P(-3;~0),~r:~9 x - 6 y + 72 = 0$ \hfill 
   [$\frac{45}{\sqrt{117}}\approx  4.16$]
  \item  $P(-10;~-7),~r:~10 x - 9 y + 27 = 0$ \hfill 
   [$\frac{10}{\sqrt{181}}\approx0.7433$]
  \item  $P(4;~0),~r:~-9 x + 4 y + 44 = 0$ \hfill 
   [$\frac{8}{\sqrt{97}}\approx0.8123$]
  \item  $P(-5;~8),~r:~10 x - 3 y - 27 = 0$ \hfill 
   [$\frac{101}{\sqrt{109}}\approx 9.674$]
  \item  $P(-11;~0),~r:~9 x + 11 y - 33 = 0$ \hfill 
   [$\frac{132}{\sqrt{202}}\approx 9.287$]
  \item  $P(-9;~-10),~r:~2 x + 4 y + 24 = 0$ \hfill 
   [$\frac{34}{\sqrt{20}}\approx 7.603$]
  \item  $P(5;~7),~r:~3 x + 1 y + 8 = 0$ \hfill 
   [$\frac{30}{\sqrt{10}}\approx 9.487$]
  \item  $P(8;~7),~r:~-10 x + 6 y + 54 = 0$ \hfill 
   [$\frac{16}{\sqrt{136}}\approx 1.372$]
  \item  $P(-2;~-6),~r:~-2 x - 6 y - 6 = 0$ \hfill 
   [$\frac{34}{\sqrt{40}}\approx 5.376$]
  \item  $P(-12;~9),~r:~-1 x + 9 y - 63 = 0$ \hfill 
   [$\frac{30}{\sqrt{82}}\approx 3.313$]
  \item  $P(-6;~4),~r:~-11 x + 10 y - 70 = 0$ \hfill 
   [$\frac{36}{\sqrt{221}}\approx 2.422$]
  \item  $P(-6;~-3),~r:~-3 y + 33 = 0$ \hfill 
   [$\frac{42}{\sqrt{9}}\approx  14.0$]
  \item  $P(7;~5),~r:~-2 x - 7 y - 35 = 0$ \hfill 
   [$\frac{84}{\sqrt{53}}\approx 11.54$]
  \item  $P(-5;~-6),~r:~-10 x + 7 y + 63 = 0$ \hfill 
   [$\frac{71}{\sqrt{149}}\approx 5.817$]
  \item  $P(-6;~11),~r:~9 x + 5 y + 55 = 0$ \hfill 
   [$\frac{56}{\sqrt{106}}\approx 5.439$]
 \end{enumeratea}
\end{esercizio}


\begin{esercizio}\label{ese:}
 Calcola la distanza tra il punto~$P$ e la retta~$r$
 \begin{enumeratea}
  \item  $P(-2;~-10),~r:~y = -\frac{1}{9} x -11$ \hfill 
   [$\frac{7}{\sqrt{82}}\approx 0.773$]
  \item  $P(7;~-9),~r:~y = \frac{2}{11} x +1$ \hfill 
   [$\frac{124}{\sqrt{125}}\approx 11.09$]
  \item  $P(6;~-2),~r:~y = -\frac{3}{4} x -1$ \hfill 
   [$\frac{28}{\sqrt{100}}\approx   2.8$]
  \item  $P(-1;~-7),~r:~y = -\frac{2}{5} x -6$ \hfill 
   [$\frac{7}{\sqrt{29}}\approx   1.3$]
  \item  $P(-4;~0),~r:~y = - x +7$ \hfill 
   [$\frac{33}{\sqrt{18}}\approx 7.778$]
  \item  $P(11;~9),~r:~y = \frac{10}{11} x +2$ \hfill 
   [$\frac{33}{\sqrt{221}}\approx  2.22$]
  \item  $P(8;~0),~r:~y = -\frac{1}{10} x -6$ \hfill 
   [$\frac{68}{\sqrt{101}}\approx 6.766$]
  \item  $P(-8;~-4),~r:~y = -\frac{9}{10} x -6$ \hfill 
   [$\frac{52}{\sqrt{181}}\approx 3.865$]
  \item  $P(2;~0),~r:~y = -\frac{6}{5} x +2$ \hfill 
   [$\frac{2}{\sqrt{61}}\approx0.2561$]
  \item  $P(9;~7),~r:~y = \frac{1}{2} x +2$ \hfill 
   [$\frac{4}{\sqrt{80}}\approx0.4472$]
  \item  $P(-3;~1),~r:~y = \frac{2}{7} x +2$ \hfill 
   [$\frac{1}{\sqrt{53}}\approx0.1374$]
  \item  $P(1;~6),~r:~y = \frac{6}{5} x +3$ \hfill 
   [$\frac{9}{\sqrt{61}}\approx 1.152$]
  \item  $P(3;~-3),~r:~y = -\frac{11}{12} x +9$ \hfill 
   [$\frac{111}{\sqrt{265}}\approx 6.819$]
  \item  $P(-11;~-7),~r:~y = \frac{3}{4} x -6$ \hfill 
   [$\frac{29}{\sqrt{25}}\approx   5.8$]
  \item  $P(1;~5),~r:~y = -\frac{6}{5} x -9$ \hfill 
   [$\frac{152}{\sqrt{244}}\approx 9.731$]
  \item  $P(5;~3),~r:~y = -\frac{5}{11} x -11$ \hfill 
   [$\frac{179}{\sqrt{146}}\approx 14.81$]
  \item  $P(-1;~10),~r:~y = 2 x -11$ \hfill 
   [$\frac{23}{\sqrt{5}}\approx 10.29$]
  \item  $P(-4;~-11),~r:~y = \frac{3}{4} x +6$ \hfill 
   [$\frac{56}{\sqrt{25}}\approx  11.2$]
  \item  $P(-8;~10),~r:~y = -\frac{2}{9} x +4$ \hfill 
   [$\frac{38}{\sqrt{85}}\approx 4.122$]
  \item  $P(-10;~-7),~r:~y = \frac{7}{4} x $ \hfill 
   [$\frac{42}{\sqrt{65}}\approx 5.209$]
 \end{enumeratea}
\end{esercizio}

\begin{esercizio}\label{ese:}
 Per ciascuna delle seguenti terne di 
punti disegna la retta~AB e calcola la sua equazione. 
Calcola la lunghezza del segmento~AB, la distanza del punto~C dalla retta~AB 
e l'area del triangolo~$ABC$
 \begin{enumeratea}
  \item  $A(-4;~10),~B(-3;~0),~C(3;~-9)$ \hfill 
   [$-10 x -30,~\sqrt{101},~\frac{51}{\sqrt{101}},~25.5$]
  \item  $A(8;~11),~B(6;~-7),~C(7;~-7)$ \hfill 
   [$9 x -61,~\sqrt{328},~\frac{9}{\sqrt{82}},~9$]
  \item  $A(11;~2),~B(2;~7),~C(11;~-1)$ \hfill 
   [$-\frac{5}{9} x +\frac{73}{9},~\sqrt{106},~\frac{27}{\sqrt{106}},~13.5$]
  \item  $A(-5;~9),~B(-8;~4),~C(9;~-5)$ \hfill 
   [$\frac{5}{3} x +\frac{52}{3},~\sqrt{34},~\frac{112}{\sqrt{34}},~56$]
  \item  $A(6;~-8),~B(-10;~-6),~C(4;~-10)$ \hfill 
   [$-\frac{1}{8} x -\frac{29}{4},~\sqrt{260},~\frac{18}{\sqrt{65}},~18$]
  \item  $A(3;~-6),~B(-5;~-2),~C(10;~-11)$ \hfill 
   [$-\frac{1}{2} x -\frac{9}{2},~\sqrt{80},~\frac{3}{\sqrt{5}},~6$]
  \item  $A(1;~-2),~B(3;~-11),~C(7;~-2)$ \hfill 
   [$-\frac{9}{2} x +\frac{5}{2},~\sqrt{85},~\frac{54}{\sqrt{85}},~27$]
  \item  $A(-6;~9),~B(11;~11),~C(1;~4)$ \hfill 
   [$\frac{2}{17} x +\frac{165}{17},~\sqrt{293},~\frac{99}{\sqrt{293}},~49.5$]
  \item  $A(2;~1),~B(6;~1),~C(-6;~-7)$ \hfill 
   [$1,~\sqrt{16},~\frac{8}{\sqrt{1}},~16.0$]
  \item  $A(1;~-4),~B(-6;~-10),~C(7;~7)$ \hfill 
   [$\frac{6}{7} x -\frac{34}{7},~\sqrt{85},~\frac{41}{\sqrt{85}},~20.5$]
  \item  $A(11;~-8),~B(-8;~9),~C(0;~-8)$ \hfill 
   [$-\frac{17}{19} x +\frac{35}{19},~\sqrt{650},~\frac{187}{\sqrt{650}},~93.5$]
  \item  $A(7;~-1),~B(-3;~-12),~C(-11;~-11)$ \hfill 
   [$\frac{11}{10} x -\frac{87}{10},~\sqrt{221},~\frac{98}{\sqrt{221}},~49$]
  \item  $A(-7;~-10),~B(9;~-6),~C(-8;~-3)$ \hfill 
   [$\frac{1}{4} x -\frac{33}{4},~\sqrt{272},~\frac{29}{\sqrt{17}},~58$]
  \item  $A(-11;~0),~B(4;~-2),~C(11;~5)$ \hfill 
   [$-\frac{2}{15} x -\frac{22}{15},~\sqrt{229},~\frac{119}{\sqrt{229}},~59.5$]
  \item  $A(-12;~-1),~B(11;~7),~C(-3;~-1)$ \hfill 
   [$\frac{8}{23} x +\frac{73}{23},~\sqrt{593},~\frac{72}{\sqrt{593}},~36$]
  \item  $A(-10;~11),~B(9;~-5),~C(-12;~2)$ \hfill 
   [$-\frac{16}{19} x +\frac{49}{19},~\sqrt{617},~\frac{203}{\sqrt{617}},~101.5$]
  \item  $A(6;~-12),~B(-4;~6),~C(10;~-8)$ \hfill 
   [$-\frac{9}{5} x -\frac{6}{5},~\sqrt{424},~\frac{56}{\sqrt{106}},~56$]
  \item  $A(9;~-10),~B(0;~-6),~C(-5;~-2)$ \hfill 
   [$-\frac{4}{9} x -6,~\sqrt{97},~\frac{16}{\sqrt{97}},~8$]
  \item  $A(3;~-11),~B(-6;~4),~C(6;~2)$ \hfill 
   [$-\frac{5}{3} x -6,~\sqrt{306},~\frac{54}{\sqrt{34}},~81$]
  \item  $A(3;~-9),~B(-8;~0),~C(-9;~9)$ \hfill 
   [$-\frac{9}{11} x -\frac{72}{11},~\sqrt{202},~\frac{90}{\sqrt{202}},~45$]
 \end{enumeratea}
\end{esercizio}

%\newpage
\begin{esercizio}\label{ese:}
 Disegna le due rette, individua le 
coordinate dell'intersezione, verifica che queste sono soluzioni di 
entrambe le equazioni.
 \begin{multicols}{2}
 \begin{enumeratea}
  \item  $r:~y = 4 x -8; \quad s:~y = -\frac{1}{4} x +9$
  \item  $r:~y = \frac{5}{8} x -4; \quad s:~y = \frac{1}{4} x -7$
  \item  $r:~y = -\frac{20}{3} x +9; \quad s:~y = -\frac{17}{3} x +6$
  \item  $r:~y = 1; s:~y = 3 x -11$
  \item  $r:~y = -\frac{5}{2} x -4; \quad s:~y = -8 x +7$
  \item  $r:~y = 4 x -3; \quad s:~y = \frac{17}{3} x -8$
  \item  $r:~y = 4 x ; \quad s:~y = 16 x -12$
  \item  $r:~y = \frac{10}{11} x -11; \quad s:~y = -\frac{5}{11} x +4$
  \item  $r:~y = \frac{1}{2} x +7; \quad s:~y = \frac{11}{4} x -11$
  \item  $r:~y = \frac{3}{2} x +7; \quad s:~y = 4$
  \item  $r:~y = 9; \quad s:~y = \frac{10}{11} x -1$
  \item  $r:~y = \frac{5}{4} x -11; \quad s:~y = \frac{11}{8} x -12$
  \item  $r:~y = \frac{9}{8} x ; \quad s:~y = -\frac{1}{4} x +11$
  \item  $r:~y = \frac{9}{4} x +6; \quad s:~y = \frac{5}{2} x +8$
  \item  $r:~y = -\frac{4}{5} x -10; \quad s:~y = \frac{3}{10} x +1$
  \item  $r:~y = 2 x +9; \quad s:~y = \frac{17}{9} x +8$
  \item  $r:~x = 0; \quad s:~x = 0$
  \item  $r:~y = \frac{9}{5} x -7; \quad s:~y = \frac{11}{10} x $
  \item  $r:~y = -\frac{2}{3} x ; \quad s:~y = -\frac{7}{9} x +1$
  \item  $r:~y = \frac{1}{9} x -11; \quad s:~y = \frac{1}{9} x -11$
 \end{enumeratea}
 \end{multicols}
\end{esercizio}

\begin{esercizio}\label{ese:}
 Disegna le due rette e calcola le coordinate dell'intersezione.
 \begin{enumeratea}
  \item  $r:~y = -\frac{3}{11} x -6; s:~y = -\frac{11}{8} x -6$ \hfill 
   [$(0;~-6)$]
  \item  $r:~y = \frac{5}{7} x -1; s:~y = \frac{1}{5} x -7$ \hfill 
   [$(-\frac{35}{3};~-\frac{28}{3})$]
  \item  $r:~y = \frac{5}{12} x +6; s:~y = -\frac{6}{7} x -8$ \hfill 
   [$(-\frac{1176}{107};~\frac{152}{107})$]
  \item  $r:~y = -\frac{11}{2} x +1; s:~y = -\frac{5}{2} x +6$ \hfill 
   [$(-\frac{5}{3};~\frac{61}{6})$]
  \item  $r:~y = 11 x -8; s:~y = -\frac{7}{11} x -1$ \hfill 
   [$(\frac{77}{128};~-\frac{177}{128})$]
  \item  $r:~y = 4; s:~y = \frac{6}{7} x $ \hfill 
   [$(\frac{14}{3};~4)$]
  \item  $r:~y = \frac{5}{2} x -11; s:~y = -\frac{11}{7} x +9$ \hfill 
   [$(\frac{280}{57};~\frac{73}{57})$]
  \item  $r:~y = -\frac{9}{2} x -5; s:~y = \frac{5}{6} x +11$ \hfill 
   [$(-3;~\frac{17}{2})$]
  \item  $r:~y = \frac{7}{9} x +1; s:~y = \frac{8}{3} x +12$ \hfill 
   [$(-\frac{99}{17};~-\frac{60}{17})$]
  \item  $r:~y = -\frac{1}{3} x +2; s:~y = 2 x +12$ \hfill 
   [$(-\frac{30}{7};~\frac{24}{7})$]
  \item  $r:~y = -7 x +1; s:~y = -5 x +11$ \hfill 
   [$(-5;~36)$]
  \item  $r:~y = 6 x -5; s:~y = 8 x -1$ \hfill 
   [$(-2;~-17)$]
  \item  $r:~y = 3 x +11; s:~y = -\frac{7}{2} x +1$ \hfill 
   [$(-\frac{20}{13};~\frac{83}{13})$]
  \item  $r:~y = \frac{2}{9} x -4; s:~y = \frac{1}{3} x -4$ \hfill 
   [$(0;~-4)$]
  \item  $r:~y = -\frac{2}{5} x -9; s:~y = \frac{4}{5} x +7$ \hfill 
   [$(-\frac{40}{3};~-\frac{11}{3})$]
  \item  $r:~y = \frac{12}{7} x +7; s:~y = \frac{1}{9} x $ \hfill 
   [$(-\frac{441}{101};~-\frac{49}{101})$]
  \item  $r:~y = 2 x +8; s:~y = \frac{1}{2} x -9$ \hfill 
   [$(-\frac{34}{3};~-\frac{44}{3})$]
  \item  $r:~y = -9 x ; s:~y = -\frac{11}{10} x +5$ \hfill 
   [$(-\frac{50}{79};~\frac{450}{79})$]
  \item  $r:~y = \frac{5}{8} x -10; s:~y = -\frac{1}{5} x -2$ \hfill 
   [$(\frac{320}{33};~-\frac{130}{33})$]
  \item  $r:~y = -\frac{8}{11} x +4; s:~y = - x +4$ \hfill 
   [$(0;~4)$]
 \end{enumeratea}
\end{esercizio}

\begin{esercizio}\label{ese:}
 Disegna le due rette e calcola le coordinate dell'intersezione.
 \begin{enumeratea}
  \item  $r:~4 x + 7 y - 63 = 0; s:~-9 x - 10 y + 110 = 0$ \hfill 
   [$(\frac{140}{23};~\frac{127}{23})$]
  \item  $r:~-7 x + 0 = 0; s:~-6 x + 6 y + 60 = 0$ \hfill 
   [$(0;~-10)$]
  \item  $r:~8 x + 0 = 0; s:~-10 x - 12 y + 72 = 0$ \hfill 
   [$(0;~6)$]
  \item  $r:~-12 x - 10 y - 60 = 0; s:~6 x - 5 y + 10 = 0$ \hfill 
   [$(-\frac{10}{3};~-2)$]
  \item  $r:~4 x - 8 y + 24 = 0; s:~-1 x + 10 y + 50 = 0$ \hfill 
   [$(-20;~-7)$]
  \item  $r:~-9 x + 2 y - 2 = 0; s:~-1 x + 10 y + 30 = 0$ \hfill 
   [$(-\frac{10}{11};~-\frac{34}{11})$]
  \item  $r:~7 x + 0 = 0; s:~-3 x + 10 y - 20 = 0$ \hfill 
   [$(0;~2)$]
  \item  $r:~-1 x - 12 y + 12 = 0; s:~-2 y + 18 = 0$ \hfill 
   [$(-96;~9)$]
  \item  $r:~-1 x + 3 y + 30 = 0; s:~11 x - 9 y - 72 = 0$ \hfill 
   [$(-\frac{9}{4};~-\frac{43}{4})$]
  \item  $r:~11 x - 1 y + 11 = 0; s:~-7 x - 8 y - 8 = 0$ \hfill 
   [$(-\frac{96}{95};~-\frac{11}{95})$]
  \item  $r:~-6 x - 10 y + 30 = 0; s:~5 x + 9 y - 45 = 0$ \hfill 
   [$(-45;~30)$]
  \item  $r:~7 x - 9 y + 63 = 0; s:~-2 x - 12 y - 120 = 0$ \hfill 
   [$(-18;~-7)$]
  \item  $r:~-10 x + 9 y + 72 = 0; s:~7 x + 1 y + 6 = 0$ \hfill 
   [$(\frac{18}{73};~-\frac{564}{73})$]
  \item  $r:~-5 x + 2 y + 12 = 0; s:~-7 x - 4 y - 24 = 0$ \hfill 
   [$(0;~-6)$]
  \item  $r:~-10 x - 10 y - 40 = 0; s:~-10 x - 4 y - 20 = 0$ \hfill 
   [$(-\frac{2}{3};~-\frac{10}{3})$]
  \item  $r:~-11 y - 99 = 0; s:~-10 x + 1 y + 5 = 0$ \hfill 
   [$(-\frac{2}{5};~-9)$]
  \item  $r:~8 x - 7 y - 77 = 0; s:~5 x - 9 y - 99 = 0$ \hfill 
   [$(0;~-11)$]
  \item  $r:~9 x + 9 y + 54 = 0; s:~10 x + 3 y + 33 = 0$ \hfill 
   [$(-\frac{15}{7};~-\frac{27}{7})$]
  \item  $r:~-1 x + 10 y - 100 = 0; s:~-6 y + 66 = 0$ \hfill 
   [$(10;~11)$]
  \item  $r:~-11 x - 9 y + 0 = 0; s:~-6 x - 5 y + 10 = 0$ \hfill 
   [$(-90;~110)$]
 \end{enumeratea}
\end{esercizio}


\subsection{Esercizi riepilogativi}

% Da Vincenzo Gentile
\begin{esercizio}\label{ese:02_01.}
Determina il circocentro, l'ortocentro, il baricentro, il perimetro e l'area 
del triangolo avente per vertici i punti~$A(-1;~-1), B(2;~-1 ), C(0;~3$).  
\end{esercizio}

\begin{esercizio}\label{ese:02_01.}
Determina la proiezione ortogonale del punto~$P(-1;~-4)$ sulla 
retta~$y = - \frac{1}{5} x - 1$ 
\end{esercizio}

\begin{esercizio}\label{ese:02_01.}
Dati i tre punti~$A(1;~3)$, $B(-1;~6)$, $C(-4;~4)$ 
determina il punto~$D$ in modo tale che il quadrilatero~$ABCD$ risulti essere 
un quadrato. 
(Suggerimento: ci sono due metodi per risolvere l'esercizio, uno è molto 
veloce...) 
\end{esercizio}

\begin{esercizio}\label{ese:02_01.}
Verifica che il triangolo di vertici~$A(3;~2)$, $B(2;~5)$, $C(-4;~3)$ è 
rettangolo e calcola l'area. \hfill[10]
\end{esercizio}

\begin{esercizio}\label{ese:02_01.}
Nel fascio di rette di centro~$A(-2;~1)$ determinare la retta~$r$ 
perpendicolare alla retta di equazione~$2x - 2y - 3 = 0$ \hfill[x + y + 1 = 0]
\end{esercizio}

\begin{esercizio}\label{ese:02_01.}
Nel fascio di rette parallele a~$y = -2x$ determinare la retta r 
passante per~$A(0;~-3)$ \hfill[2x + y + 3 = 0]
\end{esercizio}

\begin{esercizio}\label{ese:02_01.}
Dati i tre vertici di un triangolo~$A(5;~0)$ $B(1;~2)$ e $C(-3;~2)$, 
scriverne le equazioni dei lati.
 \hfill[$x + 2y - 5 = 0$  $x + 4 y - 5 = 0$ $y = 2$]
\end{esercizio}

\begin{esercizio}\label{ese:02_01.}
Scrivere l'equazione di una retta passante per~$A(4;~2)$ 
e per il punto comune alle rette~$r:~x + y = 3$ e~$s:~x - y + 1 = 0$
\hfill[$y = 2$]
\end{esercizio}

\begin{esercizio}\label{ese:02_01.}
Scrivere l'equazione della retta congiungente il punto d'intersezione delle 
rette~$a:~x + y = 3$ e~$b:~x - y + 1 = 0$, con quello d'intersezione delle 
rette~$c:~x - y = 1$ e~$d~x = -1$
 \hfill[$y = 2x$]
\end{esercizio}

\begin{esercizio}\label{ese:02_01.}
Scrivere l'equazione della retta passante per~$A(-5;~-1)$ parallela alla retta 
congiungente l'origine delle coordinate con~$B(1;~2)$
 \hfill[$2x - y + 9 = 0$]
\end{esercizio}

\begin{esercizio}\label{ese:02_01.}
La retta passante per~$A(2;~3)$ e~$B(-1;~-6)$ e quella per~$C(6;~-1)$
e~$D(-3;~2)$ come sono fra loro?
 \hfill[perpendicolari]
\end{esercizio}

% \begin{esercizio}\label{ese:02_01.}
% problema 30 pag 18 \hfill[]
% \end{esercizio}
% 
% \begin{esercizio}\label{ese:02_01.}
%  \hfill[]
% \end{esercizio}


