% (c) 2012-2013 Claudio Carboncini - claudio.carboncini@gmail.com
% (c) 2012-2014 Dimitrios Vrettos - d.vrettos@gmail.com
% (c) 2014 Daniele Zambelli - daniele.zambelli@gmail.com

\section{Esercizi}

\subsection{Esercizi dei singoli paragrafi}

\begin{comment}
\subsubsection*{\numnameref{sec:03_problemiretta}}

\begin{esercizio}\label{ese:03.1}
Dopo aver riportato in un riferimento cartesiano i seguenti punti,
calcola il il punto medio e la lunghezza dei seguenti~AB:
 \begin{enumeratea}
  \item  $ A=(9;); ~B=(-12;)$ \hfill [$M_{AB}=-1.5; ~\overline{AB}=-21$]
  \item  $ A=(-6;); ~B=(-11;)$ \hfill [$M_{AB}=-8.5; ~\overline{AB}=-5$]
  \item  $ A=(-8;); ~B=(-10;)$ \hfill [$M_{AB}=-9.0; ~\overline{AB}=-2$]
  \item  $ A=(-9;); ~B=(-12;)$ \hfill [$M_{AB}=-10.5; ~\overline{AB}=-3$]
  \item  $ A=(1;); ~B=(-11;)$ \hfill [$M_{AB}=-5.0; ~\overline{AB}=-12$]
  \item  $ A=(-2;); ~B=(7;)$ \hfill [$M_{AB}=2.5; ~\overline{AB}=9$]
  \item  $ A=(-3;); ~B=(3;)$ \hfill [$M_{AB}=0.0; ~\overline{AB}=6$]
  \item  $ A=(-4;); ~B=(9;)$ \hfill [$M_{AB}=2.5; ~\overline{AB}=13$]
  \item  $ A=(-5;); ~B=(-12;)$ \hfill [$M_{AB}=-8.5; ~\overline{AB}=-7$]
  \item  $ A=(3;); ~B=(6;)$ \hfill [$M_{AB}=4.5; ~\overline{AB}=3$]
  \item  $ A=(11;); ~B=(8;)$ \hfill [$M_{AB}=9.5; ~\overline{AB}=-3$]
  \item  $ A=(-3;); ~B=(1;)$ \hfill [$M_{AB}=-1.0; ~\overline{AB}=4$]
  \item  $ A=(-8;); ~B=(11;)$ \hfill [$M_{AB}=1.5; ~\overline{AB}=19$]
  \item  $ A=(8;); ~B=(-2;)$ \hfill [$M_{AB}=3.0; ~\overline{AB}=-10$]
  \item  $ A=(-7;); ~B=(4;)$ \hfill [$M_{AB}=-1.5; ~\overline{AB}=11$]
 \end{enumeratea}
\end{esercizio}

\end{comment}

\subsubsection*{\numnameref{sec:05_problemipiano}}

\begin{esercizio}\label{ese:03.1}
Dopo aver riportato in un riferimento cartesiano i seguenti punti,
per ogni segmento~AB calcola: punto medio, lunghezza e area sottesa.
 \begin{enumeratea}
  \item $ A=(-5; 1); ~B=(-2; -4)$ \hfill 
  [$M_{AB}=(-3.5, -1.5); ~\overline{AB}=\sqrt{34}=5.83; ~A_AB=-4.5$]
  \item $ A=(-3; 0); ~B=(-1; -4)$ \hfill 
  [$M_{AB}=(-2.0, -2.0); ~\overline{AB}=\sqrt{20}=4.47; ~A_AB=-4.0$]
  \item $ A=(-3; 0); ~B=(0; -5)$ \hfill 
  [$M_{AB}=(-1.5, -2.5); ~\overline{AB}=\sqrt{34}=5.83; ~A_AB=-7.5$]
  \item $ A=(-7; -2); ~B=(0; -6)$ \hfill 
  [$M_{AB}=(-3.5, -4.0); ~\overline{AB}=\sqrt{65}=8.06; ~A_AB=-28.0$]
  \item $ A=(-4; -1); ~B=(1; -4)$ \hfill 
  [$M_{AB}=(-1.5, -2.5); ~\overline{AB}=\sqrt{34}=5.83; ~A_AB=-12.5$]
  \item $ A=(-7; -3); ~B=(-6; -7)$ \hfill 
  [$M_{AB}=(-6.5, -5.0); ~\overline{AB}=\sqrt{17}=4.12; ~A_AB=-5.0$]
  \item $ A=(-4; -3); ~B=(-1; -6)$ \hfill 
  [$M_{AB}=(-2.5, -4.5); ~\overline{AB}=\sqrt{18}=4.24; ~A_AB=-13.5$]
  \item $ A=(-5; 0); ~B=(-3; -3)$ \hfill 
  [$M_{AB}=(-4.0, -1.5); ~\overline{AB}=\sqrt{13}=3.61; ~A_AB=-3.0$]
  \item $ A=(-7; -2); ~B=(-2; -5)$ \hfill 
  [$M_{AB}=(-4.5, -3.5); ~\overline{AB}=\sqrt{34}=5.83; ~A_AB=-17.5$]
  \item $ A=(-2; -3); ~B=(2; -6)$ \hfill 
  [$M_{AB}=(0.0, -4.5); ~\overline{AB}=\sqrt{25}=5.0; ~A_AB=-18.0$]
  \item $ A=(-4; 0); ~B=(-3; -6)$ \hfill 
  [$M_{AB}=(-3.5, -3.0); ~\overline{AB}=\sqrt{37}=6.08; ~A_AB=-3.0$]
  \item $ A=(-7; 2); ~B=(-3; -1)$ \hfill 
  [$M_{AB}=(-5.0, 0.5); ~\overline{AB}=\sqrt{25}=5.0; ~A_AB=2.0$]
  \item $ A=(-2; -2); ~B=(0; -6)$ \hfill 
  [$M_{AB}=(-1.0, -4.0); ~\overline{AB}=\sqrt{20}=4.47; ~A_AB=-8.0$]
  \item $ A=(-5; 0); ~B=(-1; -2)$ \hfill 
  [$M_{AB}=(-3.0, -1.0); ~\overline{AB}=\sqrt{20}=4.47; ~A_AB=-4.0$]
  \item $ A=(-3; -2); ~B=(-1; -8)$ \hfill 
  [$M_{AB}=(-2.0, -5.0); ~\overline{AB}=\sqrt{40}=6.32; ~A_AB=-10.0$]
 \end{enumeratea}
\end{esercizio}


\begin{esercizio}\label{ese:03.1}
Disegna i triangoli che hanno per vertici i seguenti punti
poi calcolane perimetro e area.
 \begin{enumeratea}
  \item $ A \punto{-8}{0};~B \punto{0}{-4};~C \punto{2}{3}$
  \hfill [$2p=26.66 ~A=32.0$]
  \item $ A \punto{-7}{0};~B \punto{-1}{-2};~C \punto{5}{7}$ 
  \hfill [$2p=31.03 ~A=33.0$]
  \item $ A \punto{-3}{-3};~B \punto{-2}{-6};~C \punto{3}{5}$ 
  \hfill [$2p=25.25 ~A=13.0$]
  \item $ A \punto{-4}{0};~B \punto{-2}{-8};~C \punto{6}{1}$ 
  \hfill [$2p=30.34 ~A=41.0$]
  \item $ A \punto{-7}{-3};~B \punto{1}{-6};~C \punto{5}{3}$ 
  \hfill [$2p=31.81 ~A=42.0$]
  \item $ A \punto{-6}{2};~B \punto{0}{-8};~C \punto{2}{3}$ 
  \hfill [$2p=30.90 ~A=43.0$]
  \item $ A \punto{-5}{-1};~B \punto{-3}{-3};~C \punto{5}{7}$ 
  \hfill [$2p=28.44 ~A=18.0$]
  \item $ A \punto{-6}{0};~B \punto{-5}{-3};~C \punto{-2}{7}$ 
  \hfill [$2p=21.66 ~A=9.5$]
  \item $ A \punto{-2}{-1};~B \punto{2}{-4};~C \punto{5}{3}$ 
  \hfill [$2p=20.68 ~A=18.5$]
  \item $ A \punto{-3}{0};~B \punto{-2}{-6};~C \punto{5}{4}$ 
  \hfill [$2p=27.23 ~A=26.0$]
  \item $ A \punto{-4}{0};~B \punto{0}{-4};~C \punto{2}{7}$ 
  \hfill [$2p=26.06 ~A=26.0$]
  \item $ A \punto{-6}{2};~B \punto{-2}{-3};~C \punto{4}{4}$ 
  \hfill [$2p=25.82 ~A=29.0$]
  \item $ A \punto{-5}{2};~B \punto{2}{0};~C \punto{3}{7}$ 
  \hfill [$2p=23.79 ~A=25.5$]
  \item $ A \punto{-7}{0};~B \punto{1}{-7};~C \punto{4}{2}$ 
  \hfill [$2p=31.30 ~A=46.5$]
  \item $ A \punto{-7}{2};~B \punto{-1}{-7};~C \punto{5}{6}$ 
  \hfill [$2p=37.78 ~A=66.0$]
 \end{enumeratea}
\end{esercizio}


\subsection{Esercizi riepilogativi}

\begin{comment}
\begin{esercizio}\label{ese:03.1}
Dopo aver riportato in un riferimento cartesiano i seguenti punti,
calcola il il punto medio e la lunghezza dei seguenti~AB:
 \begin{enumeratea}
  \item  $ A=(4;); ~B=(-7;)$ \hfill [$M_{AB}=-1.5; ~\overline{AB}=-11$]
  \item  $ A=(-12;); ~B=(-4;)$ \hfill [$M_{AB}=-8.0; ~\overline{AB}=8$]
  \item  $ A=(-5;); ~B=(5;)$ \hfill [$M_{AB}=0.0; ~\overline{AB}=10$]
  \item  $ A=(-11;); ~B=(2;)$ \hfill [$M_{AB}=-4.5; ~\overline{AB}=13$]
  \item  $ A=(-10;); ~B=(-3;)$ \hfill [$M_{AB}=-6.5; ~\overline{AB}=7$]
  \item  $ A=(9;); ~B=(-6;)$ \hfill [$M_{AB}=1.5; ~\overline{AB}=-15$]
  \item  $ A=(-10;); ~B=(2;)$ \hfill [$M_{AB}=-4.0; ~\overline{AB}=12$]
  \item  $ A=(3;); ~B=(8;)$ \hfill [$M_{AB}=5.5; ~\overline{AB}=5$]
  \item  $ A=(-5;); ~B=(-10;)$ \hfill [$M_{AB}=-7.5; ~\overline{AB}=-5$]
  \item  $ A=(2;); ~B=(0;)$ \hfill [$M_{AB}=1.0; ~\overline{AB}=-2$]
  \item  $ A=(10;); ~B=(-12;)$ \hfill [$M_{AB}=-1.0; ~\overline{AB}=-22$]
  \item  $ A=(-4;); ~B=(-11;)$ \hfill [$M_{AB}=-7.5; ~\overline{AB}=-7$]
  \item  $ A=(8;); ~B=(9;)$ \hfill [$M_{AB}=8.5; ~\overline{AB}=1$]
  \item  $ A=(-4;); ~B=(-9;)$ \hfill [$M_{AB}=-6.5; ~\overline{AB}=-5$]
  \item  $ A=(2;); ~B=(-8;)$ \hfill [$M_{AB}=-3.0; ~\overline{AB}=-10$]
 \end{enumeratea}
\end{esercizio}

\end{comment}

\begin{esercizio}\label{ese:03.2}
Dopo aver riportato in un riferimento cartesiano i seguenti punti,
per ogni segmento~AB calcola: punto medio, lunghezza e area sottesa.
 \begin{enumeratea}
  \item $ A=(-7; 0); ~B=(-6; -5)$ \hfill
  [$M_{AB}=(-6.5, -2.5); ~\overline{AB}=\sqrt{26}=5.10; ~A_AB=-2.5$]
  \item $ A=(-5; 2); ~B=(-4; 0)$ \hfill 
  [$M_{AB}=(-4.5, 1.0); ~\overline{AB}=\sqrt{5}=2.24; ~A_AB=1.0$]
  \item $ A=(-4; 0); ~B=(-3; -6)$ \hfill 
  [$M_{AB}=(-3.5, -3.0); ~\overline{AB}=\sqrt{37}=6.08; ~A_AB=-3.0$]
  \item $ A=(-4; 0); ~B=(0; -6)$ \hfill 
  [$M_{AB}=(-2.0, -3.0); ~\overline{AB}=\sqrt{52}=7.21; ~A_AB=-12.0$]
  \item $ A=(-3; 1); ~B=(2; -5)$ \hfill 
  [$M_{AB}=(-0.5, -2.0); ~\overline{AB}=\sqrt{61}=7.81; ~A_AB=-10.0$]
  \item $ A=(-3; -3); ~B=(0; -5)$ \hfill 
  [$M_{AB}=(-1.5, -4.0); ~\overline{AB}=\sqrt{13}=3.61; ~A_AB=-12.0$]
  \item $ A=(-4; -2); ~B=(-3; -4)$ \hfill 
  [$M_{AB}=(-3.5, -3.0); ~\overline{AB}=\sqrt{5}=2.24; ~A_AB=-3.0$]
  \item $ A=(-8; 0); ~B=(-6; -6)$ \hfill 
  [$M_{AB}=(-7.0, -3.0); ~\overline{AB}=\sqrt{40}=6.32; ~A_AB=-6.0$]
  \item $ A=(-5; -2); ~B=(-2; -6)$ \hfill 
  [$M_{AB}=(-3.5, -4.0); ~\overline{AB}=\sqrt{25}=5.0; ~A_AB=-12.0$]
  \item $ A=(-6; -2); ~B=(-4; -4)$ \hfill 
  [$M_{AB}=(-5.0, -3.0); ~\overline{AB}=\sqrt{8}=2.83; ~A_AB=-6.0$]
  \item $ A=(-5; -1); ~B=(2; -7)$ \hfill 
  [$M_{AB}=(-1.5, -4.0); ~\overline{AB}=\sqrt{85}=9.22; ~A_AB=-28.0$]
  \item $ A=(-5; 0); ~B=(-4; -3)$ \hfill 
  [$M_{AB}=(-4.5, -1.5); ~\overline{AB}=\sqrt{10}=3.16; ~A_AB=-1.5$]
  \item $ A=(-8; -1); ~B=(-3; -3)$ \hfill 
  [$M_{AB}=(-5.5, -2.0); ~\overline{AB}=\sqrt{29}=5.39; ~A_AB=-10.0$]
  \item $ A=(-4; -1); ~B=(1; -3)$ \hfill 
  [$M_{AB}=(-1.5, -2.0); ~\overline{AB}=\sqrt{29}=5.39; ~A_AB=-10.0$]
  \item $ A=(-2; -3); ~B=(2; -6)$ \hfill 
  [$M_{AB}=(0.0, -4.5); ~\overline{AB}=\sqrt{25}=5.0; ~A_AB=-18.0$] 
 \end{enumeratea}
\end{esercizio}

\begin{esercizio}\label{ese:03.1}
Disegna i triangoli che hanno per vertici i seguenti punti
poi calcolane perimetro e area.
 \begin{enumeratea}
  \item $ A=(-8; -3); ~B=(1; -6); ~C=(3; -2)$ \hfill [$2p=25.00 ~A=21.0$]
  \item $ A=(-6; -3); ~B=(-4; -5); ~C=(4; 7)$ \hfill [$2p=31.39 ~A=20.0$]
  \item $ A=(-4; 2); ~B=(2; -5); ~C=(6; 4)$ \hfill [$2p=29.27 ~A=41.0$]
  \item $ A=(-5; -2); ~B=(-1; -5); ~C=(7; 1)$ \hfill [$2p=27.37 ~A=24.0$]
  \item $ A=(-8; -1); ~B=(1; -4); ~C=(4; 1)$ \hfill [$2p=27.48 ~A=27.0$]
  \item $ A=(-6; 0); ~B=(-3; -4); ~C=(-2; 5)$ \hfill [$2p=20.46 ~A=15.5$]
  \item $ A=(-2; 2); ~B=(1; -8); ~C=(4; 5)$ \hfill [$2p=30.49 ~A=34.5$]
  \item $ A=(-3; -2); ~B=(-2; -7); ~C=(-1; 2)$ \hfill [$2p=18.63 ~A=7.0$]
  \item $ A=(-8; -3); ~B=(-2; -6); ~C=(-1; -1)$ \hfill [$2p=19.09 ~A=16.5$]
  \item $ A=(-7; 0); ~B=(-2; -3); ~C=(2; 5)$ \hfill [$2p=25.07 ~A=26.0$]
  \item $ A=(-3; -2); ~B=(1; -5); ~C=(6; 1)$ \hfill [$2p=22.30 ~A=19.5$]
  \item $ A=(-5; -2); ~B=(2; -8); ~C=(6; -1)$ \hfill [$2p=28.33 ~A=36.5$]
  \item $ A=(-4; 2); ~B=(0; 0); ~C=(3; 6)$ \hfill [$2p=19.24 ~A=15.0$]
  \item $ A=(-8; -1); ~B=(-2; -7); ~C=(6; 0)$ \hfill [$2p=33.15 ~A=45.0$]
  \item $ A=(-5; -2); ~B=(-4; -7); ~C=(6; 4)$ \hfill [$2p=32.50 ~A=30.5$]
 \end{enumeratea}
\end{esercizio}

\begin{esercizio}
\label{ese:D.18}
Per ciascuna coppia di punti indica in quale quadrante si trova, se si trova su un asse indica l'asse:
$(0;-1)$, \,$\left(\frac{3}{2};-\frac{5}{4}\right)$, \,$\left(0;\frac{1}{3}\right)$, \,$\left(\frac{5}{3};1\right)$, \,
$\left(1;-\frac{5}{3}\right)$ \,$(-8;9)$, \,$\left(-2;-\frac{1}{4}\right)$, \,$(-1;0)$

Completa l'osservazione conclusiva:
\begin{itemize*}
\item tutte le coppie del tipo~$(+;+)$ individuano punti del~$\ldots \ldots \ldots$
\item tutte le coppie del tipo~$(\ldots;\ldots)$ individuano punti del~$IV$ quadrante;
\item tutte le coppie del tipo~$(-;+)$ individuano punti del~$\ldots \ldots \ldots$
\item tutte le coppie del tipo~$(-;-)$ individuano punti del~$\ldots \ldots \ldots$
\item tutte le coppie del tipo~$(\ldots;0)$ individuano punti del~$\ldots \ldots \ldots$
\item tutte le coppie del tipo~$(\ldots;\ldots)$ individuano punti dell'asse~$y$
\end{itemize*}
\end{esercizio}

\begin{esercizio}
\label{ese:D.19}
Sono assegnati i punti~$A(3;-1)$, $B(3;5)$, $M(-1;-1)$, $N(-1;-7)$ È vero che~$\overline{AB}=\overline{MN}$?
\end{esercizio}

\begin{esercizio}
\label{ese:D.20}
Sono assegnati i punti~$A(1;5)$, $B(-4;5)$, $C(-4;-2)$, $D(5;-2)$ Quale poligono si ottiene congiungendo nell'ordine i quattro
punti assegnati? Determinare l'area del quadrilatero~$ABCD$
\end{esercizio}

\begin{esercizio}
\label{ese:D.21}
Determina l'area del quadrilatero~$MNPQ$ sapendo che~$M(6;-4)$, $N(8;3)$, $P(6;5)$, $Q(4;3)$
\end{esercizio}

\begin{esercizio}
\label{ese:D.22}
Determina~$\overline{AB}$ sapendo che~$A(7;-1)$ e~$B(-3;-6)$
\end{esercizio}

\begin{esercizio}
\label{ese:D.23}
Determina la distanza di~$P\left(-3;2,5\right)$ dall'origine del riferimento.
\end{esercizio}

\begin{esercizio}
\label{ese:D.24}
Calcola la misura del perimetro del triangolo~$ABC$ di vertici~$A(3;-2)$, $B(4;1)$, $C(7;-4)$
\end{esercizio}

\begin{esercizio}
\label{ese:D.25}
Determina il perimetro del quadrilatero di vertici~$A(1;5)$, $B(-4;5)$, $C(-4;-2)$, $D(5;-2)$
\end{esercizio}

\begin{esercizio}
\label{ese:D.26}
Determina il perimetro del quadrilatero di vertici~$M(6;-4)$, $N(8;3)$, $P(6;5)$, $Q(4;3)$
\end{esercizio}

\begin{esercizio}
\label{ese:D.27}
Determina il perimetro e la misura delle diagonali del quadrilatero di vertici~$A(1;-3)$, $B(4;3)$, $C(-3;1)$, $D(-6;-5)$
\end{esercizio}

\begin{esercizio}
\label{ese:D.28}
Verifica che il triangolo di vertici~$E(4;3)$, $F(-1;4)$, $G(3;-2)$ è isoscele.
\end{esercizio}

\begin{esercizio}
\label{ese:D.29}
Il triangolo~$ABC$ ha il lato~$BC$ appoggiato sull'asse~$x$ il vertice~$B$ ha ascissa~$\frac{5}{4}$,
il vertice~$C$ segue~$B$ e~$\overline{BC}=\frac{17}{2}$ Determina le coordinate del vertice~$C$,
l'area e il perimetro del triangolo sapendo che il terzo vertice è~$A(-1;5)$
\end{esercizio}

\begin{esercizio}
\label{ese:D.30}
I punti~$F(3;0)$, $O(0;0)$, $C(0;5)$ sono i vertici di un rettangolo; determina le coordinate del quarto vertice, il perimetro,
l'area e la misura delle diagonali del rettangolo.
\end{esercizio}

\begin{esercizio}
\label{ese:D.31}
I punti~$O(0;0)$, $A(4;5)$, $B(9;5)$, $C(3;0)$ sono i vertici di un trapezio.
Determina perimetro e area del trapezio~$OABC$
\end{esercizio}

\begin{esercizio}
\label{ese:D.32}
Determina le coordinate del punto medio dei segmenti i cui estremi sono le seguenti coppie di punti:
\begin{multicols}{2}
 \begin{enumeratea}
\item $A(-\sqrt{2};0),B(0;\sqrt{2})$
\item $A\left(\frac{2}{3};-\frac{3}{2}\right),B\left(-{\frac{1}{6}};3\right)$
\item $A(-1;4),B(1;-4)$
\item $A\left(0;-\frac{3}{2}\right),B\left(-2;-1\right)$
\item $A\left(1+\sqrt{2};\frac{1}{\sqrt{3}}\right),B\left(-\sqrt{2};-\frac{\sqrt{3}}{3}\right)$
\item $A\left(\frac{7}{5};-\frac{7}{5}\right),B(1;-1)$
\item $A\left(-3;\frac{1}{2}\right),B\left(\frac{1}{2};-3\right)$
\end{enumeratea}
\end{multicols}
\end{esercizio}

\begin{esercizio}
\label{ese:D.33}
I vertici del triangolo~$ABC$ sono i punti~$A\left(\frac{2}{3};-\frac{3}{2}\right)$, $B\left(-{\frac{1}{6}};1\right)$,
$C\left(\frac{4}{3};0\right)$, determina le coordinate dei punti~$M$, $N$, $P$, punti medi rispettivamente dei lati
$AB$, $AC$, $BC$
\end{esercizio}

\begin{esercizio}
\label{ese:D.34}
I vertici del triangolo~$ABC$ sono i punti~$A(-3;5)$, $B(3;-5)$, $C(3,5)$, i punti~$M$, $N$, $P$ sono i punti medi
rispettivamente dei lati~$AB$, $AC$, $BC$ Determina il perimetro di~$ABC$ e di~$MNP$
Quale relazione sussiste tra i perimetri ottenuti? Secondo te vale la stessa relazione anche tra le aree dei due triangoli?
\end{esercizio}

\begin{esercizio}
\label{ese:D.35}
Verifica che il triangolo di vertici~$A(2;3)$, $B(6;-1)$, $C(-4;-3)$ è rettangolo (è sufficiente verificare
che le misure dei lati verificano la relazione di Pitagora). È vero che~$CB$ è l'ipotenusa?
Verifica che~$AM$, con~$M$ punto medio di~$BC$ è metà di~$BC$ stesso. Come sono i triangoli~${AMC}$ e~${AMB}$?
\end{esercizio}

\begin{esercizio}
\label{ese:D.36}
Verifica che i segmenti~${AB}$ e~${CD}$ di estremi~$A\left(\frac{1}{2};2\right)$, $B\left(-{\frac{3}{4}};-2\right)$, $C(3;1)$,
$D\left(-{\frac{7}{2}};-1\right)$ hanno lo stesso punto medio. È vero che~${AC}={BD}$?
\end{esercizio}

% Da Vincenzo Gentile
\begin{esercizio}\label{ese:02_01.}
Verifica che il triangolo di vertici~$A(3;~2)$, $B(2;~5)$, $C(-4;~3)$ è 
rettangolo e calcola l'area. \hfill[10]
\end{esercizio}

\begin{esercizio}\label{ese:02_01.}
Verifica che il triangolo di vertici~$A(-4;~3)$, $B(-1;~-2)$, $C(1;~6)$ è 
isoscele e calcola l'area. \hfill[17]
\end{esercizio}

\begin{esercizio}\label{ese:02_01.}
Determinare la mediana relativa al lato~$AB$ del triangolo di 
vertici~$A(0;~4)$, $B(-2;~0)$, $C(2;~-2)$ \hfill[5]
\end{esercizio}

\begin{esercizio}\label{ese:02_01.}
Calcola le coordinate del baricentro~G del triangolo di 
vertici~$A(0;~0)$, $B(4;~3)$, $C(2;~-3)$ \hfill[(2;~0)]
\end{esercizio}

\begin{esercizio}\label{ese:02_01.}
Calcola le coordinate del baricentro~G del triangolo di 
vertici~$A(-3;~4)$, $B(-1;~-3)$, $C(1;~5)$ \hfill[(-1;~2)]
\end{esercizio}


