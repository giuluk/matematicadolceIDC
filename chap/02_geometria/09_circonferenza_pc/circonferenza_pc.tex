% (c) 2015 Daniele Zambelli daniele.zambelli@gmail.com

% (c) 2014 Daniele Zambelli - daniele.zambelli@gmail.com
% 
% Tutti i grafici per il capitolo relativo alle parabole
%
% 

\newcommand{\espdueterzi}{% 
    % Esponenziali con basi diverse.
    \disegno{
    \rcom{-10}{+10}{-1}{10}{gray!50, very thin, step=1}
    \begin{scope}[ultra thick, color=Maroon!50!black]
     \tkzInit[xmin=-10.3, xmax=+10.3, ymin=-0.3, ymax=+10.3]
     \tkzFct[domain=-10.3:+6]{(3./2)**x}
     \tkzFct[color=Green!50!black, domain=-6:+10.3]{(2./3)**x}
     \begin{scope}[color=Black!50!black]
      \filldraw (1, 3./2) circle (1.2pt);
      \filldraw (1, 2./3) circle (1.2pt);
     \end{scope}
     \filldraw [color=Red](0, 1) circle (1.2pt);  
    \end{scope}
    \begin{scope}[color=black]
     \draw (-7.3, 7) node{\(f(x)=\tonda{\dfrac{2}{3}}^x\)}; 
     \draw ((7.3, 7) node{\(f(x)=\tonda{\dfrac{3}{2}}^x\)};
    \end{scope}
    }
}

\newcommand{\logduebasi}{% 
    % Esponenziali con basi diverse.
    \disegno{
    \rcom{-1}{+10}{-9}{9}{gray!50, very thin, step=1}
    \begin{scope}[ultra thick, color=Maroon!50!black]
      \tkzInit[xmin=-1.3, xmax=+80, xstep=.5, ymin=-10.3,ymax=+10.3]
      \tkzFct[domain=.01:+10]{log(x)/log(2)}
      \filldraw (2, 1) circle (1.2pt);
      \begin{scope}[color=Green!50!black]
        \tkzFct[domain=-.01:+10]{log(x)/log(1./2)}
        \filldraw (2, -1) circle (1.2pt);
      \end{scope}
    \end{scope}
    \begin{scope}[color=black]
      \draw (9.5, 2.8) node{a=2}; 
      \draw (9.5, -2.8) node{a=0.5};
    \end{scope}
      \filldraw [color=Red] (1,0) circle (1.2pt);
    }
}


\chapter{La circonferenza nel piano cartesiano}

\section{TODO}

\section{Circonferenza con il centro nell'origine}
\label{sec:circ_circcentroorigine}

questo è il testo

qui metto un grafico

\begin{figure}[h]
 \centering
 \begin{minipage}[]{.48\textwidth}
  \begin{center}
   \begin{tabular}{r|l}
    $x$   & $y=x^2-2x-2$ \\
    \hline
    \dots & \dots \\
    $-5$ & $(-5)^2 -2(-5) -3 = 33$ \\
    $-4$ & $(-4)^2 -2(-4) -3 = 22$ \\
    $-3$ & $(-3)^2 -2(-3) -3 = 13$ \\
    $-2$ & $(-2)^2 -2(-2) -3 = 6$ \\
    $-1$ & $(-1)^2 -2(-1) -3 = 1$ \\
     $0$ & $(0)^2 -2(0) -3 = -2$ \\
    $+1$ & $(+1)^2 -2(+1) -3 = -3$ \\
    $+2$ & $(+2)^2 -2(+2) -3 = -2$ \\
    $+3$ & $(+3)^2 -2(+3) -3 = 1$ \\
    $+4$ & $(+4)^2 -2(+4) -3 = 6$ \\
    $+5$ & $(+5)^2 -2(+5) -3 = 13$ \\
    \dots & \dots \\
   \end{tabular}
  \caption{Alcuni valori del trinomio...} \label{tab:trinomio0}
  \end{center}
 \end{minipage}
\begin{minipage}[]{.48\textwidth}
\begin{center}
\begin{inaccessibleblock}[I punti della tabella precedente riportati nel piano 
cartesiano si dispongono lungo una curva.]
%   \puntia
  \caption{...i corrispondenti punti.} \label{fig:trinomio0}
\end{inaccessibleblock}
\end{center}
\end{minipage}
\end{figure}

\begin{center}
 $\sum \nexists$
\end{center}


\subsection{Circonferenza come luogo geometrico}
\label{subsec:circ_luogo}

\subsection{Equazione della circonferenza}
\label{subsec:circ_equazione}

% \begin{wrapfloat}{figure}{r}{0pt}
% \includegraphics[scale=0.35]{img/fig000_.png}
% \caption{...}
% \label{fig:...}
% \end{wrapfloat}
% 
% \begin{center} \input{\folder lbr/fig000_.pgf} \end{center}

\section{Circonferenza traslata}
\label{sec:circ_circtraslata}

\subsection{Traslazione lungo l'asse \(x\)}
\label{subsec:circ_traslazionex}

\subsection{Traslazione lungo l'asse \(y\)}
\label{subsec:circ_traslazioney}

\subsection{Traslazione generica)}
\label{subsec:circ_traslazione}

\section{Circonferenze e rette}
\label{sec:circ_circrette}

\subsection{Posizioni reciproche}
\label{subsec:circ_equazione}

\subsection{Tangenti ad una circonferenza}
\label{subsec:circ_tangenti}

\section{Posizioni reciproche tra circonferenze}
\label{sec:circ_posizionireciproche}

\subsection{Fasci di circonferenze}
\label{subsec:circ_fasci}

\section{Curve deducibili dall'equazione della circonferenza}
\label{sec:circ_posizionireciproche}


