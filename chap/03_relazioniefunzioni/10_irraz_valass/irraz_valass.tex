% (c) 2015 Daniele Zambelli daniele.zambelli@gmail.com

% \begin{wrapfloat}{figure}{r}{0pt}
% \includegraphics[scale=0.35]{img/fig000_.png}
% \caption{...}
% \label{fig:...}
% \end{wrapfloat}
% 
% \begin{center} \input{\folder lbr/fig000_.pgf} \end{center}

% \subsection{Funzioni, Equazioni e disequazioni con valore assoluto}
% \label{sec:irvalass_}

\chapter{Complementi di algebra}

\section{Equazioni di grado superiore al secondo}
% \label{sec:irvalass_supsec}

A questo punto siamo in grado di risolvere equazioni di primo e secondo 
grado. Impareremo ora come comportarci nel caso, più generale, di 
equazioni polinomiali di grado superiore.

\subsection{Equazioni che si possono risolvere tramite scomposizione}
% \label{subsec:irvalass_supsec_scomp}

Pensiamo, ad esempio, di dover risolvere un'equazione polinomiale del 
tipo $P(x)=0$, lo sappiamo già fare? Certo, se possiamo applicare la 
tecnica della scomposizione (raccoglimenti totali e parziali, prodotti 
notevoli, regola di Ruffini...). Scomponiamo il polinomio $P(x)$  
scrivendolo come  prodotto di più polinomi di grado minore e poi, 
mediante la legge dell'annullamento del prodotto, risolviamo le equazioni 
che abbiamo trovato.

Prima di procedere ricordiamo la legge dell'annullamento del prodotto:

\begin{definizione}
Legge dell'annullamento del prodotto: il prodotto di due o più fattori è 
uguale a zero quando almeno uno dei fattori è nullo.   
\end{definizione}

\begin{esempio}
$x^3-4x=0$
\begin{center}
\begin{tabular}{ll}
raccogliamo a fattore comune la $x$: & $x(x^2-4)=0$\\
applichiamo la legge dell'annullamento del prodotto: & $x=0 \sor (x^2-4)=0$\\
risolviamo le due equazioni ottenute: & $x=0 \sor x=\pm 2$
\end{tabular}
\end{center}
\end{esempio}

\begin{esempio}
$3x^3-2x^2-3x+2=0$
\begin{center}
\begin{tabular}{ll}
facciamo un primo raccoglimento parziale: & $3x(x^2-1)-2(x^2-1)=0$\\
raccogliamo a fattore comune la $x$: & $(x^2-1)(3x-2)=0$\\
applichiamo la legge dell'annullamento del prodotto: & $x^2-1=0 \sor 3x-2=0$\\
risolviamo le due equazioni ottenute: $x=\pm 1\sor x=\frac{2}{3}$& 
\end{tabular}
\end{center}
\end{esempio}

\begin{esempio}
$x^4-3x^3+2x^2=0$
\begin{center}
\begin{tabular}{ll}
raccogliamo a fattore comune la $x$: & $x^2(x^2-3x+2)=0$\\
applichiamo la legge dell'annullamento del prodotto: & $x^2=0 \sor x^2-3x+2=0$\\
risolviamo le due equazioni ottenute: & $x=0 \sor x=1 \sor x=2$
\end{tabular}
\end{center}
\end{esempio}

\subsection{Equazioni monomie}

\begin{definizione}[Equazione monomia]
Un'equazione si dice \textbf{monomia} se può essere scritta nella forma:
$$ax^n=0$$      
\end{definizione}

Ricordando che $$x^n=0$$
equivale a $$\underbrace{x\cdot x\cdot x\cdot x \dots \cdot x}_{\text{$n$ 
volte}}=0$$
e per la legge dell'annullamento del prodotto, abbiamo $$x=0 \sor x=0 
\sor x=0 \sor x=0 \sor \cdots \sor x=0 $$
Si può dire quindi che l'equazione $ax^n=0$ ha $n$ soluzioni 
\emph{coincidenti} uguali a 0.

\subsection{Equazioni binomie}

\begin{definizione}[Equazione binomia]
Un'equazione si dice \textbf{binomia} se può essere scritta nella forma:
$$ax^n+b=0$$
dove $n$ è un numero \emph{intero positivo} e $a$ e $b$  \emph{numeri 
reali}  non nulli.  
\end{definizione}

Il numero delle soluzioni dipende da $n$ e dal segno di $a$ e $b$.
Infatti se riscriviamo l'equazione
$$ax^n+b=0$$
e risolviamo rispetto a $x^n$, otteniamo l'equazione equivalente:
$$x^n=-\frac{b}{a}$$

\begin{itemize}
\item se $n$ è \textbf{pari} l'equazione ammette soluzioni reali 
solo se  $-\frac{b}{a}>0$ e le soluzioni saranno date da:
$$x=\pm \sqrt[n]{-\frac{b}{a}}$$
se $-\frac{b}{a}<0$ l'equazione non ammette radici reali in 
quanto non esiste la radice di indice pari di un numero negativo.
\item se $n$ \textbf{dispari} l'equazione ammette sempre 
una sola soluzione reale data da  $$x= \sqrt[n]{-\frac{b}{a}}$$ 
\end{itemize}

\begin{esempio}
$8x^3+1=0$
\begin{center}
\begin{tabular}{ll}
Risolviamo rispetto a $x^3$ e otteniamo l'equazione equivalente: & 
$x^3=-\frac{1}{8}$\\
estraiamo quindi la radice cubica: & $x=\sqrt[3]{-\frac{1}{8}} = -2$
\end{tabular}
\end{center}
\end{esempio}

\begin{esempio}
$4x^2-9=0$
\begin{center}
\begin{tabular}{ll}
Risolviamo rispetto a $x^2$ e otteniamo l'equazione equivalente: & 
$x^2=\frac{9}{4}$\\
le soluzioni di questa equazione sono due: & 
$x=\pm \sqrt{\frac{9}{4}}=\pm\frac{3}{2}$
\end{tabular}
\end{center}
\end{esempio}    

\begin{esempio}
$2x^2 +50=0$

Risolviamo rispetto a $x^2$ e otteniamo l'equazione equivalente: 

$x^2=-\frac{50}{2}=-25$

poiché non ci sono numeri reali che elevati alla seconda diano un risultato 
negativo: 

\emph{L'equazione Non Ha Soluzioni Reali}
\end{esempio}    

\begin{esempio}
$-\frac{2}{3}x^6+2=0$
\begin{center}
\begin{tabular}{ll}
Risolviamo rispetto a $x^6$ e otteniamo l'equazione equivalente: & 
$x^6=-\frac{-2}{-\frac{2}{3}}=3$\\
semplifichiamo: & $x^6=3$\\
le soluzioni di questa equazione sono due: & $x=\pm \sqrt[6]{3}$
\end{tabular}
\end{center}
\end{esempio}

\subsection{Equazioni trinomie particolari}

\begin{definizione}[Equazione trinomia particolare]
Un'equazione si dice \textbf{trinomia particolare} se può essere scritta nella 
forma:
$$ax^{2n}+bx^n+c=0$$
dove $n$ è un numero intero positivo e $a$, $b$ e $c$  numeri reali  non 
nulli. 
\end{definizione}

Possiamo distinguere tre casi:
\begin{itemize}
\item se $n=1$ l'equazione diventa $ax^{2}+bx+c=0$, si riduce 
quindi a un'equazione di secondo grado.
\item se $n=2$ l'equazione diventa $ax^{4}+bx^2+c=0$, con $a$, 
$b$ e $c$ numeri reali  non nulli e viene chiamata \textbf{equazione 
biquadratica}.
\item se $n\geq 2$ le equazioni trinomie si possono ricondurre a 
equazioni di secondo grado tramite una semplice sostituzione:\\
ponendo infatti $x^n=z$ e quindi $x^{2n}=(x^n)^2=z^2$, l'equazione 
di partenza
$$ax^{2n}+bx^n+c=0$$
diventa:
$$az^{2}+bz+c=0$$
ora non resta che risolvere questa equazione, se non troviamo 
soluzioni reali, neppure quella di partenza ammette soluzioni reali, se 
invece ammette soluzioni reali, ad esempio $z_1$ e $z_2$  le soluzioni 
dell'equazione originaria saranno le soluzioni delle due equazioni 
binomie:
\begin{center}
  $x^n=z_1$ e $x^n=z_2$.
\end{center}
\end{itemize}


\begin{comment}
\begin{esempio}

\begin{center}
\begin{tabular}{ll}
: & \\
: & \\
: & 
\end{tabular}
\end{center}
\end{esempio}
\end{comment}

\begin{esempio}
  $x^6+9x^3+8=0$

  Poniamo $x^3=z$, l'equazione diventa: $z^2+9z+8=0$

  Risolviamo questa equazione: 
  $\tonda{z+8}\tonda{z+1}=0 \sRarrow z_1=-8 \sor z_2=-1$

  Ritorniamo ora alla variabile $x$:
  \begin{itemize} [nosep]
    \item $x^3=z_1=-8 \sRarrow x_1=\sqrt[3]{-8}=-2$ 
    \item $x^3=z_2=-1 \sRarrow x_2=\sqrt[3]{-1}=-1$
  \end{itemize}
\end{esempio}

\begin{esempio}
  $x^4+x^2-6=0$
  
  Poniamo $x^2=z$, l'equazione diventa $z^2+z+-6=0$ 
  
  Questa equazione ha soluzioni:
  $\tonda{z+3}\tonda{z-2}=0 \sRarrow z_1=-3 \sor z_2=+2$
  
  Ritorniamo ora alla variabile $x$:
  \begin{itemize} [nosep]
    \item $x^2=z_1 \sRarrow x=-3$ che non ha soluzioni reali
    \item $x^2=z_2 \sRarrow x=+2$ che soluzioni $x_{1,2}=\pm \sqrt[]{2}$
  \end{itemize}
\end{esempio}

\begin{esempio}
  $x^{10}-10x^5+25=0$ 
  
  Poniamo $x^5=z$, l'equazione diventa $z^2-10z+25=0$
  
  Questa equazione ha soluzioni:
  $z_1=5$
  
  Ritorniamo ora alla variabile $x$:
  \begin{itemize} [nosep]
    \item $x^5=z_1 \sRarrow x=5$ che ha soluzione $x=\sqrt[5]{5}$
  \end{itemize}
\end{esempio}