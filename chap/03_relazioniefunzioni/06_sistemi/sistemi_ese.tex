% (c) 2012-2014 - Dimitrios Vrettos d.vrettos@gmail.com
% (c) 2014 Daniele Zambelli - daniele.zambelli@gmail.com

\section{Esercizi}

\subsection{Esercizi dei singoli paragrafi}

%\subsubsection*{22.1 - Equazione lineare in due incognite}
\subsubsection*{\numnameref{sec:sist_eqdue}}

\begin{esercizio}
 \label{ese:22.1}
Completa la tabella delle coppie di soluzioni dell'equazione~$x+2y-1=0$

\begin{tabular*}{.9\textwidth}{@{\extracolsep{\fill}}*{10}{lccccccccc}}
\toprule
$x$ & & $-1$ & 0 & &$\frac{1}{2}$ & & & 2,25 &\\
$y$ & 0 & & & $-1$ & & $\frac{3}{4}$ & 2 & & 1,5\\
\bottomrule
\end{tabular*}
\end{esercizio}

\begin{esercizio}
 \label{ese:22.2}
Completa la tabella delle coppie di soluzioni dell'equazione~$3x-2y=5$

\begin{tabular*}{.9\textwidth}{@{\extracolsep{\fill}}*{10}{lccccccccc}}
\toprule
$x$ & & 0 & 1 & & $\frac{1}{6}$ & & & $-\sqrt{2}$ & 0,25\\
$y$ & 0 & & &$-1$ & & $\frac{3}{4}$ & $\sqrt{2}$ & & \\
\bottomrule
\end{tabular*}
\end{esercizio}

\begin{esercizio}
 \label{ese:22.3}
 Completa la tabella delle coppie di soluzioni dell'equazione~$3x-2\sqrt{2}y=0$

 \begin{tabular*}{.9\textwidth}{@{\extracolsep{\fill}}*{8}{lccccccc}}
\toprule
$x$ & & 0 & & & $\frac{1}{6}$ & & $\sqrt{2}$ \\
$y$ & 0 & & 1 &$-1$ & & $\sqrt{2}$ & \\
\bottomrule
\end{tabular*}
\end{esercizio}

%%%%%%%%%%%%%%%%%%%%%%%%%%%%%%%%%%%%%%%%%%%%%%%%%%%%%%%%%%
\begin{esercizio}
 \label{ese:22.4}
Risolvi graficamente le seguenti equazioni in due incognite.
\begin{multicols}{2}
 \begin{enumeratea}
\spazielenx
\item $2x-2y+3=0$
\item $-{\dfrac{1}{5}}x-\dfrac{5}{2}y+1=0$
\item $-2y+3=0$
\item $x+2y+\dfrac{7}{4}=0$
\end{enumeratea}
\end{multicols}
\end{esercizio}

\begin{esercizio}
 \label{ese:22.5}
Risolvi graficamente le seguenti equazioni in due incognite.
\begin{multicols}{2}
 \begin{enumeratea}
\spazielenx
\item $-2x+4y-1=0$
\item $2y+\dfrac{2}{3}x+6=0$
\item $\sqrt{2}x+\sqrt{6}y=0$
\item $\sqrt{3}y+\sqrt{6}=-x$
 \end{enumeratea}
\end{multicols}
\end{esercizio}

\begin{esercizio}
 \label{ese:22.6}
 Stabilisci quali coppie appartengono all'Insieme Soluzione dell'equazione.
\TabPositions{4cm}
\begin{enumeratea}
\item $5x+7y-1=0$\tab$\left(-\frac{7}{5};0\right), 
\left(-\frac{1}{5};-1\right), \left(0;\frac{1}{7}\right), 
\left(\frac{2}{5};-\frac{1}{7}\right)$
\item $-x+\dfrac{3}{4}y-\dfrac{4}{3}=0$\tab$(0;-1), 
\left(\frac{1}{12};\frac{7}{9}\right), 
\left(-\frac{4}{3};0\right), (-3;4)$
\item $-x-y+\sqrt{2}=0$\tab$\left(\sqrt{2};0\right), 
\left(0;-\sqrt{2}\right), \left(1+\sqrt{2};-1\right), 
\left(1;-1-\sqrt{2}\right)$
\end{enumeratea}
\end{esercizio}

%%%%%%%%%%%%%%%%%%%%%%%%%%%%%%%%%%%%%%%%%%%%%%%%%%%%%%%%%%

\begin{esercizio}
 \label{ese:22.7}
Risolvi i seguenti sistemi con il metodo di sostituzione.
 \begin{multicols}{2}
 \begin{enumeratea}
  \item $\left\{\begin{array}{l}
     y=-2\\
     2x-y+2=0
    \end{array}\right.$
  \item $\left\{\begin{array}{l}
	   y=-x+1\\
	   2x+3y+4=0
       \end{array}\right.$
 \end{enumeratea}
 \end{multicols}
\end{esercizio}

\begin{esercizio}[\Ast]
 \label{ese:22.8}
Risolvi i seguenti sistemi con il metodo di sostituzione.
 \begin{multicols}{2}
 \begin{enumeratea}
  \item $\left\{\begin{array}{l}
        x=1\\
        x+y=1
       \end{array}
\right.$ \hfill $\left[(1;~0)\right]$
\item $\left\{\begin{array}{l}
        y=x\\
        2x-y+2=0
       \end{array}\right.$ \hfill $\left[(-2;~-2)\right]$
\item $\left\{\begin{array}{l}
  2x+y=1\\
  2x-y=-1
	  \end{array}\right.$ \hfill $\left[(0;~1)\right]$
\item $\left\{\begin{array}{l}
	   2y=2\\
	   x+y=1
	\end{array}\right.$ \hfill $\left[(0;~1)\right]$
\item $\left\{\begin{array}{l}3x-y=7\\x+2y=14\end{array}\right.$ 
 \hfill $\left[(4;~5)\right]$
\item $\left\{\begin{array}{l}3x-2y=1\\4y-6x=-2\end{array}\right.$ 
 \hfill $\left[indeterminato\right]$
\item $\left\{\begin{array}{l}3x+y=2\\x+2y=-1\end{array}\right.$ 
 \hfill $\left[(1;~-1)\right]$
\item $\left\{\begin{array}{l}x+4y-1=3\\
	\dfrac{x}{2}+\dfrac{y}{3}+1=-{\dfrac{x}{6}}-1
	\end{array}\right.$ 
 \hfill $\left[(-4;~2)\right]$
  \item $\left\{\begin{array}{l}2x-3y=2\\6x-9y=6\end{array}\right.$
 \hfill $\left[indeterminato\right]$
\item $\left\{\begin{array}{l}x+2y=14\\3x-y=7\end{array}\right.$
 \hfill $\left[(4;~5)\right]$
\item $\left\{\begin{array}{l}x+2y=1\\-2x-4y=2\end{array}\right.$
 \hfill $\left[impossibile\right]$
\item $\left\{\genfrac{}{}{0pt}{0}{2x-y=3}{-6x+3y=-9}\right.$
 \hfill $\left[indeterminato\right]$
 \item $\longarray\left\{\begin{array}{l}
\dfrac{x-4y}{3}=x-5y\\
x-2=6y+4 \end{array}\right.$
 \hfill $\left[(-66;~-12)\right]$
\item $\longarray\left\{\begin{array}{l}
\dfrac{y^{2}-4x+2}{5}=\dfrac{2y^{2}-x}{10}-1\\
x=-2y+8\end{array}\right.$
 \hfill $\left[(2;~3)\right]$
\item $\longarray\left\{\begin{array}{l}
3x-\dfrac{3}{4}(2y-1)=\dfrac{13}{4}(x+1)\\
\dfrac{x+1}{4}-\dfrac{y}{2}=\dfrac{1+y}{2}-\dfrac{1}{4}\end{array}\right.$
 \hfill $\left[...\right]$
\item $\longarray\left\{\begin{array}{l}
\dfrac{x}{3}-\dfrac{y}{2}=0\\
\dfrac{y-x-1}{2}+x-y+1=\dfrac{1}{2}\end{array}\right.$
 \hfill $\left[(0;~0)\right]$
\item $\longarray\left\{\begin{array}{l}
y-\dfrac{x}{3}+\dfrac{3}{4}=0\\
\dfrac{2x+1}{2}+\dfrac{2+y}{5}=-1
\end{array}\right.$
 \hfill $\left[\left(-{\frac{9}{8}};~-\frac{9}{8}\right)\right]$
\item $\longarray\left\{\begin{array}{l}
x+y=2\\
3\left(\dfrac{x}{6}+3y\right)=4\end{array}\right.$
 \hfill $\left[\left(\frac{28}{17};~\frac{6}{17}\right)\right]$
\item $\longarray\left\{\begin{array}{l}
\dfrac{1}{2}y-\dfrac{1}{6}x=5-\dfrac{6x+10}{4}\\
2(x-2)-3x=40-6\left(y-\dfrac{1}{3}\right)\end{array}\right.$
 \hfill $\left[...\right]$
\item $\longarray\left\{\begin{array}{l}
2\dfrac{y}{3}+x+1=0\\
\dfrac{y+1}{2}+\dfrac{x-1}{3}+1=0\end{array}\right.$
 \hfill $\left[(1;~-3)\right]$
%   \item $\longarray\left\{\begin{array}{l}
%   (x-2)^{2}+y=(x+1)(x-y)+(3-y)(2-x)\\
%   \dfrac{x}{4}-2y=2 \end{array}\right.$
%  \hfill $\left[\left(-4;~-{\frac{3}{2}}\right)\right]$
  \item $\left\{\begin{array}{l}
  x+y+1=0 \\
  x-y+k=0 \end{array}\right.$
 \hfill $\left[...\right]$
\item $\longarray\left\{\begin{array}{l}
y-\dfrac{3-2x}{3}=\dfrac{x-y}{3}+1\\
\dfrac{x+1}{2}+\dfrac{5}{4}=y+\dfrac{2-3x}{4}\end{array}\right.$
 \hfill $\left[\left(\frac{1}{6};~\frac{35}{24}\right)\right]$
\item $\left\{\begin{array}{l}
x-2y-3=0\\
kx+(k+1)y+1=0 \end{array}\right.$
 \hfill $\left[...\right]$
 \end{enumeratea}
 \end{multicols}
\end{esercizio}

% \begin{esercizio}
%  \label{ese:22.14}
%  Risolvere il sistema che formalizza il problema~23.4:
% \[\longarray\left\{\begin{array}{l}2x+\dfrac{1}{2}y=98
% \\2x+3y=170 \end{array}\right.,\]
% e concludere il problema determinando l'area del rettangolo.
% \end{esercizio}

\begin{esercizio}
 \label{ese:22.15}
Determinare due numeri reali~$x$ e~$y$ tali che il
triplo della loro somma sia uguale al doppio del primo aumentato di~10
e il doppio del primo sia uguale al prodotto del secondo con~5.
 \end{esercizio}

%%%%%%%%%%%%%%%%%%%%%%%%%%%%%%%%%%%%%%%%%%%%%%%%%%%%%%%%%%

%%%%%%%%%%%%%%%%%%%%%%%%%%%%%%%%%%%%%%%%%%%%%%%%%%%%%%%%%%
 \begin{esercizio}[\Ast]
 \label{ese:22.20}
Risolvere i seguenti sistemi con il metodo di riduzione.

\begin{multicols}{2}
 \begin{enumeratea}
 \item $\left\{\begin{array}{l}x+y=0\\-x+y=0\end{array}\right.$
 \hfill $\left[(0;~0)\right]$
\item $\left\{\begin{array}{l}3x+y=5\\x+2y=0\end{array}\right.$
 \hfill $\left[(2;~-1)\right]$
\item $\left\{\begin{array}{l}x=2\\x+y=3\end{array}\right.$
 \hfill $\left[(2;~1)\right]$
\item $\left\{\begin{array}{l}x=-1\\2x-y=1\end{array}\right.$
 \hfill $\left[(-1;~-3)\right]$
 \item $\left\{\begin{array}{l}y=2x-1\\y=2x\end{array}\right.$
 \hfill $\left[impossibile\right]$
\item $\left\{\begin{array}{l}x-2y=1\\2x-y=7\end{array}\right.$
 \hfill $\left[...\right]$
\item $\left\{\begin{array}{l}x+y=2\\-x-y=2\end{array}\right.$
 \hfill $\left[impossibile\right]$
 \item $\left\{\begin{array}{l}x+y=0\\-x+y=0\end{array}\right.$
 \hfill $\left[(0;~0)\right]$
 \item $\left\{\begin{array}{l}2x+y=1 \\2x-y=-1\end{array}\right.$
 \hfill $\left[(0;~1)\right]$
 \item $\left\{\begin{array}{l}2x+y=1+y\\4x+y=2\end{array}\right.$
 \hfill $\left[\left(\frac{1}{2};~0\right)\right]$
 \item $\left\{\begin{array}{l}x+y=0\\x-y=-1\end{array}\right.$
 \hfill $\left[\left(-{\frac{1}{2}};~\frac{1}{2}\right)\right]$
 \item $\left\{\begin{array}{l}x-y=0\\-2x+3y=1\end{array}\right.$
 \hfill $\left[(1;~1)\right]$
 \item $\left\{\begin{array}{l}2x=3-x\\2x+y=3\end{array}\right.$
 \hfill $\left[(1;~1)\right]$
 \hfill $\left[\left(\frac{35}{12};~\frac{19}{12}\right)\right]$
 \item $\left\{\begin{array}{l}5y+2x=1 \\3x+2y+2=0\end{array}\right.$
 \hfill $\left[\left(-{\frac{12}{11}};~\frac{7}{11}\right)\right]$
 \item $\left\{\begin{array}{l}-3x+y=2\\5x-2y=7\end{array}\right.$
 \hfill $\left[(-11;~-31)\right]$
 \item $\left\{\begin{array}{l}
 \dfrac{2}{3}x+\dfrac{2}{3}y=3\\
 \dfrac{3}{2}x-\dfrac{3}{2}y=2\end{array}\right.$
 \item $\left\{\begin{array}{l}2x=3-x\\2x+y=3\end{array}\right.$
 \hfill $\left[(1;~1)\right]$
 \item $\left\{\begin{array}{l}x+ay+a=0\\2x-ay+a=0\end{array}\right.$
 \hfill $\left[...\right]$
 \item $\left\{\begin{array}{l}2ax+2y-1=0\\ax+y=3\end{array}\right.$
 \hfill $\left[...\right]$
 \item $\left\{\begin{array}{l}2x-y=4\\x-\dfrac{1}{2}y=2\end{array}\right.$
 \hfill $\left[indeterminato\right]$
\end{enumeratea}
 \end{multicols}
\end{esercizio}

\begin{esercizio}
 \label{ese:22.19}
In un triangolo isoscele la somma della base con il doppio del lato
è~$168\unit{m}$ e la differenza tra la metà della base e~1/13 del lato 
è~$28\unit{m}$.
Indicata con~$x$ la misura della base e con~$y$ quella del lato,
risolvete con il metodo del confronto il sistema lineare che formalizza
il problema. Determinate l'area del triangolo.
 \end{esercizio}

\begin{esercizio}
 \label{ese:22.23}
Il segmento~$AB$ misura~$80\unit{cm}$ il punto~$P$ lo divide in due parti tali 
che il quadruplo della parte minore uguagli il triplo della differenza fra 
la maggiore e il triplo della minore. 
Determinare~$\overline{AP}$ e~$\overline{PB}$, formalizzando
il problema con un sistema lineare che risolverete con il metodo di
riduzione.
\begin{center}
 % (c) 2012 Dimitrios Vrettos - d.vrettos@gmail.com
\begin{tikzpicture}[font=\small,x=10mm, y=10mm]
	\draw (0,0) --(8,0);

\node [below left] at (0,0) {$A$};
\node[below right]  at (8,0) {$B$};
\draw[fill=blue] (1.5,0) circle (1.5pt) node [below]  {$P$};

\end{tikzpicture}
\end{center}
\end{esercizio}

%%%%%%%%%%%%%%%%%%%%%%%%%%%%%%%%%%%%%%%%%%%%%%%%%%%%%%%%%%
% \begin{esercizio}
%  \label{ese:22.24}
% Stabilire se è determinato il sistema:
% \[\left\{\begin{array}{l}
% (x-1)(x+1)-3(x-2)=2(x-y+3)+x^{2}\\
% x(x+y-3)+y(4-x)=x^{2}-4x+y\end{array}\right.\]
% \end{esercizio}
% 
% \begin{esercizio}
%  \label{ese:22.25}
% Verificare che il determinante della matrice del sistema è nullo:
% \[\longarray\left\{\begin{array}{l}
% \dfrac{3}{2}x-\dfrac{7}{4}y=10^{5}\\
% 6x-7y=5^{10}\end{array}\right.\]
% \end{esercizio}
% 
%  %%%%%%%%%%%%%%%%%%%%%%%%%%%%%%%%%%%%%%%%%%%%%%%%%%%%%%%%%%
%  \begin{esercizio}[\Ast]
%  \label{ese:22.26}
% Risolvere con la regola di Cramer i seguenti sistemi.
% \begin{multicols}{2}
% \begin{enumeratea}
% \item $\left\{\begin{array}{l}x-3y=2\\x-2y=2\end{array}\right.$
% \item $\left\{\begin{array}{l}2x+2y=3\\3x-3y=2\end{array}\right.$
% \item $\left\{\begin{array}{l}5y+2x=1 \\3x+2y+2=0\end{array}\right.$
% \item $\left\{\begin{array}{l}5x+2y=-1 \\3x-2y=1\end{array}\right.$
% \end{enumeratea}
% \end{multicols}
% \end{esercizio}
% 
% \begin{esercizio}[\Ast]
%  \label{ese:22.27}
% Risolvere con la regola di Cramer i seguenti sistemi.
% \begin{multicols}{2}
% \begin{enumeratea}
% \item $\longarray\left\{\begin{array}{l}
% \dfrac{1}{2}y-\dfrac{2}{3}y=2\\
% \dfrac{1}{3}x+\dfrac{1}{2}y=1 \end{array}\right.$
% \item $\longarray\left\{\begin{array}{l}
% \dfrac{y}{5}-\dfrac{x}{2}=10\\
% \dfrac{x}{3}+\dfrac{y}{2}=5 \end{array}\right.$
% \item $\left\{\begin{array}{l}
% 2(x-2y)+3x-2(y+1)=0\\
% x-2(x-3y)-5y=6(x-1) \end{array}\right.$
% \item $\longarray\left\{\begin{array}{l}
% 4-2x=\dfrac{3}{2}(y-1)\\
% \dfrac{2x+3y}{2}=\dfrac{7+2x}{2}\end{array}\right.$
% \end{enumeratea}
% \end{multicols}
% \end{esercizio}
% 
% \begin{esercizio}[\Ast]
%  \label{ese:22.28}
% Risolvere con la regola di Cramer i seguenti sistemi.
% \begin{multicols}{2}
% \begin{enumeratea}
% \item $\left\{\begin{array}{l}3x+y=-3\\-2x+3y=+2\end{array}\right.$
% \item $\left\{\begin{array}{l}6x-2y=5\\x+\frac{1}{2}y=0\end{array}\right.$
% \item $\left\{\begin{array}{l}10x-20y=-11\\x+y-1=0\end{array}\right.$
% \item $\left\{\begin{array}{l}2x-3y+1=0\\4x+6y=0\end{array}\right.$
% \end{enumeratea}
% \end{multicols}
% \end{esercizio}
% 
% \begin{esercizio}[\Ast]
%  \label{ese:22.29}
% Risolvere con la regola di Cramer i seguenti sistemi.
% \begin{multicols}{2}
% \begin{enumeratea}
% \item $\left\{\begin{array}{l}x+2y=1\\2x+4y=1\end{array}\right.$
% \item $\left\{\begin{array}{l}3x+2y=4\\\frac{3}{2}x+y=2\end{array}\right.$
% \item $\left\{\begin{array}{l}ax+ay=3a^{2}\\x-2y=-3a\end{array}\right.$
% \item $\left\{\begin{array}{l}3x-2y=8k\\x-y=3k\end{array}\right.$
% \end{enumeratea}
% \end{multicols}
% \end{esercizio}
% 
% \paragraph{\ref{ese:22.26}} a)~$(2;0)$, 
% b)~$\left(\frac{13}{12};\frac{5}{12}\right)$, 
% c)~$\left(-{\frac{12}{11}};\frac{7}{11}\right)$, 
% d)~$\left(0;-\frac{1}{2}\right)$
% 
% \paragraph{\ref{ese:22.27}} a)~$(21,-12)$, 
% b)~$\left(-{\frac{240}{19}};\frac{350}{19}\right)$, 
% c)~$\left(\frac{34}{37};\frac{16}{37}\right)$, d)~$\left(1;\frac{7}{3}\right)$
% 
% \paragraph{\ref{ese:22.28}} a)~$\left(-1;0\right)$, 
% b)~$\left(\frac{1}{2};-1\right)$, c)~$\left(\frac{3}{10};\frac{7}{10}\right)$, 
% d)~$\left(-{\frac{1}{4}};\frac{1}{6}\right)$
% 
% \paragraph{\ref{ese:22.29}} a)~impossibile, b)~indeterminato, \protect\\ 
% c)~$(a;2a)$, d)~$(2k;-k)$

% \begin{esercizio}
%  \label{ese:22.30}
%  Risolvi col metodo di Cramer il sistema
% \[\longarray\left\{\begin{array}{l}
%  25x-3y=18\\
%  \dfrac{3(y+6)}{5}=5x
% \end{array}\right.\]
% \[\longarray\left\{\begin{array}{l}
%  9x-5y=18\\
%  \dfrac{5(y+6)}{3}=3x
% \end{array}\right.\]
% Cosa osservi?
% \end{esercizio}

%%%%%%%%%%%%%%%%%%%%%%%%%%%%%%%%%%%%%%%%%%%%%%%%%%%%%%%%%%


\begin{esercizio}
 \label{ese:22.31}
Per ciascuno dei seguenti sistemi stabilisci se è determinato,
indeterminato, impossibile.
\begin{multicols}{2}
\begin{enumeratea}
\item $\left\{\begin{array}{l}x-2y=3 \\4x-8y=12\end{array}\right.$
 \hfill $\left[...\right]$
\item $\left\{\begin{array}{l}x-2y=3 \\2x-4y=5\end{array}\right.$
 \hfill $\left[...\right]$
\item $\left\{\begin{array}{l}x-2y=3 \\2x-6y=12\end{array}\right.$
 \hfill $\left[...\right]$
\item $\longarray\left\{\begin{array}{l}
\dfrac{1}{2}x-\dfrac{3}{2}y=-2\\
\dfrac{5}{4}x-\dfrac{15}{4}y=-{\dfrac{5}{2}}\end{array}\right.$
 \hfill $\left[...\right]$
\item $\longarray\left\{\begin{array}{l}
\dfrac{1}{7}x-\dfrac{4}{5}y=0\\
\dfrac{5}{4}x-7y=\dfrac{19}{2}\end{array}\right.$
 \hfill $\left[...\right]$
\item $\left\{\begin{array}{l}2x+y=1 \\2x-y=-1\end{array}\right.$
 \hfill $\left[...\right]$
\item $\left\{\begin{array}{l}-40x+12y=-3\\17x-2y=100\end{array}\right.$
 \hfill $\left[...\right]$
\item $\left\{\begin{array}{l}x-y=3 \\-x+y=1 \end{array}\right.$
 \hfill $\left[...\right]$
\item $\longarray\left\{\begin{array}{l}
-x+3y=-{\dfrac{8}{15}}\\
5x-15y=\dfrac{2^{3}}{3}\end{array}\right.$
 \hfill $\left[...\right]$
\item $\longarray\left\{\begin{array}{l}
\dfrac{x}{2}=-{\dfrac{y}{2}}+1\\
x+y=2\end{array}\right.$
 \hfill $\left[...\right]$
\end{enumeratea}
\end{multicols}
\end{esercizio}

\begin{esercizio}
 \label{ese:22.34}
La somma di due numeri reali è~16 e il doppio del
primo aumentato di~4 uguaglia la differenza tra~5 e il doppio del
secondo. Stabilisci, dopo aver formalizzato il problema con un sistema
lineare, se è possibile determinare i due numeri.
\end{esercizio}

\begin{esercizio}
 \label{ese:22.35}
Stabilisci per quale valore di~$a$ il sistema
$\left\{\begin{array}{l}ax+y=-2\\-3x+2y=0\end{array}\right.$ è
determinato. Se~$a=-{\frac{3}{2}}$ il sistema è indeterminato o
impossibile?
\end{esercizio}

\begin{esercizio}
 \label{ese:22.36}
Perché se~$a=\frac{1}{3}$ il sistema
$\left\{\begin{array}{l}x+ay=2a\\3x+y=2\end{array}\right.$ è
indeterminato?
\end{esercizio}

\begin{esercizio}
 \label{ese:22.37}
Per quale valore di~$k$ è impossibile il sistema?
\[\left\{\begin{array}{l}2x-3ky=2k\\x-ky=2k \end{array}\right.\]
\end{esercizio}

\begin{esercizio}
 \label{ese:22.38}
Per quale valore di~$k$ è indeterminato il sistema?
\[\left\{\begin{array}{l}(k-2)x+3y=6 \\(k-1)x+4y=8 \end{array}\right.\]
\end{esercizio}

%%%%%%%%%%%%%%%%%%%%%%%%%%%%%%%%%%%%%%%%%%%%%%%%%%%%%%%%%%
\begin{esercizio}[\Ast]
 \label{ese:22.39}
Risolvi graficamente i sistemi, in base al disegno verifica se le rette
sono incidenti, parallele o coincidenti e quindi se il sistema è
determinato, impossibile o indeterminato.
\begin{multicols}{2}
\begin{enumeratea}
\item $\left\{\begin{array}{l}y=2x-1\\y=2x+1\end{array}\right.$
 \hfill $\left[impossibile\right]$
\item $\left\{\begin{array}{l}y=2x-2\\y=3x+1\end{array}\right.$
 \hfill $\left[(-3;~-8)\right]$
\item $\left\{\begin{array}{l}y=x-1 \\2y=2x-2 \end{array}\right.$
 \hfill $\left[indeterminato\right]$
\item $\left\{\begin{array}{l}2x-y=2 \\2y-x=2\end{array}\right.$
 \hfill $\left[(2;~2)\right]$
\item $\left\{\begin{array}{l}{3x+y=-3}\\{-2x+3y=-2}\end{array}\right.$
 \hfill $\left[\left(-1;~0\right)\right]$
\item $\left\{\begin{array}{l}x-3y=2\\x-2y=2\end{array}\right.$
 \hfill $\left[(2;~0)\right]$
\item $\left\{\begin{array}{l}{3x+y=-3}\\{-2x+3y=+2}\end{array}\right.$
 \hfill $\left[\left(-1;~0\right)\right]$
\item $\left\{\begin{array}{l}2x=2-y\\2x-y=1\end{array}\right.$
 \hfill $\left[impossibile\right]$
\item $\left\{\begin{array}{l}5x+2y=-1 \\3x-2y=1 \end{array}\right.$
 \hfill $\left[\left(0;~-\frac{1}{2}\right)\right]$
\item $\left\{\begin{array}{l}2x=3-x\\2x+y=3\end{array}\right.$
 \hfill $\left[(1;~1)\right]$
\item $\left\{\begin{array}{l}2x=2-y\\2x-y=1\end{array}\right.$
 \hfill $\left[\left(\frac{3}{4};~\frac{1}{2}\right)\right]$
\end{enumeratea}
\end{multicols}
\end{esercizio}

\begin{esercizio}
 \label{ese:22.42}
Vero o falso?
\TabPositions{11cm}
 \begin{enumeratea}
\item Risolvere graficamente un sistema lineare significa trovare il punto 
di intersezione di due rette? 
\tab\boxV\quad\boxF
\item Un sistema lineare, determinato ha una sola coppia soluzione?
\tab\boxV\quad\boxF
\item Un sistema lineare è impossibile quando le due rette coincidono?
\tab\boxV\quad\boxF
\end{enumeratea}
\end{esercizio}

\begin{esercizio}
 \label{ese:22.43}
Completa:

\begin{itemize*}
\item se~$r_{1}\cap r_{2}=r_{1}=r_{2}$ allora il sistema è\dotfill;
\item se~$r_{1}\cap r_{2}=P$ allora il sistema è\dotfill;
\item se~$r_{1}\cap r_{2}=\emptyset $ allora il sistema è\dotfill
\end{itemize*}
\end{esercizio}

%%%%%%%%%%%%%%%%%%%%%%%%%%%%%%%%%%%%%%%%%%%%%%%%%%%%%%%%%%
% \subsubsection*{22.3 - Sistemi fratti}
% \subsubsection*{\numnameref{sec:22_fratti}}
% 
% \begin{esercizio}[\Ast]
%  \label{ese:22.44}
% Verifica l'insieme soluzione dei seguenti sistemi.
% \begin{multicols}{2}
% \begin{enumeratea}
% \item $\longarray\left\{\begin{array}{l}
% \dfrac{4y+x}{5x}=1\\\dfrac{x+y}{2x-y}=2\end{array}\right.$
% \item $\longarray\left\{\begin{array}{l}
% y=\dfrac{4x-9}{12}\\
% {\dfrac{y+2}{y-1}+\dfrac{1+2x}{1-x}+1=0}\end{array}\right.$
% \item $\longarray\left\{\begin{array}{l}
% 2+3\dfrac{y}{x}=\dfrac{1}{x}\\
% 3\dfrac{x}{y}-1=\dfrac{-2}{y} \end{array}\right.$
% \item $\longarray\left\{\begin{array}{l}
% \dfrac{y}{2x-1}=-1\\2\dfrac{x}{y-1}=1\end{array}\right.$
% \end{enumeratea}
% \end{multicols}
% \end{esercizio}
% 
% \begin{esercizio}[\Ast]
%  \label{ese:22.45}
% Verifica l'insieme soluzione dei seguenti sistemi.
% \begin{multicols}{2}
% \begin{enumeratea}
% \item 
% $\longarray\left\{\begin{array}{l}3\dfrac{x}{y}-\dfrac{7}{y}=1\\2\dfrac{y}{x}
% +\dfrac{5}{x}=1\end{array}\right.$
% \item 
% $\longarray\left\{\begin{array}{l}2\dfrac{x}{3y}-\dfrac{1}{3y}=1\\\dfrac{3}{y+2x
% }=-1\end{array}\right.$
% \item 
% $\longarray\left\{\begin{array}{l}\dfrac{x}{9y}=-{\dfrac{1}{2}}+\dfrac{1}{3y}
% \\9\dfrac{y}{2x}-1-\dfrac{3}{x}=0\end{array}\right.$
% \item 
% $\longarray\left\{\begin{array}{l}\dfrac{x}{2-\dfrac{y}{2}-2}=1\\\dfrac{x-y}{
% x+\dfrac{3}{2}y-1}=1\end{array}\right.$
% \end{enumeratea}
% \end{multicols}
% \end{esercizio}
% 
% \begin{esercizio}[\Ast]
%  \label{ese:22.46}
% Verifica l'insieme soluzione dei seguenti sistemi.
% \begin{multicols}{2}
% \begin{enumeratea}
% \item 
% $\longarray\left\{\begin{array}{l}\dfrac{\dfrac{x}{2}+\dfrac{2y}{3}-\dfrac{1}{6}
% }{x+y-2}=6\\x+y=1\end{array}\right.$
% \item 
% $\longarray\left\{\begin{array}{l}\dfrac{x-2y}{4}=\dfrac{\dfrac{x-y}{2}+2x}{4}
% \\\dfrac{x}{\dfrac{y}{3}+1}=1\end{array}\right.$
% \item $\longarray\left\{\begin{array}{l}\dfrac{x+3y-1}{x-y}=\dfrac{1}{y-x} 
% \\x=2y-10 \end{array}\right.$
% \item 
% $\longarray\left\{\begin{array}{l}\dfrac{2}{x-2}-\dfrac{3}{y+3}=1\\\dfrac{5}{y+3
% }=\dfrac{6}{2-x}-4\end{array}\right.$
% \end{enumeratea}
% \end{multicols}
% \end{esercizio}
% 
% \begin{esercizio}[\Ast]
%  \label{ese:22.47}
% Verifica l'insieme soluzione dei seguenti sistemi.
% 
% \begin{enumeratea}
% \item 
% $\longarray\left\{\begin{array}{l}y-\dfrac{x}{3}+\dfrac{3}{4}=0\\\dfrac{2x+1}{
% 1-x}+\dfrac{2+y}{y-1}=-1\end{array}\right.$
% \item 
% $\longarray\left\{\begin{array}{l}x+y=2\\y\left(\dfrac{x}{y}+3\right)=4\end{
% array}\right.$
% \item 
% $\longarray\left\{\begin{array}{l}\dfrac{x}{3}-\dfrac{y}{2}=0\\\dfrac{y(y-x-1)}{
% y+1}+x-y+1=\dfrac{1}{2}\end{array}\right.$
% \item 
% $\longarray\left\{\begin{array}{l}\dfrac{3x-7y+1}{4x^{2}-9y^{2}}=\dfrac{4}{18y^{
% 2}-8x^{2}}\\\dfrac{4(1-3x)^{2}}{2}-y=\dfrac{(12x-5)(6x-y)}{4}+3xy+2\end{array}
% \right.$
% \end{enumeratea}
% \end{esercizio}
% 
% \begin{esercizio}[\Ast]
%  \label{ese:22.48}
% Verifica l'insieme soluzione dei seguenti sistemi.
% 
% \begin{enumeratea}
% \item 
% $\longarray\left\{\begin{array}{l}\dfrac{2x-3y}{x-2y}-\dfrac{3y-1}{x+5y}=\dfrac{
% 2(x^{2}+2xy)-(3y-2)^{2}}{x^{2}+3xy-10y^{2}}\\x+y=-19\end{array}\right.$
% \item 
% $\longarray\left\{\begin{array}{l}\dfrac{x-3}{x-3y+1}+\dfrac{xy-y}{x-3y-1}
% =\dfrac{x^{2}-3xy+x^{2}y-3xy^{2}+3y^{2}}{x^{2}+9y^{2}-6xy-1}\\\dfrac{x-3}{5y-1}
% -\dfrac{y-3}{1+5y}=\dfrac{x+5y^{2}-5xy+2}{1-25y^{2}}\end{array}\right.$
% \item 
% $\longarray\left\{\begin{array}{l}\dfrac{x-2y}{x^{2}-xy-2y^{2}}-\dfrac{1}{y}
% =2\\\dfrac{4}{y}-\dfrac{5}{x+y}=-9\end{array}\right.$
% \item 
% $\longarray\left\{\begin{array}{l}{2x-y-11=0}\\{\dfrac{y+1}{x-1}+\dfrac{3-y}{
% 5x-5}-\dfrac{2}{3}=0}\end{array}\right.$
% \end{enumeratea}
% \end{esercizio}
% 
% \begin{esercizio}
%  \label{ese:22.49}
% Verifica l'insieme soluzione dei seguenti sistemi.
% \begin{multicols}{2}
% \begin{enumeratea}
% \item 
% $\longarray\left\{\begin{array}{l}{\dfrac{x+1}{x}=\dfrac{y+2}{y-2}}\\{\dfrac{
% 3x-1}{3x-2}=\dfrac{1+y}{y-2}}\end{array}\right.$
% \item 
% $\longarray\left\{\begin{array}{l}{\dfrac{2}{5x-y}=\dfrac{-3}{5y-x}}\\{\dfrac{1}
% {4x-3y}=\dfrac{2x+y-1}{3y-4x}}\end{array}\right.$
% \item 
% $\longarray\left\{\begin{array}{l}{\dfrac{\sqrt{3}}{x-\sqrt{2}}+\dfrac{2\sqrt{2}
% }{y-\sqrt{3}}=0}\\{\dfrac{1}{x-\sqrt{3}}-\dfrac{\sqrt{6}}{2\left(y+2\sqrt{2}
% \right)}=0}\end{array}\right.$
% \item 
% $\longarray\left\{\begin{array}{l}{\dfrac{x-y+1}{x+y-1}=2}\\{\dfrac{x+y+1}{x-y-1
% }=-2}\end{array}\right.$
% \item 
% $\longarray\left\{\begin{array}{l}{\dfrac{2}{x-2}=\dfrac{3}{y-3}}\\{\dfrac{1}{
% y+3}=\dfrac{-1}{2-x}}\end{array}\right.$
% \end{enumeratea}
% \end{multicols}
% \end{esercizio}

%%%%%%%%%%%%%%

% %\subsubsection*{22.4 - Sistemi letterali}
% \subsubsection*{\numnameref{sec:22_letterali}}
% 
% \begin{esercizio}[\Ast]
%  \label{ese:22.50}
% Risolvere e discutere il seguente sistema. Per quali valori di~$a$ la coppia 
% soluzione è formata da
% numeri reali positivi?
% \[\left\{\begin{array}{l}{x+ay=2a}\\\dfrac{x}{2a}+y=\dfrac{3}{2}\end{array}
% \right.\]
% \end{esercizio}
% 
% 
% \begin{esercizio}
%  \label{ese:22.51}
% Perché se il seguente sistema è determinato la coppia soluzione è accettabile?
% \[\left\{\begin{array}{l}3x-2y=0\\\dfrac{2x-y}{x+1}=\dfrac{1}{a}\end{array}
% \right.\]
% \end{esercizio}
% 
% 
% \begin{esercizio}
%  \label{ese:22.52}
% Nel seguente sistema è vero che la coppia soluzione è formata da numeri reali 
% positivi se~$a>2$?
%  
% \[\left\{\begin{array}{l}\dfrac{a-x}{a^{2}}+a+\dfrac{y-2a}{a+1}=-1\\2y=x\end{
% array}\right.\]
% \end{esercizio}
% 
% 
% \begin{esercizio}
%  \label{ese:22.53}
% Spiegate perché non esiste alcun valore di~$a$ per cui la
% coppia~$(0;2)$ appartenga a~$\IS$ del sistema:
% \[\left\{\begin{array}{l}3x-2y=0\\\dfrac{2x-y}{x+1}=\dfrac{1}{a}\end{array}
% \right.\]
% \end{esercizio}
% 
% \begin{esercizio}[\Ast]
%  \label{ese:22.54}
% Nel seguente sistema determinate i valori da attribuire al
% parametro~$a$ affinché la coppia soluzione accettabile sia formata da
% numeri reali positivi.
% \[\left\{\begin{array}{l}\dfrac{y}{x}-\dfrac{y-a}{3}=\dfrac{1-y}{3}
% \\a(x+2)+y=1\end{array}\right.\]
% \end{esercizio}
% 
% \begin{esercizio}[\Ast]
%  \label{ese:22.55}
% Risolvere i seguenti sistemi.
% \begin{multicols}{2}
%  \begin{enumeratea}
%  {\longarray
% \item 
% $\left\{\begin{array}{l}x+ay=2a\\\dfrac{x}{2a}+y=\dfrac{3}{2}\end{array}\right.$
% \item 
% $\left\{\begin{array}{l}\dfrac{x^{3}-8}{x-2}=x^{2}-3x+y-2\\\dfrac{x^{2}-4xy+3y^{
% 2}}{3y-x}=k\end{array}\right.$
% \item $\left\{\begin{array}{l}kx-y=2\\x+6ky=0\end{array}\right.$
% \item $\left\{\begin{array}{l}kx-8y=4\\2x-4ky=3\end{array}\right.$}
%  \end{enumeratea}
% \end{multicols}
% \end{esercizio}
% 
% \begin{esercizio}[\Ast]
%  \label{ese:22.56}
% Risolvere i seguenti sistemi.
% \begin{multicols}{2}
%  \begin{enumeratea}
% \item $\left\{\begin{array}{l}4x-k^{2}y=k\\kx-4ky=-3k\end{array}\right.$
% \item $\left\{\begin{array}{l}kx-4ky=-6\\kx-k^{2}y=0\end{array}\right.$
% \item $\left\{\begin{array}{l}(k-1)x+(1-k)y=0\\(2-2k)x+y=-1\end{array}\right.$
%  \end{enumeratea}
% \end{multicols}
% \end{esercizio}
% Risposte
% \paragraph{\ref{ese:22.44}} a)~indeterminato, b)~$(3;3)$,\protect\\ 
% c)~$(-\frac{5}{11};\frac{7}{11}$, d)~impossibile.
% 
% \paragraph{\ref{ese:22.45}} a)~$\left(\frac{9}{5};-\frac{8}{5}\right)$, 
% b)~$(-1;-1)$, c)~impossibile, d)~$\left(-\frac{1}{5};\frac{2}{5}\right)$
% 
% \paragraph{\ref{ese:22.46}} a)~$(39;-38)$, 
% b)~$\left(\frac{3}{4};-\frac{3}{4}\right)$, c)~$(-6;2)$, d)~$(-2;-5)$
% 
% \paragraph{\ref{ese:22.47}} a)~$\left(-\frac{9}{8};-\frac{9}{8}\right)$, 
% b)~$(1;1)$, c)~impossibile, d)~$\left(-\frac{3}{17};\frac{6}{17}\right)$
% 
% \paragraph{\ref{ese:22.48}} a)~$(-18;-1)$, 
% b)~$\left(\frac{7}{4};\frac{1}{2}\right)$, c)~$(2;-1)$
% 
% \paragraph{\ref{ese:22.50}} $a>0$
% 
% \paragraph{\ref{ese:22.54}} $-\frac{1}{2}<a<\frac{1}{2}$
% 
% \paragraph{\ref{ese:22.55}} a)~$a\neq~0\rightarrow (a;1)$, b)~determinato 
% per~$k\neq~14$, $k\neq \frac{6}{7}$ con soluzioni~$\left(\frac{k-6}{4}; 
% \frac{5k-6}{4}\right)$
% se~$k=14\vee k=\frac{6}{7}$ impossibile, c)~determinato~$\forall k$ con 
% soluzioni~$\left(\frac{12k}{6k^{2}+1};\frac{2}{6k^{2}+1}\right)$,
% d)~determinato per~$k\neq -2$, $k\neq~2$ con 
% soluzioni~$\left(\frac{4k-6}{k^{2}-4}; \frac{8-3k}{4(k^{2}-4)}\right)$ se~$k=-2 
% \vee k=2$ impossibile.
% 
% \paragraph{\ref{ese:22.56}} a)~Determinato per~$k\neq -4$, $k\neq~4$, 
% $k\neq~0$ con soluzioni~$\left(\frac{3k^{2}+4k}{16-k^{2}}; 
% \frac{k+12}{16-k^{2}}\right)$
% \protect\\ se~$k=-4\vee k=4$ impossibile; se~$k=0$ indeterminato con soluzioni 
% tipo~$(0;t)$ con~$t\in\insR$,
% b)~determinato per~$k\neq~0, k\neq~4$con soluzioni~$\left(\frac{6}{4-k}; 
% \frac{6}{k(4-k)}\right)$
% se~$k=0\vee k=4$ impossibile, c)~determinato per~$k\neq~1, k\neq \frac{3}{2}$ 
% con soluzioni~$\left(\frac{1}{2k-3}; \frac{1}{2k-3}\right)$
% se~$k=\frac{3}{2}$ impossibile; se~$k=1$ indeterminato con soluzioni del 
% tipo~$(t;-1)$

\subsubsection*{22.5 - Sistemi lineari di tre equazioni in tre incognite}
\subsubsection*{\numnameref{sec:22_treequazioni}}

\begin{esercizio}[\Ast]
 \label{ese:22.57}
 Determinare la terna di soluzione dei seguenti sistemi.
\begin{multicols}{2}
\begin{enumeratea}
\item $\left\{\begin{array}{l}x-2y+z=1 \\x-y=2 \\x+3y-2z=0 \end{array}\right.$
 \hfill $\left[(0;~-2;~3)\right]$
\item $\left\{\begin{array}{l}x+y+z=4 \\x-3y+6z=1\\x-y-z=2 \end{array}\right.$
 \hfill $\left[\left(3;\frac{8}{9};~\frac{1}{9}\right)\right]$
\item 
$\left\{\begin{array}{l}x+2y-3z=6-3y\\2x-y+4z=x\\3x-z=y+2\end{array}\right.$
 \hfill $\left[(1;~1;~0)\right]$
\item $\left\{\begin{array}{l}2x-y+3z=1 \\x-2y+z=5\\x+2z=3 \end{array}\right.$
 \hfill $\left[(-21;~-7;~12)\right]$
\item $\left\{\begin{array}{l}x+2y-z=1 \\y-4z=0\\x-2y+z=2 \end{array}\right.$
 \hfill $\left[\left(\frac{3}{2};~-\frac{2}{7};~-\frac{1}{14}\right)\right]$
\item $\left\{\begin{array}{l}x-3y+6z=1 \\x+y+z=5 \\x+2z=3 \end{array}\right.$
 \hfill $\left[(-5;~6;~4)\right]$
\item $\left\{\begin{array}{l}x-4y+6z=2 \\x+4y-z=2\\x+3y-2z=2 
\end{array}\right.$
 \hfill $\left[(2;~0;~0)\right]$
\item $\left\{\begin{array}{l}4x-y-2z=1 \\3x+2y-z=4\\x+y+2z=4 
\end{array}\right.$
 \hfill $\left[(1;~1;~1)\right]$
\item $\left\{\begin{array}{l}x-3y=3 \\x+y+z=-1\\2x-z=0 \end{array}\right.$
 \hfill $\left[(0;~-1;~0)\right]$
\item $\left\{\begin{array}{l}2x-y+3z=1 \\x-6y+8z=2\\3x-4y+8z=2 
\end{array}\right.$
 \hfill $\left[\left(\frac{2}{3};~-\frac{2}{3};~-\frac{1}{3}\right)\right]$
\item $\left\{\begin{array}{l}4x-6y-7z=-1 \\x+y-z=1\\3x+2y+6z=1 
\end{array}\right.$
 \hfill $\left[\left(\frac{9}{31};~\frac{17}{31};~-\frac{5}{31}\right)\right]$
\item $\left\{\begin{array}{l}4x-3y+z=4\\x+4y-3z=2 \\y-7z=0 \end{array}\right.$
 \hfill $\left[\left(\frac{7}{6};~\frac{7}{30};~\frac{1}{30}\right)\right]$
\item $\left\{\begin{array}{l}3x-6y+2z=1 \\x-4y+6z=5\\x-y+4z=10 
\end{array}\right.$
 \hfill $\left[(5;~3;~2)\right]$
\item $\left\{\begin{array}{l}4x-y-7z=-12 \\x+3y+z=-4\\2x-y+6z=5 
\end{array}\right.$
 \hfill $\left[\left(-{\frac{60}{43}};~-\frac{53}{43};~\frac{47}{43}\right)
         \right]$
\item $\left\{\begin{array}{l}2x+y-5z=2 \\x+y-7z=-2\\x+y+2z=1 
\end{array}\right.$
 \hfill $\left[\left(\frac{10}{3};~-3;~\frac{1}{3}\right)\right]$
\item $\left\{\begin{array}{l}3x-y+z=-1\\x-y-z=3 \\x+y+2z=1 \end{array}\right.$
 \hfill $\left[(6;~11;~-8)\right]$
\item $\left\{\begin{array}{l}x-4y+2z=7 \\-3x-2y+3z=0 \\x-2y+z=1 
\end{array}\right.$
 \hfill $\left[\left(-5;~-\frac{33}{4};~-\frac{21}{2}\right)\right]$
\item $\left\{\begin{array}{l}-2x-2y+3z=4 \\2x-y+3z=0\\2x+y=1 
\end{array}\right.$
 \hfill $\left[\left(-{\frac{5}{2}};~6;~\frac{11}{3}\right)\right]$
\end{enumeratea}
\end{multicols}
\end{esercizio}

% \begin{esercizio}
%  \label{ese:22.60}
% Quale condizione deve soddisfare il parametro~$a$ affinché il sistema seguente 
% non sia privo di
% significato? Determina la terna soluzione assegnando ad~$a$ il valore~2.
%  \[\left\{\begin{array}{l}x+y+z=\frac{a^{2}+1}{a}\\ay-z=a^{2} 
% \\y+ax=a+1+a^{2}z\end{array}\right.\]
% \end{esercizio}
% 
% \begin{esercizio}
%  \label{ese:22.61}
% Determina il dominio del sistema e stabilisci se la terna soluzione è 
% accettabile:
% \[\longarray\left\{\begin{array}{l}\frac{5}{1-x}+\frac{3}{y+2}=\frac{2x}{
% xy-2+2x-y}\\
% \frac{x+1-3(y-1)}{\text{xyz}}=\frac{1}{xy}-\frac{2}{yz}-\frac{3}{xz}\\
% x+2y+z=0\end{array}\right.\]
% \end{esercizio}

\begin{esercizio}
 \label{ese:22.62}
Verifica se il sistema è indeterminato:
\[\left\{\begin{array}{l}x+y=1 \\y-z=5
\\x+z+2=0 \end{array}\right.\]
\end{esercizio}

\begin{esercizio}
 \label{ese:22.63}
Determina il volume del parallelepipedo retto avente
per base un rettangolo, sapendo che le dimensioni della base e
l'altezza hanno come misura (rispetto al~$\unit{cm}$) i valori
di~$x, y, z$ ottenuti risolvendo il sistema:
\[\left\{\begin{array}{l}3x+1=2y+3z \\6x+y+2z=7
\\9(x-1)+3y+4z=0 \end{array}\right.\]
\end{esercizio}

%%%%%%%%%%%%%%%%%%%%%%%%%%%%%%%%%%%%%%%%%%%%

% \subsubsection*{22.6 - Sistemi da risolvere con sostituzioni delle variabili}
% \subsubsection*{\numnameref{sec:22_sostituzione}}
% 
% \begin{esercizio}[\Ast]
%  \label{ese:22.64}
%  Risolvi i seguenti sistemi per mezzo di opportune sostituzioni delle variabili.
% 
% \begin{enumeratea}
% \item 
% $\longarray\left\{\begin{array}{l}\dfrac{1}{2x}+\dfrac{1}{y}=-4\\\dfrac{2}{3x}
% +\dfrac{2}{y}=1\end{array}\right.$ \quad 
% sostituire~$u=\frac{1}{x};v=\frac{1}{y}$
% \item $\left\{\begin{array}{l}x^{2}+y^{2}=13\\x^{2}-y^{2}=5 
% \end{array}\right.$\quad sostituire~$u=x^{2};v=y^{2}$
% \item 
% $\longarray\left\{\begin{array}{l}\dfrac{1}{x+y}+\dfrac{2}{x-y}=1\\\dfrac{3}{x+y
% }-\dfrac{5}{x-y}=2\end{array}\right.$\quad sostituire~$u=\frac{1}{x+y};v=\ldots$
% \end{enumeratea}
% \end{esercizio}
% 
% \begin{esercizio}[\Ast]
%  \label{ese:22.65}
%  Risolvi i seguenti sistemi per mezzo di opportune sostituzioni delle variabili.
% \begin{multicols}{2}
% \begin{enumeratea}
% {\longarray
% \item 
% $\left\{\begin{array}{l}\dfrac{5}{2x}-\dfrac{2}{y}=2\\\dfrac{1}{x}+\dfrac{2}{y}
% =1\end{array}\right.$
% \item 
% $\left\{\begin{array}{l}\dfrac{1}{x}+\dfrac{2}{y}=3\\\dfrac{1}{x}+\dfrac{3}{y}
% =4\end{array}\right.$
% \item 
% $\left\{\begin{array}{l}\dfrac{2}{x}+\dfrac{4}{y}=-3\\\dfrac{2}{x}-\dfrac{3}{y}
% =4 \end{array}\right.$
% \item 
% $\left\{\begin{array}{l}\dfrac{1}{x+1}-\dfrac{2}{y-1}=2\\\dfrac{2}{x+1}-\dfrac{1
% }{y-1}=3\end{array}\right.$}
% \end{enumeratea}
% \end{multicols}
% \end{esercizio}
% 
% Risposte
% 
% \begin{multicols}{2}
% \paragraph{\ref{ese:22.64}} a)~$\left(-{\frac{1}{27}};\frac{2}{19}\right)$, 
% b)~$(3;2)$, $(-3;2)$, $(3;-2)$, $(-3;-2)$,
% c)~$\left(\frac{55}{9};-\frac{44}{9}\right)$
% 
% \paragraph{\ref{ese:22.65}} a)~$\left(\frac{7}{6};14\right)$, 
% b)~$\left(1;1\right)$, c)~$\left(2;-1\right)$, 
% \protect\\d)~$\left(-{\frac{1}{4}};-2\right)$

% \begin{esercizio}[\Ast]
%  \label{ese:22.66}
%  Risolvi i seguenti sistemi per mezzo di opportune sostituzioni delle variabili.
% \begin{multicols}{2}
% \begin{enumeratea}
% % \item 
% $\longarray\left\{\begin{array}{l}\dfrac{1}{x}-\dfrac{3}{y}+\dfrac{2}{z}=3 
% \\\dfrac{2}{x}-\dfrac{3}{y}+\dfrac{2}{z}=4
% % \\\dfrac{2}{x}+\dfrac{4}{y}-\dfrac{1}{z}=-3\end{array}\right.$
% \item $\left\{\begin{array}{l}x^{3}+y^{3}=9 \\2x^{3}-y^{3}=-6 
% \end{array}\right.$
% \item $\left\{\begin{array}{l}x^{2}+y^{2}=-1\\x^{2}-3y^{2}=12\end{array}\right.$
% % \item 
% $\longarray\left\{\begin{array}{l}\dfrac{4}{x^{2}}-\dfrac{2}{y^{2}}-\dfrac{2}{z^
% {2}}=0\\\dfrac{1}{x^{2}}+\dfrac{1}{z^{2}}=2
% % \\\dfrac{2}{y^{2}}-\dfrac{2}{z^{2}}=0\end{array}\right.$
% \end{enumeratea}
% \end{multicols}
% \end{esercizio}
% 
% \paragraph{\ref{ese:22.66}} a)~$\left(1;-\frac{5}{8};-\frac{5}{7}\right)$, 
% b)~$(1;2)$, c)~$\emptyset $,\protect\\
% d)~$(1;1;1)$, $(-1;1;1)$, $(1;-1;1)$, $(1;1;-1)$,\protect\\ $(-1;-1;1)$ 
% $(-1;1;-1)$, $(1;-1;-1)$, \protect\\$(-1;-1;-1)$

%%%%%%%%%%%%%%%%%%%%%%%%%%%%%%%%%%%%%%%%%%%%%%%%%%%%%%%%%%%%

\subsection{Esercizi riepilogativi}

Gli esercizi indicati con (\croce) sono tratti da \emph{Matematica }~2, 
Dipartimento di
Matematica, ITIS V. Volterra, San Donà di Piave, Versione [11-12]~[S-A11], 
pg.~53; licenza CC, BY-NC-BD, per gentile concessione dei
professori che hanno reddatto il libro. Il libro è scaricabile da 
\url{
http://www.istitutovolterra.it/dipartimenti/matematica/dipmath/docs/M2_1112.pdf}


\begin{esercizio}[\Ast]
 \label{ese:22.67}
 Risolvi i seguenti sistemi con più metodi ed eventualmente controlla
la soluzione graficamente.
\begin{multicols}{2}
\begin{enumeratea}
\item $\left\{\begin{array}{l}
y-\dfrac{3x-4}{2}=1-\dfrac{y}{4}\\
2y-2x=-{\dfrac{4}{3}}\end{array}\right.$
 \hfill $\left[\left(\frac{2}{3};~0\right)\right]$
\item $\longarray\left\{\begin{array}{l}
\dfrac{2}{3}x-y+\dfrac{1}{3}=0\\
x-\dfrac{2}{3}y+\dfrac{1}{3}=0\end{array}\right.$
 \hfill $\left[\left(-{\frac{1}{5}};~\frac{1}{5}\right)\right]$
\item $\left\{\begin{array}{l}
x=\dfrac{y-4}{3}+1\\
y=\dfrac{x+3}{3}\end{array}\right.$
 \hfill $\left[(0;~1)\right]$
\item $\left\{\begin{array}{l}
x-y+k=0\\
x+y=k-1\end{array}\right.$
 \hfill $\left[...\right]$
\item $\left\{\begin{array}{l}2x+y=1 \\2x-y=-1\end{array}\right.$
 \hfill $\left[(0;~1)\right]$
\item $\left\{\begin{array}{l}2x=1+3y\\-y-2x=3\end{array}\right.$
 \hfill $\left[(-1;~-1)\right]$
\item $\left\{\begin{array}{l}-x+2y=1 \\3x-y=3\end{array}\right.$
 \hfill $\left[\left(\frac{7}{5};~\frac{6}{5}\right)\right]$
\item $\left\{\begin{array}{l}5x-y=2\\2x+3y=-1 \end{array}\right.$
 \hfill $\left[\left(\frac{5}{17};~-\frac{9}{17}\right)\right]$
\item $\left\{\begin{array}{l}x+2y=3 \\3x-y=2\end{array}\right.$
 \hfill $\left[(1;~1)\right]$
\item $\left\{\begin{array}{l}2x-y=1 \\x+2y=2\end{array}\right.$
 \hfill $\left[\left(\frac{4}{5};~\frac{3}{5}\right)\right]$
\item $\left\{\begin{array}{l}5x+3y=2 \\3x-2y=1\end{array}\right.$
 \hfill $\left[\left(\frac{7}{19};~\frac{1}{19}\right)\right]$
\item $\left\{\begin{array}{l}7x-2y=4\\8x-6y=9 \end{array}\right.$
 \hfill $\left[\left(\frac{3}{13};~-\frac{31}{26}\right)\right]$
\item $\left\{\begin{array}{l}3x-2y=4 \\2x+3y=5\end{array}\right.$
 \hfill $\left[\left(\frac{22}{13};~\frac{7}{13}\right)\right]$
\item $\left\{\begin{array}{l}3x-y=7 \\x-2y=5 \end{array}\right.$
 \hfill $\left[\left(\frac{9}{5};~-\frac{8}{5}\right)\right]$
\item $\left\{\begin{array}{l}3x-2y=2\\2y-2x=-{\frac{4}{3}} \end{array}\right.$
 \hfill $\left[(\frac{2}{3};~0)\right]$
\item $\left\{\begin{array}{l}5x-2x=7\\-x-2y=-{\frac{1}{2}} \end{array}\right.$
 \hfill $\left[\left(\frac{7}{3};~-\frac{11}{12}\right)\right]$
{\longarray
\item 
$\left\{\begin{array}{l}\dfrac{2}{3}x-2y=-{\dfrac{1}{6}}\\-y-\dfrac{2}{3}
y=\dfrac{3}{2} \end{array}\right.$
 \hfill $\left[\left(-{\frac{59}{20}};~-\frac{9}{10}\right)\right]$
\item $\left\{\begin{array}{l}\dfrac{1}{3}x-\dfrac{3}{2}y+1=0\\9y-2x-6=0 
\end{array}\right.$
 \hfill $\left[indeterminato\right]$
\item 
$\left\{\begin{array}{l}-{\dfrac{1}{3}}x+\dfrac{3}{2}y-1=0\\3x-\dfrac{1}{5}
y+\dfrac{3}{2}=0 \end{array}\right.$
 \hfill $\left[(-{\frac{123}{266}};~\frac{75}{133})\right]$
\item $\left\{\begin{array}{l}-{\dfrac{2}{3}}y+3x=y\\x-\dfrac{1}{2}y+3=0 
\end{array}\right.$
 \hfill $\left[(-30;~-54)\right]$}
\end{enumeratea}
\end{multicols}
\end{esercizio}

\begin{esercizio}[\Ast]
 \label{ese:22.71}
 Risolvi i seguenti sistemi con più metodi ed eventualmente controlla
la soluzione graficamente.
\begin{multicols}{2}
\begin{enumeratea}
% \item $\left\{\begin{array}{l}{2x+y=0}\\{x-2y=-5}\end{array}\right.$
%  \hfill $\left[(-1;~2)\right]$
% \item $\left\{\begin{array}{l}{x+y=-1}\\{x-y=5}\end{array}\right.$
%  \hfill $\left[(2;~-3)\right]$
\item $\left\{\begin{array}{l}{2x+2y=6}\\{x-2y=-3}\end{array}\right.$
 \hfill $\left[(1;~2)\right]$
\item $\left\{\begin{array}{l}{2x-y=3}\\{x-2y=0}\end{array}\right.$
 \hfill $\left[(2;~1)\right]$
\item $\longarray\left\{\begin{array}{l}5y+\dfrac{3}{2}x=-2\\3x+10y-3=0 
\end{array}\right.$
 \hfill $\left[impossibile\right]$
\item 
$\longarray\left\{\begin{array}{l}\dfrac{1}{2}x-3y=\dfrac{1}{2}\\3(y-2)+x=0 
\end{array}\right.$
 \hfill $\left[\left(\frac{13}{3};~\frac{5}{9}\right)\right]$
\item 
$\left\{\begin{array}{l}{\dfrac{1}{3}x+3y+2=0}\\{2x+\dfrac{1}{2}y=\dfrac{11}{2}}
\end{array}\right.$
 \hfill $\left[(3;~-1)\right]$
\item 
$\left\{\begin{array}{l}{\dfrac{1}{2}x+\dfrac{1}{2}y=1}\\{\dfrac{2}{3}x+\dfrac{1
}{3}y=1}\end{array}\right.$
 \hfill $\left[(1;~1)\right]$
 {\longarray
\item 
$\left\{\begin{array}{l}
{2x-y=0}\\
{4x+\dfrac{1}{2}y=\dfrac{5}{2}}\end{array}\right.$
 \hfill $\left[\left(\frac{1}{2};~1\right)\right]$
\item $\left\{\begin{array}{l}
{2x+\dfrac{1}{2}y=-\dfrac{3}{10}}\\
{-25x+5y=6}\end{array}\right.$
 \hfill $\left[\left(-{\frac{1}{5}};~\frac{1}{5}\right)\right]$
\item $\left\{\begin{array}{l}
{2x+y-3=0}\\
{4x+2y+6=0}\end{array}\right.$
 \hfill $\left[impossibile\right]$
% \item $\left\{\begin{array}{l}
% {2x-y=-1}\\
% {x+\dfrac{1}{2}y=-{\dfrac{1}{2}}}\end{array}\right.$
%  \hfill $\left[\left(-{\frac{1}{2}};~0\right)\right]$
\item $\left\{\begin{array}{l}
{\dfrac{1}{2}x-\dfrac{1}{3}y=1}\\
{3x-2y=3}\end{array}\right.$
 \hfill $\left[\emptyset\right]$
\item $\left\{\begin{array}{l}
{10x-5y=26}\\
{x+5y=-\dfrac{42}{5}}\end{array}\right.$
 \hfill $\left[\left(\frac{8}{5};~-2\right)\right]$
\item $\left\{\begin{array}{l}
\dfrac{1}{2}(x-3)-y=\dfrac{3}{2}(y-1)\\
\dfrac{3}{2}(y-2)+x=6\left(x+\dfrac{1}{3}\right)\end{array}\right.$
 \hfill $\left[(-\frac{50}{47};~-\frac{10}{47})\right]$
\item $\left\{\begin{array}{l}
\dfrac{x+4y}{6}-3=0\\
\dfrac{x}{2}-\dfrac{y}{4}=0\end{array}\right.$
 \hfill $\left[(2;~4)\right]$}
\end{enumeratea}
\end{multicols}
\end{esercizio}

\begin{esercizio}[\Ast]
 \label{ese:22.75}
 Risolvi i seguenti sistemi con più metodi ed eventualmente controlla
la soluzione graficamente.
\begin{enumeratea}
 {\longarray
 \item $\longarray\left\{\begin{array}{l}
 \dfrac{4}{3}x-\dfrac{4y-x}{2}+\dfrac{35}{12}-\dfrac{x+y}{4}=0\\
 \dfrac{3(x+y)}{2}-\dfrac{1}{2}(5x-y)=\dfrac{1}{3}(11-4x+y)\end{array}\right.$
 \hfill $\left[(1;~2)\right]$
\item $\left\{\begin{array}{l}
\dfrac{1}{2}y-\dfrac{1}{6}x=5-\dfrac{6x+10}{8}\\
8(x-2)+3x=40-6\left(y-\dfrac{1}{6}\right)\end{array}\right.$
 \hfill $\left[(3;~4)\right]$
\item $\left\{\begin{array}{l}
3(x-4)=-{\dfrac{4y}{5}}\\
7(x+y)+8\left(x-\dfrac{3y}{8}
-2\right)=0\end{array}\right.$
 \hfill $\left[\emptyset\right]$
\item $\left\{\begin{array}{l}
\dfrac{2}{5}(y-x-1)=\dfrac{y-x}{3}-\dfrac{2}{5}\\
(x-y)^{2}-x(x-2y)=x+y(y-1)\end{array}\right.$
 \hfill $\left[...\right]$
\item $\left\{\begin{array}{l}
2x-3(x-y)=-1+3y\\
\dfrac{1}{2}x+\dfrac{1}{3}y=-{\dfrac{1}{6}}\end{array}\right.$
 \hfill $\left[(1;~-2)\right]$
\item $\left\{\begin{array}{l}
(y+2)(y-3)-(y-2)^{2}+(x+1)^{2}=(x+3)(x-3)-\dfrac{1}{2}\\
\left(y-\dfrac{1}{2}\right)\left(y+\dfrac{1}{4}\right)-(y-1)^{2}+2x+3=
    \dfrac{3}{4}\end{array}\right.$
 \hfill $\left[(-1;~\frac{1}{2})\right]$
\item $\left\{\begin{array}{l}
x^{2}+\dfrac{y}{4}-3x=\dfrac{(2x+1)^{2}}{4}-\dfrac{y}{2}\\
(y-1)^{2}=-8x+y^{2}\end{array}\right.$
 \hfill $\left[(\frac{1}{8};~1)\right]$
\item $\left\{\begin{array}{l}
\dfrac{\dfrac{x}{2}-y+5}{\dfrac{4}{3}-\dfrac{5}{6}}=
    x-\dfrac{\dfrac{x}{2}-\dfrac{y}{3}}{2}\\
-x-\dfrac{\dfrac{y}{3}-x}{2}=1\end{array}\right.$
 \hfill $\left[(-{\frac{92}{27}};~\frac{38}{9})\right]$
% \item $\left\{\begin{array}{l}
% \dfrac{\dfrac{x+1}{2}-y}{2}=y-20x\\
% x-\dfrac{y}{4}=\dfrac{x-y}{6}\end{array}\right.$
%  \hfill $\left[\left(-{\frac{1}{21}};~-{\frac{10}{21}}\right)\right]$
% \item $\left\{\begin{array}{l}
% \dfrac{4y-\dfrac{5}{2}x+\dfrac{3}{2}}{\dfrac{5}{6}}=x-2y\\
% x=3y\end{array}\right.$
%  \hfill $\left[\left(\frac{27}{26};~\frac{9}{26}\right)\right]$
}
\end{enumeratea}
\end{esercizio}

\newpage
%%%%%%%%%%%%%%%%%%%%%%%%%%%%%%%%%%%%%%%%%%%%%%%%%%%%%%%%%%%%%%%%%%%%%%%%%%
\begin{multicols}{2}
\begin{esercizio}
 \label{ese:22.77}
Determina due numeri sapendo che la loro somma è~37, la loro
differenza è~5.
\end{esercizio}

\begin{esercizio}[]
 \label{ese:22.81}
Determina due numeri la cui somma è~57 e di cui si sa che il doppio del più 
grande diminuito della metà del più piccolo è~49.
\hfill $\left[(26;~31)\right]$
\end{esercizio}

\begin{esercizio}[\Ast]
 \label{ese:22.79}
Determina tre numeri la cui somma è~81. Il secondo supera il primo
di~3. Il terzo numero è dato dalla somma dei primi due. 
\hfill $\left[18,75;~21,75;~40,5\right]$
\end{esercizio}

\begin{esercizio}[\Ast]
 \label{ese:22.78}
Il doppio della somma di due numeri è uguale al secondo numero aumentato
del triplo del primo, inoltre aumentando il primo numero di~12 si
ottiene il doppio del secondo diminuito di~6. 
\hfill $\left[(18;~18)\right]$
\end{esercizio}

\begin{esercizio}[\Ast]
 \label{ese:22.80}
Determina due numeri sapendo che la loro somma è pari al doppio del minore 
aumentato di~1/4
del maggiore, mentre la loro differenza è uguale a~$9$
\hfill $\left[(27;~36)\right]$
\end{esercizio}

\begin{esercizio}[\Ast]
 \label{ese:22.82}
Determina tre lati sapendo che il triplo del primo lato è
uguale al doppio del secondo aumentato di~$10\unit{m}$ la differenza tra il
doppio del terzo lato e il doppio del secondo lato è uguale al primo
lato aumentato di~12; la somma dei primi due lati è uguale al terzo
lato.
\hfill $\left[(12\unit{m};~13\unit{m};~25\unit{m})\right]$
\end{esercizio}

\begin{esercizio}[\Ast]
 \label{ese:22.83}
Determina un numero di due cifre sapendo che la cifra delle decine
è il doppio di quella delle unità e scambiando le due cifre si
ottiene un numero più piccolo di~27 del precedente.
\hfill $\left[63\right]$
\end{esercizio}

\begin{esercizio}[\Ast]
 \label{ese:22.84}
Determina il numero intero di due cifre di cui la cifra delle decine 
supera di~2 la cifra delle unità e la somma delle cifre è~12.
\hfill $\left[75\right]$
\end{esercizio}

\begin{esercizio}[\croce]
 \label{ese:22.85}
Determina due numeri naturali il cui quoziente è~5 e la cui differenza è~12.
\hfill $\left[84\right]$
\end{esercizio}

\begin{esercizio}[\Ast, \croce]
 \label{ese:22.86}
Determinare un numero naturale di due cifre sapendo che la loro somma
è~12 e che, invertendole, si ottiene un numero che supera di~6 la
metà di quello iniziale.
\hfill $\left[\right]$
\end{esercizio}

\begin{esercizio}[\croce]
 \label{ese:22.87}
Determinare la frazione che diventa uguale a~5/6 aumentando i suoi
termini di~2 e diventa~1/2 se i suoi termini diminuiscono di~2.
\hfill $\left[...\right]$
\end{esercizio}

\begin{esercizio}[\Ast, \croce]
 \label{ese:22.88}
La somma delle età di due coniugi è~65 anni; un settimo
dell'età del marito è uguale ad un sesto
dell'età della moglie. Determinare le età dei
coniugi.
\hfill $\left[(35;~30)\right]$
\end{esercizio}

\begin{esercizio}[\Ast, \croce]
 \label{ese:22.89}
Un numero naturale diviso per~3 dà un certo quoziente e resto~1. Un
altro numero naturale, diviso per~5, dà lo stesso quoziente e resto~3.
Sapendo che i due numeri hanno per somma~188, determinali e calcola
il quoziente.
\hfill $\left[(70;~118;~23)\right]$
\end{esercizio}

\begin{esercizio}[\Ast]
 \label{ese:22.90}
Giulio e Giulia hanno svuotato i loro
salvadanai per comparsi una bici. Nel negozio c'è
una bella bici che piace a entrambi, costa{\officialeuro}~180 e nessuno dei
due ha i soldi sufficienti per comprarla. Giulio dice:
<<Se mi dai la metà dei tuoi soldi compro io la
bici>>. Giulia ribatte: <<se mi dai la terza
parte dei tuoi soldi la bici la compro io>>. Quanti soldi
hanno rispettivamente Giulio e Giulia?
\hfill $\left[(108;~144)\right]$
\end{esercizio}

\begin{esercizio}
 \label{ese:22.91}
A una recita scolastica per beneficenza vengono incassati
{\officialeuro}~216 per un totale di~102 biglietti venduti. I ragazzi della
scuola pagano {\officialeuro}~1, i ragazzi che non sono di quella scuola
pagano {\officialeuro}~1,5, gli adulti pagano {\officialeuro}~3. Quanti sono i
ragazzi della scuola che hanno assistito alla recita?
\end{esercizio}


\begin{esercizio}
 \label{ese:22.92}
Da un cartone quadrato di lato~$12\unit{cm}$, si taglia prima una striscia
parallela a un lato e di spessore non noto, poi si taglia dal lato
adiacente una striscia parallela al lato spessa~$2\unit{cm}$ in più rispetto
alla striscia precedente. Sapendo che il perimetro del rettangolo
rimasto è~$33,6\unit{cm}$, calcola l'area del rettangolo rimasto.
\hfill $\left[...\right]$
\end{esercizio}

\begin{esercizio}[\Ast]
 \label{ese:22.93}
Al bar per pagare~4 caffè e~2 cornetti si spendono {\officialeuro}~$4,60$, per 
pagare~6 caffè
e~3 cornetti si spendono {\officialeuro}~$6,90$ È possibile
determinare il prezzo del caffè e quello del cornetto?
\hfill $\left[indeterminato\right]$
\end{esercizio}

\begin{esercizio}[\Ast]
 \label{ese:22.94}
 Al bar Mario offre la colazione agli amici perché è il suo
compleanno: per~4 caffè e~2 cornetti paga {\officialeuro}4,60. Subito dopo
arrivano tre altri amici che prendono un caffè e un cornetto
ciascuno, questa volta paga {\officialeuro}4,80. Quanto costa un caffè e
quanto un cornetto?
\hfill $\left[\officialeuro 0,7 \text{ e } \officialeuro 0,9\right]$
\end{esercizio}

\begin{esercizio}[\Ast]
 \label{ese:22.95}
Un cicloturista percorre~$218\unit{km}$ in tre giorni. Il secondo giorno 
percorre
il~$20\%$ in più del primo giorno. Il terzo giorno percorre~$14\unit{km}$ in
più del secondo giorno. Qual è stata la lunghezza delle tre tappe?
\hfill $\left[60\unit{km};~72\unit{km};~86\unit{km}\right]$
\end{esercizio}

\begin{esercizio}[\Ast]
 \label{ese:22.96}
In un parcheggio ci sono moto e auto. In tutto si contano~43 mezzi e~140
ruote. Quante sono le auto e quante le moto?
\hfill $\left[27; 16\right]$
\end{esercizio}

\begin{esercizio}
 \label{ese:22.97}
Luisa e Marisa sono due sorella. Marisa, la più grande è nata~3 anni
prima della sorella; la somma delle loro età è~59. Qual è
l'età delle due sorelle?
\hfill $\left[...\right]$
\end{esercizio}

\begin{esercizio}
 \label{ese:22.98}
Mario e Lucia hanno messo da parte del denaro. Lucia ha {\officialeuro}~5
in più di Mario. Complessivamente potrebbero comprare~45 euro di
schede prepagate per i cellulari. Quanto possiede Mario e quanto
possiede Lucia?
\hfill $\left[...\right]$
\end{esercizio}

\begin{esercizio}
 \label{ese:22.99}
Una macchina per giaccio produce~10 cubetti di giaccio al minuto, mentre
una seconda macchina per giacchio produce~7 cubetti al minuto. Sapendo
che in tutto sono stati prodotti~304 cubetti e che complessivamente le
macchine hanno lavorato per~22 minuti, quanti cubetti ha prodotto la
prima macchina e quindi ne ha prodotti la seconda.
\hfill $\left[...\right]$
\end{esercizio}

\begin{esercizio}[\Ast]
 \label{ese:22.100}
In un parcheggio ci sono automobili, camion e moto, in tutto~62 mezzi.
Le auto hanno~4 ruote, i camion ne hanno~6 e le moto ne hanno~2.
In totale le ruote sono~264. Il numero delle ruote delle auto è uguale
al numero delle ruote dei camion. Determina quante auto, quanti camion
e quante moto ci sono nel parcheggio.
\hfill $\left[30~auto;~20~camion;~12~moto\right]$
\end{esercizio}

\begin{esercizio}
 \label{ese:22.101}
 Un vasetto di marmellata pesa~$780\unit{g}$ . Quando nel vasetto rimane
metà marmellata, il vasetto pesa~$420\unit{g}$ . Quanto pesa il vasetto vuoto?
\hfill $\left[...\right]$
\end{esercizio}


\begin{esercizio}[\Ast]
 \label{ese:22.102}
Una gelateria prepara per la giornata di Ferragosto~$30\unit{kg}$ di
gelato. Vende i coni da due palline a {\officialeuro}~1,50 e i coni da tre
palline a {\officialeuro}~2,00. Si sa che da~$2\unit{kg}$ di gelato si fanno~25
palline di gelato. A fine giornata ha venduto tutto il gelato e ha
incassato {\officialeuro}~272,50. Quanti coni da due palline ha venduto?
\hfill $\left[135\right]$
\end{esercizio}

\begin{esercizio}[Prove Invalsi~2004-2005]
 \label{ese:22.103}
Marco e Luca sono fratelli. La somma delle loro età è~23
anni. Il doppio dell'età di Luca è uguale alla
differenza tra l'età del loro padre e il triplo
dell'età di Marco. Quando Luca è nato, il padre
aveva~43 anni. Determina l'età di Marco e di Luca.
\hfill $\left[...\right]$
\end{esercizio}


\begin{esercizio}[Giochi d'autunno~2010, Centro Pristem]
 \label{ese:22.104}
Oggi Angelo ha un quarto dell'età di sua
madre. Quando avrà~18 anni, sua madre avrà il triplo della sua
età. Quanti anni hanno attualmente i due?
\hfill $\left[...\right]$
\end{esercizio}


\begin{esercizio}[Giochi di Archimede, 2008]
 \label{ese:22.105}
 Pietro e Paolo festeggiano il loro onomastico in pizzeria con i
loro amici. Alla fine della cena il conto viene diviso in parti uguali
tra tutti i presenti e ciascuno dovrebbe pagare~12 euro. Con grande
generosità però gli amici decidono di offrire la cena a Pietro e
Paolo; il conto viene nuovamente diviso in parti uguali tra gli amici
di Pietro e Paolo (cioè tutti i presenti esclusi Pietro e Paolo), e
ciascuno di loro paga~16 euro. Quanti sono gli amici di Pietro e Paolo?
\hfill $\left[...\right]$
\end{esercizio}


\begin{esercizio}[\Ast]
 \label{ese:22.106}
 Al bar degli studenti, caffè e cornetto costano
{\officialeuro}~1,50; cornetto e succo di frutta costano {\officialeuro}~1,80,
caffè e succo di frutta costano {\officialeuro}~1,70. Quanto costano in
tutto~7 caffè,~5 cornetti e~3 succhi di frutta?
\hfill $\left[{\officialeuro}~11,90\right]$
\end{esercizio}

% \begin{esercizio}[\croce]
%  \label{ese:22.107}
% Un negozio ha venduto scatole contenenti~6 fazzoletti ciascuna
% ed altre contenenti~12 fazzoletti ciascuna, per un totale di~156
% fazzoletti. Il numero delle confezioni da~12 ha superato di~1 la metà
% di quello delle confezioni da~6. Quante confezioni di ogni tipo si sono
% vendute?
% \hfill $\left[...\right]$
% \end{esercizio}
% 
% 
% \begin{esercizio}[\Ast, \croce]
%  \label{ese:22.108}
% Nella città di Nonfumo gli unici negozi sono tabaccherie e
% latterie. L'anno scorso le tabaccherie erano i~2/3
% delle latterie; quest'anno due tabaccherie sono
% diventate latterie cosicché ora le tabaccherie sono i~9/16 delle
% latterie. Dall'anno scorso a
% quest'anno il numero complessivo dei negozi di Non
% fumo è rimasto lo stesso. Quante latterie c'erano
% l'anno scorso a Nonfumo?
% \hfill $\left[30\right]$
% \end{esercizio}
% 
% \begin{esercizio}
%  \label{ese:22.109}
% Un rettangolo di perimetro~$80\unit{cm}$ ha la base che è i~2/3
% dell'altezza. Calcolare l'area del rettangolo.
% \hfill $\left[...\right]$
% \end{esercizio}
% 
% \begin{esercizio}[\Ast]
%  \label{ese:22.110}
%  Un trapezio isoscele ha il perimetro di~$72\unit{cm}$ . La base minore è
% i~3/4 della base maggiore; il lato obliquo è pari alla somma dei~2/3
% della base minore con i~3/2 della base maggiore. Determina le misure
% delle basi del trapezio.
% \hfill $\left[\frac{288}{23}\unit{cm};\frac{216}{23}\unit{cm}\right]$
% \end{esercizio}
% 
% \begin{esercizio}[\Ast]
%  \label{ese:22.111}
%  Calcola l'area di un rombo le cui diagonali
% sono nel rapporto~3/2. Si sa che la differenza tra le due diagonali 
% è~$16\unit{cm}$
% \hfill $\left[1536\unit{cm}^{2}\right]$
% \end{esercizio}
% 
% \begin{esercizio}
%  \label{ese:22.112}
% In un triangolo rettangolo i~3/4 dell'angolo
% acuto maggiore sono pari ai~24/13 dell'angolo acuto
% minore. Determinare l'ampiezza degli angoli.
% \hfill $\left[...\right]$
% \end{esercizio}
% 
% \begin{esercizio}[\Ast]
%  \label{ese:22.113}
%  In un triangolo, un angolo supera di~$16\grado$ un secondo
% angolo; il terzo angolo è pari ai~29/16 della somma dei primi due.
% Determina le misure degli angoli del triangolo.
% \hfill $\left[24\grado;~40\grado;~116\grado\right]$
% \end{esercizio}
% 
% \begin{esercizio}
%  \label{ese:22.114}
% In un rettangolo di
% perimetro~$120\unit{cm}$, la base è~2/3 dell'altezza. Calcola
% l'area del rettangolo.
% \hfill $\left[...\right]$
% \end{esercizio}
% 
% \begin{esercizio}
%  \label{ese:22.115}
% Determina le misure dei tre lati~$x$, $y$, $z$ di un triangolo sapendo che il 
% perimetro è~$53\unit{cm}$. 
% Inoltre la misura~$z$ differisce di~$19\unit{cm}$ dalla somma delle
% altre due misure e che la misura~$x$ differisce di~$11\unit{cm}$ dalla 
% differenza tra~$y$ e~$z$.
% \hfill $\left[...\right]$
% \end{esercizio}
% 
% \begin{esercizio}[\Ast]
%  \label{ese:22.116}
% Aumentando la base di un rettangolo di~$5\unit{cm}$ e
% l'altezza di~$12\unit{cm}$, si ottiene un rettangolo di
% perimetro~$120\unit{cm}$ che è più lungo di~$12\unit{cm}$ del perimetro del
% rettangolo iniziale.
% \hfill $\left[impossibile\right]$
% \end{esercizio}
% 
% \begin{esercizio}[\Ast]
%  \label{ese:22.117}
% In un triangolo isoscele di perimetro~$64\unit{cm}$, la differenza tra la base e 
% la metà del lato obliquo è~$4\unit{cm}$. 
% Determina la misura della base e del lato obliquo del triangolo.
% \hfill $\left[16\unit{cm};~24\unit{cm}\right]$
% \end{esercizio}
% 
% \begin{esercizio}[\Ast]
%  \label{ese:22.118}
% Un segmento~$AB$ di~$23\unit{cm}$ viene diviso da un suo punto~$P$ in due parti 
% tali che il triplo della loro differenza è uguale al segmento minore 
% aumentato di~$20\unit{cm}$.
% Determina le misure dei due segmenti in cui resta diviso~$AB$ dal punto~$P$.
% \hfill $\left[7\unit{cm};~16\unit{cm}\right]$
% \end{esercizio}
\end{multicols}
