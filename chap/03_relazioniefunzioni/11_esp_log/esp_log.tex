% (c) 2015 Daniele Zambelli daniele.zambelli@gmail.com

% (c) 2014 Daniele Zambelli - daniele.zambelli@gmail.com
% 
% Tutti i grafici per il capitolo relativo alle parabole
%
% 

\newcommand{\espdueterzi}{% 
    % Esponenziali con basi diverse.
    \disegno{
    \rcom{-10}{+10}{-1}{10}{gray!50, very thin, step=1}
    \begin{scope}[ultra thick, color=Maroon!50!black]
     \tkzInit[xmin=-10.3, xmax=+10.3, ymin=-0.3, ymax=+10.3]
     \tkzFct[domain=-10.3:+6]{(3./2)**x}
     \tkzFct[color=Green!50!black, domain=-6:+10.3]{(2./3)**x}
     \begin{scope}[color=Black!50!black]
      \filldraw (1, 3./2) circle (1.2pt);
      \filldraw (1, 2./3) circle (1.2pt);
     \end{scope}
     \filldraw [color=Red](0, 1) circle (1.2pt);  
    \end{scope}
    \begin{scope}[color=black]
     \draw (-7.3, 7) node{\(f(x)=\tonda{\dfrac{2}{3}}^x\)}; 
     \draw ((7.3, 7) node{\(f(x)=\tonda{\dfrac{3}{2}}^x\)};
    \end{scope}
    }
}

\newcommand{\logduebasi}{% 
    % Esponenziali con basi diverse.
    \disegno{
    \rcom{-1}{+10}{-9}{9}{gray!50, very thin, step=1}
    \begin{scope}[ultra thick, color=Maroon!50!black]
      \tkzInit[xmin=-1.3, xmax=+80, xstep=.5, ymin=-10.3,ymax=+10.3]
      \tkzFct[domain=.01:+10]{log(x)/log(2)}
      \filldraw (2, 1) circle (1.2pt);
      \begin{scope}[color=Green!50!black]
        \tkzFct[domain=-.01:+10]{log(x)/log(1./2)}
        \filldraw (2, -1) circle (1.2pt);
      \end{scope}
    \end{scope}
    \begin{scope}[color=black]
      \draw (9.5, 2.8) node{a=2}; 
      \draw (9.5, -2.8) node{a=0.5};
    \end{scope}
      \filldraw [color=Red] (1,0) circle (1.2pt);
    }
}


\chapter[Esponenziali e logaritmi]{Funzioni, equazioni e disequazioni 
esponenziali e logaritmiche}

\section{TODO}

\section{Un problema}
\label{sec:esplog_problemi}

\emph{
Di ritorno da una viaggio nella foresta pluviale, mi sono portato come ricordo 
una piccola piantina che vive sulla superficie dell'acqua. 
Il primo giorno di giugno, getto la piantina nello stagno vicino a casa che ha 
una superficie di circa \(1km^2\).
Nei giorni seguenti vado a controllare lo stagno e non vedo più la pianta: il 
vento l'avrà spinta chissà dove! 
Questa specie ha la particolarità di duplicarsi ogni giorno e gli animali che 
vivono dalle nostre parti non la conoscono e non se ne cibano.
Parto per un altro viaggio, torno il trenta giugno e scopro che la piantina 
ha ricoperto tutto lo stagno.}

\begin{enumerate}
 \item
\emph{Se in 30 giorni la pianta ha ricoperto tutto lo stagno, in quanti giorni 
ne aveva ricoperto la metà?}
 \item 
\emph{Quanta superficie era ricoperta il 20 giugno?}
 \item 
\emph{Quanta superficie era ricoperta il 10 giugno?}
\end{enumerate}

\begin{enumerate}
 \item
Per risolvere questo problema non dobbiamo guardare il periodo complessivo 
ragionando in avanti. Con questa impostazione verrebbe spontaneo pensare che se 
in 30 giorni copre tutto lo stagno, metà stagno sarà coperto in 15 giorni. Non 
è così.
Noi non abbiamo informazioni su quanto la pianta ricopre lo stagno il primo 
giorno, ma su quanto stagno è ricoperto il trentesimo giorno.
Quindi dobbiamo partire dalla fine: se il primo luglio la superficie dello 
stagno è tutta coperta e, se ogni giorno la pianta raddoppia, il giorno prima 
sarà ricoperto solo la metà dello stagno quindi per ricoprire metà dello stagno 
ha impiegato~29 giorni e per ricoprire l'altra metà impiega solo un giorno.

 \item
Per calcolare quale superficie è ricoperta dalla pianta il ventesimo giorno, 
possiamo costruire una tabella, sempre partendo dalla fine:

\begin{center}
\begin{tabular}{c|c|c|c|c|c|c|c|c}
giorno & 30 & 29 & 28 & 27 & 26 & 25 & \dots & 20\\[6pt]
\hline &&&&&&&&\\ [-6pt]
superficie & 1 & \(\dfrac{1}{2}=\dfrac{1}{2^1}\) & 
 \(\dfrac{1}{4}=\dfrac{1}{2^2}\) & 
 \(\dfrac{1}{8}=\dfrac{1}{2^3}\) & 
 \(\dfrac{1}{16}=\dfrac{1}{2^4}\) & 
 \(\dfrac{1}{32}=\dfrac{1}{2^5}\) & \dots & \dots\\
\end{tabular}
\end{center}

Completando la tabella puoi scoprire quale parte dello stagno è ricoperta il 
ventesimo giorno. Ma si può ritrovare questo risultato senza calcolarsi tutti i 
risultati intermedi?
La frazione che contiene la potenza di due ci dà un suggerimento, dobbiamo 
riuscire a scriverla partendo dal valore della prima riga:

\begin{center}
\begin{tabular}{cccc}
giorno & 30 & \dots & \(n\) \\
superficie & 1 & \dots & \(\dfrac{1}{2^{\dots}}\) \\
\end{tabular}
\end{center}

 \item
Con questa formula a disposizione si può calcolare immediatamente quanta 
superficie dello stagno è ricoperta il decimo giorno:
\[superficie = \dfrac{1}{2^{\dots}} = \dfrac{1}{2^{20}} \approx 
\text{ 1 milionesimo della superficie}\]
Ciò significa che nei primi 10 giorni arriva a coprire meno di un milionesimo 
della superficie dello stagno e poi in un giorno da metà lago lo copre tutto.
 \item
Se di questa pianta ne deriva una forma mutante che continua a crescere allo 
stesso modo anche al di fuori dell'acqua, quanto tempo impiegherà a ricoprire 
tutta la terra?
\end{enumerate}

\section{Funzioni, equazioni e disequazioni esponenziali}
\label{sec:esplog_esponenziali}

Il fenomeno riportato nel problema precedente, trova un proprio modello nelle 
funzioni esponenziali. Studiamo ora come si comportano queste funzioni.

\subsection{La successione delle potenze di~2}
\label{subsec:esplog_succpotdue}

Riprendiamo l'esempio precedente, ma poniamo il giorno zero quello in cui viene 
ricoperto l'intero stagno: al valore zero corrisponde la superficie uno, al 
valore uno la superficie \dots. E nei giorni precedenti: al valore meno uno 
corrisponde la superficie un mezzo, al valore meno due la superficie \dots.

Completando la tabella e riportando i valori nel grafico si ottiene:

\begin{figure}[h]
 \centering
 \begin{minipage}[]{.48\textwidth}
 \vspace*{1cm}
  \begin{center}
   \begin{tabular}{r|l}
    $n$   & $y=2^n$ \\
    \hline
    \dots & \dots \\
    $-5$ & $2^{-5} = 0,03125$ \\
    $-4$ & $2^{-4} = 0,0625$ \\
    $-3$ & $2^{-3} = 0,125$ \\
    $-2$ & $2^{-2} = 0,25$ \\
    $-1$ & $2^{-1} = 0,5$ \\
    $0$ & $2^{0} = 1$ \\
    $+1$ & $2^{+1} = 2$ \\
    $+2$ & $2^{+2} = 4$ \\
    $+3$ & $2^{+3} = 8$ \\
    $+4$ & $2^{+4} = 16$ \\
    $+5$ & $2^{+5} = 32$ \\
    \dots & \dots \\
   \end{tabular}
 \vspace*{1.8cm}
  \caption{Alcuni valori delle potenze di~2...} \label{tab:potdue0}
  \end{center}
 \end{minipage}
\begin{minipage}[]{.48\textwidth}
\begin{center}
\begin{inaccessibleblock}[I punti della tabella precedente riportati nel piano 
cartesiano si dispongono lungo una curva che 
a sinistra si trovano appena sopra all'asse x, 
attraversano l'asse y nel punto~1, poi crescono molto rapidamente.]
  \puntia
  \caption{...e i corrispondenti punti.} \label{fig:potdue0}
\end{inaccessibleblock}
\end{center}
\end{minipage}
\end{figure}

\begin{osservazione}
Possiamo fare alcune osservazioni su questa successione.

\begin{enumerate}
 \item 
  È sempre crescente cioè se \(n_1 > n_0 \text{ allora } 2^{n_1} > 2^{n_0}\)
 \item 
  Verso sinistra i valori di \(2^n\) diventano sempre più piccoli, ma rimangono 
  sempre maggiori di zero. Perché? Quindi se \(n\) è un infinito negativo 
  allora \(2^n\) sarà \dots
 \item 
  Verso destra cresce molto rapidamente. Quindi se \(n\) è un infinito positivo 
  allora \(2^n\) sarà \dots
 \item 
  Studiamo l'incremento della funzione completando la seguente tabella:
  
\begin{center}
\begin{tabular}{cccccccccccc}
n & -4    & -3    & -2   & -1  & 0 & +1 & +2 & +3    & +4    & \dots & n\\
\hline \\
% \vspace*{4pt}
y & 0,0625 & 0,125 & 0,25 & 0,5 & 1 & 2  & 4  & \dots & \dots & \dots & \dots\\
\hline \\
% \vspace*{4pt}
\(y_{n+1} - y_n\) & 
    \dots & \dots & \dots & \dots & \dots & \dots  & \dots  & \dots & \dots & 
\dots & \dots\\
\end{tabular}
\end{center}

  Possiamo osservare che l'incremento della successione in un punto: 
  \(y_{n+1} - y_n\) 
  è uguale al valore della successione in quel punto: 
  \(y_n\) 

\end{enumerate}
\end{osservazione}

\subsection{Le potenze di~2 con esponente reale}
\label{subsec:esplog_potdue}

Abbiamo visto quanto vale \(2^0\) e \(2^1\), ma se l'esponente è in mezzo 
tra zero e uno, cosa succede? Ingrandiamo la scala del grafico disegnato sopra 
e andiamo a studiare il comportamento della funzione reale: \(y=2^x\) quando 
\(x\) assume valori non interi:

\begin{figure}[h]
 \centering
 \begin{minipage}[]{.48\textwidth}
%  \vspace*{1cm}
  \begin{center}
   \begin{tabular}{r|l}
    $n$   & $y=2^n$ \\
    \hline
    \dots & \dots \\
    $-2$ & $2^{-2} = 0,25$ \\
    $-1,5$ & $2^{-1,5} = 2^{\frac{-3}{2}} = \sqrt{2^{-3}} = 0,353553391$ \\
    $-1$ & $2^{-1} = 0,5$ \\
    $-0,5$ & $2^{-0,5} = 2^{\frac{-1}{2}} = \sqrt{2^{-1}} = 0,707106781$ \\
    $0$ & $2^{0} = 1$ \\
    $+0,5$ & $2^{0,5} = 2^{\frac{1}{2}} = \sqrt{2} = 1,414213562$ \\
    $+1$ & $2^{+1} = 2$ \\
    $+1,25$ & $2^{1,25} = 2^{\frac{5}{4}} = \sqrt[4]{2^{5}} = 2,37841423$ \\
    $+1,5$ & $2^{1,5} = 2^{\frac{3}{2}} = \sqrt{2^{3}} = 2,828427125$ \\
    $+1,75$ & $2^{1,75} = 2^{\frac{7}{4}} = \sqrt[4]{2^{7}} = 3,363585661$ \\
    $+2$ & $2^{+2} = 4$ \\
    \dots & \dots \\
   \end{tabular}
%  \vspace*{1.8cm}
  \caption{Altri valori delle potenze di~2...} \label{tab:potdue1}
  \end{center}
 \end{minipage}
\begin{minipage}[]{.48\textwidth}
\begin{center}
\begin{inaccessibleblock}[Ingrandimento del grafico precedente con alcuni 
punti interpolati.]
  \vspace*{.8cm}
  \puntib
  \vspace*{.65cm}
  \caption{...e i corrispondenti punti.} \label{fig:potdue1}
\end{inaccessibleblock}
\end{center}
\end{minipage}
\end{figure}

Disegnando punti sempre più fitti si può pensare di disegnare i punti 
corrispondenti ad ogni valore \(x \in \R\) ed ottenere così il grafico della 
funzione reale \(y=2^x\).

\begin{definizione}[Funzioni esponenziali]{
Le funzioni esponenziali
sono quelle funzioni nelle quali la variabile indipendente appare all'esponente.
}
\end{definizione}

Molti fenomeni hanno, per certi periodi, un andamento che può essere 
modellizzato da una funzione esponenziale, sono i fenomeni dove la crescita è 
proporzionale al valore in un dato momento. Alcuni esempi:

\begin{figure}[h]
 \centering
 \begin{minipage}[]{.48\textwidth}
%  \vspace*{1cm}
\begin{itemize}
 \item 
Il capitale che cresce con un certo tasso di interesse e che quindi ha una 
crescita proporzionale al valore del capitale stesso.
 \item 
La crescita di una popolazione, in condizioni favorevoli: più individui fertili 
ha una popolazione più rapidamente cresce. 
Si può pensare in particolare alla diffusione di specie non autoctone in 
territori dove non trovano predatori.
 \item 
Un caso analogo al precedente la crescita del numero di batteri o virus in un 
individuo contagiato da una malattia.
 \item 
La diffusione di un'epidemia.
 \item 
L'aumento di temperatura dovuto all'aumento di ``gas serra'' che porta allo 
scioglimento delle calotte polari con la diffusione di ulteriori quantità di 
``gas serra''.
\end{itemize}

\end{minipage}
\begin{minipage}[]{.48\textwidth}
\begin{center}
\begin{inaccessibleblock}[Grafico della funzione esponenziale che 
a sinistra si trova appena sopra all'asse x, 
attraversa l'asse y nel punto~1, poi cresce molto rapidamente.]
  \graficoesponenziale
  \caption{Grafico della funzione \(y=2^x\).} \label{fig:funx2^x}
\end{inaccessibleblock}
\end{center}
\end{minipage}
\end{figure}

\subsection{Le funzioni esponenziali}
\label{subsec:esplog_fesponenziale}

È ora il momento di generalizzare la funzione. 
Studiamo come si comporta la funzione \(y=a^x\) per diversi valori di \(a\).
Cosa succede se la base della potenza è un numero diverso da~2? 
Come cambierà il suo grafico?

\paragraph{Base negativa}
\label{par:esplog_basenegativa}

Iniziamo mettendo al posto di \(a\) un valore negativo.

Proviamo a disegnare il grafico della funzione: \(y=(-2)^x\).
Diamo a \(x\) diversi valori, calcoliamo i corrispondenti valori di \(y\) e 
riportiamoli su un grafico:

\begin{figure}[h]
 \centering
 \begin{minipage}[]{.48\textwidth}
 \vspace*{.6cm}
  \begin{center}
   \begin{tabular}{r|l}
    $x$   & $y=(-2)x$ \\
    \hline
    \dots & \dots \\
    $-5$ & $(-2)^{-5} = -0,03125$ \\
    $-4$ & $(-2)^{-4} = +0,0625$ \\
    $-3$ & $(-2)^{-3} = -0,125$ \\
    $-2$ & $(-2)^{-2} = +0,25$ \\
    $-1$ & $(-2)^{-1} = -0,5$ \\
    $0$ & $(-2)^{0} = +1$ \\
    $+1$ & $(-2)^{+1} = -2$ \\
    $+2$ & $(-2)^{+2} = +4$ \\
    $+3$ & $(-2)^{+3} = -8$ \\
    $+4$ & $(-2)^{+4} = +16$ \\
    $+5$ & $(-2)^{+5} = -32$ \\
    \dots & \dots \\
   \end{tabular}
 \vspace*{.6cm}
  \caption{Alcuni valori delle potenze di~\((-2)\)...} \label{tab:potmenodue0}
  \end{center}
 \end{minipage}
\begin{minipage}[]{.48\textwidth}
\begin{center}
\begin{inaccessibleblock}[I punti della tabella precedente riportati nel piano 
cartesiano si dispongono alcuni nel semipiano positivo e alcuni nel semipiano 
negativo.]
  \puntimenodue
  \caption{...e i corrispondenti punti.} \label{fig:potmenodue0}
\end{inaccessibleblock}
\end{center}
\end{minipage}
\end{figure}

La successione risulta un po' strana, ma è accettabile. Cosa succede però se 
l'esponente è un numero con la virgola? Proviamo a far calcolare alla 
calcolatrice le seguenti espressioni:

\[\tonda{-2}^{1,5}= \dots \qquad \tonda{-2}^{2/7}= \dots \qquad 
 \tonda{-2}^{-1,5}= \dots \qquad \tonda{-2}^{3/8}= \dots \qquad 
\]

Molto probabilmente la calcolatrice si rifiuterà di calcolare queste 
espressioni perché il loro risultato non è un numero reale. 
Quanto visto per \(-2\) vale per qualunque base negativa.
Possiamo concludere che se la base è negativa possiamo calcolare la successione 
delle sue potenze, ma non possiamo calcolare i valori della funzione 
con~\(x \in \R\). 
La funzione \(y=a^x \text{ con } a<0\) non è definita nei numeri reali 
(in \(\R\)).
Anche quando la base è uguale a zero si ottiene una funzione piuttosto strana, 
come sarà il suo grafico?

D'ora in poi studieremo solo funzioni esponenziali con la base positiva.

\paragraph{Diverse basi}
\label{par:esplog_diversebasi}

\begin{description}
 \item [\(a>1\)]
Tenendo presente il valore delle potenze che abbiamo imparato, possiamo 
affermare che maggiore è la base e più ripida diventa il grafico della funzione 
sulla destra (per valori positivi) mentre a sinistra (per valori negativi) il 
grafico si appiattisce più rapidamente sull'asse \(x\).
 \item [\(a=1\)]
% Se la base è uguale a~1 la qualunque potenza sarà uguale a~1 quindi la funzione 
% diventa una costante:~\(y=1\). 
Se la base è proprio~1 l'equazione esponenziale 
diventa molto particolare: prova tu a calcolare diversi valori di \(y=1^x\) 
facendo variare \(x\), che valori di \(y\) ottieni?
 \item [\(0<a<1\)]
Calcolando alcuni valori di una funzione con la base minore di~1, 
ad esempio \(y=\tonda{\frac{1}{2}}^x\), possiamo osservare facilmente che il 
comportamento della funzione assomiglia molto a quello della funzione
\(y=2^x\), ma i valori ottenuti sono simmetrici rispetto all'asse \(y\). 
La funzione è decrescente, i valori verso sinistra crescono rapidamente, mentre 
verso destra si appiattiscono sull'asse \(x\) rimanendo comunque sempre 
positivi.
\(\tonda{\frac{1}{2}}^x = 2^{-x}\)
In generale possiamo affermare che:

\[\tonda{\frac{1}{a}}^x = a^{-x}\]
\end{description}

Se la base si avvicina a~1, rimanendo maggiore di~1, la funzione a destra si 
appiattisce crescendo più lentamente e a sinistra si avvicina più 
lentamente all'asse \(x\). 
Se la base si avvicina a~0 o all'infinito, la funzione si avvicina agli assi.

\begin{figure}[h]
\begin{minipage}{.69\textwidth}
 \begin{inaccessibleblock}[Grafici di funzioni esponenziali con basi diverse.]
  \diversebasi
  \caption{Esponenziali con diverse basi.} \label{fig:diversebasi}
\end{inaccessibleblock}
\end{minipage}
\hspace{12pt}
\begin{minipage}{.20\textwidth}
 \begin{enumerate} [label=\alph*]
   \item :~$y=...$
   \item :~$y=...$
   \item :~$y=...$
   \item :~$y=...$
   \item :~$y=...$
   \item :~$y=...$
   \item :~$y=...$
   \item :~$y=...$
 \end{enumerate}
\end{minipage}
\end{figure}

\begin{enumerate*}
 \item La funzione esponenziale è sempre maggiore di zero qualunque sia la base 
e l'esponente.
 \item La concavità è sempre verso l'alto.
 \item Se la base $a>1$ la funzione è strettamente crescente.
 \item Se la base $0<a<1$ la funzione è strettamente decrescente.
 \item Se $a=1$ la funzione è costante.
 \item Se il valore assoluto di $a$ è grande, il grafico si avvicina agli assi.
 \item Se il valore assoluto di $a$ è vicino a uno, il grafico si avvicina alla 
retta~\(y=1\).
%  \item Nel grafico precedente ho evidenziato dei punti che dovrebbero aiutare 
% a scoprire la base della funzione esponenziale.
 \item In figura \ref{fig:diversebasi} ho evidenziato dei punti che dovrebbero 
aiutare a scoprire la base della funzione esponenziale.
\end{enumerate*}

\subsection{Le equazioni esponenziali}
\label{subsec:esplog_equazioniesponenziali}

\begin{definizione}[Equazioni esponenziali]{
Le equazioni esponenziali
sono quelle equazioni nelle quali l'incognita appare all'esponente.
}
\end{definizione}

Vediamo alcuni tipi di equazioni esponenziali che si possono risolvere 
abbastanza facilmente.

\subsubsection{Equazioni elementari}
\label{subsubsec:esplog_eq_elementari}

Sono le equazioni nella forma:

\[a^x=b \text{ con } a>0\]

Per quanto abbiamo detto precedentemente, \(b\) dovrà essere un numero non 
negativo perché nessuna potenza con base negativa può avere un valore negativo.

Per risolvere equazioni di questo tipo dovremo utilizzare una delle 
operazioni inverse della potenza: si chiama \emph{logaritmo} l'operazione che 
dà l'esponente conoscendo la base e il valore della potenza. 
In pratica la soluzione dell'equazione precedente è:

\[x = \log_a b\]

\begin{esempio}
\(5^x=10 \Rightarrow x=\log_5 10 = 1,4306765580733933\)
infatti se con la calcolatrice calcolate \(5^{1,4306765580733933}\) 
otterrete proprio~10.
\end{esempio}

Spesso gli esercizi proposti si possono risolvere con un semplice trucco che 
permette di evitare il calcolo del logaritmo. 
Il metodo nasce dall'osservazione che:

\begin{osservazione}
 Se due potenze sono uguali e hanno anche la stessa base, dovranno per 
forza avere anche lo stesso esponente: 

\(\text{Se } a^x = a^y \text{ allora } x=y\)
\end{osservazione}

\begin{esempio}
\(4^x=64\) Dato che: \(64=4^3\) possiamo scrivere: \(4^x=4^3\) 
e quindi: \(x=3\).
\end{esempio}

\begin{esempio}
\(3^{2x+3}=81\) Anche in questo caso possiamo riportarci al meccanismo 
utilizzato precedentemente: \(3^{2x+3}=3^4\) e, se tre a qualcosa è equivalente 
a tre alla qualcos'altro, qualcosa e qualcos'altro devono essere equivalenti: 
\({2x+3}=4\). In questo modo abbiamo trasformato un'equazione esponenziale in 
una semplice equazione polinomiale.
\end{esempio}

\subsubsection{Applicazione delle proprietà delle potenze}
\label{subsubsec:esplog_eq_proprpot}

A volte per riportarci al caso elementare dobbiamo applicare le proprietà delle 
potenze che sarà il caso di ripassare:

\begin{description} [nosep]% [nosepitem]
 \item [\(a^m \cdot a^n = a^{m+n}\)]
 Il prodotto di due  potenze che hanno la stessa base è una potenza che ha per 
base la stessa base e per esponente la somma degli esponenti.
 \item [\(a^m \div a^n = a^{m-n}\)]
 Il quoziente di due  potenze che hanno la stessa base è una potenza che ha per 
base la stessa base e per esponente la differenza degli esponenti.
 \item [\(\tonda{a^m}^n = a^{m \cdot n}\)]
 La potenza di una potenza è una potenza che ha per base la stessa base e per 
esponente il prodotto degli esponenti.
 \item [\(a^n \cdot b^n = \tonda{a \cdot b}^n\)]
 Il prodotto di due  potenze che hanno lo stesso esponente è una potenza che ha 
per base il prodotto delle basi e per esponente lo stesso esponente.
 \item [\(a^n \div b^n = \tonda{a \div b}^n\)]
 Il quoziente di due  potenze che hanno lo stesso esponente è una potenza che 
ha per base il quoziente delle basi e per esponente lo stesso esponente.
\end{description}

\begin{esempio}
\(7^{x^2} \div 7^5 - \frac{49}{7^{6x}} = 0\) Usando le proprietà delle potenze 
ci riportiamo ad una situazione nota. 
% \footnote{
% Un rapido richiamo alle cinque proprietà delle potenze:
% % \begin{multicols}{3}
% \begin{enumerate} [nosep]% [nosepitem]
%  \item \(a^m \cdot a^n = a^{m+n}\)
%  \item \(a^m \div a^n = a^{m-n}\)
%  \item \(\tonda{a^m}^n = a^{m \cdot n}\)
%  \item \(a^n \cdot b^n = \tonda{a \cdot b}^n\)
%  \item  \(a^n \div b^n = \tonda{a \div b}^n\)
% \end{enumerate}
% % \end{multicols}
% }
L'equazione precedente è equivalente a: 
\(7^{x^2-5} - 7^{2-6x} = 0\) che può essere riscritta come:
\(7^{x^2-5} = 7^{2-6x}\) e usando i metodo precedenti viene 
trasformata in un'equazione polinomiale facilmente risolvibile:
\(x^2-5 = 2-6x \Rightarrow \dots \Rightarrow \graffa{x_1=-7;\quad x_2=1}\)
\end{esempio}

\begin{esempio}
\({5}^{3x} \div 5^2-{2}^{9x-6}=0\)
Usando le proprietà delle potenze possiamo scrivere la divisione come una 
unica potenza:
\({5}^{3x-2}-{2}^{9x-6}=0\)

Ma a questo punto abbiamo due potenze con basi diverse, i trucchi visti sopra 
non possiamo usarli! Dobbiamo escogitare qualcos'altro \dots possiamo 
raccogliere~3 nell'esponente della seconda potenza, i questo modo otteniamo due 
potenze con basi diverse, ma con gli esponenti che si assomigliano:
\({5}^{3x-2}-{2}^{3\tonda{3x-2}}=0\)

Sfruttando la terza proprietà delle potenze possiamo scrivere l'equazione nel 
seguente modo:
\({5}^{3x-2}-{8}^{3x-2}=0\)

Ora con un po' di magia, sposto dall'altra parte dell'uguale una potenza:
\({5}^{3x-2}={8}^{3x-2}\)

e divido entrambi i membri per l'espressione che si trova a secondo membro 
(posso farlo? perché?):
\(\tonda{\frac{5}{8}}^{3x-2}=1\)

Ma ogni potenza con esponente zero è uguale a uno quindi al posto di~1 posso 
scrivere \(\tonda{\frac{5}{8}}^{0}\) e, a questo punto, il gioco è fatto:
\(\tonda{\frac{5}{8}}^{3x-2}=\tonda{\frac{5}{8}}^{0} \Rightarrow 3x-2=0 \dots\) 
\end{esempio}

\subsubsection{Sostituzione di variabile}
\label{subsubsec:esplog_sostituzione}

A volte un'opportuna sostituzione di variabile rende l'equazione più semplice. 
Si può effettuare la sostituzione, risolvere l'equazione più semplice ottenuta, 
poi ripristinare la variabile originale nelle soluzioni trovate. 

Qualche esempio può aiutare a capire.

\begin{esempio} 
\(5^{2x} +6 \cdot 5^x -7=0\) 

\textbf{Sostituzione}:
ponendo \(5^x=z\) l'equazione diventa: \(z^2 +6z -7=0\) 
che risulta una equazione di secondo grado, facile da risolvere:

\(\tonda{z+7}\tonda{z-1}=0 \Rightarrow z_1=-7; \quad z_2=+1\) 

\textbf{Risostituzione}: al posto di \(z\) mettiamo le soluzioni trovate:

\(5^x=1 \Rightarrow 5^x=5^0 \Rightarrow x=0\)

\(5^x=-7\) Non ha soluzioni reali.
\end{esempio}

\begin{esempio} 
\(2^{\frac{4}{3}x} +4 \cdot 2^{\frac{2}{3}x} -32=0\) 

\textbf{Sostituzione}:
ponendo \(2^{\frac{2}{3}x}=z\) l'equazione diventa: \(z^2 +4z -32=0\) 
che risulta una equazione di secondo grado, facile da risolvere:

\(\tonda{z+8}\tonda{z-4}=0 \Rightarrow z_1=-8; \quad z_2=+4\) 

\textbf{Risostituzione}: al posto di \(z\) mettiamo le soluzioni trovate:

\(2^{\frac{2}{3}x}=-8\) Non ha soluzioni reali.

\(2^{\frac{2}{3}x}=4 \Rightarrow 2^{\frac{2}{3}x}=2^2
\Rightarrow \dfrac{2}{3}x=2 \Rightarrow x=3\)
\end{esempio}

\subsection{Le disequazioni esponenziali}
\label{subsubsec:esplog_disequazioniesponenziali}

L'altro problema che ci resta da risolvere è la soluzione di disequazioni in 
cui la variabile indipendente si trova ad esponente. Innanzitutto recuperiamo 
il metodo seguito fin'ora per risolvere le disequazioni:

\begin{enumerate} [noitemsep]
 \item Studio del segno:
 \begin{enumerate} [noitemsep]
  \item Equazione Associata;
  \item Funzione Associata;
 \end{enumerate}
 \item Individuazione dell'intervallo soluzione:
 \begin{enumerate} [noitemsep]
  \item modo Grafico;
  \item con i Predicati;
  \item con le Parentesi;
 \end{enumerate} 
\end{enumerate}

Proviamo a vedere in un caso semplice se possiamo seguire lo stesso meccanismo.

\begin{esempio}
 \(\tonda{\dfrac{1}{2}}^x \geqslant \dfrac{1}{64}\)
 
\begin{enumerate} [noitemsep]
 \item Studio del segno:
 \begin{enumerate} [noitemsep]
  \item Equazione Associata: 
  \(\tonda{\dfrac{1}{2}}^x = \dfrac{1}{64} \Rightarrow\)
   \(\tonda{\dfrac{1}{2}}^x = \tonda{\dfrac{1}{2}}^6 \Rightarrow x= 6\);
  \item Funzione Associata: \(y=\tonda{\dfrac{1}{2}}^x -\dfrac{1}{64}\)
 \begin{inaccessibleblock}[Funzione esponenziale decrescente che taglia 
l'asse~\(x\) nel punto~6 con un segno più prima di questo valore e meno dopo.]
  \grafdiseq{.8}{6}{-4}
 \end{inaccessibleblock}
 prima di \(x=6\) la funzione è positiva, dopo questo valore è negativa e non 
può ridiventare positiva dato che continua a calare.
 \end{enumerate}
 \item Individuazione dell'intervallo soluzione:
 Dato che la disequazione richiede che la funzione sia maggiore di zero, i 
valori che la rendono tale sono quelli positivi.
 \begin{enumerate} [noitemsep]
  \item modo Grafico;
 \begin{inaccessibleblock}[un asse delle~\(x\) con evidenziati i punti che nel 
grafico precedente erano segnati come positivi.]
  \dissolincl{-6}{6}
 \end{inaccessibleblock}
  \item con i Predicati: \(x \leqslant 6\);
  \item con le Parentesi \(\left]\infty;~6\right]\);
 \end{enumerate} 
\end{enumerate}

\end{esempio}

Prima di procedere facciamo un'osservazione importante: 

Se una funzione è crescente, da \(f(a)>f(b)\) consegue che \(a>b\).

Se una funzione è decrescente, da \(f(a)>f(b)\) consegue che \(a<b\).

E viceversa. Applicando questa osservazione alle funzioni esponenziali possiamo 
concludere che:

\begin{itemize}
 \item Se \(base^a>base^b \wedge base>1\) consegue che \(a>b\).
 \item Se \(base^a<base^b \wedge base>1\) consegue che \(a<b\).
 \item Se \(base^a>base^b \wedge base<1\) consegue che \(a<b\).
 \item Se \(base^a<base^b \wedge base<1\) consegue che \(a>b\).
\end{itemize}

Vediamo subito come utilizzare questa osservazione.


\begin{esempio}
 \(\tonda{\dfrac{1}{2}}^x \geqslant \dfrac{1}{64}\) (non è un errore di stampa, 
è proprio la stessa disequazione dell'esempio precedente).
 
  \(\tonda{\dfrac{1}{2}}^x \geqslant \dfrac{1}{64} \Rightarrow
    \tonda{\dfrac{1}{2}}^x \geqslant \tonda{\dfrac{1}{2}}^6 \Rightarrow 
    x \leqslant 6\)
    
Se mi ricordo di cambiar verso al predicato, quando necessario, questo metodo 
risulta decisamente più rapido.
\end{esempio}


\section{Funzioni, equazioni e disequazioni logaritmiche}
\label{sec:esplog_logaritmiche}
















