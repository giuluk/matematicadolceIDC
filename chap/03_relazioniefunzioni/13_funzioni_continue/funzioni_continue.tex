% (c) 2015 Daniele Zambelli daniele.zambelli@gmail.com

\chapter{Funzioni continue}

\section{TODO}
\begin{comment}
 
Schema del capitolo
===================

Limiti
------

Continuità
----------

  C. in un punto
  ''''''''''''''
  
    Definizione
    ...........
  
      Definizioni equivalenti
      
    Punti di discontinuità e di non derivabilità
    ............................................
 
  C. in un intervallo
  '''''''''''''''''''
  
    Definizione
    ...........
  
      Teorema funzioni elementari
      Teorema composizione di funzioni
      
    Insieme e sottoinsiemi delle funzioni
    .....................................
    
Massimi e minimi
----------------

  Definizione
  '''''''''''
  
    Teorema del punto critico
    
  Proprietà delle f. continue
  '''''''''''''''''''''''''''
  
    I numeri iperinteri
    ...................
    
    Teorema degli zeri
    
    Teorema dei valori estremanti
    
    Teorema di Rolle
    
    Teorema di Lagrange
    
    Corollario

\end{comment}


\section{Limiti}
\label{sec:cont_limiti}

In alcuni problemi non siamo interessati a sapere come si comporta una funzione 
per un valore ben preciso, dove magari non è definita, ci interessa di più 
sapere come si comporta quando si \emph{avvicina} a quel valore.

Per trattare queste situazioni i matematici si sono inventati il concetto di 
\emph{limite}.

TODO problema sensato sui limiti.

\begin{definizione}
\(l\) è il \textbf{limite} di una funzione \(f(x)\) per \(x\) che tende a un 
valore \(c\), se, quando \(x\) è infinitamente vicino a \(c\), 
ma diverso da \(c\), 
allora \(f(x)\) è infinitamente vicino a \(l\). E si scrive:

\[l=\lim_{x \rightarrow c} f(x) \Leftrightarrow 
\forall x \tonda{\tonda{x \approx c \wedge x \neq c} \Rightarrow 
\tonda{f(x) \approx l}}\]

\end{definizione}


\section{Continuità}
\label{sec:cont_continuita}

\subsection{Definizione di continuità in un punto}
\label{subsec:cont_definizione}


\begin{definizione}
Diremo che una funzione è continua in un punto \(c\), se è definita in \(c\) e, 
quando \(x\) è infinitamente vicino a \(c\), 
allora \(f(x)\) è infinitamente vicino a \(f(c)\). E si scrive::

\[f \text{ è continua in } c \Leftrightarrow 
\forall x \tonda{\tonda{x \approx c} \Rightarrow 
\tonda{f(x) \approx f(c)}}\]

\end{definizione}

\begin{esempio}
 Data la funzione \(f(x)=x^2-3x\) dimostrare che \(f(x)\) è continua in~4.
 
 La funzione è continua in~4 se per ogni \(x\) infinitamente vicino a~4 
 \(f(x)\) è infinitamente vicino a \(f(4)\). Cioè se \(x -4=\epsilon\) allora
 \(f(x) -f(4) = \delta\) dove \(\epsilon\) e \(\delta\) sono due infinitesimi.
 
\emph{dimostrazione}

Da \(x-4=\epsilon\) si ricava che \(x=4+\epsilon\), quindi: 
\[f(x) -f(4) = f(4+\epsilon) -f(4) = 
\tonda{4+\epsilon}^2-3\tonda{4+\epsilon}-\tonda{4^2-3\cdot 4}=\]
\[=\cancel{16}+8\epsilon+\epsilon^2\cancel{-12} 
  -3\epsilon\cancel{-16}\cancel{+12} = 
  +8\epsilon+\epsilon^2 -3\epsilon = 
\epsilon \tonda{5 + \epsilon}\]

Ora, il prodotto tra un infinitesimo e un finito è un infinitesimo, quindi, se 
la distanza tra \(x\) e \(4\) è infinitesima, anche la distanza tra 
\(f(x)\) e \(f(4)\) è infinitesima. qed 
 
\end{esempio}

\begin{esempio}
 Dimostrare che \(f(x)=\frac{\abs{x}}{x}\) non è continua in~0.
 
\emph{dimostrazione}
TODO
 
\end{esempio}

\begin{esempio}
 Studiare la continuità di una funzione definita a tratti nel punto di 
giunzione.
 
TODO
 
\end{esempio}

\subsection{Definizione di continuità in un intervallo}
\label{subsec:cont_definizione}

Dimostrare che una funzione è continua in un punto è piuttosto laborioso, pur 
non essendo complicato, ma presenta un problema quando sono interessato a 
studiare la continuità di una funzione in un intervallo. Infatti in un 
intervallo, anche piccolo, i punti sono infiniti e dimostrare la continuità per 
ognuno di essi risulta piuttosto lungo\dots.

Per superare questo scoglio, i matematici hanno pensato un approccio diverso:

\begin{itemize}
 \item dimostrare che alcune funzioni elementari sono continue;
 \item dimostrare che la somma, il prodotto, il quoziente e la composizione di 
funzioni continue è ancora una funzione continua.
\end{itemize}

In questo modo si può riconoscere la continuità di un gran numero di funzioni 
senza fare noiosi calcoli. Di seguito vediamo qualcuno di questi teoremi.

\subsubsection{Funzioni elementari}
\label{subsubsec:cont_funzionielementari}

Dimostriamo la continuità di alcune funzioni elementari.

\begin{teorema}[Continuità delle costanti]
Le funzioni costanti sono continue.
\end{teorema}

\noindent Ipotesi: \(f(x)=k\).\tab Tesi: \(f(x)\) è continua.

\begin{proof}
Per la definizione di continuità vogliamo dimostrare che 
\[\forall x \text{ se } x_0 \approx x \text{ allora } f(x_0) \approx f(x)\]
Poniamo \(x_0=x+\epsilon\), ma, essendo la funzione costante, anche 
\[f(x+\epsilon)=k\] 
che, ovviamente, è infinitamente vicino a \(k\). In simboli:
\[f(x+\epsilon) = k \approx k = f(x)\] 
\end{proof}

\begin{teorema}[Continuità della funzione identica]
La funzione identica (\(y=x\)) è continua.
\end{teorema}

\noindent Ipotesi: \(f(x)=x\).\tab Tesi: \(f(x)\) è continua.

\begin{proof}
Per la definizione di continuità vogliamo dimostrare che 
\[\forall x \text{ se } x_0 \approx x \text{ allora } f(x_0) \approx f(x)\]
Poniamo \(x_0=x+\epsilon\), \(f(x+\epsilon)=x+\epsilon\). 
Dato che la differenza:
\[f(x+\epsilon)-f(x) = x+\epsilon-x= \epsilon\]
è un infinitesimo, allora i due valori sono infinitamente vicini. In simboli:
\[f(x+\epsilon) = x+\epsilon \approx x = f(x)\] 
\end{proof}

\begin{teorema}[Continuità della funzione esponenziale]
La funzione esponenziale (\(y=a^x\)) è continua.
\end{teorema}

\noindent Ipotesi: \(f(x)=a^x\).\tab Tesi: \(f(x)\) è continua.

\begin{proof}
Usando la prima proprietà delle potenze:
\[f(x+\epsilon) =
a^{x+\epsilon} = a^{x} \cdot a^{\epsilon} \approx a^{x} \cdot 1 = a^{x}
f(x)\]
\end{proof}

Oltre alle funzioni precedenti, anche altre funzioni elementari sono continue, 
il seguente elenco riporta le principali funzioni continue:

\begin{itemize} [noitemsep]
 \item \(y=k\)
 \item \(y=x\)
 \item \(y=\frac{1}{x}\) \quad \textasteriskcentered
 \item \(y=\sqrt[n]{x}\) \quad \textasteriskcentered
 \item \(y=\abs{x}\)
 \item \(y=a^x\)
 \item \(y=\log_a x\) \quad \textasteriskcentered
 \item \(y=\sen x\)
 \item \(y=\cos x\)
 \item \(y=\tg x\) \quad \textasteriskcentered
\end{itemize}

\begin{osservazione}
Le funzioni segnate da \textasteriskcentered sono continue non su tutto \(\R\), 
ma solo \textbf{all'interno del loro campo di esistenza}.
\end{osservazione}

\subsubsection{Composizione di funzioni}
\label{subsubsec:cont_composizionefunzioni}

Vediamo ora che anche componendo in alcuni modi funzioni continue otteniamo 
ancora funzioni continue.

\begin{teorema}[Somma di funzioni continue]
Se \(f\) e \(g\) sono funzioni continue, anche \(f+g\) è continua.
\end{teorema}

\noindent Ipotesi: 
\(f(x) \text{ e} g(x)\) sono continue
\tab Tesi: 
\(f(x)+g(x)\) è continua.

\begin{proof}
Dato che sono continue: 
\[f(x+\epsilon) + g(x+\epsilon) = f(x)+\alpha + g(x)+\beta\]
Ma la somma di due infinitesimi è ancora un infinitesimo quindi:
\[f(x)+g(x)+\tonda{\alpha + \beta} \approx f(x)+g(x)\]
\end{proof}

\begin{teorema}[Prodotto di funzioni continue]
Se \(f\) e \(g\) sono funzioni continue, anche \(f \cdot g\) è continua.
\end{teorema}

\noindent Ipotesi: 
\(f(x) \text{ e} g(x)\) sono continue
\tab Tesi: 
\(f(x) \cdot g(x)\) è continua.

\begin{proof}
Dato che sono continue: 
\[f(x+\epsilon) \cdot g(x+\epsilon) = 
\tonda{f(x)+\alpha} \cdot \tonda{g(x)+\beta} = 
f(x) \cdot g(x) + f(x) \cdot \beta + g(x) \cdot \alpha + \alpha \cdot \beta
\approx f(x) \cdot g(x)\]

Dato che sia il prodotto tra un numero finito e un infinitesimo, sia il 
prodotto tra due infinitesimi sono infinitesimi e lo è anche la loro somma. 
\end{proof}

\begin{corollario}
 Ogni funzione polinomiale è continua.
\end{corollario}

\begin{proof}
Dato che una funzione polinomiale si può ottenere partendo da funzioni 
costanti e da funzioni identiche attraverso moltiplicazioni e addizioni, 
la tesi consegue dai teoremi precedenti. 
\end{proof}

\begin{esempio}
 Dimostrare che \(f(x)=2x^2 + 3\) è una funzione continua.

\begin{proof}

\(f(x)=2x^2 + 3\) è continua perché è somma di due funzioni continue: 
% \begin{itemize}[noitemsep]
%  \item \(f(x)=2x^2 + 3\) è continua perché è somma di due funzioni continue: 
 \begin{itemize}[nosep]
  \item \(y=2x^2\) è continua perché è prodotto di due funzioni continue:
  \begin{itemize}[nosep]
   \item \(y=2\) è continua perché è una costante;
   \item \(y=x^2\) è continua perché è prodotto di due funzioni continue:
   \begin{itemize}[nosep]
    \item \(y=x\) è continua perché è una funzione identica;
    \item \(y=x\) è continua perché è una funzione identica;
   \end{itemize}
  \end{itemize}
  \item \(y=3\) è continua perché è una costante;
 \end{itemize}
% \end{itemize}
\end{proof}

\end{esempio}

\begin{teorema}[Funzioni di funzioni]
Se \(f(x)\) e \(g(x)\) sono funzioni continue, anche \(f(g(x))\) è continua.
\end{teorema}

\noindent Ipotesi: 
\(f(x) \text{ e} g(x)\) sono continue
\tab Tesi: 
\(f(x) \star g(x) = f(g(x))\) è continua.

\begin{proof}
Dato che \(g\) è continua: 
\[f(g(x+\epsilon)) = f(g(x)+\alpha)\]
e dato che \(f\) è continua: 
\[f(g(x)+\alpha)=f(g(x))+\beta\]
quindi: 
\[f(g(x+\epsilon)) = f(g(x)+\alpha) = f(g(x))+\beta \approx f(g(x))\]
\end{proof}

\section{Massimi e minimi}
\label{sec:cont_massimiminimi}

Teorema del punto critico

Metodo per trovare i massimi e minimi

\section{Massimi e minimi: applicazioni}
\label{sec:cont_applicazioni}

\section{Derivate e grafico di funzioni}
\label{sec:cont_derivate_studiof}

\section{Proprietà delle funzioni continue}
\label{sec:cont_proprieta}

\subsection{iperinteri}
\label{subsec:cont_iperinteri}

