% (c) 2015 Daniele Zambelli daniele.zambelli@gmail.com

% (c) 2014 Daniele Zambelli - daniele.zambelli@gmail.com
% 
% Tutti i grafici per il capitolo relativo alle parabole
%
% 

\newcommand{\espdueterzi}{% 
    % Esponenziali con basi diverse.
    \disegno{
    \rcom{-10}{+10}{-1}{10}{gray!50, very thin, step=1}
    \begin{scope}[ultra thick, color=Maroon!50!black]
     \tkzInit[xmin=-10.3, xmax=+10.3, ymin=-0.3, ymax=+10.3]
     \tkzFct[domain=-10.3:+6]{(3./2)**x}
     \tkzFct[color=Green!50!black, domain=-6:+10.3]{(2./3)**x}
     \begin{scope}[color=Black!50!black]
      \filldraw (1, 3./2) circle (1.2pt);
      \filldraw (1, 2./3) circle (1.2pt);
     \end{scope}
     \filldraw [color=Red](0, 1) circle (1.2pt);  
    \end{scope}
    \begin{scope}[color=black]
     \draw (-7.3, 7) node{\(f(x)=\tonda{\dfrac{2}{3}}^x\)}; 
     \draw ((7.3, 7) node{\(f(x)=\tonda{\dfrac{3}{2}}^x\)};
    \end{scope}
    }
}

\newcommand{\logduebasi}{% 
    % Esponenziali con basi diverse.
    \disegno{
    \rcom{-1}{+10}{-9}{9}{gray!50, very thin, step=1}
    \begin{scope}[ultra thick, color=Maroon!50!black]
      \tkzInit[xmin=-1.3, xmax=+80, xstep=.5, ymin=-10.3,ymax=+10.3]
      \tkzFct[domain=.01:+10]{log(x)/log(2)}
      \filldraw (2, 1) circle (1.2pt);
      \begin{scope}[color=Green!50!black]
        \tkzFct[domain=-.01:+10]{log(x)/log(1./2)}
        \filldraw (2, -1) circle (1.2pt);
      \end{scope}
    \end{scope}
    \begin{scope}[color=black]
      \draw (9.5, 2.8) node{a=2}; 
      \draw (9.5, -2.8) node{a=0.5};
    \end{scope}
      \filldraw [color=Red] (1,0) circle (1.2pt);
    }
}


\chapter{Funzioni continue}

\section{TODO}
\begin{comment}
 
Schema del capitolo
===================

Limiti
------

Continuità
----------

  C. in un punto
  ''''''''''''''
  
    Definizione
    ...........
  
      Definizioni equivalenti
      
    Punti di discontinuità e di non derivabilità
    ............................................
 
  C. in un intervallo
  '''''''''''''''''''
  
    Definizione
    ...........
  
      Teorema funzioni elementari
      Teorema composizione di funzioni
      
    Insieme e sottoinsiemi delle funzioni
    .....................................
    
Massimi e minimi
----------------

  Definizione
  '''''''''''
  
    Teorema del punto critico
    
  Proprietà delle f. continue
  '''''''''''''''''''''''''''
  
    I numeri iperinteri
    ...................
    
    Teorema degli zeri
    
    Teorema dei valori estremanti
    
    Teorema di Rolle
    
    Teorema di Lagrange
    
    Corollario

\end{comment}

\begin{comment}
\begin{center}
\begin{inaccessibleblock}[TODO.]
  \telescopio
%   \caption{...e i corrispondenti punti.} \label{fig:potdue0}
\end{inaccessibleblock}
\end{center}
\end{comment}

\begin{center}
\begin{inaccessibleblock}[TODO.]
%   \telescopio
%   
%   \iperinteri
%   \caption{...e i corrispondenti punti.} \label{fig:potdue0}
\end{inaccessibleblock}
\end{center}

\section{Limiti}
\label{sec:cont_limiti}

In alcuni problemi non siamo interessati a sapere come si comporta una funzione 
per un valore ben preciso, dove magari non è definita, ci interessa di più 
sapere come si comporta quando si \emph{avvicina} a quel valore.

Per trattare queste situazioni i matematici si sono inventati il concetto di 
\emph{limite}.

TODO problema sensato sui limiti.

\begin{definizione}
\(l\) è il \textbf{limite} di una funzione \(f(x)\) per \(x\) 
che tende a un 
valore \(c\), se, quando \(x\) è infinitamente vicino a \(c\), 
ma diverso da \(c\), 
allora \(f(x)\) è infinitamente vicino a \(l\). E si scrive:

\[l=\lim_{x \rightarrow c} f(x) \Leftrightarrow 
\forall x \tonda{\tonda{x \approx c \wedge x \neq c} \Rightarrow 
\tonda{f(x) \approx l}}\]

\end{definizione}


\section{Continuità}
\label{sec:cont_continuita}

\subsection{Definizione di continuità in un punto}
\label{subsec:cont_definizione}


\begin{definizione}
Diremo che una funzione è \textbf{continua} in un punto \(c\), 
se è definita in \(c\) e, 
quando \(x\) è infinitamente vicino a \(c\), 
allora \(f(x)\) è infinitamente vicino a \(f(c)\). E si scrive::

\[f \text{ è continua in } c \Leftrightarrow 
\forall x \tonda{\tonda{x \approx c} \Rightarrow 
\tonda{f(x) \approx f(c)}}\]

\end{definizione}

\begin{esempio}
 Data la funzione \(f(x)=x^2-3x\) dimostrare che \(f(x)\) è continua in~4.
 
 La funzione è continua in~4 se per ogni \(x\) infinitamente vicino a~4 
 \(f(x)\) è infinitamente vicino a \(f(4)\). Cioè se \(x -4=\epsilon\) allora
 \(f(x) -f(4) = \delta\) dove \(\epsilon\) e \(\delta\) sono due infinitesimi.
 
\emph{dimostrazione}

Da \(x-4=\epsilon\) si ricava che \(x=4+\epsilon\), quindi: 
\[f(x) -f(4) = f(4+\epsilon) -f(4) = 
\tonda{4+\epsilon}^2-3\tonda{4+\epsilon}-\tonda{4^2-3\cdot 4}=\]
\[=\cancel{16}+8\epsilon+\epsilon^2\cancel{-12} 
  -3\epsilon\cancel{-16}\cancel{+12} = 
  +8\epsilon+\epsilon^2 -3\epsilon = 
\epsilon \tonda{5 + \epsilon}\]

Ora, il prodotto tra un infinitesimo e un finito è un infinitesimo, quindi, se 
la distanza tra \(x\) e \(4\) è infinitesima, anche la distanza tra 
\(f(x)\) e \(f(4)\) è infinitesima. qed 
 
\end{esempio}

\begin{esempio}
 Dimostrare che \(f(x)=\frac{\abs{x}}{x}\) non è continua in~0.
 
\emph{dimostrazione}
TODO
 
\end{esempio}

\begin{esempio}
 Dimostrare che \(f(x)=\frac{\abs{x}}{3}\) è continua in~0.
 
\emph{dimostrazione}
TODO
 
\end{esempio}

\begin{esempio}
 Studiare la continuità di una funzione definita a tratti nel punto di 
giunzione.
 
TODO
 
\end{esempio}

Per riassumere, data una funzione \(y=f(x)\) definita in \(c\), le seguenti 
affermazioni sono equivalenti:

\begin{enumerate}[noitemsep]
 \item \(f\) è continua in \(c\);
 \item se \(x \approx c\) allora \(f(x) \approx f(c)\);
 \item se \(\st(x) = c\) allora \(\st(f(x)) = f(c)\);
 \item \(\lim_{x \to c} f(x) = f(c)\);
 \item se \(x\) si allontana da \(c\) di un infinitesimo allora 
 \(f(x)\) si allontana da \(f(c)\) di un infinitesimo.
%  \item se \(\Delta x\) è infinitesimo allora \(\Delta y\) è infinitesimo.
\end{enumerate}

\begin{teorema}[Derivabilità e continuità]
Se una funzione è derivabile in un punto allora è continua in quel punto.
\end{teorema}

\noindent Ipotesi: 
\(f(x) \text{ è derivabile in } c\)
\tab Tesi: 
\(f(x) \text{ è continua in } c\).

\begin{proof}
TODO
\end{proof}

\subsection{Definizione di continuità in un intervallo}
\label{subsec:cont_definizione}

Dimostrare che una funzione è continua in un punto è piuttosto laborioso, pur 
non essendo complicato, ma presenta un problema quando sono interessato a 
studiare la continuità di una funzione in un intervallo. Infatti in un 
intervallo, anche piccolo, i punti sono infiniti e dimostrare la continuità per 
ognuno di essi risulta piuttosto lungo\dots.

Per superare questo scoglio, i matematici hanno pensato un approccio diverso:

\begin{itemize}
 \item dimostrare che alcune funzioni elementari sono continue;
 \item dimostrare che la somma, il prodotto, il quoziente e la composizione di 
funzioni continue è ancora una funzione continua.
\end{itemize}

In questo modo si può riconoscere la continuità di un gran numero di funzioni 
senza fare noiosi calcoli. Di seguito vediamo qualcuno di questi teoremi.

\subsubsection{Funzioni elementari}
\label{subsubsec:cont_funzionielementari}

Dimostriamo la continuità di alcune funzioni elementari.

\begin{teorema}[Continuità delle costanti]
Le funzioni costanti sono continue.
\end{teorema}

\noindent Ipotesi: \(f(x)=k\).\tab Tesi: \(f(x)\) è continua.

\begin{proof}
Per la definizione di continuità vogliamo dimostrare che 
\[\forall x \text{ se } x_0 \approx x \text{ allora } f(x_0) \approx f(x)\]
Poniamo \(x_0=x+\epsilon\), essendo la funzione costante, anche 
\[f(x+\epsilon)=k\] 
che, ovviamente, è infinitamente vicino a \(k\). In simboli:
\[f(x_0) = f(x+\epsilon) = k \approx k = f(x)\] 
\end{proof}

\begin{teorema}[Continuità della funzione identica]
La funzione identica (\(y=x\)) è continua.
\end{teorema}

\noindent Ipotesi: \(f(x)=x\).\tab Tesi: \(f(x)\) è continua.

\begin{proof}
Per la definizione di continuità vogliamo dimostrare che 
\[\forall x \text{ se } x_0 \approx x \text{ allora } f(x_0) \approx f(x)\]
Poniamo \(x_0=x+\epsilon\), \(f(x_0) = f(x+\epsilon)=x+\epsilon\). 
Dato che la differenza:
\[f(x+\epsilon)-f(x) = x+\epsilon-x= \epsilon\]
è un infinitesimo, allora i due valori sono infinitamente vicini. In simboli:
\[f(x+\epsilon) = x+\epsilon \approx x = f(x)\] 
\end{proof}

\begin{teorema}[Continuità della funzione seno]
La funzione seno (\(y=\sen x\)) è continua.
\end{teorema}

\noindent Ipotesi: \(f(x)=\sen x\).\tab Tesi: \(f(x)\) è continua.

\begin{proof}
Usando la prima proprietà delle potenze:
\[f(x+\epsilon) =
\sen{(x+\epsilon)} = \sen x \cos \epsilon - \cos x \sen \epsilon\]
Se un angolo è infinitamente vicino a zero avrà il coseno infinitamente vicino 
a uno e il seno infinitamente vicino a zero. Quindi:
\[\sen{(x+\epsilon)} = \sen x + \delta\]
Perciò:
\[f(x+\epsilon) =
\sen{(x+\epsilon)} = \sen{(x+\epsilon)} = 
\sen x + \delta \approx \sen x = f(x)\]
\end{proof}

\begin{teorema}[Continuità della funzione esponenziale]
La funzione esponenziale (\(y=a^x\)) è continua.
\end{teorema}

\noindent Ipotesi: \(f(x)=a^x\).\tab Tesi: \(f(x)\) è continua.

\begin{proof}
Usando la prima proprietà delle potenze:
\[f(x+\epsilon) =
a^{x+\epsilon} = a^{x} \cdot a^{\epsilon} \approx a^{x} \cdot 1 = a^{x} = f(x)\]
\end{proof}

Oltre alle funzioni precedenti, anche altre funzioni elementari sono continue, 
il seguente elenco riporta le principali funzioni continue:

\begin{itemize} [noitemsep]
 \item \(y=k\)
 \item \(y=x\)
 \item \(y=\frac{1}{x}\) \quad \textasteriskcentered
 \item \(y=\sqrt[n]{x}\) \quad \textasteriskcentered
 \item \(y=\abs{x}\)
 \item \(y=a^x\)
 \item \(y=\log_a x\) \quad \textasteriskcentered
 \item \(y=\sen x\)
 \item \(y=\cos x\)
 \item \(y=\tg x\) \quad \textasteriskcentered
\end{itemize}

\begin{osservazione}
Le funzioni segnate da \textasteriskcentered sono continue non su tutto \(\R\), 
ma solo \textbf{all'interno del loro campo di esistenza}.
\end{osservazione}

\subsubsection{Composizione di funzioni}
\label{subsubsec:cont_composizionefunzioni}

Vediamo ora che anche componendo in alcuni modi funzioni continue otteniamo 
ancora funzioni continue.

\begin{teorema}[Somma di funzioni continue]
Se \(f\) e \(g\) sono funzioni continue, anche \(f+g\) è continua.
\end{teorema}

\noindent Ipotesi: 
\(f(x) \text{ e} g(x)\) sono continue
\tab Tesi: 
\(f(x)+g(x)\) è continua.

\begin{proof}
Dato che sono continue: 
\[f(x+\epsilon) + g(x+\epsilon) = f(x)+\alpha + g(x)+\beta\]
Ma la somma di due infinitesimi è ancora un infinitesimo quindi:
\[f(x)+g(x)+\tonda{\alpha + \beta} \approx f(x)+g(x)\]
\end{proof}

\begin{teorema}[Prodotto di funzioni continue]
Se \(f\) e \(g\) sono funzioni continue, anche \(f \cdot g\) è continua.
\end{teorema}

\noindent Ipotesi: 
\(f(x) \text{ e} g(x)\) sono continue
\tab Tesi: 
\(f(x) \cdot g(x)\) è continua.

\begin{proof}
Dato che sono continue: 
\[f(x+\epsilon) \cdot g(x+\epsilon) = 
\tonda{f(x)+\alpha} \cdot \tonda{g(x)+\beta} = 
f(x) \cdot g(x) + f(x) \cdot \beta + g(x) \cdot \alpha + \alpha \cdot \beta
\approx f(x) \cdot g(x)\]

Dato che sia il prodotto tra un numero finito e un infinitesimo, sia il 
prodotto tra due infinitesimi sono infinitesimi e lo è anche la loro somma. 
\end{proof}

\begin{corollario}
 Ogni funzione polinomiale è continua.
\end{corollario}

\begin{proof}
Dato che una funzione polinomiale si può ottenere partendo da funzioni 
costanti e da funzioni identiche attraverso moltiplicazioni e addizioni, 
la tesi consegue dai teoremi precedenti. 
\end{proof}

\begin{esempio}
 Dimostrare che \(f(x)=2x^2 + 3\) è una funzione continua.

\begin{proof}

\(f(x)=2x^2 + 3\) è continua perché è somma di due funzioni continue: 
% \begin{itemize}[noitemsep]
%  \item \(f(x)=2x^2 + 3\) è continua perché è somma di due funzioni continue: 
 \begin{itemize}[nosep]
  \item \(y=2x^2\) è continua perché è prodotto di due funzioni continue:
  \begin{itemize}[nosep]
   \item \(y=2\) è continua perché è una costante;
   \item \(y=x^2\) è continua perché è prodotto di due funzioni continue:
   \begin{itemize}[nosep]
    \item \(y=x\) è continua perché è una funzione identica;
    \item \(y=x\) è continua perché è una funzione identica;
   \end{itemize}
  \end{itemize}
  \item \(y=3\) è continua perché è una costante;
 \end{itemize}
% \end{itemize}
\end{proof}

\end{esempio}

\begin{teorema}[Funzioni di funzioni]
Se \(f(x)\) e \(g(x)\) sono funzioni continue, anche \(f(g(x))\) è continua.
\end{teorema}

\noindent Ipotesi: 
\(f(x) \text{ e} g(x)\) sono continue
\tab Tesi: 
\(f(x) \star g(x) = f(g(x))\) è continua.

\begin{proof}
Dato che \(g\) è continua: 
\[f(g(x+\epsilon)) = f(g(x)+\alpha)\]
e dato che \(f\) è continua: 
\[f(g(x)+\alpha)=f(g(x))+\beta\]
quindi: 
\[f(g(x+\epsilon)) = f(g(x)+\alpha) = f(g(x))+\beta \approx f(g(x))\]
\end{proof}

\section{Massimi e minimi}
\label{sec:cont_massimiminimi}

Data una funzione definita in un certo intervallo, può darsi che questa 
funzione abbia un massimo  un minimo in questo intervallo.

\begin{definizione}
 Chiamiamo \textbf{massimo} di una funzione in un intervallo \(I\) un punto 
\(\punto{c}{f(c)}\) tale che, per ogni \(x\) appartenente all'intervallo, 
\(f(c)\) sia maggiore o uguale a \(f(x)\):
\[\punto{c}{f(c)} \text{ è un massimo se } \forall x \in I \quad
f(c) \geqslant f(x)\]
\end{definizione}

La definizione di \textbf{minimo} in un intervallo si ottiene facilmente 
modificando quella di massimo (scrivila tu e poi confrontala con quella scritta 
dagli altri tuoi compagni.

In un intervallo, una funzione potrebbe avere \emph{più} minimi o massimi. 
Oppure potrebbe \emph{non} avere minimi o massimi.

\begin{esempio}
 \(y=3\) TODO
\end{esempio}

\begin{esempio}
 \(y=\sen{x}\) TODO
\end{esempio}

\begin{esempio}
 \(y=\dfrac{1}{x}\) TODO
\end{esempio}

Un importante teorema che riguarda i massimi e i minimi delle funzioni 
continue dice che se una funzione è continua e ha in punto di massimo (o 
di minimo), allora questo può trovarsi:
\begin{enumerate}[label=\roman*), noitemsep]
 \item o in un estremo;
 \item o in un punto non derivabile;
 \item o in un punto la cui derivata vale zero.
\end{enumerate}

Noi dimostreremo il seguente teorema:
\begin{teorema}[Teorema di Fermat]
% Se una funzione è continua, 
Se una funzione è  
definita in un intervallo chiuso, 
ha un massimo (minimo) in un punto interno all'intervallo
e in quel punto è derivabile, 
allora in quel punto ha derivata nulla.
\end{teorema}

\noindent Ipotesi:
\begin{enumerate}[nosep]
%  \item \(f\) è una funzione continua nell'intervallo chiuso \(\intervcc{a}{b}\)
 \item \(f\) è una funzione definita nell'intervallo chiuso \(\intervcc{a}{b}\)
 \item \(c\) appartiene all'intervallo aperto \(\intervaa{a}{b}\)
 \item \(f(c)\) è un massimo (minimo);
 \item \(f\) è derivabile in \(c\)
\end{enumerate}

\noindent Tesi: 

la derivata \(f'(c)=0\)
 
% \end{teorema}
% \begin{minipage}[t]{.52\textwidth}
% \noindent Ipotesi:
% \begin{enumerate}[nosep]
%  \item \(f\) è una funzione continua nell'intervallo chiuso \(intervcc{a}{b}\);
%  \item \(c\) appartiene all'intervallo aperto \(intervaa{a}{b}\);
%  \item \(f(c)\) è un massimo (minimo);
%  \item \(f\) è derivabile in \(c\).
% \end{enumerate}
% \end{minipage}
% \hspace{.5cm}
% \begin{minipage}[t]{.44\textwidth} 
% Tesi: la derivata \(f'(c)=0\).
% % \begin{enumerate}[label=\roman*), nosep]
% %  \item \(c\) è un estremo o
% %  \item \(c\) è un punto non derivabile o
% %  \item \(c\) è un punto la cui derivata vale zero.
% % \end{enumerate}
% \end{minipage} 
 
\begin{proof}
Consideriamo un valore \(\Delta\) positivo abbastanza piccolo 
in modo che \(c+\Delta\) appartenga ancora all'intervallo \([a;~b]\).
Poiché \(f(c)\) è un massimo: 
\[f(c+\Delta) \leqslant f(c) \Rightarrow f(c+\Delta) - f(c) \leqslant 0\]
Dividendo entrambi i membri per \(\Delta\) otteniamo:
\[\frac{f(c+\Delta) - f(c)}{\Delta} \leqslant 0\]
Questa disuguaglianza continua a valere anche se \(\Delta\) è un infinitesimo:
\[\frac{f(c+\delta) - f(c)}{\delta} \leqslant 0\] 
e prendendo la parte standard dell'espressione otteniamo:
\[f'_+(c)=\pst{\frac{f(c+\delta) - f(c)}{\delta}} \leqslant 0\] 
Ora possiamo ripetere le stesse considerazioni prendendo un valore \(\Delta\) 
negativo.
Poiché \(f(c)\) è un massimo: 
\[f(c+\Delta) \leqslant f(c) \Rightarrow f(c+\Delta) - f(c) \leqslant 0\]
Questa volta dividendo entrambi i membri per \(\Delta\) dobbiamo tener conto 
che \(\Delta\) è negativo quindi dobbiamo invertire il verso del predicato:
\[\frac{f(c+\Delta) - f(c)}{\Delta} \geqslant 0\]
Questa disuguaglianza continua a valere anche se \(\Delta\) è un infinitesimo:
\[\frac{f(c+\delta) - f(c)}{\delta} \geqslant 0\] 
e prendendo la parte standard dell'espressione otteniamo:
\[f'_-(c)=\pst{\frac{f(c+\delta) - f(c)}{\delta}} \geqslant 0\] 

Ma poiché per ipotesi la funzione è derivabile in \(c\), il valore 
dell'espressione: \(\pst{\frac{f(c+\delta) - f(c)}{\delta}}\) non dipende dal 
valore dell'infinitesimo \(\delta\) perciò: \(f'_-(c) = f'_+(c)\). 
Quindi:

\[0 \leqslant f'_-(c) = f'_+(c) \leqslant 0\]

Da cui si ricava la tesi:
\[f(c) = 0\]

\end{proof}

Metodo per trovare i massimi e minimi

\section{Massimi e minimi: applicazioni}
\label{sec:cont_applicazioni}

\section{Derivate e grafico di funzioni}
\label{sec:cont_derivate_studiof}

\section{Proprietà delle funzioni continue}
\label{sec:cont_proprieta}

\subsection{Numeri iperinteri}
\label{subsec:cont_iperinteri}

Per affrontare alcuni dei prossimi argomenti, abbiamo bisogno di un altro 
strumento matematico: l'insieme dei numeri \emph{Iperinteri}.
Non è difficile visualizzare sulla retta dei numeri questo insieme.

Per dare una definizione rigorosa dei numeri Iperinteri abbiamo bisogno di 
usare la funzione \emph{parte intera} di un numero: che si indica con il 
simbolo:
\(\quadra{x}\).
\begin{center}
\begin{tabular}{ccccccccccccccc}
\(x\) & 
-3&-2,4&-2,01&-1,2&-1,03&-0,3&0,2&1&1,6&1,99&2&2.03&2.9&3,42\\
\hline
\(y=\quadra{x}\) & 
-3&-3  &-3   &-2  &-2   &-1  &0  &1&1  &1   &2&2   &2  &3
\end{tabular}
\end{center}

La parte intera di un numero \(x\) è il più grande numero intero \(n\) 
minore o uguale a \(x\). Attenzione che mentre per i numeri positivi il 
concetto è abbastanza naturale, per quelli negativi il concetto non è 
altrettanto immediato, vedi la tabella precedente.

Applicando la funzione parte intera ai numeri Iperreali otteniamo gli 
Iperinteri.

\begin{definizione}
 I numeri \textbf{Iperinteri} sono quei numeri Iperreali per cui vale 
l'uguaglianza:
 \[x=\quadra{x}\]
\end{definizione}

Possiamo fare alcune osservazioni sugli Iperinteri:

\begin{enumerate}
 \item 
La somma algebrica di due numeri Iperinteri è un numero Iperintero.
 \item 
Ogni numero Iperreale si trova tra due numeri Iperinteri:
\[\forall x \in \IR \quadra{x} \leqslant x < \quadra{x}+1\]
\end{enumerate}

Possiamo usare gli Interi per dividere un intervallo Reale \([a;~b]\)
in \(n\) parti uguali. Ciascuna di queste \(n\) parti 
uguali è lunga \(l=\frac{b-a}{n}\).

Gli \(n\) sotto intervalli che si ottengono sono:
\[\left[a;~a+l \right[,~[a+l;~a+2l[,~\dots,~
[a+jl;~a+(j+1)l[,~\dots,~
[a+\tonda{n-1}l;~a+nl=b]\]

Gli estremi di questi intervalli sono chiamati punti di partizione 
dell'intervallo:
\[a;~a+l;~a+2l;~a+3l;~\dots;~a+\tonda{n-1}l;~a+nl=b\]

Possiamo ora estendere questo procedimento ai numeri Iperreali.
Scegliamo un numero infinito iperintero \(H\) e dividiamo in parti 
uguali l'intervallo di numeri Iperreali \(\intervcc{a}{b}\). Ogni 
sotto intervallo avrà la stessa lunghezza infinitesima \(\delta=\frac{b-a}{H}\).

Gli \(H\) sotto intervalli che si ottengono sono:
\[\left[a;~a+\delta \right[,~[a+\delta;~a+2\delta[,~
[a+j\delta;~a+(j+1)\delta[,~\dots,~
[a+\tonda{H-1}\delta;~a+H\delta=b]\]

e i punti di partizione sono:
\[a;~a+\delta;~a+2\delta;~\dots;~a+j\delta;~\dots;~a+H\delta=b\]
cioè i punti \(a+j\delta\) con \(j\) che varia da~0 a~\(H\).

Ogni numero iperreale \(x\) appartenente all'intervallo \(\intervca{a}{b}\)
apparterrà a uno dei sotto intervalli infinitesimi:
\[x \in \intervca{a+j\delta}{a+\tonda{j+1}\delta} \quad \Rightarrow \quad 
  a+j\delta \leqslant x < a+\tonda{j+1}\delta\]

Possiamo ora affrontare alcuni teoremi riguardanti le funzioni continue.

\subsection{Alcuni teoremi delle funzioni continue}
\label{subsec:cont_iperinteri}

Il prossimo teorema riguarda gli zeri di una funzione. Con zero di una funzione 
si intende un valore della \(x\) che rende la funzione uguale a zero:

\begin{definizione}
 \(c\) è uno \textbf{zero} della funzione \(f(x)\) se \(f(c)=0\)
\end{definizione}

\begin{teorema}[Teorema degli zeri]
Supponiamo che una funzione \(f(x)\) sia continua nell'intervallo chiuso
\(\intervcc{a}{b}\) e agli estremi dell'intervallo assuma valori di segno 
opposto allora la funzione ha uno zero nell'intervallo 
aperto~\(\intervaa{a}{b}\).
\end{teorema}

\noindent Ipotesi:
\begin{enumerate}[nosep]
 \item \(f\) è una funzione continua;
 \item \(f\) è definita nell'intervallo chiuso \(\intervcc{a}{b}\);
 \item \(f(a) \cdot f(b) < 0\) 
 (equivale a dire che \(f(a)\) e \(f(b)\) hanno valori discordi).
\end{enumerate}

\noindent Tesi: 

\(\exists~c \in \intervaa{a}{b} \text{ tale che } f(c)=0\).

% \begin{minipage}[t]{.52\textwidth}
% \noindent Ipotesi:
% \begin{enumerate}[nosep]
%  \item \(f\) è una funzione continua;
%  \item \(f\) è definita nell'intervallo chiuso \(\intervcc{a}{b}\);
%  \item \(f(a) \cdot f(b) < 0\) 
%  (equivale a dire che \(f(a)\) e \(f(b)\) hanno valori discordi).
% \end{enumerate}
% \end{minipage}
% \hspace{.5cm}
% \begin{minipage}[t]{.44\textwidth} 
% Tesi: \(\exists~c \in \intervaa{a}{b} \text{ tale che } f(c)=0\).
% \end{minipage} 

\begin{proof}
Supponiamo che \(f(a)<0\) e \(f(b)>0\), posto \(H\) un Iperintero 
infinito, dividiamo l'intervallo iperreale~\(\intervcc{a}{b}\) in~\(H\) 
parti uguali
\[a;~a+\delta;~a+2\delta;~\dots;~a+j\delta;~\dots;~a+H\delta=b\]
Chiamiamo \(k\) l'indice per il quale:
\[f(a+k\delta) \leqslant 0 < f(a+(k+1)\delta)\]
% Ma \(a+k\delta\) è infinitamente vicino a \(a+(k+1)\delta\), quindi hanno la 
% stessa parte standard che chiamiamo \(c\):
% \[c=\pst{a+k\delta}=\pst{a+(k+1)\delta}\]
Dato che \(f\) è continua:
\[a+k\delta \approx a+(k+1)\delta \quad \Rightarrow \quad 
f(a+k\delta) \approx f(a+(k+1)\delta)\] 
Ma l'unico numero standard che è infinitamente vicino ad un numero
minore di zero e anche ad un numero maggiore di zero è zero. 
Quindi chiamando:\(c=\pst{a+k\delta}=\pst{a+(k+1)\delta}\)
\[f(c)=f(\pst{a+k\delta})=\pst{f(a+k\delta)}=0\]

In modo analogo si dimostra il caso in cui \(f(a)>0>f(b)\).
\end{proof}

Il teorema dimostra che in \(\intervcc{a}{b}\) c'è almeno uno zero della 
funzione, ma non dice nulla sul numero degli zeri.

La dimostrazione di questo teorema permette di dimostrarne facilmente degli 
altri: 

\begin{corollario}
Il teorema dei valori intermedi dice che se una funzione è continua in un
intervallo \(\intervcc{a}{b}\) allora tra \(a\) e \(b\) assume tutti i valori 
compresi tra \(f(a)\) e \(f(b)\).
\end{corollario}

\begin{proof}
Suggerimento: 
supponiamo che \(f(a) \leqslant h < f(b)\), 
costruiamo una nuova funzione: \(g(x)=f(x)-h\).
La funzione \(g\) soddisfa tutte le ipotesi del teorema precedente per cui:
\[\exists~c \in \intervaa{a}{b} \text{ tale che } g(c)=0\]
e sostituendo la funzione \(g\) otteniamo:
\[g(c)=0 \Rightarrow f(c)-h=0 \Rightarrow f(c)=h\].
\end{proof}

\begin{corollario}
Se una funzione è continua in un
intervallo \(\intervcc{a}{b}\) e 
\(f(x)\neq 0 \quad \forall x \in \intervcc{a}{b}\) 
allora:
\begin{enumerate}[nosep]
 \item se \(f(c)<0\) per un qualunque \(c \in \intervcc{a}{b}\)
 allora \(f(x)<0 \quad \forall x \in \intervcc{a}{b}\);
 \item se \(f(c)>0\) per un qualunque \(c \in \intervcc{a}{b}\)
 allora \(f(x)>0 \quad \forall x \in \intervcc{a}{b}\).
\end{enumerate}
\end{corollario}

\begin{proof}
Suggerimento: 
consideriamo il primo caso: \(f(c)>0\) 
se esistesse un valore \(d \in \intervcc{a}{b}\) per cui
\(f(d)>0\) allora nell'intervallo \(\intervcc{c}{d}\) 
esiste un punto \(e \Rightarrow f(e)=0\)
essendo \(\intervcc{c}{d} \subseteq \intervcc{a}{b}\) 
ciò contraddirebbe l'ipotesi.
In modo analogo si dimostra il secondo caso.
\end{proof}

Un altro importante teorema è il seguente.

\begin{teorema}[Teorema di Weierstrass]
Supponiamo che una funzione \(f(x)\) sia continua nell'intervallo chiuso
\(\intervcc{a}{b}\) allora, in questo intervallo assume un valore massimo e un 
valore minimo.
\end{teorema}

% \begin{minipage}[t]{.52\textwidth}
% \noindent Ipotesi:
% \begin{enumerate}[nosep]
%  \item \(f\) è una funzione continua;
%  \item \(f\) è definita nell'intervallo chiuso \(\intervcc{a}{b}\)
% \end{enumerate}
% \end{minipage}
% \hspace{.5cm}
% \begin{minipage}[t]{.44\textwidth} 
% Tesi: 
% \begin{enumerate}[nosep]
%  \item \(\exists~c \in \intervcc{a}{b} \text{ tale che }
%          f(c) \geqslant f(x) \forall x \in \intervcc{a}{b}\);
%  \item \(\exists~c \in \intervcc{a}{b} \text{ tale che }
%          f(c) \leqslant f(x) \forall x \in \intervcc{a}{b}\).
% \end{enumerate}
% \end{minipage} 

\noindent Ipotesi:
\begin{enumerate}[nosep]
 \item \(f\) è una funzione continua;
 \item \(f\) è definita nell'intervallo chiuso \(\intervcc{a}{b}\)
\end{enumerate}

\noindent Tesi: 
\begin{enumerate}[nosep]
 \item \(\exists~c \in \intervcc{a}{b} \text{ tale che }
         f(c) \geqslant f(x) \quad \forall x \in \intervcc{a}{b}\);
 \item \(\exists~c \in \intervcc{a}{b} \text{ tale che }
         f(c) \leqslant f(x) \quad \forall x \in \intervcc{a}{b}\).
\end{enumerate}

\begin{proof}
Dimostriamo la prima tesi.

Operiamo una divisione infinita dell'intervallo \(\intervcc{a}{b}\)
ottenendo i punti di partizione: 
\[a;~a+\delta;~a+2\delta;~\dots;~a+j\delta;~\dots;~a+H\delta=b\]
Per il principio di tranfer posso confrontare tra di loro tutti i valori della 
funzione in questi punti e troverò che uno di questi è maggiore o uguale a 
tutti gli altri, supponiamo che questo punto sia: \(a+(k+1)\delta\):
\[f(a+k\delta) \geqslant f(a+j\delta) \text{ per ogni j Iperintero}\]
Considerando la parte standard 
se \(c=\pst{a+k\delta}\) e \(d=\pst{a+j\delta}\)
ne deriva che:
\[f(c) \geqslant f(d)\]

In modo analogo si dimostra la seconda tesi.
\end{proof}


\begin{teorema}[Teorema di Rolle]
Supponiamo che una funzione \(f(x)\) continua nell'intervallo chiuso
\(\intervcc{a}{b}\),
sia derivabile nell'intervallo aperto
\(\intervaa{a}{b}\) 
e, agli estremi, assuma lo stesso valore: \(f(a) = f(b)\) allora
esiste un punto \(c\) dell'intervallo \(\intervaa{a}{b}\) nel quale 
la derivata è nulla: \(f'(c)=0\).
\end{teorema}

\noindent Ipotesi:
\begin{enumerate}[nosep]
 \item \(f\) è una funzione continua 
 nell'intervallo chiuso \(\intervcc{a}{b}\)
 \item \(f\) è una funzione derivabile 
 nell'intervallo aperto \(\intervaa{a}{b}\)
 \item \(f(a)=f(b)\)
\end{enumerate}

\noindent Tesi: 

\(\exists~c \in \intervaa{a}{b} \text{ tale che } f'(c)=0\);

\begin{proof}
Dato che valgono le ipotesi del teorema di Weierstrass, 
nell'intervallo \(\intervcc{a}{b}\) 
esisterà un massimo \(M\) e un minimo \(m\).
Si possono distinguere 3 casi:
\begin{enumerate} %[nosep]
 \item Se \(M = m = f(a) = f(b)\), la funzione è costante. 
 In questo caso la dimostrazione è banale 
 poiché~\(f'(c)=0 \forall c \in \intervaa{a}{b}\)
 \item Se \(M > f(a)\) vuol dire che la funzione ha un massimo 
 in~\(c\) ovvero~\(f(c)=M\). 
 La funzione \(f\) nel punto \(c\) soddisfa tutte le ipotesi del teorema 
 del punto critico, 
\begin{enumerate}[noitemsep]
 \item \(f\) è una funzione continua nell'intervallo chiuso \(\intervcc{a}{b}\);
 \item \(c\) appartiene all'intervallo aperto \(\intervaa{a}{b}\);
 \item \(f(c)\) è un massimo;
 \item \(f\) è derivabile in \(c\).
\end{enumerate}
 quindi in \(c\) la funzione ha derivata nulla: \(f'(c)=0\).
 \item Se \(m < f(a)\) vuol dire che la funzione ha un minimo 
 in~\(c\) ovvero~\(f(c)=m\).
 Si dimostra in modo analogo al punto 2.
\end{enumerate}
\end{proof}

\begin{teorema}[Teorema di Lagrange o della pendenza media]
La pendenza media di una funzione \(f\) in un intervallo 
\(\intervcc{a}{b}\) è data da:
\[\text{pendenza media} = \frac{f(b)-f(a)}{b-a}\]
Se una funzione \(f\) è continua nell'intervallo chiuso \(\intervcc{a}{b}\) e
è derivabile nell'intervallo aperto \(\intervaa{a}{b}\) allora
esiste un punto \(c\) dell'intervallo \(\intervaa{a}{b}\) nel quale 
la derivata è ha lo stesso valore della 
pendenza media:~\(f'(c)=\text{pendenza media}\).
\end{teorema}

\noindent Ipotesi:
\begin{enumerate}[nosep]
 \item \(f\) è una funzione continua 
 nell'intervallo chiuso \(\intervcc{a}{b}\)
 \item \(f\) è una funzione derivabile 
 nell'intervallo aperto \(\intervaa{a}{b}\)
\end{enumerate}

\noindent Tesi: 

\(\exists~c \in \intervaa{a}{b}\text{ tale che }f'(c)=\dfrac{f(b)-f(a)}{b-a}\);

\begin{proof}
Chiamiamo \(m\) la pendenza media: \(m=\dfrac{f(b)-f(a)}{b-a}\). 
La funzione lineare che congiunge i due 
punti \(\punto{a}{f(a)}\) e \(\punto{b}{f(b)}\) è:
\[l(x) = f(a) + m(x-a)\]
Costruiamo una nuova funzione uguale alla distanza in verticale tra \(f(x)\) 
e \(l(x)\):
\[h(x) = f(x) - l(x)\]
Questa nuova funzione soddisfa tutte le ipotesi del teorema di Rolle:
\begin{enumerate}[nosep]
 \item \(h\) è una funzione continua 
 nell'intervallo chiuso \(\intervcc{a}{b}\)
 poiché è somma di due funzioni continue;
 \item \(h\) è una funzione derivabile 
 nell'intervallo aperto \(\intervaa{a}{b}\)
 poiché è somma di due funzioni derivabili;
 \item \(h(a)=h(b)\)  poiché:
 
 \(h(a)=f(a) - (f(a) + m(a-a))=f(a)-f(a)-0=0\)
 
 e 
 
 \(h(b)=f(b) - (f(a) + m(b-a))=
 f(b)-f(a)-\dfrac{f(b)-f(a)}{b-a}\cdot \tonda{b-a}=\)\\
 \(=f(b)-f(a)-f(b)+f(a)=0\)
\end{enumerate}
 
 Quindi esiste un punto \(c\) dell'intervallo \(\intervaa{a}{b}\) tale che:
 \[h'(c)=0\]
 Sostituendo la funzione \(h\) con la sua definizione otteniamo:
 \[h'(c) = f'(c)-l'(c)= f'(c)-m=0 \Rightarrow f'(c)=m\]
 E sostituendo \(m\) otteniamo la tesi:
 \[f'(c)=\dfrac{f(b)-f(a)}{b-a}\]
\end{proof}

\begin{corollario}
 Derivata e andamento di una funzione. Se in una funzione \(f'(x)>0\) per ogni 
punto di un certo intervallo \(I\), allora la funzione è crescente in tutto 
l'intervallo.
\end{corollario}
% 
% \newcommand{\sand}{~ \wedge ~}
% \newcommand{\sor}{~ \vee ~}
% \newcommand{\sRarrow}{~ \Rightarrow ~}

\begin{proof}
Consideriamo due punti \(x_0 < x_1 ~ \in I\), 
per il teorema della pendenza media:
\[\exists c \in I \text { tale che } f'(c) = \dfrac{f(x_1)-f(x_0)}{x_1-x_0}\]
Dato che 
\[\forall c \in I \Rightarrow f'(c) > 0\]
si ha:
\[\dfrac{f(x_1)-f(x_0)}{x_1-x_0}>0 \]
E poiché il denominatore è positivo, \(x_1 - x_0 > 0\)
\[f(x_1)-f(x_0)>0 \sRarrow f(x_1)>f(x_0)\]
\end{proof}

% \begin{teorema}[De L'H\^opital]
% \(f(x)\) e \(g(x)\).
% \end{teorema}
% 
% \noindent Ipotesi: 
% \(f(x) \text{ e} g(x)\)
% \tab Tesi: 
% \(f(x) \star g(x) = f(g(x))\).
% 
% \begin{proof}
% Dato che \(g\) è continua: 
% \[f(g(x+\epsilon)) = f(g(x)+\alpha)\]
% e dato che \(f\) è continua: 
% \[f(g(x)+\alpha)=f(g(x))+\beta\]
% quindi: 
% \[f(g(x+\epsilon)) = f(g(x)+\alpha) = f(g(x))+\beta \approx f(g(x))\]
% \end{proof}












