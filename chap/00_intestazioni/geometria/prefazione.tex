% Copyright (c) 2015 Daniele Masini - d.masini.it@gmail.com

\pagestyle{matc3page}
\chapter*{Prefazione}
\addcontentsline{toc}{chapter}{Prefazione}
\markboth{Prefazione}{Prefazione}
Guardando i libri di testo sia con gli occhi dell'insegnante che li usa, sia dell'autore che li scrive, ci si rende conto di un fatto banale: chi scrive i manuali scolastici sono gli insegnanti, chi li usa sono sempre gli insegnanti. Dal momento che oggi ci sono gli strumenti, sia quelli elettronici, sia il sistema della stampa \textit{on demand}, che permettono di ``circuitare'' direttamente autori e fruitori, mi sono deciso a intraprendere la creazione di un manuale di matematica ``libero'', nel senso più ampio che oggi, nell'era delle tecnologie dell'informazione e della comunicazione, si riesce a dare a questo termine. Tuttavia, adottare ``ufficialmente'' un testo scolastico nella scuola italiana è un fatto semplice solo se si segue un percorso consolidato nel tempo, fatto più che altro di prassi e abitudini che non di leggi specifiche. Per rispondere a queste esigenze questo Manuale è fatto di Autori, Contenuti, Supporti e Dati legali.

\paragraph{Obiettivi} Il progetto ``\serie{}'' ha per obiettivo la realizzazione di manuali di matematica e geometria, per tutto il percorso scolastico e per ogni tipologia di scuola, scritti in forma collaborativa e con licenza \textit{Creative Commons}. Si propone, quindi, di abbattere i costi dell'istruzione, ridurre il peso dei libri, invogliare gli studenti a usare il testo e promuovere l'autoformazione per chi è fuori dai percorsi scolastici. Ha inoltre l'ambizione di avviare una sfida ``culturale'' più ampia in una scuola più democratica, più libera, dove ognuno possa accedere gratuitamente almeno alle risorse di base.

\paragraph{Autori} I manuali sono scritti in forma collaborativa da diverse decine di docenti delle relative discipline scientifiche, sulla base della loro esperienza reale di insegnamento nelle diverse scuole. Alla sua realizzazione hanno contribuito anche studenti e appassionati. Tutti hanno contribuito in maniera gratuita e libera.

\paragraph{Contenuti} \serie{} si presenta quindi come un \textit{work in progress} sempre aggiornato e migliorabile da parte di tutti, docenti e studenti. Può essere liberamente personalizzato da ciascun insegnante per adeguarlo alla scuola in cui insegna, al proprio modo di lavorare, alle esigenze dei suoi studenti. \`E pensato non tanto per lo studio della teoria, che rimane principalmente un compito dell'insegnante, quanto per fornire un'ampia scelta di esercizi da cui attingere per ``praticare'' l'apprendimento. Lo stile scelto è quello di raccontare la materia allo stesso modo in cui l'insegnante la racconta in classe di fronte agli studenti. Gli argomenti sono trattati secondo un approccio laboratoriale, senza distinguere eccessivamente tra teoria ed esercizi; teoria, esempi svolti, esercizi guidati, esercizi da svolgere vengono presentati come un tutt'uno.

\paragraph{Supporti}
\serie{} è scaricabile dal sito \url{http://www.matematicamente.it}. \`E disponile in formato elettronico PDF completamente gratuito; i sorgenti in \href{http://www.latex-project.org}{\LaTeX} sono liberi e disponibili sullo stesso sito. I diversi volumi che compongono l'opera possono essere stampati, fotocopiati in proprio o stampati in tipografia per le sole le parti che occorrono, in nessun caso ci sono diritti d'autore da pagare agli autori o all'editore. Il docente che vorrà sperimentare nuove forme d'uso può usarlo in formato elettronico su tablet, PC, netbook o più semplicemente PC portatili, può proiettarlo direttamente sulla lavagna interattiva (LIM) interagendo con il testo, svolgendo direttamente esempi ed esercizi, personalizzando con gli alunni definizioni ed enunciati; ricorrendo eventualmente a contenuti multimediali esterni presenti sui siti internet, confrontando definizioni e teoremi su \href{http://www.wikipedia.it}{Wikipedia}, cercando sull'enciclopedia libera notizie storiche sugli autori, ricorrendo eventualmente a contenuti multimediali esterni presenti sui siti internet (sul sito \url{http://www.matematicamente.it} sono disponibili gratuitamente test interattivi e alcune videolezioni). A casa lo studente potrà usare il testo per mezzo dello stesso dispositivo che ha utilizzato in classe (tablet, notebook) con le annotazioni e le modifiche fatte dall'insegnante, potrà svolgere gli esercizi sul computer o sul libro cartaceo, potrà scambiare file attraverso i \textit{social network} o i sistemi di messaggistica istantanea, particolarmente diffusi tra i ragazzi.

\paragraph{Quarta edizione} Modifiche sostanziali presenti in questa edizione: prima versione \LaTeX{} a cura di Daniele~Masini, revisione dei risultati di alcuni esercizi, aggiunta di alcuni esercizi, correzioni di refusi.

\paragraph{Dati legali} \serie{}, eccetto dove diversamente specificato, è rilasciato nei termini della licenza Creative Commons Attribuzione allo stesso modo 3.0 Italia (CC BY 3.0) il cui testo integrale è disponibile al sito~\url{http://creativecommons.org/licenses/by/3.0/deed.it}.

\begin{flushright}
Il coordinatore del progetto\\
prof. Antonio Bernardo.
\end{flushright}

\cleardoublepage
