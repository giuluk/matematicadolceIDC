% (c) 2015 Daniele Zambelli daniele.zambelli@gmail.com

\chapter{Modelli Economici}

La parte più discorsiva di questo capitolo è tratta da wikipedia:

\url{it.wikipedia.org/wiki/Economia}

\url{it.wikipedia.org/wiki/Microeconomia}

a cui si rimanda per la bibliografia e per i collegamenti di approfondimento.

\section{TODO}

% \section{Gli elementi di base dell'economia}
% \label{sec:modec_elementi_di_base}
\begin{comment}
 
Introduzione
  Micro economia
  Macro economia
  Variabili discrete
I mercati
  bene
  mercato
  consumatore
  produttore
  concorrenza perfetta
  monopolio
  oligopolio
  concorrenza monopolistica
Funzione di domanda e offerta
  leggi della domanda
    domanda
    offerta
    d=f(p) p>0 decrescente
      modello lineare
      modello parabolico
      modello esponenziale
      modello iperbolico
    esempi
    elasticità della domanda
      totalmente elastica
      elastica
      unitaria
      rigida
      totalmente rigida
  leggi dell'offerta
    r=f(p) p>0 crescente
    coefficiente di elasticità dell'offerta (positivo)
      elastica
      unitaria
      rigida
Legge di mercato
  Regime di concorrenza perfetta
    condizioni
      molti consumatori
      molti produttori
      libertà di acquisto e di vendita
      ogni operatore può entrare o uscire dal mercato 
      c'è trasparenza
      non ci sono coalizioni
      prodotti a larga diffusione
    prezzo di equilibrio
  Cambiamento del prezzo di equilibrio
    a parità di domanda
    a parità di offerta
    con cambiamento sia di domanda sia di offerta
Funzione di utilità
  paniere di consumo
    relazione tra panieri
      preferito
      indifferente
    proprietà
      riflessiva
      transitiva
      completezza
      continuità
      non sazietà
      stretta convessità
    funzione di utilità
      curva di livello o curve di indifferenza
      saggio marginale di sostituzione
        caratteristiche delle curve di indifferenza
          decrescenti
          concavità verso l'alto
          non si intersecano
          curve più alte->soddisfazione più elevata
        casi estremi
          sostituti perfetti
          complementari perfetti

\end{comment}

\section{Economia}
\label{sec:modelli_economia}

Per ``economia'' – dal greco %οἶκος
(\emph{oikos}), ``casa'' inteso anche come ``beni di famiglia'', e %νόμος 
(\emph{nomos}), ``norma'' o ``legge'' – si intende sia l'organizzazione 
dell'utilizzo di risorse scarse (limitate o finite) quando attuata al 
fine di soddisfare al meglio bisogni individuali o collettivi, sia 
un sistema di interazioni che garantisce un tale tipo di organizzazione, 
sistema detto anche \emph{sistema economico}.
% <ref>«\emph{Economics is the 
% science which studies human behaviour as a relationship between ends and 
% scarce 
% means which have alternative uses}»; la definizione, del~1932, è tratta da 
(\emph{Lionel Robbins, Essay on the Nature and Significance of Economic 
Science, Macmillan, London, 1945}
\url{http://mises.org/books/robbinsessay2.pdf})

Per l'economista e politico francese Raymond Barre
(\emph{Raymond Barre, Economie politique, Presses universitaires de France, 
1959}): 
\begin{quote}
 L'economia 
è la scienza della gestione delle risorse scarse. Essa prende in esame le forme 
assunte dal comportamento umano nella gestione di tali risorse; analizza e 
spiega le modalità secondo le quali un individuo o una società destinano mezzi 
limitati alla soddisfazione di esigenze molteplici ed illimitate.
\end{quote}


Per l'economista inglese Alfred Marshall:
(\emph{Alfred Marshall, Principi di Economia, 1890})
\begin{quotation}
 L'economia è uno studio del genere umano negli affari ordinari della vita. 
\end{quotation}

I soggetti che creano tali sistemi di organizzazione possono essere persone, 
organizzazioni o istituzioni. Normalmente si considerano i 
soggetti (detti anche "agenti" o "attori" o "operatori" economici) come attivi 
nell'ambito di un dato territorio; peraltro si tiene conto anche delle 
interazioni con altri soggetti attivi fuori dal territorio.

\section{Il sistema economico}
\label{sec:modelli_sistemaeconomico}

Il sistema economico, secondo la visione dell'economia di mercato della 
moderna società occidentale, è la rete di interdipendenze ed interconnessioni 
tra operatori o soggetti economici che svolgono le attività di produzione, 
consumo, scambio, lavoro, risparmio e investimento per soddisfare i bisogni 
individuali e realizzare il massimo profitto, ottimizzando l'uso delle risorse, 
evitando sprechi e aumentando la produttività individuale anche diminuendo il 
costo del lavoro.

\subsection{Componenti o sottosistemi}

I componenti o sottosistemi del sistema economico sono:

\begin{description}
 \item [Sistema di produzione], attraverso la produzione promuove e 
determina l'offerta di beni e servizi sotto 
continua spinta all'investimento per produrre innovazione 
(aziende e imprese).
 \item [Sistema dei consumatori], promuove e determina 
attraverso il consumo la domanda e offerta di beni e 
servizi (es. famiglie e in parte anche imprese).
 \item [Sistema creditizio-finanziario], da esso i 
precedenti sottosistemi afferiscono fondi di liquidità (capitali) e 
strumenti finanziari per 
promuovere e raggiungere i loro obiettivi (produzione e/o consumo) 
(banche e istituti di intermediazione finanziaria).
 \item [Mercato], è l'ambiente di interazione dei precedenti sottosistemi 
dove avviene lo scambio di beni, servizi e denaro tipicamente regolati 
dalla legge della domanda e dell'offerta.
 \item [Stato], alimenta il sistema economico attraverso la spesa 
pubblica (offerta di servizi pubblici a fronte di prelievo fiscale), 
regolandolo anche attraverso interventi mirati di politica economica 
(politica di bilancio e politica monetaria).
\end{description}

Il livello di sviluppo ed efficienza di tali sottosistemi e del relativo 
sistema economico riflette il livello di sviluppo della società stessa e varia 
in funzione delle epoche storiche o della parte del mondo o Stato considerato. 
Storicamente si passa da economie prettamente agricole ad economie 
agricole-industriali fino ad arrivare a economie 
agricole-industriali-terziarie, mentre attualmente e geograficamente si 
classifica l'efficienza dei sistemi economici con le denominazioni di primo 
mondo, secondo mondo, terzo mondo e quarto mondo. Il processo di 
globalizzazione sta gradualmente portando ad una progressiva 
omogeneizzazione dei vari sistemi economici mondiali grazie all'interdipendenza 
a livello internazionale dei vari mercati nazionali (internazionalizzazione).

\subsection{Operatori economici e loro funzioni}

% File:Trieste Assicurazioni Generali 04032007 01.jpg Le Assicurazioni 
% Generali a Trieste

Il sistema economico può definirsi, altresì, come l'ambiente o l'insieme delle 
attività promosse dagli operatori economici per le suddette finalità. Gli 
operatori economici svolgono una o più delle seguenti funzioni:
% <ref>La classificazione e le definizioni che seguono sono quelle usate in 
% ambito internazionale (secondo gli standard SNA delle Organizzazione delle 
% Nazioni Unite e Sistema europeo dei conti nazionali e regionali, SEC, da esso 
% derivato, dell'Unione europea) per l'analisi della struttura complessiva di 
un 
% sistema economico, di suoi aspetti specifici (ruolo dello Stato, sottosistemi 
% regionali ecc.), della sua evoluzione nel tempo, delle relazioni con altri 
% sistemi economici; in particolare, sono tratte da:
% \item {{en}} Eurostat, 
% [http://circa.europa.eu/irc/dsis/nfaccount/info/data/esa95/en/titelen.htm 
% European system of accounts ESA 1995];
% \item ISTAT, \emph{I conti degli italiani}, Bologna, Il Mulino, 2001.</ref>

\begin{itemize} [noitemsep]
\item produzione di beni e servizi;
\item consumo di beni e servizi;
\item intermediazione finanziaria;
\item accumulazione di ricchezza;
\item redistribuzione del reddito e della ricchezza;
\item assicurazione.
\end{itemize}

\subsubsection{Classificazione degli operatori}
Gli operatori economici vengono classificati secondo le funzioni svolte. Si 
hanno:

\begin{itemize} [noitemsep]
\item le famiglie, che consumano beni e servizi prodotti (prodotti nel 
territorio considerato, o importati, a cura di altri operatori, dal "resto del 
mondo"), ma possono anche produrre e accumulare (Impresa|imprese 
individuali, Azienda|aziende familiari);
\item le società che svolgono attività finalizzate al 
conseguimento di utili ed all'accumulazione:
  \begin{itemize} [noitemsep]
  \item le società di intermediazione finanziaria (in primo luogo le banche; 
  in Italia vi sono poi le Società di Intermediazione Mobiliare (SIM), le 
  Società di gestione del risparmio (SGR), le SICAV ecc.);
  \item le società di assicurazione;
  \item le società (dalle grandi società per azioni alle piccole società di 
  persone) che producono altri beni e servizi;
  \end{itemize}
\item la pubblica amministrazione, in tutte le sue articolazioni, che 
contribuisce al consumo (cosiddetti consumi collettivi), produce 
prevalentemente servizi non destinati alla vendita (istruzione, ordine 
pubblico, 
Ministero della difesa ecc.) e redistribuisce il reddito e la ricchezza 
tra gli operatori del sistema;
\item altre organizzazioni senza finalità di lucro, che erogano servizi a 
beneficio 
delle famiglie (partiti, sindacati dei 
lavoratori, organizzazioni religiose, associazioni culturali ricreative e 
sportive, enti di beneficenza ed assistenza).
\item Professionisti (avvocati, commercialisti, farmacisti...) che offrono 
servizi 
regolati da ordini professionali.
\end{itemize}

\subsubsection{Le operazioni economiche}

% File:Euro coins and banknotes.jpg Monete e banconote in euro.

Gli operatori interagiscono con operazioni economiche che possono 
essere:
\begin{description} [noitemsep]
 \item [operazioni su beni e servizi]: sono sia quelle che danno origine a beni 
e servizi mediante la produzione o l'importazione, sia quelle che ad essi 
danno destinazione (consumi intermedi o finali, 
investimenti, esportazioni);
\item [operazioni finanziarie]: consistono nell'acquisizione o cessione di 
attività finanziarie (acquisto di azioni o altri 
titoli, apertura di depositi, erogazione di prestiti ecc.);
\item [operazioni di distribuzione e redistribuzione del reddito e della 
ricchezza]: 
fanno sì che il valore aggiunto generato dall'attività produttiva venga sia 
distribuito fra i fattori della produzione (percezione del profitto e 
del reddito da lavoro autonomo, distribuzione di redditi da 
capitale da parte delle società, pagamento di Redditi di lavoro 
redditi da lavoro dipendente), sia redistribuito tra gli operatori 
(riscossione di imposte e tasse, erogazione di contributi).
\end{description}

Vi sono poi altre operazioni quali gli ammortamenti o lo 
scambio di attività non finanziarie non prodotte (terreni, brevetti, licenze).

Tutte le operazioni indicate costituiscono \emph{flussi}; vengono pertanto 
misurate tenendo conto delle variazioni (creazione, trasformazione, scambio, 
trasferimento di valore) che intervengono in un dato periodo di tempo. Ad 
esempio, si misura l'insieme delle vendite effettuate da una società, oppure 
l'insieme delle imposte percepite dalla pubblica amministrazione, nel corso di 
un anno.

Le operazioni possono avere o non avere una contropartita. Nel primo caso (ad 
esempio, la vendita di un bene), ad un flusso di denaro o in natura corrisponde 
un flusso di beni o servizi di pari valore; nel secondo caso (ad esempio, 
l'erogazione delle pensioni) non vi è una diretta contropartita e si parla 
di operazioni unilaterali o trasferimenti.

\subsection{I settori economici}

Le diverse attività di produzione di beni e servizi vengono ripartite in 
settori economici.

Al livello più generale si usa la tradizionale distinzione tra:
\begin{description} [noitemsep]
\item [settore primario], che comprende l'agricoltura, la selvicoltura, 
la pesca, lo sfruttamento delle cave e delle miniere;
\item [settore secondario], che comprende l'industria in senso stretto, 
l'edilizia e l'artigianato;
\item [settore terziario], che produce e fornisce servizi.
\end{description}

Vengono attualmente utilizzate, tuttavia, classificazioni più articolate:
\begin{itemize} [noitemsep]
\item l'ESCAP delle Organizzazione delle Nazioni Unite 
propone una classificazione che individua 20 settori 
economici.
\url{http://www.unescap.org/publications/accsectors.asp|titolo=Sett
ori economici ESCAP}
\item la Divisione Statistica delle Organizzazione delle Nazioni 
Unite usa l'ISIC (International Standard Industrial Classification of All 
Economic Activities), che individua 21 settori (detti "sezioni");
\item l'Eurostat, organo statistico della Commissione europea, usa la 
classificazione NACE, derivata dall'ISIC;
\item in Italia, l'ISTAT adotta la classificazione ATECO, traduzione 
italiana del NACE.
\end{itemize}

\subsection{La ricchezza di un sistema economico}

% File:Palazzo della banca d'Italia (firenze) 03.JPG Palazzo della 
% Banca d'Italia, Firenze

Gli operatori che svolgono la funzione di accumulazione danno luogo a 
variazioni delle attività del sistema. Altre variazioni possono 
manifestarsi indipendentemente dalla loro volontà (incendi, catastrofi 
naturali, ecc.).

Le attività si dividono in non finanziarie e finanziarie. Tra le 
prime rientrano:
\begin{description} [noitemsep]
\item [attività fisse materiali]: terreni, abitazioni, macchine e impianti, 
mezzi di trasporto, giacimenti minerari ecc.;
\item [attività fisse immateriali]: opere artistiche, software, brevetti ecc.;
\item [scorte di materie prime], prodotti in corso di lavorazione, prodotti 
finiti;
\item [oggetti di valore]: pietre e metalli preziosi, oggetti di antiquariato 
ecc.
\item [attività finanziarie]: vi sono monete, depositi, azioni ed altri titoli 
ecc.

\end{description}
La misurazione delle attività ad una certa data consente di determinare la 
ricchezza, a quella data, di un sistema economico (si tratta di uno 
\emph{stock}, non di un \emph{flusso}).

\section{Tipi di sistemi economici}
Si possono individuare diversi tipi di sistemi economici, sulla base della 
presenza di tutti, o solo di alcuni, degli operatori sopra indicati, della 
maggiore importanza di alcuni rispetto ad altri, di diverse modalità di 
esplicazione delle loro funzioni, di diverse regole per l'esecuzione delle 
operazioni. Su tali aspetti influiscono 
le Istituzione|istituzioni politiche e sociali, le tecnologie disponibili, 
aspetti culturali e ideologici.

Nel corso della storia si sono susseguiti diversi sistemi economici, mentre 
altri sono stati solo ideati e mai realizzati.

\subsection{Sistemi economici nella storia}

\subsubsection{Antichità}

% File:Port of Piraeus Panoramic View.JPG Il porto del Pireo oggi

Vi è stata una grande varietà di sistemi economici nell'antichità. In generale 
si può dire che, per millenni, hanno dominato l'agricoltura, finalizzata 
prevalentemente all'autoconsumo, ed il commercio lungo vie d'acqua anche con 
terre lontane. Si faceva inoltre ampio ricorso alla schiavitù.

% I Sumeri erano divisi in varie città-stato indipendenti, spesso in 
% conflitto tra loro per il controllo di canali che 
% delimitavano i territori e consentivano di irrigare i terreni drenando le acque 
% in eccesso e trasportandole alle zone più lontane. Nelle città avevano grande 
% importanza i templi, sia come luoghi di culto che come sedi di 
% raccolta e di redistribuzione delle eccedenze agricole.
% 
% Presso i Babilonesi il re era anche il maggiore 
% proprietario terriero e le sue terre erano coltivate dagli schiavi. Il codice 
% di Hammurabi ci rivela che vi erano tre classi sociali: uomini liberi, che 
% potevano essere proprietari terrieri ma anche medici, commercianti o artigiani; 
% uomini semiliberi, senza possedimenti, e schiavi. Erano anche stati definiti 
% contratti per molte operazioni economiche: baratto, 
% compravendita, prestito, donazione, deposito, pegno, 
% assunzione di lavoratori al momento del raccolto.
% 
% % File:Poikile quadriportico Villa Adriana.jpg Villa Adriana a Tivoli
% 
% In Grecia coesistevano diversi sistemi economici. A Sparta la 
% popolazione era divisa in tre gruppi: gli spartiati erano i soli cittadini a 
% pieno titolo ed erano tenuti a curare l'addestramento militare ed a dotarsi di 
% armi pesanti; i perieci erano liberi, curavano il commercio e l'artigianato, ma 
% erano obbligati a pagare tributi senza godere di alcun diritto politico; gli 
% iloti erano schiavi di proprietà dello Stato, come la terra. Lo Stato affidava 
% agli spartiati sia appezzamenti di terra, sia iloti per lavorarla. L'economia 
% spartana aveva quindi come fulcro la coltivazione di terre conquistate grazie 
% alla guerra. Storia di Atene, invece, cercò la propria espansione 
% economica nel commercio marittimo, soprattutto con Pisistrato, che favorì 
% la crescita di una classe di commercianti, e con Pericle, che usò i tributi 
% per collegare la città al porto del Pireo e per incrementare la flotta 
% mercantile.
% 
% Roma privilegiò l'espansione territoriale, quindi l'agricoltura, fin 
% dall'origine. Si possono distinguere due fasi: all'inizio prevalevano i piccoli 
% e medi proprietari terrieri, che costituivano anche il nerbo dell'esercito; 
% successivamente prevalse il latifondo e si dovette creare un esercito di 
% mercenari. Il cambiamento fu indotto dalla crisi economica successiva alla 
% seconda guerra punica, che rovinò molti proprietari terrieri; ne seguirono 
% anche la crisi della repubblica e, dopo lotte interne durate due secoli, la 
% nascita dell'impero. Il latifondo dette gradualmente vita all'"economia delle 
% ville romane", centri di produzione agricola sempre più ampi e 
% sontuosi.
% 
% Sia ad Atene che a Roma venne dato grande impulso alle opere pubbliche.

\subsubsection{Medioevo}

% File:Rolandfealty.jpg Carlo Magno investe Rolando e gli consegna Durlindana
% Si distinguono due fasi principali: Alto e Basso Medioevo.

Nell'Alto Medioevo si diffuse in un primo tempo l'economia curtense. 
Derivate dalle ville romane, le corti 
costituivano unità produttive autosufficienti, in cui il commercio aveva un 
ruolo limitato e gli scambi avvenivano spesso in natura. 

% Si distinguevano in esse una \emph{pars dominica}, gestita direttamente dal 
% "signore", ed una \emph{pars massaricia}, gestita da contadini, liberi o 
% asserviti, che avevano comunque l'obbligo di versare al signore un terzo del 
% prodotto e di svolgere alcune giornate lavorative gratuite sulla \emph{pars 
% dominica} (corvée).

Con l'affermazione dell'Impero Carolingio, l'economia curtense si trasformò 
in economia feudale. 
% In un primo tempo le terre appartenevano 
% all'imperatore, che ne assegnava in comodato parti, dette feudi, 
% a persone di sua fiducia dette vassalli. Questi ne curavano 
% l'amministrazione e potevano a loro volta assegnarne parti ai valvassori; i 
% vassalli riuscirono presto ad ottenere anche il diritto di trasmettere il feudo 
% ai loro eredi.
% 
% Vi erano poi i servi della gleba, che erano obbligati a 
% coltivare le terre padronali, dalle quali non potevano allontanarsi per 
% trasferirsi altrove; potevano coltivare nel tempo libero le terre dette 
% "servili", riconoscendo peraltro un'imposta detta decima al clero.

Nel Basso Medioevo si ebbero graduali ma significativi progressi sia 
nell'agricoltura che nei commerci. Nell'Europa settentrionale iniziarono a 
diffondersi la rotazione triennale e l'uso dell'aratro pesante, 
che consentirono aumenti delle rese agricole e, con ciò, la disponibilità di 
maggiori eccedenze da dedicare al commercio. Lo sviluppo del commercio favorì, 
a sua volta, la nascita e la crescente importanza delle città.

In Italia le città acquisirono importanza tale da costituirsi in comuni 
(trasformatisi poi in signorie) e, in qualche caso, in 
repubbliche marinare. 

% Tra le città italiane più importanti si possono ricordare:
% 
% \begin{description} [nosep]
% \item [Venezia], che aveva acquistato, con la diplomazia e con 
% la guerra, il dominio di quei pochi territori dell'entroterra necessari ai 
% traffici e utili per l'incremento delle entrate governative, ma curava 
% soprattutto l'espansione commerciale via mare;
% \item [Milano], che curava soprattutto l'agricoltura e 
% l'allevamento del bestiame, le lavorazioni artigianali dei metalli e dei 
% tessuti (sotto gli Sforza si svilupparono la coltivazione del gelso e la 
% lavorazione della seta) ed il commercio interno, grazie anche ad una rete di 
% canali che penetravano dentro la città; perseguì quindi l'espansione 
% territoriale, ottenendo sotto i Visconti il controllo di buona parte 
% dell'Italia 
% centrosettentrionale;
% \item [Firenze], che sviluppò notevolmente, fin dal XII 
% secolo, sia l'artigianato che il commercio internazionale, tanto da essere 
% definita la \emph{Wall street} del medioevo. I traffici internazionali si 
% giovavano della valle dell'Arno e della Via Francigena che, 
% collegando Roma e Canterbury, costituiva una delle più importanti vie 
% di comunicazione europee in epoca medioevale. I mercanti fiorentini si 
% inserirono presto nel circuito degli scambi europei: importavano l'allume 
% dal Levante e panni semilavorati dalle Fiandre e 
% dalla Francia; raffinavano quindi i tessuti ottenendone preziose stoffe che 
% esportavano con notevoli guadagni. L'esigenza di mezzi di pagamento idonei al 
% commercio internazionale favorì, a sua volta, una forte crescita del 
% sistema bancario (i Medici erano banchieri). Nel corso del XV 
% secolo Firenze da sola aveva un reddito superiore a quello dell'intera 
% Inghilterra, grazie alle industrie e alle grandi banche fiorentine, circa 
% ottanta tra sedi e filiali, queste ultime sparse in buona parte d'Europa.
% \end{description}

Nel resto d'Europa si formarono invece fin dal XIII secolo i primi 
Stati nazionali, che furono poi i protagonisti dell'età 
moderna.

\subsubsection{Età moderna}

% File:Columbus Taking Possession.jpg Cristoforo Colombo
% sbarcato nel Nuovo Mondo

L'età moderna è caratterizzata, in estrema sintesi, 
dall'espansione territoriale nelle regioni rese accessibili dalle scoperte 
geografiche, dallo sviluppo del commercio marittimo internazionale, dalla 
progressiva affermazione degli Stati nazionali come Stati 
assoluti, dall'affermazione di una aristocrazia fondiaria e di un ceto 
borghese dedito al commercio ed alla finanza.

% L'Impero portoghese privilegiò la ricerca di rotte per raggiungere 
% l'India, da cui provenivano le spezie importate in Europa, con l'obiettivo 
% commerciale di scavalcare l'intermediazione araba ed il monopolio commerciale 
% di Storia di Venezia. L'Impero spagnolo preferì invece la 
% conquista territoriale e lo sfruttamento agricolo e minerario dell'America 
% meridionale.
% 
% L'Inghilterra e i Paesi Bassi riuscirono poi a conquistare gradualmente 
% le basi portoghesi dal Capo di Buona Speranza all'Oceano pacifico, 
% affermandosi a loro volta come potenze commerciali. Nel XVII secolo, 
% Amsterdam divenne il porto più importante del mondo e un centro di finanza 
% internazionale. Successivamente, le guerre contro l'Inghilterra e la 
% Francia indebolirono i Paesi Bassi a favore dell'Inghilterra. Qui 
% la Gloriosa rivoluzione portò ad una forma di monarchia costituzionale 
% basata sull'equilibrio tra il sovrano, i proprietari terrieri e la borghesia, 
% nella quale venivano disciplinati i modi di finanziamento dello Stato sia 
% attraverso i tributi (che dovevano essere approvati dal Parlamento), sia 
% attraverso il debito pubblico (la Banca d'Inghilterra, una delle prime 
% banche centrali, venne fondata nel 1694).
% 
% I Paesi Bassi, poi imitati dall'Inghilterra, furono anche la culla 
% della prima rivoluzione agricola. 
% Nei Paesi Bassi l'agricoltura veniva finalizzata prevalentemente alle esigenze 
% del commercio (lino per le tele, coloranti per il panno, ecc.), mentre 
% l'Inghilterra dette grande impulso alla coltivazione dei cereali, 
% all'allevamento del bestiame ed alla produzione della lana e della seta.

\subsubsection{Età contemporanea}

% File:12 (236012210).jpg Una catena di montaggio

L'età contemporanea inizia, da un punto di vista economico, con la rivoluzione 
industriale: un processo di evoluzione che da un'economia 
agricola-artigianale-commerciale 
portò ad un'economia industriale moderna, caratterizzata dall'uso 
generalizzato di macchine azionate da energia meccanica e 
dall'utilizzo di nuove fonti energetiche inanimate (in 
primo luogo i combustibili fossili).

Ne sono seguiti il progressivo declino dell'agricoltura (il numero degli 
occupati nel settore agricolo iniziò a diminuire costantemente dopo la Grande 
depressione del 1873-1895, detta \emph{Long Depression}) e, con esso, quello 
dell'aristocrazia, la crescente importanza della borghesia produttiva, lo 
sviluppo sostenuto delle città, l'estensione della produzione per il mercato e 
la tendenziale scomparsa di quella per l'autoconsumo, la nascita di un mercato 
del lavoro.

Attraverso grandi momenti di crisi economica (la \emph{Long Depression} e il 
crollo di Wall Street del 1929) e politica (la 
Prima guerra mondiale, la Rivoluzione russa, la Repubblica di 
Weimar), si sono affermati nel XX secolo tre diversi sistemi economici:
\begin{description} [noitemsep]
\item [l'economia di mercato]: è basata sull'interazione degli operatori 
economici privati, con un ruolo limitato dello Stato (ordine pubblico, 
difesa, giustizia, istruzione, costruzione di infrastrutture);
\item [l'economia pianificata]: in essa la gestione delle dinamiche del sistema 
economico compete allo Stato, che elabora piani di breve-media durata che 
stabiliscono gli obiettivi e regolano conseguentemente l'impiego delle risorse;
\item [l'economia mista]: accanto all'interazione degli operatori privati, lo 
interviene direttamente nel funzionamento del sistema economico, a sostegno 
della produzione e dell'occupazione, utilizzando la spesa pubblica ed 
avvalendosi di politiche fiscali e monetarie.
\end{description}

Nelle economie moderne il motore della crescita economica spesso è stato 
rappresentato dall'innovazione tecnologica: questa componente è stata infatti in 
grado di generare un effetto a catena/valanga sulle altre variabili 
macroeconomiche con conseguenziale aumento dei consumi, della produttività (PIL) 
e dell'occupazione. 
Fondamentale per la creazione di innovazione sotto forma di ricerca e 
sviluppo è ed è stato anche l'accesso al credito degli istituti di 
credito da parte delle imprese per la promozione dei loro 
investimenti, cioè l'interazione forte tra i sottosistemi di 
produzione e consumo e il sistema creditizio-finanziario all'interno del 
sistema economico stesso.

\subsection{Sistemi economici ideati e mai realizzati pienamente}

Aspetti economici possono ravvisarsi in molte utopie. Nel XX secolo 
vi sono stati, peraltro, sistemi economici "ideali" che sono stati assunti come 
obiettivo da partiti politici:
\begin{description} [noitemsep]
\item [il comunismo], caratterizzato dall'abolizione della proprietà privata, 
dalla proprietà collettiva dei mezzi di produzione ed ispirato al motto "da 
ciascuno secondo le sue capacità, a ciascuno secondo le sue necessità";
\item [il fascismo], basato anch'essa sulla proprietà collettiva dei mezzi di 
produzione, ma nell'ambito dello Stato corporativo.
\end{description}

Oltre questi sistemi economici ne esiste un altro, diverso da essi perché 
apolitico: è il Venus Project, ideato da Jacque 
Fresco, basato sull'abbondanza delle risorse attraverso l'utilizzo della 
tecnologia odierna.

Un altro sistema economico apolitico è quello fondato sul modello di 
Ayres-Warr (base della green economy), simile alla teoria 
"dell'astronave" ove la terra è considerata un sistema chiuso, come una 
grande nave, la cui somma delle risorse non è infinita e in cui occorre quindi 
fare attenzione al rapporto tra lo sfruttamento delle risorse del territorio e 
le esigenze dell'umanità. In questo modello il saldo 
entropico viene escluso dalle convenzionali esternalità negative 
dell'economia neoclassica, perché fondate sulla fisica newtoniana.

\section{Studio dei sistemi economici}

% File:GDP PPP per capita 2007 IMF.pngStati per PIL (PPA) pro 
% Paesi in base al PIL (PPA) pro capite del 2007

L'Economia politica studia i sistemi economici per individuarne le leggi di 
funzionamento. L'economia politica in senso moderno nasce quando si afferma 
la separazione tra etica e politica e ci si pone espressamente il problema 
della potenza economica degli Stati. Per lungo tempo tale disciplina 
si è occupata prevalentemente di sistemi economici nazionali;
% <ref>Cfr. il 
% Mercantilismo, l'Indagine sulla natura e le cause della ricchezza delle 
% nazioni} di Adam Smith, il \emph{Sistema nazionale di economia politica} 
% di Friedrich List ecc.</ref> 
i suoi concetti e metodi si sono tuttavia 
progressivamente estesi allo studio sia di sistemi sociali di ogni genere 
(economia aziendale), sia di singoli settori economici (economia 
agraria, economia industriale ecc.).

La Statistica economica ha invece come obiettivo la misurazione degli 
aspetti quantitativi di un'economia, dalla misura di grandezze semplici e di 
loro aggregati, all'analisi della dinamica e alle previsioni economiche, alla 
stima e alla verifica di modelli di comportamenti economici. Ad esempio, lo 
stato di un'economia nazionale viene rilevato mediante la contabilità economica 
nazionale (in Europa si usa il sistema di conti detto Sec95).

La Storia economica tenta di ricostruire il funzionamento di sistemi 
economici del passato, avvalendosi sia dei concetti dell'economia politica 
che dei metodi della statistica economica.

A partire dalla conoscenza o analisi del sistema economico è possibile agire 
sul sistema economico stesso con misure o interventi di politica economica 
mirati a stimolarne la stabilità o la crescita economica.

La Filosofia dell'economia è una branca della filosofia che studia le 
questioni relative all'economia o, in alternativa, il settore dell'economia che 
si occupa delle proprie fondamenta e del proprio \emph{status} di scienza 
umana.

\section{Microeconomia}
\label{sec:modelli_microeconomia}

La \emph{microeconomia} è quella branca della teoria economica che studia il 
comportamento dei singoli agenti economici, o sistemi con un numero limitato di 
agenti, che operano in condizioni di\emph{scarsità di risorse}. Assieme alla 
\emph{macroeconomia}, che studia sistemi a livello aggregato, costituisce la 
macro-categoria in cui si possono raggruppare tutte le discipline legate 
all'economia politica.

\subsection{Differenze con la macroeconomia}

La macroeconomia si 
occupa delle grandezze economiche cosiddette ''aggregate``, come, per esempio, 
il livello e il tasso di crescita del prodotto nazionale, i tassi di 
interesse, la disoccupazione e l'inflazione, le quali dipendono in 
qualche modo dalla ''somma`` delle grandezze microeconomiche ovvero dai 
comportamenti microeconomici globali dei consumatori. La filosofia di fondo è 
dunque quella del riduzionismo classico: il sistema economico globale è 
descritto a partire dalla somma delle azioni o comportamenti dei singoli 
consumatori.

Il confine tra la microeconomia e la macroeconomia è diventato negli ultimi 
anni sempre meno netto. Il motivo principale è dovuto al fatto che anche la 
macroeconomia ha a che fare con l'analisi dei mercati. Per capire come 
funzionano, infatti, è necessario comprendere prima di tutto il comportamento 
dei singoli operatori che costituiscono questi mercati. Quindi i 
macroeconomisti sono diventati sempre più attenti ai 
fondamenti microeconomici dei fenomeni economici aggregati.

\subsection{L'uso e i limiti della teoria microeconomica}

Come ogni scienza, l'economia si occupa della \emph{spiegazione} e della 
\emph{previsione} dei fenomeni osservati. La spiegazione e la previsione sono 
fondate su \emph{teorie}, le quali servono a spiegare i fenomeni osservati, in 
termini di un insieme di regole e di ipotesi di base. 
La teoria 
dell'impresa, per esempio, nasce da una semplice ipotesi: le imprese cercano 
di massimizzare il profitto (anche se in alcuni particolari mercati non è 
sempre così: per esempio secondo la teoria di Baumol in mercati monopolistici 
le imprese potrebbero perseguire il fine di massimizzare i ricavi totali, 
mantenendo il pareggio di bilancio del profitto come semplice vincolo sotto cui 
non eccedere). La teoria utilizza questa ipotesi per spiegare come le imprese 
scelgono l'ammontare di forza lavoro, di capitale e di materie prime da usare 
per la produzione, così come le quantità di beni da produrre. Questa teoria 
serve anche a spiegare in che modo queste scelte dipendono dai \emph{prezzi} 
dei fattori produttivi e qual è il prezzo che le imprese sono in grado di 
ottenere per i loro prodotti.

Le teorie economiche servono anche da presupposto per fare previsioni. Quindi, 
la teoria dell'impresa ci dice se il livello di produzione di un'impresa 
aumenterà o diminuirà in seguito ad un aumento dei salari o a una diminuzione 
del prezzo delle materie prime. Utilizzando tecniche statistiche ed 
econometriche, la teoria può dunque essere usata per costruire \emph{modelli}, 
sui quali poi basare previsioni di tipo quantitativo. Un modello è una 
rappresentazione di tipo matematico, fondato sulla teoria economica di 
un'impresa, di un mercato o di qualche altro tipo di entità economica.

Nessuna teoria è perfettamente corretta. Ognuna parte da assunzioni di base o 
da approssimazioni più o meno ragionevoli o realistiche della realtà. L'utilità 
e la validità di una teoria dipendono dalla capacità che essa ha di spiegare e 
prevedere l'insieme dei fenomeni reali che si vogliono studiare. Dato questo 
obiettivo, le teorie sono continuamente messe a confronto (testate) con le 
osservazioni della realtà; in seguito a questo confronto, esse sono spesso 
soggette a modifica e riformulazione, e a volte anche al rigetto. Il processo 
di verifica e riformulazione è di primaria importanza per lo sviluppo 
dell'economia come scienza. Per valutare una teoria, è importante tenere 
presente che essa è necessariamente imperfetta.

\subsection{Analisi positiva e analisi normativa}

Le teorie nascono per spiegare i fenomeni, vengono confrontate con 
l'osservazione e sono utilizzate per costruire modelli su cui basare le 
previsioni. L'uso della teoria economica per formulare previsioni è importante 
sia per i manager delle imprese sia per le politiche economiche pubbliche. 

La microeconomia dà risposta a diversi interrogativi siano essi di natura 
\emph{positiva} o di natura \emph{normativa}. Gli interrogativi di natura 
''positiva`` hanno a che fare con la spiegazione e la previsione, mentre le 
questioni di natura ''normativa`` riguardano ciò che dovrebbe essere.

A volte si vuole andare oltre la spiegazione e la previsione per porsi domande 
del tipo: <<Che cosa sarebbe meglio fare?>>. 
È questo il campo dell'analisi \emph{normativa}, anch'essa importante sia per i 
manager d'impresa sia per coloro che devono prendere decisioni di politica 
economica. 
L'analisi normativa non si occupa soltanto delle diverse opzioni di politica 
economica, ma riguarda anche l'implementazione delle politiche prescelte. Questa 
analisi è spesso accompagnata da giudizi di valore. Ogni volta che sono 
necessari giudizi di valore, la microeconomia non è in grado di dirci quale sia 
la soluzione migliore, ma può chiarire i vari trade-off (scelte alternative) e 
aiutare quindi a individuare i problemi e a mettere a fuoco i termini della 
questione.

\section{Teoria del consumatore}
\label{sec:modelli_teoria_consumatore}


% * [[Teoria del consumatore]]: legge di [[Hermann Heinrich Gossen|Gossen]], 
% [[utilità (economia)]], [[utilità marginale]], [[curve di indifferenza]], 
% funzioni di domanda, [[elasticità (economia)|elasticità]], [[surplus del 
% consumatore]], [[teoria della preferenza rivelata]], [[indice del costo della 
% vita]], [[Teoria dell'utilità attesa]], [[paradosso di Allais]]

La teoria del comportamento del consumatore si fonda su un modello razionale di 
scelta che si può riassumere dicendo che fra tutte le alternative possibili il 
consumatore sceglie quella che egli ritiene migliore. 
La teoria neoclassica del 
% consumatore trae la sua origine dagli scritti degli autori marginalisti, in 
% particolare [[Hermann Heinrich Gossen]], [[Léon Walras]], [[Francis Ysidro 
% Edgeworth]] e [[Vilfredo Pareto]].
I due pilastri di questa teoria sono il \emph{vincolo di bilancio} e le 
\emph{preferenze}.

\begin{description}
 \item {Il vincolo di bilancio}
Il consumatore dispone di una certa somma (il suo reddito o le sue risorse) per 
acquistare dei beni o dei servizi. Il prezzo di questi beni è fisso. Il vincolo 
di bilancio ci dice che la somma spesa per l'acquisto di questi beni non deve 
essere superiore al reddito disponibile. Se si fa l'ipotesi di non sazietà, 
allora tutto il reddito sarà speso e, nel caso di due beni, il vincolo di 
bilancio può essere rappresentato graficamente da una retta con una pendenza 
negativa. 
 \item [Le preferenze]
Le preferenze del consumatore sono espresse da una funzione di utilità 
quasi-concava (curve di indifferenza convesse). 
Graficamente e nel caso di due beni si utilizza il medesimo metodo delle carte 
geografiche o meteorologiche. Si prende un valore dell'utilità e si costruisce 
una curva di indifferenza. La pendenza di questa curva è chiamata il saggio 
marginale di sostituzione poiché esprime quante unità del secondo bene devono 
essere sostituite con un'unità del primo bene allo scopo di avere la medesima 
utilità.

% I primi autori della teoria del consumatore pensavano che l'utilità poteva 
% essere misurata, come la temperatura. Si parla allora di utilità cardinale.
% In seguito ci si rese conto che ciò non era possibile e d'altronde non era 
% neanche necessario. Basta un concetto ordinale come quello espresso dalle 
% curve 
% d'indifferenza.

Paul Samuelson ha proposto di dedurre le preferenze osservando il 
consumatore mentre fa gli acquisti. La sua teoria della preferenza rivelata 
permette una verifica operazionale del modello del consumatore.
\end{description}

% 
% \subsection{}
% 
% Nel caso di un ribasso di quantità o di un sussidio per l'acquisto delle 
% prime 
% unità di un bene, il vincolo di bilancio sarà più difficile da rappresentare 
% graficamente ma il principio di una barriera che non può essere sorpassata 
% resta 
% valevole. 
% 
% Questo modello statico può essere generalizzato introducendo diversi periodi. 
% In 
% questo caso il consumatore può risparmiare in un periodo per spendere di più 
% in 
% un altro o il contrario.

\subsection{Teoria delle scelte}

Siccome la teoria del consumatore serve a spiegare le sue scelte (i beni 
acquistati), si può costruire una teoria delle scelte senza passare per la 
funzione d'utilità o le curve di indifferenza. Prendiamo dei complessi o panieri 
di beni \(\mathbf{A}\), \(\mathbf{B}\), \(\mathbf{C}\), eccetera. 
Un paniere può essere costituito, per 
esempio, di 2 kg di pane, 3 litri di vino, 1 giornale, eccetera. Esprimiamo le 
preferenze del consumatore utilizzando la relazione binaria \(\succcurlyeq\), 
per esempio \(\mathbf{A}\succcurlyeq\mathbf{B}\) 
(\(\mathbf{A}\) preferito o uguale a \(\mathbf{B}\), 
oppure \(\mathbf{B}\) almeno tanto buono quanto \(\mathbf{A}\)). 
Questa relazione è simile al segno 
matematico \(\ge\) (maggiore o uguale a). Supponiamo che questa 
relazione binaria soddisfi gli assiomi seguenti:

1) riflessività: \(\mathbf{A}\succcurlyeq\mathbf{A}\)

2) transitività: \(\mathbf{A}\succcurlyeq\mathbf{B}\) e 
\(\mathbf{B}\succcurlyeq\mathbf{C}\) implica 
\(\mathbf{A}\succcurlyeq\mathbf{C}\)

3) completezza: si ha \(\mathbf{A}\succcurlyeq\mathbf{B}\) oppure 
\(\mathbf{B}\succcurlyeq\mathbf{A}\) o i due casi (indifferenza)

Se queste condizioni sono soddisfatte abbiamo una relazione di ordine totale 
che può essere utilizzato per spiegare le scelte del consumatore. 
Basta però aggiungere l'assioma seguente (una condizione matematica):

4) continuità: \(\left \{ \mathbf{A} \in X | \mathbf{A} \succcurlyeq 
\mathbf{B} \right \}\) e \(\left \{ \mathbf{A} \in X | \mathbf{B} 
\succcurlyeq \mathbf{A} \right \}\) sono insiemi chiusi,
e allora esiste una funzione di utilità. 
Le preferenze che non possono essere espresse da una funzione di utilità sono 
dei casi speciali (per esempio l'ordine lessicografico).

\subsection{Equilibrio del consumatore}

Il consumatore sceglie il paniere di beni che preferisce, tenendo conto del 
reddito disponibile. Matematicamente, si tratta di massimizzare l'utilità sotto 
il vincolo di bilancio. Utilizzando il metodo di Lagrange, si può scrivere, 
nel caso di due beni \( x_1 \, , \, x_2 \):

\[ L = u(x_1,x_2) + \lambda ( y - p_1 x_1 - p_2 x_2). \, \]

dove \( p_1 \, , \, p_2 \) sono i prezzi e y il reddito disponibile.

Le derivate parziali (condizione di primo ordine) sono

\begin{align}
\frac{\partial L}{\partial x_1}       &= {\frac{\partial u}{\partial x_1}} -  
\lambda p_1 &&= 0, \\
\frac{\partial L}{\partial x_2}       &= {\frac{\partial u}{\partial x_2}} - 
\lambda p_2   &&= 0,  \\
\frac{\partial L}{\partial \lambda} &= y - p_1 x_1 - p_2 x_2       &&= 0. 
\end{align}

Eliminando \(\lambda\), si ottiene:

:\( \frac{\frac{\partial u}{\partial x_1}}{p_1} =  \frac{\frac{\partial 
u}{\partial x_2}}{p_2}= \lambda \)

L'utilità marginale (\( \frac{\partial u}{\partial x_1}\)), divisa 
per il prezzo, deve essere uguale per tutti i beni. Si tratta della seconda 
legge di Hermann Heinrich Gossen. Graficamente, il saggio marginale di 
sostituzione deve essere uguale al rapporto dei prezzi.

[[File:ConsumersOptimum.png|framed|center|Paniere ottimo del consumatore dati 
due beni x<sub>1</sub> e x<sub>2</sub> e reddito y]]

Se la curva di indifferenza è convessa, questa condizione garantisce un 
massimo di utilità. Una curva concava è poco probabile poiché allora il 
consumatore acquista un solo bene. Una soluzione ad angolo si presenta quando 
un consumatore non acquista un bene, anche se lo desidera, poiché costa troppo.

\subsection{La funzione di domanda}

La teoria del consumatore serve a spiegare la domanda di beni e servizi. 
Prendendo l'esempio di due beni, sviluppato qui sopra, si ottiene, risolvendo 
il sistema di equazioni delle condizioni di primo ordine:

\(x_1 = \phi_1 (p_1, p_2, y) \quad ; \quad x_2= \phi_2 (p_1,p_2,y)\)

La domanda dipende dunque dal prezzo di tutti i beni e dal reddito del 
consumatore. 
% Le condizioni di primo ordine ci dicono che non c'è illusione monetaria. 

Se il prezzo di tutti i beni e il reddito raddoppiano, la domanda non cambia. 
Per esempio, il passaggio dalla lira all'euro non doveva avere nessun effetto 
sulla domanda (tutto è diviso per 1936.27) se non ci fosse stato il problema 
degli arrotondamenti.

Gli effetti di un cambiamento dei prezzi o del reddito sono studiati 
utilizzando il concetto di elasticità della domanda. 
Designiamo con il simbolo \( \varepsilon_{ij}\) l'elasticità della domanda del 
bene i quando il prezzo j aumenta. 
Se la domanda del bene i è elastica (superiore all'unità in valore 
assoluto), allorquando il suo prezzo aumenta la spesa diminuisce e viceversa 
per una domanda inelastica.

L'elasticità della domanda quando il reddito aumenta è chiamata 
elasticità-reddito. Designiamo con il simbolo \( \eta_i \) questa 
elasticità. L'elasticità-reddito è superiore all'unità per i beni superiori, 
inferiore all'unità per i beni necessari (questi due tipi di beni sono chiamati 
beni normali) e infine negativa per i beni inferiori. La legge di Engel ci 
dice che i beni alimentari sono dei beni necessari.

\subsection{Effetto di sostituzione}

L'effetto di sostituzione è l'effetto osservato qualora vi siano modifiche nei 
prezzi relativi delle merci.

Il grafico sottostante mostra l'effetto di un aumento di prezzo per il bene Y. 
Se il prezzo di Y aumenta, il vincolo di bilancio farà perno da BC2 a BC1. 

Per massimizzare l'utilità, con la riduzione del vincolo di bilancio, BC1, il 
consumatore potrà riassegnare il reddito per raggiungere la curva di 
indifferenza più elevata disponibile, che è tangente alla BC1. Come mostrato 
nel diagramma qui di seguito, che la curva è I1, e quindi la quantità di bene 
acquistato Y si sposterà da Y2 di Y1, e l'importo del bene acquistato X per il 
passaggio da X2 a X1. L'effetto opposto si verifica se il prezzo di Y 
diminuisce causando il passaggio da BC2 per BC3, e I2 di I3. 

[[File:effetto sostituzione.JPG|400px|left]]


\subsection{Effetto di reddito}

L'effetto di reddito è il fenomeno osservabile mediante cambiamenti osservati 
in potere d'acquisto. Si rivela il cambiamento in termini di quantità richiesta 
promossa da un cambiamento in termini di reddito reale (utilità). Graficamente, 
fintanto che i prezzi rimangono costanti, la modifica del reddito creerà un 
parallelo spostamento del vincolo di bilancio. 
Aumentando il reddito si sposterà a destra il vincolo di bilancio, in quanto si 
possono acquistare maggiori quantità di entrambi i beni, e la diminuzione del 
reddito si sposterà a sinistra.

[[File:effetto reddito normale.JPG|400px|left]]

% ------------- ?
Eugen Slutsky ha utilizzato la teoria del consumatore per determinare il segno 
di una variazione del prezzo. 
Normalmente, 
quando il prezzo aumenta la domanda diminuisce (segno negativo). C'è però il 
caso dei beni inferiori che impedisce una risposta univoca. Slutsky ha allora 
diviso in due parti l'effetto di una variazione del prezzo. Un aumento del 
prezzo diminuisce il reddito reale del consumatore. Se si vuole ottenere 
l'effetto puro, bisogna eliminare questo effetto di reddito. Supponiamo allora 
che il consumatore sia compensato affinché possa ancora acquistare il medesimo 
paniere di beni. Slutsky ha mostrato che il consumatore non lo acquista ma 
diminuisce la domanda del bene che è diventato più caro. L'equazione ottenuta 
da Slutsky è:

\(\frac{\partial x_1}{\partial p_1} = \frac{\partial x_1}{\partial p_1} 
\quad \text{(se consumatore compensato)} \quad - \frac{\partial x_1}{\partial 
y} x_1 \)

Utilizzando l'elasticità-prezzo e l'elasticità-reddito si può scrivere:

\( \varepsilon_{11} = \xi_{11} - \omega_1 \eta_1  \)

Dove \( \xi_{11} \) è chiamata l'elasticità-prezzo pura (senza 
l'effetto di reddito) e \( \omega_1 \) è la parte del reddito 
consacrato al bene 1.

Il primo termine è sempre negativo. Il secondo è pure negativo per i beni 
normali. Se un bene è inferiore allora il secondo effetto è positivo e 
l'effetto totale è indeterminato. 
C'è quindi il caso teorico di un effetto di reddito che 
può più che compensare l'effetto di sostituzione è arrivare ad un effetto 
totale positivo. 
Questo bene è chiamato un bene Giffen dal nome dell'economista che 
avrebbe trovato questo caso durante la carestia irlandese degli anni 1740-1741. 
L'esistenza di beni Giffen è contestata. Possono essere considerati una 
curiosità teorica. In definitiva, la curva di domanda ha una pendenza 
negativa.

Gli effetti incrociati (variazione della quantità del bene 1 quando il prezzo 
del bene 2 aumenta) possono essere positivi o negativi. Nel caso generale, 
l'equazione di Slutsky diventa:

\( \varepsilon_{ij} = \xi_{ij} - \omega_j \eta_i \)

Un bene sostituto ha un'elasticità positiva (effetto lordo o non compensato se 
\( \varepsilon_{ij} > 0 \); 
effetto netto o compensato se \( \xi_{ij} > 0 \)). 
Un bene complemento ha un'elasticità negativa (effetto lordo o netto secondo le 
elasticità).
% ------------- ?
% 
% \subsection{Bibliografia}
% 
% * G. Debreu, Theory of Value, New Haven, 1959
% * H. Varian, Microeconomic Analysis, London, 1992
% * J.M. Perloff, Microeconomics, London, 2008
% * Eugen Slutsky “Sulla teoria del bilancio del consumatore”, Giornale degli 
% economisti, 1915, pp. 1-26
% 
% [[Categoria:Microeconomia]]
% 



% [[Immagine:Isoquants_2d.jpg|framed|right|Figura 1: Curve di indifferenza nel 
% caso di [[funzione di utilità Cobb-Douglas]] a due fattori]]
% 
% [[Immagine:Indifference_curves_3d.jpg|framed|right|Figura 2: Curve di 
% indifferenza nel caso di funzione di utilità Cobb-Douglas a tre fattori]]

La \emph{curva di indifferenza} in microeconomia è l'insieme dei beni che 
garantiscono al consumatore lo stesso livello di utilità. 
In termini formali, data una generica funzione di utilità del tipo:

\(\ U = f(x_1,x_2,\ldots,x_n)\)

dove ''U`` è  l'utilità e x<sub>i</sub> è il bene 
i-esimo, la curva di indifferenza è definita come:

\(I = \{(x_1,x_2,\ldots,x_n)\ |\ f(x_1,x_2,\ldots,x_n)= \bar U\}\)

Ad ogni livello di utilità corrisponde dunque una diversa curva di indifferenza.

Supponiamo per esempio che un consumatore acquisti delle combinazioni diverse 
di beni (o più verosimilmente di due panieri di beni). L'ottenimento di una 
quantità maggiore di un bene compensa la rinuncia ad alcune unità di un secondo 
bene.

La convessità della curva è dovuta alla \emph{sostituzione}. 
Se il bene è scarso avremo un effetto di sostituzione maggiore e l'utilità 
marginale sarà maggiore dell'utilità marginale del bene più scarso.

L'inclinazione in ogni punto della curva di indifferenza è il 
\emph{saggio marginale di sostituzione}, che misura il rapporto di scambio tra 
due beni tale da non far variare il livello di utilità, ed è quindi una misura 
della sostituibilità soggettiva tra beni.

La combinazione fra il vincolo di bilancio rilevante, determinato dal prezzo 
relativo dei beni e dalla ricchezza, e la famiglia di curve di indifferenza 
consente di risolvere il problema di massimizzazione vincolata dell'utilità del 
consumatore. In particolare, il paniere ottimale è quello in corrispondenza del 
quale, data qualsiasi coppia di beni, il saggio marginale di sostituzione 
eguaglia il loro prezzo relativo.    
Per il calcolo analitico del punto dove il consumatore massimizza la 
soddisfazione (punto di tangente tra il vincolo di bilancio e la curva di 
indifferenza), ricordando la condizione di tangenza \(\ y=(p_x/p_y)x\) 
e tenendo conto che essendo \(\ R=p_xx+p_yy\), risolvendo con qualche 
passaggio algebrico avremo che \(\ x=R/2p_x\) ed \(\ 
y=(p_x/p_y)R/2p_x\). Sostituendo i valori noti di \(\ R\), \(\ 
p_x\) e \(\ p_y\) troveremo l'esatto punto di tangenza. 


\subsection{Elasticità della domanda}

In microeconomia si parla di \emph{elasticità della domanda}, o 
sensibilità alla variazione dei prezzi, 
quando all'aumento dei prezzi corrisponde una riduzione della domanda e alla 
riduzione dei prezzi un aumento della domanda.
% risiede fondamentalmente in base al 
% principio: "''la domanda è elastica se aumenta e contemporaneamente i prezzi 
% diminuiscono, oppure, se diminuisce e i prezzi aumentano''".

L' elasticità della domanda rispetto ai prezzi venne elaborata 
dall'economista Léon Walras, e indica l'attesa variazione percentuale della 
domanda di un dato prodotto/servizio (quantità venduta Q) rispetto ad una 
variazione percentuale del prezzo dello stesso prodotto o di altri prodotti 
(elasticità incrociata):

\( e_p = \frac {\frac {\Delta Q} {Q} } { \frac {\Delta p} {p}}\)

Gli incrementi possono essere fatti tendere a zero, considerando variazioni 
infinitesime anziché variazioni finite, per usare le derivate e i relativi 
strumenti di calcolo:

\( e_p = \frac {\frac {\partial Q} {Q} } { \frac {\partial p} {p}}\)

Ogni bene differisce dall'altro per quanto riguarda l'elasticità, ossia la 
sensibilità alle variazioni del prezzo. L'elasticità della domanda dipende da 
numerosi fattori economici, anche se tende ad essere più elevata per i beni di 
lusso, per i quali sono disponibili beni sostitutivi. Vi sono diverse categorie 
di elasticità:

\begin{description}
 \item [domanda elastica rispetto al prezzo]
Quando una variazione del prezzo dell'1\% genera una variazione della 
quantità domandata superiore all'1\%.
 \item [domanda a elasticità unitaria]
Quando una variazione del prezzo dell'1\% genera una variazione della 
domanda dell'1\%.
 \item [domanda rigida rispetto al prezzo]
Quando una variazione del prezzo dell'1\% genera una variazione della 
quantità domandata inferiore all'1\%.
\end{description}

Se tale valore è negativo, la funzione di domanda, 
inclinata negativamente, è del tipo \(P = a -b \cdot Q\). La pendenza 
(o coefficiente angolare) di una retta è \(b = \frac {\partial P} {\partial Q}\) 
(oppure \(b = \frac {\partial Q} {\partial P}\), se 
si considera la funzione inversa \(Q = a - b \cdot P\)).

Per quanto riguarda la relazione tra l'elasticità ed il ricavo, dove il ricavo 
è dato dal prodotto fra quantità e prezzo, sappiamo che:
\begin{itemize}
 \item se la domanda è \emph{rigida} rispetto al prezzo, una diminuzione del 
prezzo riduce il ricavo totale.
 \item se la domanda è \emph{elastica} rispetto al prezzo, una diminuzione 
del prezzo aumenta il ricavo totale.
 \item se la domanda è a \emph{elasticità unitaria}, una diminuzione del 
prezzo non modifica il ricavo totale.
\end{itemize}

Lo stato è incentivato a tassare beni con curva di domanda anelastica.

È evidente che l'elasticità è strettamente legata alla funzione di domanda. 
Solitamente si usa per misurare la reattività del mercato rispetto al prezzo 
(\(Q\) è la variabile dipendente e \(P\) la leva sulla quale 
il produttore può agire).

L'elasticità può però essere usata anche in senso opposto, per misurare la 
variazione del prezzo in seguito ad un aumento dell'offerta: in questo caso si 
sostituisce a Q la funzione domanda \(Q = a - b \cdot P\). Tale calcolo 
può per esempio essere utile quando una nuova azienda entra in un settore, per 
cui le aziende già presenti, per favorire una rapida espulsione dal mercato del 
nuovo concorrente, aumentano l'offerta in modo da abbassare i prezzi, proprio 
nel momento in cui l'altro ha contratto dei debiti, anche a causa agli alti 
costi fissi, proprio dovuti per l'ingresso sul mercato.

L'\emph{elasticità incrociata} ha una notevole importanza in quanto misura la 
sostituibilità di beni succedanei o alternativi alla funzione 
prezzo. Per esempio burro contro margarina, oppure Mercedes classe E contro 
BMW serie 5.

L'elasticità parte dal presupposto che nessun soggetto economico ha un tale 
potere sul mercato da poter determinare contemporaneamente sia la quantità che 
il prezzo. Se uno è conosciuto, l'altro è determinato dal mercato e non è noto 
a priori, ma è una grandezza da misurare.

La funzione della domanda è in generale una curva, 
semplificata in una retta la cui equazione è un caso particolare 
di quella della curva. Se la domanda è approssimata con una retta, 
essendo una funzione lineare, è anche invertibile, e come detto si può 
anche esprimere la quantità in funzione del prezzo.

\subsection{Utilità marginale}

L'\emph{utilità marginale} di un bene è concetto cardine della teoria 
neoclassica del valore in economia ed è definibile come 
l'incremento del livello di utilità (\(\ \Delta U\)), ovvero della 
soddisfazione che un individuo trae dal consumo di un bene, ricollegabile ad 
aumenti marginali nel consumo del bene (\(\ \Delta x_i\)), dato e 
costante il consumo di tutti gli altri beni.

\begin{definizione}
Si dice utilità marginale la quantità di soddisfazione che fornisce ogni 
singola dose di un bene consumato.
\end{definizione}

In termini non formali, l'utilità marginale può definirsi come l'utilità 
apportata dall'ultima unità o dose consumata di un bene.

In modo più formale, data una funzione di utilità \(\ 
U(x_1,x_2,\ldots,x_n)\), una funzione cioè che 
lega il consumo di quantità date di beni e servizi al livello di utilità, 
l'utilità marginale del bene \(\ x_i\) è data dalla derivata 
parziale della funzione rispetto ad \(\ x_i\); in simboli:

:\(U_i = \frac{\partial U(x_1,x_2,\ldots,x_n)}{\partial x_i} (>0)\)

La legge del \emph{utilità marginale decrescente} afferma che all'aumentare del 
consumo di un bene, l'utilità marginale di quel bene diminuisce.
La \emph{condizione di equilibrio} afferma che ogni individuo effettua le 
proprie scelte di consumo in modo che ogni singolo bene fornisca le stesse 
utilità marginali per euro di spesa.
Il \emph{principio di utilità marginali uguali per euro di spesa per ciascun 
bene} afferma che la condizione essenziale per ottenere massimo 
soddisfacimento o utilità è la seguente: di fronte ai prezzi di mercato dei 
beni un consumatore con reddito dato ottiene il massimo soddisfacimento quando 
l'utilità marginale dell'ultimo euro speso per un bene è esattamente uguale 
all'utilità marginale dell'ultimo euro speso per qualsiasi altro bene.

\subsubsection{L'ipotesi di utilità marginale decrescente}

Al concetto di utilità marginale risulta strettamente collegato l'assunto di 
``utilità marginale decrescente''. In pratica si assume che l'utilità marginale 
di un bene diminuisca al crescere del livello assoluto di consumo del bene. 
Formalmente questo comporta assumere che: 
% <ref name="nota_dipendenza"> Va 
% comunque notato che, nell'ipotesi in cui il bene in questione generasse una 
% qualche forma di dipendenza, l'utilità marginale dovrebbe crescere al crescere 
% del livello assoluto di consumo del bene.
% </ref>

[[File:Erstes gossensches gesetz.png|thumb|upright=2|Figura 1:Funzione di 
utilità]]
:\(\ U_{ii} = \frac{\partial^2 U(x_1,x_2,\ldots,x_n)}{\partial x_i^2} < 0\) 

Queste due ipotesi implicano che la funzione di utilità sia monotona crescente 
e concava rispetto al consumo dei singoli beni.

Solitamente si assume anche che:

\(\lim_{x_i \rightarrow 0} \frac{\partial U}{\partial x_i} = + 
\infty\)

\(\lim_{x_i \rightarrow +\infty} \frac{\partial U}{\partial x_i} = 0\)

\subsection{Un esempio}

Per comprendere meglio i concetti esposti si può pensare all'atteggiamento che 
l'individuo medio potrebbe avere di fronte ad una crostata di fragole.

Il primo pezzo di torta sarebbe molto gradito, apportando un incremento \(\ 
\Delta u_1\). L'incremento di utilità che genererebbe un secondo pezzo di 
crostata, sebbene consistente, sarebbe sicuramente minore del primo (\(\ 
\Delta u_2 < \Delta u_1\)). L'incremento del terzo ancora minore e così 
via.

Nel caso della crostata di fragole è poi anche verosimile immaginare che vi 
sarà 
un punto in cui il nostro consumatore sarà "sazio".

Una volta raggiunto il \emph{punto di sazietà} eventuali altri incrementi del 
consumo del bene (il mangiare altri pezzi di torta) probabilmente apporteranno 
una \emph{disutilità}, diminuiranno cioè il livello di soddisfazione 
individuale.

In corrispondenza del punto di sazietà l'utilità marginale è nulla (il 
consumatore è indifferente se mangiare il pezzo di crostata oppure no) ed il 
suo livello di utilità è massimo.
% <ref name="nota_saziet"> Nell'ipotesi in cui si 
% assuma che la derivata prima si annulli solo all'infinito si esclude 
% l'esistenza di punti di sazietà.
% </ref>

% \subsubsection{Storia della nozione di utilità marginale in economia}
% 
% La nozione di utilità marginale e l'ipotesi di utilità marginale decrescente 
% erano già note nella prima metà del Settecento. Daniel Bernoulli ad esempio 
% le utilizzò nella risoluzione del famoso paradosso di San Pietroburgo. 
% 
% % Queste nozioni vennero anche utilizzate, sebbene in modo non formalizzato, tra 
% % gli altri, da [[Jeremy Bentham]] ([[1789]]-[[1802]]) e [[Nassau William 
% % Senior]] 
% % ([[1790]]-[[1864]]).
% 
% % Fu tuttavia l'ingegnere francese Jules Dupuit (1804-1866) il primo 
% % a collegare in modo chiaro il concetto di utilità marginale e l'ipotesi di 
% % utilità marginale decrescente con l'inclinazione negativa della funzione di 
% % domanda.
% % <ref>Va notato incidentalmente come in realtà ciò sia vero solo nel 
% % caso di ''funzioni d'utilità additive'', del tipo:
% % :\(\ U = \sum_i u_i(x_i)\).
% % Nel caso più generale, è possibile ottenere funzioni di domanda inclinate 
% % negativamente anche in casi di utilità marginale costante. Così, ad esempio, 
% % ipotizzando una [[funzione di utilità Cobb-Douglas]] del tipo:
% % :\(\ U = x_1 x_2\)
% % nonostante l'utilità marginale sia costante:
% % :\(\ \frac{\partial U}{\partial x_i} = x_j\)
% % La funzione di domanda che ne deriva è:
% % :\(\ x_i = \frac{w}{2 p_i}\)
% % dove ''w'' è la ricchezza e p<sub>i</sub> il prezzo del bene, ed in cui la 
% % quantità consumata dipende negativamente dal prezzo.</ref>
% 
% L'impostazione di Dupuit venne poi ulteriormente chiarita e formalizzata da 
% [[Hermann Heinrich Gossen]] ([[1854]]), che anticipò molta della 
% [[marginalismo|rivoluzione marginalista]], sebbene il suo lavoro sia stato del 
% tutto trascurato all'epoca.
% 
% La [[teoria del valore|teoria soggettiva del valore]] marginalista, centrata 
% sul 
% concetto di utilità marginale, si sviluppò quindi a partire dai contributi 
% indipendenti di [[William Stanley Jevons]], [[Carl Menger]] e [[Léon Walras]].
% 
% Quando [[Gabriel Cramer]] e [[Bernoulli]] introdussero il concetto di utilità 
% marginale decrescente, affrontarono il [[paradosso di San Pietroburgo]] 
% introducendo i [[fattori di rischio]] e incertezza fino ad allora sconosciuti, 
% e 
% offrendo un nuovo spunto riguardante una concezione quantitativa di utilità: 
% [[utilità attesa]].
% 
% L'ipotesi dell'utilità attesa di Bernoulli e di altri è stata ripresa da vari 
% pensatori del 20° secolo, tra i quali Ramsey<ref>Ramsey, Frank Plumpton; "Truth 
% and Probability" (PDF), Chapter VII in The Foundations of Mathematics and other 
% Logical Essays (1931).</ref> (1926), [[John von Neumann|Von Neumann]] e [[Oskar 
% Morgenstern|Morgenstern]]<ref>Von Neumann, John and Oskar Morgenstern; [[Theory 
% of Games and Economic Behavior]] (1944).</ref> (1944) che, attraverso una 
% formula matematica, garantivano la rappresentabilità della ''struttura delle 
% preferenze'', e Savage (1954) il quale, per la prima volta, affrontò il tema 
% della ''soggettività dell'utilità'' introducendo le situazioni in 
% incertezza.<ref>Savage, Leonard Jimmie: Foundations of Statistics (1954).</ref>
%  
% Una delle principali ragioni per cui i modelli di utilità attesa sono influenti 
% ancora oggi è che il rischio e l'incertezza sono stati riconosciuti come temi 
% centrali della [[Teoria economica|teoria economica contemporanea]]<ref>Diamond, 
% Peter, and Michael Rothschild, eds.: Uncertainty in Economics (1989). Academic 
% Press.</ref> poiché semplificano l'analisi delle decisioni rischiose, perché 
% ''l'utilità marginale decrescente'' implica l'[[avversione al 
% rischio]].<ref>Demange, Gabriel, and Guy Laroque: Finance and the Economics of 
% Uncertainty (2006), Ch. 3, pp. 71-72. Blackwell Publishing.</ref>

% \subsection{Utilità}
% 
% A seconda di quale teoria di utilità viene utilizzata, l'interpretazione di 
% utilità marginale può essere più o meno significativa.
%  
% Gli economisti hanno comunemente descritto l'utilità come se fosse qualcosa di 
% quantificabile, cioè, come se i diversi livelli di utilità potessero essere 
% confrontati lungo una scala numerica, come ad esempio la ricchezza per 
% [[Bernoulli]], o calcolati tramite formule matematiche come per [[John von 
% Neumann|Von Neumann]] e [[Oskar Morgenstern|Morgenstern]] o [[Jeremy 
% Bentham|Bentham]]. 
% 
% La teoria economica tradizionale moderna presuppone che le strutture delle 
% preferenze rappresentino il risultato dell'associazione di beni, servizi o il 
% loro uso con le quantità, definendo l'utilità come tale 
% quantificazione.<ref>[[David M. Kreps|Kreps, David Marc]]; A Course in 
% Microeconomic Theory, Chapter two: The theory of consumer choice and 
% demand,Utility representations.</ref>
%  
% Un'altra concezione è la filosofia di [[Jeremy Bentham|Bentham]], che 
% equiparava 
% l'utilità con la ''produzione di piacere e l'annullamento del dolore'', assunti 
% come oggetti delle operazioni aritmetiche.<ref>[[Jeremy Bentham|Bentham, 
% Jeremy]]; Introduction to the Principles of Morals and Legislation.</ref>
% 
% Egli affermava: 
% 
% {{quote|L'utilità è la tendenza di un oggetto o di un'azione di accrescere o 
% ridurre la felicità complessiva.|[[Jeremy Bentham]]}} 
% 
% Il solo scopo è la massimizzazione del proprio benessere personale, 
% indipendentemente da ciò che potrebbe dover essere sacrificato durante il 
% tragitto.
% 
% Gli economisti britannici, influenzati da questa filosofia (in particolare per 
% mezzo di [[John Stuart Mill]]), consideravano l'utilità come "le sensazioni di 
% piacere e di dolore" e in seguito come "quantità di sentimento "(enfasi 
% aggiunta).<ref>Jevons, William Stanley; “Brief Account of a General 
% Mathematical 
% Theory of Political Economy”, Journal of the Royal Statistical Society v29 
% (June 
% 1866).</ref> 
% 
% Al di fuori dei metodi tradizionali, vi sono concezioni di [[Utilità 
% (economia)|utilità]] che non si basano sulla quantificazione: ad esempio, la 
% [[scuola austriaca]] generalmente attribuisce valore alla ''soddisfazione dei 
% bisogni'',<ref> Menger, Carl; Grundsätze der Volkswirtschaftslehre(Principles 
% of 
% Economics).</ref><ref>Mc Culloch, James Huston; 
% “[http://www.mises.org/etexts/mcCulloch.pdf The Austrian Theory of the Marginal 
% Use and of Ordinal Marginal Utility]”,Zeitschrift für Nationalökonomie 37 
% (1977) 
% #3&4 (September).</ref><ref>[[Nicholas Georgescu-Roegen|Georgescu-Roegen, 
% Nicholas]]; Utility, International Encyclopedia of the Social Sciences 
% (1968).</ref> e, talvolta, respinge la possibilità di 
% quantificazione.<ref>[[Ludwig von Mises|von Mises, Ludwig Heinrich]]; Theorie 
% des Geldes und der Umlaufsmittel (1912).</ref> In questo modo è possibile 
% considerare razionali preferenze che sarebbero altrimenti escluse.<ref>Mc 
% Culloch, James Huston; “[http://www.mises.org/etexts/mcCulloch.pdf The Austrian 
% Theory of the Marginal Use and of Ordinal Marginal Utility]”,Zeitschrift für 
% Nationalökonomie 37 (1977).</ref> 
% 
% In ogni scenario standard, lo stesso oggetto può avere diverse utilità 
% marginali 
% per persone diverse che riflettono diverse preferenze o circostanze 
% individuali.<ref>Davenport, Herbert Joseph; The Economics of 
% Enterprise(1913).</ref>

\subsection{Prezzo di mercato e l'utilità marginale decrescente}

Se un individuo possiede un bene o un servizio la cui utilità marginale, per 
lui, è inferiore a quella di qualche altro bene o servizio per il quale avrebbe 
potuto scambiarlo, allora è nel suo interesse effettuare il commercio. 

Se l'utilità marginale di un bene o un servizio sta diminuendo e l'altro non è 
in aumento, un individuo tenterà di ottenere un rapporto sempre maggiore tra 
ciò che si acquista a ciò che viene venduto.

In economia, l'utilità marginale di una quantità è chiaramente associata al 
miglior bene o servizio che si potrebbe acquistare a parità di prezzo.
Questo concetto è alla base della teoria della domanda e 
dell'offerta, nonché degli aspetti essenziali dei modelli di concorrenza 
imperfetta.

\subsection{Il paradosso di acqua e diamanti}

Il ``paradosso dell'acqua e dei diamanti'', più comunemente associato ad 
Adam Smith,
% <ref>Gordon, Scott (1991). "The Scottish Enlightenment of the 
% eighteenth century". History and Philosophy of Social Science: An Introduction. 
% Routledge. ISBN 0-415-09670-7.</ref> sebbene riconosciuto a pensatori 
% precedenti,<ref>Marshall, Alfred; Principles of Economics, Bk 3 Ch 3 
% Note.</ref> 
è l'apparente contraddizione che l'acqua possiede un valore di gran lunga 
inferiore a quello dei diamanti, anche se l'acqua risulta essere vitale per un 
essere umano.

Il prezzo è determinato sia dall'utilità marginale che dal costo marginale: la 
chiave per il paradosso è che il costo marginale dell'acqua è di gran 
lunga inferiore a quello dei diamanti.
Questo non vuol dire che il prezzo di un qualsiasi bene o servizio è 
semplicemente l'utilità marginale che ha per un individuo, piuttosto, gli 
individui sono disposti a negoziare sulla base delle rispettive utilità 
marginali dei beni che hanno o che desiderano, dunque i prezzi risultano 
vincolati da tali utilità marginali.

% \subsection{Note}
% <references/>
% 
% \subsection{Voci correlate}
% * [[Utilità (economia)]]
% * [[Marginalismo]]
% * [[Produttività marginale]]
% * [[Teoria del valore]]
% 
% {{portale|economia}}
% 
% [[Categoria:Microeconomia]]
% [[Categoria:Storia del pensiero economico]]



% {{W|economia|settembre 2009}}

\subsection{La legge di Engel}

La \emph{legge di Engel} indica che la proporzione del reddito di una famiglia 
che viene consacrato all'alimentazione diminuisce quando il reddito aumenta.

Esaminando le spese di circa 200 famiglie belghe, l'economista tedesco Ernst 
Engel ha costatato che la proporzione del reddito che viene consacrato 
all'alimentazione diminuisce quando il reddito aumenta. 
Siccome questa relazione è valevole per tutti i paesi, si parla oggi di legge di 
Engel.
Ciò significa che l'elasticità della domanda rispetto al reddito è inferiore 
all'unità. Infatti, sia y il reddito, p il prezzo e q la quantità 
di beni alimentari. La proporzione del reddito consacrato all'alimentazione è:

\(  \omega = \frac{p \cdot q}{y} \)

Se questa proporzione diminuisce quando il reddito aumenta la derivata è 
negativa. Si ha allora:

\( \frac{\mathrm{d} \omega}{\mathrm{d} y}=\frac{py \frac{\mathrm{d} 
q}{\mathrm{d} y}-p q}{y^2} < 0  \rightarrow \eta = \frac{\mathrm{d} 
q}{\mathrm{d} y}\frac{y}{q} < 1 \)

La legge di Engel è una delle leggi tra le più generali in economia. Tutte 
le stime effettuate mostrano che la legge è verificata per  i paesi 
sviluppati come per i paesi in via di sviluppo. Ciò non significa che le 
elasticità siano uguali. Si trova sovente delle elasticità basse per i paesi 
sviluppati o per i redditi alti e delle elasticità più alte per i paesi in via 
di sviluppo o per i redditi bassi.

Si può rappresentare graficamente la relazione tra la spesa per un bene e il 
reddito. Questa relazione è chiamata la curva di Engel.

% \subsection{Bibliografia}
% 
% * E. Ducpétiaux - ''Budgets économiques des classes ouvrières en Belgique'' - 
% Bruxelles, 1855
% * E. Engel - ''« Die Productions- und Consumtionsverhältnisse des Königreichs 
% Sachsen », Statistischen Büreaus des Königlich Sächsischen Ministeriums des 
% Innern'' - 1857
% * [[Hendrik Houthakker|H. Houthakker]] - ''« An International Comparison of 
% Household Expenditure Patterns, Commemorating The Centenary of Engel's Law »'' 
% - 
% Econometrica, vol. 25, 1957, pp.&nbsp;532–551
% 
% [[Categoria:Microeconomia]]
% [[Categoria:Leggi economiche|Engel]]




\section{La teoria della produzione}

% * [[Teoria della produzione]]: [[funzione di produzione]], [[economia di 
% scala]], [[produttività marginale]], [[isocosto]], [[elasticità di 
% sostituzione]], la funzione di costo, [[costo marginale]]


La\emph{teoria della produzione} è lo studio della produzione o del 
processo economico di conversione di beni 
primari (``input'') in beni e prodotti finali a 
valore aggiunto (``output''). Dato che la produzione è un processo che si 
verifica attraverso il tempo e lo spazio, possiede uno stretto contatto con il 
concetto matematico di flusso. La produzione è infatti 
misurata come "tasso di produzione in output in rapporto all'unità di 
tempo".

I tre aspetti principali del processo di produzione sono:
% <ref name="Varian">{{cita|Varian||Varian, 1998}}</ref>
\begin{enumerate} [noitemsep]
 \item la quantità di beni o servizi da produrre;
 \item la forma del bene o servizio creato;
 \item la distribuzione spazio-temporale del bene o servizio prodotto.
\end{enumerate}

\subsection{Fattori di produzione}

Gli ingressi o le risorse utilizzate nel processo di produzione vengono 
chiamati fattori di produzione. La miriade di possibili fattori di solito è 
raggruppata in cinque categorie, ovvero:

\begin{enumerate} [noitemsep]
 \item materia prima;
 \item macchinario;
 \item lavoro;
 \item capitale;
 \item terra.
\end{enumerate}

Nel lungo periodo, tutti questi fattori possono essere variabili. Il 
breve periodo, invece, è definito come un periodo in cui è fissato almeno 
uno dei fattori di produzione. %<ref name="Varian"/>

Un fattore di produzione fisso è quello la cui quantità non può essere 
facilmente modificata. Alcuni esempi sono le attrezzature più importanti o lo 
spazio dedito alle fabbriche.

Un fattore di produzione variabile è uno il cui utilizzo può mutare facilmente. 
Gli esempi includono il consumo di energia elettrica, i servizi di trasporto e 
i materiali di base.

\subsection{Funzione di produzione}

[[File:Total product curve small.png|thumb|Esempio di funzione di produzione]]

Dato un insieme di produzione (l'insieme di tutte le combinazioni di 
input/output tecnicamente realizzabili), la \emph{funzione di produzione} 
consiste nella frontiera di tale insieme. Essa descrive il massimo livello di 
output dato un vettore di input.
% <ref name="Varian"/>

Generalmente la funzione di produzione è descritta come:

\(q=f(\mathbf X),\)

dove \(q\) è la quantità di prodotto e 
\(\mathbf X=(x_1, x_2, ..., x_n)\) è il vettore di input.

[[File:Average and marginal product curves small.png|thumb|Esempio di curve di 
produttività marginale e media]]

Data una funzione di produzione, la variazione del livello di output in 
corrispondenza di una variazione della i-esima variabile è chiamata 
\emph{produttività marginale} del fattore \(x_i\), ed è definita 
come:
% <ref name="Varian"/>

\(PMG_i=\frac{\partial{f(\mathbf X)}}{\partial{x_i}}.\)

Il rapporto tra il livello di output e l'ammontare complessivo dell'input 
utilizzato è invece chiamato \emph{produttività media} definita come:
% <ref 
name="Varian"/>

\(PME_i=\frac{d{f(x_i)}}{d{x_i}}.\)

È degno di nota che, finché PMG<sub>i</sub>>PME<sub>i</sub>, la produttività 
media dell'i-esimo input sarà crescente. %<ref name="Varian"/>

Dimostrazione

\(\frac{dPME_i}{dx_i}>0;\)

\(\frac{f'(x_i)}{x_i}-\frac{f(x_i)}{{x_i}^2}>0;\)

\(\frac{f'(x_i)}{x_i}>\frac{f(x_i)}{{x_i}^2};\)

\({f'(x_i)}>\frac{f(x_i)}{x_i};\)

\(PME_i>PMG_i.\)

\subsection{Isoquanti e isocosti}

% {{vedi anche|Isoquanto|Isocosto}}
% [[File:Indifference curves 3d.jpg|thumb|Mappa degli isoquanti in tre 
% [[dimensioni]]]]

Fissato un valore di output \(q_0\), si otterrà la relazione

\(q_0=f(x_1, x_2),\)

la quale descrive la mappa dell'isoquanto (ovvero, tutte le combinazioni di 
input che restituiscono il valore di output fissato).

Ora, detta \(C=c(w_1,w_2,q)\) la funzione di costo (funzione che 
esprime i costi minimi necessari per produrre \(q\) unità di output, 
dati i prezzi di input \(w_1\) e \(w_2\)), per un determinato 
costo \(c_0\) tutte le combinazioni di input che generano tale costo 
possono essere rappresentate mediante la cosiddetta retta di isocosto, 
la cui equazione è:

\(x_1=-\frac{w_2}{w_1}\cdot x_2+\frac{c_0}{w_1},\)

avente come coefficiente angolare (la pendenza della retta rispetto agli 
assi) \(- \frac{w_2}{w_1}\). È da notare che, se \(w_1\) e 
\(w_2\) sono costanti, le rette di isocosto saranno tutte 
parallele fra loro.

[[File:TE-Production-Isoquant.png|thumb|Retta di isocosto tangente alla curva 
di isoquanto]]

Dato un isoquanto, per minimizzare i costi di produzione, è necessario 
individuare il punto di tangenza con la retta di isocosto più bassa 
possibile. %<ref name="Varian"/>

\subsection{Saggio di sostituzione tecnica}

Il saggio di sostituzione tecnica (o saggio tecnico di sostituzione, 
STS)
% <ref>In 
% [[Lingua inglese|inglese]] si utilizza sia la sigla TRS (''technical rate of 
% substitution'') che RTS (''rate of technical substitution'').</ref> 
rappresenta la misura della sostituibilità degli input, fissato un output, ed è 
dato da:

\(STS_{2,1}=\frac{\Delta x_2}{\Delta x_1}.\)

[[File:MRTS small.png|thumb|Saggio marginale di sostituzione tecnica]]

Il saggio marginale tecnico di sostituzione (o saggio marginale di 
sostituzione tecnica, SMST),
% <ref>In inglese è indicato con MRTS (''marginal 
% rate of technical substitution'').</ref> 
invece, rappresenta la pendenza dell'isoquanto, ed è dato da:

\(SMST_{2,1}=\frac{dx_2}{dx_1}.\)

Esiste un teorema che dimostra quanto segue:

\(SMST_{2,1}=-\frac{PMG_1}{PMG_2}.\)

Dimostrazione
Dato che, se si parla di isoquanti, \(q\) risulta essere fisso, la sua 
derivata sarà nulla. Ma, per la formula del differenziale totale:

\(dq=dx_1\cdot \frac{\partial f}{\partial x_1}+dx_2\cdot \frac{\partial 
f}{\partial x_2};\)

\(\Rightarrow dx_1\cdot \frac{\partial f}{\partial x_1}+dx_2\cdot 
\frac{\partial f}{\partial x_2}=0;\)

\(\Rightarrow \frac{dx_2}{dx_1}=-\frac{\partial f/\partial x_1}{\partial 
f/\partial x_2};\)

\(\Rightarrow SMST_{2,1}=-\frac{PMG_1}{PMG_2}.\)

\subsection{Concorrenza perfetta e monopolio}

Si è in presenza di una \emph{concorrenza perfetta} laddove si riscontrino le 
seguenti ipotesi: %<ref name="Varian"/>
\begin{description}
 \item[Polverizzazione (o[atomizzazione) del mercato]: esistono molti 
piccoli produttori dello stesso bene.
 \item[Omogeneità del prodotto]: le imprese non hanno la possibilità di 
differenziare i propri prodotti. Di conseguenza, il consumatore percepisce in 
maniera identica il valore dello stesso prodotto di due imprese distinte.
 \item[Assenza di barriere all'entrata]: le imprese che vogliono entrare nel 
mercato non incontrano alcun ostacolo.
\end{description}

Se sussistono tali condizioni, le imprese del mercato possono essere 
considerate price-taker. 
Il prezzo, infatti, dipende solo dalla domanda e dall'offerta 
del mercato del bene in questione. %<ref name="Varian"/>

Nell'analisi della produzione, una delle caratteristiche fondamentali del 
regime di concorrenza perfetta è che l'impresa, per massimizzare il proprio 
profitto, sceglie il prezzo eguagliandolo al costo marginale di produzione.

Dimostrazione
Detto \(p^e\) il prezzo di equilibrio:

\(\frac{\partial \pi}{\partial q}=0;\)

\(\Rightarrow \frac{\partial (q\cdot p^e-C(q))}{\partial 
q}=p^e-\frac{\partial C(q)}{\partial q}=0;\)

\(\Rightarrow p^e=\frac{\partial C(q)}{\partial q}=CMG .\)

Il \emph{monopolio}, invece, è una forma di mercato in cui una merce, di cui 
non esiste un sostituto equivalente, è prodotta da un'unica 
impresa.
% <ref>{{cita|Varian|p. 403|Varian, 1998}}</ref> 
Sono inoltre presenti delle barriere all'entrata, quindi non è possibile per le 
altre imprese entrare facilmente nel mercato. 
L'impresa che detiene il monopolio viene detta price maker.

% == Relazioni verticali ==
% {{...||economia|arg2=ingegneria}}
% 
% == Note ==
% <references/>
% 
% == Bibliografia ==
% *{{cita libro|titolo=Microeconomia|autore=[[Hal R. 
% Varian]]|editore=Cafoscarina|anno=1998|città=Venezia|pagine=662|edizione=4 
% ed.|id=ISBN 88-85613-75-6|cid=Varian, 1998}}
% 
% == Voci correlate ==
% *[[Elasticità (economia)]]
% 
% {{portale|economia|ingegneria}}
% 
% [[Categoria:Economia della produzione| ]]
% [[Categoria:Ingegneria industriale]]
% [[Categoria:Gestione della produzione]]




% * Teoria dei mercati ([[concorrenza perfetta]]): [[domanda e offerta]], 
% [[Equilibrio economico|equilibrio]], [[ciclo del maiale]]


\section{concorrenza perfetta}

% {{F|microeconomia|data=febbraio 2008}}
% {{S|microeconomia}}

In economia, la \emph{concorrenza perfetta} (in inglese 
``perfect competition'' o ``pure competition'') è una forma di mercato
caratterizzata dall'impossibilità degli imprenditori di 
fissare il prezzo di vendita dei beni prodotti,  che è fissato invece 
dall'incontro della domanda e dell'offerta, che a loro volta sono espressione 
dell'utilità e del costo marginale. 
L'impresa non può determinare contemporaneamente quantità e prezzo d'equilibrio 
del mercato.

La definizione di concorrenza perfetta fa riferimento a quella situazione in 
cui, per il numero degli operatori economici presenti sul mercato, ciascuno di 
essi (sia esso espressione della domanda ovvero \emph{consumatore} e/o sia 
esso espressione dell'offerta ovvero \emph{produttore}) non ha la possibilità 
di influenzare in alcun modo, attraverso i propri comportamenti, il prezzo di 
vendita dei beni e/o servizi scambiati sul mercato.

==Descrizione==

La curva di domanda è semplificata con una retta, ovvero una funzione 
lineare e quindi invertibile di prezzo e quantità, inclinata negativamente.
Il mercato di concorrenza perfetta, lungi dall'essere una rappresentazione 
veritiera della realtà, costituisce un presupposto alla base di molti modelli 
economici di analisi dell'equilibrio.

L'equilibrio concorrenziale si contrappone ad altri modelli, ma possiede delle 
caratteristiche che lo rendono desiderabile rispetto a questi ultimi dal punto 
di vista dell'efficienza economica.

Un mercato si può definire perfettamente concorrenziale quando si verificano le 
seguenti ipotesi:
\begin{enumerate}
 \item il bene prodotto è omogeneo, quindi le unità di un certo tipo di bene 
sono tutte uguali tra loro;
 \item le imprese operano in condizione di "informazione completa e simmetrica" 
(trasparenza di mercato), ossia tutti gli agenti economici (produttori e 
consumatori) dispongono di informazioni complete in merito ai costi di 
produzione, ai prezzi, alle caratteristiche dei beni, alla disponibilità sul 
mercato, al salario reale di equilibrio, ecc.;
 \item le imprese che operano sul mercato hanno una dimensione atomistica, tale 
da non poter influenzare in alcun modo i prezzi di vendita (le imprese sono 
Price-taker);
 \item i consumatori hanno chiare le loro preferenze e le imprese conoscono le 
tecnologie messe a loro disposizione, che sono uguali per tutti e non possono 
essere sostituite;
 \item gli agenti economici dispongono delle stesse informazioni in maniera 
certa;
 \item la chiusura di un'impresa, giungerà quando essa non sarà più in grado di 
coprire i costi variabili, e quando il prezzo di vendita del bene sul mercato 
sarà inferiore al [[costo variabile]] unitario del bene;
 \item libertà di entrata e uscita dal mercato, quindi non c'è il vincolo dei 
costi di transazione;
 \item sono resi certi i diritti di proprietà delle risorse disponibili, in modo 
da conferire agli agenti economici una certa responsabilità nell'impiego dei 
propri mezzi;
\end{enumerate}

Secondo la teoria microeconomica classica, la concorrenza 
perfetta è il meccanismo ottimale per l'allocazione efficiente delle risorse in 
quanto il prezzo di vendita che si forma sul mercato è quello che remunera 
tutti i fattori di produzione in base alla loro produttività marginale e non 
consente: creazione di extra profitti e sfruttamento del lavoro. Inoltre il 
prezzo (o meglio il sistema dei prezzi relativi) è anche quello che consente ai 
consumatori di massimizzare la loro soddisfazione. Questo risultato è noto 
come primo teorema dell'economia del benessere.

La concorrenza perfetta è stata studiata in particolar modo da Léon Walras.

\subsection{Equilibrio e impostazione del prezzo}

Impresa è \emph{price taker} in un mercato di concorrenza perfetta. L'impresa 
in questo caso trova l'equilibrio e può essere nel breve periodo o nel lungo 
periodo agendo sui costi CMg = RMg = P un'impresa che massimizza i profitti 
stabilisce di produrre un livello di output al quale il costo marginale è 
uguale al prezzo. Graficamente questo significa che la curva del costo 
marginale di un'impresa corrisponde alla curva di offerta.

Questa equazione è la condizione di massimo profitto, che si ricava ponendo 
a zero la derivata del profitto rispetto alla quantità prodotta, per trovarne 
il massimo.

La derivata dei ricavi totali rispetto alla quantità è il prezzo, 
mentre il termine che riguarda i costi fissi (indipendenti 
dalla quantità prodotta) si azzera durante la derivazione.

Il ricavo marginale e costo marginale sono l'incremento di ricavo e 
costo per l'impresa per ogni unità \emph{nuova} di prodotto; se il costo e 
ricavo marginale si eguagliano, l'aggravio di costi e l'incremento di ricavi 
che genera un nuovo prodotto si compensano e danno un incremento di profitto 
pari a zero. È coerente il fatto che il massimo profitto si trovi in 
corrispondenza di un incremento dei profitti nullo (il valore è massimo quando 
la variabile non può più aumentare).

La decisione di chiusura in concorrenza perfetta:

\begin{itemize} [noitemsep]
 \item nel breve periodo i costi fissi sono irrecuperabili quindi l'impresa 
deve coprire almeno i costi variabili;
 \item nel lungo periodo l'impresa deve coprire tutti i costi;
 \item se non si coprono i costi si cessa l'attività.
\end{itemize}

Quando la retta del ricavo marginale, che coincide con il prezzo, scende al di 
sotto del minimo della curva del costo variabile medio, non è più conveniente 
produrre. Tale punto di minimo è detto punto di fuga (dei produttori dal 
mercato).

Quando il prezzo è compreso fra il minimo del costo totale medio e quello del 
costo variabile medio, ci si trova in una situazione in cui è conveniente 
produrre in perdita pur di ammortizzare gli alti costi fissi.

Di seguito, riportiamo i principali grafici che descrivono un regime di 
concorrenza perfetta.

\begin{enumerate} [noitemsep]
\item il mercato fa il prezzo;
\item confronto prezzo dato dal mercato e costi marginali (CMg);
\item confronto CMg con costi variabili (CV);
\item nel lungo periodo prezzo uguale al minimo del costo medio e al costo 
marginale.
\end{enumerate}

[[File:Dom-off Cmg.JPG]]
[[File:Me-cmg.JPG]]

\subsection{Critica al modello}

Intorno agli anni '40 nasce la tesi della concorrenza non-perfetta, e quindi 
nuovi approcci che tentano di risolvere il problema dell'informazione 
incompleta:
\begin{itemize} [noitemsep]
 \item la teoria dei giochi;
 \item la teoria dei contratti.
\end{itemize}

Nella realtà:
\begin{itemize} [noitemsep]
 \item Non c'è omogeneità di prodotto, ma differenziazione per massimizzare il 
profitto e ottenere una posizione vantaggiosa sul mercato; questo consente di 
avere un margine di manovra sul prezzo praticato, benché minimo (le imprese non 
sono price-taker): monopolio. Inoltre la differenziazione di prodotto genera 
anche dei tentativi di carpire quali sono i gusti del consumatore.
 \item È impossibile che ci siano infiniti operatori sul mercato, per via delle 
economie di scala: oligopolio.
\end{itemize}


\emph{Friedrich von Hayek}: “L'informazione in mano ai vari operatori 
presenti sul mercato è imperfetta”; quando si analizza la collettività, non si 
può definire l'informazione come data, acquisita, rigida, ma è necessario 
capire quanta e quale informazione hanno gli individui a loro disposizione; 
infatti, una comunità di individui che interagiscono non ha informazioni 
immutabili e informazioni sono anche le decisioni che gli altri individui 
prendono, influenzando in questo modo l'ambiente circostante (anche le imprese).
La concorrenza perfetta è un processo, un punto di partenza. Attraverso la sua 
evoluzione porta al risultato di uno stato di equilibrio ideale (il prezzo 
unico) che è il punto di arrivo.

\emph{Milton Friedman}: non può esserci un punto d'equilibrio nella 
concorrenza. Una teoria è corretta quando non viene smentita dalla realtà: per 
arrivare ad una teoria, è necessario formulare un'ipotesi, che deve essere 
generale e astratta (non reale); la teoria non descrive, non analizza 
comportamenti, ma individua delle condizioni. Il modello di concorrenza 
perfetta rappresenta una legge generale, con cui analizzare e osservare i 
mercati reali.

Sia Hayek che Friedman sostengono che, pur non potendo essere assunta a verità 
assoluta, la concorrenza perfetta è una teoria utile per formulare previsioni.

% == Voci correlate ==
% 
% * [[prezzo di costo]]
% 
% {{Portale|diritto|economia}}
% [[Categoria:Economia industriale]]
% [[Categoria:Microeconomia]]

\section{Domanda e offerta}

[[File:Supply-and-demand.svg|thumb|
Il prezzo P di un prodotto, che rende 
massimo il ricavo, è determinato dall'equilibrio tra le due curve della domanda 
D e dell'offerta S. Il grafico mostra l'effetto di una crescita della curva di 
domanda da D1 a D2: il prezzo P e la quantità totale Q venduta aumentano 
ambedue.]]

In economia \emph{domanda e offerta} è un modello matematico di 
determinazione del prezzo nell'ambito del sistema 
matematico denominato tecnicamente, con termine intuitivo, mercato.

\subsection{Domanda}

In microeconomia per \emph{domanda} s'intende la quantità di consumo 
richiesta dal mercato e dai consumatori di un certo bene o servizio, dato 
un determinato prezzo e quanto spenderebbero se tale prezzo variasse. In ottica 
macroeconomica, per la scuola neoclassica l'insieme 
delle domande dei singoli consumatori costituisce la domanda collettiva o 
``domanda aggregata''.

Ci sono diversi fattori che influenzano la domanda:
\begin{enumerate} [noitemsep]
 \item Il prezzo del bene acquistato;
 \item Il prezzo dei beni complementari e succedanei;
 \item Il reddito del consumatore;
 \item Le aspettative soggettive dei consumatori;
 \item Il costo del denaro;
 \item L'elasticità o la rigidità della domanda;
 \item Bisogni del consumatore.
\end{enumerate}

\subsubsection{Caratteristiche della domanda}

% {{Vedi anche|Elasticità della domanda}}

La domanda si caratterizza principalmente per quattro fattori:
\begin{description}
 \item [Concentrazione della domanda]: la domanda può costituirsi di un unico 
acquirente (in genere in una ben definita area geografica, ad esempio 
un'impresa produttrice di binari ferroviari in Italia ha come unico acquirente 
le Ferrovie dello Stato) e in tal caso si parla di monopsonio; di pochi 
grandi acquirenti e si parla di oligopsonio; oppure di tanti piccoli 
acquirenti per cui si parla di domanda polverizzata.
 \item [Elasticità della domanda]: indica la variabilità della domanda in 
relazione ad un determinata variabile (prezzo, reddito, ecc.). Una domanda 
molto elastica varia notevolmente in seguito ad un minimo aumento/riduzione del 
prezzo, ad esempio. La curva del grafico sarà tendenzialmente orizzontale. Ad 
una domanda infinitamente elastica corrisponde una curva del tutto orizzontale, 
dato che i consumatori acquistano la quantità massima di un 
bene in corrispondenza di un singolo prezzo.
 \item [Differenziazione della domanda], che definisce tante più domande (e 
offerte) quanto più sono i segmenti di mercato.
 \item [Rigidità della domanda]: indica la costante in funzione dell'offerta 
variabile. Ciò significa che all'aumento del prezzo, la domanda cala 
lentamente. Questo succede ai beni di prima necessità.
\end{description}

\subsubsection{Domanda e offerta}

[[File:Domanda.png|thumb|Grafico della domanda individuale]]

In generale tutte le ``curve di domanda'' hanno pendenza negativa (in caso di 
beni o servizi normali), questo significa che più il prezzo di un bene o 
servizio è alto, meno ne viene richiesto. Viceversa più un bene o servizio è a 
buon mercato, più ne viene venduto. La relazione tra quantità e prezzo è dunque 
inversa. I beni per cui la funzione di domanda ha pendenza positiva sono detti 
[[beni di Giffen]].

\subsubsection{Domanda di mercato}

Per ottenere la domanda di mercato bisogna sommare orizzontalmente tutte le 
domande individuali; graficamente:
* la pendenza (e quindi la relazione) rimane negativa ma può cambiare (rimane 
uguale solo in caso tutte le pendenze siano perfettamente uguali);
* mentre la curva, detta funzione di domanda "a gomito", si sposta verso destra 
(ovviamente la quantità richiesta aumenta se sommiamo tutte le curve).

[[File:Sommaorizzontale.gif|thumb|upright=3|center|Grafico della domanda di 
mercato|451x451px]]

Influiscono sulla domanda di mercato: 
\begin{itemize} [noitemsep]
 \item gusti e redditi dei consumatori;
 \item i prezzi degli altri beni di consumo; 
 \item l'aumento demografico che porta la curva a slittare verso destra; 
 \item la concentrazione del reddito nazionale.
\end{itemize}

Se la domanda è molto elastica rispetto al prezzo, la curva del grafico sarà 
tendenzialmente piatta, dato che ad una piccola variazione di prezzo 
corrisponde una notevole variazione nella richiesta. Ad una domanda 
infinitamente elastica corrisponde una curva del tutto orizzontale: i 
consumatori acquistano la quantità massima di un bene in corrispondenza di un 
singolo prezzo.

\subsection{Offerta}

In economia, per \emph{offerta} si intende la quantità di un certo bene o 
servizio che viene messa in vendita in un dato momento a un dato prezzo.

Si suppone che per ogni bene si possa tratteggiare una curva di offerta (curva 
con pendenza positiva per la [[legge dei rendimenti decrescenti]]), 
rappresentante le diverse quantità messe in vendita dalle imprese di un bene o 
servizio in corrispondenza di ciascun prezzo.

Le caratteristiche dell'offerta influenzano il prezzo e il tipo di [[mercato]] 
per un dato bene.
Essa viene influenzata da diversi fattori:
\begin{description} [noitemsep]
 \item [Costi di produzione]: la diminuzione dei salari percepiti dagli 
operai nel settore, abbassa i costi e incrementa l'offerta.
 \item [Tecnologia]: migliore tecnologia comporta un'iniziale spesa maggiore 
per la Ricerca e lo Sviluppo, ma poi riduce i costi di produzione e incrementa 
l'offerta.
 \item [Prezzi]
 \item [Politiche governative]: l'abolizione dei dazi doganali determina un 
aumento dell'offerta dei prodotti esportabili.
\end{description}

Se esiste un solo venditore si parla di monopolio, di duopolio nel caso 
vi siano due soli venditori, di oligopolio se i venditori sono pochi. In 
presenza di moltitudini di venditori, ognuno dei quali non è in grado di 
determinare il prezzo di vendita si parla di ``concorrenza perfetta''.

L'offerta ``individuale'' di un bene è la quantità di quel bene che i venditori 
sono disposti a offrire sul mercato in un determinato momento e a un certo 
prezzo. L'offerta ``collettiva'' è l'insieme delle offerte individuali.

\subsubsection{Curva di Offerta del lavoro}

Il lavoro non è un bene omogeneo. 
Nel caso del lavoro, la ``domanda'' è fatta dalle imprese, mentre ``l'offerta'' 
è fornita dai lavoratori.
Il lavoro è caratterizzato da una retta a pendenza negativa (con assi con 
ordinata W, profitto, e ascissa L, lavoro) e da differenziali compensativi: si 
tende a pagare di più per lavori che nessuno ``vuole'' fare.

\begin{itemize} [noitemsep]
 \item La ''curva di offerta del lavoro'' è una curva simile all'offerta, ma 
con un'involuzione finale che riporta la curva verso quantità di lavoro minori. 
Questo perché, più il salario tende a salire più l'utilità marginale diventa 
piccola; per questa ragione, la curva S, per W molto alti, tende a diventare 
verticale o addirittura a tornare indietro.
 \item Significativo è sottolineare la differenza tra ``non lavoratori'' e 
``disoccupati'': mentre i primi hanno scelto di non lavorare, i secondi sono 
disposti a lavorare al salario corrente ma non trovano lavoro.
\end{itemize}


%   [[Categoria:Economia della produzione]]
% [[Categoria:Microeconomia]]
% [[Categoria:Macroeconomia]]
% [[Categoria:Leggi economiche|Domanda e offerta]]

\subsubsection{Un problema di equilibrio}

% {{S|economia internazionale}}
In ambito economico, l'equilibrio è una condizione del sistema economico in cui 
le forze sono equilibrate: in assenza di fattori esterni, le variabili micro e 
macroeconomiche rimangono immutate.

L'equilibrio è la condizione in cui il prezzo di mercato è determinato in 
un regime di concorrenza perfetta: tale prezzo è strettamente legato alla 
domanda e offerta. Adam Smith, economista affiliato alla 
teoria liberista, ribattezzò questo fenomeno di equilibrio 
dinamico ``mano invisibile''.

Per risolvere il problema di prezzo di equilibrio si devono tracciare le curve 
d'offerta e di domanda ed eguagliarle in equazione.

Un esempio può essere: 

\begin{alignat}{2}
 Q_s & = 124 + 1.5 \cdot P \\
 Q_d & = 189 - 2.25 \cdot P \\
\\
 Q_s & = Q_d \\
\\
 124 + 1.5 \cdot P & = 189 - 2.25 \cdot P \\
 (1.5 + 2.25) \cdot P & = (189 - 124) \\
 P & = \frac{189 - 124}{1.5 + 2.25} \\
 P & = \frac{65}{3.75} \\
 P & = 17.33 \\
\end{alignat}


In qualsiasi prezzo al di sopra P offerta supera la domanda, mentre ad un 
prezzo inferiore P, la quantità eccede l'offerta. In altre parole, se i prezzi 
di domanda e offerta non sono uguali, allora sono definiti punti di squilibrio, 
e creano eccesso o carenza di bene.  Cambiamenti nelle condizioni di domanda o 
di offerta sposteranno a loro volta le curve di offerta e di domanda finché non 
si troveranno in equilibrio.

Si consideri il seguente schema di domanda e offerta:

\begin{center}
\begin{tabular}{rrr}
\toprule
Prezzo (€) & Domanda & Offerta\\
\midrule
8.00 & 6 000 & 18 000\\
7.00 & 8 000 & 16 000\\
6.00 & 10 000 & 14 000\\
5.00 & 12 000 & 12 000\\
4.00 & 14 000 & 10 000\\
3.00 & 16 000 & 8 000\\
2.00 & 18 000 & 6 000\\
\bottomrule
\end{tabular}
\end{center}

\begin{itemize} [noitemsep]
 \item Il prezzo di equilibrio del mercato è pari a € 5,00 dove la domanda e 
l'offerta sono pari a 12.000 unità.
 \item Se il prezzo corrente di mercato è di € 3,00 - vi sarebbe un eccesso di 
domanda per 8.000 unità, creando una carenza di bene.
 \item Se il prezzo corrente di mercato fosse di € 8,00 - vi sarebbe eccesso di 
offerta di 12.000 unità.
\end{itemize}

Quando vi è una carenza nel mercato vediamo che, al fine di correggere questo 
disequilibrio, il prezzo del bene sarà aumentato a un prezzo di € 5,00, 
riducendo così il quantitativo domandato e aumentando quello offerto, così da 
mantenere il mercato in equilibrio.

Quando vi è un eccesso di un bene, ad esempio quando il prezzo è al di sopra di 
\$ 6.00, allora vediamo che i produttori di diminuiranno il prezzo al fine di 
aumentare la domanda di richiesta per il bene, in modo da eliminare l'eccesso e 
tornare all'equilibrio di mercato. 

% {{Teoria dei giochi}}
% {{Portale|Economia}}
% 
% [[Categoria:Teorie in economia]]

\subsubsection{ciclo del maiale}

[[File:Schweinezyklus.svg|thumb|upright=1.2|Grafico che illustra il ciclo del 
maiale]] 

Il \emph{ciclo del maiale} è un termine originariamente derivato dalle 
scienze agrarie per descrivere una fluttuazione periodica del 
mercato dei maiali. L'uso si è poi esteso anche nell'ambito 
dell'economia.

Quando i prezzi tendono a raggiungere valori elevati, gli 
investimenti tendono ad aumentare. Questo effetto è però 
ritardato dal tempo necessario alla riproduzione degli animali. Il mercato 
diviene quindi saturo e ciò si traduce in una diminuzione dei prezzi. Come 
conseguenza la produzione diminuisce ma gli effetti vengono osservati solo 
dopo un certo periodo di tempo, quindi si tornerà a una situazione di aumento 
della domanda e a un aumento dei prezzi.
% <ref> {{cita 
% libro|autore=Sjoukje Osinga|autore2=Gert Jan Hofstede|autore3=Tim 
% Verwaart|titolo=Emergent Results of Artificial Economics|editore=Springer 
% Science & Business Media|anno=2011|pagine=29|ISBN= 3-642-21108-9}} </ref> 
Questo processo si ripete ciclicamente. Il grafico risultante 
offerta/domanda somiglia a una ragnatela. L'economista agrario Mordecai 
Ezekiel fu uno dei primi a dare una interpretazione del ciclo del maiale.

In economia questo concetto è stato similmente applicato per la descrizione di 
vari tipi di mercato. Ad esempio, nell'ambito del mercato del lavoro 
salari alti o maggiori possibilità di inserimento lavorativo 
associati a un particolare settore provocano un aumento di immatricolazioni 
universitarie relativamente ad un certo indirizzo di studi. Dopo un po' di 
tempo si giungerà a una condizione di saturazione dell'offerta lavoro e il 
numero di immatricolazioni tenderà a diminuire. La produzione di beni 
industriali, il mercato immobiliare e la produzione di 
petrolio e suoi derivati sono altri importanti esempi in cui è applicato 
proficuamente il ciclo del maiale.

% ==Note==
% <references/>
% 
% == Voci correlate ==
% *[[Modello della ragnatela]]
% 
% {{portale|economia}}
% 
% [[Categoria:Microeconomia]]
% [[Categoria:Economia e politica agraria]]



% * Teoria dei mercati ([[concorrenza monopolistica]]): [[monopolio]], 
% [[duopolio]] (modelli di [[duopolio di Cournot|Cournot]], [[Joseph Louis 
% François Bertrand|Bertrand]], [[Modello di Hotelling|Hotelling]], [[duopolio 
% di 
% Stackelberg|Stackelberg]]), [[oligopolio]] (generalizzazione del modello di 
% [[Oligopolio di Cournot|Cournot]], [[modello di Salop]])

\section{Concorrenza monopolistica}

La \emph{concorrenza monopolistica} è una forma di mercato molto diffusa. 
Spesso caratterizza i mercati di libri, ristoranti, 
film, abbigliamento, ecc.

\subsection{La differenziazione del prodotto}

Si instaura quando un certo numero di venditori offre sul mercato beni o 
prodotti che, nati per soddisfare lo stesso bisogno, si presentano in modo 
diverso.
Esiste quindi per ogni prodotto una domanda stabile e 
ripetuta dalla clientela che apprezza quelle caratteristiche.
Una variazione di prezzo crea comunque una variazione di segno contrario 
della domanda.

Il prodotto può essere differenziato in vari modi:
\begin{itemize} [nosep]
 \item secondo il tipo e lo stile;
 \item secondo la localizzazione;
 \item secondo la qualità.
\end{itemize}

\subsection{Modello di Chamberlin}

Edward Chamberlin
% <ref>Edward H. Chamberlin, Teoria della concorrenza 
% monopolistica, Firenze, 1961</ref> 
osserva che in molti rami economici ci sono 
numerose imprese che vendono dei beni differenziati (marca, qualità, 
localizzazione differenti). Il potere di monopolio di queste imprese è 
molto limitato poiché ci sono molti beni sostituti. A lungo termine i ricavi 
della maggioranza di queste imprese coprono appena i costi di vendita. I 
piccoli negozi di quartiere, i ristoranti, i parrucchieri o gli idraulici sono 
degli esempi di queste imprese.
% <ref>David Begg, Stanley Fisher, Rudiger 
% Dornbush, Microeconomia, McGraw-Hill, 2008</ref>.[[File :M-Chamberlin.pdf 
% |upright=1.8|thumb| ]]

La curva di domanda dell'impresa rappresentativa ha un pendenza negativa 
siccome la differenziazione dei prodotti le conferisce un certo potere di 
mercato. Se il prezzo aumenta l'impresa non perde tutti i suoi clienti. 

Nel breve periodo l'impresa genera profitti positivi (e.g. un ristorante che 
apre in un nuovo quartiere). Il primo grafico illustra questa situazione. 
L'equilibrio dell'impresa implica l'uguaglianza tra il ricavo marginale (Rm) e 
il costo marginale (Cm). Il prezzo di vendita è p*. Il profitto 
(\(\pi\)) è dato dalla differenza tra il ricavo medio (RM), la curva 
di domanda, e il costo medio (CM) moltiplicata per la quantità prodotta (q*).

Questi profitti extra attirano dei nuovi operatori economici (e.g. ristoranti 
che vedono la possibilità di fare degli ottimi affari nel medesimo quartiere). 
Siccome non ci sono delle barriere all'entrata nel settore, la curva di domanda 
per i prodotti dell'impresa si sposta a sinistra (il primo ristorante avrà 
perso una parte dei suoi clienti). Tale movimento continua fino alla scomparsa 
del profitto (equilibrio a lungo termine = extra profitti nulli). Dal punto di 
vista analitico, la curva dei costi medi (CM) sarà tangente alla curva di 
domanda RM.

Rispetto alla [[concorrenza perfetta]], il prezzo sarà più alto e l'impresa non 
produce ai costi minimi. C'è un eccesso di capacità in questo ramo (si può 
produrre di più e a dei costi più bassi) ma c'è anche una più grande varietà.

\subsection{Altri modelli}

Dixit e Stiglitz introducono esplicitamente le preferenze 
per le varietà nella funzione di utilità del consumatore rappresentativo
% <ref>Avinash Dixit and Joseph Stiglitz, " Monopolistic Competition and Optimum 
% Product Diversity ", American Economic Review, 1977, p. 297-308</ref>. 
Due casi 
sono esaminati: nel primo le preferenze sono rappresentate da una funzione di 
utilità a elasticità di sostituzione costante; nel secondo l'elasticità è 
variabile.

Produrre al livello di costi minimi non è una soluzione ottimale dal punto di 
vista dell'utilità sociale quando i diversi beni non sono dei sostituti 
perfetti. Non c'è dunque troppa varietà nella concorrenza monopolistica.

Il modello di Dixit e Stiglitz è stato utilizzato da Paul Krugman nei suoi 
modelli di commercio internazionale e sulla concorrenza spaziale delle attività 
economiche
% <ref>Paul Krugman, " The Increasing Returns Revolution in Trade and 
% Geography ", American Economic Review, 2009, p. 561-571</ref>
.

Ogni modello di concorrenza monopolistica deve basarsi sulle condizioni 
seguenti:
% <ref> Oliver D. Hart, " Monopolistic Competition in the Spirit of 
% Chamberlin: A General Model ", Review of Economic Studies, 1985, p. 529</ref>

\begin{itemize} [noitemsep]
 \item numerose imprese che producono dei beni differenziati;
 \item le imprese sono piccole e hanno degli effetti trascurabili sulle altre 
imprese;
 \item la libera entrata sul mercato conduce a lungo termine alla scomparsa del 
profitto;
 \item la domanda dei prodotti delle imprese ha una pendenza negativa.
\end{itemize}

Hart
% <ref>Oliver D. Hart, " Monopolistic Competition in the Spirit of 
% Chamberlin: Special Results ", Economic Journal, 1985, p. 889-908</ref> 
suppone numerosi consumatori con dei gusti differenti e numerose imprese che 
producono dei beni differenti. I consumatori sono interessati ad un numero 
limitato di questi beni. Ogni bene ha la medesima probabilità di essere 
desiderato dal consumatore. Gli effetti di un cambiamento dell'offerta di 
un'impresa sono allora ripartiti su tutte le altre imprese. Hart arriva alla 
conclusione che, secondo i casi, ci possono essere troppe o troppo poche 
varietà.

% == Note ==
% <references/>
% 
% == Voci correlate ==
% * [[Monopolio]]
% * [[Oligopolio]]
% * [[Modello di Hotelling]]
% * [[Modello di Salop]]
% 
% == Bibliografia ==
% 
% * Avinash Dixit and Joseph Stiglitz, " Monopolistic Competition and Optimum 
% Product Diversity ", American Economic Review, 1977, p. 297-308
% * Oliver D. Hart, " Monopolistic Competition in the Spirit of Chamberlin: A 
% General Model ", Review of Economic Studies, 1985, p. 529-546
% * [[Paul Krugman]], [[Robin Wells]], ''Microeconomia'', [[Bologna]], [[Nicola 
% Zanichelli Editore|Zanichelli]], 2005. ISBN 88-08-17842-0
% * K. Lancaster, " Socially Optimal Product Differentiation ", American 
% Economic % Review, 1975, p. 567-585
% * A.M. Spence, " Product Selection, Fixed Costs and Monopolistic Competition 
", % Review of Economic Studies, 1976, p. 217-235
% 
% {{Controllo di autorità}}
% {{Portale|economia}}
% 
% [[Categoria:Economia industriale]]
% [[Categoria:Microeconomia]]

\subsection{Monopolio}

\emph{Monopolio} (dal greco ``mònos'' <<solo>> 
e  ``pólion''  <<vendere>>) è una forma di 
mercato, dove un unico venditore offre un prodotto 
o un servizio per il quale non esistono sostituti stretti (``monopolio 
naturale'') oppure opera in ambito protetto (``monopolio legale'', protetto da 
barriere giuridiche).
% <ref>{{cita libro|titolo=Microeconomia|autore=[[Hal R. 
% Varian]]|editore=Cafoscarina|anno=1998|città=Venezia|pagina=403|edizione=4 
% ed.|isbn=88-85613-75-6|cid=Varian, 1998}}</ref> 
Consiste insomma 
nell'accentramento dell'offerta o della domanda del 
mercato di un dato bene o servizio nelle mani di un solo 
venditore o di un solo compratore.

\subsubsection{Storia}

Il primo grande monopolio della [[storia moderna]] fu la [[Compagnia britannica 
delle Indie orientali|Compagnia delle Indie orientali]], che per tutto il 
[[1600]] e gran parte del [[1700]] fu l'unica compagnia dell'[[Civiltà 
occidentale|Occidente]] a controllare il commercio di merci provenienti 
dall'[[Estremo Oriente]], e in particolare dell'[[Oceano Indiano]].

\subsubsection{Cause}

Una situazione di monopolio può crearsi come conseguenza di:

\begin{enumerate} [noitemsep]
 \item esclusività sul controllo di ''input'' essenziali (es. diamanti grezzi 
''[[De Beers]]'');
 \item economie di scala: i costi di produzione rendono ottimale la presenza 
di un solo produttore invece che di una moltitudine di produttori diversi. Ciò 
è dovuto al fatto che per quel singolo produttore la curva del costo medio di 
lungo periodo è decrescente, quindi un aumento della produzione, diluendo i 
costi su più unità di prodotto, ne riduce l'incidenza media (si viene a 
determinare un monopolio naturale); un esempio è il caso delle ferrovie o 
delle autostrade;
 \item brevetti;
 \item licenze governative.
\end{enumerate}

\subsubsection{Forme di monopolio}
I monopoli sono spesso caratterizzati in base alle circostanze da cui hanno 
origine. Tra le categorie principali si hanno monopoli che sono il risultato di 
leggi o regolamentazione (monopoli legali), monopoli che hanno origine dalla 
struttura dei costi di un dato sistema produttivo ([[monopolio naturale]]). I 
fautori del [[liberismo]] in [[economia]] sostengono che una classificazione 
più fondamentale dovrebbe distinguere tra monopoli che nascono e prosperano 
grazie a una violazione dei principi del [[libero mercato]] (monopolio 
coercitivo) e quelli che si mantengono tali grazie alla superiorità del 
prodotto o servizio offerto rispetto a quello dei potenziali concorrenti.

\paragraph{Monopolio legale}
Un monopolio basato su leggi che esplicitamente limitano la concorrenza le 
quali fungono da intermediazione dei diritti su opere tutelate di 
rappresentazione, esecuzione e recitazione, radiodiffusione riproduzione 
meccanica e cinematografica è detto monopolio legale (o ''de jure''). Il 
monopolio legale inoltre può proteggere l'interesse privato nella concessione 
di diritti esclusivi per offrire un servizio particolare in una regione 
specifica (ad es invenzioni brevettate), accettando di avere le loro politiche 
e dei prezzi controllati. Il monopolio legale è regolamentato dall'[[art 180]]
Un monopolio legale può assumere la forma di un monopolio di governo in cui lo 
Stato possiede i mezzi di produzione ([[monopolio di stato]]).

Un esempio classico per poter più facilmente comprendere la questione e la 
funzione di questo monopolio è: supponendo che il bene prodotto in regime di 
monopolio dovuto a brevetto sia una nuova medicina. Da un lato, la 
[[concorrenza perfetta]] consentirebbe un maggiore livello di produzione ad un 
prezzo più contenuto; dall'altro, se non ci fosse stata la possibilità di agire 
in una condizione di monopolio grazie al brevetto, il bene in questione non 
sarebbe stato introdotto sul mercato. Quando un'impresa investe tempo e denaro 
per sviluppare un nuovo prodotto desidera che tale investimento renda: il 
brevetto è un modo per garantire tale rendimento poiché, almeno per un certo 
numero di anni, l'impresa potrà raccogliere i frutti della propria inventiva.

Questo tipo di monopolio è di solito in contrasto con [[monopolio di fatto]] 
che è una vasta categoria di monopoli che non vengono creati dal governo.

\paragraph{Monopolio naturale}
Situazione in cui un'impresa è in grado di 
generare l'intera produzione del mercato a un costo inferiore a quello che 
sarebbe praticato in presenza di diverse imprese. 

% === Concorrenza monopolistica ===
% {{vedi anche|Concorrenza monopolistica}}

\subsection{Analisi economica}

Caratteristiche del monopolio:
\begin{description}
 \item [Venditore unico]
In un monopolio ``puro'', un'unica impresa è l'unico produttore di un bene, o 
il solo fornitore di un servizio, solitamente a causa di restrizioni 
all'entrata nel mercato. La definizione del mercato, e dunque della natura di 
monopolio, può essere d'altronde non univoca: ad esempio, l'impresa che produce 
il gelato ``A'' è monopolista nel mercato per il gelato ``A'', ma non nel 
mercato dei gelati in generale; questo porta al punto seguente.
 \item [Assenza di beni o servizi sostitutivi]
Il prodotto o servizio deve essere unico in una maniera che vada al di là 
della vera identità del marchio, e non può essere facilmente rimpiazzato (la 
[[Coca-Cola]], per esempio, ``non'' è un monopolista).
 \item [Comportamento da ``price maker'']
In un monopolio ``puro'', l'impresa monopolista controlla l'intera offerta del 
bene o servizio, ed è in grado di esercitare un rilevante controllo sul prezzo, 
cambiando la quantità prodotta, adottando, dunque, un comportamento da ``price 
maker'' (in opposizione al comportamento da ``price taker'' dell'impresa che 
opera in concorrenza perfetta).
 \item [Barriere all'entrata]
La ragione per cui un monopolista non ha concorrenti è che barriere di un 
qualche tipo limitano la possibilità che altre imprese accedano al mercato. A 
seconda della forma di monopolio, tali barriere possono essere economiche, 
tecniche, legali (ad es. nel caso di brevetti o concessioni), innocenti 
(ricerca-sviluppo, tecnologia, licenze, brevetti, economie di scala e curve di 
esperienza), strategiche (guerra dei prezzi, costi all'entrata, minaccia).
\end{description}

\subsection{Confronto tra Monopolio e Concorrenza Perfetta}

\begin{itemize} %[noitemsep]
 \item Il monopolista fa profitti positivi (P- Cm) Q, mentre l'industria in 
concorrenza perfetta non fa profitti.
 \item Dal punto di vista del consumatore è meglio trovarsi in una situazione 
di 
concorrenza perfetta poiché vi è una maggiore quantità di merci a prezzo più 
basso, mentre dal punto di vista del produttore è più vantaggioso il monopolio 
perché comporta ricavi maggiori.
 \item Per quanto riguarda il "benessere complessivo" (surplus del produttore + 
surplus del consumatore) in concorrenza perfetta i consumatori pagano di meno 
di quello che sarebbero disposti a pagare per ogni singolo prodotto,quindi la 
concorrenza perfetta rispetto al monopolio crea un maggiore benessere 
complessivo.

In alcuni casi però monopolio e concorrenza perfetta possono essere 
equivalenti dal punto di vista del benessere complessivo.
 \item \emph{[[Discriminazione di prezzo]] di primo grado}: caso teorico in cui 
il 
monopolista conosce perfettamente la disponibilità a pagare degli acquirenti 
(presuppone l'impossibilità che gli acquirenti si possano rivendere tra di loro 
i beni), allora il surplus del produttore risulta essere uguale al surplus del 
consumatore nel caso della concorrenza perfetta.
\end{itemize}

\subsection{Produzione in condizioni di monopolio}

A differenza delle imprese che operano in condizioni di concorrenza 
perfetta, l'impresa che opera in condizioni di monopolio deve soddisfare 
l'intera domanda di mercato per il suo prodotto. Si suppone che la domanda sia, 
``ceteris paribus'', una funzione decrescente del 
prezzo; rovesciando questa argomentazione, il prezzo del ``lato della 
domanda'', che i consumatori sono disposti a pagare per acquistare il prodotto, 
è una funzione decrescente della quantità offerta, 
\(\ p(q)\) tale che \(\ \frac{dp}{dq}<0\).

Il monopolista fissa la quantità di prodotto ottima \(\ q^{*}\) in 
maniera tale da rendere massimo il proprio profitto; risolve dunque 
implicitamente un problema di ottimizzazione:

\(\ q^{*}=\arg\max_{q}\pi(q)=p(q)q-c(q)\)

dove \(\ \pi(q)\) è la [[Funzione (matematica)|funzione]] profitto, 
``p(q)q'' sono i ricavi e ``c(q)'' denota i costi sostenuti per la produzione, 
anch'essi funzione della quantità prodotta. La condizione del primo ordine per 
un massimo è:

\(\ \frac{d\pi}{dq} = p(q)+p'(q)q-c'(q)=0\)

Le quantità \(\ p(q)+p'(q)q\) e ``c<nowiki>'</nowiki>(q)'' sono dette 
rispettivamente ``ricavo marginale'' 
% (MR, dall'[[Lingua inglese|inglese]] ''Marginal Revenue'') 
e ''costo marginale'' 
% (MC, dall'[[Lingua inglese|inglese]] ''Marginal Cost'')
; condizione per l'ottimalità della produzione è dunque:

\(\ \textrm{MR}(q)=\textrm{MC}(q)\)

Dividendo ambo i membri per la quantità non negativa ``p(q)'' e riorganizzando 
i termini, tale condizione può essere riscritta come:

\(\frac{p(q)-c'(q)}{p(q)}=-\frac{dp}{dq}\frac{q}{p(q)}=-\frac{1}{\eta}\)

dove \(\eta\) denota l'\emph{elasticità della domanda} rispetto al 
prezzo, \(\ \eta=\frac{p(q)}{q}\frac{dq}{dp}\) (ossia la variazione 
percentuale della quantità domandata in risposta a una variazione 
infinitesimale del prezzo). Dunque la condizione di ottimalità della produzione 
in condizioni di monopolio può scriversi come:

\(p(q)\left(1+\frac{1}{\eta}\right)=c'(q)\)

Quest'ultima espressione giustifica il cosiddetto ``indice di Lerner'' di 
potere di mercato, dato da:

\(\frac{p(q)}{c'(q)}=\frac{1}{1+\frac{1}{\eta}}\)

che misura la "distanza" del prezzo di mercato dal costo marginale, a cui 
sarebbe pari in condizioni di concorrenza perfetta, e del quale è maggiore 
in condizioni di monopolio.

In più, dobbiamo considerare le leggi dell'Economia Politica, le quali dicono 
che in assenza di monotonicità, convessità e transitività, le curve di 
indifferenza dei consumatori, in regime di monopolio, si intersecheranno e non 
si avrà più soddisfazione maggiore passando ad una curva di indifferenza più 
``alta''.
Stessa cosa varrà per il monopolista, anche se questa cosa riguarderà le curve 
isoquanti, i quali rappresentano le combinazioni di fattori che producono 
output diversi.

\subsection{Rappresentazione grafica}

[[File:Equilibrio monopolista.svg|frame|Curve dei costi medio e marginale, del 
prezzo e del ricavo marginale in regime di monopolio]]

Le curve del costo medio e del costo marginale sono identiche a quelle 
che si assumono per la concorrenza perfetta. Tuttavia:
\begin{itemize}
 \item il prezzo è funzione decrescente della quantità offerta;
 \item il ricavo marginale non è uguale al prezzo (come invece accade in 
concorrenza perfetta), ma è anch'esso decrescente; inoltre, un aumento delle 
vendite comporta una diminuzione del prezzo non solo per l'ultima unità 
venduta, ma anche per quelle che, prima delle maggiori vendite, avevano un 
prezzo più alto; ne segue che il ricavo marginale decresce più rapidamente del 
prezzo;
 \item l'impresa raggiunge il suo equilibrio nel punto in cui il costo marginale 
ed il ricavo marginale sono uguali, vendendo la quantità \(\ q^*\) al 
prezzo ``P''; in regime di concorrenza, l'impresa avrebbe venduto la maggiore 
quantità ``q<nowiki>'</nowiki>'' al prezzo inferiore ``P<nowiki>'</nowiki>'';
 \item l'impresa monopolistica consegue un maggiore profitto rispetto a quella 
concorrenziale; nel lungo periodo, infatti, l'impresa concorrenziale è in 
equilibrio quando sono uguali [[costo marginale]], [[costo medio]] e prezzo; 
ciò comporta che ricavi totali (quantità per prezzo) e costi totali (quantità 
per costo medio) sono uguali e il profitto è nullo;
% <ref>"Profitto nullo", si 
% rammenta, non vuol dire che il capitale non viene remunerato, in quanto i costi 
% comprendono la remunerazione del capitale come fattore di produzione.</ref> 
l'impresa monopolista, invece, sopporta un costo medio pari a ''C'', quindi 
costi totali pari al rettangolo \(\ 0CEq^{*}\) nella figura a lato, e 
ricavi pari a \(\ 0PAq^{*}\), con un profitto pari a \(CPAE\);
 \item il consumatore, dovendo sopportare un maggior prezzo, perde una parte del 
suo surplus, quella corrispondente al trapezio \(P'PAF\);
 \item l'impresa monopolistica si appropria di una parte del surplus perso dal 
consumatore, il rettangolo \(P' P A B\), ma, vendendo una quantità 
minore di quella che avrebbe venduto in concorrenza, perde la parte del proprio 
surplus corrispondente alla regione \(BDF\);
 \item la parte del surplus del consumatore di cui l'impresa non si appropria, 
il 
triangolo \(ABF\), e la parte del surplus del produttore perso 
dall'impresa (\(BDF\)) costituiscono, insieme, la cosiddetta "perdita 
netta di monopolio" (\(ADF\)).
\end{itemize}

\subsubsection{Monopolio come fallimento del mercato}

Il monopolio può dar luogo a un [[fallimento del mercato]]; esso dà infatti 
adito a una perdita secca di [[surplus del consumatore]] rispetto alla 
[[concorrenza perfetta]]; tuttavia, qualora fosse fornito un sussidio alla 
produzione in modo tale che questa raggiunga lo stesso livello che avrebbe in 
condizioni di concorrenza perfetta, il benessere sarebbe comunque massimizzato. 
Tale passaggio sposta però il problema sulla equità nella distribuzione del 
surplus piuttosto che sulla sua massimizzazione.
Per questa ragione in democrazia i monopoli (e gli [[oligopoli]]) privati sono 
combattuti con leggi dello Stato, fatto salvo per i monopoli statali, che di 
solito riguardano beni o servizi di particolare importanza per la comunità, che 
in questo caso sono di proprietà di tutti i cittadini.

% == Note ==
% <references />
% 
% == Voci correlate ==
% * [[Cartello]]
% * [[Concorrenza monopolistica]]
% * [[Fallimento del mercato]]
% * [[Mercato contendibile]]
% * [[Monopolio naturale]]
% * [[Oligopolio]]
% 
% == Altri progetti ==
% {{interprogetto|etichetta=monopolio|wikt=monopolio}}
% 
% {{Controllo di autorità}}
% {{Portale|economia}}
% 
% [[Categoria:Economia industriale]]
% [[Categoria:Economia politica]]
% [[Categoria:Microeconomia]]


\begin{comment}
{{S|microeconomia}}

Un \emph{duopolio} costituisce una situazione limite della struttura di 
mercato [[oligopolio|oligopolistico]], in cui operino due sole imprese che 
offrano prodotti identici, con costi marginali simili, e che entrambe conoscano 
le informazioni di domanda (simmetria informativa) e l'impatto delle mosse del 
concorrente sulla propria situazione.

È una concezione teorica, sviluppata per studiare ed evindenziare le 
caratteristiche del modello [[oligopolio|oligopolista]].

Tale modello di riferimento è la base delle analisi sull'oligopolio condotte da 
economisti di grande rilievo, quali [[Antoine Augustine Cournot|Cournot]], 
[[Joseph Louis François Bertrand|Bertrand]], [[Modello di Hotelling|Hotelling]] 
e [[John Nash|Nash]].

Il modello di duopolio è usato come riferimento per le [[teoria dei 
giochi|teorie dei giochi]].

== Il duopolio televisivo in Italia ==
In Italia il termine ''duopolio'' è dai ''media'' usato specialmente per 
descrivere la situazione che si ha in campo televisivo per quello che riguarda 
la ''televisione analogica'', dove i due principali ''competitors'' , la 
[[Rai]] e [[Mediaset]] distanziavano in modo molto evidente gli altri 
comprimari.

La rottura del ''monopolio televisivo'' sancita dalle sentenze della [[Corte 
Costituzionale]] del 1976, determinò il sorgere di una ''situazione di fatto'' 
non regolamentata da provvedimenti legislativi, con una pluralità di soggetti 
che lanciavano diverse iniziative. La crisi di diversi di questi tentativi vide 
un concentrarsi della televisioni commerciali in un gruppo che faceva capo a 
Silvio Berlusconi.

La catena di trasmittenti televisive riuscivano a superare l'ambito locale, 
inizialmente previsto, con alcuni artifici, come la cosiddetta interconnessione 
(trasmissione dello stesso contenuto in differita rispetto all'ambito locale 
milanese, ma una differita di alcuni secondi soltanto): il [[Pretore 
(ordinamenti moderni)|pretore]] di Roma, così come altri due pretori, basandosi 
su una normativa tratta dal [[codice postale]], aveva stabilito la 
illegittimità di tale situazione.

Il "decreto [[Berlusconi]]" creò il duopolio: esso fu varato da [[Craxi]] che 
ottenne dal Governo da lui presieduto il varo di un decreto-legge ristabilire 
le frequenze dei canali [[Fininvest]] di [[Silvio Berlusconi]] chiusi 
dall'ordinanza del pretore. La misura fu preceduta da un'accorta regia 
mediatica di [[Berlusconi]], facendo inondare di telefonate furenti il 
centralino di Palazzo Chigi e gli apparecchi dei tre pretori "colpevoli", da 
parte di telespettatori desiderosi "di godersi in santa pace le proprie serate 
televisive: Dynasty, Dallas, i Puffi... Quando infine Berlusconi piomba a Roma, 
i giornali raccontano già ampiamente questa levata di scudi dei 
telespettatori"<ref>"Inchiesta sul signor TV", di Giovanni Ruggeri e Mario 
Guarino, Kaos ed. 1994.</ref>.

In buona parte dell'[[opinione pubblica]] si diffuse l'idea che Craxi 
proteggesse politicamente Berlusconi e quest'ultimo gli concedesse ampio spazio 
nelle sue [[televisione|televisioni]]; Craxi e Berlusconi tra l'altro erano 
legati da una lunga e stretta [[amicizia]]. Altri invece ribadiscono che il 
decreto rientrava in un progetto a largo raggio di Craxi per scardinare il 
[[monopolio]] della [[Rai]] e aprire alla concorrenza il [[mercato]] 
televisivo<ref>"Craxi voleva rompere gli schemi del monopolio 
dell'informazione": così Rino Formica nell'intervista a Claudio Sabelli 
Fioretti per “La Stampa” del 10 dicembre 2008.</ref>.

La conversione del decreto in legge fu abbastanza travagliata, essendosi 
arenata sullo scoglio della decadenza per mancato riconoscimento dei 
presupposti costituzionali di necessità ed urgenza. In base alla prassi 
dell'epoca, il decreto fu reiterato e, trascorsi i sessanta giorni prescritti 
dall'art. 77 della [[Costituzione della Repubblica Italiana]], il provvedimento 
fu convertito dal Parlamento solo grazie ad una precisa iniziativa politica di 
Craxi, che minacciò la crisi di governo e le elezioni anticipate. A poche ore 
dal termine ultimo per la conversione, i parlamentari del [[Partito Comunista 
Italiano]] garantirono il numero legale con la loro presenza, senza porre in 
essere alcuno ostruzionismo, consentendo così la conversione del decreto. 
L'apporto del [[Partito Comunista Italiano]] fu determinante: come 
contropartita, i comunisti avrebbero ricevuto il placet di Craxi per ottenere 
il controllo di [[Raitre]]<ref>Michele De Lucia “Il baratto. Il Pci e le 
televisioni. Le intese e gli scambi tra il comunista Veltroni e l'affarista 
Berlusconi negli anni Ottanta” (Kaos edizioni)</ref>; secondo altri, invece, 
l'errore politico del PCI fu di temere una sconfitta elettorale oppure di 
preferire che la legislatura avesse seguito nell'erronea convinzione che lo 
svolgimento del referendum sulla scala mobile avrebbe visto la prevalenza dei 
sì all'abolizione del decreto di San Valentino.

Il duopolio fu poi consacrato nel 1990 dalla "[[Legge Mammì]]".

L'anomalia di tale situazione è stata più volte rilevata dalla [[Corte 
costituzionale della Repubblica Italiana|Corte 
Costituzionale]]<ref>[
http://www.uonna.it/466-2002-sentenza-corte-costituzionale.htm Sentenza del 
2002]</ref> e dal [[Parlamento della Repubblica Italiana|Parlamento]] 
<ref>[
http://new.camera.it/_dati/leg14/lavori/bollet/200301/0123/html/0709//comunic.ht
m Atti parlamentari]</ref>. Gli interventi legislativi che si sono succeduti 
sull'argomento non hanno di fatto portato alcun concreto temperamento al 
problema, che ha spinto il legislatore a puntare sull'anticipazione dei tempi 
dell'introduzione della [[televisione digitale]].

% == Note ==
% <references/>
% 
% == Voci correlate ==
% * [[Oligopolio]]
% * [[Antoine Augustin Cournot]]
% * [[Joseph Louis François Bertrand]]
% * [[John Nash]]
% * [[Teoria dei giochi]]
% 
% [[Categoria:Mercati finanziari]]
% [[Categoria:Microeconomia]]
% [[Categoria:Teoria dei giochi]]



* [[Equilibrio economico generale]]: [[scatola di Edgeworth]], esistenza, 
unicità e stabilità


In [[economia]] la teoria dell<nowiki>'</nowiki>'''[[equilibrio economico]] 
generale} cerca di spiegare come domanda, offerta e prezzi di diversi 
prodotti siano interrelati e determinati simultaneamente in un esito denominato 
di "equilibrio generale". Per contro, l'analisi di [[equilibrio parziale]] 
analizza domanda, offerta e prezzi di singoli mercati. La teoria 
dell'equilibrio economico generale risale al lavoro pionieristico di [[Léon 
Walras]], che nel suo ''Éléments d'économie politique pure'' (1899, 4th ed.; 
1926, éd. définitive) ipotizzò l'esistenza di un insieme di prezzi per il quale 
la domanda e l'offerta di ciascun prodotto potessero equivalersi. Fu solo nel 
1952 che gli economisti matematici [[Lionel McKenzie]] e, separatamente in un 
lavoro congiunto, [[Kenneth Arrow]] e [[Gérard Debreu]] dimostrarono 
matematicamente l'esistenza di tale equilibrio in condizioni generali.  

== Introduzione ==
L'idea alla base della teoria dell'equilibrio economico generale è che in un 
sistema di mercato, i prezzi e le scelte di produzione e di consumo dei diversi 
beni (ivi compresi "beni" quali il denaro, o "prezzi" quali il tasso 
d'interesse) siano interrelati. Un cambiamento nel prezzo di un bene, ad 
esempio il pane, potrebbe influenzare un altro prezzo, ad esempio il salario 
dei panettieri. A seconda delle [[utilità (economia)|preferenze]] dei 
panettieri, la [[domanda e offerta#Domanda|domanda]] di pane sarà allora 
influenzata dalla variazione nel salario dei panettieri, andando a incidere 
nuovamente sul prezzo del pane, e così via. Dunque, la determinazione del 
prezzo di un singolo bene risulta potenzialmente collegata a quella del prezzo 
di qualunque altro bene nell'intera economia.

La teoria dell'equilibrio generale spiega come un'economia decentralizzata, 
composta da numerosi agenti indipendenti che agiscono secondo il loro 
interesse, sia compatibile con un equilibrio su tutti i mercati. Questo 
equilibrio è ottenuto senza che ci sia un organismo che si occupa della 
logistica economica. Si cita sovente il caso di una grande città dove nessuno è 
incaricato della distribuzione del pane e del latte. Ciononostante, c'è 
abbastanza pane e latte per tutti gli abitanti. 
[[Adam Smith]] parla di una [[mano invisibile]] che conduce gli agenti verso un 
equilibrio che ha molte proprietà interessanti.
L'esistenza dell'equilibrio generale è stata studiata per la prima volta da 
[[Léon Walras]]. Gli studi furono continuati da [[Vilfredo Pareto]] e altri 
discepoli della Scuola di Losanna. Il nome è dovuto al luogo dove Walras e 
Pareto insegnavano. 

La prima prova rigorosa dell'esistenza dell'equilibrio generale è stata 
presentata da [[Abraham Wald]] nel 1936. Dopo la Seconda guerra mondiale, 
[[Gérard Debreu]] e [[Kenneth Arrow]] hanno pubblicato delle prove più generali 
e più complete (v. [[Modello di Arrow-Debreu]]).
L'equilibrio generale suppone che i consumatori e, in generale, tutti gli 
agenti economici, considerano i prezzi come un dato (''price taker'') e, su 
questa base, esprimono le loro domande e offerte. Per esempio, se il prezzo 
dello zucchero aumenta, la domanda di zucchero ma anche quella di miele cambia, 
come pure la produzione di barbabietole. Ci sono dunque degli effetti diretti e 
indiretti che influiscono sul prezzo d'equilibrio. L'equilibrio generale è 
ottenuto quando su tutti i mercati la domanda è uguale all'offerta. Walras dà 
l'esempio del banditore d'asta che grida un prezzo e guarda se la [[domanda e 
offerta#Domanda|domanda]] è uguale all'[[offerta]]. Se c'è una differenza, 
grida un prezzo più alto quando l'offerta è insufficiente e un prezzo più basso 
nel caso contrario. Si arriva all'equilibrio dopo alcuni tentativi 
(Tâtonnement). Tutti gli scambi si fanno al prezzo d'equilibrio.

== La legge di Walras ==

{{vedi anche|Legge di Walras}}

Walras aveva osservato che la domanda dipende solo dai prezzi relativi. Se un 
contadino si reca al mercato per vendere delle mele e comperare del pane e i 
prezzi raddoppiano, la sua domanda e la sua offerta non cambiano. Infatti, la 
spesa raddoppia ma pure il ricavo. La domanda e l'offerta possono allora essere 
espresse in termini di prezzi relativi. Nel caso di \emph{m} beni, abbiamo 
dunque solo \emph{m-1} variabili. Secondo la teoria dei sistemi d'equazioni 
lineari, che Walras conosceva, c'è una soluzione unica quando le equazioni 
indipendenti sono pure m-1.
Walras è riuscito a mostrare che effettivamente ciò era il caso. Questo 
risultato è chiamato oggi la [[legge di Walras]]: se \emph{m-1} mercati sono 
in equilibrio allora il mercato restante deve pure essere in equilibrio.

== Esistenza dell'equilibrio ==

Le funzioni di domanda non sono necessariamente delle [[Funzione 
lineare|funzioni lineari]]. La legge di Walras non permette dunque di mostrare 
l'esistenza dell'equilibrio generale. Le dimostrazioni moderne utilizzano 
[[topologia|approcci topologici]] all'esistenza del [[punto fisso]], come il 
[[Teorema del punto fisso di Brouwer]] o [[Teorema di Kakutani|quello più 
generale]] di [[Shizuo Kakutani]]. 

Si tratta di puri [[teoremi di esistenza]], non [[Costruttivismo 
matematico|costruttivi]], che non forniscono alcuna indicazione su come trovare 
il punto fisso. Nel 1967, Scarf ha proposto un algoritmo di calcolo del punto 
fisso. I modelli d'equilibrio generale hanno allora potuto essere applicati per 
risolvere dei problemi concreti, come gli effetti di un cambiamento del sistema 
fiscale o la soppressione dei dazi doganali.

== Unicità e stabilità dell'equilibrio generale ==

Le condizioni per un equilibrio unico o stabile sono molto restrittive. Se un 
equilibrio non è stabile, sarà difficile trovarlo. Il [[teorema 
Sonnenschein–Mantel–Debreu]] stabilisce che, in caso di mercati 
interdipendenti, è impossibile escludere casi di instabilità dell'equilibrio. 
Modelli con equilibri multipli sono proposti in certi casi.

% ==Voci correlate==
% * [[Economia classica]]
% * [[Economia keynesiana]]
% 
% == Bibliografia ==
% 
% * [[Kenneth Arrow]] and [[Frank Hahn|Frank H. Hahn]], ''General Competitive 
% Analysis'', San Francisco, 1971
% * [[Luitzen Brouwer|Luitzen E.J. Brouwer]], “Uber Abbildung von 
% Mannigfaltigkeiten”, Mathematische Annalen, vol. 71, 1911, pp. 97-115
% * [[Vilfredo Pareto]], ''Manuel d'économie politique'', 1897
% * H. Scarf, ''The Computation of Economic Equilibria'', New Haven, 1973
% * [[Léon Walras]], ''Eléments d'économie politique pure'', Lausanne, 1874
% 
% [[Categoria:Microeconomia]]
% [[Categoria:Economia matematica]]






* Ottimo economico: [[ottimo paretiano]], efficienza della concorrenza 
perfetta, funzione di utilità sociale ([[teorema dell'impossibilità di 
Arrow]]), [[teoremi dell'economia del benessere]]

* [[Esternalità]] e [[beni pubblici]]: teorema di [[Paul Samuelson|Samuelson]], 
[[teorema di Coase]], [[modello di Lindahl]], [[rivelazioni delle preferenze]] 
(Clarke, Groves)

* Economia dell'informazione: [[azzardo morale]], [[asimmetria informativa]], 
[[selezione avversa]], [[il mercato dei limoni]], modello di Spence, [[modello 
principale-agente]]

\subsection{Ipotesi alla base del comportamento del consumatore}

La teoria microeconomica, nella sua versione mainstream, pone alla base della 
sua analisi due ipotesi fondamentali più una accessoria:
\emph{Completezza}: Il consumatore, se posto di fronte alla scelta tra due 
panieri di beni (es: X e Y), sa sempre dire quale delle due preferisce o se gli 
sono indifferenti;
\emph{Transitività}: Avendo date quantità di tre beni \emph{X}, \emph{Y} 
e \emph{Z}, se il consumatore preferisce una unità di \emph{X} a una unità 
di \emph{Y} e una unità di \emph{Y} a una unità di \emph{Z} allora preferirà 
naturalmente anche una unità di \emph{X} a una di \emph{Z}.
\emph{Non Sazietà}: Il consumatore è via via più soddisfatto se consuma 
panieri che hanno la stessa quantità del bene X e una quantità via via maggiore 
del bene Y.
\emph{Continuità}: Le curve di indifferenza (ossia gli insiemi dei panieri 
tra cui il consumatore è indifferente) sono funzioni continue.
\emph{Convessità stretta}: Dato un paniere X' l'insieme dei panieri X 
preferiti a X' è strettamente convesso.

La \emph{Teoria della preferenza rivelata} propone di ottenere le preferenze 
del consumatore osservando il suo comportamento.



\begin{comment}
 
\subsection{Voci correlate}
* [[Ricavo]]
* [[Prezzo]]
* [[Costo marginale]]
* [[Utilità marginale]]
* [[Produttività marginale]]
* [[Moltiplicatore monetario]]
* [[Rendita finanziaria]]
* [[Avversione al rischio]]

\subsection{Altri progetti}
{{
interprogetto|wikt|v=Materia:Microeconomia|v_preposizione=sulla|etichetta=microe
conomia}}

\subsection{Collegamenti esterni}
* {{Thesaurus BNCF}}
{{Controllo di autorità}}
{{Portale|economia}}

[[Categoria:Microeconomia| ]]






\section{Note}
<references />

\subsection{Bibliografia}
* {{cita libro|autore=Pierluigi Ciocca|titolo=Il tempo dell'economia. 
Strutture, fatti, interpreti del Novecento|editore=Bollati 
Boringhieri|anno=2004, ISBN 978-88-339-1559-3}}
* {{cita libro|autore=Sidney Pollard|titolo=L'economia internazionale dal 
1945 ad oggi|editore=Editori Laterza|anno=1999 |ISBN= 88-420-5791-6}}
* {{cita libro|autore=André Gauthier|titolo=L'economia mondiale dal 1945 
a oggi|editore=Il Mulino|anno=1998 |ISBN= 88-15-06381-1}}

\subsection{Voci correlate}
{{div col|cols=4}}
* Autarchia
* Capitalismo
* Commercio
* Dirigismo
* Economia aziendale
* Economia del benessere
* Economia del lavoro
* Economia della conoscenza
* Economia dell'informazione
* Economia dello sviluppo
* Economia d'Italia
* Economia di mercato
* Economia e politica agraria
* Economia finanziaria
* Economia industriale
* Economia internazionale
* Economia keynesiana
* Economia mista
* Economia mondiale
* Economia monetaria
* Economia neoclassica
* Economia pianificata
* Economia politica
* Economisti classici
* Finanza
* Fisiocrazia
* Globalizzazione
* Glossario economico
* Liberismo
* Macroeconomia
* Marginalismo
* Mercantilismo
* Mercato
* Microeconomia
* Monetarismo
* Politica
* Politica economica
* Settore economico
* Statalismo
* Statistica economica
* Storia del pensiero economico
* Storia economica
{{div col end}}

\subsection{Altri progetti}
{{interprogetto|etichetta=economia|wikt=economia|s=Categoria:Testi di 
economia|s_preposizione=di|q|commons=Category:Economics}}

\subsection{Collegamenti esterni}
* {{Thesaurus BNCF}}
* {{cita web|http://www.treccani.it/enciclopedia/tag/economia/|Lemma 
Economia 
sull’Enciclopedia Treccani online}}
* {{cita web|http://www.sapere.it/enciclopedia/econom%C3%ACa.html|Lemma 
Economia nell’Enciclopedia Sapere online}}

{{Scienze sociali}}

{{Controllo di autorità}}
{{Portale|economia}}

Categoria:Economia| Economia



\end{comment}









































