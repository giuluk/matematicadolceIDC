%%%%%%%%%%%%%%%%%%%%%%%%%%%%%%%%%%%%%%%%%%%%%%%%%%%%%%%%%%%%%%%%%%%%
%        Matematica dolce
%
% Copyright 2016 Daniele Zambelli
%
%------------------------------
% Matematica dolce per i licei Linguistico e economico sociale, volume 1
%
% m_d_licei_17_1.tex
%------------------------------
%
% This work may be distributed and/or modified under the
% conditions of the LaTeX Project Public License, either version 1.3
% of this license or (at your option) any later version.
% The latest version of this license is in
%   http://www.latex-project.org/lppl.txt
% and version 1.3 or later is part of all distributions of LaTeX
% version 2005/12/01 or later.
%
% This work has the LPPL maintenance status `maintained'.
%
% The Current Maintainer of this work is
% Daniele Zambelli - daniele.zambelli@gmail.com
%
% This work consists of the files:
%  -  m_d_1_les.tex (this file)
%  -  createpdf.py
%  -  readme.md
%  -  the part of code of the files under \matdir , img/  and lbr/ directories
%%%%%%%%%%%%%%%%%%%%%%%%%%%%%%%%%%%%%%%%%%%%%%%%%%%%%%%%%%%%%%%%%%%%

%========================
% Definizione delle directory
%========================
% \newcommand{\basedir}{matematicadolce/}
\newcommand{\depdir}{deposito/}
\newcommand{\magdir}{\depdir magazzino/}
\newcommand{\matdir}{\depdir materiali/}

\newcommand{\moddir}{modifiche/}                    % MODIFICATO

%========================
% Variabili del progetto
%========================
\input{\magdir variabili}

%========================
% Variabili per questo volume
%========================
\newcommand{\editore}{Matematicamente.it}
\newcommand{\serie}{Matematica \(C^3\)}
\newcommand{\titolo}{Geometria Razionale}     % MODIFICATO
\newcommand{\pdftitolo}{Matematica dolce}  % MODIFICATO
\newcommand{\docvers}{\texttt{5.0.1}}
\newcommand{\edizione}{2019}
\newcommand{\Edizione}{Edizione}
\newcommand{\tipo}{ (versione completa)}
\newcommand{\descr}{Testo per il primo biennio \protect\\
                    della Scuola Secondaria di II grado \\ \null
                    istituto professionale \\       % MODIFICATO
                    servizi socio-sanitari}         % MODIFICATO
\newcommand{\oggi}{19 settembre 2019}               % MODIFICATO
\newcommand{\mese}{settembre}                       % MODIFICATO
\newcommand{\anno}{2019}
\newcommand{\mcisbn}{}

%========================
% Lettura preambolo
%========================
\input{\magdir packages}
\input{\magdir definizioni}
\input{\magdir definizioni_tikz}

%========================
% Con o senza linguaggio di programmazione
%========================
\newif\ifcoding
%\codingtrue    % commentare questa linea se NON si vuole il coding
\codingfalse    % commentare questa linea se si vogliono le parti di coding

%========================
% Documento
%========================
\begin{document}
\frontmatter
\intestazione{\matdir 00_intestazioni/les/}{frontespizio}
\intestazione{\matdir 00_intestazioni/les/}{colophon}
\intestazione{\matdir 00_intestazioni/les/}{indice}
\intestazione{\moddir 00_intestazioni/les/}{prefazione} % MODIFICATO
% ..................................................
%
% Terzo tema: Geometria
%
\mainmatter
\parte{\moddir 0_a_parti/}{part_03_1_d}
\capitolo{\matdir 05_Geometria_euclidea/01_Assiomi_teoremi/fond01/}
         {fondamenti1}
\capitolo{\matdir 05_Geometria_euclidea/02_Triangoli_rette/tri01/}
         {triangoli1}
\capitolo{\matdir 05_Geometria_euclidea/02_Triangoli_rette/parall01/}
         {parallelismo1}
\capitolo{\matdir 05_Geometria_euclidea/02_Triangoli_rette/quadr01/}
         {quadrilateri1}
\capitolo{\matdir 05_Geometria_euclidea/03_Circonferenza/circ01/}
         {circonferenza1}
\capitolo{\matdir 05_Geometria_euclidea/04_Equiestensione/equie01/}
         {equiestensione1}
\capitolo{\matdir 05_Geometria_euclidea/04_Equiestensione/prop01/}
         {proporzionalita1}
%..................................................
%
% Azzeramento numerazione capitoli
%
\renewcommand{\thechapter}{\Alph{chapter}}
\setcounter{chapter}{0}

\end{document}
