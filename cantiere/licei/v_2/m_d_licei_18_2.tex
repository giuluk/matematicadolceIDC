%%%%%%%%%%%%%%%%%%%%%%%%%%%%%%%%%%%%%%%%%%%%%%%%%%%%%%%%%%%%%%%%%%%%
%        Matematica dolce
%
% Copyright 2016 Daniele Zambelli
%
%------------------------------
% Matematica dolce per i licei Linguistico e economico sociale, volume 1
%
% m_d_2_les.tex
%------------------------------
%
% This work may be distributed and/or modified under the
% conditions of the LaTeX Project Public License, either version 1.3
% of this license or (at your option) any later version.
% The latest version of this license is in
%   http://www.latex-project.org/lppl.txt
% and version 1.3 or later is part of all distributions of LaTeX
% version 2005/12/01 or later.
%
% This work has the LPPL maintenance status `maintained'.
% 
% The Current Maintainer of this work is 
% Dimitrios Vrettos - d.vrettos@gmail.com
%
% This work consists of the files:
%  -  m_d_2_les.tex (this file)
%  -  createpdf.py
%  -  readme.md
%  -  the part of code of the files under \matdir , img/  and lbr/ directories
%%%%%%%%%%%%%%%%%%%%%%%%%%%%%%%%%%%%%%%%%%%%%%%%%%%%%%%%%%%%%%%%%%%%

%========================
% Definizione delle directory
%========================
% \newcommand{\basedir}{matematicadolce/}
\newcommand{\depdir}{deposito/}
\newcommand{\magdir}{\depdir magazzino/}
\newcommand{\matdir}{\depdir materiali/}

%========================
% Variabili del progetto
%========================
\input{\magdir variabili}

%========================
% Variabili per questo volume
%========================
\newcommand{\editore}{Matematicamente.it}
\newcommand{\serie}{Matematica $C^3$}
\newcommand{\titolo}{Matematica dolce 2 - licei}        
\newcommand{\pdftitolo}{Matematica dolce 2 - licei}
\newcommand{\docvers}{\texttt{4.0.1}}
\newcommand{\edizione}{2018}
\newcommand{\Edizione}{Edizione}
\newcommand{\tipo}{ (versione completa a colori)}
\newcommand{\descr}{Testo per il primo biennio \protect\\ 
                    della Scuola Secondaria di $II$ grado \\ \null 
                    licei}
\newcommand{\oggi}{27 giugno 2018}
\newcommand{\mese}{giugno}
\newcommand{\anno}{2018}
\newcommand{\mcisbn}{9788899988012}

%========================
% Lettura preambolo
%========================
\input{\magdir packages}
\input{\magdir definizioni}
\input{\magdir definizioni_tikz}

%========================
% Con o senza linguaggio di programmazione
%========================
\newif\ifcoding
\codingtrue     % commentare questa linea se NON si vuole il coding
% \codingfalse    % commentare questa linea se si vogliono le parti di coding

%========================
% Documento
%========================
\begin{document}
\frontmatter
% \intestazione{\matdir 00_intestazioni/les/}{frontespizio}
% \intestazione{\matdir 00_intestazioni/les/}{colophon}
% \intestazione{\matdir 00_intestazioni/les/}{indice}
% \intestazione{\matdir 00_intestazioni/les/}{prefazione}

%..................................................
%%
%% Primo tema: Aritmetica e algebra
%%
\mainmatter
\parte{\matdir 0_a_parti/}{part_01_2_d}
% \capitolo{\matdir 02_Insiemi_numerici/05_Numeri_reali/re01/}
%          {reali2}
\capitolo{\matdir 03_Polinomi/05_Radicali/rad01/}
         {radicali1}
%..................................................
%%
%% Secondo tema: Geometria
%%
\parte{\matdir 0_a_parti/}{part_02_2_d}
\capitolo{\matdir 07_Geometria_analitica/02_Retta/retta01/}
         {retta}
\capitolo{\matdir 05_Geometria_euclidea/02_Triangoli_rette/parall01/}
         {parallelismo1}
\capitolo{\matdir 05_Geometria_euclidea/02_Triangoli_rette/quadr01/}
         {quadrilateri1}
\capitolo{\matdir 05_Geometria_euclidea/04_Equiestensione/equie01/}
         {equiestensione1}
\capitolo{\matdir 06_Trasformazioni/01_Trasf_geometriche/trasf01/}
         {trasformazioni}
\capitolo{\matdir 06_Trasformazioni/02_Isometrie/isom01/}
         {isometrie1}
%..................................................
%%
%% Terzo tema: Relazioni e funzioni
%%
\parte{\matdir 0_a_parti/}{part_03_2_d}
\capitolo{\matdir 04_Equazioni/02_Diseq_grado_1/dis01/}
         {disequazioni}
\capitolo{\matdir 04_Equazioni/03_Sistemi_lineari/sist01/}
         {sistemi}
%%
%% Quarto tema: Dati e previsioni
%%
\parte{\matdir 0_a_parti/}{part_04_2_d}
\capitolo{\matdir 15_Probabilita/02_Calcolo_probabilita/prob01/}
         {probabilita}
%..................................................
%%
%% Quinto tema: Elementi di informatica
%%
\parte{\matdir 0_a_parti/}{part_05_2_d}
\capitolo{\matdir 17_Strumenti_digitali/02_Python/02_Geo_inter/geoint01/}
         {geointerattiva2}
%..................................................
%%
%% Azzeramento numerazione capitoli
%%
\renewcommand{\thechapter}{\Alph{chapter}}
\setcounter{chapter}{0}

\end{document}
