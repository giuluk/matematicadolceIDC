%%%%%%%%%%%%%%%%%%%%%%%%%%%%%%%%%%%%%%%%%%%%%%%%%%%%%%%%%%%%%%%%%%%%
%        Matematica dolce
%
% Copyright 2016 Daniele Zambelli
%
%------------------------------
% Matematica dolce per i licei Linguistico e economico sociale, volume 1
%
% m_d_1_les.tex
%------------------------------
%
% This work may be distributed and/or modified under the
% conditions of the LaTeX Project Public License, either version 1.3
% of this license or (at your option) any later version.
% The latest version of this license is in
%   http://www.latex-project.org/lppl.txt
% and version 1.3 or later is part of all distributions of LaTeX
% version 2005/12/01 or later.
%
% This work has the LPPL maintenance status `maintained'.
% 
% The Current Maintainer of this work is 
% Daniele Zambelli - daniele.zambelli@gmail.com
%
% This work consists of the files:
%  -  m_d_1_les.tex (this file)
%  -  createpdf.py
%  -  readme.md
%  -  the part of code of the files under \matdir , img/  and lbr/ directories
%%%%%%%%%%%%%%%%%%%%%%%%%%%%%%%%%%%%%%%%%%%%%%%%%%%%%%%%%%%%%%%%%%%%

%========================
% Definizione delle directory
%========================
% \newcommand{\basedir}{matematicadolce/}
\newcommand{\depdir}{deposito/}
\newcommand{\magdir}{\depdir magazzino/}
\newcommand{\matdir}{\depdir materiali/}

%========================
% Variabili del progetto
%========================
\input{\magdir variabili}

%========================
% Variabili per questo volume
%========================
\newcommand{\editore}{Matematicamente.it}
\newcommand{\serie}{Matematica $C^3$}
\newcommand{\titolo}{Matematica dolce 4 - licei}        
\newcommand{\pdftitolo}{Matematica dolce 4 - licei}
\newcommand{\docvers}{\texttt{0.0.9}}
\newcommand{\edizione}{2017}
\newcommand{\Edizione}{Edizione}
\newcommand{\tipo}{ (versione completa a colori)}
\newcommand{\descr}{Testo per il secondo biennio \protect\\ 
                    della Scuola Secondaria di $II$ grado \\ \null 
                    licei}
\newcommand{\oggi}{27 giugno 2017}
\newcommand{\mese}{giugno}
\newcommand{\anno}{2017}
\newcommand{\mcisbn}{9788899988036}

%========================
% Lettura preambolo
%========================
\input{\magdir packages}
\input{\magdir definizioni}
\input{\magdir definizioni_tikz}

%========================
% Caratteri sans serif
%========================
\renewcommand{\familydefault}{\sfdefault}

%========================
% Con o senza linguaggio di programmazione
%========================
\newif\ifcoding
\codingtrue     % commentare questa linea se NON si vuole il coding
% \codingfalse    % commentare questa linea se si vogliono le parti di coding

%========================
% Documento
%========================
\begin{document}
\frontmatter
\intestazione{\matdir 00_intestazioni/les/}{frontespizio}
\intestazione{\matdir 00_intestazioni/les/}{colophon}
\intestazione{\matdir 00_intestazioni/les/}{indice}
\intestazione{\matdir 00_intestazioni/les/}{prefazione}

%..................................................
%% --------------------------------
%% Capitoli
%% --------------------------------
%....... Proposta di Luciana ......................
% Complementi di algebra (irrazionali, valori assoluti)
% Ellisse
% Iperbole
% Funzioni esponenziale e logaritmica
% Calcolo approssimato
% Calcolo combinatorio
% Probabilità
% Modelli
% Geometria euclidea nello spazio
%..................................................
\mainmatter
\parte{\matdir 0_a_parti/}{part_01_d}
\capitolo{\matdir 04_Equazioni/06_Grado_2+/compl201/}
         {grasup2_irraz_valass}
\capitolo{\matdir 05_Geometria_euclidea/03_Circonferenza/circ01/}
         {circonferenza1}
\capitolo{\matdir 07_Geometria_analitica/04_Circonferenza/circ01/}
         {circonferenza_pc}
\capitolo{\matdir 07_Geometria_analitica/05_Ellisse_iperbole/ell01/}
         {ellisse}
\capitolo{\matdir 07_Geometria_analitica/05_Ellisse_iperbole/iper01/}
         {iperbole}
\capitolo{\matdir 07_Geometria_analitica/06_Luoghi_geometrici/luo01/}
         {luoghigeometrici}
\capitolo{\matdir 10_Esp_log/01_Funz_eq_diseq/esplog01/}
         {esp_log}
% TODO \capitolo{\matdir 01_aritmeticaealgebra/11_calc_approssimato/}
%               {calc_approssimato }
\capitolo{\matdir 15_Probabilita/01_Calcolo_combinatorio/comb01/}
         {calc_combinatorio}
\capitolo{\matdir 15_Probabilita/02_Calcolo_probabilita/prob01/}
          {probabilita}
\capitolo{\matdir 16_Economia/01_Funzioni_economiche/funzec01/}
         {funzionieconomiche}
\capitolo{\matdir 07_Geometria_analitica/08_Geometria_3d/g3d01/}
         {geometria3d}
%..................................................
%
%% Azzeramento numerazione capitoli
%%
\renewcommand{\thechapter}{\Alph{chapter}}
\setcounter{chapter}{0}

\end{document}
