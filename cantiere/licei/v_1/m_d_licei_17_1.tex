%%%%%%%%%%%%%%%%%%%%%%%%%%%%%%%%%%%%%%%%%%%%%%%%%%%%%%%%%%%%%%%%%%%%
%        Matematica dolce
%
% Copyright 2016 Daniele Zambelli
%
%------------------------------
% Matematica dolce per i licei Linguistico e economico sociale, volume 1
%
% m_d_licei_17_1.tex
%------------------------------
%
% This work may be distributed and/or modified under the
% conditions of the LaTeX Project Public License, either version 1.3
% of this license or (at your option) any later version.
% The latest version of this license is in
%   http://www.latex-project.org/lppl.txt
% and version 1.3 or later is part of all distributions of LaTeX
% version 2005/12/01 or later.
%
% This work has the LPPL maintenance status `maintained'.
% 
% The Current Maintainer of this work is 
% Daniele Zambelli - daniele.zambelli@gmail.com
%
% This work consists of the files:
%  -  m_d_1_les.tex (this file)
%  -  createpdf.py
%  -  readme.md
%  -  the part of code of the files under \matdir , img/  and lbr/ directories
%%%%%%%%%%%%%%%%%%%%%%%%%%%%%%%%%%%%%%%%%%%%%%%%%%%%%%%%%%%%%%%%%%%%

%========================
% Definizione delle directory
%========================
% \newcommand{\basedir}{matematicadolce/}
\newcommand{\depdir}{deposito/}
\newcommand{\magdir}{\depdir magazzino/}
\newcommand{\matdir}{\depdir materiali/}

%========================
% Variabili del progetto
%========================
\input{\magdir variabili}

%========================
% Variabili per questo volume
%========================
\newcommand{\editore}{Matematicamente.it}
\newcommand{\serie}{Matematica $C^3$}
\newcommand{\titolo}{Matematica dolce 1 - licei}        
\newcommand{\pdftitolo}{Matematica dolce 1 - licei}
\newcommand{\docvers}{\texttt{3.0.1}}
\newcommand{\edizione}{2017}
\newcommand{\Edizione}{Edizione}
\newcommand{\tipo}{ (versione completa a colori)}
\newcommand{\descr}{Testo per il primo biennio \protect\\ 
                    della Scuola Secondaria di $II$ grado \\ \null 
                    licei}
\newcommand{\oggi}{27 giugno 2016}
\newcommand{\mese}{giugno}
\newcommand{\anno}{2016}
\newcommand{\mcisbn}{9788899988005}

%========================
% Lettura preambolo
%========================
\input{\magdir packages}
\input{\magdir definizioni}
\input{\magdir definizioni_tikz}

%========================
% Caratteri sans serif
%========================
% \renewcommand{\familydefault}{\sfdefault}

%========================
% Con o senza linguaggio di programmazione
%========================
\newif\ifcoding
\codingtrue     % commentare questa linea se NON si vuole il coding
% \codingfalse    % commentare questa linea se si vogliono le parti di coding

%========================
% Documento
%========================
\begin{document}
\frontmatter
\intestazione{\matdir 00_intestazioni/les/}{frontespizio}
\intestazione{\matdir 00_intestazioni/les/}{colophon}
\intestazione{\matdir 00_intestazioni/les/}{indice}
\intestazione{\matdir 00_intestazioni/les/}{prefazione}
%..................................................
%
% Primo tema: Aritmetica e algebra
% 
\mainmatter
\parte{\matdir 0_a_parti/}{part_01_1_d}
\capitolo{\matdir 02_Insiemi_numerici/01_Numeri_naturali/nat01/}{naturali}
\capitolo{\matdir 02_Insiemi_numerici/02_Numeri_interi/int01/}{interi}
\capitolo{\matdir 02_Insiemi_numerici/03_Numeri_razionali/raz01/}{razionali}
\capitolo{\matdir 03_Polinomi/01_Monomi_polinomi/calclett01/}
         {calcololetterale}
%..................................................
%
% Secondo tema: Geometria
%
\parte{\matdir 0_a_parti/}{part_02_1_d}
\capitolo{\matdir 05_Geometria_euclidea/01_Assiomi_teoremi/fond01/}
         {fondamenti1}
\capitolo{\matdir 05_Geometria_euclidea/02_Triangoli_rette/tri01/}
         {triangoli1}
\capitolo{\matdir 07_Geometria_analitica/01_Punti_segmenti/pc01/}
         {pianocartesiano}
% ..................................................
% 
% Terzo tema: Relazioni e funzioni
%
\parte{\matdir 0_a_parti/}{part_03_1_d}
\capitolo{\matdir 01_Logica/01_Insiemi_logica/ins01/}{insiemi}
\capitolo{\matdir 04_Equazioni/01_Eq_grado_1/eq01/}{equazioni}
\capitolo{\matdir 01_Logica/02_Relazioni/rel01/}{relazioni}
% ..................................................
% 
% Quarto tema: Dati e previsioni
%
\parte{\matdir 0_a_parti/}{part_04_1_d}
\capitolo{\matdir 14_Statistica/01_Statistica_descrittiva/stat01/}
         {statistica}
%..................................................
%
% Quinto tema: Elementi di informatica
%
\parte{\matdir 0_a_parti/}{part_05_1_d}
\capitolo{\matdir 17_Strumenti_digitali/01_Foglio_calcolo/calc01/}
         {fogliodicalcolo}
\capitolo{\matdir 17_Strumenti_digitali/02_Python/02_Geo_inter/geoint01/}
         {geointerattiva1}
%..................................................
%
% Azzeramento numerazione capitoli
%
\renewcommand{\thechapter}{\Alph{chapter}}
\setcounter{chapter}{0}

\end{document}

