%%%%%%%%%%%%%%%%%%%%%%%%%%%%%%%%%%%%%%%%%%%%%%%%%%%%%%%%%%%%%%%%%%%%
%        Matematica dolce
%
% Copyright 2016 Daniele Zambelli
%
%------------------------------
% Matematica dolce per i licei Linguistico e economico sociale, volume 1
%
% m_d_licei_17_5.tex
%------------------------------
%
% This work may be distributed and/or modified under the
% conditions of the LaTeX Project Public License, either version 1.3
% of this license or (at your option) any later version.
% The latest version of this license is in
%   http://www.latex-project.org/lppl.txt
% and version 1.3 or later is part of all distributions of LaTeX
% version 2005/12/01 or later.
%
% This work has the LPPL maintenance status `maintained'.
% 
% The Current Maintainer of this work is 
% Daniele Zambelli - daniele.zambelli@gmail.com
%
% This work consists of the files:
%  -  m_d_les_5.tex (this file)
%  -  createpdf.py
%  -  readme.md
%  -  the part of code of the files under chap/, img/  and lbr/ directories
%%%%%%%%%%%%%%%%%%%%%%%%%%%%%%%%%%%%%%%%%%%%%%%%%%%%%%%%%%%%%%%%%%%%

%========================
% Definizione delle directory
%========================
% \newcommand{\basedir}{matematicadolce/}
\newcommand{\depdir}{deposito/}
\newcommand{\magdir}{\depdir magazzino/}
\newcommand{\matdir}{\depdir materiali/}

%========================
% Variabili del progetto
%========================
\input{\magdir variabili}

%========================
% Variabili per questo volume
%========================
\newcommand{\editore}{Matematicamente.it}
\newcommand{\serie}{Matematica $C^3$}
\newcommand{\titolo}{Matematica dolce 5}        
\newcommand{\pdftitolo}{Matematica C3 - Matematica dolce 5}
\newcommand{\docvers}{\texttt{0.0.9}}
\newcommand{\edizione}{ 2016}
\newcommand{\Edizione}{2016 Edizione}
\newcommand{\tipo}{ (versione completa a colori)}
\newcommand{\descr}{Testo per il secondo biennio \protect\\ 
                    della Scuola Secondaria di $II$ grado}
\newcommand{\oggi}{27 giugno 2016}
\newcommand{\mese}{giugno}
\newcommand{\anno}{2016}
\newcommand{\mcisbn}{}

%========================
% Lettura preambolo
%========================
\input{\magdir packages}
\input{\magdir definizioni}
\input{\magdir definizioni_tikz}

%========================
% Caratteri sans serif
%========================
% \renewcommand{\familydefault}{\sfdefault}

%========================
% Con o senza linguaggio di programmazione
%========================
\newif\ifcoding
\codingtrue     % commentare questa linea se NON si vuole il coding
% \codingfalse    % commentare questa linea se si vogliono le parti di coding

%========================
% Documento
%========================
\begin{document}
\frontmatter
% \intestazione{\matdir 00_intestazioni/les/}{frontespizio}
% \intestazione{\matdir 00_intestazioni/les/}{colophon}
% \intestazione{\matdir 00_intestazioni/les/}{indice}
% \intestazione{\matdir 00_intestazioni/les/}{prefazione}
%..................................................
%% --------------------------------
%% Capitoli
%% --------------------------------
%....... Proposta di Luciana ......................
% 
% Argomenti proposti da Luciana in sintesi
% Funzioni
% Elementi di Topologia
% Limiti
% Funzioni economiche
% Continuità
% Derivate
% Ottimizzazione e studio di funzione
% Integrali

%..................................................
\mainmatter
% % % \parte{\matdir 01_aritmeticaealgebra/}{part_01_d}
% \capitolo{\matdir 12_Analisi/01_Topologia_retta/funztop01/}
%          {funzionitopologia}
% \capitolo{\matdir 02_Insiemi_numerici/07_Numeri_iperreali/iperr02/}
%           {iperreali}
% \capitolo{\matdir 12_Analisi/04_Derivata/diff01/}
%           {differenziazione}
% \capitolo{\matdir 12_Analisi/03_Funzioni_continue/cont01/}
%          {funzionicontinue}
% \capitolo{\matdir 12_Analisi/05_Studio_funzione/studiof01/}
%          {studiofunzioni}
% \capitolo{\matdir 12_Analisi/06_Integrali/int01/}
%          {integrali_matematicamente_bruno2}
\capitolo{\matdir 12_Analisi/06_Integrali/int01/}
         {integrazione}
% \capitolo{\matdir 16_Economia/02_Modelli_economici/modec01/}
%          {modellieconomici}
%..................................................
%%
%% Azzeramento numerazione capitoli
%%
\renewcommand{\thechapter}{\Alph{chapter}}
\setcounter{chapter}{0}

\end{document}
\grid
