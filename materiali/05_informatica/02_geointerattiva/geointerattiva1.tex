% (c) 2014 Daniele Zambelli - daniele.zambelli@gmail.com

\begin{comment}

% per disegnare il simbolo >>> in 05_02_tartaruga.
%\newcommand{\tggg}[0]{\textgreater\textgreater\textgreater}

\end{comment}

% Forse è meglio mettere questa istruzione nel file delle intestazioni dopo la 
% lettura della libreia??????????????
\lstset{numbers=left, numberstyle=\tiny, frame=trbl, frameround=ftff,
        language=Python}

\chapter{Geometria interattiva 1}

\section{Introduzione}
\label{sec:introduzione}

\emph{Cos'è la geometria interattiva, i primi oggetti.}

La geometria interattiva è un programma che permette di creare gli oggetti 
della geometria euclidea in un computer.
Questa geometria viene detta interattiva perché gli elementi di base possono 
essere mossi con il mouse e quindi le figure create possono venir deformate.

La geometria interattiva permette di visualizzare facilmente elementi
varianti e invarianti di una certa costruzione geometrica.

La geometria interattiva ci mette a disposizione alcuni elementi primitivi tra 
cui: punti, rette, semirette, segmenti, circonferenze, angoli, testi \dots
(i testi non sono oggetti geometrici, ma ci possono essere utili per aggiungere 
delle informazioni al disegno che stiamo realizzando).

% \begin{itemize} [noitemsep]
% \item punti;
% \item rette (semirette, segmenti);
% \item circonferenze;
% \item angoli;
% \item testi.
% \end{itemize}

Esistono molti programmi che permettono di operare con la geometria
interattiva, a questo indirizzo se ne possono trovare molti:

\url{en.wikipedia.org/wiki/List_of_interactive_geometry_software}

Un altro progetto interessante che contiene anche la geometria interattiva è:

\url{www.kogics.net/sf:kojo}

In questo testo propongo l'uso del linguaggio 
Python con la libreria \texttt{pyig}.

È comunque possibile seguire il percorso proposto anche con un programma
\emph{punta e clicca} invece che con un linguaggio.

Prima di poter usare la geometria interattiva dobbiamo installare i  software 
necessario.

\subsection{Installiamo un interprete}
\label{sec:05_02installazione}

\emph{Cosa installare per lavorare con la geometria interattiva.}

\subsubsection{Python}

Chi usa come sistema operativo Windows può installare \texttt{Python} 
a partire dal sito:

\url{www.python.org/downloads}

E installare la versione più recente della serie 3.x.x.

Chi utilizza altri sistemi operativi può installarlo partendo dal proprio
gestore di pacchetti installando \texttt{Python3} e anche \texttt{IDLE}.

\subsubsection{\texttt{pygraph}}

Si può scaricare l'intero pacchetto da:

\url{bitbucket.org/zambu/pygraph/downloads}

A questo punto bisogna fare a mano alcune operazioni che dipendono dal 
proprio sistema operativo:

\subsubsection*{Windows}

\begin{itemize} [noitemsep]
\item {} Scompattare il file scaricato.
\item {} Entrare nella cartella \texttt{pygraph}.
\item {} Selezionare il file \texttt{pygraph.pth} e la cartella 
  \texttt{pygraph} lì presenti.
\item {} Copiarli nella cartella 
\texttt{C:\textbackslash Python3x\textbackslash Lib\textbackslash site-package}
\end{itemize}

Dove ``Python3x'' potrebbe essere: ``Python34'', ``Python35'' ...

\subsubsection*{MacOSX}

\begin{itemize} [noitemsep]
\item {} Scompattare il file scaricato.
\item {} Entrare nella cartella \texttt{pygraph}.
\item {} Selezionare il file \texttt{pygraph.pth} e la cartella 
  \texttt{pygraph} lì presenti.
\item {} Copiarli nella cartella \texttt{HD/libreria/python/3.x/site-package}
\end{itemize}

Se in ``HD/libreria/python/'' non è presente la cartella 
``3.4/site-packages'', bisogna crearla.

\subsubsection*{GNU/Linux}

\begin{itemize} [noitemsep]
\item {} Scompattare il file scaricato.
\item {} Entrare nella directory \texttt{pygraph}.
\item {} Aprire un terminale in questa directory.
\item {} Copiare la cartella \texttt{pygraph} e il file \texttt{pygraph.pth}
  nella cartella
  
  \texttt{/usr/lib/python3/dist-packages/}
  
  Dato che in Linux, per modificare le directory di sistema bisogna essere 
  amministratori, il comando da dare assomiglierà a questo:
  
  \texttt{sudo cp -R python* /usr/lib/python3/dist-packages/}
\end{itemize}

A questo punto se tutto è andato bene dovremmo essere in grado di avviare 
Python-IDLE e dare il comando:

\texttt{import pyig as ig}

Se non succede nulla vuol dire che tutto è andato a buon fine, 
se invece appare una scritta rossa, bisogna leggere almeno l'ultima riga
e cercare di capire cosa non è andato bene. Magari ci si può far aiutare
da qualcuno esperto nell'installazione di programmi.
Se tutto è andato per il verso giusto possiamo procedere.

\subsection{Riassumendo}
\begin{itemize} [nosep]
\item La geometria interattiva permette di creare e di muovere gli oggetti 
della geometria euclidea.
\item Ci sono molti programmi che permettono di giocare con la geometria
interattiva, noi utilizzeremo il linguaggio Python con la libreria 
\texttt{pyig}.
\end{itemize}

\section{Elementi fondamentali}
\label{sec:geo_int_elementi_fondamentali}

Ogni disegno che realizzeremo parte dalla creazione di alcuni punti liberi che 
sono punti che possono essere trascinati con il mouse. La Geometria interattiva 
mette in evidenza quali sono le caratteristiche invarianti e quali quelle 
variabili di una certa costruzione geometrica.

Un metodo per imparare, non solo l'informatica o la geometria, consiste nei 
seguenti tre passi: \textbf{copiare}, \textbf{capire}, \textbf{modificare}.
Lo utilizzeremo per imparare la geometria e l'informatica.
Studieremo la geometria interattiva ponendoci alcuni problemi di costruzione 
grafica e descrivendo la loro soluzione. Per ogni problema viene data una 
possibile soluzione. Riportala sul computer e verifica che funzioni. Non 
preoccuparti se non ti è chiara immediatamente, ogni programma viene spiegato 
in dettaglio. 

\subsection{Un piano vuoto}
\label{subsec:geo_int_pianovuoto}

\subsubsection{problema}

Il primo problema che ci poniamo è quello di disegnare un piano vuoto. Il 
risultato ha poco di geometrico (anzi niente), ma ci permette di chiarire come 
è fatto un programma minimale.

\subsubsection{copiare}

\begin{procedura}
Scrivi il programma.
\begin{enumerate} [noitemsep]
 \item Avvia IDLE (da Windows: menu-programmi-Python-IDLE);
 \item Crea una nuova finestra di editor (menu-File-New File);
 \item Ricopia il seguente programma;
 \item Esegui il programma (menu-Run-Run Module o più semplicemente <F5>)
 \item Correggi gli errori finché il programma non produce una finestra con un 
riferimento cartesiano vuoto.
\end{enumerate}
\end{procedura}

\lstinputlisting{\folder src/00piano_vuoto.py}

\subsubsection{capire}

Spiegazione del programma:

\begin{description}
 \item [linee 1-3] 
 tutto quello che segue il carattere cancelletto ('\#') viene ignorato 
dall'interprete Python perciò queste righe non vengono eseguite. Ma ogni 
programma che scriveremo deve iniziare con alcune informazioni: 
la data di creazione, l'autore o gli autori, un titolo.
% \begin{multicols}{3}
% \begin{itemize} [nosep]
%  \item la data di creazione,  
%  \item l'autore o gli autori, 
%  \item un titolo.
% \end{itemize}
% \end{multicols}
 \item [linea 4]
 Una riga vuota può aiutare a rendere più leggibile il codice (vedi anche le 
linee:~8,~11).
 \item [linee 5-7]
 Questa è una stringa che si estende su più righe; inizia e termina con tre 
caratteri di virgolette doppie. In questa stringa mettiamo una descrizione di 
cosa fa il programma, nel nostro caso, scriviamo qui il problema che vogliamo 
risolvere con questo programma.
 \item [linee 9, 12, 14]
 Sono dei commenti che servono da \emph{titolo} alle righe seguenti.
 \item [linea 10]
 Questa è la prima istruzione che viene effettivamente eseguita 
dall'interprete Python: Legge la libreria \lstinline{pyig} cioè estende ciò 
che sa fare Python con tutto ciò che è contenuto in questa libreria. 
 \item [linea 13]
 Questa è la prima istruzione del programma principale, serve per creare un 
piano interattivo. Il piano creato viene associato all'\emph{identificatore}
\lstinline{ip}. La \emph{classe} \lstinline{InteractivePlane} è definita 
all'interno della libreria \lstinline{pyig} è per questo che per creare un 
piano interattivo devo devo scrivere il nome della libreria seguito da un punto 
seguito dal nome della classe di cui voglio creare un oggetto. Se 
questa linea dà un errore bisogna controllare di aver rispettato le maiuscole e 
minuscole e di aver messo le parentesi tonde. 
 \item [linea 15]
 Questa istruzione rende attiva (e interattiva) la finestra grafica. È 
importante che resti sempre l'ultima istruzione del programma.
\end{description}

\subsubsection{osservazioni}

\begin{itemize}
 \item È normale ottenere degli errori quando si crea un programma. Gli errori 
che si 
ottengono sono di due tipi:
\begin{description} [noitemsep]
 \item [Sintattici]
 Il compilatore non riesce a capire quello che avete scritto. Potrebbero 
mancare dei simboli, una parentesi che è stata aperta non è stata chiusa, ...
Questi errori vengono individuati prima ancora di incominciare ad eseguire il 
programma e vengono segnalati da una finestra \emph{pop-up} aperta 
dall'editor dove è stato scritto il programma.
 \item [Semantici]
 Certe istruzioni, se pur formalmente corrette, non possono essere eseguite.
 Questi errori vengono riportati nella finestra della \emph{Shell} di IDLE, in 
rosso. L'ultima riga dà un'indicazione del motivo dell'errore, quelle subito 
precedenti indicano l'istruzione che ha creato l'errore e la sua posizione nel 
programma.
\end{description}
 \item Per semplificare la correzione degli errori, organizzate il vostro 
desktop in modo da avere sempre in vista contemporaneamente sia il programma 
sia la finestra della \emph{shell} dove appaiono gli errori.
\end{itemize}

\subsubsection{modificare}

\begin{enumerate} [noitemsep]
 \item Commenta, ponendo un cancelletto ('\#') a inizio riga, una alla volta le 
tre istruzioni del programma e osserva cosa succede.
 \item Nella shell di Idle scrivi le seguenti istruzioni:
 \begin{lstlisting}[numbers=none]
>>> import pyig
>>> help(pyig.InteractivePlane.__init__)
 \end{lstlisting}
Quello che ottieni è la descrizione dei parametri che possono essere passati al 
costruttore del piano interattivo.
Prova a modificare la finestra grafica cambiando la riga 13 del programma in 
questo modo:
 \begin{lstlisting}[firstnumber=13]
ip = pyig.InteractivePlane(name='il MIO piano')
 \end{lstlisting}
 \item Personalizza il piano cartesiano modificando altri parametri.
\end{enumerate}

\subsection{Oggetti di base}
\label{subsec:geo_int_oggetti_base}

Disegniamo alcuni elementi di base: un punto, un segmento, una retta, una 
circonferenza.

\subsubsection{copiare}

Con lo stesso procedimento utilizzato sopra, scrivi il seguente programma e 
correggilo finché non funziona.

\lstinputlisting{\folder src/01oggetti_di_base.py}

\subsubsection{capire}

\begin{description}
 \item [Struttura] 
 Il programma ha la stessa struttura del precedente:
\begin{itemize} [nosep]
 \item un'intestazione con alcune informazioni (linee 1-3); 
 \item la descrizione del problema che risolve (linee 5-8);
 \item la lettura delle librerie (linea 11);
 \item il programma principale (linee 14-28);
 \item notare sempre l'ultima istruzione del programma principale che rende 
attiva la finestra grafica.
\end{itemize}
 \item [linea 16]
 Viene creato un oggetto \lstinline{Point} della libreria \lstinline{pyig}. 
In questa istruzione vengono anche precisati: un colore e uno spessore e 
un'etichetta.
Queste due informazioni non sono strettamente necessarie si può creare un punto 
anche fornendo solo le coordinate che indicano dove disegnarlo:
\lstinline{pyig.Point(3,5)}
 \item [linee 18-20]
 Per disegnare un segmento (\lstinline{pyig.Segment}) ho bisogno di indicare i 
suoi due estremi che sono due punti quindi devo prima creare i due punti, 
associarli a due identificatori e poi passare questa informazione al 
costruttore 
del segmento. 
\begin{itemize} [nosep]
 \item la linea 18 crea un punto e lo associa all'identificatore 
 \lstinline{p_1}
 \item la linea 19 crea un punto e lo associa all'identificatore 
 \lstinline{p_2}
 \item la linea 20 crea un segmento che va da 
 \lstinline{p_1} a \lstinline{p_1}
\end{itemize}
 \item [linee 22-23]
 Per disegnare una retta (\lstinline{pyig.Line}) ho bisogno di indicare due suoi
punti quindi posso seguire lo stesso meccanismo usato per il segmento:
\begin{lstlisting}[numbers=none]
p_3 = pyig.Point(-3, -8, width=6, name="D")
p_4 = pyig.Point(-8, -5, width=6, name="E")
pyig.Line(p_3, p_4)
\end{lstlisting}
ma il metodo proposto nell'esempio è equivalente e un po' più rapido.
Da notare che queste due righe descrivono un'unica istruzione e che alla fine 
ci sono due parentesi tonde: la prima per chiudere la costruzione del secondo 
punto, la seconda per chiudere la costruzione della retta.
Se avete dei dubbi provate a cancellare le due parentesi e a riscriverle 
prestando ben attenzione a cosa avviene nell'editor.
 \item [linee 25-26]
 Per disegnare una circonferenza (\lstinline{pyig.Circle}) ci sono diversi modi 
quello usato qui richiede che siano passati al costruttore il centro e un punto 
della circonferenza. Lo si può fare con il metodo usato per il segmento, ma 
dato che richiede una riga di codice in meno, io preferisco il metodo usato per 
la retta.
\end{description}

\subsubsection{osservazioni}

\begin{itemize}
 \item Può sembrare una banalità preferire il metodo utilizzato per disegnare 
la retta al metodo usato per il segmento, si risparmia solo una riga di codice. 
Ma una riga su tre equivale al 33\% che non è poco, e inoltre è 
preferibile non introdurre degli identificatori se non servono. Una regola 
generale recita: ``Quello che non c'è non si può rompere'', meno istruzioni ci 
sono in un programma e meno errori avrò.

 \item Muovi i punti base del disegno e osserva come si comportano gli oggetti 
grafici.
\end{itemize}

\subsubsection{modificare}

\begin{enumerate} [noitemsep]
 \item Cambia a tuo piacere i colori dei vari elementi grafici. Quali colori 
possono assumere?
 \item Aggiungi al programma una circonferenza con centro nell'origine.
\end{enumerate}

\subsection{Intersezioni}
\label{subsec:geo_int_intersezioni}

Disegniamo:
\begin{itemize} [nosep]
 \item la retta passante per \(\punto{-2}{7}\) e \(\punto{5}{7}\); 
 \item la circonferenza c, di centro \(\punto{2}{-5}\) e 
 passante per \(\punto{4}{0}\);
 \item la retta s, passante per \(\punto{-3}{2}\) e \(\punto{4}{2}\);
 \item l'intersezione tra le due rette s e r;
 \item le intersezioni tra la retta s e la circonferenza c. 
\end{itemize}

\subsubsection{copiare}

\lstinputlisting{\folder src/02intersezioni.py}

\subsubsection{capire}

\begin{description}
 \item [Struttura] 
 Il programma ha sempre la stessa struttura;
 \item [linee 20-27]
Vengono create le due rette e la circonferenza, da notare che viene data 
un'etichetta a questi oggetti, non ai punti.
 \item [linea 29]
Viene creata l'intersezione tra le due rette assegnando a questo punto il 
colore verde.
 \item [linee 31-32]
Vengono create le due intersezione della retta con la circonferenza. 
\lstinline{pyig} permette di creare un'intersezione alla volta quindi per 
distinguere le due intersezioni è obbligatorio aggiungere un argomento che può 
valere solo~\(-1\) o~\(+1\), se ci dimentichiamo di precisare quale 
intersezione vogliamo, Python ci segnala un errore
 \item [linee ]
\end{description}

\subsubsection{osservazioni}

\begin{itemize}
 \item Provate a muovere la retta s, quando questa incontra l'altra retta o la 
circonferenza, appaiono i punti di intersezione.
 \item Provate a muovere gli altri due oggetti\dots non c'è modo di farlo 
perché i punti su cui si basano sono stati creati con il parametro 
\lstinline{visible} 
posto a \lstinline{False}.
\end{itemize}

\subsubsection{modificare}

\begin{enumerate} [noitemsep]
 \item Rendi modificabili anche la retta r e la circonferenza.
 \item Rendi un po' più visibili i punti di intersezione.
\end{enumerate}

\subsection{Altri oggetti primitivi}
\label{subsec:geo_int_altrioggetti}

Disegniamo, nei diversi quadranti: una semiretta, un angolo, un angolo con un 
solo lato visibile, un angolo con entrambi i lati visibili.

\subsubsection{copiare}

\lstinputlisting{\folder src/03altri_oggetti.py}

\subsubsection{capire}

\begin{description}
 \item [linee 16-17]
La costruzione di una semiretta avviene in modo molto simile a quella di una 
retta.
 \item [linee 19-21]
Un modo per costruire un angolo è quello di dare un punto dl primo lato, il 
vertice e un punto del secondo lato. Se non viene specificato altro i lati non 
vengono disegnati ma appare solo l'archetto che indica l'angolo.
 \item [linee 23-25]
Per far disegnare il primo lato faccio seguire ai tre punti, necessari per la 
costruzione, una lista che contiene il numero zero: [0].
 \item [linee 27-29]
Per far disegnare entrambi i lati faccio seguire ai tre punti, una lista che 
contiene i numeri zero e uno: [0, 1].
\end{description}

\subsubsection{osservazioni}

\begin{itemize}
 \item L'ultimo dei tre angoli può un po' stupire. Osservate bene qual è 
il primo lato 
e quale il secondo e ricordatevi che l'angolo procede sempre in verso 
antiorario dal primo al secondo. Provate a muovere il punto~C più a sinistra 
del punto~D.
\end{itemize}

\subsubsection{modificare}

\begin{enumerate} [noitemsep]
 \item Cambia il colore e lo spessore della semiretta.
 \item Cambia il colore e lo spessore dell'angolo.
 \item Cambia la costruzione del secondo angolo in modo che venga visualizzato 
il secondo lato.
\end{enumerate}

\subsection{Poligoni}
\label{subsec:geo_int_poligoni}

Disegniamo un triangolo nel primo quadrante, un quadrilatero nel secondo, un 
pentagono nel terzo e un esagono nel quarto.

\subsubsection{copiare}

\lstinputlisting{\folder src/04poligoni.py}

\subsubsection{capire}

In questo programma vengono usati metodi diversi per fare la stessa cosa; dato 
che tutti funzionano, non è essenziale capirli tutti.

\begin{description}
 \item [linee 16-21]
Vengono disegnati, prima tre punti, poi i tre segmenti che li congiungono 
formando un triangolo.
 \item [linee 23-26]
Vengono disegnati 4 punti.
 \item [linee 27-28]
Viene disegnato il poligono che ha questi 4 punti come vertici.
Il primo argomento passato a \lstinline{pyig.Polygon} è una lista che contiene 
tutti i vertici.
Gli altri argomenti sono: lo spessore del bordo, il suo colore e il colore 
dell'interno.
 \item [linee 30-32]
Viene creata una lista di cinque punti e viene associata all'identificatore 
\lstinline{vertici_p}.
 \item [linea 33]
Viene creato un poligono a cui viene passato come primo argomento la lista di 
punti associata a \lstinline{vertici_p}.
 \item [linea 35]
Viene creata una lista di coordinate (coppie di numeri).
 \item [linea 36]
Con un metodo che si chiama ``list comprehension'' viene creata una lista di 
punti partendo dalla lista di coordinate.
 \item [linea 37]
Viene creato il poligono passando come primo argomento: la lista di vertici.
\end{description}

\subsubsection{osservazioni}

\begin{itemize}
 \item Il metodo usato per disegnare il triangolo permette di scegliere colori 
diversi per i diversi lati, ma non di colorare la superficie del poligono.
 \item Il quadrilatero potrebbe sembrare un quadrato, e in effetti inizialmente 
lo è, ma non è una proprietà della figura infatti con il mouse è facile 
deformarla.
 \item 
Il primo argomento di un poligono deve essere una sequenza di punti, se 
l'istruzione delle righe 27-27 l'avessi scritta così:
\begin{lstlisting}[firstnumber=27]
pyig.Polygon(p_0, p_1, p_2, p_3,
             width=6, color="dark green", intcolor="light steel blue")
\end{lstlisting}
cioè senza le parentesi che raggruppano i punti, non avremmo ottenuto un 
quadrilatero, ma un errore.
 \item La ``list comprehension'' è un potente strumento, presente in alcuni 
linguaggi di programmazione, che permette di costruire delle liste di oggetti a 
partire da altre liste di oggetti. Nel nostro esempio trasforma una lista di 
coppie di numeri in una lista di punti.
\end{itemize}

\subsubsection{modificare}

\begin{enumerate} [noitemsep]
 \item Muovi, con il mouse, i poligoni in modo che siano contenuti uno 
nell'altro.
 \item Cambia il colore dei vertici del pentagono, cambia il colore dei vertici 
dell'esagono. Quali differenze puoi notare nei due casi?
 \item Disegna, sempre nel primo quadrante un altro triangolo ma utilizzando, 
questa volta, l'oggetto \lstinline{pyig.Polygon}.
 \item Disegna un poligono con la superficie colorata e con i lati di colori 
diversi.
\end{enumerate}

\subsection{Riassumendo}
\begin{itemize} [noitemsep]
\item Per usare la geometria interattiva di Python, per prima cosa dobbiamo 
caricare la libreria \lstinline{pyig};
\item Il programma principale inizia creando un piano interattivo, gli altri 
oggetti geometrici che verranno creati vengono inseriti in quel piano.
\item Gli oggetti di base sono:
\begin{multicols}{3}
\begin{itemize} [nosep]
\item punti:
\begin{itemize} [nosep]
\item liberi,
\item vincolati,
\item intersezioni;
\end{itemize}

\item rette:
\begin{itemize} [nosep]
\item rette,
\item semirette,
\item segmenti;
\end{itemize}
\item circonferenze;
\item angoli;
\end{itemize}
\end{multicols}
\end{itemize}

\section{Altri problemi}

\begin{enumerate} [noitemsep]
\item Installa Python.

\item Installa la libreria pygraph.

\item Crea un nuovo programma che disegni un segmento di colore viola,
con due estremi rosa, grandi a piacere.

\item Crea un programma che disegni un rettangolo. Muovendo i punti base
continua a rimanere un rettangolo?

\item Crea un programma che disegni un triangolo. Muovendo i punti base
continua a rimanere un triangolo?

\item Crea un programma che disegni un quadrilatero delimitato da semirette.

\item Crea un programma che disegni tre punti \texttt{A}, \texttt{B} e 
\texttt{C}, disegna poi le tre circonferenze:
\begin{itemize} [nosep]
\item di centro \texttt{A} e passante per \texttt{B};
\item di centro \texttt{B} e passante per \texttt{C};
\item di centro \texttt{C} e passante per \texttt{A};
\end{itemize}

\item Disegna due circonferenze concentriche. Muovendo i punti base, si 
mantiene la proprietà ``essere concentriche''?

\item Disegna una circonferenza \texttt{c\_0} con il centro nell'origine,
una retta \texttt{r\_0} e un'altra circonferenza \texttt{c\_1}.
Disegna in modo evidente le intersezioni tra la retta \texttt{r\_0} e la
circonferenza \texttt{c\_0} e tra la circonferenza \texttt{c\_1} e la
circonferenza \texttt{c\_0}.

\item Disegna una circonferenza e una retta. Poi disegna un'intersezione tra
la retta e la circonferenza e assegna a questa intersezione il nome:
``Ciao''. Poi disegna una circonferenza che ha centro nell'intersezione
e passa per il punto (3; 1).

\item Disegna le intersezioni tra due circonferenze che hanno centro in un 
estremo di un segmento e passano per l'altro estremo del segmento.

\item
Crea un piano e disegna:
\begin{itemize} [nosep]
\item nel primo quadrante: 
 due punti e il triangolo equilatero costruito su quei due punti;
\item nel secondo quadrante: 
 un segmento e l'asse di quel segmento;
\item nel terzo quadrante: 
 un angolo e la bisettrice di quell'angolo;
\item nel quarto quadrante: 
 due punti e il quadrato costruito su quei due punti.
\end{itemize}

\item Disegna un quadrato dati due vertici opposti.

\item Disegna un esagono regolare dati due vertici consecutivi.

\item Disegna un esagono regolare dato il centro e un vertice.

\item Disegna un pentagono regolare dati due vertici consecutivi.

\item Disegna un parallelogramma dati tre vertici consecutivi.
\end{enumerate}
