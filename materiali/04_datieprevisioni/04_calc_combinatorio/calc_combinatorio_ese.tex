% (c) 2015 Daniele Zambelli daniele.zambelli@gmail.com
% (c) 2016 Andrea Sellaroli andrea.sellaroli@istruzione.it

\section{Esercizi}

\subsection{Esercizi dei singoli paragrafi}

\subsubsection*{\numnameref{sec:01_introduzione}}

\begin{esercizio}
\label{ese:I.1}
Per andare da Verona a Vicenza ci sono 3 possibili percorsi. Per andare da Vicenza a Padova ci sono 5 possibili strade. Nessuna strada collega direttamente Verona con Padova. Quanti percorsi possibili ci sono per andare da Verona a Padova?
\hfill $\left[15\right]$
\end{esercizio}

\begin{esercizio}
\label{ese:I.2}
Il menu di un ristorante permette di scegliere tra 5 primi, 7 secondi e, a scelta, tra frutta o dolce. In quanti modi si può ordinare un pranzo completo?
\hfill $\left[70\right]$
\end{esercizio}

\begin{esercizio}\label{ese:I.3}
Quanti sono i numeri dispari formati da 4 cifre tutte diverse tra loro (un numero di quattro cifre non può iniziare per 0, altrimenti sarebbe di 3 cifre).
\hfill $\left[4500\right]$
\end{esercizio}


\subsubsection*{\numnameref{sec:02_permutazioni}}
\begin{esercizio}
\label{ese:P.1}
Cinque persone sono in fila per entrare in un negozio. In quanti modi potrebbero entrare se non rispettassero l'ordine di arrivo
\hfill $\left[120\right]$
\end{esercizio}

\begin{esercizio}\label{ese:P.2}
Disegna un albero per rappresentare le permutazioni nell'insieme:
 \begin{enumeratea}
  \item $A=\{ a,b\}$
  \item $B=\{ x,y,z,k\}$
  \item $C=\{\text{rosso, verde, giallo, marrone, blu}\}$
 \end{enumeratea}
\end{esercizio}

\begin{esercizio}\label{P.3}
Scrivi tutti gli anagrammi, anche privi di significato, delle parole:
\begin{multicols}{3}
 \begin{enumeratea}
  \item PIU
  \item MENO
  \item VENTI
 \end{enumeratea}
 \end{multicols}
\end{esercizio}

\subsubsection*{\numnameref{sec:03_disposizioni}}

\begin{esercizio}\label{D.1}
Calcola il numero di disposizioni di 8 elementi in 3 posti.
\hfill $\left[336\right]$
\end{esercizio}

\begin{esercizio}\label{D.2}
Calcola il numero di disposizioni di 25 elementi in 5 posti.
\hfill $\left[6375600\right]$
\end{esercizio}

\begin{esercizio}\label{D.3}
Quante melodie si possono comporre formate da 3 note diverse (le note sono 12).
\hfill $\left[1320\right]$
\end{esercizio}

\begin{esercizio}\label{D.4}
Una password è formata da 3 cifre dispari tutte diverse tra loro. Se non ricordo la password, quante prove dovrò fare?
\hfill $\left[60\right]$
\end{esercizio}

\begin{esercizio}\label{D.5}
Quanti sono i numeri dispari formati da 4 cifre tutte diverse tra loro
\hfill $\left[4500\right]$
\end{esercizio}

\subsubsection*{\numnameref{sec:04_combinazioni}}

\begin{esercizio}
\label{ese:C.1}
In un compito in classe si devono scegliere 3 esercizi tra 5 proposti. In quanti modi diversi posso sceglierli?
\end{esercizio}

\begin{esercizio}
\label{ese:C.2}
Calcola il numero dei possibili terni al lotto.
\end{esercizio}

\begin{esercizio}
\label{ese:C.3}
Determina il numero di sottoinsiemi di 5 elementi da un insieme di 8 elementi
\end{esercizio}

\begin{esercizio}
\label{ese:C.4}
In una classe di 22 persone devono essere scelte a caso 3 persone per essere interrogate contemporanemente. In quanti modi è possibile effettuare la scelta?
\end{esercizio}


\begin{comment}
\subsection{Esercizi riepilogativi}

\begin{esercizio}
\label{ese:R.1}
Quante sono le diagonali di un ottagono?
\end{esercizio}

\begin{esercizio}
\label{ese:R.2}
In un torneo di calcio a otto squadre, ogni squadra deve giocare una partita contro ognuna delle altre. Quante partite si giocheranno?
\end{esercizio}
\end{comment}
