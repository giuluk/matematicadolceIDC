%%%%%%%%%%%%%
% Claudoio Carboncini
%%%

\theoremstyle{plain}
% le seguenti due righe ora danno un errore penso vadano semplicemente tolte:
% \newshadetheorem{postulato}{\thmcolor{Postulato}}[chapter]
% \newshadetheorem{proposizione}{\thmcolor{Proposizione}}[chapter]

%%%%%%%%%%%%%
% Daniele Zambelli
%%%

%
% per mantenere allineati i riferimenti presenti
% negli esercizi e nelle soluzioni.
\newcommand{\numnameref}[1]{\ref{#1} \nameref{#1}}
%
% per disegnare il simbolo >>>
% in 05_02_tartaruga.
\newcommand{\tggg}[0]{\textgreater\textgreater\textgreater}

% Inizializza la ``variabile globale'' folder
\newcommand{\folder}{./}

% Crea una nuova parte
\newcommand{\parte}[2]{
  \renewcommand{\folder}{#1}
  \graphicspath{\folder}
  \include{\folder #2}
}

% Crea un nuova parte di intestazione
\newcommand{\intestazione}[2]{
  \renewcommand{\folder}{#1}
  \graphicspath{{\folder}}
  \include{\folder #2}
  \cleardoublepage
}

% Crea un nuovo capitolo
\newcommand{\capitolo}[2]{
  \renewcommand{\folder}{#1}
  \graphicspath{{\folder}}
  \include{\folder #2}
  \newpage
  \include{\folder #2_ese}
  \cleardoublepage
}

% Per contrassegnare e sostituire i blocchi inacessibili ai ciechi.
\newenvironment{inaccessibleblock}[1][]{}{}

% % \usepackage{enumerate}
% \usepackage{fancyhdr} % per intestazioni e pie di pagina
% \setlength{\columnseprule}{1pt}
% \def\columnseprulecolor{\color{black}}
% % \usepackage{widetable} % Per tabelle con larghezza fissa

%--------------------------
% Nuovi comandi di testo:

%--------------------------
% Insiemi numerici:
\newcommand{\N}{\mathbb{N}}
\newcommand{\Z}{\mathbb{Z}}
\newcommand{\Q}{\mathbb{Q}}
\newcommand{\A}{\mathbb{A}}
\newcommand{\R}{\mathbb{R}}
\newcommand{\IR}{{}^*\hspace{-.12em}\mathbb{R}} % Iperreali
\newcommand{\C}{\mathbb{C}}
\newcommand{\K}{\mathbb{K}}
\newcommand{\effestar}{{}^*\hspace{-.15em}f}

%--------------------------
% varianti di lettere greche:
\renewcommand{\epsilon}{\varepsilon}
\renewcommand{\theta}{\vartheta}
\renewcommand{\rho}{\varrho}
\renewcommand{\phi}{\varphi}

%--------------------------
% Delimitatori e parentesi:
\newcommand{\tonda}[1]{\left( #1 \right)}
\newcommand{\quadra}[1]{\left[ #1 \right]}
\newcommand{\graffa}[1]{\left \{ #1 \right \}}
\newcommand{\abs}[1]{\left \lvert #1 \right \rvert}
\newcommand{\modulo}[1]{\left| #1 \right|}
\newcommand{\angolare}[1]{\left \langle #1 \right \rangle}

%--------------------------
% Sisteni, vettori, matrici:
\newcommand{\sistema}[1]{\left\{\begin{array}{lcl}#1\end{array}\right.}
\newcommand{\fatratti}[1]{\left\{\begin{array}{lclcl}#1\end{array}\right.}
\newcommand{\vettore}[1]{\left(\begin{array}{c}#1\end{array}\right)}
\newcommand{\matrice}[2]{\tonda{\begin{array}{#1}#2\end{array}}}
\newcommand{\vect}[1]{\overrightarrow{#1}}
\newcommand{\punto}[2]{\tonda{#1;~#2}}

%--------------------------
% Intervalli:
\newcommand{\intervcc}[2]{\left[#1;~#2\right]}
\newcommand{\intervac}[2]{\left]#1;~#2\right]}
\newcommand{\intervca}[2]{\left[#1;~#2\right[}
\newcommand{\intervaa}[2]{\left]#1;~#2\right[}

%--------------------------
% simboli con l'aggiunta di uno spazio prima e dopo:
\newcommand{\sand}{~ \wedge ~}
\newcommand{\sor}{~ \vee ~}
\newcommand{\sLarrow}{~ \Leftarrow ~}
\newcommand{\sLRarrow}{~ \Leftrightarrow ~}
\newcommand{\sRarrow}{~ \Rightarrow ~}
\newcommand{\slarrow}{~ \leftarrow ~}
\newcommand{\slrarrow}{~ \leftrightarrow ~}
\newcommand{\srarrow}{~ \rightarrow ~}

%--------------------------
% Operatori per le funzioni circolari e per gli Iperreali:
\DeclareMathOperator{\sen}{sen}
\DeclareMathOperator{\tg}{tg}
\DeclareMathOperator{\st}{st}              % standard
\newcommand{\pst}[1]{\st \tonda{#1}}       % parte standard di 
\DeclareMathOperator{\monade}{mon}         % monade
\newcommand{\mon}[1]{\monade \tonda{#1}}   % monade di 
\DeclareMathOperator{\Galassia}{Gal}       % Galassia
\newcommand{\Gal}[1]{\Galassia \tonda{#1}} % Galassia di 


%--------------------------
% Elenco numerato. Esempio di chiamata: 
% \elenconumerato{{$3+2$, $4 \cdot 5$, {con, VIRGOLE, $5^3$}}{\vspace{1cm}}}
\newcommand{\elenconumerato}[2]{%
\begin{enumerate}
 \foreach \x in #1 {\item \x #2}
\end{enumerate}
}

%--------------------------
% Centra un elemento utile nelle tabelle centra anche verticalmente: 
\newcommand{\centra}[1]{\begin{center}#1\end{center}}

%--------------------------
% Per il testo di geometria (tratto da Geoetria Razionale C3): 
\theoremstyle{plain} 
% \newshadetheorem{postulato}{\thmcolor{Postulato}}[chapter]
% \newshadetheorem{corollario}{\thmcolor{Corollario}}[chapter]
% \newshadetheorem{proposizione}{\thmcolor{Proposizione}}[chapter]
\newshadetheorem{corollario}[teorema]{Corollario}
% \newshadetheorem{proposizione}[teorema]{Proposizione}

%\newshadetheorem{legge}[teorema]{Legge}
%\newshadetheorem{principio}[teorema]{Principio}
%\newshadetheorem{procedura}[teorema]{Procedura}
%\newshadetheorem{proprieta}[teorema]{Propriet\`a}

\newcommand{\defintextsep}{\intextsep}


% \newtheorem{theorem}{Theorem}[section]
% \newtheorem{proposizione}[theorem]{Proposizione}
% 
% \newtheorem{teorema}[theorem]{Teorema}
% \newtheorem{osservazione}[theorem]{Osservazione} % non usato qui
% \newtheorem{lemma}[theorem]{Lemma} % usato da Centomo
% \newtheorem{corollario}[theorem]{Corollario} % non usato qui
% 
% %\theoremstyle{definizione}
% \newtheorem{definizione}[theorem]{Definizione}
% % 
% % %\theoremstyle{esempio}
% \newtheorem{esempio}[theorem]{Esempio} % non usato qui
% % \newtheorem{notazione}[theorem]{Notazione} % non usato qui
% % \newtheorem{esercizio}[theorem]{Esercizio} % non usato qui
% 
% %--------------------------------------------
% % Centomo
% \newenvironment{proof}
%                {\noindent\textbf{Dimostrazione\ }}
%                {\hspace*{\fill}$\Box$\medskip}
% \newtheorem{definition}{Definition} % usato da Centomo
% % \newtheorem{lemma}{Lemma}
% % \newtheorem{theorem}{Theorem}


% \newcommand{\piedipagina}[3]{% scrive il piedipagina
%   \def \sinistra{#1}
%   \def \centro{#2}
%   \def \destra{#3}
%   \pagestyle{fancy}
%   \renewcommand{\headrulewidth}{0pt}
%   \renewcommand{\footrulewidth}{0pt}
%   \lfoot{\scriptsize \sinistra}
%   \cfoot{\scriptsize \centro}
%   \rfoot{\scriptsize \destra}
% }
