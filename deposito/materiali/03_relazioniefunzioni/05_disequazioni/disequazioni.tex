% (c) 2012 -2014 Dimitrios Vrettos - d.vrettos@gmail.com
% (c) 2014 Daniele Zambelli - daniele.zambelli@gmail.com

\chapter{Disequazioni}

\section{Disuguaglianze chiuse e aperte}
\label{sec:dis_disguagianze}

Consideriamo le seguenti proposizioni:

\begin{enumeratea}
\item 5 è minore di~12;
\item $48-90$ è maggiore di~30;
\item il quadrato di un numero reale è maggiore o uguale a zero;
\item sommando ad un numero la sua metà si ottiene un numero minore
o uguale a~1.
\end{enumeratea}

Esse possono essere tradotte in linguaggio matematico usando i simboli
$>$ (maggiore), $<$~(minore), ${\geq}$ (maggiore o uguale), ${\leq}$ 
(minore o uguale) e precisamente:

\begin{multicols}{4}
 \begin{enumeratea}
\item $5<12$
\item $48-90>30$
\item $x^{2}\ge~0$
\item $x+\frac{1}{2}x\le~1$.
 \end{enumeratea}
\end{multicols}

Le formule che contengono variabili si dicono aperte; quelle che
contengono solo numeri si dicono chiuse. Quindi a) e b) sono formule
chiuse; c) e d) sono formule aperte.

\begin{definizione}
Chiamiamo \emph{disuguaglianza} una formula chiusa
costruita con uno dei predicati:~$<$ (essere minore);
$>$ (essere maggiore); ${\leq}$ (essere minore o uguale);
${\geq}$ (essere maggiore o uguale).
\end{definizione}

Di essa sappiamo subito stabilire il valore di verità, quando è
stabilito l'ambiente in cui vengono enunciate.

\begin{definizione}
Chiamiamo \emph{disequazione} una formula aperta,
definita in~$\insR$ e costruita con uno dei seguenti predicati:~$<$
(essere minore); $>$ (essere maggiore); ${\leq}$
(essere minore o uguale); ${\geq}$ (essere
maggiore o uguale).
\end{definizione}

Analogamente a quanto detto per le equazioni, chiamiamo
\emph{incognite} le variabili che compaiono nella disequazione,
\emph{primo membro} e \emph{secondo membro} le due espressioni che
compaiono a sinistra e a destra del segno di disuguaglianza.

% \begin{exrig}
 \begin{esempio}
 Disuguaglianze vere e false.

 \begin{enumeratea}
\item in~$\insN$, la formula~$5>0$ è una disuguaglianza: vera;
\item in~$\insZ$, la formula~$-6>-4$ è una disuguaglianza: falsa;
\item la formula~$5x>0$ è una disequazione; quando
all'incognita sostituiamo un numero essa si trasforma
in una disuguaglianza e solo allora possiamo stabilirne il valore di
verità. Nel caso proposto è vera se sostituiamo
alla variabile un qualunque numero positivo, falsa se
sostituiamo zero o un numero negativo.
\end{enumeratea}
 \end{esempio}

% \end{exrig}

% \ovalbox{\risolvi \ref{ese:21.8}}

\begin{definizione}
Chiamiamo \emph{soluzione} di una disequazione l'insieme dei valori che 
sostituiti all'incognita rendono vera la disuguaglianza.
\end{definizione}

Mentre le soluzioni di un'equazione determinata sono dei valori isolati, 
dei numeri, le soluzioni delle disequazioni sono degli intervalli di
numeri.
Ad esempio una disequazione può essere verificata per tutti i numeri positivi, 
oppure per tutti i numeri compresi tra~$-5$ e~$+4,72$.

I numeri li conosciamo bene, sappiamo come rappresentarli, gli intervalli
un po' meno.
Prima di affrontare il nuovo argomento, vediamo dunque come rappresentare gli 
intervalli.

\section{Intervalli sulla retta reale}
\label{sec:dis_intervalli}

I numeri reali possono essere messi in corrispondenza biunivoca con i
punti di una retta: ogni numero reale ha per immagine un punto della
retta e viceversa ogni punto della retta è immagine di un numero
reale. Un intervallo di numeri può essere messo in corrispondenza con una 
semiretta o un segmento.
Un segmento della retta è l'insieme di tutti i punti della retta compresi 
tra due punti detti estremi. Un intervallo numerico è l'insieme di tutti i 
numeri compresi tra due numeri detti estremi dell'intervallo. Ad esempio 
possiamo considerare tutti i numeri compresi tra~$-7$ e~$-2$. 

\osservazione Quando rappresentiamo un intervallo poniamo attenzione di 
scrivere prima il numero minore poi il maggiore.

``Tutti i numeri compresi tra~$-7$ e~$-2$'' è una frase ambigua. 
È chiaro che:~$-5;~-4;~-3; \dots$, ma anche:~$-5,2;~-4,37;~-2,001; \dots$  
appartengono all'intervallo, ma cosa dire di~$-7$ e di~$-2$? A seconda dei 
gusti possiamo sostenere che gli estremi appartengono oppure no all'intervallo, 
non c'è una ragione logica per preferire una o l'altra interpretazione. 
Quindi i matematici parlano di due tipi di intervalli: 

\begin{itemize} [noitemsep]
 \item Intervalli \emph{chiusi}: quelli che comprendono anche gli estremi;
 \item Intervalli \emph{aperti}: quelli che non comprendono gli estremi.
\end{itemize}

Possiamo distinguere gli intervalli anche in base ad un'altra caratteristica:

\begin{itemize} [noitemsep]
 \item Intervalli \emph{limitati}: formati dai numeri compresi tra due numeri;
 \item Intervalli \emph{illimitati}: formati dai numeri minori (o maggiori) di
  un dato numero.
\end{itemize}

Possiamo dare la seguente definizione:

\begin{definizione}
 Si chiama \emph{intervallo} di un insieme ordinato, 
 un sottoinsieme che contiene tutti gli elementi compresi tra due 
 valori detti \emph{estremi}. 
 Questi valori possono appartenere oppure non appartenere all'intervallo. 
\end{definizione}

La situazione non è semplice, perché in un intervallo potrebbe essere compreso 
un estremo e non l'altro quindi possiamo avere intervalli aperti/chiusi
a destra o a sinistra. Non solo, ma un intervallo potrebbe avere un inizio
e poi continuare all'infinito. Vediamo i vari casi possibili.

Per quanto riguarda gli intervalli di numeri reali, quelli che si possono
mettere in corrispondenza biunivoca con i punti della retta, possiamo avere 
i casi presentati nella tabella:~\ref{tab:intervalli}

\begin{table}[h!]
\caption{Intervalli}
\center
\label{tab:intervalli}
 \begin{tabular}{p{4cm}|c|c|c}
  a parole   & con i predicati & con le parentesi & sulla retta \\
  \hline
  i numeri compresi tra~$a$ e~$b$ estremi esclusi & 
  $a < x < b$ & $(a;~b)$ o $]a;~b[$ & 
  % (c) 2014 Daniele Zambelli - daniele.zambelli@gmail.com

%%%
% Intervallo ]a; b[
%%%%
 
\begin{tikzpicture}[x=1.5mm, y=1.5mm, smooth]

\coordinate (a) at (-3, 0);
\coordinate (b) at (6, 0);

% (c) 2014 Daniele Zambelli - daniele.zambelli@gmail.com

%%%
% Asse cartesiano R
%%%%

% (c) 2014 Daniele Zambelli - daniele.zambelli@gmail.com

%%%
% Asse cartesiano generico
%%%%

\draw [-{Stealth[length=2mm, open, round]}] (-10, 0) -- (10, 0);

\node [below] at (10, 0) {$\mathbb{R}$};


\begin{scope}[blue,thick]
\draw [-,decorate,decoration=snake] (a) -- (b);
\draw[fill=white] (a) circle (2pt) node [above] {a};
\draw[fill=white] (b) circle (2pt) node [above] {b};
\end{scope}

\end{tikzpicture}
 \\
  \hline
  i numeri compresi tra~$a$ e~$b$ estremi inclusi & 
  $a \le x \le b$ & $[a;~b]$ & 
  % (c) 2014 Daniele Zambelli - daniele.zambelli@gmail.com

%%%
% Intervallo [a; b]
%%%%
 
\begin{tikzpicture}[x=1.5mm, y=1.5mm, smooth]

\coordinate (a) at (-3, 0);
\coordinate (b) at (6, 0);

% (c) 2014 Daniele Zambelli - daniele.zambelli@gmail.com

%%%
% Asse cartesiano R
%%%%

% (c) 2014 Daniele Zambelli - daniele.zambelli@gmail.com

%%%
% Asse cartesiano generico
%%%%

\draw [-{Stealth[length=2mm, open, round]}] (-10, 0) -- (10, 0);

\node [below] at (10, 0) {$\mathbb{R}$};


\begin{scope}[blue,thick]
\draw [-,decorate,decoration=snake] (a) -- (b);
\draw[fill] (a) circle (2pt) node [above] {a};
\draw[fill] (b) circle (2pt) node [above] {b};
\end{scope}

\end{tikzpicture}
 \\
  \hline
  i numeri compresi tra~$a$ e~$b$, a~incluso, b~escluso & 
  $a \le x < b$ & $[a;~b)$ o $[a;~b[$ & 
  % (c) 2014 Daniele Zambelli - daniele.zambelli@gmail.com

%%%
% Intervallo [a; b[
%%%%
 
\begin{tikzpicture}[x=1.5mm, y=1.5mm, smooth]

\coordinate (a) at (-3, 0);
\coordinate (b) at (6, 0);

% (c) 2014 Daniele Zambelli - daniele.zambelli@gmail.com

%%%
% Asse cartesiano R
%%%%

% (c) 2014 Daniele Zambelli - daniele.zambelli@gmail.com

%%%
% Asse cartesiano generico
%%%%

\draw [-{Stealth[length=2mm, open, round]}] (-10, 0) -- (10, 0);

\node [below] at (10, 0) {$\mathbb{R}$};


\begin{scope}[blue,thick]
\draw [-,decorate,decoration=snake] (a) -- (b);
\draw[fill] (a) circle (2pt) node [above] {a};
\draw[fill=white] (b) circle (2pt) node [above] {b};
\end{scope}

\end{tikzpicture}
 \\
  \hline
  i numeri compresi tra~$a$ e~$b$, a~escluso, b~incluso & 
  $a < x \le b$ & $(a;~b]$ o $]a;~b]$ & 
  % (c) 2014 Daniele Zambelli - daniele.zambelli@gmail.com

%%%
% Intervallo ]a; b]
%%%%
 
\begin{tikzpicture}[x=1.5mm, y=1.5mm, smooth]

\coordinate (a) at (-3, 0);
\coordinate (b) at (6, 0);

% (c) 2014 Daniele Zambelli - daniele.zambelli@gmail.com

%%%
% Asse cartesiano R
%%%%

% (c) 2014 Daniele Zambelli - daniele.zambelli@gmail.com

%%%
% Asse cartesiano generico
%%%%

\draw [-{Stealth[length=2mm, open, round]}] (-10, 0) -- (10, 0);

\node [below] at (10, 0) {$\mathbb{R}$};


\begin{scope}[blue,thick]
\draw [-,decorate,decoration=snake] (a) -- (b);
\draw[fill=white] (a) circle (2pt) node [above] {a};
\draw[fill] (b) circle (2pt) node [above] {b};
\end{scope}

\end{tikzpicture}
 \\
  \hline
  i numeri fino ad~$a$, $a$~escluso & 
  $x < a$ & $(-\infty;~a)$ o $]-\infty;~a[$ & 
  % (c) 2014 Daniele Zambelli - daniele.zambelli@gmail.com

%%%
% Intervallo ]oo; a[ 
%%%%
 
\begin{tikzpicture}[x=1.5mm, y=1.5mm, smooth]

\coordinate (a) at (-3, 0);
\coordinate (b) at (-10, 0);

% (c) 2014 Daniele Zambelli - daniele.zambelli@gmail.com

%%%
% Asse cartesiano R
%%%%

% (c) 2014 Daniele Zambelli - daniele.zambelli@gmail.com

%%%
% Asse cartesiano generico
%%%%

\draw [-{Stealth[length=2mm, open, round]}] (-10, 0) -- (10, 0);

\node [below] at (10, 0) {$\mathbb{R}$};


\begin{scope}[blue,thick]
\draw [-,decorate,decoration=snake] (a) -- (b);
\draw[fill=white] (a) circle (2pt) node [above] {a};
\end{scope}

\end{tikzpicture}
 \\
  \hline
  i numeri fino ad~$a$, $a$~incluso & 
  $x \le a$ & $(-\infty;~a]$ o $]-\infty;~a]$ & 
  % (c) 2014 Daniele Zambelli - daniele.zambelli@gmail.com

%%%
% Intervallo ]oo; a]
%%%%
 
\begin{tikzpicture}[x=1.5mm, y=1.5mm, smooth]

\coordinate (a) at (-3, 0);
\coordinate (b) at (-10, 0);

% (c) 2014 Daniele Zambelli - daniele.zambelli@gmail.com

%%%
% Asse cartesiano R
%%%%

% (c) 2014 Daniele Zambelli - daniele.zambelli@gmail.com

%%%
% Asse cartesiano generico
%%%%

\draw [-{Stealth[length=2mm, open, round]}] (-10, 0) -- (10, 0);

\node [below] at (10, 0) {$\mathbb{R}$};


\begin{scope}[blue,thick]
\draw [-,decorate,decoration=snake] (a) -- (b);
\draw[fill] (a) circle (2pt) node [above] {a};
\end{scope}

\end{tikzpicture}
 \\
  \hline
  i numeri da~$a$ in poi, $a$~escluso & 
  $x > a$ o $a < x$ & $(a;~-\infty)$ o $]a;~-\infty[$ & 
  % (c) 2014 Daniele Zambelli - daniele.zambelli@gmail.com

%%%
% Intervallo ]a; oo[
%%%%
 
\begin{tikzpicture}[x=1.5mm, y=1.5mm, smooth]

\coordinate (a) at (-3, 0);
\coordinate (b) at (10, 0);

% (c) 2014 Daniele Zambelli - daniele.zambelli@gmail.com

%%%
% Asse cartesiano R
%%%%

% (c) 2014 Daniele Zambelli - daniele.zambelli@gmail.com

%%%
% Asse cartesiano generico
%%%%

\draw [-{Stealth[length=2mm, open, round]}] (-10, 0) -- (10, 0);

\node [below] at (10, 0) {$\mathbb{R}$};


\begin{scope}[blue,thick]
\draw [-,decorate,decoration=snake] (a) -- (b);
\draw[fill=white] (a) circle (2pt) node [above] {a};
\end{scope}

\end{tikzpicture}
 \\
  \hline
  i numeri da~$a$ in poi, $a$~incluso & 
  $x \ge a$ o $a \le x$ & $[a;~-\infty)$ o $[a;~-\infty[$ & 
  % (c) 2014 Daniele Zambelli - daniele.zambelli@gmail.com

%%%
% Intervallo [a; oo[ 
%%%%
 
\begin{tikzpicture}[x=1.5mm, y=1.5mm, smooth]

\coordinate (a) at (-3, 0);
\coordinate (b) at (10, 0);

% (c) 2014 Daniele Zambelli - daniele.zambelli@gmail.com

%%%
% Asse cartesiano R
%%%%

% (c) 2014 Daniele Zambelli - daniele.zambelli@gmail.com

%%%
% Asse cartesiano generico
%%%%

\draw [-{Stealth[length=2mm, open, round]}] (-10, 0) -- (10, 0);

\node [below] at (10, 0) {$\mathbb{R}$};


\begin{scope}[blue,thick]
\draw [-,decorate,decoration=snake] (a) -- (b);
\draw[fill] (a) circle (2pt) node [above] {a};
\end{scope}

\end{tikzpicture}
 
 \end{tabular}
\end{table}

% \begin{exrig}
\begin{esempio}
$H=\{x\in \insR/x<3\}$ intervallo illimitato 
inferiormente~$H = ]-\infty;~3) = (-\infty;~3)$.

L'insieme~$H$ è rappresentato da tutti i punti della
semiretta che precedono il punto immagine del numero~3, esclusa
l'origine della semiretta. Nella figura questi punti sono evidenziati e 
per mettere in evidenza che l'origine della semiretta non
appartiene all'insieme abbiamo messo un pallino vuoto sul punto.
\begin{center}
 % (c) 2014 Daniele Zambelli - daniele.zambelli@gmail.com

%%%
% Intervallo illimitato inferiormente
%%%%
 
\begin{tikzpicture}[x=5mm, y=5mm, smooth]

\coordinate (a) at (3, 0);
\coordinate (b) at (-10, 0);

% (c) 2014 Daniele Zambelli - daniele.zambelli@gmail.com

%%%
% Asse cartesiano x
%%%%

% (c) 2014 Daniele Zambelli - daniele.zambelli@gmail.com

%%%
% Asse cartesiano generico
%%%%

\draw [-{Stealth[length=2mm, open, round]}] (-10, 0) -- (10, 0);

\node [below] at (10, 0)  {$x$};


\begin{scope}[blue,thick]
\draw [-,decorate,decoration=snake] (a) -- (b);
\draw[fill=white] (a) circle (2pt) node [below] {3};
% \draw[fill] (b) circle (2pt) node [above] {b};
\end{scope}

\end{tikzpicture}

\end{center}
\end{esempio}

\begin{esempio}
$\insP=\{x\in \insR/x\ge -5\}$ intervallo illimitato superiormente chiuso 
a sinistra~$\insP = [-5;~+\infty[ = [-5;~+\infty)$.

Segniamo sulla retta~$r$ il punto immagine di~$-5$
l'insieme~$\insP$ è rappresentato dalla semiretta di tutti
i punti che seguono~$-5$, compreso lo stesso~$-5$. Nel disegno, la
semiretta dei punti che appartengono a~$\insP$ è stata disegnata con una
linea più spessa, per indicare che il punto~$-5$ appartiene
all'intervallo abbiamo messo un pallino pieno sul punto.
\begin{center}
 % (c) 2014 Daniele Zambelli - daniele.zambelli@gmail.com

%%%
% Intervallo [-5; oo[
%%%%
 
\begin{tikzpicture}[x=5mm, y=5mm, smooth]

% \clip (-7.5, -5.5) rectangle (10.9, 10.9);

\coordinate (a) at (-5, 0);
\coordinate (b) at (10, 0);

% (c) 2014 Daniele Zambelli - daniele.zambelli@gmail.com

%%%
% Asse cartesiano x
%%%%

% (c) 2014 Daniele Zambelli - daniele.zambelli@gmail.com

%%%
% Asse cartesiano generico
%%%%

\draw [-{Stealth[length=2mm, open, round]}] (-10, 0) -- (10, 0);

\node [below] at (10, 0)  {$x$};


\begin{scope}[blue,thick]
\draw [-,decorate,decoration=snake] (a) -- (b);
\draw[fill] (a) circle (2pt) node [below] {$-5$};
\end{scope}

\end{tikzpicture}

\end{center}
\end{esempio}

\begin{esempio}
 $D=\{x\in \insR/-2<x<6\}$ intervallo limitato 
 aperto~$D = ]-2;~6[ = (-2;~6)$.

Segniamo sulla retta reale i punti immagine degli estremi del segmento,
$-2$ e~6. L'insieme~$D$ è rappresentato dal segmento che
ha per estremi questi due punti. Nel disegno il segmento è stato
disegnato con una linea più spessa, i due estremi del segmento sono
esclusi, pertanto su ciascuno di essi abbiamo messo un pallino vuoto.
\begin{center}
 % (c) 2014 Daniele Zambelli - daniele.zambelli@gmail.com

%%%
% Intervallo ]2; 6[
%%%%
 
\begin{tikzpicture}[x=5mm, y=5mm, smooth]

\coordinate (a) at (-2, 0);
\coordinate (b) at (6, 0);

% (c) 2014 Daniele Zambelli - daniele.zambelli@gmail.com

%%%
% Asse cartesiano x
%%%%

% (c) 2014 Daniele Zambelli - daniele.zambelli@gmail.com

%%%
% Asse cartesiano generico
%%%%

\draw [-{Stealth[length=2mm, open, round]}] (-10, 0) -- (10, 0);

\node [below] at (10, 0)  {$x$};


\begin{scope}[blue,thick]
\draw [-,decorate,decoration=snake] (a) -- (b);
\draw[fill=white] (a) circle (2pt) node [below] {$-2$};
\draw[fill=white] (b) circle (2pt) node [below] {6};
\end{scope}

\end{tikzpicture}

\end{center}
\end{esempio}

\begin{esempio}
$T=\{x\in \insR/-2<x\le~6\}$ intervallo limitato chiuso a 
destra~$T = ]-2;~6] = (-2;~6]$.

Rispetto al caso precedente, il segmento che rappresenta
l'insieme~$T$ è chiuso a destra, ossia è incluso
nell'intervallo anche il~6, è escluso invece il
punto~$-2$.
\begin{center}
 % (c) 2014 Daniele Zambelli - daniele.zambelli@gmail.com

%%%
% Intervallo ]2: 6]
%%%%
 
\begin{tikzpicture}[x=5mm, y=5mm, smooth]

\coordinate (a) at (-2, 0);
\coordinate (b) at (6, 0);

% (c) 2014 Daniele Zambelli - daniele.zambelli@gmail.com

%%%
% Asse cartesiano x
%%%%

% (c) 2014 Daniele Zambelli - daniele.zambelli@gmail.com

%%%
% Asse cartesiano generico
%%%%

\draw [-{Stealth[length=2mm, open, round]}] (-10, 0) -- (10, 0);

\node [below] at (10, 0)  {$x$};


\begin{scope}[blue,thick]
\draw [-,decorate,decoration=snake] (a) -- (b);
\draw[fill=white] (a) circle (2pt) node [below] {$-2$};
\draw[fill] (b) circle (2pt) node [below] {6};
\end{scope}

\end{tikzpicture}

\end{center}
\end{esempio}

\begin{esempio}
$S=\{x\in \insR/-2\le x\le~6\}$ intervallo chiuso e 
limitato~$S = [2;~6]$.

Il segmento che rappresenta l'insieme~$S$ contiene tutti e
due i suoi estremi:
\begin{center}
 % (c) 2014 Daniele Zambelli - daniele.zambelli@gmail.com

%%%
% Intervallo [2; 6]
%%%%
 
\begin{tikzpicture}[x=5mm, y=5mm, smooth]

\coordinate (a) at (-2, 0);
\coordinate (b) at (6, 0);

% (c) 2014 Daniele Zambelli - daniele.zambelli@gmail.com

%%%
% Asse cartesiano x
%%%%

% (c) 2014 Daniele Zambelli - daniele.zambelli@gmail.com

%%%
% Asse cartesiano generico
%%%%

\draw [-{Stealth[length=2mm, open, round]}] (-10, 0) -- (10, 0);

\node [below] at (10, 0)  {$x$};


\begin{scope}[blue,thick]
\draw [-,decorate,decoration=snake] (a) -- (b);
\draw[fill] (a) circle (2pt) node [below] {$-2$};
\draw[fill] (b) circle (2pt) node [below] {6};
\end{scope}

\end{tikzpicture}

\end{center}
\end{esempio}

\begin{esempio}
Altri particolari sottoinsiemi dei numeri reali sono:

\begin{itemize} [noitemsep]
\item $\insR^{+}=\{x\in \insR/x>0\} = ]0;~\infty[$. 
Semiretta di origine~0 costituita da tutti i numeri positivi:
\begin{center}
 % (c) 2014 Daniele Zambelli - daniele.zambelli@gmail.com

%%%
% Intervallo ]a; oo[
%%%%
 
\begin{tikzpicture}[x=5mm, y=5mm, smooth]

\coordinate (a) at (0, 0);
\coordinate (b) at (10, 0);

% (c) 2014 Daniele Zambelli - daniele.zambelli@gmail.com

%%%
% Asse cartesiano x
%%%%

% (c) 2014 Daniele Zambelli - daniele.zambelli@gmail.com

%%%
% Asse cartesiano generico
%%%%

\draw [-{Stealth[length=2mm, open, round]}] (-10, 0) -- (10, 0);

\node [below] at (10, 0)  {$x$};


\begin{scope}[blue,thick]
\draw [-,decorate,decoration=snake] (a) -- (b);
\draw[fill=white] (a) circle (2pt) node [below] {0};
\end{scope}

\end{tikzpicture}

\end{center}
\item $\insR^{-}=\{x\in \insR/x<0\} = ]- \infty;~0[$. 
Semiretta di origine~0 costituita da tutti i numeri reali negativi:
\begin{center}
 % (c) 2014 Daniele Zambelli - daniele.zambelli@gmail.com

%%%
% Intervallo ]oo; 0[
%%%%
 
\begin{tikzpicture}[x=5mm, y=5mm, smooth]

\coordinate (a) at (0, 0);
\coordinate (b) at (-10, 0);

% (c) 2014 Daniele Zambelli - daniele.zambelli@gmail.com

%%%
% Asse cartesiano x
%%%%

% (c) 2014 Daniele Zambelli - daniele.zambelli@gmail.com

%%%
% Asse cartesiano generico
%%%%

\draw [-{Stealth[length=2mm, open, round]}] (-10, 0) -- (10, 0);

\node [below] at (10, 0)  {$x$};


\begin{scope}[blue,thick]
\draw [-,decorate,decoration=snake] (a) -- (b);
\draw[fill=white] (a) circle (2pt) node [below] {0};
\end{scope}

\end{tikzpicture}

\end{center}
\subitem Il punto~0 non appartiene a nessuna delle due semirette; il numero 
 zero non appartiene né 
 a~$\insR^{+}$ né a~$\insR^{-}$:~$\insR=\insR^{+}\cup\insR^{-}\cup\{0\}$.
\item $\insR_{0}^{+}=\{x\in \insR/x\ge~0\} = [0;~\infty[$
\item $\insR_{0}^{-}=\{x\in \insR/x\le~0\} = ]- \infty;~0]$.
\end{itemize}
\end{esempio}
% \end{exrig}

% \ovalbox{\risolvii \ref{ese:21.1}, \ref{ese:21.2}, \ref{ese:21.3}, 
% \ref{ese:21.4}, \ref{ese:21.5}, \ref{ese:21.6}, \ref{ese:21.7}}

\section{Segno di un binomio di primo grado}
\label{sec:dis_binomio}

Prima di affrontare lo studio delle disequazioni è importante capire come 
studiare il segno di un'espressione contenente una variabile. 
In questo modo, le disequazioni si ridurranno ad una applicazione dello 
studio del segno.
Studiare il segno di un'espressione che contiene la variabile~$x$, vuol dire 
stabilire per quali valori della variabile l'espressione è positiva e per 
quali valori è negativa.
Come esempio possiamo studiare i valori che assumono i due 
binomi~$f(x) = -4 x +4$ e~$g(x) = 3 x +6$ al variare di~$x$:

\begin{table}[h!]
\label{tab:valoripolinomio}
\caption{Valori di un polinomio}
\center
\label{tab:valori}
 \begin{tabular}{c|c|c}
  $x$ & $f(x) = -4 x +4$ & $g(x) = 3 x +6$ \\
  \hline
  -4 &  20 & -6 \\
  -3 &  16 & -3 \\
  -2 &  12 &  0 \\
  -1 &   8 &  3 \\
   0 &   4 &  6 \\
   1 &   0 &  9 \\
   2 &  -4 & 12 \\
   3 &  -8 & 15 \\
   4 & -12 & 18 \\
   5 & -16 & 21
 \end{tabular}
\end{table}

Si può osservare che il primo binomio è sempre positivo finché~$x$ è più 
piccolo di~1, quando~$x$ vale proprio 1 il binomio vale~0, quando~$x$ è 
maggiore di~1 il binomio assume un valore negativo.
Il secondo binomio ha un comportamento diverso. Finché~$x$ si mantiene 
minore di~$-2$ è negativo, quando~$x$ vale~$-2$ il binomio vale~0, quando~$x$ 
supera il valore~$-2$ il binomio diventa positivo.

In realtà noi abbiamo verificato solo un piccolissimo insieme di valori, 
ma l'andamento regolare dei risultati dovrebbe convincerci che i segni 
rimangono immutati anche per valori molto diversi da quelli testati.

Il grafico della funzione~$y=f(x)$, dove~$f(x)$ è il polinomio, è una retta. 
In corrispondenza dei valori positivi del polinomio, la retta si trova al di 
sopra dell'asse~$x$ (tratto blu), quando invece il polinomio assume valori 
negativi, la retta si trova sotto all'asse~$x$ (tratto rosso).

Così i due polinomi possono essere associati alle 
funzioni:~$f(x) = -4 x +4$ e~$g(x) = \frac{3}{2} x +3$ che hanno le seguenti 
rappresentazioni nel piano cartesiano:

\begin{inaccessibleblock}[Figura: TODO]
 \begin{figure}[h]
 \centering
 \begin{minipage}[]{.45\textwidth}
  \centering% (c) 2014 Daniele Zambelli - daniele.zambelli@gmail.com

%%%
% Retta per due punti
%%%%
 
\begin{tikzpicture}[x=5mm, y=5mm, smooth]

\clip (-5.5, -5.5) rectangle (5.5, 5.9);

% (c) 2014 Daniele Zambelli - daniele.zambelli@gmail.com

%%%
% Piano cartesiano: da -5 a +5 scala 0,5
%%%%
% \usepgflibrary{arrows.meta}

% Griglia
\draw[gray!50, very thin, step=1] (-5.2, -5.2) grid (5.2, 5.2);

%Asse x
\draw [-{Stealth[length=2mm, open, round]}] (-5.3,0) -- (5.5,0) node [below]  {$x$};

%Asse y
\draw [-{Stealth[length=2mm, open, round]}] (0, -5.3) -- (0, 5.5) node [left]  {$y$};

\node [below] at (-.3, 0) {O};


\coordinate (inizio) at (-1, 8);
\coordinate (zero) at (1, 0);
\coordinate (fine) at (3, -8);

\draw [-] [ultra thick,blue!50!black] (inizio) -- (zero);
\draw [-] [ultra thick,red!50!black] (zero) -- (fine);

\end{tikzpicture}

  \caption{Retta~$f(x) = -4 x +4$}
 \end{minipage}
 \begin{minipage}[]{.45\textwidth}
  \centering% (c) 2014 Daniele Zambelli - daniele.zambelli@gmail.com

%%%
% Retta per due punti
%%%%
 
\begin{tikzpicture}[x=5mm, y=5mm, smooth]

\clip (-5.5, -5.5) rectangle (5.5, 5.9);

% (c) 2014 Daniele Zambelli - daniele.zambelli@gmail.com

%%%
% Piano cartesiano: da -5 a +5 scala 0,5
%%%%
% \usepgflibrary{arrows.meta}

% Griglia
\draw[gray!50, very thin, step=1] (-5.2, -5.2) grid (5.2, 5.2);

%Asse x
\draw [-{Stealth[length=2mm, open, round]}] (-5.3,0) -- (5.5,0) node [below]  {$x$};

%Asse y
\draw [-{Stealth[length=2mm, open, round]}] (0, -5.3) -- (0, 5.5) node [left]  {$y$};

\node [below] at (-.3, 0) {O};


\coordinate (inizio) at (-6, -6);
\coordinate (zero) at (-2, 0);
\coordinate (fine) at (2, 6);

\draw [-] [ultra thick,red!50!black] (inizio) -- (zero);
\draw [-] [ultra thick,blue!50!black] (zero) -- (fine);

\end{tikzpicture}

  \caption{Retta~$g(x) = \frac{3}{2} x +3$}
 \end{minipage}
\end{figure}
\end{inaccessibleblock}

Disegnare una retta nel piano cartesiano è un'abilità molto utile da 
possedere, ma per il nostro problema si può tracciare il grafico in modo 
molto approssimato: sono due gli aspetti che dobbiamo riportare nel grafico:

\begin{itemize} [noitemsep]
 \item 
  lo zero del polinomio, cioè il punto in cui la retta interseca l'asse~$x$
 \item
  la pendenza della retta: cioè se la retta è crescente o decrescente.
\end{itemize}

Si capisce facilmente se la retta è crescente o decrescente guardando la sua 
equazione infatti le rette \emph{crescenti} hanno il coefficiente della~$x$ 
\emph{positivo}, mentre le rette \emph{decrescenti} hanno il coefficiente 
della~$x$ \emph{negativo}.
Quindi i grafici possono essere tracciati semplicemente in questo modo:

\begin{inaccessibleblock}[Figura: TODO]
 \begin{figure}[h]
 \centering
 \begin{minipage}[]{.45\textwidth}
  \centering% (c) 2014 Daniele Zambelli - daniele.zambelli@gmail.com

%%%
% Retta decrescente
%%%%
 
\begin{tikzpicture}[x=1.5mm, y=1.5mm, smooth]

% \clip (-7.5, -5.5) rectangle (10.9, 10.9);

\coordinate (inizio) at (-10, 4);
\coordinate (zero) at (0, 0);
\coordinate (fine) at (10, -4);

% (c) 2014 Daniele Zambelli - daniele.zambelli@gmail.com

%%%
% Asse cartesiano x
%%%%

% (c) 2014 Daniele Zambelli - daniele.zambelli@gmail.com

%%%
% Asse cartesiano generico
%%%%

\draw [-{Stealth[length=2mm, open, round]}] (-10, 0) -- (10, 0);

\node [below] at (10, 0)  {$x$};


\draw [-] [ultra thick,blue!50!black] (inizio) -- (zero);
\draw [-] [ultra thick,red!50!black] (zero) -- (fine);

\begin{scope}[blue,thick]
\draw[fill=white] (zero) circle (2pt) node [above] {};
\end{scope}

\end{tikzpicture}

  \caption{Retta~$f(x) = -4 x +4$}
 \end{minipage}
 \begin{minipage}[]{.45\textwidth}
  \centering% (c) 2014 Daniele Zambelli - daniele.zambelli@gmail.com

%%%
% Retta crescente
%%%%
 
\begin{tikzpicture}[x=1.5mm, y=1.5mm, smooth]

% \clip (-7.5, -5.5) rectangle (10.9, 10.9);

\coordinate (inizio) at (-10, -4);
\coordinate (zero) at (0, 0);
\coordinate (fine) at (10, 4);

% (c) 2014 Daniele Zambelli - daniele.zambelli@gmail.com

%%%
% Asse cartesiano x
%%%%

% (c) 2014 Daniele Zambelli - daniele.zambelli@gmail.com

%%%
% Asse cartesiano generico
%%%%

\draw [-{Stealth[length=2mm, open, round]}] (-10, 0) -- (10, 0);

\node [below] at (10, 0)  {$x$};


\draw [-] [ultra thick,blue!50!black] (inizio) -- (zero);
\draw [-] [ultra thick,red!50!black] (zero) -- (fine);

\begin{scope}[blue,thick]
\draw[fill=white] (zero) circle (2pt) node [above] {};
\end{scope}

\end{tikzpicture}

  \caption{Retta~$g(x) = \frac{3}{2} x +3$}
 \end{minipage}
\end{figure}
\end{inaccessibleblock}

Su questi ultimi grafici si possono aggiungere le informazioni che 
interessano lo studio del segno:

\begin{itemize} [noitemsep]
 \item il valore della~$x$ che rende uguale a 0 il polinomio;
 \item l'intervallo dell'asse~$x$ per il quale il polinomio è positivo;
 \item l'intervallo dell'asse~$x$ per il quale il polinomio è negativo.
\end{itemize}

Riassumendo, per studiare il segno di un binomio di primo grado dobbiamo: 

\begin{procedura}
 Studio del segno di un binomio di primo grado:
\begin{enumeratea}
\item calcolare lo zero del polinomio risolvendo un'equazione associata al 
 polinomio;
\item disegnare il grafico della funzione associata al polinomio, 
 tenendo conto se la retta è crescente o decrescente.
\item riportare su questo grafico lo zero del polinomio e segnare con un 
 "+" i tratti positivi (quelli sopra l\'asse delle~$x$) e con un "-" i tratti 
 negativi (quelli nei quali la retta è tracciata sotto l'asse delle~$x$).
\end{enumeratea}
\end{procedura}

Sempre riferendoci agli esempi precedenti:

\begin{inaccessibleblock}[Figura: TODO]
 \begin{figure}[h]
 \centering
 \begin{minipage}[]{.45\textwidth}
  \centering% (c) 2014 Daniele Zambelli - daniele.zambelli@gmail.com

%%%
% Segno di un polinomio: decrescente, zero=1.
%%%%
 
\begin{tikzpicture}[x=1.5mm, y=1.5mm, smooth]

% (c) 2014 Daniele Zambelli - daniele.zambelli@gmail.com

%%%
% Retta decrescente con segni
%%%%
 
\coordinate (inizio) at (-10, 4);
\coordinate (zero) at (0, 0);
\coordinate (fine) at (10, -4);

% (c) 2014 Daniele Zambelli - daniele.zambelli@gmail.com

%%%
% Asse cartesiano x
%%%%

\input{lbr/assiepiani/asse10.pgf}
\node [below] at (10, 0)  {$x$};


\draw [-] [ultra thick, blue!50!black] (inizio) -- (zero);
\draw [-] [ultra thick, red!50!black] (zero) -- (fine);

\node [xshift=-25, yshift=-3, above] at (zero) {$+$};
\draw[blue, thick, fill=white] (zero) circle (2pt);
\node [xshift=25, yshift=-3, above] at (zero) {$-$};

\node [above] {$1$};

\end{tikzpicture}

  \caption{Segno di~$f(x) = -4 x +4$}
 \end{minipage}
 \begin{minipage}[]{.45\textwidth}
  \centering% (c) 2014 Daniele Zambelli - daniele.zambelli@gmail.com

%%%
% Segno di un polinomio: crescente, zero=-2.
%%%%
 
\begin{tikzpicture}[x=1.5mm, y=1.5mm, smooth]

% (c) 2014 Daniele Zambelli - daniele.zambelli@gmail.com

%%%
% Retta crescente con segni
%%%%
 
\coordinate (inizio) at (-10, -4);
\coordinate (zero) at (0, 0);
\coordinate (fine) at (10, 4);

% (c) 2014 Daniele Zambelli - daniele.zambelli@gmail.com

%%%
% Asse cartesiano x
%%%%

\input{lbr/assiepiani/asse10.pgf}
\node [below] at (10, 0)  {$x$};


\draw [-] [ultra thick, red!50!black] (inizio) -- (zero);
\draw [-] [ultra thick, blue!50!black] (zero) -- (fine);

\node [xshift=-25, yshift=-3, above] at (zero) {$-$};
\draw[blue, thick, fill=white] (zero) circle (2pt);
\node [xshift=25, yshift=-3, above] at (zero) {$+$};

\node [above] {$-2$};

\end{tikzpicture}

  \caption{Segno di~$g(x) = \frac{3}{2} x +3$}
 \end{minipage}
\end{figure}
\end{inaccessibleblock}

Riassumendo, lo studio del segno del binomio di primo grado:~$-4 x +4$,  
si riduce a svolgere questi due passi:

\begin{enumerate}
 \item
  Equazione Associata:~$-4 x +4 = 0 \quad \Rightarrow \quad x = 1$
 \item 
  \begin{minipage}{.45\textwidth}
  Funzione Associata: $y = -4 x +4 \quad \rightarrow$
  \end{minipage}
  \begin{minipage}{.30\textwidth}
  % (c) 2014 Daniele Zambelli - daniele.zambelli@gmail.com

%%%
% Segno di un polinomio: decrescente, zero=1.
%%%%
 
\begin{tikzpicture}[x=1.5mm, y=1.5mm, smooth]

% (c) 2014 Daniele Zambelli - daniele.zambelli@gmail.com

%%%
% Retta decrescente con segni
%%%%
 
\coordinate (inizio) at (-10, 4);
\coordinate (zero) at (0, 0);
\coordinate (fine) at (10, -4);

% (c) 2014 Daniele Zambelli - daniele.zambelli@gmail.com

%%%
% Asse cartesiano x
%%%%

\input{lbr/assiepiani/asse10.pgf}
\node [below] at (10, 0)  {$x$};


\draw [-] [ultra thick, blue!50!black] (inizio) -- (zero);
\draw [-] [ultra thick, red!50!black] (zero) -- (fine);

\node [xshift=-25, yshift=-3, above] at (zero) {$+$};
\draw[blue, thick, fill=white] (zero) circle (2pt);
\node [xshift=25, yshift=-3, above] at (zero) {$-$};

\node [above] {$1$};

\end{tikzpicture}

  \end{minipage}
\end{enumerate}

\section{Segno di un prodotto}
\label{sec:dis_prodotto}

Imparato come studiare il segno di un binomio di primo grado possiamo 
incominciare a complicare le cose...
Se dobbiamo studiare il segno di un trinomio di secondo grado, possiamo 
seguire un procedimento formato da questi 3 passi:

\begin{procedura}
 Studio del segno del prodotto di polinomi di primo grado:
\begin{enumeratea}
\item scomporre in fattori il polinomio;
\item studiare il segno di ogni singolo fattore;
\item applicare la regola dei segni.
\end{enumeratea}
\end{procedura}

Per quanto riguarda i primi due punti seguiamo le indicazioni precedenti, 
il terzo lo si risolve con un grafo in cui riportiamo tre assi, due per i 
segni dei fattori e uno per il segno del prodotto.

Costruiamo una tabella con tanti assi~$x$ quanti sono i fattori, tante 
linee verticali quanti sono i diversi zeri dei polinomi calcolati. 
Intestiamo ogni riga verticale con il valore di uno zero del polinomio
stando ben attenti 
di riportarli in ordine crescente e intestiamo ogni spazio orizzontale con 
l'indicazione del fattore di cui vogliamo riportare il segno. 
Tracciamo un tondino in corrispondenza degli zeri dei polinomi e riportiamo 
i segni già studiati precedentemente. 
Sopra al terzo asse~$x$ riportiamo il segno del prodotto ottenuto 
seguendo la solita regola: un prodotto è positivo se i fattori negativi 
sono in numero pari~(~0,~2,~\dots), è negativo se i fattori negativi sono in 
numero dispari, è nullo se almeno un fattore è nullo.

% \begin{exrig}
 \begin{esempio}
Applichiamo questo procedimento allo studio del segno del 
prodotto: 

$(x-4)(x+2)$

% \begin{itemize} [noitemsep]
%  \item Studio del segno del primo fattore~$F_1$:
%  \subitem E.A.:~$x - 4 = 0 \quad \Rightarrow \quad x = 4$
%  \subitem
%   \begin{minipage}{.25\textwidth}
%   F.A.:~$y = x - 4 \quad \rightarrow$
%   \end{minipage}
%   \begin{minipage}{.30\textwidth}
%   % (c) 2014 Daniele Zambelli - daniele.zambelli@gmail.com

%%%
% Segno di un polinomio: crescente, zero=4.
%%%%
 
\begin{tikzpicture}[x=1.5mm, y=1.5mm, smooth]

% (c) 2014 Daniele Zambelli - daniele.zambelli@gmail.com

%%%
% Retta crescente con segni
%%%%
 
\coordinate (inizio) at (-10, -4);
\coordinate (zero) at (0, 0);
\coordinate (fine) at (10, 4);

% (c) 2014 Daniele Zambelli - daniele.zambelli@gmail.com

%%%
% Asse cartesiano x
%%%%

\input{lbr/assiepiani/asse10.pgf}
\node [below] at (10, 0)  {$x$};


\draw [-] [ultra thick, red!50!black] (inizio) -- (zero);
\draw [-] [ultra thick, blue!50!black] (zero) -- (fine);

\node [xshift=-25, yshift=-3, above] at (zero) {$-$};
\draw[blue, thick, fill=white] (zero) circle (2pt);
\node [xshift=25, yshift=-3, above] at (zero) {$+$};

\node [above] {$4$};

\end{tikzpicture}

%   \end{minipage}
%  \item Studio del segno del secondo fattore~$F_2$:
%  \subitem E.A.:~$x + 2 = 0 \quad \Rightarrow \quad x=-2$
%  \subitem
%   \begin{minipage}{.25\textwidth}
%   F.A.:~$y = x + 2 \quad \rightarrow$
%   \end{minipage}
%   \begin{minipage}{.30\textwidth}
%   % (c) 2014 Daniele Zambelli - daniele.zambelli@gmail.com

%%%
% Segno di un polinomio: crescente, zero=-2.
%%%%
 
\begin{tikzpicture}[x=1.5mm, y=1.5mm, smooth]

% (c) 2014 Daniele Zambelli - daniele.zambelli@gmail.com

%%%
% Retta crescente con segni
%%%%
 
\coordinate (inizio) at (-10, -4);
\coordinate (zero) at (0, 0);
\coordinate (fine) at (10, 4);

% (c) 2014 Daniele Zambelli - daniele.zambelli@gmail.com

%%%
% Asse cartesiano x
%%%%

\input{lbr/assiepiani/asse10.pgf}
\node [below] at (10, 0)  {$x$};


\draw [-] [ultra thick, red!50!black] (inizio) -- (zero);
\draw [-] [ultra thick, blue!50!black] (zero) -- (fine);

\node [xshift=-25, yshift=-3, above] at (zero) {$-$};
\draw[blue, thick, fill=white] (zero) circle (2pt);
\node [xshift=25, yshift=-3, above] at (zero) {$+$};

\node [above] {$-2$};

\end{tikzpicture}

%   \end{minipage}
%  \item Grafo dei segni:
%   % (c) 2014 Daniele Zambelli - daniele.zambelli@gmail.com

%%%
% Studio dei segni di un prodotto
%%%%
 
\begin{tikzpicture}[x=2.5mm, y=5mm, smooth]

\coordinate (a_top) at (-3.3, 1);
\coordinate (a_bottom) at (-3.3, -2);
\coordinate (b_top) at (3.3, 1);
\coordinate (b_bottom) at (3.3, -2);

% (c) 2014 Daniele Zambelli - daniele.zambelli@gmail.com

%%%
% Grafo per il calcolo del segno con tre assi
%%%%
 
% (c) 2014 Daniele Zambelli - daniele.zambelli@gmail.com

%%%
% Asse cartesiano x
%%%%

\input{lbr/assiepiani/asse10.pgf}
\node [below] at (10, 0)  {$x$};

\begin{scope}[yshift= -.5cm]
  % (c) 2014 Daniele Zambelli - daniele.zambelli@gmail.com

%%%
% Asse cartesiano x
%%%%

\input{lbr/assiepiani/asse10.pgf}
\node [below] at (10, 0)  {$x$};

  \begin{scope}[yshift= -.5cm]
    % (c) 2014 Daniele Zambelli - daniele.zambelli@gmail.com

%%%
% Asse cartesiano x
%%%%

\input{lbr/assiepiani/asse10.pgf}
\node [below] at (10, 0)  {$x$};

  \end{scope}
\end{scope}

\draw [-] [] (a_top) -- (a_bottom);
\draw [-] [] (b_top) -- (b_bottom);

\node [above] at (-3.3, 1) {$-2$};
\node [above] at (3.3, 1) {$+4$};

\node [above left] at (-10, 0) {$x - 4$};
\node [above] at (-6.5, 0) {$-$};
\node [above] at (0, 0) {$-$};
\draw (3.3, .5) circle (3pt);
\node [above] at (6.5, 0) {$+$};

\node [above left] at (-10, -1) {$x + 2$};
\node [above] at (-6.5, -1) {$-$};
\draw (-3.3, -.5) circle (3pt);
\node [above] at (0, -1) {$+$};
\node [above] at (6.5, -1) {$+$};

\node [above left] at (-10, -2.15) {$(x - 4)(x + 2)$};
\node [above] at (-6.5, -2) {$+$};
\draw (-3.3, -1.5) circle (3pt);
\node [above] at (0, -2) {$-$};
\draw (3.3, -1.5) circle (3pt);
\node [above] at (6.5, -2) {$+$};

\end{tikzpicture}

% \end{itemize}

\begin{comment}
 
 \begin{minipage}{.45\textwidth}
  
 \end{minipage}
 \begin{minipage}{.25\textwidth}
  
 \end{minipage}
 \begin{minipage}{.3\textwidth}
  
 \end{minipage}
 
\end{comment}

\begin{itemize} [noitemsep]
 \item Studio del segno del primo fattore~$F_1$:\\
 \begin{minipage}{.45\textwidth}
  E.A.:~$x - 4 = 0 \quad \Rightarrow \quad x = 4$
 \end{minipage}
 \begin{minipage}{.25\textwidth}
  F.A.:~$y = x - 4 \quad \rightarrow$
 \end{minipage}
 \begin{minipage}{.3\textwidth}
  % (c) 2014 Daniele Zambelli - daniele.zambelli@gmail.com

%%%
% Segno di un polinomio: crescente, zero=4.
%%%%
 
\begin{tikzpicture}[x=1.5mm, y=1.5mm, smooth]

% (c) 2014 Daniele Zambelli - daniele.zambelli@gmail.com

%%%
% Retta crescente con segni
%%%%
 
\coordinate (inizio) at (-10, -4);
\coordinate (zero) at (0, 0);
\coordinate (fine) at (10, 4);

% (c) 2014 Daniele Zambelli - daniele.zambelli@gmail.com

%%%
% Asse cartesiano x
%%%%

\input{lbr/assiepiani/asse10.pgf}
\node [below] at (10, 0)  {$x$};


\draw [-] [ultra thick, red!50!black] (inizio) -- (zero);
\draw [-] [ultra thick, blue!50!black] (zero) -- (fine);

\node [xshift=-25, yshift=-3, above] at (zero) {$-$};
\draw[blue, thick, fill=white] (zero) circle (2pt);
\node [xshift=25, yshift=-3, above] at (zero) {$+$};

\node [above] {$4$};

\end{tikzpicture}

 \end{minipage}
 \item Studio del segno del secondo fattore~$F_2$:\\
 \begin{minipage}{.45\textwidth}
  E.A.:~$x + 2 = 0 \quad \Rightarrow \quad x=-2$
 \end{minipage}
 \begin{minipage}{.25\textwidth}
  F.A.:~$y = x + 2 \quad \rightarrow$
 \end{minipage}
 \begin{minipage}{.3\textwidth}
  % (c) 2014 Daniele Zambelli - daniele.zambelli@gmail.com

%%%
% Segno di un polinomio: crescente, zero=-2.
%%%%
 
\begin{tikzpicture}[x=1.5mm, y=1.5mm, smooth]

% (c) 2014 Daniele Zambelli - daniele.zambelli@gmail.com

%%%
% Retta crescente con segni
%%%%
 
\coordinate (inizio) at (-10, -4);
\coordinate (zero) at (0, 0);
\coordinate (fine) at (10, 4);

% (c) 2014 Daniele Zambelli - daniele.zambelli@gmail.com

%%%
% Asse cartesiano x
%%%%

\input{lbr/assiepiani/asse10.pgf}
\node [below] at (10, 0)  {$x$};


\draw [-] [ultra thick, red!50!black] (inizio) -- (zero);
\draw [-] [ultra thick, blue!50!black] (zero) -- (fine);

\node [xshift=-25, yshift=-3, above] at (zero) {$-$};
\draw[blue, thick, fill=white] (zero) circle (2pt);
\node [xshift=25, yshift=-3, above] at (zero) {$+$};

\node [above] {$-2$};

\end{tikzpicture}

 \end{minipage}
 \item Grafo dei segni:
 % (c) 2014 Daniele Zambelli - daniele.zambelli@gmail.com

%%%
% Studio dei segni di un prodotto
%%%%
 
\begin{tikzpicture}[x=2.5mm, y=5mm, smooth]

\coordinate (a_top) at (-3.3, 1);
\coordinate (a_bottom) at (-3.3, -2);
\coordinate (b_top) at (3.3, 1);
\coordinate (b_bottom) at (3.3, -2);

% (c) 2014 Daniele Zambelli - daniele.zambelli@gmail.com

%%%
% Grafo per il calcolo del segno con tre assi
%%%%
 
% (c) 2014 Daniele Zambelli - daniele.zambelli@gmail.com

%%%
% Asse cartesiano x
%%%%

\input{lbr/assiepiani/asse10.pgf}
\node [below] at (10, 0)  {$x$};

\begin{scope}[yshift= -.5cm]
  % (c) 2014 Daniele Zambelli - daniele.zambelli@gmail.com

%%%
% Asse cartesiano x
%%%%

\input{lbr/assiepiani/asse10.pgf}
\node [below] at (10, 0)  {$x$};

  \begin{scope}[yshift= -.5cm]
    % (c) 2014 Daniele Zambelli - daniele.zambelli@gmail.com

%%%
% Asse cartesiano x
%%%%

\input{lbr/assiepiani/asse10.pgf}
\node [below] at (10, 0)  {$x$};

  \end{scope}
\end{scope}

\draw [-] [] (a_top) -- (a_bottom);
\draw [-] [] (b_top) -- (b_bottom);

\node [above] at (-3.3, 1) {$-2$};
\node [above] at (3.3, 1) {$+4$};

\node [above left] at (-10, 0) {$x - 4$};
\node [above] at (-6.5, 0) {$-$};
\node [above] at (0, 0) {$-$};
\draw (3.3, .5) circle (3pt);
\node [above] at (6.5, 0) {$+$};

\node [above left] at (-10, -1) {$x + 2$};
\node [above] at (-6.5, -1) {$-$};
\draw (-3.3, -.5) circle (3pt);
\node [above] at (0, -1) {$+$};
\node [above] at (6.5, -1) {$+$};

\node [above left] at (-10, -2.15) {$(x - 4)(x + 2)$};
\node [above] at (-6.5, -2) {$+$};
\draw (-3.3, -1.5) circle (3pt);
\node [above] at (0, -2) {$-$};
\draw (3.3, -1.5) circle (3pt);
\node [above] at (6.5, -2) {$+$};

\end{tikzpicture}

 \item Possiamo concludere che il prodotto $(x -4)(x +2)$ è:
\begin{itemize} [noitemsep]
 \item positivo per~$x < -2$ o per~$x > +4$
 \item nullo per~$x = -2$ o per~$x = +4$
 \item negativo per~$x > -2$ e~$x < +4$
\end{itemize}
\end{itemize}

 \end{esempio}
% \end{exrig}

\section{Segno di un quoziente}
\label{sec:dis_quoziente}

Dato che la regola del segno del prodotto è uguale alla regola del segno del 
quoziente si può utilizzare un metodo simile a quello presentato sopra anche 
per studiare il segno di quozienti di polinomi.

C'è un'unica \emph{piccola} differenza: perché si possa calcolare una frazione, 
il suo denominatore deve essere diverso da zero. Quindi gli zeri del 
denominatore sono dei valori di~$x$ che non possiamo mai accettare. 
Per ricordarci di questo, nel grafo dei segni, li indichiamo con una crocetta 
invece che con un cerchietto.

% \begin{exrig}
 \begin{esempio}
Applichiamo questo procedimento allo studio del segno della frazione:

\[\frac{x(1 -2 x)(1 + 2 x)}{(x -2)(x +5)}\]

Chiamiamo:~$N_1$,~$N_2$~e~$N_3$ i fattori che si trovano al numeratore 
e:~$D_1$~e~$D_2$ i fattori che si trovano al denominatore.

\begin{itemize} [noitemsep]
 \item Studio del segno del fattore~$N_1$:\\
 \begin{minipage}{.45\textwidth}
  E.A.:~$x=0 \Rightarrow x=0$
 \end{minipage}
 \begin{minipage}{.25\textwidth}
  F.A.:~$y = x \quad \rightarrow$
 \end{minipage}
 \begin{minipage}{.3\textwidth}
  % (c) 2014 Daniele Zambelli - daniele.zambelli@gmail.com

%%%
% Retta crescente zero in 0
%%%%
 
\begin{tikzpicture}[x=1.5mm, y=1.5mm, smooth]

% (c) 2014 Daniele Zambelli - daniele.zambelli@gmail.com

%%%
% Retta crescente con segni
%%%%
 
\coordinate (inizio) at (-10, -4);
\coordinate (zero) at (0, 0);
\coordinate (fine) at (10, 4);

% (c) 2014 Daniele Zambelli - daniele.zambelli@gmail.com

%%%
% Asse cartesiano x
%%%%

\input{lbr/assiepiani/asse10.pgf}
\node [below] at (10, 0)  {$x$};


\draw [-] [ultra thick, red!50!black] (inizio) -- (zero);
\draw [-] [ultra thick, blue!50!black] (zero) -- (fine);

\node [xshift=-25, yshift=-3, above] at (zero) {$-$};
\draw[blue, thick, fill=white] (zero) circle (2pt);
\node [xshift=25, yshift=-3, above] at (zero) {$+$};

\node [above] (0, 0) {$0$};

\end{tikzpicture}

 \end{minipage}
 \item Studio del segno del fattore~$N_2$:\\
 \begin{minipage}{.45\textwidth}
  E.A.:~$-2 x +1=0 \Rightarrow x=\frac{1}{2}$
 \end{minipage}
 \begin{minipage}{.25\textwidth}
  F.A.:~$y = -2 x +1 \quad \rightarrow$
 \end{minipage}
 \begin{minipage}{.3\textwidth}
  % (c) 2014 Daniele Zambelli - daniele.zambelli@gmail.com

%%%
% Retta decrescente zero in 1/2
%%%%
 
\begin{tikzpicture}[x=1.5mm, y=1.5mm, smooth]

% (c) 2014 Daniele Zambelli - daniele.zambelli@gmail.com

%%%
% Retta decrescente con segni
%%%%
 
\coordinate (inizio) at (-10, 4);
\coordinate (zero) at (0, 0);
\coordinate (fine) at (10, -4);

% (c) 2014 Daniele Zambelli - daniele.zambelli@gmail.com

%%%
% Asse cartesiano x
%%%%

\input{lbr/assiepiani/asse10.pgf}
\node [below] at (10, 0)  {$x$};


\draw [-] [ultra thick, blue!50!black] (inizio) -- (zero);
\draw [-] [ultra thick, red!50!black] (zero) -- (fine);

\node [xshift=-25, yshift=-3, above] at (zero) {$+$};
\draw[blue, thick, fill=white] (zero) circle (2pt);
\node [xshift=25, yshift=-3, above] at (zero) {$-$};

\node [above] {$+\frac{1}{2}$};

\end{tikzpicture}

 \end{minipage}
 \item Studio del segno del fattore~$N_3$:\\
 \begin{minipage}{.45\textwidth}
  E.A.:~$2 x +1=0 \Rightarrow x=- \frac{1}{2}$
 \end{minipage}
 \begin{minipage}{.25\textwidth}
  F.A.:~$y=2 x +1 \quad \rightarrow$
 \end{minipage}
 \begin{minipage}{.3\textwidth}
  % (c) 2014 Daniele Zambelli - daniele.zambelli@gmail.com

%%%
% Retta crescente zero in -1/2
%%%%
 
\begin{tikzpicture}[x=1.5mm, y=1.5mm, smooth]

% (c) 2014 Daniele Zambelli - daniele.zambelli@gmail.com

%%%
% Retta crescente con segni
%%%%
 
\coordinate (inizio) at (-10, -4);
\coordinate (zero) at (0, 0);
\coordinate (fine) at (10, 4);

% (c) 2014 Daniele Zambelli - daniele.zambelli@gmail.com

%%%
% Asse cartesiano x
%%%%

\input{lbr/assiepiani/asse10.pgf}
\node [below] at (10, 0)  {$x$};


\draw [-] [ultra thick, red!50!black] (inizio) -- (zero);
\draw [-] [ultra thick, blue!50!black] (zero) -- (fine);

\node [xshift=-25, yshift=-3, above] at (zero) {$-$};
\draw[blue, thick, fill=white] (zero) circle (2pt);
\node [xshift=25, yshift=-3, above] at (zero) {$+$};

\node [above] {$-\frac{1}{2}$};

\end{tikzpicture}

 \end{minipage}
 \item Studio del segno del fattore~$D_1$:\\
 \begin{minipage}{.45\textwidth}
  E.A.:~$x-2=0 \Rightarrow x=2$
 \end{minipage}
 \begin{minipage}{.25\textwidth}
  F.A.:~$y=x-2 \quad \rightarrow$
 \end{minipage}
 \begin{minipage}{.3\textwidth}
  % (c) 2014 Daniele Zambelli - daniele.zambelli@gmail.com

%%%
% Retta crescente zero in 2
%%%%
 
\begin{tikzpicture}[x=1.5mm, y=1.5mm, smooth]

% (c) 2014 Daniele Zambelli - daniele.zambelli@gmail.com

%%%
% Retta crescente con segni
%%%%
 
\coordinate (inizio) at (-10, -4);
\coordinate (zero) at (0, 0);
\coordinate (fine) at (10, 4);

% (c) 2014 Daniele Zambelli - daniele.zambelli@gmail.com

%%%
% Asse cartesiano x
%%%%

\input{lbr/assiepiani/asse10.pgf}
\node [below] at (10, 0)  {$x$};


\draw [-] [ultra thick, red!50!black] (inizio) -- (zero);
\draw [-] [ultra thick, blue!50!black] (zero) -- (fine);

\node [xshift=-25, yshift=-3, above] at (zero) {$-$};
\draw[blue, thick, fill=white] (zero) circle (2pt);
\node [xshift=25, yshift=-3, above] at (zero) {$+$};

\node [above] {$+2$};

\end{tikzpicture}

 \end{minipage}
 \item Studio del segno del fattore~$D_2$:\\
 \begin{minipage}{.45\textwidth}
  E.A.:~$x+5=0 \Rightarrow x=-5$
 \end{minipage}
 \begin{minipage}{.25\textwidth}
  F.A.:~$y=x+5 \quad \rightarrow$
 \end{minipage}
 \begin{minipage}{.3\textwidth}
  % (c) 2014 Daniele Zambelli - daniele.zambelli@gmail.com

%%%
% Retta crescente zero in -5
%%%%
 
\begin{tikzpicture}[x=1.5mm, y=1.5mm, smooth]

% (c) 2014 Daniele Zambelli - daniele.zambelli@gmail.com

%%%
% Retta crescente con segni
%%%%
 
\coordinate (inizio) at (-10, -4);
\coordinate (zero) at (0, 0);
\coordinate (fine) at (10, 4);

% (c) 2014 Daniele Zambelli - daniele.zambelli@gmail.com

%%%
% Asse cartesiano x
%%%%

\input{lbr/assiepiani/asse10.pgf}
\node [below] at (10, 0)  {$x$};


\draw [-] [ultra thick, red!50!black] (inizio) -- (zero);
\draw [-] [ultra thick, blue!50!black] (zero) -- (fine);

\node [xshift=-25, yshift=-3, above] at (zero) {$-$};
\draw[blue, thick, fill=white] (zero) circle (2pt);
\node [xshift=25, yshift=-3, above] at (zero) {$+$};

\node [above] {$-5$};

\end{tikzpicture}

 \end{minipage}
 \item Applichiamo la regola dei segni ricordandoci di segnare con una ``X''
  gli zeri del denominatore:
  % (c) 2014 Daniele Zambelli - daniele.zambelli@gmail.com

%%%
% Segno di una frazione
%%%%
 
\begin{tikzpicture}[x=5mm, y=5mm, smooth]

\coordinate (a_top) at (-6.7, 1);
\coordinate (a_bottom) at (-6.7, -5);
\coordinate (b_top) at (-3.3, 1);
\coordinate (b_bottom) at (-3.3, -5);
\coordinate (c_top) at (0, 1);
\coordinate (c_bottom) at (0, -5);
\coordinate (d_top) at (3.3, 1);
\coordinate (d_bottom) at (3.3, -5);
\coordinate (e_top) at (6.7, 1);
\coordinate (e_bottom) at (6.7, -5);

% (c) 2014 Daniele Zambelli - daniele.zambelli@gmail.com

%%%
% Grafo per il calcolo del segno con sei assi
%%%%
 
% (c) 2014 Daniele Zambelli - daniele.zambelli@gmail.com

%%%
% Asse cartesiano x
%%%%

\input{lbr/assiepiani/asse10.pgf}
\node [below] at (10, 0)  {$x$};

\begin{scope}[yshift= -.5cm]
  % (c) 2014 Daniele Zambelli - daniele.zambelli@gmail.com

%%%
% Asse cartesiano x
%%%%

\input{lbr/assiepiani/asse10.pgf}
\node [below] at (10, 0)  {$x$};

  \begin{scope}[yshift= -.5cm]
    % (c) 2014 Daniele Zambelli - daniele.zambelli@gmail.com

%%%
% Asse cartesiano x
%%%%

\input{lbr/assiepiani/asse10.pgf}
\node [below] at (10, 0)  {$x$};

    \begin{scope}[yshift= -.5cm]
      % (c) 2014 Daniele Zambelli - daniele.zambelli@gmail.com

%%%
% Asse cartesiano x
%%%%

\input{lbr/assiepiani/asse10.pgf}
\node [below] at (10, 0)  {$x$};

      \begin{scope}[yshift= -.5cm]
        % (c) 2014 Daniele Zambelli - daniele.zambelli@gmail.com

%%%
% Asse cartesiano x
%%%%

\input{lbr/assiepiani/asse10.pgf}
\node [below] at (10, 0)  {$x$};

        \begin{scope}[yshift= -.5cm]
          % (c) 2014 Daniele Zambelli - daniele.zambelli@gmail.com

%%%
% Asse cartesiano x
%%%%

\input{lbr/assiepiani/asse10.pgf}
\node [below] at (10, 0)  {$x$};

        \end{scope}
      \end{scope}
    \end{scope}
  \end{scope}
\end{scope}

\draw [-] [] (a_top) -- (a_bottom);
\draw [-] [] (b_top) -- (b_bottom);
\draw [-] [] (c_top) -- (c_bottom);
\draw [-] [] (d_top) -- (d_bottom);
\draw [-] [] (e_top) -- (e_bottom);

\node [above] at (a_top) {$-5$};
\node [above] at (b_top) {$-\frac{1}{2}$};
\node [above] at (c_top) {$0$};
\node [above] at (d_top) {$\frac{1}{2}$};
\node [above] at (e_top) {$2$};

\node [above left] at (-10, 0) {$x$};
\node [above] at (-8, 0) {$-$};
\node [above] at (-5, 0) {$-$};
\node [above] at (-1.7, 0) {$-$};
\draw (0, .5) circle (3pt);
\node [above] at (1.7, 0) {$+$};
\node [above] at (5, 0) {$+$};
\node [above] at (8, 0) {$+$};

\node [above left] at (-10, -1) {$-2 x + 1$};
\node [above] at (-8, -1) {$+$};
\node [above] at (-5, -1) {$+$};
\node [above] at (-1.7, -1) {$+$};
\node [above] at (1.7, -1) {$+$};
\draw (3.3, -.5) circle (3pt);
\node [above] at (5, -1) {$-$};
\node [above] at (8, -1) {$-$};

\node [above left] at (-10, -2) {$2 x + 1$};
\node [above] at (-8, -2) {$-$};
\node [above] at (-5, -2) {$-$};
\draw (-3.3, -1.5) circle (3pt);
\node [above] at (-1.7, -2) {$+$};
\node [above] at (1.7, -2) {$+$};
\node [above] at (5, -2) {$+$};
\node [above] at (8, -2) {$+$};

\node [above left] at (-10, -3) {$x - 2$};
\node [above] at (-8, -3) {$-$};
\draw (-6.7 -.2, -2.5 -.2) -- (-6.7 +.2, -2.5 +.2) 
      (-6.7 -.2, -2.5 +.2) -- (-6.7 +.2, -2.5 -.2);
\node [above] at (-5, -3) {$+$};
\node [above] at (-1.7, -3) {$+$};
\node [above] at (1.7, -3) {$+$};
\node [above] at (5, -3) {$+$};
\node [above] at (8, -3) {$+$};

\node [above left] at (-10, -4) {$x + 5$};
\node [above] at (-8, -4) {$-$};
\node [above] at (-5, -4) {$-$};
\node [above] at (-1.7, -4) {$-$};
\node [above] at (1.7, -4) {$-$};
\node [above] at (5, -4) {$-$};
\draw (6.7 -.2, -3.5 -.2) -- (6.7 +.2, -3.5 +.2) 
      (6.7 -.2, -3.5 +.2) -- (6.7 +.2, -3.5 -.2);
\node [above] at (8, -4) {$+$};

\node [above left] at (-10, -5.15) {$f(x)$};
\node [above] at (-8, -5) {$+$};
\draw (-6.7 -.2, -4.5 -.2) -- (-6.7 +.2, -4.5 +.2) 
      (-6.7 -.2, -4.5 +.2) -- (-6.7 +.2, -4.5 -.2);
\node [above] at (-5, -5) {$-$};
\draw (-3.3, -4.5) circle (3pt);
\node [above] at (-1.7, -5) {$+$};
\draw (0, -4.5) circle (3pt);
\node [above] at (1.7, -5) {$-$};
\draw (3.3, -4.5) circle (3pt);
\node [above] at (5, -5) {$+$};
\draw (6.7 -.2, -4.5 -.2) -- (6.7 +.2, -4.5 +.2) 
      (6.7 -.2, -4.5 +.2) -- (6.7 +.2, -4.5 -.2);
\node [above] at (8, -5) {$-$};

\end{tikzpicture}

 \item Possiamo concludere che la frazione$\frac{x(1 -2 x)(1 + 2 x)}{(x -2)(x 
+5)}$ è:
\begin{itemize} [noitemsep]
 \item positiva per tutti i valori di~$x$ minori di~$-5$
 \item non definita per~$x=-5$
 \item negativa per tutti i valori di~$x$ compresi tra~$-5$ e~$-\frac{1}{2}$
 \item zero per~$x=-\frac{1}{2}$
 \item positiva per tutti i valori di~$x$ compresi tra~$-\frac{1}{2}$ e~$0$
 \item zero per~$x=0$
 \item negativa per tutti i valori di~$x$ compresi tra~$0$ e~$\frac{1}{2}$
 \item zero per~$x=\frac{1}{2}$
 \item positiva per tutti i valori di~$x$ compresi tra~$\frac{1}{2}$ e~$2$
 \item non definita per~$x=2$
 \item negativa per tutti i valori di~$x$ maggiori di~$2$.
\end{itemize}
\end{itemize}
 \end{esempio}
% \end{exrig}




O in simboli:

$x \in \mathbb{R} - \{ -5;~2 \}$ (insieme di esistenza)

$\begin{array}{cccccccccc}
f(x) & > & 0 & \Leftrightarrow &
 x < -5 & \lor & -\frac{1}{2} < x < 0 & \lor & \frac{1}{2} < x < 2 \\
 
f(x) & = & 0 & \Leftrightarrow &
 x = -\frac{1}{2} & \lor & x = 0 & \lor & x = \frac{1}{2} \\
 
f(x) & < & 0 & \Leftrightarrow &
 -5 < x < -\frac{1}{2} & \lor & 0 < x < \frac{1}{2} & \lor & x > 2 \\
\end{array}$

\section{Disequazioni numeriche}
\label{sec:dis_numeriche}

\subsection{Principi di equivalenza delle disequazioni}

Per lavorare sulle disequazioni si ricorre a due principi, che derivano 
direttamente dalle proprietà delle disuguaglianze.

Stabiliamo innanzitutto la seguente definizione:

\begin{definizione}
Due disequazioni si dicono \emph{equivalenti} se hanno lo
stesso insieme soluzione.
\end{definizione}

Stabilito questo, possiamo formulare due principi di equivalenza
simili a quelli validi per le equazioni

\begin{principio}[I principio]
\label{ppd}
Addizionando o sottraendo a ciascuno dei due membri di
una disequazione una stessa espressione (definita per qualunque
valore attribuito all'incognita), si ottiene una
disequazione equivalente alla data.
\end{principio}

Regola pratica: questo principio ci permette di
``spostare'' un addendo da un membro
all'altro cambiandogli segno o di
``eliminare'' da entrambi i membri
gli addendi uguali.

\begin{principio}[II principio]
Moltiplicando o dividendo ciascuno dei due membri di
una disequazione per una stessa espressione definita e positiva, 
si ottiene una disequazione equivalente alla data.
Moltiplicando o dividendo ciascuno dei due membri di
una disequazione per una stessa espressione definita e negativa, 
e cambiando il verso della disuguaglianza,
si ottiene una disequazione equivalente alla data.
\end{principio}

Ora si può osservare che il primo principio è semplice e esattamente uguale a 
quello delle equazioni, il secondo invece è insidioso... 
Per fortuna per risolvere le disequazioni basta usare il primo ed al massimo 
la prima parte del secondo.

% \begin{exrig}
 \begin{esempio}
Consideriamo la disequazione:~$3x + 2 > 5 x -4$.
\begin{itemize} [noitemsep]
 \item sommiamo ad entrambi i membri l'espressione:~$-5 x +4$ 
 \item la disequazione di partenza è diventata:~$-2 x +6 > 0$ e
  il primo principio ci assicura che le soluzioni di questa disequazione sono 
  tutte e sole le soluzioni della disequazione di partenza.
\end{itemize}
 \end{esempio}
% \end{exrig}

Chiamiamo disequazione scritta in forma \emph{normale} (o \emph{canonica})
la disequazione trasformata in modo da avere il secondo membro uguale a zero.

\subsection{Soluzione di una disequazione lineare}

Usando il primo principio si può sempre scrivere una qualunque disequazione 
lineare in forma normale. A questo punto è facile studiare il segno del 
polinomio che si trova a primo membro e quindi risolvere la disequazione.

% \begin{exrig}
 \begin{esempio}
Riprendiamo la disequazione precedente:~$-2 x +6 > 0$.
\begin{itemize} [noitemsep]
 \item Studio del segno del polinomio: \\
 \begin{minipage}{.45\textwidth}
  E.A.:~$-2 x +6=0 \Rightarrow  x=3$
 \end{minipage}
 \begin{minipage}{.25\textwidth}
  F.A.:~$y=-2 x +6 \rightarrow $
 \end{minipage}
 \begin{minipage}{.3\textwidth}
  % (c) 2014 Daniele Zambelli - daniele.zambelli@gmail.com

%%%
% Retta decrescente zero in 3
%%%%
 
\begin{tikzpicture}[x=1.5mm, y=1.5mm, smooth]

% (c) 2014 Daniele Zambelli - daniele.zambelli@gmail.com

%%%
% Retta decrescente con segni
%%%%
 
\coordinate (inizio) at (-10, 4);
\coordinate (zero) at (0, 0);
\coordinate (fine) at (10, -4);

% (c) 2014 Daniele Zambelli - daniele.zambelli@gmail.com

%%%
% Asse cartesiano x
%%%%

\input{lbr/assiepiani/asse10.pgf}
\node [below] at (10, 0)  {$x$};


\draw [-] [ultra thick, blue!50!black] (inizio) -- (zero);
\draw [-] [ultra thick, red!50!black] (zero) -- (fine);

\node [xshift=-25, yshift=-3, above] at (zero) {$+$};
\draw[blue, thick, fill=white] (zero) circle (2pt);
\node [xshift=25, yshift=-3, above] at (zero) {$-$};

\node [above] {$3$};

\end{tikzpicture}

 \end{minipage}
 \item Quindi i valori di~$x$ che rendono positivo il binomio sono quelli che 
si trovano a sinistra di~3 cioè quelli minori di~3. 
 \subitem 
  \begin{minipage}{.35\textwidth}
   rappresentazione grafica: 
  \end{minipage}
  \begin{minipage}{.30\textwidth}
   % (c) 2014 Daniele Zambelli - daniele.zambelli@gmail.com

%%%
% Soluzione di una disequazione ]oo; 3[
%%%%
 
\begin{tikzpicture}[x=1.5mm, y=1.5mm, smooth]

% \clip (-7.5, -5.5) rectangle (10.9, 10.9);

\coordinate (a) at (0, 0);
\coordinate (b) at (-10, 0);

% (c) 2014 Daniele Zambelli - daniele.zambelli@gmail.com

%%%
% Asse cartesiano x
%%%%

% (c) 2014 Daniele Zambelli - daniele.zambelli@gmail.com

%%%
% Asse cartesiano generico
%%%%

\draw [-{Stealth[length=2mm, open, round]}] (-10, 0) -- (10, 0);

\node [below] at (10, 0)  {$x$};


\begin{scope}[blue,thick]
\draw [-,decorate,decoration=snake] (a) -- (b);
\draw[fill=white] (a) circle (3pt) node [above] {3};
\end{scope}

\end{tikzpicture}

  \end{minipage}
 \subitem rappresentazione con i predicati:~$x < 3$ 
 \subitem rappresentazione con le parentesi:~$]-∞; 3[$. 
\end{itemize}
 \end{esempio}
% \end{exrig}

Riassumendo:

\begin{procedura}
 Per risolvere una disequazione:
\begin{enumeratea}
 \item scrivere la disequazione in forma normale;
 \item studiare il segno dell'espressione a sinistra del predicato;
 \item rappresentare, con i diversi metodi visti, 
  gli intervalli che risolvono la disequazione.
\end{enumeratea}
\end{procedura}

\subsection{Un caso particolare}

A volte nel risolvere una disequazione ci imbattiamo in un'equazione associata 
impossibile. La prima reazione istintiva è quella di pensare che se l'E.A. è 
impossibile lo sarà anche la disequazione, ma non è così! 
Se l'E.A. è impossibile ciò significa che la retta non interseca l'asse~$x$, 
cioè è parallela all'asse~$x$. In questo caso tutti i valori della funzione 
staranno dalla stessa parte dell'asse cioè saranno tutti positivi o tutti 
negativi. Vediamo qualche esempio.

% \begin{exrig}

\begin{esempio}
$\dfrac{1}{2} (x+5)-x>\dfrac{1}{2} (3-x).$
Il~$\mcm$ è~2, positivo; moltiplichiamo ambo i membri per~2; svolgiamo
i calcoli:

\[2 \left[\frac{1}{2}(x+5)-x\right]>2
\left[\frac{1}{2}(3-x)\right]\Rightarrow x+5-2x>3-x\Rightarrow -x+5>3-x.\]

La forma canonica è~$0 x + 2 > 0$ che si riduce alla disuguaglianza~$0>-2$
vera per qualunque~$x$ reale:~$\IS=\insR$.
\begin{itemize} [noitemsep]
 \item Studio del segno del polinomio:\\
 \begin{minipage}{.45\textwidth}
  E.A.:~$0 x + 2 = 0  \Rightarrow \quad $ eq. impossibile;
 \end{minipage}
 \begin{minipage}{.25\textwidth}
  F.A.:~$y=0 x + 2 \rightarrow $
 \end{minipage}
 \begin{minipage}{.3\textwidth}
  % (c) 2014 Daniele Zambelli - daniele.zambelli@gmail.com

%%%
% Segno di una costante negativa.
%%%%
 
\begin{tikzpicture}[x=1.5mm, y=1.5mm, smooth]

% (c) 2014 Daniele Zambelli - daniele.zambelli@gmail.com

%%%
% Segno di una costante positiva.
%%%%
 
\coordinate (inizio) at (-10, 3);
\coordinate (zero) at (0, 0);
\coordinate (fine) at (10, 3);

% (c) 2014 Daniele Zambelli - daniele.zambelli@gmail.com

%%%
% Asse cartesiano x
%%%%

\input{lbr/assiepiani/asse10.pgf}
\node [below] at (10, 0)  {$x$};


\draw [-] [ultra thick, blue!50!black] (inizio) -- (fine);

\node [xshift=-25, yshift=-3, above] at (zero) {$+$};
\node [xshift=25, yshift=-3, above] at (zero) {$+$};


\end{tikzpicture}

 \end{minipage}
 \item Quindi per ogni valore di~$x$ il polinomio è sempre positivo. 
 \subitem 
  \begin{minipage}{.35\textwidth}
   rappresentazione grafica: 
  \end{minipage}
  \begin{minipage}{.30\textwidth}
   % (c) 2014 Daniele Zambelli - daniele.zambelli@gmail.com

%%%
% Intervallo ]oo; 2/9[ 
%%%%
 
\begin{tikzpicture}[x=1.5mm, y=1.5mm, smooth]

% \clip (-7.5, -5.5) rectangle (10.9, 10.9);

\coordinate (a) at (-10, 0);
\coordinate (b) at (+10, 0);

% (c) 2014 Daniele Zambelli - daniele.zambelli@gmail.com

%%%
% Asse cartesiano x
%%%%

% (c) 2014 Daniele Zambelli - daniele.zambelli@gmail.com

%%%
% Asse cartesiano generico
%%%%

\draw [-{Stealth[length=2mm, open, round]}] (-10, 0) -- (10, 0);

\node [below] at (10, 0)  {$x$};


\begin{scope}[blue,thick]
\draw [-,decorate,decoration=snake] (a) -- (b);
\end{scope}

\end{tikzpicture}

  \end{minipage}
 \subitem rappresentazione con i predicati:~$\mathbb{R}$ 
 \subitem rappresentazione con le parentesi:~$]-\infty; +\infty[$. 
\end{itemize}
\end{esempio}


\begin{esempio}
$(x+2)^2-4(x+1)<x^{2}-1.$
Svolgiamo i calcoli ed eliminiamo i monomi simili:
\[x^{2}+4x+4-4x-4<x^{2}-1\Rightarrow~0 x + 1 < 0,\]
che è la disuguaglianza~$0<-1$ falsa per qualunque~$x$ reale:~$\IS=\emptyset $.
%\vspace*{1.05ex}
\begin{itemize} [noitemsep]
 \item Studio del segno del polinomio:\\
 \begin{minipage}{.45\textwidth}
  E.A.:~$0 x + 1 = 0  \Rightarrow \quad $ eq. impossibile;
 \end{minipage}
 \begin{minipage}{.25\textwidth}
  F.A.:~$y=0 x + 1 \rightarrow $
 \end{minipage}
 \begin{minipage}{.3\textwidth}
  % (c) 2014 Daniele Zambelli - daniele.zambelli@gmail.com

%%%
% Segno di una costante negativa.
%%%%
 
\begin{tikzpicture}[x=1.5mm, y=1.5mm, smooth]

% (c) 2014 Daniele Zambelli - daniele.zambelli@gmail.com

%%%
% Segno di una costante positiva.
%%%%
 
\coordinate (inizio) at (-10, 3);
\coordinate (zero) at (0, 0);
\coordinate (fine) at (10, 3);

% (c) 2014 Daniele Zambelli - daniele.zambelli@gmail.com

%%%
% Asse cartesiano x
%%%%

\input{lbr/assiepiani/asse10.pgf}
\node [below] at (10, 0)  {$x$};


\draw [-] [ultra thick, blue!50!black] (inizio) -- (fine);

\node [xshift=-25, yshift=-3, above] at (zero) {$+$};
\node [xshift=25, yshift=-3, above] at (zero) {$+$};


\end{tikzpicture}

 \end{minipage}
 \item Quindi per ogni valore di~$x$ il polinomio è sempre positivo. 
  Ma a noi servono i valori di~$x$ che rendono il polinomio negativo 
  quindi\dots
 \subitem 
  \begin{minipage}{.35\textwidth}
   rappresentazione grafica: 
  \end{minipage}
  \begin{minipage}{.30\textwidth}
   % (c) 2014 Daniele Zambelli - daniele.zambelli@gmail.com

%%%
% Intervallo ]oo; 2/9[ 
%%%%
 
\begin{tikzpicture}[x=1.5mm, y=1.5mm, smooth]

% \clip (-7.5, -5.5) rectangle (10.9, 10.9);

\coordinate (a) at (-10, 0);
\coordinate (b) at (+10, 0);

% (c) 2014 Daniele Zambelli - daniele.zambelli@gmail.com

%%%
% Asse cartesiano x
%%%%

% (c) 2014 Daniele Zambelli - daniele.zambelli@gmail.com

%%%
% Asse cartesiano generico
%%%%

\draw [-{Stealth[length=2mm, open, round]}] (-10, 0) -- (10, 0);

\node [below] at (10, 0)  {$x$};


\end{tikzpicture}

  \end{minipage}
 \subitem rappresentazione con i simboli:~$\emptyset$. 
\end{itemize}
\end{esempio}
% \end{exrig}

\subsection{Prodotto o quoziente di polinomi}
\label{sec:dis_prod_quo}

Se una disequazione è formata dal prodotto o dal quoziente di polinomi, 
una volta scritta in forma normale,
dobbiamo studiare il segno di tutto il prodotto o il quoziente e poi risolvere
la disequazione.
Vediamo un esempio.

% \begin{exrig}
 \begin{esempio}
Consideriamo:~$\frac{(x-2)(-2x+1)}{x+3} \ge 0$.

Dobbiamo studiare il segno di ogni singolo fattore:
\begin{itemize} [noitemsep]
 \item segno di $(x-2)$\\
 \begin{minipage}{.45\textwidth}
  E.A.:~$x-2=0 \Rightarrow x=2$
 \end{minipage}
 \begin{minipage}{.25\textwidth}
  F.A.:~$y=x-2 \rightarrow $
 \end{minipage}
 \begin{minipage}{.3\textwidth}
  % (c) 2014 Daniele Zambelli - daniele.zambelli@gmail.com

%%%
% Retta crescente zero in 2
%%%%
 
\begin{tikzpicture}[x=1.5mm, y=1.5mm, smooth]

% (c) 2014 Daniele Zambelli - daniele.zambelli@gmail.com

%%%
% Retta crescente con segni
%%%%
 
\coordinate (inizio) at (-10, -4);
\coordinate (zero) at (0, 0);
\coordinate (fine) at (10, 4);

% (c) 2014 Daniele Zambelli - daniele.zambelli@gmail.com

%%%
% Asse cartesiano x
%%%%

\input{lbr/assiepiani/asse10.pgf}
\node [below] at (10, 0)  {$x$};


\draw [-] [ultra thick, red!50!black] (inizio) -- (zero);
\draw [-] [ultra thick, blue!50!black] (zero) -- (fine);

\node [xshift=-25, yshift=-3, above] at (zero) {$-$};
\draw[blue, thick, fill=white] (zero) circle (2pt);
\node [xshift=25, yshift=-3, above] at (zero) {$+$};

\node [above] {$+2$};

\end{tikzpicture}

 \end{minipage}
 \item segno di $(-2x+1)$\\
 \begin{minipage}{.45\textwidth}
  E.A.:~$-2x+1=0 \Rightarrow x=\frac{1}{2}$
 \end{minipage}
 \begin{minipage}{.25\textwidth}
  F.A.:~$y=-2x+1 \rightarrow $
 \end{minipage}
 \begin{minipage}{.3\textwidth}
  % (c) 2014 Daniele Zambelli - daniele.zambelli@gmail.com

%%%
% Retta decrescente zero in 1/2
%%%%
 
\begin{tikzpicture}[x=1.5mm, y=1.5mm, smooth]

% (c) 2014 Daniele Zambelli - daniele.zambelli@gmail.com

%%%
% Retta decrescente con segni
%%%%
 
\coordinate (inizio) at (-10, 4);
\coordinate (zero) at (0, 0);
\coordinate (fine) at (10, -4);

% (c) 2014 Daniele Zambelli - daniele.zambelli@gmail.com

%%%
% Asse cartesiano x
%%%%

\input{lbr/assiepiani/asse10.pgf}
\node [below] at (10, 0)  {$x$};


\draw [-] [ultra thick, blue!50!black] (inizio) -- (zero);
\draw [-] [ultra thick, red!50!black] (zero) -- (fine);

\node [xshift=-25, yshift=-3, above] at (zero) {$+$};
\draw[blue, thick, fill=white] (zero) circle (2pt);
\node [xshift=25, yshift=-3, above] at (zero) {$-$};

\node [above] {$\frac{1}{2}$};

\end{tikzpicture}

 \end{minipage}
 \item Segno del denominatore:\\
 \begin{minipage}{.45\textwidth}
  E.A.:~$x + 3=0 \Rightarrow x=-3$
 \end{minipage}
 \begin{minipage}{.25\textwidth}
  F.A.:~$y=x +3 \rightarrow $
 \end{minipage}
 \begin{minipage}{.3\textwidth}
  % (c) 2014 Daniele Zambelli - daniele.zambelli@gmail.com

%%%
% Retta decrescente zero in 1/2
%%%%
 
\begin{tikzpicture}[x=1.5mm, y=1.5mm, smooth]

% (c) 2014 Daniele Zambelli - daniele.zambelli@gmail.com

%%%
% Retta decrescente con segni
%%%%
 
\coordinate (inizio) at (-10, 4);
\coordinate (zero) at (0, 0);
\coordinate (fine) at (10, -4);

% (c) 2014 Daniele Zambelli - daniele.zambelli@gmail.com

%%%
% Asse cartesiano x
%%%%

\input{lbr/assiepiani/asse10.pgf}
\node [below] at (10, 0)  {$x$};


\draw [-] [ultra thick, blue!50!black] (inizio) -- (zero);
\draw [-] [ultra thick, red!50!black] (zero) -- (fine);

\node [xshift=-25, yshift=-3, above] at (zero) {$+$};
\draw[blue, thick, fill=white] (zero) circle (2pt);
\node [xshift=25, yshift=-3, above] at (zero) {$-$};

\node [above] {$\frac{3}{2}$};

\end{tikzpicture}

 \end{minipage}
 \item Con la regola dei segni calcolo il segno della frazione 
  % (c) 2014 Daniele Zambelli - daniele.zambelli@gmail.com

%%%
% Studio dei segni di un prodotto
%%%%
 
\begin{tikzpicture}[x=2.5mm, y=5mm, smooth]

\coordinate (a_top) at (-5, 1);
\coordinate (a_bottom) at (-5, -3);
\coordinate (b_top) at (0, 1);
\coordinate (b_bottom) at (0, -3);
\coordinate (c_top) at (5, 1);
\coordinate (c_bottom) at (5, -3);

% (c) 2014 Daniele Zambelli - daniele.zambelli@gmail.com

%%%
% Grafo per il calcolo del segno con tre assi
%%%%
 
% (c) 2014 Daniele Zambelli - daniele.zambelli@gmail.com

%%%
% Asse cartesiano x
%%%%

\input{lbr/assiepiani/asse10.pgf}
\node [below] at (10, 0)  {$x$};

\begin{scope}[yshift= -.5cm]
  % (c) 2014 Daniele Zambelli - daniele.zambelli@gmail.com

%%%
% Asse cartesiano x
%%%%

\input{lbr/assiepiani/asse10.pgf}
\node [below] at (10, 0)  {$x$};

  \begin{scope}[yshift= -.5cm]
    % (c) 2014 Daniele Zambelli - daniele.zambelli@gmail.com

%%%
% Asse cartesiano x
%%%%

\input{lbr/assiepiani/asse10.pgf}
\node [below] at (10, 0)  {$x$};

    \begin{scope}[yshift= -.5cm]
      % (c) 2014 Daniele Zambelli - daniele.zambelli@gmail.com

%%%
% Asse cartesiano x
%%%%

\input{lbr/assiepiani/asse10.pgf}
\node [below] at (10, 0)  {$x$};

    \end{scope}
  \end{scope}
\end{scope}

\draw [-] [] (a_top) -- (a_bottom);
\draw [-] [] (b_top) -- (b_bottom);
\draw [-] [] (c_top) -- (c_bottom);

\node [above] at (-5, 1) {$-2$};
\node [above] at (0, 1) {$+\frac{3}{2}$};
\node [above] at (5, 1) {$+2$};

\node [above left] at (-10, 0) {$x-2$};
\node [above] at (-7.5, 0) {$-$};
\node [above] at (-2.5, 0) {$-$};
\node [above] at (2.5, 0) {$-$};
\draw (5, .5) circle (3pt);
\node [above] at (7.5, 0) {$+$};

\node [above left] at (-10, -1) {$x+2$};
\node [above] at (-7.5, -1) {$-$};
\draw (-5, -.5) circle (3pt);
\node [above] at (-2.5, -1) {$+$};
\node [above] at (2.5, -1) {$+$};
\node [above] at (7.5, -1) {$+$};

\node [above left] at (-10, -2) {$-2x+3$};
\node [above] at (-7.5, -2) {$+$};
% \draw (0 -.4, -1.5 -.2) -- (0 +.4, -1.5 +.2) 
%       (0 -.4, -1.5 +.2) -- (0 +.4, -1.5 -.2);
\node [above] at (-2.5, -2) {$+$};
\draw (0, -1.5) circle (3pt);
\node [above] at (2.5, -2) {$-$};
\node [above] at (7.5, -2) {$-$};

\node [above left] at (-10, -3.15) {$P(x)$};
\node [above] at (-7.5, -3) {$+$};
\draw (-5, -2.5) circle (3pt);
\node [above] at (-2.5, -3) {$-$};
\draw (0, -2.5) circle (3pt);
\node [above] at (2.5, -3) {$+$};
\draw (5, -2.5) circle (3pt);
\node [above] at (7.5, -3) {$-$};

\end{tikzpicture}

       %{\folder lbr/fig035_grafosegni03.pgf} 
 \item Quindi i valori di~$x$ che rendono vera la disequazione, cioè i valori
  che rendono~$f(x)$ non negativo, sono quelli 
  che si trovano a sinistra di~$-2$ oppure che si trovano a destra di~$+3$. 
 \subitem 
  \begin{minipage}{.35\textwidth}
   rappresentazione grafica: 
  \end{minipage}
  \begin{minipage}{.30\textwidth}
   % (c) 2014 Daniele Zambelli - daniele.zambelli@gmail.com

%%%
% Valori esterni all'intervallo -2; 3
%%%%
 
\begin{tikzpicture}[x=1.5mm, y=1.5mm, smooth]

% \clip (-7.5, -5.5) rectangle (10.9, 10.9);

\coordinate (m_i) at (-10, 0);
\coordinate (a) at (-5, 0);
\coordinate (b) at (0, 0);
\coordinate (c) at (5, 0);
\coordinate (p_i) at (10, 0);

% (c) 2014 Daniele Zambelli - daniele.zambelli@gmail.com

%%%
% Asse cartesiano x
%%%%

% (c) 2014 Daniele Zambelli - daniele.zambelli@gmail.com

%%%
% Asse cartesiano generico
%%%%

\draw [-{Stealth[length=2mm, open, round]}] (-10, 0) -- (10, 0);

\node [below] at (10, 0)  {$x$};


\begin{scope}[blue,thick]
\draw [-,decorate,decoration=snake] (m_i) -- (a);
\draw[fill=white] (a) circle (2pt) node [above] {$-3$};
\draw [-,decorate,decoration=snake] (b) -- (c);
\draw[fill] (b) circle (2pt) node [above] {$\frac{1}{2}$};
\draw [fill=white] (c) -- (p_i);
\draw[fill] (c) circle (2pt) node [above] {$2$};
\end{scope}

\end{tikzpicture}
%{\folder lbr/fig036_diseq02.pgf}
  \end{minipage}
 \subitem rappresentazione con i 
   predicati:~$x < -3 \lor \frac{1}{2} \le x \le 3$ 
 \subitem rappresentazione con le 
  parentesi:~$]-\infty; -3[ \quad \cup \quad [\frac{1}{2}; 2;]$. 
\end{itemize}
 \end{esempio}
% \end{exrig}

\subsubsection{Soluzione di disequazioni fratte}

Quando una disequazione contiene la variabile al denominatore ma non è composta
da un'unica frazione, dobbiamo seguire la procedura:

\begin{procedura}
 Per risolvere una disequazione fratta:
\begin{enumeratea}
 \item spostare tutti i termini a primo membro e sommarli in modo da ottenere 
 una sola frazione e a secondo membro solo lo zero;
 \item studiare il segno della frazione;
 \item rappresentare, con i diversi metodi visti, 
  gli intervalli che risolvono la disequazione.
\end{enumeratea}
\end{procedura}

% \begin{exrig}
 \begin{esempio}
Consideriamo:~$\frac{-3 x +4}{x+2} \le -1$.
\begin{itemize} [noitemsep]
 \item scrivere la disequazione in forma normale:
 $\frac{-3 x +4}{x+2} \le -1 \Rightarrow \frac{-3 x +4}{x+2} +1 \le 0 \newline
 \frac{-3 x +4 + x +2}{x+2} \le 0 \rightarrow \frac{-2 x +6}{x+2} \le 0$
 \item Segno del numeratore:\\
 \begin{minipage}{.45\textwidth}
  E.A.:~$-2 x +6=0 \Rightarrow x=3$
 \end{minipage}
 \begin{minipage}{.25\textwidth}
  F.A.:~$y=-2 x +6 \rightarrow $
 \end{minipage}
 \begin{minipage}{.3\textwidth}
  % (c) 2014 Daniele Zambelli - daniele.zambelli@gmail.com

%%%
% Retta decrescente zero in 3
%%%%
 
\begin{tikzpicture}[x=1.5mm, y=1.5mm, smooth]

% (c) 2014 Daniele Zambelli - daniele.zambelli@gmail.com

%%%
% Retta decrescente con segni
%%%%
 
\coordinate (inizio) at (-10, 4);
\coordinate (zero) at (0, 0);
\coordinate (fine) at (10, -4);

% (c) 2014 Daniele Zambelli - daniele.zambelli@gmail.com

%%%
% Asse cartesiano x
%%%%

\input{lbr/assiepiani/asse10.pgf}
\node [below] at (10, 0)  {$x$};


\draw [-] [ultra thick, blue!50!black] (inizio) -- (zero);
\draw [-] [ultra thick, red!50!black] (zero) -- (fine);

\node [xshift=-25, yshift=-3, above] at (zero) {$+$};
\draw[blue, thick, fill=white] (zero) circle (2pt);
\node [xshift=25, yshift=-3, above] at (zero) {$-$};

\node [above] {$3$};

\end{tikzpicture}

 \end{minipage}
 \item Segno del denominatore:\\
 \begin{minipage}{.45\textwidth}
  E.A.:~$x + 2=0 \Rightarrow x=-2$
 \end{minipage}
 \begin{minipage}{.25\textwidth}
  F.A.:~$y=x +2 \rightarrow $
 \end{minipage}
 \begin{minipage}{.3\textwidth}
  % (c) 2014 Daniele Zambelli - daniele.zambelli@gmail.com

%%%
% Retta crescente zero in -2
%%%%
 
\begin{tikzpicture}[x=1.5mm, y=1.5mm, smooth]

% (c) 2014 Daniele Zambelli - daniele.zambelli@gmail.com

%%%
% Retta crescente con segni
%%%%
 
\coordinate (inizio) at (-10, -4);
\coordinate (zero) at (0, 0);
\coordinate (fine) at (10, 4);

% (c) 2014 Daniele Zambelli - daniele.zambelli@gmail.com

%%%
% Asse cartesiano x
%%%%

\input{lbr/assiepiani/asse10.pgf}
\node [below] at (10, 0)  {$x$};


\draw [-] [ultra thick, red!50!black] (inizio) -- (zero);
\draw [-] [ultra thick, blue!50!black] (zero) -- (fine);

\node [xshift=-25, yshift=-3, above] at (zero) {$-$};
\draw[blue, thick, fill=white] (zero) circle (2pt);
\node [xshift=25, yshift=-3, above] at (zero) {$+$};

\node [above] {$-2$};

\end{tikzpicture}

 \end{minipage}
 \item Con la regola dei segni calcolo il segno della frazione 
  % (c) 2014 Daniele Zambelli - daniele.zambelli@gmail.com

%%%
% Studio dei segni di una frazione
%%%%
 
\begin{tikzpicture}[x=2.5mm, y=5mm, smooth]

\coordinate (a_top) at (-3.3, 1);
\coordinate (a_bottom) at (-3.3, -2);
\coordinate (b_top) at (3.3, 1);
\coordinate (b_bottom) at (3.3, -2);

% (c) 2014 Daniele Zambelli - daniele.zambelli@gmail.com

%%%
% Grafo per il calcolo del segno con tre assi
%%%%
 
% (c) 2014 Daniele Zambelli - daniele.zambelli@gmail.com

%%%
% Asse cartesiano x
%%%%

\input{lbr/assiepiani/asse10.pgf}
\node [below] at (10, 0)  {$x$};

\begin{scope}[yshift= -.5cm]
  % (c) 2014 Daniele Zambelli - daniele.zambelli@gmail.com

%%%
% Asse cartesiano x
%%%%

\input{lbr/assiepiani/asse10.pgf}
\node [below] at (10, 0)  {$x$};

  \begin{scope}[yshift= -.5cm]
    % (c) 2014 Daniele Zambelli - daniele.zambelli@gmail.com

%%%
% Asse cartesiano x
%%%%

\input{lbr/assiepiani/asse10.pgf}
\node [below] at (10, 0)  {$x$};

  \end{scope}
\end{scope}

\draw [-] [] (a_top) -- (a_bottom);
\draw [-] [] (b_top) -- (b_bottom);

\node [above] at (-3.3, 1) {$-2$};
\node [above] at (3.3, 1) {$+3$};

\node [above left] at (-10, 0) {$-3 x + 4$};
\node [above] at (-6.5, 0) {$+$};
\node [above] at (0, 0) {$+$};
\draw (3.3, .5) circle (3pt);
\node [above] at (6.5, 0) {$-$};

\node [above left] at (-10, -1) {$x + 2$};
\node [above] at (-6.5, -1) {$-$};
\draw (-3.3 -.4, -.5 -.2) -- (-3.3 +.4, -.5 +.2) 
      (-3.3 -.4, -.5 +.2) -- (-3.3 +.4, -.5 -.2);
\node [above] at (0, -1) {$+$};
\node [above] at (6.5, -1) {$+$};

\node [above left] at (-10, -2.15) {$f(x)$};
\node [above] at (-6.5, -2) {$-$};
\draw (-3.3 -.4, -1.5 -.2) -- (-3.3 +.4, -1.5 +.2) 
      (-3.3 -.4, -1.5 +.2) -- (-3.3 +.4, -1.5 -.2);
\node [above] at (0, -2) {$+$};
\draw (3.3, -1.5) circle (3pt);
\node [above] at (6.5, -2) {$-$};

\end{tikzpicture}
 
 \item Quindi i valori di~$x$ che rendono vera la disequazione, cioè i valori
  che rendono~$f(x)$ negativo, sono quelli 
  che si trovano a sinistra di~$-2$ oppure che si trovano a destra di~$+3$. 
 \subitem 
  \begin{minipage}{.35\textwidth}
   rappresentazione grafica: 
  \end{minipage}
  \begin{minipage}{.30\textwidth}
   % (c) 2014 Daniele Zambelli - daniele.zambelli@gmail.com

%%%
% Valori esterni all'intervallo -2; 3
%%%%
 
\begin{tikzpicture}[x=1.5mm, y=1.5mm, smooth]

% \clip (-7.5, -5.5) rectangle (10.9, 10.9);

\coordinate (m_i) at (-10, 0);
\coordinate (a) at (-3, 0);
\coordinate (b) at (3, 0);
\coordinate (p_i) at (10, 0);

% (c) 2014 Daniele Zambelli - daniele.zambelli@gmail.com

%%%
% Asse cartesiano x
%%%%

% (c) 2014 Daniele Zambelli - daniele.zambelli@gmail.com

%%%
% Asse cartesiano generico
%%%%

\draw [-{Stealth[length=2mm, open, round]}] (-10, 0) -- (10, 0);

\node [below] at (10, 0)  {$x$};


\begin{scope}[blue,thick]
\draw [-,decorate,decoration=snake] (m_i) -- (a);
\draw[fill=white] (a) circle (2pt) node [above] {$-2$};
\draw [-,decorate,decoration=snake] (b) -- (p_i);
\draw[fill] (b) circle (2pt) node [above] {$3$};
\end{scope}

\end{tikzpicture}

  \end{minipage}
 \subitem rappresentazione con i predicati:~$x < -2 \lor x \ge 3$ 
 \subitem rappresentazione con le 
  parentesi:~$]-\infty; -2[ \quad \cup \quad ]3; \infty[$. 
\end{itemize}
 \end{esempio}
% \end{exrig}

\osservazione Per comodità (o per pigrizia), d'ora in poi riuniremo in un 
unico grafo lo studio dei segni e la rappresentazione grafica della soluzione: 

% (c) 2014 Daniele Zambelli - daniele.zambelli@gmail.com

%%%
% Segni e soluzione di una disequazione fratta
%%%%
 
\begin{tikzpicture}[x=2.5mm, y=5mm, smooth]

\coordinate (m_i) at (-10, -2);
\coordinate (a_top) at (-3.3, 1);
\coordinate (a_bottom) at (-3.3, -2);
\coordinate (b_top) at (3.3, 1);
\coordinate (b_bottom) at (3.3, -2);
\coordinate (p_i) at (10, -2);

% (c) 2014 Daniele Zambelli - daniele.zambelli@gmail.com

%%%
% Grafo per il calcolo del segno con tre assi
%%%%
 
% (c) 2014 Daniele Zambelli - daniele.zambelli@gmail.com

%%%
% Asse cartesiano x
%%%%

\input{lbr/assiepiani/asse10.pgf}
\node [below] at (10, 0)  {$x$};

\begin{scope}[yshift= -.5cm]
  % (c) 2014 Daniele Zambelli - daniele.zambelli@gmail.com

%%%
% Asse cartesiano x
%%%%

\input{lbr/assiepiani/asse10.pgf}
\node [below] at (10, 0)  {$x$};

  \begin{scope}[yshift= -.5cm]
    % (c) 2014 Daniele Zambelli - daniele.zambelli@gmail.com

%%%
% Asse cartesiano x
%%%%

\input{lbr/assiepiani/asse10.pgf}
\node [below] at (10, 0)  {$x$};

  \end{scope}
\end{scope}

\draw [-] [] (a_top) -- (a_bottom);
\draw [-] [] (b_top) -- (b_bottom);

\node [above] at (-3.3, 1) {$-2$};
\node [above] at (3.3, 1) {$+3$};

\node [above left] at (-10, 0) {$-3 x + 4$};
\node [above] at (-6.5, 0) {$+$};
\node [above] at (0, 0) {$+$};
\draw (3.3, .5) circle (3pt);
\node [above] at (6.5, 0) {$-$};

\node [above left] at (-10, -1) {$x + 2$};
\node [above] at (-6.5, -1) {$-$};
\draw (-3.3 -.4, -.5 -.2) -- (-3.3 +.4, -.5 +.2) 
      (-3.3 -.4, -.5 +.2) -- (-3.3 +.4, -.5 -.2);
\node [above] at (0, -1) {$+$};
\node [above] at (6.5, -1) {$+$};

\node [above left] at (-10, -2.15) {$f(x)$};
\node [above] at (-6.5, -2) {$-$};
\draw (-3.3 -.4, -1.5 -.2) -- (-3.3 +.4, -1.5 +.2) 
      (-3.3 -.4, -1.5 +.2) -- (-3.3 +.4, -1.5 -.2);
\node [above] at (0, -2) {$+$};
\draw (3.3, -1.5) circle (3pt);
\node [above] at (6.5, -2) {$-$};

\begin{scope}[blue,thick]
\draw [-,decorate,decoration=snake] (m_i) -- (a_bottom);
\draw[fill=white] (a_bottom) circle (2pt);
\draw [-,decorate,decoration=snake] (b_bottom) -- (p_i);
\draw[fill] (b_bottom) circle (2pt);
\end{scope}

\end{tikzpicture}


Stiamo ben attenti ai simboli che stiamo utilizzando:
\begin{multicols}{2}
% (c) 2014 Daniele Zambelli - daniele.zambelli@gmail.com

%%%
% Cerchietto
%%%%
 
\begin{tikzpicture}[smooth]

\draw (0, 0) circle (3pt);

\end{tikzpicture}
 
sta per: \emph{zero accettabile}
% (c) 2014 Daniele Zambelli - daniele.zambelli@gmail.com

%%%
% Studio dei segni di una frazione
%%%%
 
\begin{tikzpicture}[smooth]

\draw ( -.1,  -.1) -- (+.1, +.1) 
      ( -.1,  +.1) -- (+.1, -.1);

\end{tikzpicture}
 
sta per: \emph{zero non accettabile}
% (c) 2014 Daniele Zambelli - daniele.zambelli@gmail.com

%%%
% Studio dei segni di una frazione
%%%%
 
\begin{tikzpicture}[smooth]

\draw[blue, thick, fill=white] (0, 0) circle (2pt);

\end{tikzpicture}
 
sta per: \emph{estremo escluso}
% (c) 2014 Daniele Zambelli - daniele.zambelli@gmail.com

%%%
% Studio dei segni di una frazione
%%%%
 
\begin{tikzpicture}[smooth]

\draw[fill, blue, thick] (0, 0) circle (2pt);

\end{tikzpicture}
 
sta per: \emph{estremo incluso}
\end{multicols}

In particolare stiamo attenti a non confondere il primo segno 
con il terzo che, pur assomigliandosi, 
hanno significato completamente diverso.

\subsection{Sistema di disequazioni}
\label{sec:dis_sistemi}

In alcune situazioni occorre risolvere contemporaneamente più
disequazioni. Vediamo un problema.

\begin{problema}
Il doppio di un numero reale positivo diminuito di~1 non supera la sua
metà aumentata di~2. Qual è il numero?
\end{problema}

Incognita del problema è il numero reale che indichiamo con~$x$. Di esso
sappiamo che deve essere positivo, quindi~$x>0$ e che deve verificare
la condizione~$2x-1\le \frac{1}{2}x+2$

Le due disequazioni devono verificarsi contemporaneamente quindi 
il problema può essere formalizzato con un \emph{sistema di disequazioni}:

\[\bigg \{%
\begin{array}{l}
 x>0\\
 2x-1\le\frac{1}{2}x+2.
\end{array}\]

Scriviamo in forma normale anche la seconda disequazione e risolviamola:

$d_2:4x-2\le x+4 \quad \Rightarrow \quad 3x -6 \le 0$

Possiamo vedere che la soluzione dell'E.A. è~2 
il grafico della F.A. è una retta crescente
quindi la soluzione è l'insieme di numeri minori o uguali a~2.

Dobbiamo ora determinare~$\IS=\IS_{1}\cap\IS_{2}$ che è l'insieme
dei numeri positivi minori di~2.

% \begin{exrig}
 \begin{esempio}

\emph{Risolvere un sistema di disequazioni} significa trovare
l'insieme dei numeri reali che sono soluzioni comuni
alle due disequazioni, cioè che le verificano entrambe.

La soluzione di un sistema di disequazioni è l'insieme dei valori della 
variabile~$x$ per i quali sono verificate tutte le disequazioni.
La soluzione di un sistema è l'intersezione tra le soluzioni di tutte le 
disequazioni.

Se indichiamo con~$\IS_{1}$ e~$\IS_{2}$
rispettivamente gli insiemi soluzione della prima e della seconda
disequazione, l'insieme soluzione del sistema è dato
dall'intersezione~$\IS=\IS_{1}\cap\IS_{2}$.

Quindi per risolvere un sistema di disequazioni prima si risolvono una alla 
volta tutte le disequazioni che lo compongono, poi si opera l'intersezione 
tra tutte le soluzioni. Iniziamo con un esempio semplice:

Risolviamo il seguente sistema:

$\left\{\begin{array}{l}
  2 (x -5) \le 3 + 4 x \\
  6 x -4 < -3 x -2 \\
\end{array}\right.$

\begin{itemize} [noitemsep]
 \item Per prima cosa scriviamo il sistema in forma normale:
 
$\left\{\begin{array}{l}
  2 x -10 -3 - 4 x \le 0 \\
  6 x -4 +3 x +2 < 0
\end{array}\right.$
$\left\{\begin{array}{ll}
  -2 x -13 \le 0 & \quad \mbox{(1)} \\
   9 x -2 < 0 & \quad \mbox{(2)}
\end{array}\right.$

 \item Soluzione della prima disequazione:\\
 \begin{minipage}{.45\textwidth}
  E.A.:~$-2 x -13=0 \Rightarrow x=-\frac{13}{2}$
 \end{minipage}
 \begin{minipage}{.25\textwidth}
  F.A.:~$y=-2 x -13 \rightarrow $
 \end{minipage}
 \begin{minipage}{.3\textwidth}
  % (c) 2014 Daniele Zambelli - daniele.zambelli@gmail.com

%%%
% Retta decrescente con zero in 13/2
%%%%
 
\begin{tikzpicture}[x=1.5mm, y=1.5mm, smooth]

% (c) 2014 Daniele Zambelli - daniele.zambelli@gmail.com

%%%
% Retta decrescente con segni
%%%%
 
\coordinate (inizio) at (-10, 4);
\coordinate (zero) at (0, 0);
\coordinate (fine) at (10, -4);

% (c) 2014 Daniele Zambelli - daniele.zambelli@gmail.com

%%%
% Asse cartesiano x
%%%%

\input{lbr/assiepiani/asse10.pgf}
\node [below] at (10, 0)  {$x$};


\draw [-] [ultra thick, blue!50!black] (inizio) -- (zero);
\draw [-] [ultra thick, red!50!black] (zero) -- (fine);

\node [xshift=-25, yshift=-3, above] at (zero) {$+$};
\draw[blue, thick, fill=white] (zero) circle (2pt);
\node [xshift=25, yshift=-3, above] at (zero) {$-$};

\node [above] {$- \frac{13}{2}$};

\end{tikzpicture}

 \end{minipage}
 \subitem
  \begin{minipage}{.40\textwidth}
   Soluzione di:~$-2 x -13 \le 0$
  \end{minipage}
  \begin{minipage}{.30\textwidth}
  % (c) 2014 Daniele Zambelli - daniele.zambelli@gmail.com

%%%
% Soluzione di una disequazione [-13/2; oo[
%%%%
 
\begin{tikzpicture}[x=1.5mm, y=1.5mm, smooth]

% \clip (-7.5, -5.5) rectangle (10.9, 10.9);

\coordinate (a) at (0, 0);
\coordinate (b) at (10, 0);

% (c) 2014 Daniele Zambelli - daniele.zambelli@gmail.com

%%%
% Asse cartesiano x
%%%%

% (c) 2014 Daniele Zambelli - daniele.zambelli@gmail.com

%%%
% Asse cartesiano generico
%%%%

\draw [-{Stealth[length=2mm, open, round]}] (-10, 0) -- (10, 0);

\node [below] at (10, 0)  {$x$};


\begin{scope}[blue,thick]
\draw [-,decorate,decoration=snake] (a) -- (b);
\draw[fill] (a) circle (2pt) node [above] {$- \frac{13}{2}$};
\end{scope}

\end{tikzpicture}

  \end{minipage}
 \item Soluzione della seconda disequazione:\\
 \begin{minipage}{.45\textwidth}
  E.A.:~$9 x -2=0 \Rightarrow x=\frac{2}{9}$
 \end{minipage}
 \begin{minipage}{.25\textwidth}
  F.A.:~$y=9 x -2 \rightarrow $
 \end{minipage}
 \begin{minipage}{.3\textwidth}
  % (c) 2014 Daniele Zambelli - daniele.zambelli@gmail.com

%%%
% Segno di un polinomio con coefficiente della x positivo.
%%%%
 
\begin{tikzpicture}[x=1.5mm, y=1.5mm, smooth]

% (c) 2014 Daniele Zambelli - daniele.zambelli@gmail.com

%%%
% Retta crescente con segni
%%%%
 
\coordinate (inizio) at (-10, -4);
\coordinate (zero) at (0, 0);
\coordinate (fine) at (10, 4);

% (c) 2014 Daniele Zambelli - daniele.zambelli@gmail.com

%%%
% Asse cartesiano x
%%%%

\input{lbr/assiepiani/asse10.pgf}
\node [below] at (10, 0)  {$x$};


\draw [-] [ultra thick, red!50!black] (inizio) -- (zero);
\draw [-] [ultra thick, blue!50!black] (zero) -- (fine);

\node [xshift=-25, yshift=-3, above] at (zero) {$-$};
\draw[blue, thick, fill=white] (zero) circle (2pt);
\node [xshift=25, yshift=-3, above] at (zero) {$+$};

\node [above]{$\frac{2}{9}$};

\end{tikzpicture}

 \end{minipage}
 \subitem
  \begin{minipage}{.40\textwidth}
   Soluzione di:~$9 x -2 < 0$
  \end{minipage}
  \begin{minipage}{.30\textwidth}
  % (c) 2014 Daniele Zambelli - daniele.zambelli@gmail.com

%%%
% Intervallo ]oo; 2/9[ 
%%%%
 
\begin{tikzpicture}[x=1.5mm, y=1.5mm, smooth]

% \clip (-7.5, -5.5) rectangle (10.9, 10.9);

\coordinate (a) at (0, 0);
\coordinate (b) at (-10, 0);

% (c) 2014 Daniele Zambelli - daniele.zambelli@gmail.com

%%%
% Asse cartesiano x
%%%%

% (c) 2014 Daniele Zambelli - daniele.zambelli@gmail.com

%%%
% Asse cartesiano generico
%%%%

\draw [-{Stealth[length=2mm, open, round]}] (-10, 0) -- (10, 0);

\node [below] at (10, 0)  {$x$};


\begin{scope}[blue,thick]
\draw [-,decorate,decoration=snake] (a) -- (b);
\draw[fill=white] (a) circle (2pt) node [above] {$\frac{2}{9}$};
\end{scope}

\end{tikzpicture}

  \end{minipage}
 \item A questo punto dobbiamo solo eseguire l'intersezione tra i due 
intervalli 
  che rappresentano le soluzioni delle due disequazioni, per farlo possiamo 
  utilizzare uno schema nel quale riportiamo i due assi con le due soluzioni 
  più un terzo nel quale evidenziamo gli intervalli che sono comuni ai due 
  precedenti: 
  % (c) 2014 Daniele Zambelli - daniele.zambelli@gmail.com

%%%
% Disequazione fratta
%%%%
 
\begin{tikzpicture}[x=2.5mm, y=5mm, smooth]

% Punti grafo:
% (-10, 1)    (-3.3, 1)   (3.3, 1)   (10, 1)
% (-10, 0)    (-3.3, 0)   (3.3, 0)   (10, 0)
% (-10, -1)   (-3.3, -1)  (3.3, -1)  (10, -1)
% (-10, -2)   (-3.3, -2)  (3.3, -2)  (10, -2)

\coordinate (a_top) at (-3.3, 1);
\coordinate (a_bottom) at (-3.3, -2);
\coordinate (b_top) at (3.3, 1);
\coordinate (b_bottom) at (3.3, -2);

% (c) 2014 Daniele Zambelli - daniele.zambelli@gmail.com

%%%
% Grafo per il calcolo del segno con tre assi
%%%%
 
% (c) 2014 Daniele Zambelli - daniele.zambelli@gmail.com

%%%
% Asse cartesiano x
%%%%

\input{lbr/assiepiani/asse10.pgf}
\node [below] at (10, 0)  {$x$};

\begin{scope}[yshift= -.5cm]
  % (c) 2014 Daniele Zambelli - daniele.zambelli@gmail.com

%%%
% Asse cartesiano x
%%%%

\input{lbr/assiepiani/asse10.pgf}
\node [below] at (10, 0)  {$x$};

  \begin{scope}[yshift= -.5cm]
    % (c) 2014 Daniele Zambelli - daniele.zambelli@gmail.com

%%%
% Asse cartesiano x
%%%%

\input{lbr/assiepiani/asse10.pgf}
\node [below] at (10, 0)  {$x$};

  \end{scope}
\end{scope}

\draw [-] [] (a_top) -- (a_bottom);
\draw [-] [] (b_top) -- (b_bottom);

\node [above] at (-3.3, 1) {$- \frac{13}{2}$};
\node [above] at (3.3, 1) {$\frac{2}{9}$};

\node [left] at (-10, 0) {$-2 x -13$};
\begin{scope}[blue,thick]
\draw [-,decorate,decoration=snake] (-3.3, 0) -- (10, 0);
\draw[fill] (-3.3, 0) circle (2pt);
\end{scope}

\node [left] at (-10, -1) {$9 x - 2$};
\begin{scope}[blue,thick]
\draw [-,decorate,decoration=snake] (3.3, -1) -- (-10, -1);
\draw[fill=white] (3.3, -1) circle (2pt);
\end{scope}

\node [left] at (-10, -2) {$f(x)$};

\begin{scope}[blue,thick]
\draw [-,decorate,decoration=snake] (-3.3, -2) -- (3.3, -2);
\draw[fill] (-3.3, -2) circle (2pt);
\draw[fill=white] (3.3, -2) circle (2pt);
\end{scope}

\end{tikzpicture}

 \item rappresentazione con i predicati:~$-\frac{13}{2} \le x < \frac{2}{9}$ 
 \item rappresentazione con le parentesi:~$\left[-\frac{13}{2}; \; \frac{2}{9} 
\right[$. 
\end{itemize}
 \end{esempio}
% \end{exrig}

\osservazione Consideriamo questo schema e quello usato nello studio del 
segno del prodotto, pur essendo formati entrambi da assi orizzontali e 
da linee verticali i due schemi sono completamente diversi: 
nel primo riportiamo dei segni ed eseguiamo il prodotto di segni, 
nel secondo riportiamo degli intervalli e eseguiamo l'intersezione tra 
insiemi. 

\subsection{Soluzione di disequazioni letterali}
\label{sec:dis_tetterali}

Qualunque sia una disequazione letterale di primo grado nella variabile~$x$ 
può sempre essere scritta, utilizzando il primo principio di equivalenza e un 
po' di calcoli, come:

$A x + B > 0$

Alcune osservazioni sulla formula precedente:

\begin{itemize} [noitemsep]
 \item il predicato può essere uno di questi:~$>, \; <, \; \le, \; \ge$.
 \item A e B sono espressioni letterali contenenti cioè dei parametri.
\end{itemize}

Partiamo da un  esempio e cerchiamo di seguire il metodo già usato:

$k \left( x -1 \right )\le k \left ( k - x \right ) + x$

Innanzitutto la scriviamo in forma normale:

$k x - k \le k^2 - k x + x $

$k x - k - k^2 + k x - x \le 0 $

$2 k x - x - k^2 - k \le 0 $

$\left (2 k - 1 \right ) x - k^2 - k \le 0$

A questo punto si può vedere che il metodo utilizzato fin qui non può più 
essere seguito pedissequamente; infatti se non conosciamo il valore di~$k$,
non possiamo dire se il coefficiente della x è negativo, nullo o positivo.

\begin{itemize} [noitemsep]
 \item se~$2 k - 1$ è minore di zero la funzione associata è decrescente;
 \item se~$2 k - 1$ è uguale a zero, l'equazione associata non ha soluzione;
 \item se~$2 k - 1$ è maggiore di zero la funzione associata è crescente.
\end{itemize}

Ma il valore dell'espressione~$2 k - 1$ dipende dal valore del 
parametro~$k$. 

Quindi dobbiamo sospendere la soluzione della disequazione iniziale per 
dedicarci allo studio del segno del coefficiente della~$x$. 

Applicando la solita tecnica otteniamo:

\begin{itemize} [noitemsep]
 \item
  Equazione Associata:~$2 k -1=0 \Rightarrow k=\frac{1}{2}$
 \item 
  \begin{minipage}{.40\textwidth}
  Funzione Associata:~$y=2 k -1 \rightarrow$
  \end{minipage}
  \begin{minipage}{.30\textwidth}
  % (c) 2014 Daniele Zambelli - daniele.zambelli@gmail.com

%%%
% Retta k crescente zero in 1/2
%%%%
 
\begin{tikzpicture}[x=1.5mm, y=1.5mm, smooth]

% (c) 2014 Daniele Zambelli - daniele.zambelli@gmail.com

%%%
% Retta k crescente con segni
%%%%
 
\coordinate (inizio) at (-10, -4);
\coordinate (zero) at (0, 0);
\coordinate (fine) at (10, 4);

% (c) 2014 Daniele Zambelli - daniele.zambelli@gmail.com

%%%
% Asse cartesiano x
%%%%

\input{lbr/assiepiani/asse10.pgf}
\node [below] at (10, 0)  {$k$};


\draw [-] [ultra thick, red!50!black] (inizio) -- (zero);
\draw [-] [ultra thick, blue!50!black] (zero) -- (fine);

\node [xshift=-25, yshift=-3, above] at (zero) {$-$};
\draw[blue, thick, fill=white] (zero) circle (2pt);
\node [xshift=25, yshift=-3, above] at (zero) {$+$};

\node [above] {$\frac{1}{2}$};

\end{tikzpicture}

  \end{minipage}
\end{itemize}

Ora possiamo studiare i 3 casi che si ottengono a seconda che il 
parametro~$k$ renda il coefficiente della~$x$ negativo, uguale a zero 
o positivo:

\begin{enumerate}
 \item Se~$k < \frac{1}{2} \Rightarrow 2 k - 1 < 0$
 \subitem E.A.:~$(2 k - 1) x - k^2 - k = 0 \Rightarrow x=\frac{k^2 - k}{2 k - 
1}$
 \subitem
  \begin{minipage}{.4\textwidth}
   F.A.:~$y=(2 k - 1) x - k^2 - k \rightarrow$
  \end{minipage}
  \begin{minipage}{.30\textwidth}
  % (c) 2014 Daniele Zambelli - daniele.zambelli@gmail.com

%%%
% Retta decrescente con zero in f(k).
%%%%
 
\begin{tikzpicture}[x=1.5mm, y=1.5mm, smooth]

% (c) 2014 Daniele Zambelli - daniele.zambelli@gmail.com

%%%
% Retta decrescente con segni
%%%%
 
\coordinate (inizio) at (-10, 4);
\coordinate (zero) at (0, 0);
\coordinate (fine) at (10, -4);

% (c) 2014 Daniele Zambelli - daniele.zambelli@gmail.com

%%%
% Asse cartesiano x
%%%%

\input{lbr/assiepiani/asse10.pgf}
\node [below] at (10, 0)  {$x$};


\draw [-] [ultra thick, blue!50!black] (inizio) -- (zero);
\draw [-] [ultra thick, red!50!black] (zero) -- (fine);

\node [xshift=-25, yshift=-3, above] at (zero) {$+$};
\draw[blue, thick, fill=white] (zero) circle (2pt);
\node [xshift=25, yshift=-3, above] at (zero) {$-$};

\node [above, yshift=2] {$\frac{k^2 - k}{2 k - 1}$};

\end{tikzpicture}

  \end{minipage}
 \item Se~$k = \frac{1}{2} \Rightarrow 0 x - \frac{3}{4} \le 0$
  \subitem E.A.:~$0 x - \frac{3}{4} = 0 \Rightarrow$ "Impossibile"
  \subitem
  \begin{minipage}{.4\textwidth}
   F.A.:~$y=(2 k - 1) x - k^2 - k \rightarrow$
  \end{minipage}
  \begin{minipage}{.30\textwidth}
  % (c) 2014 Daniele Zambelli - daniele.zambelli@gmail.com

%%%
% Segno di una costante negativa.
%%%%
 
\begin{tikzpicture}[x=1.5mm, y=1.5mm, smooth]

% (c) 2014 Daniele Zambelli - daniele.zambelli@gmail.com

%%%
% Retta crescente
%%%%
 
\coordinate (inizio) at (-10, -3);
\coordinate (zero) at (0, 0);
\coordinate (fine) at (10, -3);

% (c) 2014 Daniele Zambelli - daniele.zambelli@gmail.com

%%%
% Asse cartesiano x
%%%%

\input{lbr/assiepiani/asse10.pgf}
\node [below] at (10, 0)  {$x$};


\draw [-] [ultra thick, red!50!black] (inizio) -- (fine);

\node [xshift=-25, yshift=-3, above] at (zero) {$-$};
\node [xshift=25, yshift=-3, above] at (zero) {$-$};


\end{tikzpicture}

  \end{minipage}
 \item Se~$k > \frac{1}{2} \Rightarrow 2 k - 1 > 0$
  \subitem E.A.:~$(2 k - 1 ) x - k^2 - k = 0 \rightarrow x=\frac{k^2 - k}{2 k - 
1}$
  \subitem
  \begin{minipage}{.4\textwidth}
   F.A.:~$y=(2 k - 1 ) x - k^2 - k \rightarrow$
  \end{minipage}
  \begin{minipage}{.30\textwidth}
  % (c) 2014 Daniele Zambelli - daniele.zambelli@gmail.com

%%%
% Retta crescente con zero in f(k).
%%%%
 
\begin{tikzpicture}[x=1.5mm, y=1.5mm, smooth]

% (c) 2014 Daniele Zambelli - daniele.zambelli@gmail.com

%%%
% Retta crescente con segni
%%%%
 
\coordinate (inizio) at (-10, -4);
\coordinate (zero) at (0, 0);
\coordinate (fine) at (10, 4);

% (c) 2014 Daniele Zambelli - daniele.zambelli@gmail.com

%%%
% Asse cartesiano x
%%%%

\input{lbr/assiepiani/asse10.pgf}
\node [below] at (10, 0)  {$x$};


\draw [-] [ultra thick, red!50!black] (inizio) -- (zero);
\draw [-] [ultra thick, blue!50!black] (zero) -- (fine);

\node [xshift=-25, yshift=-3, above] at (zero) {$-$};
\draw[blue, thick, fill=white] (zero) circle (2pt);
\node [xshift=25, yshift=-3, above] at (zero) {$+$};

\node [above, yshift=2] {$\frac{k^2 - k}{2 k - 1}$};

\end{tikzpicture}

  \end{minipage}
\end{enumerate}

La soluzione della disequazione letterale è:

\begin{itemize} [noitemsep]
 \item Se~$k < \frac{1}{2} \rightarrow x \le \frac{k^2 - k}{2 k - 1}$
 \item Se~$k = \frac{1}{2} \rightarrow \forall x \in \mathbb{R}$
 \item Se~$k > \frac{1}{2} \rightarrow x \ge \frac{k^2 - k}{2 k - 1}$
\end{itemize}

Riassumendo possiamo seguire questo metodo:

\begin{procedura}
 Per risolvere una disequazione letterale:
\begin{enumeratea}
 \item scrivere la disequazione in forma normale;
 \item studiare il segno del coefficiente della~$x$
 \item risolvere le tre disequazioni che si ottengono a seconda il segno 
  precedente sia minore, uguale o maggiore di zero.
\end{enumeratea}
\end{procedura}


%%%%%%%%%%%%%%%%%%%%%%%%%%%%%%%%%%%%%%%%%%%%%%%%%%%%%%%%%%%%%%%%%%%%%%%%%%%%%%%%
%%%%

% TODO di quanto segue selezionare solo ciò che non è già stato scritto.


%%%%%%%%%%%%%%%%%%% Originale %%%%%%%%%%%%%%%%%%%%%%%%%%%%%%%%%%%

% Esempi un po' più complicati:
%  \begin{esempio}
%  $\dfrac{(x+1)^{2}}{4}-\dfrac{2+3x}{2}>\dfrac{(x-1)^{2}}{4}.$
% \end{esempio}
% Il~$\mcm$ è~4 numero positivo, moltiplicando per~4 si ha
% 
% \[4\cdot\left[\frac{(x+1)^{2}}{4}-\frac{2+3x}{2}\right]>
% \frac{4\cdot{(x-1)^{2}}}{4}.\]
% Semplificando:~$(x+1)^{2}-2\cdot (2+3x)>(x-1)^{2}$.
% 
% Eseguiamo i prodotti:~$x^{2}+2x+1-4-6x>x^{2}-2x+1$
% 
% Eliminiamo dai due membri i termini uguali~$x^{2}$ e~1,
% trasportiamo a sinistra i monomi con l'incognita e a
% destra i termini noti; infine sommiamo i monomi simili:
% 
% \[\cancel{{x^{2}}}+2x\cancel{+1}-4-6x>\cancel{{x^{2}}}-2x\cancel{+1} 
% \Rightarrow~2x+2x-6x>+4
% \Rightarrow \ -2x>4.\]
% 
% Il coefficiente dell'incognita è negativo, applicando
% il terzo principio dividiamo ambo i membri per~$-2$ e cambiamo il verso
% della disuguaglianza:
% \[\frac{-2}{-2}x<\frac{4}{-2}\Rightarrow x<-2.\]
% 
% \begin{center}
%  % (c) 2012 Dimitrios Vrettos - d.vrettos@gmail.com
\begin{tikzpicture}[font=\small,x=10mm, y=5mm]

\draw[->] (0,0) -- (8,0) node [below right] () {$r$};
\node[above]  at (4,0) {$-2$};
\begin{scope}[blue,thick]
\draw (0,0) -- (4,0);
\draw[fill=white] (4,0)circle (1.5pt);
\end{scope}

\end{tikzpicture}
% \end{center}
%  $\IS=\{x\in \insR/x<-2\}=(-\infty,-2)$.
% 
% 
% Giunti alla forma~$-2x>4$ potevano trasportare a destra del
% segno di disuguaglianza il monomio con l'incognita e a
% sinistra mettere il termine noto; ovviamente per il primo principio
% spostando questi termini cambiano segno e otteniamo~$-4>2x$. Il coefficiente
% dell'incognita è positivo dunque applichiamo il
% secondo principio dividendo per~2,
% abbiamo~$\frac{-4}{2}>\frac{2}{2}x\Rightarrow -2>x$, che letta da destra a 
% sinistra dice che i
% valori da attribuire ad~$x$ per soddisfare la disequazione assegnata sono
% tutti i numeri reali minori di~$-2$.
% \vspace*{1.05ex}
% % \end{exrig}

% \ovalbox{\risolvii \ref{ese:21.9}, \ref{ese:21.10}, \ref{ese:21.11}, 
% \ref{ese:21.12}, \ref{ese:21.13}, \ref{ese:21.14}, \ref{ese:21.15}}

\subsection{Problemi con le disequazioni}
\label{sec:dis_problemi}

 \begin{problema}[Tariffe telefoniche]
 Sto analizzando due proposte di compagnie telefoniche per poi stipulare
il contratto più conveniente per le mie esigenze. La compagnia
T\textsubscript{1} prevede una spesa fissa di~5 centesimi di scatto
alla risposta da sommare alla spesa di~1 centesimo per ogni minuto di
telefonata. La compagnia T\textsubscript{2} non prevede spesa per lo
scatto alla risposta, ma per ogni minuto di telefonata la spesa è di~2 
centesimi.
Dopo quanti minuti di telefonata la seconda tariffa è
più conveniente della prima?
 \end{problema}

 \begin{soluzione}
 Indichiamo con~$x$ la durata in minuti di una telefonata e con
$t_{1}$ e~$t_{2}$ rispettivamente la spesa con
la prima e la seconda compagnia:

\[t_{1}=(5+1\cdot x)\text{ centesimi};\quad t_{2}=(2\cdot x)\text{ centesimi.}\]

La~$t_2$ sarà più conveniente di~$t_1$ se~$2\cdot x<5+x$.

Il problema è formalizzato con una disequazione
nell'incognita~$x$, di primo grado. Dobbiamo trovare l'$\IS$.

Risolvendo la disequazione si ottiene:
$2\cdot x-x<5\Rightarrow x<5\unit{min}$.

Conclusione: se le mie telefonate durano meno di~5~minuti allora mi
conviene il contratto con T\textsubscript{2}, altrimenti se faccio
telefonate più lunghe di~5~minuti mi conviene T\textsubscript{1}. Le
due tariffe sono uguali se la telefonata dura esattamente~5~minuti.
 \end{soluzione}

 \begin{problema}[L'abbonamento]
 Su un tragitto ferroviario, il biglietto costa~8,25 euro.
L'abbonamento mensile costa~67,30 euro. Qual è il
numero minimo di viaggi che occorre effettuare in un mese perché
l'abbonamento sia più conveniente?
 \end{problema}

 \begin{soluzione}
 Indichiamo con~$x$ il numero di viaggi. Il costo del biglietto di~$x$ viaggi
è~$8,25\cdot x$. L'abbonamento è più
conveniente quando~$8,25\cdot x>67,30$ da cui~$x>\frac{67,30}{8,25}$
e quindi~$x>8,16$. In conclusione se si fanno~8 viaggi in un
mese conviene acquistare i biglietti singoli, da~9 viaggi in poi
conviene l'abbonamento.
 \end{soluzione}

%  \ovalbox{\risolvii \ref{ese:21.16}, \ref{ese:21.17}, \ref{ese:21.18}, 
% \ref{ese:21.19}, \ref{ese:21.20}, \ref{ese:21.21}, \ref{ese:21.22}, 
% \ref{ese:21.23}, \ref{ese:21.24}, \ref{ese:21.25}, \ref{ese:21.26}}

% \vspazio\ovalbox{\ref{ese:21.27}, \ref{ese:21.28}, \ref{ese:21.29}, 
% \ref{ese:21.30}, \ref{ese:21.31}, \ref{ese:21.32}}

% \section{Sistemi di disequazioni}
% \label{sec:21_sistemi}

%  \begin{soluzione}
% Risolviamo separatamente le due disequazioni e determiniamo gli
% insiemi delle soluzioni.
% 
% Questa ricerca può essere facilitata rappresentando graficamente i due
% intervalli in uno stesso schema. Disegniamo l'asse dei
% numeri reali~$r$ e su esso indichiamo i numeri che entrano in gioco, lo~0
% e il~2. Disegniamo una prima linea dove rappresentiamo con una linea
% spessa~$\IS_{1}$, disegniamo una seconda linea dove
% rappresentiamo con una linea più spessa~$\IS_2$.
% 
% Su una terza linea rappresentiamo l'insieme degli
% elementi comuni a~$\IS_{1}$ e~$\IS_{2}$, che
% è appunto l'insieme delle soluzioni del sistema di
% disequazioni.
% \begin{center}
%  % (c) 2012 Dimitrios Vrettos - d.vrettos@gmail.com
\begin{tikzpicture}[font=\small,x=10mm, y=10mm]

\draw[->] (0,0) -- (8,0) node [below right] () {$r$};

\foreach \x in {2,6}
\draw(\x,3pt)--(\x,-3pt);

\node[above]  at (2,0) {$0$};
\node[above]  at (6,0) {2};

\begin{scope}[dotted]
\draw (2,0) -- (2,-1.5);
\draw (6,0) -- (6,-1.5);
\draw (0,-.5) -- (2,-.5);
\draw (6,-.5) -- (8,-.5);
\end{scope}

\pattern[pattern= north east lines, pattern color=red] (2,-2) rectangle (6,-1.5);

\node[below] () at (4,-2) {$\IS$};

\begin{scope}[blue,thick]
\draw (2,-.5) -- (8,-.5);
\draw (0,-1) -- (6,-1);
\draw[fill=blue] (6,-1)circle (1.5pt);
\draw[fill=white] (2,-.5)circle (1.5pt);
\end{scope}

\end{tikzpicture}
% \end{center}
% 
% Non ci rimane che descrivere
% l'intervallo delle soluzioni in forma insiemistica:
% \[\IS=\{x\in \insR/0<x\le~2\}=(0,2].\]
% 
% \end{soluzione}
% 
% \begin{problema}
%  In un triangolo il lato maggiore misura~$13\unit{m}$, gli altri due lati
% differiscono tra di loro di~$2\unit{m}$. Come si deve scegliere il lato minore
% affinché il perimetro non superi i~$100\unit{m}$?
% \end{problema}
% 
% \emph{Dati}:~$\overline{AB}=13\unit{m}$, 
$\overline{BC}-\overline{AC}=2\unit{m}$.
% Riferendoci alla figura, $AC$ è il lato minore; indichiamo con~$x$ la sua
% misura.
% \begin{center}
%  % (c) 2012 Dimitrios Vrettos - d.vrettos@gmail.com
\begin{tikzpicture}[font=\small,x=10mm, y=10mm]

\draw (0,0) -- (2,3)--(6,0)--(0,0);

\node[left] at (0,0) {$A$};
\node[right] at (6,0) {$B$};
\node[above] at (2,3) {$C$};
\end{tikzpicture}
% \end{center}
% 
% \emph{Obiettivo}: determinare~$x$ in modo che~$2p\le~100$.
% 
% \begin{soluzione}
%  $\overline{AC}=x; \overline{BC}=2+x; \overline{AB}=13\text{ con }x>0.$
% 
% L'obiettivo in linguaggio matematico si scrive:~$x+(2+x)+13\le~100$.
% 
% Per la ``disuguaglianza triangolare''
% si deve avere~$13<x+(2+x)$. Il problema è formalizzato dal
% sistema:
% 
% \[\left\{%
% \begin{array}{l}
% x>0\\
% x+(x+2)+13\le~100\\
% 13<x+(x+2)
% \end{array}
% \right.,\]
% Risolvendo ciascuna disequazione si ottiene
% {\longarray\[\left\{%
% \begin{array}{l}
% x>0\\
% x\le\dfrac{85}{2}\\
% \vspace*{1.05ex} x>\dfrac{11}{2}
% \end{array}
% \right..\]}
% 
% A questo punto basta risolvere il sistema di disequazioni.
% 
% % Determiniamo l'insieme soluzione aiutandoci con una
% % rappresentazione grafica.
% % \begin{center}
% %  % (c) 2012 Dimitrios Vrettos - d.vrettos@gmail.com
\begin{tikzpicture}[font=\small,x=10mm, y=10mm]

\draw[->] (-1.5,0) -- (8,0) node [below right] () {$r$};

\foreach \x in {0,1,7}{
\draw(\x,3pt)--(\x,-3pt);
\begin{scope}[dotted]
\draw (\x,0) -- (\x,-2);
\draw (-1.5,-.5) -- (0,-.5);
\draw (-1.5,-1) -- (1,-1);
\draw (7,-1.5) -- (8,-1.5);
\end{scope}}

\node[above]  at (0,0) {$0$};
\node[above]  at (1,0) {$\frac{11}{2}$};
\node[above]  at (7,0) {$\frac{85}{2}$};
\pattern[pattern= north east lines, pattern color=red] (1,-2) rectangle (7,-1.5);

\node[below] () at (4,-2) {$\IS$};

\begin{scope}[blue,thick]
\draw (0,-.5) -- (8,-.5);
\draw (1,-1) -- (8,-1);
\draw (-1.5,-1.5) -- (7,-1.5);

\draw[fill=white] (0,-.5)circle (1.5pt);
\draw[fill=white] (1,-1)circle (1.5pt);
\draw[fill=blue] (7,-1.5)circle (1.5pt);

\end{scope}

\end{tikzpicture}
% % \end{center}
% % Affinché il perimetro non superi~$100\unit{m}$ la misura in metri del
% % lato minore deve essere un numero dell'insieme:
% % \[\IS=\left\{x\in \insR/\frac{11}{2}<x\le\frac{85}{2}\right\}.\]
% \end{soluzione}

% Risolviamo delle disequazioni più articolate nel calcolo algebrico.
% 
% % \begin{exrig}
%  \begin{esempio}
% Risolvere il seguente sistema di disequazioni.
% \longarray{
% \[\left\{%
%  \begin{array}{l}
%   x>\dfrac{2x-11}{8}+\dfrac{19-2x}{4}\\
%   \dfrac{1}{5}(x+1)>\dfrac{x}{3}-\dfrac{15+2x}{9}
%   \end{array}
% \right..\]}
% 
% Risolviamo separatamente le due disequazioni:
% 
% \[d_{1}: 8x>2x-11+38-4x\Rightarrow~10x>27\Rightarrow 
% x>\frac{27}{10}\rightarrow\IS_{1}=\left\{x\in\insR/x>\frac{27}{10}\right\},\]
% \[d_{2}:9x+9>15x-75-10x\Rightarrow~4x>-84\Rightarrow x>-21\rightarrow 
% \IS_{2}=\left\{x\in\insR/x>-21\right\}.\]
% 
% Rappresentiamo graficamente le soluzioni e determiniamo~$\IS=\IS_{1}\cap 
% \IS_{2}$:
% \begin{center}
% % (c) 2012 Dimitrios Vrettos - d.vrettos@gmail.com
\begin{tikzpicture}[font=\small,x=10mm, y=10mm]

\draw[->] (0,0) -- (8,0) node [below right] () {$r$};

\foreach \x in {2,6}{
\draw(\x,3pt)--(\x,-3pt);
\begin{scope}[dotted]
\draw (\x,0) -- (\x,-1.5);
\draw (0,-.5) -- (2,-.5);
\draw (0,-1) -- (6,-1);
\end{scope}}

\node[above]  at (2,0) {$-21$};
\node[above]  at (6,0) {$\frac{27}{10}$};
\pattern[pattern= north east lines, pattern color=red] (6,-1) rectangle (8,-1.5);

\node[below] () at (7,-1.5) {$\IS$};

\begin{scope}[blue,thick]
\draw (2,-.5) -- (8,-.5);
\draw (6,-1) -- (8,-1);

\draw[fill=white] (2,-.5)circle (1.5pt);
\draw[fill=white] (6,-1)circle (1.5pt);

\end{scope}

\end{tikzpicture}
% \end{center}
% \[\IS=\left\{x\in\insR/x>\frac{27}{10}\right\}.\]
%  \end{esempio}
% 
%  \begin{esempio}
%  Risolvere il seguente sistema di disequazioni.
%  \longarray{
%  \[\left\{%
%  \begin{array}{l}
%   2\cdot (x+1)+(-2)^{2}\cdot x>3\cdot(2x-3)\\
%   \dfrac{(x-3)^{2}}{4}-\dfrac{(2x-1)^{2}}{16}<\dfrac{35}{16}
%  \end{array}
% \right..\]}
% 
% Risolviamo separatamente le due disequazioni:
% 
% \[D_{1}:2x+2+4x>6x-9\Rightarrow~0x>-11\rightarrow\IS_{1}=\insR,\]
% \[D_{2}:4x^{2}+36-24x-4x^{2}-1+4x-35<0\Rightarrow -20x<0\Rightarrow 
% x>0\rightarrow\IS_{2}=\left\{x\in \insR/x>0\right\}.\]
% 
% Determiniamo~$\IS=\IS_{1}\cap \IS_{2}$.
% \begin{center}
% % (c) 2012 Dimitrios Vrettos - d.vrettos@gmail.com
\begin{tikzpicture}[font=\small,x=10mm, y=10mm]

\draw[->] (0,0) -- (8,0) node [below right] () {$r$};

\draw(4,3pt)--(4,-3pt);

\begin{scope}[dotted]
\draw (4,0) -- (4,-1.5);
\draw (0,-.5) -- (2,-.5);
\draw (0,-1) -- (4,-1);
\end{scope}

\node[above]  at (4,0) {$0$};
\pattern[pattern= north east lines, pattern color=red] (4,-1) rectangle (8,-1.5);

\node[below] () at (6,-1.5) {$IS$};

\begin{scope}[blue,thick]
\draw (0,-.5) -- (8,-.5);
\draw (4,-1) -- (8,-1);

\draw[fill=white] (4,-1)circle (1.5pt);

\end{scope}

\end{tikzpicture}
% \end{center}
%  \[\IS=\left\{x\in \insR/x>0\right\}.\]
%  \end{esempio}
% 
%  \begin{esempio}
%  Risolvere il seguente sistema di disequazioni.
% \longarray{
% \[\left\{%
%  \begin{array}{l}
%   (x-2)\cdot (x+3)\ge x+(x-1)\cdot (x+1)\\
%   (x-1)^{3}\le x^{2}\cdot(x-3)+2\left(-{\dfrac{1}{2}}x+1\right)
%  \end{array}
% \right..\]}
% 
% Risolviamo separatamente le disequazioni:
% \[D_{1}: 
% x^{2}-2x+3x-6>x+x^{2}-1\Rightarrow~0x\ge~5\rightarrow\IS_{1}=\emptyset.\]
% 
% Poiché la prima equazione non ha soluzioni non avrà soluzioni
% nemmeno il sistema. È superfluo quindi risolvere la
% seconda disequazione. La risolviamo per esercizio.
% 
% \[D_{2}:x^{3}-3x^{2}+3x-1\le x^{3}-3x^{2}-x+2\Rightarrow~4x\le~3\Rightarrow 
% x\le\frac{3}{4}\rightarrow\IS_{2}=\left\{x\in \insR/x\le\frac{3}{4}\right\}.\]
% 
% \[\IS=\IS_{1}\cap \IS_{2}=\emptyset\cap\IS_{2}=\emptyset.\]
%  \end{esempio}
% 
%  \begin{esempio}
%  Risolvere il seguente sistema di disequazioni.
% \longarray{%
% \[\left\{%
%  \begin{array}{l}
%   
% \dfrac{1}{3}\cdot\left(x-\dfrac{1}{2}\right)-\dfrac{1}{2}\cdot\left(x-\dfrac{1}{
% 3}\right)\le\dfrac{1}{6}\\
%   x+1\le\dfrac{2x-1}{3}+\dfrac{1-2x}{4}
%  \end{array}
% \right..\]}
% 
% Risolviamo separatamente le due disequazioni:
% 
% \[D_{1}:\frac{1}{3}x-\frac{1}{2}x\le 
% \frac{1}{6}\Rightarrow~2x-3x\le~1\Rightarrow x\ge -1\rightarrow 
% \IS_{1}=\{x\in\insR/x\ge -1\},\]
% \[D_{2}:12x+12\le~8x-4+3-6x\Rightarrow~10x\le -13\Rightarrow x\le 
% -{\frac{13}{10}}\rightarrow\IS_{2}=\left\{x\in\insR/x\le 
% -{\frac{13}{10}}\right\}.\]
% 
% Rappresentiamo le soluzioni e determiniamo
% $\IS=\IS_{1}\cap \IS_{2}$.
% \begin{center}
% % (c) 2012 Dimitrios Vrettos - d.vrettos@gmail.com
\begin{tikzpicture}[font=\small,x=10mm, y=10mm]

\draw[->] (0,0) -- (8,0) node [below right] () {$r$};

\foreach \x in {2,5}{
\draw(\x,3pt)--(\x,-3pt);
\begin{scope}[dotted]
\draw (\x,0) -- (\x,-1.5);
\draw (0,-.5) -- (5,-.5);
\draw (2,-1) -- (8,-1);
\end{scope}}

\node[above]  at (5,0) {$-1$};
\node[above]  at (2,0) {$-\frac{13}{10}$};

\begin{scope}[blue,thick]
\draw (5,-.5) -- (8,-.5);
\draw (0,-1) -- (2,-1);

\draw[fill=blue] (5,-.5)circle (1.5pt);
\draw[fill=blue] (2,-1)circle (1.5pt);

\end{scope}

\end{tikzpicture}
% \end{center}
% 
% Il grafico mette in evidenza che i due insiemi soluzione non hanno
% elementi in comune, pertanto~$\IS=\emptyset $.
%  \end{esempio}
% % \end{exrig}

% \ovalbox{\ref{ese:21.33}, \ref{ese:21.34}, \ref{ese:21.34}, \ref{ese:21.35}, 
% \ref{ese:21.36}, \ref{ese:21.37}, \ref{ese:21.38}, \ref{ese:21.39}, 
% \ref{ese:21.40}, \ref{ese:21.41}}

% \section{Disequazioni polinomiali di grado superiore al primo}
% \label{sec:21_gradosup}
% 
% \begin{problema}
% \label{pro:22.1}
% Determinare i valori di~$x$ che rendono il polinomio~$p=(3x-7)(2-x)$ positivo.
% \end{problema}
% 
% Il problema chiede di determinare l'insieme delle
% soluzione della disequazione di secondo grado~$(3x-7)(2-x)>0$. La
% disequazione si presenta nella forma di prodotto di due fattori di
% primo grado e proprio la sua forma di prodotto ci faciliterà la
% risposta al quesito.
% \begin{wrapfloat}{figure}{r}{0pt}
%  % (c) 2012 Dimitrios Vrettos - d.vrettos@gmail.com
\begin{tikzpicture}[font=\small,x=10mm, y=10mm]

\matrix (a)[matrix of nodes]{
$\times$& $+$ $-$\\
$+$& $+$ $-$\\
$-$& $-$ $+$\\
};

\begin{scope}[orange]
\draw (a-1-1.north east)--(a-3-1.south east);
\draw (a-1-1.south west)--(a-1-2.south east);
\end{scope}
\end{tikzpicture}
% \end{wrapfloat}
% 
% Sappiamo che nell'insieme dei numeri relativi il segno
% del prodotto di due fattori segue la regola dei segni visualizzata
% dalla tabella a lato: ``il segno di un prodotto è
% positivo se i due fattori sono concordi''. Questo
% fatto si traduce nei due metodi risolutivi del problema proposto.
% 
% \begin{soluzione}
% \textbf{Metodo I}: impostiamo due sistemi di disequazioni, formalizzando
% l'osservazione precedente:
% 
% \[\left\{\begin{array}{l}
% 	 3x-7>0\\
% 	 2-x>0
% 	\end{array}
% 	 \right.\vee
%  \left\{\begin{array}{l}
% 	 3x-7<0\\
% 	 2-x<0
% 	\end{array}
%  \right..\]
% Risolvendo i due sistemi e unendo le loro soluzioni otteniamo
% l'insieme delle soluzioni della disequazione
% originaria:~$\IS=\IS_{1}\cup \IS_{2}$.
% 
%  \[\IS_{1}: \left\{\begin{array}{l}
% \t	  3x-7>0\\
% \t	  2-x>0
% \t	 \end{array}
% 	  \right.
% \Rightarrow\left\{\begin{array}{l}
% \t	 x>\dfrac{7}{3}\\
% \t	 x<2
% \t	 \end{array}
% 	  \right.
% \rightarrow \IS_{1}=\emptyset,\]
% \[\IS_{2}: \left\{\begin{array}{l}
% \t	 3x-7<0\\
% \t	 2-x<0
% \t	 \end{array}\right.
% \Rightarrow\left\{\begin{array}{l}
% \t	 x<\dfrac{7}{3}\\
% \t	 x>2
% \t	 \end{array}\right.
% \rightarrow\IS_{2}=\left\{x\in\insR/2<x<\dfrac{7}{3}\right\}.\]
% 
% Quindi~$\IS=\IS_{1}\cup\IS_{2}=\left\{x\in\insR/2<x<\dfrac{7}{3}\right\}$.
% 
% \textbf{Metodo~II}: Torniamo alla disequazione iniziale~$(3x-7)(2-x)>0$ e
% applichiamo un altro metodo. Osserviamo che quando risolviamo la
% disequazione~$3x-7>0$ determiniamo l'insieme
% dei valori che attribuiti alla variabile rendono il polinomio~$p=3x-7$
% positivo, precisamente sono i valori~$x>\frac{7}{3}$ Rappresentiamo
% l'~$\IS$ con una semiretta in grassetto come in figura:
% \begin{center}
% % (c) 2012 Dimitrios Vrettos - d.vrettos@gmail.com
\begin{tikzpicture}[font=\small,x=10mm, y=10mm]

\draw[->] (0,0) -- (8,0) node [below right] () {$r$};

\draw(4,3pt)--(4,-3pt);

\node[above]  at (4,0) {$\frac{7}{3}$};

\begin{scope}[blue,thick,->]
\draw (4,0) -- (8,0);
\draw[fill=white] (4,0)circle (1.5pt);
\end{scope}

\end{tikzpicture}
% \end{center}
% 
% In realtà, nel grafico sono contenute tutte le informazioni sul segno
% del polinomio:
% 
% \begin{itemize*}
% \item la semiretta in grassetto rappresenta i valori che rendono il polinomio 
% positivo;
% \item il valore~$x = \frac{7}{3}$ è quello che annulla il polinomio;
% \item la semiretta non in grassetto rappresenta i valori che rendono il 
% polinomio negativo.
% \end{itemize*}
% 
% \begin{center}
%  % (c) 2012 Dimitrios Vrettos - d.vrettos@gmail.com
\begin{tikzpicture}[font=\small,x=10mm, y=10mm]

\draw[->] (0,0) -- (8,0) node [below right] () {$r$};
\draw[dotted] (4,0) -- (4,-.5);

\draw(4,3pt)--(4,-3pt);

\node[above]  at (4,0) {$\frac{7}{3}$};
\begin{scope}[below]
\node at (6,0) {$+$};
\node at (2,0) {$-$};
\end{scope}
\begin{scope}[blue,thick,->]
\draw (4,0) -- (8,0);
\draw[fill=white] (4,0)circle (1.5pt);
\end{scope}

\end{tikzpicture}
% \end{center}
% \end{soluzione}
% 
% % \ovalbox{\risolvii \ref{ese:21.42}, \ref{ese:21.43}}
% 
% % \begin{exrig}
%  \begin{esempio}
%  $(3x-7)\cdot (2-x)>0.$
% 
%  La disequazione equivale a determinare i valori che attribuiti alla
% variabile~$x$ rendono positivo il polinomio~$p=(3x-7)\cdot (2-x)$.
% 
% Studiamo separatamente il segno dei due fattori:
% 
% \[F_{1}:3x-7>0\Rightarrow x>\frac{7}{3},\quad
% F_{2}: 2-x>0\Rightarrow x<2.\]
% 
% Per risolvere la disequazione iniziale ci è di particolare aiuto un
% grafico che sintetizzi la situazione.
% \begin{center}
%  % (c) 2013 Claudio Carboncini - claudio.carboncini@gmail.com
\begin{tikzpicture}[font=\small,x=10mm, y=10mm]

\draw[->] (0,0) -- (8,0) node [below right] () {$r$};

\foreach \x in {2,3.5,5,6.5}{
\draw(\x,3pt)--(\x,-3pt);
\begin{scope}[dotted]
\draw (\x,0) -- (\x,-2);
\draw (0,-.5) -- (2,-.5);
\draw (6.5,-.5) -- (8,-.5);
\draw (3.5,-1) -- (5,-1);
\draw (0,-1.5) -- (2,-1.5);
\draw (3.5,-1.5) -- (5,-1.5);
\draw (6.5,-1.5) -- (8,-1.5);
\end{scope}}

\node[above]  at (2,0) {$\frac{3-2 \sqrt{6}}{3}$};
\node[above]  at (3.5,0) {$\frac{3- \sqrt{5}}{2}$};
\node[above]  at (5,0) {$\frac{3+ \sqrt{5}}{2}$};
\node[above]  at (6.5,0) {$\frac{3+2 \sqrt{6}}{3}$};
\pattern[pattern= north east lines, pattern color=red] (2,-2) rectangle (3.5,-1.5);
\pattern[pattern= north east lines, pattern color=red] (5,-2) rectangle (6.5,-1.5);

\node[] () at (-.5,-.5) {$\IS_{1}$};
\node[] () at (-.5,-1) {$\IS_{2}$};
\node[] () at (-.5,-1.75) {$\IS$};

\begin{scope}[blue,thick]
\draw (2,-.5) -- (6.5,-.5);
\draw (0,-1) -- (3.5,-1);
\draw (5,-1) -- (8,-1);
\draw (2,-1.5) -- (3.5,-1.5);
\draw (5,-1.5) -- (6.5,-1.5);

\draw[fill=blue] (2,-.5)circle (1.5pt);
\draw[fill=blue] (6.5,-.5)circle (1.5pt);
\draw[fill=white] (3.5,-1)circle (1.5pt);
\draw[fill=white] (5,-1)circle (1.5pt);
\draw[fill=blue] (2,-1.5)circle (1.5pt);
\draw[fill=blue] (6.5,-1.5)circle (1.5pt);
\draw[fill=white] (3.5,-1.5)circle (1.5pt);
\draw[fill=white] (5,-1.5)circle (1.5pt);

\end{scope}

\end{tikzpicture}

% \end{center}
% 
% Applicando poi la regola dei
% segni otteniamo il segno del polinomio~$p=(3x-7)\cdot (2-x)$.
% 
% Ricordiamo che la disequazione che stiamo risolvendo~$(3x-7)\cdot(2-x)>0$
% è verificata quando il polinomio~$p=(3x-7)\cdot (2-x)$ è
% positivo, cioè nell'intervallo in cui abbiamo
% ottenuto il segno ``$+$''. Possiamo
% concludere~$\IS=\left\{x\in\insR/2<x<\frac{7}{3}\right\}$.
%  \end{esempio}
% 
%  \begin{esempio}
% $(x-3)\cdot (2x-9)\cdot (4-5x)>0.$
% 
% Determiniamo il segno di ciascuno dei suoi tre fattori:
% 
% \[
% F_{1}: x-3>0\Rightarrow x>3;\quad
% F_{2}:2x-9>0\Rightarrow x>\dfrac{9}{2};\quad
% F_{3}:4-5x>0\Rightarrow x<\dfrac{4}{5}.
% \]
% Costruiamo la tabella dei segni:
% \begin{center}
%  % (c) 2013 Claudio Carboncini - claudio.carboncini@gmail.com
\begin{tikzpicture}[font=\small,x=10mm, y=10mm]

\draw[->] (0,0) -- (6,0) node [below right] () {$r$};

\foreach \x in {1.5,3,4.5}{
\draw(\x,3pt)--(\x,-3pt);
\begin{scope}[dotted]
\draw (\x,0) -- (\x,-1.7);
\end{scope}}

\node[left] at (0,-0.15) {Segno di};
\node[left] at (0,-0.5) {$x-1$};
\node[left] at (.5,-1) {$2x^2-7x+3$};
\node[left] at (0,-1.5) {$d_1$};
\node[above]  at (1.5,0) {$\frac{1}{2}$};
\node[above]  at (3,0) {$1$};
\node[above]  at (4.5,0) {$3$};
\node[] at (.75,-0.5) {$-$};
\node[] at (2.25,-0.5) {$-$};
\node[] at (3.75,-0.5) {$+$};
\node[] at (5.25,-0.5) {$+$};
\node[] at (.75,-1) {$+$};
\node[] at (2.25,-1) {$-$};
\node[] at (3.75,-1) {$-$};
\node[] at (5.25,-1) {$+$};
\node[] at (.75,-1.5) {$-$};
\node[] at (2.25,-1.5) {$+$};
\node[] at (3.75,-1.5) {$-$};
\node[] at (5.25,-1.5) {$+$};

\draw[fill=blue] (3,-.5)circle (1.5pt);
\draw[fill=blue] (1.5,-1)circle (1.5pt);
\draw[fill=blue] (4.5,-1)circle (1.5pt);
\draw[fill=blue] (3,-1.5)circle (1.5pt);
\draw[fill=blue] (1.5,-1.5)circle (1.5pt);
\draw[fill=blue] (4.5,-1.5)circle (1.5pt);

\end{tikzpicture}

% \end{center}
% La disequazione è verificata negli intervalli dove è presente il
% segno ``$+$''.
% \[\IS=\left\{x\in\insR/x<\frac{4}{5}\vee~3<x<\frac{9}{2}\right\}.\]
% \end{esempio}
% 
% \begin{esempio}
%  $4x^{3}+4x^{2}\le~1+x.$
% La disequazione è di terzo grado; trasportiamo al primo membro tutti i
% monomi:
% \[4x^{3}+4x^{2}-1-x\le~0.\]
% 
% Possiamo risolverla se riusciamo a scomporre in fattori di primo grado
% il polinomio al primo membro:
% \[4x^{3}+4x^{2}-1-x=4x^{2}(x+1)-(x+1)=(x+1)(4x^2-1) \Rightarrow 
% (x+1)(2x-1)(2x+1)\le~0.\]
% 
% Studiamo ora il segno di ciascun fattore, tenendo conto che sono
% richiesti anche i valori che annullano ogni singolo fattore (legge di
% annullamento del prodotto):
% 
% \[ F_{1}:x+1\ge~0\Rightarrow x\ge -1;\quad F_{2}:2x-1\ge~0\Rightarrow x\ge 
% \dfrac{1}{2},\quad F_{3}:2x+1\ge~0\Rightarrow x\ge -{\dfrac{1}{2}}.\]
% Possiamo ora costruire la tabella dei segni.
% Ricordiamo che la disequazione di partenza~$4x^{3}+4x^{2}\le~1+x$ è
% verificata dove compare il segno~``$-$'':
% 
% \begin{center}
% % (c) 2013 Claudio Carboncini - claudio.carboncini@gmail.com
\begin{tikzpicture}[font=\small,x=10mm, y=10mm]

\draw[->] (0,0) -- (6,0) node [below right] () {$r$};

\foreach \x in {1.5,3,4.5}{
\draw(\x,3pt)--(\x,-3pt);
\begin{scope}[dotted]
\draw (\x,0) -- (\x,-2.2);
\end{scope}}

\node[left] at (0,-0.1) {Segno di};
\node[left] at (.25,-0.5) {$x^2+x+1$};
\node[left] at (0,-1) {$x$};
\node[left] at (0,-1.5) {$x^2-1$};
\node[left] at (0,-2) {$d_2$};
\node[above]  at (1.5,0) {$-1$};
\node[above]  at (3,0) {$0$};
\node[above]  at (4.5,0) {$1$};
\node[] at (.75,-0.5) {$+$};
\node[] at (2.25,-0.5) {$+$};
\node[] at (3.75,-0.5) {$+$};
\node[] at (5.25,-0.5) {$+$};
\node[] at (.75,-1) {$-$};
\node[] at (2.25,-1) {$-$};
\node[] at (3.75,-1) {$+$};
\node[] at (5.25,-1) {$+$};
\node[] at (.75,-1.5) {$+$};
\node[] at (2.25,-1.5) {$-$};
\node[] at (3.75,-1.5) {$-$};
\node[] at (5.25,-1.5) {$+$};
\node[] at (.75,-2) {$-$};
\node[] at (2.25,-2) {$+$};
\node[] at (3.75,-2) {$-$};
\node[] at (5.25,-2) {$+$};

\begin{scope}[blue,thick]
\draw[fill=white] (3,-1)circle (1.5pt);
\draw[fill=white] (1.5,-1.5)circle (1.5pt);
\draw[fill=white] (4.5,-1.5)circle (1.5pt);
\draw[fill=white] (3,-2)circle (1.5pt);
\draw[fill=white] (1.5,-2)circle (1.5pt);
\draw[fill=white] (4.5,-2)circle (1.5pt);
\end{scope}

\end{tikzpicture}

% \end{center}
% \[\IS=\left\{x\in \insR/x\le-1\text{ oppure }-\frac{1}{2}\le 
% x\le\frac{1}{2}\right\}.\]
% \end{esempio}
% % \end{exrig}
% 
% \begin{procedura}
%  Determinare l'$\IS$ Di una disequazione polinomiale di grado
% superiore al primo:
% 
% \begin{enumeratea}
%  \item scrivere la disequazione nella forma~$p\leq0$, $p\geq~0$,
% $p<0$, $p>0$;
% \item scomporre in fattori irriducibili il polinomio;
% \item determinare il segno di ciascun fattore, ponendolo sempre maggiore
% di zero, o maggiore uguale a zero a seconda della richiesta del
% problema;
% \item costruire la tabella dei segni, segnando con un punto ingrossato
% gli zeri del polinomio;
% \item determinare gli intervalli in cui il polinomio assume il segno
% richiesto.
% \end{enumeratea}
% \end{procedura}
% 
% % \ovalbox{\risolvii \ref{ese:21.44}, \ref{ese:21.45}, \ref{ese:21.46}, 
% \ref{ese:21.47}, \ref{ese:21.48}, \ref{ese:21.49}, \ref{ese:21.50}, 
% \ref{ese:21.51},
% % \ref{ese:21.52}, \ref{ese:21.53}}

% \section{Disequazioni frazionarie}
% \label{sec:21_frazionarie}

% Un'espressione contenente operazioni tra frazioni
% algebriche ha come risultato una frazione algebrica. Con la condizione
% di esistenza che il denominatore della frazione sia diverso da zero la
% ricerca del segno di una frazione algebrica viene effettuata con la
% stessa procedura seguita per il prodotto di due o più fattori.
% 
% % \begin{exrig}
%  \begin{esempio}
%  $p=\dfrac{3x-7}{2-x}\ge~0$.
% 
%  Poniamo innanzi tutto la~$\CE: 2-x\neq~0$
%  cioè~$x\neq~2$ e procediamo studiando il segno del
% numeratore e del denominatore. Terremo conto della~$\CE$ ponendo il
% denominatore semplicemente maggiore di zero e non maggiore uguale.
% \[N\ge~0\Rightarrow~3x-7\ge~0\Rightarrow x\ge \frac{7}{3},\]
% \[D>0\Rightarrow2-x>0\Rightarrow \ x<2.\]
% \begin{center}
% % (c) 2012 Dimitrios Vrettos - d.vrettos@gmail.com
\begin{tikzpicture}[font=\small,x=10mm, y=10mm]

\draw[->] (0,0) -- (8,0) node [below right] () {$r$};

\foreach \x in {3,5.5}{
\draw(\x,3pt)--(\x,-3pt);
\begin{scope}[dotted]
\draw (\x,0) -- (\x,-1.5);
\draw (0,-.5) -- (5.5,-.5);
\draw (3,-1) -- (8,-1);
\end{scope}}

\node[above]  at (3,0) {$2$};
\node[above]  at (5.5,0) {$\frac{7}{3}$};

\begin{scope}[blue,thick]
\draw (5.5,-.5) -- (8,-.5);
\draw (3,-1) -- (0,-1);

\draw[fill=blue] (5.5,-.5)circle (1.5pt);
\draw[fill=white] (3,-1)circle (1.5pt);
\end{scope}

\foreach \x in {-1.5}{
\node  at (\x,-.25) {segno di $N$:};
\node  at (\x,-.75) {segno di $D$:};
\node  at (\x,-1.25) {segno di $p$:};
}
\foreach \z in {1.5,4.25}
\node  at (\z,-.25) {$-$};

\foreach \zi in {4.25,6.75}
\node  at (\zi,-.75) {$-$};

\node  at (6.75,-.25) {$+$};
\node  at (1.5,-.75) {$+$};

\begin{scope}[red]
\foreach \y in {-1.25}{
\foreach \ziv in {4.25}
	\node at (\ziv,\y) {$+$};
\foreach \zv in {1.5,6.75}
\node at (\zv,\y) {$-$};
}
\end{scope}
\end{tikzpicture}
% \end{center}
% Analogamente a quanto fatto per il prodotto, dalla tabella dei segni otteniamo
% \[\IS=\left\{x\in \insR/2<x\le \frac{7}{3}\right\}\]
% in cui vediamo già compresa la~$\CE$ che inizialmente avevamo posto.
%  \end{esempio}
% % \end{exrig}
% 
% \begin{procedura}
%  Procedura per determinare~$\IS$ di una disequazione frazionaria:
% 
% \begin{enumeratea}
% \item applicare il primo principio e trasportare tutti i termini al primo 
% membro;
%  \item eseguire i calcoli dell'espressione al primo membro per arrivare a una 
% disequazione nella forma:
%  \subitem $\dfrac{N(x)}{D(x)}>0$ oppure~$\dfrac{N(x)}{D(x)}\ge~0$ 
% oppure~$\dfrac{N(x)}{D(x)}<0$ oppure~$\dfrac{N(x)}{D(x)}\le~0$;
%  \item studiare il segno del numeratore e del denominatore, ponendo~$N(x)>0$ 
% (oppure~$N(x)\geq~0$ a secondo della richiesta) e~$D(x)>0$;
%  \item costruire la tabella dei segni, segnando con un punto in grassetto gli 
% zeri del numeratore;
%  \item determinare gli intervalli in cui il polinomio assume il segno 
% richiesto.
% \end{enumeratea}
% \end{procedura}
% 
% % \begin{exrig}
%  \begin{esempio}
%  
% $\dfrac{x-1}{2x+2}+\dfrac{2x+1}{4x-2}>\dfrac{4x^{2}(2x+1)+1}{8x^{3}+8x^{2}-2x-2}
% .$
% 
% Trasportiamo tutti i termini al primo 
% membro~$\dfrac{x-1}{2x+2}+\dfrac{2x+1}{4x-2}-\dfrac{4x^{2}(2x+1)+1}{8x^{3}+8x^{2
% }-2x-2}>0$.
% 
% Scomponiamo in fattori i denominatori, determiniamo il minimo comune
% multiplo e sommiamo le frazioni per arrivare alla forma~$\frac{N(x)}{D(x)}>0$:
% 
% \begin{align}
% 
% &\frac{x-1}{2(x+1)}+\frac{2x+1}{2(2x-1)}-\frac{4x^{2}(2x+1)+1}{
% 2(x+1)(2x-1)(2x+1)}>0 \notag\\
% \Rightarrow & 
% \frac{(x-1)(2x-1)(2x+1)+(2x+1)(2x+1)(x+1)-4x^{2}(2x+1)+1}{2(x+1)(2x-1)(2x+1)}>0 
% \notag\\
% \Rightarrow & \frac{4x+1}{2(x+1)(2x-1)(2x+1)}>0. \label{eq:21.1}
% \end{align}
% Studiamo separatamente il segno di tutti i fattori che compaiono nella
% frazione, sia quelli al numeratore sia quelli al denominatore e
% costruiamo la tabella dei segni:
%  \[\begin{gathered}
%  N>0\Rightarrow~4x+1>0\Rightarrow x>-{\frac{1}{4}},\\
%  D>0\Rightarrow\left\{\begin{array}{l}
% 	\t\tx+1>0\Rightarrow x>-1 \\
% 	\t\t2x-1>0\Rightarrow x>\frac{1}{2}\\
% 	\t\t2x+1>0\Rightarrow x>-{\frac{1}{2}}
% \t	  \end{array}\right..
% \end{gathered}\]
% \begin{center}
% % (c) 2012 Dimitrios Vrettos - d.vrettos@gmail.com
\begin{tikzpicture}[font=\small,x=10mm, y=10mm]

  \draw[->] (0,0) -- (8,0) node [below right] () {$r$};

  \foreach \x in {1.5,2.75,3.75,5.75}{
    \draw(\x,3pt)--(\x,-3pt);
    
    \begin{scope}[dotted]
      \draw (\x,0) -- (\x,-2.5);
      \draw (0,-.5) -- (3.75,-.5);
      \draw (0,-1) -- (1.5,-1);
      \draw (0,-1.5) -- (5.75,-1.5);
      \draw (0,-2) -- (2.75,-2);
    \end{scope}
  }

  \node[above]  at (1.5,0) {$-1$};
  \node[above] at (2.75,0) {$-\frac{1}{2}$};
  \node[above]  at (3.75,0) {$-\frac{1}{4}$};
  \node[above]  at (5.75,0) {$\frac{1}{2}$};

  \begin{scope}[blue,thick]
    \draw (3.75,-.5) -- (8,-.5);
    \draw (1.5,-1) -- (8,-1);
    \draw (5.75,-1.5) -- (8,-1.5);
    \draw (2.75,-2) -- (8,-2);

    \draw[fill=white] (3.75,-.5)circle (1.5pt);
    \draw[fill=white] (1.5,-1)circle (1.5pt);
    \draw[fill=white] (5.75,-1.5)circle (1.5pt);
    \draw[fill=white] (2.75,-2)circle (1.5pt);
  \end{scope}

  \foreach \x in {-1.5}{
    \node  at (\x,-.25) {segno di $N$:};
    \node(d1)  at (\x,-.75) {segno di $d_1$:};
    \node  at (\x,-1.25) {segno di $d_2$:};
    \node (d3) at (\x,-1.75) {segno di $d_3$:};
    \node  at (\x,-2.25) {segno di $f$:};
    }

  \draw[decorate, decoration={brace, mirror}] let \p1=(d1.north west), \p2=(d3.south west) in(\p1 ) -- (\p2) node[midway, left=2pt] {$D:$};

  \foreach \z in {.75, 2.125,3.25}
    \node  at (\z,-.25) {$-$};

  \foreach \zi in {4.75, 6.875}
    \node  at (\zi,-.25) {$+$};

  \foreach \zii in {2.125,3.25,4.75, 6.875}
    \node  at (\zii,-.75) {$+$};

  \foreach \ziii in {.75,2.125,3.25,4.75}
    \node  at (\ziii,-1.25) {$-$};

  \foreach \ziv in {.75,2.125}
    \node at (\ziv,-1.75) {$-$};

  \foreach \zv in {3.25,4.75, 6.875}
    \node at (\zv,-1.75) {$+$};

  \node  at (.75,-.75) {$-$};
  \node  at (6.875,-1.25) {$+$};

  \begin{scope}[red]
    \foreach \y in {-2.25}{
      \foreach \ziv in {.75,3.25,6.875}
	\node at (\ziv,\y) {$+$};
      \foreach \zv in {2.125,4.75}
	\node at (\zv,\y) {$-$};
    }
  \end{scope}
\end{tikzpicture}
% \end{center}
% Non abbiamo posto le~$\CE$ in quanto già rispettate dalle disequazioni
% del denominatore.
% Prendiamo gli intervalli in cui il segno della frazione è positivo
% come richiesto dalla disequazione~\ref{eq:21.1}:
%  \[\IS=\left\{x\in \insR/x<-1\vee -{\frac{1}{2}}<x<-{\frac{1}{4}}\vee 
% x>\frac{1}{2}\right\}.\]
% \end{esempio}
% 
%  \begin{esempio}
% $\dfrac{x}{2}-\dfrac{2}{3}\cdot 
% {\dfrac{2x-3}{x-1}}+\dfrac{10x-3}{6x-6}\le\dfrac{3}{2}\cdot 
% {\dfrac{x^{2}+2}{3x-2}}+\dfrac{1}{3(x-1)(3x-2)}.$
% 
% Trasportiamo tutti i termini al primo membro:
% 
% \[\frac{x}{2}-\frac{2}{3}\cdot\frac{2x-3}{x-1}+\frac{10x-3}{6x-6}-\frac{3}{2}
% \cdot\frac{x^{2}+2}{3x-2}-\frac{1}{3(x-1)(3x-2)}\le~0.\]
% 
% Eseguiamo le operazioni per semplificare la frazione e ridurla alla
% forma~$\frac{N(x)}{D(x)}\le~0$:
% 
% \begin{align}
%   
% &\frac{x}{2}-\frac{4x-6}{3(x-1)}+\frac{10x-3}{6(x-1)}-\frac{3x^{2}+6}{2(3x-2)}
% -\frac{1}{3(x-1)(3x-2)}\le~0\notag\\
%   \Rightarrow 
% &\frac{3x(x-1)(3x-2)-2(4x-6)(3x-2)+(10x-3)(3x-2)-3(3x^{2}+6)(x-1)-2}{
% 6(x-1)(3x-2)}\le~0\notag\\
%   \Rightarrow &\frac{11x-2}{6(x-1)(3x-2)}\le~0. \label{eq:22.2}
% \end{align}
% 
% Studiamo il segno del numeratore e dei fattori del denominatore:
%  \[\begin{gathered}N\ge~0\Rightarrow~11x-2\ge~0\Rightarrow x\ge\frac{2}{11},\\
% \t	  D>0\Rightarrow\left\{\begin{array}{l}
% 	\t\td_{1}>0\Rightarrow x-1>0\Rightarrow x>1\\
% 	\t\td_{2}>0\Rightarrow~3x-2>0\Rightarrow x>\dfrac{2}{3}
% 	\t	\end{array}\right.. \end{gathered}\]
% \begin{center}
% % (c) 2012 Dimitrios Vrettos - d.vrettos@gmail.com
\begin{tikzpicture}[font=\small,x=10mm, y=10mm]

\draw[->] (0,0) -- (8,0) node [below right] () {$r$};

\foreach \x in {1,3.72,5.56}{
\draw(\x,3pt)--(\x,-3pt);
\begin{scope}[dotted]
\draw (\x,0) -- (\x,-2);
\draw (0,-.5) -- (1,-.5);
\draw (0,-1) -- (5.56,-1);
\draw (0,-1.5) -- (3.72,-1.5);

\end{scope}}


\node[above] at (1,0) {$\frac{2}{11}$};
\node[above]  at (3.72,0) {$\frac{2}{3}$};
\node[above]  at (5.56,0) {$1$};

\begin{scope}[blue,thick]
\draw (1,-.5) -- (8,-.5);
\draw (5.56,-1) -- (8,-1);
\draw (3.72,-1.5) -- (8,-1.5);

\draw[fill=blue] (1,-.5)circle (1.5pt);
\draw[fill=white] (5.56,-1)circle (1.5pt);
\draw[fill=white] (3.72,-1.5)circle (1.5pt);
\end{scope}

\foreach \x in {-1.5}{
\node  at (\x,-.25) {segno di $N$:};
\node(d1)  at (\x,-.75) {segno di $d_1$:};
\node (d2) at (\x,-1.25) {segno di $d_2$:};
\node (d3) at (\x,-1.75) {segno di $f$:};
}

 \draw[decorate, decoration={brace, mirror}]  let \p1=(d1.north west), \p2=(d2.south west) in(\p1 ) -- (\p2) node[midway, left=2pt] {$D:$};

\foreach \z in {2.36,4.64,6.78}
\node  at (\z,-.25) {$+$};

 \foreach \zi in {.5,2.36,4.64}
 \node  at (\zi,-.75) {$-$};

\foreach \zii in {.5,2.36}
 \node  at (\zii,-1.25) {$-$};

 \foreach \ziii in {4.64,6.78}
\node  at (\ziii,-1.25) {$+$};

\node  at (.5,-.25) {$-$};
\node  at (6.78,-.75) {$+$};

\begin{scope}[red]
\foreach \y in {-1.75}{
\foreach \ziv in {.5,4.64}
	\node at (\ziv,\y) {$-$};
\foreach \zv in {2.36,6.78}
\node at (\zv,\y) {$+$};
}
\end{scope}
\end{tikzpicture}
% \end{center}
% Non abbiamo posto le~$\CE$ in quanto già rispettate dalle disequazioni
% del denominatore. Prendiamo gli intervalli in cui il segno della frazione è 
% positivo o
% nullo come dalla disequazione~\ref{eq:22.2}:
% \[\IS=\left\{x\in \insR/x\le\frac{2}{11}\vee \frac{2}{3}<x<1\right\}.\]
%  \end{esempio}
% % \end{exrig}
% % \ovalbox{\risolvii \ref{ese:21.54}, \ref{ese:21.55}, \ref{ese:21.56}, 
% \ref{ese:21.57}, \ref{ese:21.58}, \ref{ese:21.59}, \ref{ese:21.60}, 
% \ref{ese:21.61}, \ref{ese:21.62}, \ref{ese:21.63}}
% 
% % \vspazio\ovalbox{\ref{ese:21.64}, \ref{ese:21.65}, \ref{ese:21.66}, 
% \ref{ese:21.67}, \ref{ese:21.68}, \ref{ese:21.69}}
