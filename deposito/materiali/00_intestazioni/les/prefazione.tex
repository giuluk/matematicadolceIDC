\pagestyle{matc3page}
\chapter*{Prefazione}
\addcontentsline{toc}{chapter}{Prefazione}
\markboth{Prefazione}{Prefazione}

\begin{flushleft}
\emph{Ciao Daniele,
ho appena inoltrato il tuo lavoro al mio professore, \\
lui apprezza molto il progetto \emph{Matematica $C^3$} \\
e penso che la tua versione gli possa far comodo \\
soprattutto per i primi anni del nostro serale. \\
Già l'anno scorso ha tentato 
l'adozione ufficiale del $C^3$ normale, \\
ma, come precario, è riuscito a 
strappare solo una promessa, \\
promessa che verrà mantenuta solo se tra un paio di settimane \\
(quando inizierà per me e per lui la scuola) lo rivedrò in cattedra. \\
}
\emph{In ogni caso, che ci sia lui o no, proporrò lo stesso al coordinatore il 
progetto $C^3$, \\
``Software Libero, Conoscenza Libera, Scuola Libera'', giusto?\\
}

\emph{
Buon lavoro,
}

\emph{
Alice
}
\end{flushleft}

Giusto, Alice.\\
La cosa importante è che il testo non sia considerato un oggetto scritto da
altri, da un gruppo di professori più o meno strambi, ma sia una traccia.
Una traccia lasciata sul terreno di un territorio sconosciuto, 
a volte inospitale a volte stupefacente.\\
Una traccia come quella scritta su una mappa del tesoro: un po'
bruciacchiata consumata e piena di incrostazioni.
A volte incomprensibile, con degli errori che portano fuori pista, 
a volte scritta male, con alcune parti mancanti oppure con alcune parti 
inutili che confondono.\\
Non seguire acriticamente la mappa, non fidarti del testo, leggilo con
la penna in mano, correggi, cambia, cancella e aggiungi, parlane in
classe. \\
Contribuisci alla sua evoluzione.\\
Grazie, ciao.


\paragraph{\serie}
Diversi anni fa, Antonio Bernardo ha avuto il coraggio di coordinare un 
gruppo di insegnanti che mettendo insieme le proprie competenze hanno creato 
un testo di matematica per il biennio dei licei scientifici: \emph{\serie}.
Con grande generosità e lungimiranza, il gruppo ha scelto di rilasciare il
lavoro con una licenza \textit{Creative Commons} libera. 
Questa licenza permette 
a chiunque di riprodurre l'opera e divulgarla liberamente, ma permette anche 
di creare altre opere derivate da \emph{\serie}.

\paragraph{Specificità di questa versione} Questa versione modifica 
\emph{\serie} in modo da adattarlo ai programmi delle scuole diverse dal liceo 
scientifico. Nell'organizzazione del testo si è tenuto conto delle 
indicazioni ministeriali per la matematica dei licei.

Viene dato più spazio alla geometria nel piano cartesiano proponendo in 
prima: i punti, i segmenti, le figure; in seconda: le rette.
Le trasformazioni geometriche sono proposte sotto forma di schede 
che guidano l'attività di laboratorio di matematica.
Nei numeri naturali viene proposto l'uso di grafi ad albero nella soluzione 
delle espressioni e nella scomposizione in fattori dei numeri.
Nelle disequazioni, il centro dell'attenzione è posto nello studio
del segno di un'espressione.

Per quanto riguarda il tema dell'informatica, in prima viene presentato il  
foglio di calcolo e la geometria della tartaruga mentre 
in seconda, la geometria interattiva con l'uso di un linguaggio di 
programmazione e di una apposita libreria grafica. 

\paragraph{Adozione} Questo manuale non vorrebbe essere adottato nel 
senso di essere \emph{scelto} dal collegio docenti; vorrebbe essere 
\emph{adottato} nel senso di essere preso in carico, da insegnanti, alunni, 
famiglie, come un proprio progetto, bisognoso di cure e attenzioni.
Ha senso adottarlo se siamo disposti a contribuire alla sua crescita. 
Si può contribuire in diversi modi:
 usando il testo o anche solo qualche capitolo, magari per supportare 
 attività di recupero o per trattare temi non presenti nel libro di 
 testo in adozione;
 segnalando errori, parti scritte male o esercizi non adeguati;
 proponendo cambiamenti alla struttura;
 scrivendo o riscrivendo parti del testo;
 creando esercizi;
 realizzando illustrazioni.

\paragraph{Obiettivi} Il progetto \emph{\serie} ha per obiettivo la 
realizzazione di un manuale di matematica, per tutto il percorso scolastico 
e per ogni tipo di scuola, scritto in forma collaborativa e con licenza 
\textit{Creative Commons}. 
Seguendo l'esempio di questa versione, altri insegnanti, studenti, 
appassionati di matematica, potrebbero proporre delle modifiche per adattare 
il testo alle esigenze di altri percorsi scolastici.

\paragraph{Supporti}
\serie\ è scaricabile dal sito \url{www.matematicamente.it}. 
Mentre il cantiere in cui si lavora a questa versione si trova in:
\url{bitbucket.org/zambu/matematicadolce}. 
È disponile in formato elettronico pdf direttamente visualizzabile o 
stampabile. 
Sullo stesso sito sono disponibili i sorgenti in {\LaTeX}, 
che ne permettono la modifica. 
I diversi volumi che compongono l'opera possono essere stampati, 
fotocopiati in proprio o stampati in tipografia per le sole le parti che 
occorrono. 
Oppure può essere usato in formato elettronico su pc, netbook, tablet, 
smartphone.
Può essere proiettato direttamente sulla lavagna interattiva 
interagendo con il testo, svolgendo direttamente esempi ed esercizi, 
personalizzando con gli alunni definizioni ed enunciati; 
ricorrendo eventualmente a contenuti multimediali esterni presenti 
sui siti internet, confrontando definizioni e teoremi su Wikipedia, 
cercando sull'enciclopedia libera notizie storiche sugli autori, 
ricorrendo eventualmente a contenuti multimediali esterni presenti sui siti 
internet (sul sito \url{www.matematicamente.it} sono disponibili 
gratuitamente test interattivi e alcune videolezioni). 

\begin{flushright}
Daniele Zambelli
\end{flushright}

\newpage

\section{Prefazione alla seconda edizione}

Un anno di lavoro ha messo in luce alcuni errori che sono stati corretti,
la nuova versione è scaricabile da:

\url{bitbucket.org/zambu/mc3_a1_dolce_2ed}

e 

\url{bitbucket.org/zambu/mc3_a2_dolce_2ed}. 


Ma, soprattutto, in questo anno è sorta una interessante opportunità: 
è stato finanziato un progetto per tradurre il testo in braille. 
Il lavoro sta procedendo e alcuni capitoli sono già stati tradotti. 
Quanto fatto lo si può trovare in:

\url{oer.veia.it}

Buon divertimento con la matematica!

\begin{flushright}
Daniele Zambelli
\end{flushright}

\section{Prefazione all'edizione 2016}

Cambia il modo di indicare le edizioni. 

Ma soprattutto è cambiata l'organizzazione del materiale: ora tutto il 
progetto è contenuto in un unico repository.

Matematica Dolce, oltre ad essere un libro \emph{libero} è anche 
\emph{polimorfo}: ora è molto semplice creare nuovi libri partendo dal 
materiale presente nel repository. Già da quest'anno, oltre alla versione 
orientata ai licei non scientifici, sta prendendo vita una versione per gli 
istituti professionali.
Il tutto è ospitato in:

\url{bitbucket.org/zambu/matematicadolce}

Quest'anno altri colleghi si sono uniti al progetto e un alunno ha fornito le 
immagini per le copertine.

Per quanto riguarda i contenuti, riporto i principali cambiamenti:
\begin{itemize} [nosep]
 \item la geometria è stata inserita nel testo di matematica;
 \item nel terzo volume è stato inserito un capitolo che introduce ai numeri 
Iperreali;
 \item è stata riscritta la parte di linguaggio di programmazione per la 
geometria interattiva;
 \item è stato aggiunto il quarto volume.
\end{itemize}

Abbiamo svolto un gran lavoro, ora è il momento di usarlo.

Buon divertimento con la matematica!

\begin{flushright}
Daniele Zambelli
\end{flushright}

\section{Prefazione all'edizione 2017}

Raggiunto il traguardo dei cinque volumi: l'opera è completa!

Comunque, chi ha voglia di partecipare alla realizzazione di Matematica Dolce 
può stare tranquillo: 
c'è ancora molto lavoro da fare.

Buon divertimento con la matematica!

\begin{flushright}
Daniele Zambelli
\end{flushright}

\section{Prefazione all'edizione 2018}

Sbozzata l'opera, c'è molto lavoro di raspa per farla diventare uno 
strumento più adatto alle nostre esigenze.

Abbiamo cercato di asciugare un po' il primo volume, abbiamo ridistribuito 
il materiale tra la terza e la quarta e aggiunto, in quinta le variabili 
aleatorie e un nuovo modo di proporre le funzioni,
oltre ad apportare tutte le correzioni di errori segnalati e buona parte 
delle richieste di miglioramenti.

Abbiamo ``scoperto'' che non c'è modo di sapere dove il testo è stato 
adottato. Sarebbe carino se chi lo ha adottato ce lo facesse sapere e 
contribuisse con segnalazioni di errori o proposte di miglioramento.

Il libro è vivo e libero solo se chi lo usa partecipa alla sua evoluzione.
E questo progetto ha senso solo se evolve.

Buon divertimento con la matematica!

\begin{flushright}
Daniele Zambelli
\end{flushright}

% \cleardoublepage
