% (c) 2015 Daniele Zambelli daniele.zambelli@gmail.com

\subsubsection*{\numnameref{sec:insnum_iperreali}}

Di seguito sono riportate alcune domande, scrivi sul quaderno una risposta e 
poi confrontala con quella riportata sotto.

\paragraph{Domande}
\footnote{Queste domande e le rispettive risposte sono state messe a 
disposizione dal prof. Giorgio Goldoni}

\begin{multicols}{2}

\begin{esercizio}\label{ese:iper_001}
Enunciare l'assioma di Eudosso-Archimede per i 
segmenti e discutere in quale
senso esso esclude l'esistenza di segmenti infinitesimi e infiniti.
\end{esercizio}

\begin{esercizio}\label{ese:iper_002}
Cosa intendiamo per numeri standard e per segmenti standard?
\end{esercizio}

\begin{esercizio}\label{ese:iper_003}
Che cos'è un segmento infinitesimo?
\end{esercizio}

\begin{esercizio}\label{ese:iper_004}
Che cos'è un segmento infinito?
\end{esercizio}

\begin{esercizio}\label{ese:iper_005}
Che cos'è un segmento finito?
\end{esercizio}

\begin{esercizio}\label{ese:iper_006}
Che cos'è un segmento non infinitesimo?
\end{esercizio}

\begin{esercizio}\label{ese:iper_007}
Che cos'è un segmento finito non infinitesimo?
\end{esercizio}

\begin{esercizio}\label{ese:iper_008}
Che cos'è un numero infinitesimo?
\end{esercizio}

\begin{esercizio}\label{ese:iper_009}
Che cos'è un numero infinito?
\end{esercizio}

\begin{esercizio}\label{ese:iper_010}
Che cos'è un numero finito?
\end{esercizio}

\begin{esercizio}\label{ese:iper_011}
Che cos'è un numero non infinitesimo?
\end{esercizio}

\begin{esercizio}\label{ese:iper_012}
Che cos'è un numero finito non infinitesimo?
\end{esercizio}

\begin{esercizio}\label{ese:iper_013}
Cosa sono i numeri iperreali?
\end{esercizio}

\begin{esercizio}\label{ese:iper_014}
Che cos'è la retta iperreale?
\end{esercizio}

\begin{esercizio}\label{ese:iper_015}
Come vengono classificati in tipi i numeri iperreali?
\end{esercizio}

\begin{esercizio}\label{ese:iper_016}
Come si comportano i tipi di numeri iperreali con le operazioni aritmetiche?
\begin{description} [nosep]
 \item [Addizione/sottrazione:]
 \item [Moltiplicazione:]
 \item [Reciproco:]
 \item [Divisione:]
\end{description}
\end{esercizio}

\begin{esercizio}\label{ese:iper_017}
Che cosa si intendono per forme indeterminate?
\end{esercizio}

\begin{esercizio}\label{ese:iper_018}
Quando due numeri si dicono infinitamente vicini(\(\approx\))?
\end{esercizio}

\begin{esercizio}\label{ese:iper_019}
Di quali proprietà gode la relazione \(x \approx y\)?
\end{esercizio}

% \begin{esercizio}\label{ese:iper_020}
% Che cos'è una monade e che cosa si intende per monade principale?
% \end{esercizio}

\begin{esercizio}\label{ese:iper_021}
Quando due numeri si dicono a distanza finita (\(\approx\))?
\end{esercizio}

\begin{esercizio}\label{ese:iper_022}
Di quali proprietà gode la relazione \(x \approx y\)?
\end{esercizio}

% \begin{esercizio}\label{ese:iper_023}
% Che cos'è una galassia e che cosa si intende per galassia principale?
% \end{esercizio}

\begin{esercizio}\label{ese:iper_024}
Come si confrontano due infinitesimi non nulli?
\end{esercizio}

\begin{esercizio}\label{ese:iper_025}
Come si confrontano due infiniti?
\end{esercizio}

\begin{esercizio}\label{ese:iper_026}
Che cos'è la parte standard di un numero finito?
\end{esercizio}

\begin{esercizio}\label{ese:iper_027}
Elencare le proprietà salienti della parte standard.
\end{esercizio}

\begin{esercizio}\label{ese:iper_028}
Quando due numeri non nulli si dicono indistinguibili (\(\sim\))?
\end{esercizio}

\begin{esercizio}\label{ese:iper_029}
Di quali proprietà gode la relazione \(x \sim y\)?
\end{esercizio}

\begin{esercizio}\label{ese:iper_030}
Quando sostituiamo il simbolo ∼ con quello di uguaglianza?
\end{esercizio}

\begin{esercizio}\label{ese:iper_031}
Che cosa si intende per microscopio standard?
\end{esercizio}

\begin{esercizio}\label{ese:iper_032}
Che cosa si intende per telescopio standard?
\end{esercizio}

\begin{esercizio}\label{ese:iper_033}
Che cosa si intende per zoom standard?
\end{esercizio}

\begin{esercizio}\label{ese:iper_034}
Che cosa si intende per microscopio non-standard?
\end{esercizio}

\begin{esercizio}\label{ese:iper_035}
Che cosa si intende per telescopio non-standard?
\end{esercizio}

\begin{esercizio}\label{ese:iper_036}
Che cosa si intende per zoom non-standard?
\end{esercizio}

\begin{esercizio}\label{ese:iper_037}
Cosa intendiamo per scala naturale di ingrandimento?
\end{esercizio}

\begin{esercizio}\label{ese:iper_038}
Come possiamo visualizzare un numero infinitesimo non nullo sulla retta 
iperreale?
\end{esercizio}

\begin{esercizio}\label{ese:iper_039}
Come possiamo visualizzare un numero infinito sulla retta iperreale?
\end{esercizio}

\begin{esercizio}\label{ese:iper_040}
Come possiamo visualizzare un numero finito non infinitesimo sulla retta 
iperreali?
\end{esercizio}

\begin{esercizio}\label{ese:iper_041}
Come possiamo visualizzare il fatto che \(\epsilon\) è un infinitesimo di 
ordine superiore a \(\delta\)?
\end{esercizio}

\begin{esercizio}\label{ese:iper_042}
Come possiamo visualizzare il fatto che \(\epsilon\) e \(\delta\) sono due 
infinitesimo dello stesso ordine?
\end{esercizio}

\begin{esercizio}\label{ese:iper_043}
Come possiamo visualizzare il fatto che M è un infinito di ordine superiore 
a N?
\end{esercizio}

\begin{esercizio}\label{ese:iper_044}
Come possiamo visualizzare il fatto che M e N sono due infiniti dello 
stesso ordine?
\end{esercizio}

\end{multicols}

\paragraph{Risposte}
~

\ref{ese:iper_001} 
\emph{Enunciare l'assioma di Eudosso-Archimede per i segmenti e 
discutere in quale senso esso esclude l'esistenza di segmenti infinitesimi e 
infiniti.}

Assioma di Eudosso-Archimede: Dati due segmenti diversi, esiste sempre un 
multiplo del minore che supera il maggiore o, equivalentemente, esiste sempre 
un sottomultiplo del maggiore che è più piccolo del minore.
L'assioma nega l'esistenza di segmenti infiniti poiché afferma che, fissato 
un 
segmento arbitrario come unità di misura, ogni segmento, per quanto grande, 
risulta superato da un opportuno multiplo finito dell'unità di misura. 
Equivalentemente, esso nega l'esistenza di segmenti infinitesimi in quanto 
afferma che ogni segmento, per quanto piccolo, risulta maggiore di un 
opportuno 
sottomultiplo finito dell'unità di misura.

\ref{ese:iper_002} 
\noindent\emph{Cosa intendiamo per numeri standard e per segmenti standard?}

I numeri standard sono i numeri reali \(\R\) e i segmenti standard sono i 
segmenti la cui misura può essere espressa mediante un numero reale positivo.

\ref{ese:iper_003} 
\emph{Che cos'è un segmento infinitesimo?}

Un segmento infinitesimo è un segmento minore di ogni segmento standard. 
Nessun segmento standard è quindi infinitesimo.

\ref{ese:iper_004} 
\emph{Che cos'è un segmento infinito?}

Un segmento infinito è un segmento maggiore di ogni segmento standard. 
Nessun segmento standard è quindi infinito.

\ref{ese:iper_005} 
\emph{Che cos'è un segmento finito?}

Un segmento finito è un segmento non infinito e quindi un segmento minore di 
almeno un segmento standard. 
Tutti i segmenti standard sono quindi segmenti finiti.

\ref{ese:iper_006} 
\emph{Che cos'è un segmento non infinitesimo?}

Un segmento non infinitesimo è un segmento maggiore di almeno un segmento 
standard. Tutti i segmenti standard sono quindi non infinitesimi.

\ref{ese:iper_007} 
\emph{Che cos'è un segmento finito non infinitesimo?}

Un segmento finito non infinitesimo è un segmento compreso tra due segmenti 
standard. Tutti i segmenti standard sono quindi finiti non infinitesimi.

\ref{ese:iper_008} 
\emph{Che cos'è un numero infinitesimo?}

Un numero infinitesimo è un numero in valore assoluto minore di ogni numero 
standard positivo. L'unico numero standard infinitesimo è lo zero.

\ref{ese:iper_009} 
\emph{Che cos'è un numero infinito?}

Un numero infinito è un numero in valore assoluto maggiore di ogni numero 
standard. Nessun numero standard è quindi infinito.

\ref{ese:iper_010} 
\emph{Che cos'è un numero finito?}

Un numero finito è un numero non infinito e quindi un numero in valore 
assoluto 
minore di almeno un numero standard. 
Tutti i numeri standard sono quindi numeri finiti.

\ref{ese:iper_011} 
\emph{Che cos'è un numero non infinitesimo?}

Un numero non infinitesimo è un numero in valore assoluto maggiore di almeno 
un 
numero standard positivo. Tutti i numeri standard tranne lo zero sono quindi 
non infinitesimi.

\ref{ese:iper_012} 
\emph{Che cos'è un numero finito non infinitesimo?}

Un numero finito non infinitesimo è un numero in valore assoluto compreso tra 
due numeri standard positivi. Tutti i numeri standard tranne lo zero sono 
quindi finiti non infinitesimi.

\ref{ese:iper_013} 
\emph{Cosa sono i numeri iperreali?}

Negando l'Assioma di Eudosso/Archimede, accettiamo l'esistenza di segmenti 
maggiori di ogni multiplo dell'unità di misura e minori di ogni suo 
sottomultiplo e accettiamo quindi l'esistenza di segmenti infiniti e 
infinitesimi. Analogamente, accettiamo l'esistenza di numeri infiniti e 
infinitesimi. I numeri che si ottengono combinando i numeri standard con i 
numeri infiniti e infinitesimi mediante le operazioni aritmetiche sono 
chiamati 
numeri iperreali e il loro insieme si indica con \(\IR\).

\ref{ese:iper_014} 
\emph{Che cos'è la retta iperreale?}

La retta iperreale è una retta i cui punti sono in corrispondenza biunivoca 
con 
i numeri iperreali.

\ref{ese:iper_015} 
\emph{Come vengono classificati in tipi i numeri iperreali?}

I numeri iperreali si dividono in \emph{finiti} (f) e \emph{infiniti} (I). 
I finiti a loro volta si dividono in \emph{finiti non infinitesimi} (fni) e 
in 
\emph{infinitesimi} (i) e 
questi ultimi in \emph{infinitesimi non nulli} (inn) e lo \emph{zero}. 
Si distinguono quindi quattro tipi di iperreali: 
\begin{center}
\textbullet ~ infiniti, \qquad 
\textbullet ~ finiti non infinitesimi, \qquad 
\textbullet ~ infinitesimi non nulli, \qquad 
\textbullet ~ zero.
\end{center}

\ref{ese:iper_016} 
\emph{Come si comportano i tipi di numeri iperreali con le operazioni 
aritmetiche?}

\noindent\begin{minipage}{.30\textwidth}
 Addizione/sottrazione:\\
\(inn \mp inn = i\)\\
\(inn \mp fni = fni\)\\
\(inn \mp I = I\)\\
\(fni \mp fni = f\)\\
\(fni \mp I = I\)
\vspace{24pt}
\end{minipage}
\noindent\begin{minipage}{.25\textwidth}
 Moltiplicazione:\\
\(inn \times inn = inn\)\\
\(inn \times fni = inn\)\\
\(fni \times fni = fni\)\\
\(fni \times I = I\)\\
\(I \times I = I\)
\vspace{24pt}
\end{minipage}
\noindent\begin{minipage}{.15\textwidth}
 Reciproco:\\
\(\dfrac{1}{inn} = I\)\\
\(\dfrac{1}{fni} = fni\)\\
\(\dfrac{1}{I} = inn\)\\
\vspace{12pt}
\end{minipage}
\noindent\begin{minipage}{.25\textwidth}
 Divisione:\\
\(inn : fni = inn\)\\
\(inn : I = inn\)\\
\(fni : inn = I\)\\
\(fni : fni = fni\)\\
\(fni : I = inn\)\\
\(I : inn = I\)\\
\(I : fni = I\)
\end{minipage}

\ref{ese:iper_017} 
\emph{Che cosa si intendono per forme indeterminate?}

Si chiamano forme indeterminate le operazioni per le quali la sola conoscenza 
dei tipi degli operandi non consente di determinare il tipo del risultato. Le 
forme indeterminate relative alle operazioni aritmetiche sono:
\(I \mp I; \quad inn \times I; \quad inn : inn; \quad I : I\).

\ref{ese:iper_018} 
\emph{Quando due numeri si dicono infinitamente vicini?}

Due numeri si dicono infinitamente vicini se la loro differenza è un 
infinitesimo. Indichiamo il fatto che x è infinitamente vicino a y con 
\(x \approx y\). 
In particolare un numero x è infinitesimo se e solo se \(x \approx 0\).

\ref{ese:iper_019} 
\emph{Di quali proprietà gode la relazione \(x \approx x\)?}

La relazione \(x \approx x\) è riflessiva, simmetrica e transitiva ed è 
dunque 
una relazione di equivalenza. In simboli:\\
\(x \approx x; \quad 
  x \approx y \sRarrow y \approx x; \quad
  x \approx y \sand y \approx z \sRarrow x \approx z\)
  
Inoltre, se due numeri sono infinitamente vicini allora sono dello stesso 
tipo.

% \ref{ese:iper_020} 
% \emph{Che cos'è una monade e che cosa si intende per monade principale?}
% 
% La monade di un numero è l'insieme di tutti i numeri infinitamente vicini ad 
% esso ed è quindi una classe di equivalenza della relazione x ≈ y. Ne segue 
% che 
% due monadi sono sempre disgiunte o coincidenti e che le monadi formano una 
% partizione dei numeri iperreali. In particolare la monade di un numero 
% standard 
% x consiste, oltre che del numero stesso, di tutti i numeri non standard ad 
% esso 
% infinitamente vicini, cioè dei numeri del tipo \(x + \epsilon\). La monade 
% principale è la monade dello zero, che è costituita esattamente da tutti gli 
% infinitesimi. Indichiamo la monade di \(x\) con \(\mon{x}\).

\ref{ese:iper_021} 
\emph{Quando due numeri si dicono a distanza finita?}

Due numeri si dicono a distanza finita quando la loro differenza è un numero 
finito. 
Per indicare che due numeri x e y sono a distanza finita scriviamo 
\(x \approx y\).

\ref{ese:iper_022} 
\emph{Di quali proprietà gode la relazione \(x \approx y\)?}

La relazione gode della proprietà riflessiva, simmetrica e transitiva ed è 
quindi una relazione di equivalenza. In simboli:\\
\(x \approx x\);\\
\(x \approx y \sRarrow y \approx x\);\\
\(x \approx y \sand y \approx z \sRarrow x \approx z\).\\
Se un numero è a distanza finita da un finito è finito, se è a distanza finita 
da un infinito è un infinito.

% \ref{ese:iper_023} 
% \emph{Che cos'è una galassia e che cosa si intende per galassia principale?}
% 
% Due numeri si dicono appartenere a una stessa galassia se sono a distanza 
% finita tra loro. Una galassia è dunque una classe di equivalenza rispetto alla 
% relazione di essere a distanza finita. In particolare, tutti i numeri standard 
% appartengono a una stessa galassia, detta galassia principale. Indichiamo la 
% galassia del numero x con \(\Gal{x}\).

\ref{ese:iper_024} 
\emph{Come si confrontano due infinitesimi non nulli?}

Per confrontare due infinitesimi non nulli \(\epsilon\) e \(\delta\) si 
considera il loro quoziente \(\frac{\epsilon}{\delta}\). 

Se è:
\begin{description} [nosep]
 \item [infinitesimo] 
diciamo che \(\epsilon\) è un infinitesimo di ordine superiore a \(\delta\) o 
che \(\delta\) è un infinitesimo di ordine inferiore a 
\(\epsilon\) e 
scriviamo \(\epsilon = o(\delta)\).
 \item [finito non infinitesimo] 
diciamo che \(\epsilon\) e \(\delta\) sono infinitesimi dello stesso ordine e 
scriviamo \(\epsilon = O(\delta)\) o \(\delta = O(\epsilon)\).
 \item [infinito]
 allora diciamo \(\epsilon\) è un infinitesimo di ordine inferiore a \(\delta\) 
o che \(\delta\) è un infinitesimo di ordine superiore a \(\epsilon\) e 
scriviamo \(\delta = o(\epsilon)\).
\end{description}

\ref{ese:iper_025} 
\emph{Come si confrontano due infiniti?}

Dati due infiniti M e N si considera il loro quoziente \(\frac{M}{N}\). 

Se è:
\begin{description} [nosep]
 \item [infinito] 
diciamo che M è un infinito di ordine superiore a N o che N è un infinito di 
ordine inferiore a M e scriviamo \(M \gg N\) o \(N \ll M\).
 \item [finito non infinitesimo] 
diciamo che M e N sono infiniti dello stesso ordine e scriviamo \\
\(M = O(N)\) o \(N = O(M)\).
 \item [infinitesimo]
diciamo M è un infinito di ordine inferiore a N o che N è un 
infinito di ordine superiore a M e scriviamo \(M \ll N\) o \(N \gg M\).
\end{description}

\ref{ese:iper_026} 
\emph{Che cos'è la parte standard di un numero finito?}

Ogni numero finito risulta infinitamente vicino a un numero standard, detto 
appunto sua parte standard. In altre parole, ogni numero finito x può essere 
scritto in modo unico nella forma \(x = s + \epsilon\), dove s è standard e 
\(\epsilon\) è un infinitesimo eventualmente nullo. La parte standard di x si 
indica con \(\pst{x}\).

% \newpage %---------------------------------------------

\ref{ese:iper_027} 
\emph{Elencare le proprietà salienti della parte standard.}

Indicando con a e b due numeri finiti (eventualmente infinitesimi o nulli):
\vspace{-.5em}
\begin{multicols}{2}
\(\pst{a} = a ⇔ a\) è standard

\(\pst{a} = 0 ⇔ a\) è infinitesimo

\(\pst{a \pm b} = \pst{a} \pm \pst{b}\)

\(\pst{ab} = \pst{a} \pst{b}\)

\(\pst{\frac{a}{b}} = \frac{\pst{a}}{\pst{b}}\)

\(\pst{a} > 0 ⇒ a > 0\)

\(a > 0 \sRarrow \pst{a} ≥ 0\)
\end{multicols}
\vspace{-.5em}
\ref{ese:iper_028} 
\emph{Quando due numeri non nulli si dicono indistinguibili?}

Due numeri non nulli si dicono indistinguibili se il loro rapporto è 
infinitamente vicino all'unità o, equivalentemente, se la loro differenza è 
infinitesima rispetto a ciascuno di essi. Indichiamo il fatto che x è 
indistinguibile da y con \(x \sim y\). 
In simboli \(x \sim y\) se e solo se vale una 
delle seguenti condizioni equivalenti:
\begin{center}
\textbullet ~ \(\dfrac{x}{y}\approx 1\) \qquad 
\textbullet ~ \(\pst{\dfrac{x}{y}}=1\) \qquad 
\textbullet ~ \(\dfrac{x - y}{x} \approx 0\) \qquad 
\textbullet ~ \(\dfrac{x - y}{y} \approx 0\)
\end{center}


\ref{ese:iper_029} 
\emph{Di quali proprietà gode la relazione \(x \sim y\)?}

Si tratta di una relazione riflessiva, simmetrica e transitiva e quindi di una 
relazione di equivalenza sugli iperreali non nulli. In simboli:
\begin{center}
\textbullet ~ \(x \sim x\) \qquad
\textbullet ~ \(x \sim y \sRarrow y \sim x\) \qquad
\textbullet ~ \(x \sim y \sand y \sim z \sRarrow x \sim z\) \qquad
\end{center}
Inoltre, se due numeri sono indistinguibili allora sono dello stesso tipo, cioè 
entrambi infinitesimi, finiti non infinitesimi o infiniti.

\ref{ese:iper_030} 
\emph{Quando sostituiamo il simbolo \(\sim\) con quello di uguaglianza?}

Quando siamo portati a identificare due numeri indistinguibili e, in questo 
caso, sostituiamo un numero con uno da esso indistinguibile e il più possibile 
semplice.

\ref{ese:iper_031} 
\emph{Che cosa si intende per microscopio standard?}

Per microscopio standard si intende uno strumento ottico ideale che, puntato su 
un numero, consente di vedere una porzione di retta ingrandita di un fattore n. 
Il microscopio standard può essere utilizzato per esplorare il campo visivo 
di ogni altro strumento ottico standard o non standard.

\ref{ese:iper_032} 
\emph{Che cosa si intende per telescopio standard?}

Per telescopio standard si intende uno strumento ottico ideale in grado di 
mostrare una parte remota di retta nella stessa scala della parte vicina. Il 
telescopio standard può essere utilizzato per esplorare il campo visivo di ogni 
altro strumento ottico standard o non standard.

\ref{ese:iper_033} 
\emph{Che cosa si intende per zoom standard?}

Per zoom standard si intende uno strumento ottico ideale che, puntato 
nell'origine consente di vedere una parte di retta centrata nell'origine e in 
una scala rimpicciolita di un fattore n. Lo zoom standard può esser utilizzato 
per esplorare il campo visivo di ogni altro strumento ottico standard o non 
standard in cui sia visibile lo zero.

\ref{ese:iper_034} 
\emph{Che cosa si intende per microscopio non-standard?}

Per microscopio non-standard si intende uno strumento ottico ideale che, 
puntato su un numero, consente di vedere un'opportuna porzione di numeri 
infinitamente vicini a quel numero. 
Il microscopio non-standard può essere utilizzato per esplorare il campo 
visivo di ogni altro strumento ottico standard o non standard.

% Per microscopio non-standard si intende uno strumento ottico ideale che, 
% puntato su un numero, consente di vedere un'opportuna porzione della monade 
% di quel numero. 
% Il microscopio non-standard può essere utilizzato per esplorare il campo 
% visivo di ogni altro strumento ottico standard o non standard.

\ref{ese:iper_035} 
\emph{Che cosa si intende per telescopio non-standard?}

Per telescopio standard si intende uno strumento ottico ideale in grado di 
mostrare una parte di retta a distanza infinita nella stessa scala della parte 
vicina. Il telescopio non-standard può essere utilizzato per esplorare il 
campo visivo di ogni altro strumento ottico standard o non standard.

\ref{ese:iper_036} 
\emph{Che cosa si intende per zoom non-standard?}

Per zoom non-standard si intende uno strumento ottico ideale che, puntato 
nell'origine consente di vedere una parte di retta centrata nell'origine e in 
una scala rimpicciolita in modo tale da far entrare nel campo visivo numeri 
infiniti. Lo zoom non-standard può essere utilizzato per esplorare il campo 
visivo di ogni altro strumento ottico standard o non standard in cui sia 
visibile lo zero.

\ref{ese:iper_037} 
\emph{Cosa intendiamo per scala naturale di ingrandimento?}

Una scala di rappresentazione della retta in cui il punto di coordinata 1 sia 
visibile e ben distinguibile dallo zero.

\ref{ese:iper_038} 
\emph{Come possiamo visualizzare un numero infinitesimo non nullo sulla retta
iperreale?}

Un numero infinitesimo non nullo può essere visualizzato come un numero che 
nella scala naturale risulta non separato dallo zero e che non può essere 
separato dallo zero con nessun microscopio standard. Occorre invece un 
microscopio non-standard per separarlo dallo zero.

\ref{ese:iper_039} 
\emph{Come possiamo visualizzare un numero infinito sulla retta iperreale?}

Un numero infinito può essere visualizzato come un numero che nella scala 
naturale risulta esterno al campo visivo e che non può essere fatto entrare nel 
campo visivo di nessuno zoom standard. Occorre invece uno zoom non-standard per 
farlo entrare nel campo visivo.

\ref{ese:iper_040} 
\emph{Come possiamo visualizzare un numero finito non infinitesimo sulla retta
iperreali?}

Un numero finito non infinitesimo può essere visualizzato come un numero che 
già nella scala naturale rientra nel campo visivo e ben separato dallo zero; 
oppure come un numero che nella scala naturale risulta non separato dallo zero, 
ma che è separabile con un microscopio standard; infine, come un numero che 
nella scala naturale non rientra nel campo visivo, ma che possiamo far 
rientrare nel campo visivo di uno zoom standard.

\ref{ese:iper_041} 
\emph{Come possiamo visualizzare il fatto che \(\epsilon\) è un infinitesimo di 
ordine superiore a \(\delta\)?}

Nella scala naturale i due numeri risultano non separati dallo zero e non è 
possibile separarli con nessun microscopio standard. Usando un microscopio 
non-standard riusciamo a separare dallo zero in numero \(\delta\), mentre il 
numero \(\epsilon\) risulta non separato dallo zero e non si riesce a separarlo 
con nessun microscopio standard. In altri termini, nella scala in cui 
\(\delta\) risulta visibile e separato dallo zero, 
\(\epsilon\) risulta infinitesimo.

\ref{ese:iper_042} 
\emph{Come possiamo visualizzare il fatto che \(\epsilon\) e \(\delta\) sono 
due infinitesimo dello stesso ordine?}

Nella scala naturale i due numeri risultano non separati dallo zero e non è 
possibile separarli con nessun microscopio standard. Usando un microscopio 
non-standard riusciamo a separare dallo zero entrambi i numeri; oppure 
riusciamo a separane solo uno, mentre l'altro risulta ancora non separato dallo 
zero, ma basta un microscopio standard per separare anche il secondo.

\ref{ese:iper_043} 
\emph{Come possiamo visualizzare il fatto che M è un infinito di ordine 
superiore a N?}

Nella scala naturale i due numeri risultano esterni al campo visivo e non è 
possibile farli rientrare nel campo visivo di nessuno zoom standard. Usando uno 
zoom non standard possiamo far rientrare nel campo visivo in numero N, mentre 
il numero M continua a rimanere esterno al campo visivo e non si riesce a far 
rientrare con nessuno zoom standard. In altri termini, nella scala in cui N 
risulta visibile e separato dallo zero, M risulta infinito.

\ref{ese:iper_044} 
\emph{Come possiamo visualizzare il fatto che M e N sono due infiniti dello 
stesso ordine?}

Nella scala naturale i due numeri risultano esterni al campo visivo e non è 
possibile farli entrare nel campo visivo di nessuno zoom standard. Usando uno 
zoom non-standard riusciamo a far rientrare nel campo visivo e separati dallo 
zero entrambi i numeri; oppure riusciamo a farne entrare solo uno, separato 
dallo zero, mentre l'altro risulta ancora esterno al campo visivo, ma basta uno 
zoom standard per far rientrare anche il anche il secondo.

\subsubsection*{\numnameref{subsec:insnum_operazioni}}

Nei problemi di questa sezione si assuma che: 
\(\epsilon, \delta\, \dots\) siano \emph{inn} postivi, 
\(H, K, \dots\) siano \emph{I} postivi.

\begin{esercizio}\label{ese:iper_op_01}
Determina se le seguenti espressioni sono equivalenti a un numero 
\emph{inn}, \emph{fni}, \emph{I}.
\begin{multicols}{3}
\begin{enumeratea}
 \item \(7,3 \cdot 10^23 \cdot \epsilon\)
 \item \(8 +\dfrac{1}{\epsilon}\)
 \item \(\dfrac{2}{\sqrt{\epsilon}}\) 
 \item \(\dfrac{H}{9^{9^9}}\) 
 \item \(\dfrac{5 \epsilon^3 -4 \epsilon^4}
               {7 \epsilon -3 \epsilon^2 +\epsilon}\)
 \item \(\tonda{3 +\epsilon} \tonda{3 -\epsilon} -6\) 
 \item \(\dfrac{4 \epsilon - 5 \epsilon^2}{2 \epsilon^2 -3 \epsilon^3}\) 
 \item \(\dfrac{2}{\sqrt{\epsilon} - \epsilon}\) 
 \item \(\dfrac{4}{5 \epsilon} \cdot \dfrac{2 \epsilon}{6}\) 
 \item \(\dfrac{H + K}{HK}\) 
 \item \(H^2 - H\)  
%  \item \(\tonda{H + \dfrac{1}{H}}^2 - \tonda{H - \dfrac{1}{H}}^2\)
 \item \(\dfrac{5 H^4 -4 H +7} {7 H^3 -4}\)
 \item \(\dfrac{1}{\epsilon} \tonda{\dfrac{1} {4 + \epsilon} -
                                    \dfrac{1}{\epsilon}}\)
 \item \(7 \epsilon + 5 \delta\)
 \item \(3 \epsilon^3 +2 \epsilon^2 - \epsilon +4\)
 \item \(\tonda{5 - \epsilon}^2 - 25\)
 \item \(\dfrac{5 \epsilon^3 + 7  \epsilon^4}{2 \epsilon^3 + 3  \epsilon^4}\)
 \item \(\dfrac{\sqrt{\epsilon} -4 \epsilon}{\sqrt{\epsilon} + 3}\)
 \item \(\dfrac{H -7 +\epsilon}{H^2 + 3 \epsilon}\)
\end{enumeratea}
\end{multicols}
\end{esercizio}

\subsubsection*{\numnameref{subsec:insnum_espressioni}}

\begin{esercizio}\label{ese:iper_op_01}
Esegui i seguenti calcoli nell'insieme degli \(\IR\) sapendo che 
x è un infinitesimo non nullo.
\begin{multicols}{2}
\begin{enumeratea}
 \item \(\pst{9 -3x}\)
 \item \(\pst{5 +2x -x^2}\)
 \item \(\pst{7 +2x^3}\) 
 \item \(\pst{2+8x+x^2}\) 
 \item \(\pst{\dfrac{2-5x}{6 +7x}}\)
 \item \(\pst{\sqrt{3 +x} + \sqrt{3 -x}}\) 
 \item \(\pst{\dfrac{x^4 - x^2 + 4x}{3x^2 -2x -3}}\) 
 \item \(\pst{\dfrac{x^4 - x^3 + x^2}{2x^2}}\) 
 \item \(\pst{\dfrac{4x^4 - 3x^3 + 2x^2}{3x^4 -2x^3 +x^2}} \) 
 \item \(\pst{\tonda{2+x}\tonda{3-x^2}} \)  
\end{enumeratea}
\end{multicols}
\end{esercizio}

\begin{esercizio}\label{ese:iper_op_01}
Esegui i seguenti calcoli nell'insieme degli \(\IR\) sapendo che 
x è un infinito positivo.
\begin{multicols}{2}
\begin{enumeratea}
 \item \(\pst{9 -3x}\)
 \item \(\pst{5 +2x -x^2}\)
 \item \(\pst{\dfrac{2x +4}{3x-6}}\)
 \item \(\pst{\dfrac{6 x-7}{x^2 +2}}\)
 \item \(\pst{\dfrac{3x^2 -5x +2}{x^2 +1}}\)
 \item \(\pst{\dfrac{x^4 +3x^2 +1}{4x^4 +2x^2 -1}}\)
 \item \(\pst{\dfrac{x^4 - x^2 + 4x}{-3x^2 -2x -3}}\) 
 \item \(\pst{\dfrac{x^4 - x^3 + x^2}{2x^2}}\) 
 \item \(\pst{\dfrac{-4x^4 - 3x^3 + 2x^2}{3x^4 -2x^3 +x^2}} \) 
 \item \(\pst{\tonda{2+x}\tonda{3-x^2}} \)  
\end{enumeratea}
\end{multicols}
\end{esercizio}

