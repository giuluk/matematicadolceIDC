% (c) 2015 Daniele Zambelli daniele.zambelli@gmail.com


\section{Esercizi}
\subsection{Esercizi dei singoli paragrafi}

\subsubsection*{\numnameref{sec:insnum_iperreali}}

\paragraph{Riflessioni}
Le risposte alle domande che seguono vanno fatte individualmente. Al termine è 
utile un confronto in classe.
\begin{esercizio}
Se $d$ è un numero positivo estremamente piccolo e se $\delta$ è un 
infinitesimo positivo
\begin{enumerate} [noitemsep]
\item  cosa ne pensi della grandezza di $d^2$, $\delta^2$, $2d$, $2\delta$ e 
$-d$, $-\delta$?
\item come immagini $10+d$, $10+\delta$ e $10-d$, $10-\delta$?
\item cosa pensi di $\dfrac{2}{d}$ e $\dfrac{2}{\delta}$?
\item cosa pensi di $\dfrac{d}{2}$ e $\dfrac{\delta}{2}$?
\item a quali valori per $d$ e per $\delta$ hai fatto riferimento mentre
 rispondevi a queste domande?
\end{enumerate}
\end{esercizio}

\begin{esercizio}
Se $X$ è un numero grandissimo e $\Xi$ un infinito, entrambi positivi,
\begin{enumerate} [noitemsep]
\item cosa ne pensi della grandezza di $X^2$, $2X$ e $-X$? e di $\Xi^2$, $2\Xi$
e $-\Xi$?
\item come immagini $X+3$ e $X-3$? e di $\Xi+3$ e $\Xi-3$?
\item cosa pensi di $\dfrac{1}{X}$ e di $\dfrac{1}{\Xi}$?
\item cosa pensi di $\dfrac{X}{2}$ e di $\dfrac{\Xi}{2}$?
\item a quali valori per $X$ e per $\Xi$ hai fatto riferimento mentre
 rispondevi a queste domande?
 \end{enumerate}
\end{esercizio}

\begin{esercizio}
 Se $\epsilon$ è un infinitesimo positivo, 
\end{esercizio}

\begin{esercizio}
Sia $f:x\mapsto x^2$,  e sia $\delta$ infinitesimo positivo.

\begin{enumerate} [noitemsep]
\item Disegna quanto risulta applicando un microscopio centrato su $\punto{1}{1}$
in modo che $\delta$ sia visibile.

Nel disegno mostra i valori $1$ e $f(1)$,  $1+\delta$ e 
$f(1+\delta)$,  $1-\delta$ and $f(1-\delta)$.

Disegna il tratto di curva che appartiene al grafico della funzione: che 
aspetto ha?

Quale ingrandimento stai applicando?

\item Per la stessa funzione, disegna il campo visivo di un microscopio
centrato su $\punto{2}{4}$.

Nel disegno mostra i valori $2$ e $f(2)$, $2+\delta$ e $f(2+\delta)$, 
$2-\delta$ e $f(2-\delta)$.

Quale ingrandimento stai applicando?

\item Come il precedente per un microscopio centrato su $\punto {0}{0}$.
\item Come il precedente per un microscopio centrato su $\punto {-1}{1}$.
\end{enumerate}
\end{esercizio}

\input{\folder iperreali_ese.tex}

