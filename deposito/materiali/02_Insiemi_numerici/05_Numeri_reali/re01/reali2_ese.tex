% (c)~2014 Claudio Carboncini - claudio.carboncini@gmail.com
% (c)~2014 Dimitrios Vrettos - d.vrettos@gmail.com
\section{Esercizi}
\subsection{Esercizi dei singoli paragrafi}
\subsubsection*{1.1 - Dai numeri naturali ai numeri irrazionali}

\begin{esercizio}
\label{ese:1.1}
Dimostra, con un ragionamento analogo a quello fatto per $\sqrt 2$, che $\sqrt 3$ non è razionale.
\end{esercizio}

\begin{esercizio}
\label{ese:1.2}
 Per ciascuno dei seguenti numeri reali scrivi una sequenza di sei numeri razionali che lo approssimano per difetto e sei numeri razionali che lo approssimano per eccesso. Esempio:
$\sqrt 3$:\,$A=\{1; 1,7; 1,73; 1,732; 1,7320; 1,73205\},$\,$B=\{2; 1,8; 1,74; 1,733; 1,7321; 1,73206\}$
\begin{enumeratea}
 \item$\sqrt{5}:\, A=\{\ldots\ldots\ldots\ldots\ldots\ldots\ldots\ldots\},\quad B=\{\ldots\ldots\ldots\ldots\ldots\ldots\ldots\ldots\}$
 \item$\dfrac{6}{7}:\, A=\{\ldots\ldots\ldots\ldots\ldots\ldots\ldots\ldots\},\quad B=\{\ldots\ldots\ldots\ldots\ldots\ldots\ldots\ldots\}$
 \item$\dfrac{1}{7}:\, A=\{\ldots\ldots\ldots\ldots\ldots\ldots\ldots\ldots\},\quad B=\{\ldots\ldots\ldots\ldots\ldots\ldots\ldots\ldots\}$
\end{enumeratea}
\end{esercizio}

\subsubsection*{1.2 - I numeri reali}

\begin{esercizio}
\label{ese:1.3}
 Per ciascuno dei seguenti numeri reali scrivi una sequenza di almeno sei numeri razionali che lo approssimano per difetto e sei numeri razionali che lo approssimano per eccesso:
\begin{multicols}{2}
\begin{enumeratea}
 \item~$\sqrt {2}+\sqrt {3}$
 \item~$\sqrt {2}\cdot\sqrt {3}$
\end{enumeratea}
\end{multicols}
\end{esercizio}

\begin{esercizio}
\label{ese:1.4}
 Determina per ciascuno dei seguenti numeri irrazionali i numeri interi tra i quali è compreso. Esempio: $5<\sqrt{30}<6$
\begin{multicols}{4}
\begin{enumeratea}
 \item~$\sqrt{50}$
 \item~$\sqrt{47}$
 \item~$\sqrt{91}$
 \item~$\sqrt{73}$
 \item~$\sqrt{107}$
 \item~$\sqrt{119}$
 \item~$\sqrt 5+\sqrt 3$
 \item~$2\sqrt 7$
 \item~$2+\sqrt 7$
 \item~$\sqrt{20}-\sqrt{10}$
 \item~$\sqrt{\frac 7{10}}$
 \item~$7+\sqrt{\frac 1 2}$
\end{enumeratea}
\end{multicols}
\end{esercizio}

\begin{esercizio}
\label{ese:1.5}
 Disponi in ordine crescente i seguenti numeri reali:
 \begin{enumeratea}
 \item $\sqrt 2$,\quad $1$,\quad $\dfrac 2 3$,\quad $2,0\overline{13}$,\quad $\sqrt 5$,\quad $\dfrac 3 2$\quad $0,75$
 \item $\pi$,\quad $\sqrt 3$,\quad $\dfrac{11} 5$,\quad $0,\overline 9$,\quad $\sqrt{10}$,\quad $3,1\overline 4$,\quad $\sqrt[3]{25}$
 \end{enumeratea}
\end{esercizio}

\begin{esercizio}
\label{ese:1.6}
 Rappresenta con un diagramma di Eulero-Venn l'insieme dei numeri reali $\insR$, suddividilo nei seguenti sottoinsiemi: l'insieme dei numeri naturali $\insN$, l'insieme dei numeri interi relativi~$\insZ$, l'insieme dei numeri razionali $\insQ$, l'insieme $\insJ$ dei numeri irrazionali. Disponi in maniera opportuna i seguenti numeri: $\sqrt 3$,\quad $\sqrt[3]5$,\quad$\pi$,\quad $0,\overline 3$,\quad $3,14$,\quad $\frac 3 2$,\quad$-2$
\end{esercizio}

% \newpage

\begin{esercizio}
\label{ese:1.7}
Indica il valore di verità delle seguenti affermazioni:
\begin{enumeratea}
\item un numero decimale finito è sempre un numero razionale;
\item un numero decimale illimitato è sempre un numero irrazionale;
\item un numero decimale periodico è un numero irrazionale;
\item la somma algebrica di due numeri razionali è sempre un numero razionale;
\item la somma algebrica di due numeri irrazionali è sempre un numero irrazionale;
\item il prodotto di due numeri razionali è sempre un numero razionale;
\item il prodotto di due numeri irrazionali è sempre un numero irrazionale.
\end{enumeratea}
\end{esercizio}

\subsubsection*{1.3 - Richiami sul valore assoluto}
\begin{esercizio}
\label{ese:1.8}
 Calcola il valore assoluto dei seguenti numeri:
\begin{multicols}{3}
 \begin{enumeratea}
 \item~$\valass{-5}$
 \item~$\valass{+2}$
 \item~$\valass{-1}$
 \item~$\valass{0}$
 \item~$\valass{-10}$
 \item~$\valass{3-5(2)}$
 \item~$\valass{-3+5}$
 \item~$\left|{(-1)^3}\right|$
 \item~$\valass{-1-2-3}$
% \item~$\valass{3(-2)-5}$
 \end{enumeratea}
 \end{multicols}
\end{esercizio}

\begin{esercizio}
\label{ese:1.9}
Dati due numeri reali $x$ ed $y$ entrambi non nulli e di segno opposto, verifica le seguenti relazioni con gli esempi numerici riportati sotto.
Quali delle relazioni sono vere in alcuni casi e false in altri, quali sono sempre vere, quali sono sempre false?
\begin{center}
 \begin{tabular}{lcccc}
\toprule
Relazione & $x=-3, y=5$&$x=-2, y=2$ &$x=-10, y=1$&$x=1, y=-5$\\
\midrule
$\valass{x}<\valass{y}$& \boxV\qquad\boxF& \boxV\qquad\boxF&\boxV\qquad\boxF&\boxV\qquad\boxF\\
$\valass{x}=\valass{y}$& \boxV\qquad\boxF& \boxV\qquad\boxF&\boxV\qquad\boxF&\boxV\qquad\boxF\\
$\valass{x}<y$& \boxV\qquad\boxF& \boxV\qquad\boxF&\boxV\qquad\boxF&\boxV\qquad\boxF\\
$\valass{x+y}<\valass{x}+\valass{y}$& \boxV\qquad\boxF& \boxV\qquad\boxF&\boxV\qquad\boxF&\boxV\qquad\boxF\\
$\valass{x-y}=\valass{x}-\valass{y}$& \boxV\qquad\boxF& \boxV\qquad\boxF&\boxV\qquad\boxF&\boxV\qquad\boxF\\
$\left|{\valass{x}-\valass{y}}\right|=\valass{x-y}$& \boxV\qquad\boxF& \boxV\qquad\boxF&\boxV\qquad\boxF&\boxV\qquad\boxF\\
\bottomrule
\end{tabular}
\end{center}
\end{esercizio}

\begin{esercizio}
\label{ese:1.10}
 Elimina il segno di valore assoluto dalle seguenti espressioni sostituendole con una funzione definita per casi:
 \begin{multicols}{2}
 \begin{enumeratea}
 \item~$f(x)=\left|x+1\right|$
 \item~$f(x)=\left|x-1\right|$
 \item~$f(x)=\left|x^2+1\right|$
 \item~$f(x)=\left|(x+1)^2\right|$
 \item~$f(x)=\left|x^2-1\right|$
 \item~$f(x)=\left|x^3-1\right|$
 \item~$f(x)=\left|x^2-6x+8\right|$
 \item~$f(x)=\left|x^2+5x+4\right|$
 \end{enumeratea}
 \end{multicols}
\end{esercizio}

\begin{esercizio}
\label{ese:1.11}
 Elimina il segno di valore assoluto dalle seguenti espressioni sostituendole con una funzione definita per casi:
 \begin{multicols}{2}
 \begin{enumeratea}
 \item~$f(x)=\dfrac{\left|x+1\right|}{\left|x+2\right|}$
 \item~$f(x)=\left|\dfrac{x+1}{x-1}\right|$
 \item~$f(x)=\left|x+1\right|+\left|x-2\right|$
 \item~$f(x)=\left|x+2\right|+\left|x-2\right|$
 \item~$f(x)=\left|x-2\right|+\left|x-3\right|$
 \item~$f(x)=\left|x+1\right|\cdot \left|x+2\right|$
 \item~$f(x)=\left|\dfrac{x+1} 4\right|+\left|\dfrac{x+2}{x+1}\right|$
 \item~$f(x)=\left|\dfrac{x+1}{x+2}\right|+\left|\dfrac{x+2}{x+1}\right|$
 \end{enumeratea}
 \end{multicols}
\end{esercizio}

