% (c) 2017 Daniele Zambelli - daniele.zambelli@gmail.com
% 
% Tutti i grafici per il capitolo relativo alle funzioni_new
% 

% \draw (0,0) .. controls (6,1) and (9,1) ..
% node[near start,sloped,above] {near start}
% node {midway}
% node[very near end,sloped,below] {very near end} (12,0);


% \newcommand{\deffunzione}{% 
%   % Definizione di funzione
%   \disegno{
%     \shadedraw [shading=ball]
%       [postaction={decoration={text along path, text={Insieme di Esistenza},
%        text align={align=left}}, below, decorate}] 
%       (0,0) circle [x radius=2, y radius=3, ball color=red!20, rotate=+20];
%     \shadedraw[
%       top color=yellow!70,
%       bottom color=red!70,
%       shading angle={45},
%       opacity=.5,
%       x radius=4, y radius=5] (0, 0) circle; 
%     \shadedraw [shading=ball] (12,0) circle [x radius=1.6, y radius=2.4,
%                                            ball color=red!20, rotate=-20];
%     \shadedraw[
%       top color=yellow!70,
%       bottom color=blue!70,
%       shading angle={45},
%       opacity=.5,
%       x radius=4, y radius=5] (12, 0) circle;
%     \draw [|->, very thick] (0, 0) to [out=20, in=160,
%        edge node={node [sloped,above] {$f$}}] (12, 0);
%     \draw (-1, 2.824) to [out=20, in=160] (12.8, 2.26);
%     \draw (+1.1,-2.8) to [out=20, in=160] (11.2,-2.27);
%   }
% }

% \draw (0,0) to [out=90,in=180, edge node={node [sloped,above] {x}}] (3,2);


\newcommand{\deffunzione}[4]{% 
  % Definizione di funzione
  \def \nomea{#1}
  \def \nomeb{#2}
  \def \nomec{#3}
  \def \nomed{#4}
  \disegno{
    \shadedraw [shading=ball]
      (0,0) circle [x radius=2, y radius=3, ball color=red!20, rotate=+20];
    \draw [red!0, dashed]
          [postaction={decoration={text along path, text={\nomeb},
           text align={align=center}}, decorate}]
          (-2.5,1) .. controls (-2,4) and (1,4) .. (2,1);
    \shadedraw[
      top color=yellow!70,
      bottom color=red!70,
      shading angle={45},
      opacity=.5,
      x radius=4, y radius=5] (0, 0) circle; 
    \draw [red!0, dashed]
          [postaction={decoration={text along path, text={\nomea},
           text align={align=center}}, decorate}]
      (-3.5,3) .. controls (-2,6) and (2,6) .. (3.5,3);
    \shadedraw [shading=ball] (12,0) circle [x radius=1.6, y radius=2.4,
                                           ball color=red!20, rotate=-20];
    \draw [red!0, dashed]
          [postaction={decoration={text along path, text={\nomed},
           text align={align=center}}, decorate}]
          (10.2,.5) .. controls (11,3.5) and (14,3.5) .. (14,.5);
    \shadedraw[
      top color=yellow!70,
      bottom color=blue!70,
      shading angle={45},
      opacity=.5,
      x radius=4, y radius=5] (12, 0) circle;
    \draw [red!0, dashed]
          [postaction={decoration={text along path, text={\nomec},
           text align={align=center}}, decorate}]
      (8.5,3) .. controls (10,6) and (14,6) .. (15.5,3);
    \draw [|->, very thick] (0, 0) to [out=20, in=160,
       edge node={node [sloped,above] {$f$}}] (12, 0);
    \draw (-1, 2.824) to [out=20, in=160] (12.8, 2.26);
    \draw (+1.1,-2.8) to [out=20, in=160] (11.2,-2.27);
  }
}

\newcommand{\dueassivuoti}{% 
  % Disegno di due assi paralleli
  \disegno{
    \assecontrattini{-12}{+12}{0}{\(x\)}
    \assecontrattini{-12}{+12}{3}{\(y\)}
    \foreach \x in {-12, ..., 11}{
      \draw  (\x, 0) [below, font=\small] node {\x};
      \draw  (\x, 3) [above, font=\small] node {\x};
    }
  }
}

\newcommand{\dueassi}[3]{% 
  % Funzione rappresentata su due assi
  \def \mix{#1} \def \max{#2} \def \funct{#3}
  \disegno{
    \assecontrattini{-12}{+12}{0}{\(x\)}
    \assecontrattini{-12}{+12}{3}{\(y\)}
    \foreach \x in {-12, ..., 11}{
      \draw  (\x, 0) [below, font=\small] node {\x};
      \draw  (\x, 3) [above, font=\small] node {\x};
    }
    \foreach \x / \y in {\mix, ..., \max}{
      \draw [->] (\x, 0) to [out=90, in=270] (\funct, 3);
    }
  }
}


% 
% \clip[draw] (0.5,0.5) circle (.6cm);
% 

% \newcommand{\micx}[6]{% 
%   % interno del microscopio posto sull'asse x.
%   \def \basexa{#1} \def \basexb{#2} \def \basey{#3}
%   \def \xa{#4}     \def \xb{#5}     \def \yab{#6}
%   \draw (\basexa, \basey) -- (\basexb, \basey);
%   \fill [Cyan!50]  (\xa, \yab) -- (\xa, \basey) -- 
%                    (\xb, \basey) -- (\xb, \yab) -- cycle;
%   \draw [dashed] (\xa, \yab) -- (\xa, \basey) 
%         node [below, xshift=-1mm] {$x_0$};
%   \draw [dashed] (\xb, \yab) -- (\xb, \basey) 
%         node [below, xshift=+2.5mm] {$x_0 + \epsilon$};
% }
%   \draw [thick, color=Red!50!black] (pb |- pb) -- (pb |- pa)
%         node [midway, right] {$df(x)$};
% }
