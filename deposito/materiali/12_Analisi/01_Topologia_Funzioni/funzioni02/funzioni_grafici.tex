% (c) 2017 Daniele Zambelli - daniele.zambelli@gmail.com
% 
% Tutti i grafici per il capitolo relativo alle funzioni_new
% 

\newcommand{\deffunzione}[4]{% 
  % Definizione di funzione
  \def \nomea{#1}
  \def \nomeb{#2}
  \def \nomec{#3}
  \def \nomed{#4}
  \disegno{
    \shadedraw [shading=ball]
      (0,0) circle [x radius=2, y radius=3, ball color=red!20, rotate=+20];
    \draw [red!0, dashed]
          [postaction={decoration={text along path, text={\nomeb},
           text align={align=center}}, decorate}]
          (-2.5,1) .. controls (-2,4) and (1,4) .. (2,1);
    \shadedraw[
      top color=yellow!70,
      bottom color=red!70,
      shading angle={45},
      opacity=.5,
      x radius=4, y radius=5] (0, 0) circle; 
    \draw [red!0, dashed]
          [postaction={decoration={text along path, text={\nomea},
           text align={align=center}}, decorate}]
      (-3.5,3) .. controls (-2,6) and (2,6) .. (3.5,3);
    \shadedraw [shading=ball] (12,0) circle [x radius=1.6, y radius=2.4,
                                           ball color=red!20, rotate=-20];
    \draw [red!0, dashed]
          [postaction={decoration={text along path, text={\nomed},
           text align={align=center}}, decorate}]
          (10.2,.5) .. controls (11,3.5) and (14,3.5) .. (14,.5);
    \shadedraw[
      top color=yellow!70,
      bottom color=blue!70,
      shading angle={45},
      opacity=.5,
      x radius=4, y radius=5] (12, 0) circle;
    \draw [red!0, dashed]
          [postaction={decoration={text along path, text={\nomec},
           text align={align=center}}, decorate}]
      (8.5,3) .. controls (10,6) and (14,6) .. (15.5,3);
    \draw [|->, very thick] (0, 0) to [out=20, in=160,
       edge node={node [sloped,above] {$f$}}] (12, 0);
    \draw (-1, 2.824) to [out=20, in=160] (12.8, 2.26);
    \draw (+1.1,-2.8) to [out=20, in=160] (11.2,-2.27);
  }
}

\newcommand{\dueassivuoti}{% 
  % Disegno di due assi paralleli
  \disegno{
    \assecontrattini{-12}{+12}{0}{\(x\)}
    \assecontrattini{-12}{+12}{3}{\(y\)}
    \foreach \x in {-12, ..., 11}{
      \draw  (\x, 0) [below, font=\small] node {\x};
      \draw  (\x, 3) [above, font=\small] node {\x};
    }
  }
}

\newcommand{\dueassi}[3]{% 
  % Funzione rappresentata su due assi
  \def \mix{#1} \def \max{#2} \def \funct{#3}
  \disegno{
    \assecontrattini{-12}{+12}{0}{\(x\)}
    \assecontrattini{-12}{+12}{3}{\(y\)}
    \foreach \x in {-12, ..., 11}{
      \draw  (\x, 0) [below, font=\small] node {\x};
      \draw  (\x, 3) [above, font=\small] node {\x};
    }
    \foreach \x / \y in {\mix, ..., \max}{
      \draw [->] (\x, 0) to [out=90, in=270] (\funct, 3);
    }
  }
}

\newcommand{\puntonelpc}[2]{
  % Punto che rappresenta la relazione tra argomento e risultato 
  % di una funzione.
  \def \argomento{#1} \def \risultato{#2}
  \disegno{
    \rcomvar{-7}{+7}{-3}{+10}{gray!50, very thin, step=1}
            {argomento}{risultato}
    \draw [dashed, red!50!black] (\argomento, \risultato) --
      (0, \risultato) node [left, blue!50!black] {$f(x)$};
    \draw [dashed, blue!50!black] (\argomento, \risultato) --
      (\argomento, 0) node [below, red!50!black] {$x$};
    \filldraw [Maroon!50!black] (\argomento, \risultato) circle (1.5pt)
      node [above] {P(arg; ris)};
  }
}

\newcommand{\graficoa}{% Alcuni punti del grafico di una funzione.
  \disegno{
    \rcom{-5}{+7}{-10}{+10}{gray!50, very thin, step=1}
    \foreach \x in {-3,...,6}
      \filldraw [Blue!50!black] (\x, -2*\x+3) circle (2pt);
  }
}

\newcommand{\graficob}{% Grafico di una funzione con punti evidenziati.
  \disegno{
    \graficoxy{-5}{+7}{-10}{+10}{-2*x+3}
    \foreach \x in {-3,...,6}
      \filldraw [Blue!50!black] (\x, -2*\x+3) circle (2pt);
  }
}


\newcommand{\intervallimonotonia}{% Interv. di monotonia di una funzione.
  \disegno{
    \graficoxy{-10}{+10}{-10}{+10}{.03*x**3-.2*x**2-x}
    \draw [Maroon!50!black] (-1.8, 1) -- (-1.8, 0) node [below] {$a$};
    \draw [Maroon!50!black] (6.25, -6.8) -- (6.25, 0) node [above] {$b$};
  }
}


% \newcommand{\graficoxy}[5]{  % Grafico di una funzione.
%   % Esempio di chiamata:
%   %% \disegno{\graficoxy{-7}{+7}{-11}{+10}{2*x+3}
%   \def \grxmi{#1}
%   \def \grxma{#2}
%   \def \grymi{#3}
%   \def \gryma{#4}
%   \def \funzione{#5}
%   \pgfmathparse{\grxmi-.3)} \let\grxxmi\pgfmathresult
%   \pgfmathparse{\grxma+.3)} \let\grxxma\pgfmathresult
%   \pgfmathparse{\grymi-.3)} \let\gryymi\pgfmathresult
%   \pgfmathparse{\gryma+.3)} \let\gryyma\pgfmathresult
%     \rcom{\grxmi}{\grxma}{\grymi}{\gryma}{gray!50, very thin, step=1}
%     \tkzInit[xmin=\grxxmi,xmax=\grxxma,ymin=\gryymi,ymax=\gryyma]
%     \tkzFct[domain=\grxxmi:\grxxma, ultra thick, color=Maroon!50!black]
%            {\funzione}
% }


%     \foreach \pi in {
%     (-3, 13), (-2, 6), (-1, 1), (0, -2), (1, -3), (2, -2), (3, 1), 
%     (4, 6), (5, 13),
%     (-2.5, 9.25), (-1.5, 3.25), (-0.5, -0.75), (0.5, -2.75), 
%     (1.5, -2.75), (2.5, -0.75), (3.5, 3.25), (4.5, 9.25) 
%     }
%     \filldraw [Maroon!50!black] \pi circle (1.5pt);
    
    
% 
% \clip[draw] (0.5,0.5) circle (.6cm);
% 

% \newcommand{\micx}[6]{% 
%   % interno del microscopio posto sull'asse x.
%   \def \basexa{#1} \def \basexb{#2} \def \basey{#3}
%   \def \xa{#4}     \def \xb{#5}     \def \yab{#6}
%   \draw (\basexa, \basey) -- (\basexb, \basey);
%   \fill [Cyan!50]  (\xa, \yab) -- (\xa, \basey) -- 
%                    (\xb, \basey) -- (\xb, \yab) -- cycle;
%   \draw [dashed] (\xa, \yab) -- (\xa, \basey) 
%         node [below, xshift=-1mm] {$x_0$};
%   \draw [dashed] (\xb, \yab) -- (\xb, \basey) 
%         node [below, xshift=+2.5mm] {$x_0 + \epsilon$};
% }
%   \draw [thick, color=Red!50!black] (pb |- pb) -- (pb |- pa)
%         node [midway, right] {$df(x)$};
% }
