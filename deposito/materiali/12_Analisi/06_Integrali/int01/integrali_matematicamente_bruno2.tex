% (c) 2015 Daniele Zambelli daniele.zambelli@gmail.com

\chapter{L'integrale}

Uno dei problemi che l'integrazione vuole affrontare è quello di determinare 
l'area tra una curva, che è il grafico di una funzione continua f in un 
intervallo I, l'asse delle ascisse e due rette verticali per gli estremi a e 
b (a<b) di un intervallo chiuso contenuto in I sull'asse delle ascisse.

Fissata la funzione f, tale area dipenderà dalla scelta degli estremi a e b 
dell'intervallo, ed è possibile indicarla come A(a,b). Questa funzione 
binaria A ha due tipiche proprietà che potranno essere usate per 
determinarla. Esse sono: 

\begin{enumerate}
 \item 
la proprietà additiva, cioè se a, b e c 
appartengono all'intervallo I e 
\(a < b < c\)
allora A(a,c) = A(a,b)+A(b,c),
 \item 
la proprietà rettangolare rispetto alla funzione f, 
cioè \(m\cdot(b-a) < A(a,b)< M\cdot(b-a)\), 
con m minimo delle funzione f in \(\intervcc{a}{b}\) e M massimo delle 
funzione f in \(\intervcc{a}{b}\).
\end{enumerate}
 
Detto altrimenti, la proprietà 1) afferma che se una superficie 
del tipo che si sta considerando viene divisa in due parti da una retta 
verticale, allora l'area totale è la somma delle aree delle due parti, e la 
2) dice che l'area considerata è maggiore dell'area di un rettangolo di ugual 
base e altezza il minimo della funzione f nell'intervallo \(\intervcc{a}{b}\) 
e minore di 
un rettangolo con la stessa base e altezza il massimo della funzione f 
nell'intervallo \(\intervcc{a}{b}\).

Seguendo l'approccio non standard al calcolo integrale, consideriamo una 
funzione continua f su un intervallo I cui appartengono i punti a e b, e una 
partizione dell'intervallo chiuso \(\intervcc{a}{b}\) in sotto intervalli di 
uguale 
ampiezza \(\Delta (\Delta>0)\) 
separati dai punti: \(a = x_0 < x_1 < … < x_h \le b\) 
dove \(x_i = x_0 +i \Delta\), \(x_{i+1} - x_i= \Delta\) e h è il massimo  
numero naturale tale che 
\(x_0 + h \Delta \le b\), e dunque \(b-x_h \le \Delta\).

Si può considerare la sommatoria
\[\left(\sum_{i=0}^{h-1} f(x_i)\cdot\Delta \right)+f(x_i)\cdot(b-x_h)\] 
che è l'area dell'unione dei rettangoli che 
hanno per base i segmenti \(\intervcc{x_i}{x_{i+1}}\) 
(tutti di lunghezza \(\Delta\)) e 
il segmento 
\(\intervcc{x_{h+1}}{b}\) (di lunghezza minore di \(\Delta\)) e altezza 
\(f(x_i)\). 

Al diminuire della distanza \(\Delta\) tra i punti di partizione 
l'area coperta da questa unione di rettangoli sarà sempre più vicina all'area 
cercata.

Per come è stata definita questa sommatoria dipende solo dagli estremi a e b 
e da \(\Delta\), essendo $h$ e la successione degli \(x_i\) determinati dalle 
quantità indicate, sicché la denoteremo come 
\[\sum_a^b f(x) \cdot\Delta\] 
che, fissati a e b, è una 
funzione reale unaria di \(\Delta\), chiamata somma di Riemann finita. 

Nello spirito dei metodi non standard, questa funzione avrà un'estensione 
naturale che può essere valutata anche per un infinitesimo positivo \(dx\) 
al posto di \(\Delta\).

Allora il massimo numero \(H\) tale che \(x_H \le b\) sarà un ipernaturale 
infinito (\(x_H = x_0 + H \cdot dx\)).


Se la funzione f(x) è continua in \(\intervcc{a}{b}\), come si sta 
supponendo, 
ricordando che una funzione continua in un intervallo chiuso ha massimo e 
minimo, per la somma di Riemann finita si ha che 
\[m\cdot(b - a) \le \sum_a^b f(x) \cdot\Delta \le M\cdot(b - a)\] 
dove m è il minimo della funzione f in \(\intervcc{a}{b}\), e 
M è il massimo della funzione f in \(\intervcc{a}{b}\), sicché, per transfer, 
anche le per 
la somma di Riemann infinita vale che 
\[m \cdot (b - a) \le  \sum_a^b f(x) \cdot dx  \le M\cdot(b - a)\]
e pertanto la sommatoria sarà un numero iperreale finito che così ha parte 
standard. 
Ciò ci permette di definire l'\emph{integrale definito} di una funzione 
standard \(f\) continua su \(\intervcc{a}{b}\) come:
\[\int_a^b f \cdot dx = \pst{\sum_a^b f(x) \cdot dx}\]

Se poi, 
partendo dalle disequazioni precedenti si passa alle parti standard, che 
preservano le disequazioni, poiché il primo e l'ultimo termine sono parti 
standard di se stessi in quanto reali, si ha: 
\[m \cdot(b - a) \le  \int_a^b f \cdot dx  \le M\cdot(b - a)\] 
e si è dimostrata la proprietà rettangolare per l'integrale.

Dalla definizione seguono le prime proprietà dell'integrale:
\[\int_a^a f \cdot dx=0\] 
\[\int_a^b c \cdot dx = c \cdot (b-a)\] 
\[\int_a^b c \cdot f \cdot dx = c\cdot\int_a^b f \cdot dx\] 
\[\int_a^b (f+g)\cdot dx = 
  \int_a^b f \cdot dx + \int_a^b g\cdot dx\]
se \(f \le g\) per ogni x in \(\intervcc{a}{b}\) allora 
\[\int_a^b f \cdot dx < \int_a^b g\cdot dx\]

Si definisce poi  
\[\int_b^a f \cdot dx = - \int_a^b f \cdot dx\]

Si vuole dimostrare ora che l'integrale è indipendente dalla scelta 
dell'infinitesimo non nullo \(dx\). Cioè, anche se \(dx\) e \(du\) sono due 
diversi infinitesimi non nulli si ha:
\[\int_a^b f(x) \cdot dx = \int_a^b f(u) \cdot du\]
Poiché entrambi gli integrali sono numeri reali per mostrare che coincidono 
basta far vedere che sono infinitamente vicini, cioè basta mostrare che, per 
ogni reale positivo r:      
\[\int_a^b f \cdot dx < r + \int_a^b f \cdot du\]
Poiché \(r\) è un reale positivo lo si può dividere per \((b-a)\) e ottenere 
il reale positivo \(c\) tale che 
\[r = c \cdot (b-a) = \int_a^b c \cdot dx\] 
sicché la disuguaglianza precedente equivale a:
\[\int_a^b f \cdot dx \le \int_a^b (f+c)\cdot du\] 
per ogni numero reale positivo \(c\). Questa affermazione sarà dimostrata 
per assurdo. Si supponga che 
\[\int_a^b f\cdot dx > \int_a^b (f+c)\cdot du\] 
per qualche reale positivo c, cioè che esista un reale positivo c tale che 
\[\sum_a^b f \cdot dx > \sum_a^b (f+c)\cdot du\]
Più precisamente:

\begin{flalign*}
 \sum_a^b f \cdot dx &=\\
   &=\tonda{\sum_{i=0}^{H-1} f(x_i) \cdot dx} + 
     \tonda{f \tonda{a+H \cdot dx} \cdot \tonda{b-\tonda{a+H \cdot dx}}}>&&\\
   &>\tonda{\sum_{j=0}^{K-1} \tonda{f(u_j) +c} \cdot du} + 
     \tonda{\tonda{f \tonda{a+K \cdot du} +c} \cdot 
           \tonda{b-\tonda{a+K \cdot du}}} = &&\\
   &=\sum_a^b (f+c)\cdot du &&
\end{flalign*}

(dove \(H\) è il massimo naturale tale che 
\(a+H \cdot dx < b\) 
e \(K\) è il massimo naturale tale che
\(a+K \cdot du < b\))
e, per il transfer: 
\begin{flalign*}
 &\tonda{\sum_{i=0}^{h-1} f(x_i) \cdot \Delta_x} + 
     \tonda{f \tonda{a+h \cdot \Delta_x} \cdot 
            \tonda{b-\tonda{a+h \cdot \Delta_x}}} > &&\\
 & \qquad > \tonda{\sum_{j=0}^{k-1} \tonda{f(u_j) +c} \cdot du} + 
  \tonda{\tonda{f \tonda{a+k \cdot \Delta_u} +c} \cdot 
        \tonda{b-\tonda{a+k \cdot \Delta_u}}} &
\end{flalign*}

% \[\sum_{i=0}^H f(x_i) \cdot \Delta_x = 
%   \sum_a^b f(x) \cdot \Delta_x > \sum_a^b (f(u)+c)\cdot \Delta_u = 
% \sum_{j=0}^K(f(u_j)+c)\cdot \Delta_u\]

(dove h è il massimo naturale tale che 
\(a+h \cdot \Delta x < b\) 
e \(k\) è il massimo naturale tale che
\(a+k \cdot \Delta_u < b\)).
Affinché ciò succeda ci devono essere almeno 
due punti 
\(\underline{x}\) e \(\underline{u}\)
tali che 
\[\underline{x} - \Delta_u < \underline{u} < \underline{x} + \Delta_u \]
e 
\[f(\underline{x}) > (f(\underline{u})+c)\] 
altrimenti i rettangoli della prima sommatoria sono tutti contenuti in quelli 
della seconda e questa non può essere minore. Tornando con il transfer 
all'ambiente non standard si ha che ci devono essere almeno due iperreali 
\(\underline{x}\) e \(\underline{u}\)
tali che 
\[\underline{x} - du < \underline{u} < \underline{x}+ du\]
e 
\[f(\underline{x}) > (f(\underline{u})+c)\] 
ma allora \(\underline{x}\) e \(\underline{u}\)
sono infinitamente vicini,  
\(\underline{x} \approx \underline{u}\)
e, per la continuità, anche 
\(f(\underline{x})\) e \((f(\underline{u})+c)\) 
devono essere infinitamente vicini 
\[f(\underline{x}) \approx (f(\underline{u})+c)\] 
contraddicendo la relazione 
\[f(\underline{x}) - f(\underline{u}) > c\]
con \(c\) reale positivo. 

Il che prova che l'integrale è indipendente dalla scelta dell'infinitesimo non 
nullo $dx$.
Il risultato ottenuto permette di giungere a mostrare che 
l'integrale ha la proprietà additiva.

Si deve mostrare che 
\[\int_a^b f \cdot dx + \int_b^c f \cdot dx = \int_a^c f \cdot dx\] 
Passando alle sommatorie 
\[\sum_a^b f \cdot dx,\quad \sum_b^c f \cdot dx \quad\text{ e } \quad 
  \sum_a^c f \cdot dx\] 
e prendendo 
\(dx = (b-a)/H\)
con \(H\) ipernaturale infinito, 
i punti di ripartizione degli intervalli \(\intervcc{a}{b}\) e 
\(\intervcc{b}{c}\) coincidono con quelli dell'intervallo \(\intervcc{a}{c}\)
sicché la terza sommatoria è evidentemente la somma delle altre due.
Passando alla loro parti standard, operazione che preserva l'addizione, si 
ottiene il risultato voluto. 

Finalmente si dimostra l'unicità della funzione binaria che soddisfa la 
proprietà additiva e la proprietà rettangolare rispetto alla funzione f.

Iniziamo definendo le somme di Riemann finite superiore e inferiore. Queste 
si distinguono dalla somma di Riemann finita perché invece di considerare il 
fattore nei vari addendi che è il valore della funzione f nell'estremo 
sinistro del sottointervallo considerato, lo si sostituisce con il valore 
massimo e con il valore minimo della f rispettivamente sempre nel 
sottointervallo considerato. Se \({}_{-}f(x_i)\) e \({}^{-}f(x_i)\) sono 
rispettivamente il minimo e il massimo della funzione \(f\) nel 
sottointervallo 
di estremi \(x_i\) e \(x_i+1\) 
e \(\Delta\) (con \(\Delta =x_{i+1} - x_i)\) 
è l'n-esima parte di \((b - a)\) 
(assunzione che non limita la generalità, come è stato giustificato), 
allora la somma di Riemann finita inferiore può essere denotata come 
\[\sum_{i=0}^h{}_{-}f(x_i)\cdot\Delta = \sum_a^b{}_{-} f(x) \cdot \Delta\] 
e quella superiore è come 
\[\sum_{i=0}^h {}^{-}f(x_i)\cdot\Delta = 
  \sum_a^b {}^{-}f(x) \cdot \Delta\] 
con il significato dei simboli già precisato.
Evidentemente 
\[\sum_a^b{}_{-}f(x) \cdot \Delta \le \sum_a^b f(x) \cdot \Delta \le 
\sum_a^b {}^{-}f(x) \cdot \Delta\] 
ed è immediato (seguendo un percorso corrispondente a quanto già fatto) che 
anche le somme di Riemann superiore e inferiore hanno le proprietà rettangolare 
e additiva. 
  Passando alle loro estensioni naturali non standard risulta ancora 
\[\sum_a^b {}_{-}f \cdot dx \le \sum_a^b f \cdot dx \le 
  \sum_a^b {}^{-}f \cdot dx\] 
e si mantengono le proprietà rettangolare e additiva. 
Inoltre la differenza tra la somma di Riemann superiore e quella inferiore è 
un infinitesimo. Infatti, per ogni reale positivo r, 
\[\sum_a^b {}^{-}f \cdot dx < \sum_a^b {}_{-}f \cdot dx +r\] 
Se per assurdo non fosse così, ci dovrebbero essere almeno un reale 
positivo \(r_0\) e un punto \(x_0 \in \intervcc{a}{b}\) tali che 
\({}^{-}f(x_0) \ge {}_{-}f(x_0)+r_0\) 
ma il minimo e il massimo di una funzione continua in un intervallo chiuso e 
di lunghezza infinitesima non possono che esser infinitamente vicini e quindi 
non distanti più di un reale, contraddizione che prova quanto si voleva 
asserire. Così 
\[\int_b^a {}_{-}f \cdot dx = \int_b^a f \cdot dx = 
  \int_b^a {}^{-}f \cdot dx\]
  
Ora si vuole mostrare che la funzione binaria che soddisfa le proprietà 
additiva e rettangolare rispetto alla funzione f continua nell'intervallo 
cui appartengono i punti a e b è unica.
Allo scopo siano \(A(a,b)\) e \(B(a,b)\) due tali funzioni. 
Sia n un qualsiasi numero naturale e si divida l'intervallo 
\(\intervcc{a}{b}\) in n parti. 
Per la proprietà 
additiva sia \(A(a,b)\) che \(B(a,b)\) sono le somme delle \(n\) 
parti in cui sono stati divisi. 
Ciascuna parte, per la proprietà rettangolare, è minore del 
rettangolo con la stessa base di quella parte e altezza il massimo della 
funzione f su quella base, e maggiore del rettangolo con la stessa base di 
quella parte e altezza il minimo della funzione f su quella base. 

Così, 
considerando le estensioni naturali agli iperreali, si ha sia:
\[\sum_a^b {}_{-}f \cdot dx \le A(a,b) \text{, } \quad
  \sum_a^b {}_{-}f \cdot dx \le B(a,b)\]
sia 
\[A(a,b) \le \sum_a^b {}^{-}f \cdot dx \text{, } \quad
  B(a,b) \le \sum_a^b {}^{-}f \cdot dx\] 
e, passando alle parti standard, si ottiene 
\[\int_b^a f \cdot dx = A(a,b) = B(a,b)\]
e l'unicità delle funzioni binarie che soddisfano sia le proprietà 1) che 2) 
è dimostrata. 

Direi che questo risultato è già di per sè di grande interesse in una scuola 
secondaria. 

Ora si vuole mostrare il teorema fondamentale del calcolo (la derivata della 
funzione integrale è la funzione integranda) sfruttando i 
metodi dell'analisi non standard, mettendo in luce la semplificazione che si 
ottiene, semplificazione che proviene già dalla più semplice e diretta 
nozione di integrale che non deve far ricorso a limiti al tender a zero della 
lunghezza massima dei sottointervalli di partizione dell'intervallo 
\(\intervcc{a}{b}\). 

Con i metodi non standard ci sono due dimostrazioni sostanzialmente diverse 
del teorema fondamentale del calcolo, una che parte dagli integrali indefiniti 
(dalla primitive) e l'altra che parte dalla derivazione la funzione integrale. 
Le vedremo entrambe perché sono entrambe semplici. 

La prima richiede di 
premettere il teorema di unicità delle funzioni che hanno le proprietà 1) e 
2) ricordate in precedenza relativamente a a una funzione continua \(f\) in 
un intervallo \(I\). 

Sia allora \(F(x)\) una funzione la cui derivata su un intervallo è 
\(f(x)\):  
\(F'(x) = f(x)\). 
È banale vedere con i metodi non standard che se anche 
\(G'(x) = f(x)\) 
su quell'intervallo allora \(F(x) - G(x) = k\) una costante. 
Infatti, la funzione 
\(F(x) - G(x)\) avrebbe derivata 0 su un intervallo e, per il teorema 
del valor medio, deve essere una costante. 

Ricordiamo che il teorema del valor medio è conseguenza del teorema
di Rolle che si appoggia sul teorema dei valori estremi per funzioni continue 
in un intervallo chiuso, e questo è facilmente dimostrabile in analisi non 
standard. 

Si consideri ora la funzione 
\(D(a,b) = F(b)-F(a)\). 
È evidente che soddisfa la proprietà additiva. Per il teorema 
del valor medio (le cui ipotesi sono rispettate), c'è c tale che 
\(a < c < b\) 
e 
\[F(b)-F(a) = F'(c)\cdot(b - a) = f(c)\cdot(b - a)\]
da cui segue la proprietà rettangolare poiché \(f(c)\) è minore del massimo 
della funzione \(f\) 
nell'intervallo e maggiore del minimo della funzione f nell'intervallo.
Dunque, 
per l'unicità della funzione che soddisfa le condizione 1) e 2) deve essere:
\[\int_a^b f \cdot dx = F(b)-F(a)\]
con \(F\) funzione che ha per derivata la funzione \(f\). 
Poiché questa uguaglianza vale qualunque sia l'estremo superiore di 
integrazione in un intervallo dove la funzione \(f\) è continua, si ha che
\[\int_a^x f \cdot dx = F(x) – F(a)\]
e la funzione \(F(x) – F(a)\) viene chiamata funzione integrale.
Così si è ottenuto quanto afferma il teorema fondamentale del
calcolo.

Un altro approccio allo stesso teorema è il seguente.
Si voglia determinare la derivata della funzione integrale che è: 
\[F(u) = \int_a^u f \cdot dx\]
usando i metodi non standard. Denotando sempre con \(F\) l'estensione 
naturale della funzione reale \(F\), si avrà
\[F'(u) = \pst{\frac{F(u+du)-F(u)}{du}} = 
\pst{\frac{\int_a^{u+du} f\cdot dx - \int_a^u f\cdot dx}{du}} =\] 
che per l'arbitrarietà dell'incremento (l'incremento d'integrazione \(dx\) 
può essere \(du\)) è uguale a
\[\pst{\frac{\int_a^{u+du} f\cdot du - \int_a^u f\cdot du}{du}} =\]
per l'additività degli integrali consideriamo che
\[\pst{\frac{\int_u^{u+du} f \cdot du}{du}} = 
    \pst{\frac{f(u) \cdot du}{du}} = \pst{f(u)} = f(u)\]
essendo \(f(u)\) un numero reale. La prima uguaglianza della riga
precedente è giustificata dal fatto che, per definizione 
\[\int_u^{u+du} f \cdot du \approx \sum_u^{u+du} f \cdot du = 
  f(u) \cdot du\]
% essendo questo un numero reale.

 Così si è visto che la derivata della funzione integrale è la funzione 
integranda, provando per altra via il teorema fondamentale del calcolo.

Volendo raggiungere l'espressione che consente di calcolare un integrale 
definito dai valori di una primitiva negli estremi di integrazione, si può 
osservare che, ricordando ancora che le primitive di una funzione continua in 
un intervallo differiscono per una costante, si può osservare che
\[F(b) = \int_a^b f\cdot dx = G(b) + k\]
dove G è una qualsiasi primitiva di f e k dipende da G. Per determinare k, si 
osservi che
\[0 = \int_a^a f\cdot dx = G(a) + k\],
sicché deve essere \(k = - G(a)\) , e si ottiene nuovamente la classica 
relazione 
tra integrale definito e primitive della funzione integranda 
\[\int_a^b f \cdot dx = G(b) - G(a)\]

 Si noti che la seconda dimostrazione non ha dovuto far ricorso all'unicità 
della funzione binaria con le proprietà 1) e 2). Inoltre la prima 
dimostrazione incorre nella difficoltà di spiegare come mai tra tutte le 
possibili funzioni d'area si sia scelta proprio quella determinata mediante 
una primitiva della funzione integranda. La seconda dimostrazione risponde 
immediatamente a questo interrogativo facendo emergere la risposta dalla 
ricerca, del tutto naturale, della derivata della funzione integrale.

Un commento: il teorema fondamentale dell'analisi non dipende dalla 
trattazione classica o non standard dell'argomento. L'integrale che abbiamo 
ottenuto, detto anche integrale definito, c'è per ogni funzione reale 
continua, e, si dimo­stra facilmente, anche per funzioni continue a tratti o 
continue su intervalli illimitati, ma seguendo la definizione non si riesce a 
trovarne il valore dovendo prendere parti standard di somme infinite. Può 
essere approssimato mediante le somme di Riemann finite, ma il lavoro 
richiesto è enorme e spesso poco fattibile anche con i più avanzati strumenti 
di calcolo automatico. 
Così la determinazione mediante le primitive può esser un ottimo modo per 
determinare il risultato. Ma anche in questo caso ci sono difficoltà. Non c'è 
un calcolo delle primitiva, data una funzione f per trovarne una primitiva 
bisogna ricordare di avere incontrato f come derivata di una certa funzione. 
I metodi d'integrazione aiutano a modificare la situazione nella speranza di 
arrivare al punto di dover trovare primitive di funzioni che si ricorda di 
aver incontrato come derivate di altre funzioni. Ci sono poi funzioni 
continue le cui primitive (che esistono proprio per il teorema fondamentale 
del calcolo) non sono esprimibili con il linguaggio che descrive le 
funzioni, a volte lo sono come limiti di successioni di funzioni. Così, non 
essendoci un calcolo (metodo automatico, con precise regole sulla scrittura 
delle funzioni, che non richiede la conoscenza dei significati) la 
determinazione delle primitive, è quasi una sfida aperta.

\section{Osservazioni}

Caro Luciano e cari amici,
       nonostante la revisione che vi ho inviato la scorsa notte, rimangono 
ancora dei punti trattati ap­prossimativamente. Ad esempio uno potrebbe 
rimanere perplesso nel leggere \(\int_u^{u+du} f(x) \cdot du\), 
doman­dandosi che ci sta a 
fare la x in questa espressione. Infatti, a rigore, non dovrebbe esserci. 
f(x) indica la quantità che si ottiene calcolando la funzione unaria f nel 
valore x della variabile indipendente, mentre la funzione è indicata dalla 
sola f. Così, se si volesse essere precisi fino in fondo si sarebbe dovuto 
scrivere \(\int_u^{u+du} f \cdot du\), e non solo in questa occasione. 
Infatti, l'integrale 
vuole rappresentare l'area limitata da una curva che è il grafico della 
funzione f e non il valore di questa funzione in un certo punto. Purtroppo, 
tradizionalmente s'indica impropriamente la funzione come f(x) poiché non si 
fa tanto riferimento alla funzione in sé ma al modo linguistico con cui viene 
descritta. Ad esempio, si usa scrivere \(f(x) = 5x^2+sin x\) per indicare la 
funzione che a un certo valore fa corrispondere 5 volte il quadrato di quel 
valore più il seno dello stesso. Con una notazione più precisa si dovrebbe 
scrivere \(f: x \mapsto 5x^2+sin x\) in cui correttamente la funzione è 
indicata con f 
e quanto segue i due punti sta a descrivere di quale funzione ci si sta 
interessando e come a un valore ne corrisponde un altro. La stessa idea è 
colta dalla seguente notazione per una funzione, 
\(f = \graffa{(x, 5x^2+sin x): x \in \R}\). 
Ancora si nota la netta separazione tra la notazione della funzione e la 
descrizione del suo comportamento. Uno potrebbe obiettare che nel descrivere 
delle funzioni a volte s'incontrano altre variabili o lettere che indicano 
parametri, e bisogna indicare quale simbolo indica la variabile indipendente 
della funzione unaria che si vuole considerare. Ad esempio vorremmo mantenere 
la scrittura \(f(x) = 5ax^2+sin xy\) in cui il riferimento alla variabile x è 
essenziale. Ma le notazioni 
\(f: x \mapsto 5ax2+sin xy\) e 
\(f=\graffa{(x, 5ax2+sin xy): x \in \R}\), 
che separano il simbolo f dalla sua definizione, sono altrettanto esplicite, 
e, se proprio si vuole enfatizzare che la funzione è una funzione della 
variabile \(x\), si può usare la notazione 
\(f_x\), sicché la scrittura \(f_x(c)\) 
indica il valore che questa funzione assume in corrispondenza del valore 
\(c\) attribuito alla variabile \(x\).
Tornando a noi, tutte le volte che si vuole indicare una funzione si sarebbe 
dovuto scrivere solo \(f\), anche nei numeri precedenti. Così, se abbiamo 
scritto 
imprecisamente \(f(x)\), ciò è dovuto al peso di una certa tradizione che in 
qualche modo giustifica la notazione usata, ma non evita che questa debba 
essere interpretata correttamente da persone ammaestrate a farlo, e che 
inoltre induca imprecisioni. 

Nel numero 207 avevamo correttamente indicato il differenziale e la derivata 
con le notazioni \(df\) e \(df / dx\). 
Avevamo anche usato le notazioni 
\(G(x) = g(f(x)) e G'(x) = g'(f(x))·f'(x)\)  
per indicare la funzione composta e la sua 
derivata. Ciò può essere accettato perché si sta cercando di descrivere la 
funzione ottenuta dai valori che assume quando le componenti assumono certi 
valori in corrispondenza di valori della variabile indipendente, anche se si 
tratta di un abuso di linguaggio e richiede l'addestramento sufficiente per 
superare questa imprecisione.

Nel numero 210, avevamo correttamente indicato le somme con la notazione 
\((\sum_{i=0}^{h-1} f(x_i)\cdot\Delta) + f(xh)\cdot(b-xh)\), 
poiché qui davvero si considerano i 
valori che la funzione \(f\) assume nei punti indicati. Ma dopo, osservando 
che 
la somma dipendeva solo da \(\Delta\) e dagli estremi dell'intervallo, oltre 
che 
dalla funzione f, si era scritto \(\sum_a^b f(x) \cdot \Delta\), e qui ci si 
è chinati 
all'imprecisione della tradizione e dal voler richiamare che il prodotto era 
tra valori specifici della funzione e \(\Delta\) (anche se i valori specifici 
non sono indicati da f(x)) invece di indicare \(\sum_a^b f\Delta\). 
Questa accettazione 
della consuetudine è proseguita coerentemente anche nella notazione 
\(\int_a^b f(x) \cdot dx = st (\sum_a^b f(x) \cdot dx)\), anche se sarebbe 
stato più esatto scrivere 
\(\int_a^b fdx = st(\sum_a^b fdx)\), e similmente in tutto quello che segue. 
Essendoci conformati alla notazione corrente, si può a buona ragione 
continuare a seguire questa notazione, sapendo quali abusi linguistici 
comporta e come sviluppare correttamente il percorso.
La difficoltà che questa scelta comporta è messa in luce proprio dalla 
difficoltà di leggere l'espressione \(\int_u^{u+du} f(x).du\) , difficoltà 
che ha motivato questo mio intervento. 
E il modo giusto di leggere è come se al 
posto di f(x) fosse scritto solo f perché quello che si sta dicendo non 
riguarda assolutamente la funzione calcolata in  un punto x (che non si sa 
neppure qual è) ma solo la funzione f. 
Che fare del nostro lavoro dopo tutte queste considerazioni? Niente, e 
incrociare le dita sperando che i lettori leggano nel modo da noi inteso. Ma 
si potrebbe anche inserire una frase al momento opportuno per chiarire che a 
volte bisognerebbe leggere la scrittura f(x) come ci fosse solo la f, poiché 
non si sta applicando la funzione a uno specifico valore x ma si sta solo 
indicando la funzione, però non vedo dove inserirla (e quindi cosa dire 
esattamente) senza appesantire l'esposizione con parentesi potenzialmente 
fuorvianti. Una terza azione potrebbe prevedere di preparare un ulteriore 
lavoro che discuta degli abusi linguistici in matematica; questa scelta 
potrebbe anche essere combinata con una delle precedenti.
  Ruggero
