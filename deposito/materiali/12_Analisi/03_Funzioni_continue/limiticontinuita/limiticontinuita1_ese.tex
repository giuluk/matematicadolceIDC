% (c) 2015 Daniele Zambelli daniele.zambelli@gmail.com

\section{Esercizi}

\subsection{Esercizi dei singoli paragrafi}

\subsubsection*{\numnameref{sec:cont_limiti}}

\begin{esercizio}\label{ese:03.1}
Calcola i seguenti limiti:
 \begin{enumeratea}
  \item \(\displaystyle \lim_{x \rightarrow 3} \tonda{x^2-4x+2}\)
  \item \(\displaystyle \lim_{x \rightarrow 1} \tonda{4x^2-5x+8}\)
  \item \(\displaystyle \lim_{x \rightarrow -1} \tonda{7x^2+3x+5}\)
  \item \(\displaystyle \lim_{x \rightarrow -2} \tonda{3x^2+2x+6}\)
 \end{enumeratea}
\end{esercizio}

\begin{esercizio}\label{ese:03.1}
Calcola i seguenti limiti:
 \begin{enumeratea}
  \item \(\displaystyle \lim_{x \rightarrow 0} \frac{x^2-4x+2}{x-1}\)
  \hfill \(-2\)
  \item \(\displaystyle \lim_{x \rightarrow } \frac{}{}\)
  \hfill \(\frac{}{}\)
  \item \(\displaystyle \lim_{x \rightarrow } \frac{}{}\)
  \hfill \(\frac{}{}\)
  \item \(\displaystyle \lim_{x \rightarrow } \frac{}{}\)
  \hfill \(\frac{}{}\)
  \item \(\displaystyle \lim_{x \rightarrow } \frac{}{}\)
  \hfill \(\frac{}{}\)
 \end{enumeratea}
\end{esercizio}

\begin{comment}
\subsection{Esercizi riepilogativi}

\begin{esercizio}
\label{ese:D.19}
testo esercizio
\end{esercizio}

\begin{esercizio}\label{ese:03.1}
Consegna:
 \begin{enumeratea}
  \item  
 \end{enumeratea}
\end{esercizio}
\end{comment}
