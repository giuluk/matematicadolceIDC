% (c) 2017 Daniele Zambelli - daniele.zambelli@gmail.com
% 
% Tutti i grafici per il capitolo relativo allo studio di funzioni
% 
% 

% x^3/(x-1)^2 . Non simmetrie, una discontinuità, un minimo, un flesso, asintoti 
% obliqui.
% Più semplice: x^4-4x^2+5. Continua, parità, due min e un max. non asintoti
% Intermedia: x^2/(x-2). Un asint. vert, 1 asint obliquo,1 max un min, niente 
% flessi

\newcommand{\funzionea}{% 
  % prima funzione.
  \disegno{
  \rcom{-10}{+10}{-7}{+13}{gray!50, very thin, step=1}
  \tkzInit[xmin=-10.3, xmax=+10.3, ymin=-7.3, ymax=+13.3]
    \tkzFct[ultra thick, color=Green!50!black, domain=-10.3:+.9]
          {x**3 / ((x-1)**2)}
    \tkzFct[ultra thick, color=Green!50!black, domain=1.1:+10.3]
          {x**3 / (x-1)**2}
  }
}

\newcommand{\segnoespressionea}{% 
  % segno della prima funzione.
  \disegno{
  \assex{-5}{+5}{0}
  \draw (0, 0.3) -- (0, 0) node [below] {\(0\)}
        (-2, 0) node [above] {\(-\)}
        (2, 0) node [above] {\(+\)};
  }
}

\newcommand{\cefunca}{% 
  % prima funzione.
  \rcom{-10}{+10}{-7}{+13}{gray!50, very thin, step=1}
  \draw [ultra thick, dotted] (1, 13.3) -- (1, -7.3);
}

\newcommand{\cefunzionea}{% 
  % prima funzione.
  \disegnod{2}{
    \cefunca
  }
}

\newcommand{\assifunca}{% 
  % prima funzione.
  \filldraw (0, 0) [red!50!black] circle (.3);
}

\newcommand{\assifunzionea}{% 
  % prima funzione.
  \disegnod{2}{
    \cefunca
    \assifunca
  }
}

\newcommand{\segnofunca}{% 
  % prima funzione.
  \foreach \ax/\ay/\bx/\by in {-10/10/-7.5/13, -10/7/-4.5/13,
                               -10/4/-1.5/13, -10/1/0/11.5,
                               -8.5/0/0/8.5, -5.5/0/0/5.5,
                               -2.5/0/0/2.5}
    \draw (\ax, \ay) -- (\bx, \by);
  \foreach \ax/\ay/\bx/\by in {2.5/0/0/-2.5, 5.5/0/0/-5.5,
                               8.5/0/1.5/-7, 10/-1.5/4.5/-7,
                               10/-4.5/7.5/-7}
    \draw (\ax, \ay) -- (\bx, \by);
}

\newcommand{\segnofunzionea}{% 
  % prima funzione.
  \disegnod{2}{
    \cefunca
    \assifunca
    \segnofunca
  }
}

\newcommand{\asintoticofunca}{% 
  % prima funzione.
  \begin{scope}[thick, green!40!black, ->]
    \draw (.25, 9) -- (.75, 12); 
    \draw (2.25, 9) -- (1.75, 12);
    \draw (5, 8) -- (8, 11);
    \draw (-5, -2) -- (-8, -5);
  \end{scope}
}

\newcommand{\asintoticofunzionea}{% 
  % prima funzione.
  \disegnod{2}{
    \cefunca
    \assifunca
    \segnofunca
    \asintoticofunca
  }
}

\newcommand{\asintotifunca}{% 
  % prima funzione.
  \begin{scope}[thick, blue!40!black]
    \draw (1, -7.3) -- (1, 13.3); 
    \draw (-9.3, -7.3) -- (10.3, 12.3);
  \end{scope}
}

\newcommand{\asintotifunzionea}{% 
  % prima funzione.
  \disegnod{2}{
    \cefunca
    \assifunca
    \segnofunca
    \asintoticofunca
    \asintotifunca
  }
}

\newcommand{\stazionarifunca}{% 
  % prima funzione.
  \begin{scope}[thick]
    \draw (-.5, 0) -- (+.5, 0);
    \filldraw (0, 0) circle (.2);
    \draw (2.5, 6.75) -- (3.5, 6.75);
    \filldraw (3, 6.75) circle (.2);
  \end{scope}
}

\newcommand{\stazionarifunzionea}{% 
  % prima funzione.
  \disegnod{2}{
    \cefunca
    \assifunca
    \segnofunca
    \asintoticofunca
    \asintotifunca
    \stazionarifunca
  }
}

\newcommand{\segnotrinomioa}{% 
  % segno della prima funzione.
  \disegno{
  \assex{-2}{+6}{0}
  \draw (1, 0.3) -- (1, 0) node [below] {\(1\)};
  \draw (3, 0.3) -- (3, 0) node [below] {\(3\)}
        (0, 0) node [above] {\(+\)}
        (2, 0) node [above] {\(-\)}
        (4, 0) node [above] {\(+\)};
  }
}

\newcommand{\segnoderivataa}{% 
  % segno della prima funzione.
  \disegno{
  \assex{-4}{+7}{0}
  \draw (0, 0.3) -- (0, 0) node [below] {\(0\)};
  \draw (1, 0.3) -- (1, 0) node [below] {\(1\)};
  \draw (3, 0.3) -- (3, 0) node [below] {\(3\)};
  \draw (-2, 0) node [above] {\(+\)}
        (.5, 0) node [above] {\(+\)}
        (2, 0) node [above] {\(-\)}
        (5, 0) node [above] {\(+\)};
  \begin{scope}[thick, ->]
    \draw (-3, -2) -- (-.2, -1); 
    \draw (0, -2) -- (1, -1);
    \draw (1.2, -1) -- (2.8, -2);
    \draw (3.2, -2) -- (6, -1);
  \end{scope}
  }
}

\newcommand{\segnoderivatasecondaa}{% 
  % segno della prima funzione.
  \disegnod{10}{
  \assex{-3}{+4}{0}
  \draw (0, 0.15) -- (0, 0) node [below] {\(0\)};
  \draw (1, 0.15) -- (1, 0) node [below] {\(1\)};
  \draw (-1.5, 0) node [above] {\(-\)}
        (.5, 0) node [above] {\(+\)}
        (2.5, 0) node [above] {\(+\)};
  \begin{scope}[thick]
    \draw (-2.8, -1) .. controls (-2, 0) and (-1, 0) .. (-.2, -1);
    \draw (.2, -.5) .. controls (.4, -1) and (.6, -1) .. (.8, -.5);
    \draw (1.2, -.25) .. controls (2, -1.25) and (3, -1.25) .. (3.8, -.25);
  \end{scope}
  }
}

\newcommand{\funca}{% 
  % prima funzione.
  \tkzInit[xmin=-10.3, xmax=+10.3, ymin=-7.3, ymax=+13.3]
    \tkzFct[ultra thick, color=Green!50!black, domain=-10.3:+.9]
          {x**3 / ((x-1)**2)}
    \tkzFct[ultra thick, color=Green!50!black, domain=1.1:+10.3]
          {x**3 / (x-1)**2}
}

\newcommand{\tuttoassiemea}{% 
  % prima funzione.
  \disegnod{3}{
    \cefunca
    \assifunca
    \segnofunca
    \asintoticofunca
    \asintotifunca
    \stazionarifunca
    \funca
  }
}

\newcommand{\funzioneb}{% 
    \disegno{
    \rcom{-5}{+5}{-2}{+10}{gray!50, very thin, step=1}
    \tkzInit[xmin=-5.3, xmax=+5.3, ymin=-2.3, ymax=+10.3]
     \tkzFct[ultra thick, color=Green!50!black, domain=-5.3:+5.3]
            {x**4-4*x**2+3}
    }
}

\newcommand{\funzionec}{% 
    \disegno{
    \rcom{-10}{+10}{-5}{+15}{gray!50, very thin, step=1}
    \tkzInit[xmin=-10.3, xmax=+10.3, ymin=-5.3, ymax=+15.3]
     \tkzFct[ultra thick, color=Green!50!black, domain=-10.3:-1.1]
            {(x**2+12*x+20)/(2*x+2)}   % x**2/(2*x-6)
     \tkzFct[ultra thick, color=Green!50!black, domain=-0.9:+10.3]
            {(x**2+12*x+20)/(2*x+2)}
    }
}

% Esercizi: descrivi i seguenti grafici

\def \uni{3.7}
\newcommand{\grafesea}{%  
  \def \func {(-2*x-5)/(x**2-x-6)}
    \disegnod{\uni}{
    \rcom{-6}{+6}{-6}{+6}{gray!50, very thin, step=1}
    \tkzInit[xmin=-6.3, xmax=+6.3, ymin=-6.3, ymax=+6.3]
     \tkzFct[ultra thick, color=Green!50!black, domain=-6.3:-2.01]
            {\func}
     \tkzFct[ultra thick, color=Green!50!black, domain=-1.99:+2.9]
            {\func}
     \tkzFct[ultra thick, color=Green!50!black, domain=+3.1:6.3]
            {\func}
    }
}

\newcommand{\grafeseb}{%  
  \def \func {(x**2-x-5)/(-2*x**2-x-1)}
    \disegnod{\uni}{
    \rcom{-6}{+6}{-2}{+10}{gray!50, very thin, step=1}
    \tkzInit[xmin=-6.3, xmax=+6.3, ymin=-2.3, ymax=+10.3]
     \tkzFct[ultra thick, color=Green!50!black, domain=-6.3:+6.3]
            {\func}
    }
}

\newcommand{\grafesec}{% 
  \def \func {(-4*x**2-2*x+2)/(-x**2+3*x-4)}
    \disegnod{\uni}{
    \rcom{-6}{+6}{-2}{+10}{gray!50, very thin, step=1}
    \tkzInit[xmin=-6.3, xmax=+6.3, ymin=-2.3, ymax=+10.3]
     \tkzFct[ultra thick, color=Green!50!black, domain=-6.3:+6.3]
            {\func}
    }
}


\newcommand{\grafesed}{%  
  \def \func {(x**2+2*x-4)/(-x**2-4*x-3)}
    \disegnod{\uni}{
    \rcom{-8}{+4}{-6}{+6}{gray!50, very thin, step=1}
    \tkzInit[xmin=-8.3, xmax=+4.3, ymin=-6.3, ymax=+6.3]
     \tkzFct[ultra thick, color=Green!50!black, domain=-8.3:-3.01]
            {\func}
     \tkzFct[ultra thick, color=Green!50!black, domain=-2.99:-1.01]
            {\func}
     \tkzFct[ultra thick, color=Green!50!black, domain=-.99:4.3]
            {\func}
    }
}

\newcommand{\grafesee}{% 
  \def \func {(x**2 -6*x +5)/(x**2 -9)}
    \disegnod{\uni}{
    \rcom{-6}{+6}{-6}{+6}{gray!50, very thin, step=1}
    \tkzInit[xmin=-6.3, xmax=+6.3, ymin=-6.3, ymax=+6.3]
     \tkzFct[ultra thick, color=Green!50!black, domain=-6.3:-3.1]
            {\func}
     \tkzFct[ultra thick, color=Green!50!black, domain=-2.9:+2.9]
            {\func}
     \tkzFct[ultra thick, color=Green!50!black, domain=+3.1:6.3]
            {\func}
    }
}

\newcommand{\grafesef}{%  
  \def \func {(2*x**2+5*x+4)/(3*x**2-x-2)}
    \disegnod{\uni}{
    \rcom{-6}{+6}{-6}{+6}{gray!50, very thin, step=1}
    \tkzInit[xmin=-6.3, xmax=+6.3, ymin=-6.3, ymax=+6.3]
     \tkzFct[ultra thick, color=Green!50!black, domain=-6.3:-0.67]
            {\func}
     \tkzFct[ultra thick, color=Green!50!black, domain=-0.66:+0.99]
            {\func}
     \tkzFct[ultra thick, color=Green!50!black, domain=+1.01:6.3]
            {\func}
    }
}

\newcommand{\grafeseg}{% 
  \def \func {(4*x**2 -6*x +5)/(x**2 +6)}
    \disegnod{\uni}{
    \rcom{-6}{+6}{-6}{+6}{gray!50, very thin, step=1}
    \tkzInit[xmin=-6.3, xmax=+6.3, ymin=-6.3, ymax=+6.3]
     \tkzFct[ultra thick, color=Green!50!black, domain=-6.3:-0.67]
            {\func}
     \tkzFct[ultra thick, color=Green!50!black, domain=-0.66:+0.99]
            {\func}
     \tkzFct[ultra thick, color=Green!50!black, domain=+1.01:6.3]
            {\func}
    }
}

\newcommand{\grafeseh}{%  
  \def \func {(x**3 +1)/(x**2 -3*x -10)}
    \disegnod{\uni}{
    \rcom{-6}{+6}{-6}{+6}{gray!50, very thin, step=1}
    \tkzInit[xmin=-6.3, xmax=+6.3, ymin=-6.3, ymax=+6.3]
     \tkzFct[ultra thick, color=Green!50!black, domain=-6.3:-2.1]
            {\func}
     \tkzFct[ultra thick, color=Green!50!black, domain=-1.9:+4.9]
            {\func}
     \tkzFct[ultra thick, color=Green!50!black, domain=+5.1:6.3]
            {\func}
    }
}

\newcommand{\grafesei}{%  
  \def \func {(4*x**2 -6*x +5)/(x +5)}
    \disegnod{\uni}{
    \rcom{-6}{+6}{-6}{+6}{gray!50, very thin, step=1}
    \tkzInit[xmin=-6.3, xmax=+6.3, ymin=-6.3, ymax=+6.3]
     \tkzFct[ultra thick, color=Green!50!black, domain=-6.3:-5.1]
            {\func}
     \tkzFct[ultra thick, color=Green!50!black, domain=-4.9:+6.3]
            {\func}
    }
}

\newcommand{\grafesej}{%  
  \def \func {(3*x**2 -3)/(-2*x**2 +4*x +6)}
    \disegnod{\uni}{
    \rcom{-6}{+6}{-6}{+6}{gray!50, very thin, step=1}
    \tkzInit[xmin=-6.3, xmax=+6.3, ymin=-6.3, ymax=+6.3]
     \tkzFct[ultra thick, color=Green!50!black, domain=-6.3:+2.9]
            {\func}
     \tkzFct[ultra thick, color=Green!50!black, domain=3.1:+6.3]
            {\func}
    }
}

