% (c) 2015 Daniele Zambelli daniele.zambelli@gmail.com

\section{Esercizi}

% \subsection{Esercizi dei singoli paragrafi}
% 
% \subsubsection*{\numnameref{sec:01_}}

\begin{esercizio}\label{ese:03.1}
Descrivi a parole il grafico poi analizza la seguente funzione.

\begin{multicols}{2}
 \begin{enumeratea}
  \item 
\(y=x^4-4x^2+3\)
 \begin{center}
 \scalebox{.8}{\funzioneb}
 \end{center}
  \item 
\(y=\dfrac{x^2+12x+20}{2x+2}\)
 \begin{center}
 \scalebox{.5}{\funzionec}
 \end{center}
 \end{enumeratea}
\end{multicols}
\end{esercizio}

% \begin{esercizio} \label{ese:}
% Descrivi a parole il grafico poi analizza la seguente funzione.
% 
% \begin{minipage}{.19\linewidth}
% \[y=x^4-4x^2+3\]
% \vspace{40mm}
% \end{minipage}
% \hfill
% \begin{minipage}{.79\linewidth}
%  \begin{center}
%  \scalebox{.6}{\funzioneb}
%  \end{center}
% \end{minipage}
% \end{esercizio}
% 
% % \vspace{-30mm}
% 
% \begin{esercizio} \label{ese:}
%  Descrivi a parole il grafico poi analizza la seguente funzione.
%  
% \begin{minipage}{.19\linewidth}
%  \begin{center}
% \[y=\frac{x^2+12x+20}{2x+2}\]
%  \end{center}
% \end{minipage}
% \hfill
% \begin{minipage}{.79\linewidth}
%  \begin{center}
%  \scalebox{.6}{\funzionec}
%  \end{center}
% \end{minipage}
% \end{esercizio}

\begin{esercizio}\label{ese:03.1}
Descrivi i seguenti grafici:

\begin{multicols}{2}
 \begin{enumeratea}
  \item \grafesea
  \item \grafeseb 
  \item \grafesec
  \item \grafesed 
  \item \grafesee
  \item \grafesef
  \item \grafeseg
  \item \grafeseh
  \item \grafesei
  \item \grafesej
 \end{enumeratea}
\end{multicols}
\end{esercizio}

\begin{esercizio}\label{ese:03.1}
Analizza le seguenti funzioni:
\begin{multicols}{2}
 \begin{enumeratea}
  \item \(\dfrac{-2x-5}{x^2-x-6}\)
  \item \(\dfrac{2x^2+5x+4}{3x^2-x-2}\) 
  \item \(\dfrac{x^2-x-5}{-2x^2-x-1}\) 
  \item \(\dfrac{3x^2 -3}{-2x^2 +4x +6}\) 
  \item \(\dfrac{4x^2 -6x +5}{x^2 +6}\) 
  \item \(\dfrac{-4x^2-2x+2}{-x^2+3x-4}\) 
  \item \(\dfrac{x^3 +1}{x^2 -3x -10}\) 
  \item \(\dfrac{4x^2 -6x +5}{x +5}\) 
  \item \(\dfrac{x^2 -6x +5}{x^2 -9}\) 
  \item \(\dfrac{x^2+2x-4}{-x^2-4x-3}\) 
 \end{enumeratea}
\end{multicols}
\end{esercizio}

\begin{comment}
 
\begin{esercizio}
\label{ese:}
\end{esercizio}

\end{comment}
