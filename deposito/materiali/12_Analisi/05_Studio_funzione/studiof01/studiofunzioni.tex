% (c) 2015 Daniele Zambelli daniele.zambelli@gmail.com

% (c) 2014 Daniele Zambelli - daniele.zambelli@gmail.com
% 
% Tutti i grafici per il capitolo relativo alle parabole
%
% 

\newcommand{\espdueterzi}{% 
    % Esponenziali con basi diverse.
    \disegno{
    \rcom{-10}{+10}{-1}{10}{gray!50, very thin, step=1}
    \begin{scope}[ultra thick, color=Maroon!50!black]
     \tkzInit[xmin=-10.3, xmax=+10.3, ymin=-0.3, ymax=+10.3]
     \tkzFct[domain=-10.3:+6]{(3./2)**x}
     \tkzFct[color=Green!50!black, domain=-6:+10.3]{(2./3)**x}
     \begin{scope}[color=Black!50!black]
      \filldraw (1, 3./2) circle (1.2pt);
      \filldraw (1, 2./3) circle (1.2pt);
     \end{scope}
     \filldraw [color=Red](0, 1) circle (1.2pt);  
    \end{scope}
    \begin{scope}[color=black]
     \draw (-7.3, 7) node{\(f(x)=\tonda{\dfrac{2}{3}}^x\)}; 
     \draw ((7.3, 7) node{\(f(x)=\tonda{\dfrac{3}{2}}^x\)};
    \end{scope}
    }
}

\newcommand{\logduebasi}{% 
    % Esponenziali con basi diverse.
    \disegno{
    \rcom{-1}{+10}{-9}{9}{gray!50, very thin, step=1}
    \begin{scope}[ultra thick, color=Maroon!50!black]
      \tkzInit[xmin=-1.3, xmax=+80, xstep=.5, ymin=-10.3,ymax=+10.3]
      \tkzFct[domain=.01:+10]{log(x)/log(2)}
      \filldraw (2, 1) circle (1.2pt);
      \begin{scope}[color=Green!50!black]
        \tkzFct[domain=-.01:+10]{log(x)/log(1./2)}
        \filldraw (2, -1) circle (1.2pt);
      \end{scope}
    \end{scope}
    \begin{scope}[color=black]
      \draw (9.5, 2.8) node{a=2}; 
      \draw (9.5, -2.8) node{a=0.5};
    \end{scope}
      \filldraw [color=Red] (1,0) circle (1.2pt);
    }
}


% \begin{wrapfloat}{figure}{r}{0pt}
% \includegraphics[scale=0.35]{img/fig000_.png}
% \caption{...}
% \label{fig:...}
% \end{wrapfloat}
% 
% \begin{center} \input{\folder lbr/fig000_.pgf} \end{center}

\chapter{Studio di funzioni}

In generale rappresentiamo una funzione con un'espressione simile 
a:~\(y0f(x)\) dove \(y\) è il risultato dell'espressione \(f(x)\)
che contiene la variabile \(x\). 

\(y\) viene anche detta variabile dipendente e \(x\) variabile indipendente.

Una funzione può anche essere rappresentata su un piano cartesiano da un 
grafico. 
Il grafico di una funzione avrà la particolarità di intersecare ogni retta 
parallela all'asse \(y\) al massimo in un punto.

Scopo di questo capitolo è descrivere il comportamento di una funzione.

\section{Descrizione del grafico}
\label{sec:descrizione_grafico}
Iniziamo da un esempio non troppo banale. 
Consideriamo la seguente funzione che è rappresentata sia come espressione 
matematica che come grafico.

\begin{minipage}{.20\linewidth}
 \begin{center}
\[y=\frac{x^3}{(x-1)^2}\]
 \end{center}
% \caption{Funzione~\(y=f(x)\).}
\end{minipage}
\hfill
\begin{minipage}{.78\linewidth}
 \begin{center}
\funzionea
 \end{center}
% \caption{Grafico della funzione.} \label{fig:funzionea}
\end{minipage}

Per ora ci fidiamo che quello di destra è proprio il grafico corrispondente 
all'espressione scritta a sinistra.

La descrizione:

``Sono due linee che vanno su e giù nel piano cartesiano.''

è un po' troppo generica e potrebbe descrivere una grande quantità di grafici.

Di seguito vediamo quante informazioni possiamo ricavare dal grafico.

\subsection{Descrizione a parole}

Ogni volta che dovrai descrivere una funzione tieni conto dei seguenti punti:

% \begin{description}
%  \item [Le prime caratteristiche\\]
 
 \begin{enumerate} [nosep]
  \item \emph{Prime caratteristiche}
  \begin{enumerate} [nosep]
  \item \emph{Campo di esistenza}\\
  È definita per ogni valore di \(x\) tranne che nella zona 
attorno a \(+1\).
  \item \emph{Continuità}\\
  È continua su tutto \(\R\) tranne che attorno a \(+1\).
  \item \emph{Intersezioni con gli assi}\\
  Interseca l'asse \(y\) nel punto zero e l'asse \(x\) solo nel 
punto zero.
  \item \emph{Segno}\\
  È negativa quando \(x<0\) e positiva quando \(x>0\).
  \item \emph{Simmetrie}\\
  Non è simmetrica né rispetto all'asse \(y\) né rispetto all'origine.
  \end{enumerate}
  
  \item \emph{Comportamento agli estremi del campo di esistenza}
  \begin{enumerate} [nosep]
   \item Quando \(x\) è un infinito negativo anche \(y\) è un infinito 
negativo.
   \item Quando \(x\) è infinitamente vicino a~\(1\), da entrambi i lati, 
\(y\) è un infinito positivo.
   \item Quando \(x\) è un infinito positivo anche \(y\) è un infinito 
positivo.
  \end{enumerate}
  
  \item \emph{Asintoti}
  \begin{enumerate} [nosep]
  \item \emph{Asintoti verticali}\\
  Vicino a~\(1\) c'è un asintoto verticale.
  \item \emph{Asintoti orizzontali}\\
  Non ci sono asintoti orizzontali.
  \item \emph{Asintoti obliqui}\\
  Potrebbe esserci un asintoto obliquo.
  \end{enumerate}
  
  \item \emph{Punti stazionari}
  \begin{enumerate} [nosep]
  \item In \(\punto{0}{0}\) c'è un flesso orizzontale.
  \item Quando \(x\) vale circa~3 c'è un punto di minimo.
  \end{enumerate}
 
  \item \emph{Andamento}
  \begin{enumerate} [nosep]
  \item È crescente fino a~1.
  \item È decrescente da~1 a~3.
  \item È crescente da~3 in poi.
  \end{enumerate}
  
  
  \item \emph{Concavità}
  \begin{enumerate} [nosep]
  \item Fino a~0 ha la concavità verso il basso.
  \item Da~0 a~1 ha la concavità verso l'alto.
  \item Da~1 in poi ha la concavità verso l'alto.
  \end{enumerate}
  
  \item \emph{Insieme immagine}\\
  L'insieme immagine è tutto \(\R\) perché ogni valore di \(y\) è immagine di 
almeno un valore di \(x\).
  
 \end{enumerate}

\osservazione{
Abbiamo tratto una serie di informazioni da un grafico limitato lavorando 
molto di fantasia: chi ci dice che allontanandomi molto a sinistra o a destra 
il comportamento della funzione non sia completamente diverso da quello che 
appare nel piccolo spazio visualizzato? Potrebbero esserci degli 
intervalli in cui non è definita, oppure in cui inverte la pendenza, ...

Cosa possiamo dire della funzione nell'intervallo attorno a \(x=1\) dove non 
è visualizzata? Potrebbe non essere definita in un intervallo o solo in un 
punto o avere dei valori molto distanti da quelli visualizzati nel nostro 
piccolo disegno.

E siamo sicuri che un ingrandimento in un punto qualsiasi non potrebbe 
rivelare un comportamento imprevedibile? Potrebbero esserci dei buchi, delle 
oscillazioni, ...
}

\section{Analisi della funzione}
\label{sec:analisi_della_funzione}

L'analisi dell'espressione matematica della funzione può darci informazioni 
più precise e più sicure di quelle ricavate osservando il grafico.Riprendiamo 
quindi l'espressione matematica della funzione e andiamo a studiarne le sue 
proprietà:
\[y=\frac{x^3}{(x-1)^2}\]

\osservazione{Man mano che procediamo con l'analisi, riportiamo su un 
grafico le informazioni ottenute in questo modo possiamo effettuare un 
controllo di coerenza dei risultati e, alla fine, avremo la possibilità di 
disegnare il grafico con una buona precisione.} 

\subsection{Le prime caratteristiche}
\label{subsec:prime_caratteristiche}
\mbox{ }

\begin{minipage}{.60\linewidth}

\subsubsection{Campo di esistenza}
% \label{subsubsec:01_}
Nella funzione è presente una sola operazione critica: la divisione. La 
divisione non dà risultato se il divisore è uguale a zero quindi questa 
funzione non è definita se:
\[(x-1)^2 = 0 \sRarrow x=1\]
Questo significa che il campo di esistenza della funzione è:
\[\text{CE} = \R - {1}\]
Cioè la funzione è definita per qualunque valore di \(x\) tranne che per 
\(x=1\).

\subsubsection{Continuità}
% \label{subsubsec:01_}
Dato che la funzione è data dalla combinazione di tutte funzioni continue 
tranne la divisione, sarà continua in tutto~\(\R\) tranne che in~1 
dove questa divisione, e quindi la funzione, non è definita.

\end{minipage}
\hfill
\begin{minipage}{.38\linewidth}
 \begin{center}
\cefunzionea
 \end{center}
% \caption{Grafico della funzione.} \label{fig:funzionea}
\end{minipage}


\subsubsection{Intersezioni con gli assi}
% \label{subsubsec:01_}
 
\mbox{ }

\begin{minipage}{.60\linewidth}
\paragraph{Intersezione con l'asse y}
Dato che l'asse y ha equazione \(x=0\) l'intersezione può essere trovata 
risolvendo il sistema:
\[\sistema{y=\dfrac{x^3}{(x-1)^2} \\ x=0} \sRarrow 
           y=\dfrac{0^3}{(0-1)^2} \sRarrow 
           y=\dfrac{0}{1} \sRarrow y=0\]

Quando x vale zero, anche y vale zero: la funzione passa per l'origine degli 
assi.
\paragraph{Intersezione con l'asse x}

Dato che l'asse x ha equazione \(y=0\) l'intersezione può essere trovata 
risolvendo il sistema:
\[\sistema{y=\dfrac{x^3}{(x-1)^2} \\ y=0} \sRarrow 
           \dfrac{x^3}{(x-1)^2}=0 \sRarrow x^3=0 \sRarrow x=0\]

Ci ricordiamo infatti che una frazione vale zero solo quando è nullo il suo 
numeratore.
Quindi y vale zero solo quando anche x vale zero.

\end{minipage}
\hfill
\begin{minipage}{.38\linewidth}
 \begin{center}
\assifunzionea
 \end{center}
% \caption{Grafico della funzione.} \label{fig:funzionea}
\end{minipage}

\begin{minipage}{.60\linewidth}
\subsubsection{Segno}
% \label{subsubsec:01_}

Per studiare il segno di una funzione dobbiamo studiare il segno del 
numeratore del denominatore e calcolare poi il segno della funzione usando la 
regola dei segni della divisione. Nel nostro caso, però, possiamo osservare 
che il denominatore non è mai negativo, essendo il quadrato di una funzione, 
quindi il segno della frazione è uguale al segno del numeratore:
\begin{center}
 \segnoespressionea
\end{center}

Quindi l'intera funzione è negativa quando x è negativo e positiva quando x è 
positivo. Nel piano cartesiano cancelliamo le superfici in cui non è presente 
la funzione.

\end{minipage}
\hfill
\begin{minipage}{.38\linewidth}
 \begin{center}
\segnofunzionea
 \end{center}
% \caption{Grafico della funzione.} \label{fig:funzionea}
\end{minipage}

\subsubsection{Simmetrie}
% \label{subsubsec:01_}

Il campo di esistenza non è simmetrico rispetto all'origine questo basta per
dire che anche la funzione non potrà avere alcuna simmetria rispetto 
all'origine, non sarà né pari né dispari.

Viceversa, se il campo di esistenza fosse simmetrico, questo non basta per 
dire che la funzione sia simmetrica rispetto all'asse di simmetria 
(funzione pari) o rispetto all'origine (funzione dispari).

Per stabilire questo, bisogna sostituire nell'espressione \(x\) con \(-x\) e 
verificare quale di queste tre situazioni si ottiene:

\begin{enumerate} [nosep]
 \item \(f(-x)=f(x)\): funzione pari (simmetrica rispetto all'asse \(y\);
 \item \(f(-x)=-f(x)\): funzione dispari (simmetrica rispetto all'origine;
 \item \emph{altrimenti}: nessuna simmetria.
\end{enumerate}

Nel nostro caso: 
\[f(-x)=\frac{(-x)^3}{((-x)-1)^2} = -\frac{(x)^3}{(-x-1)^2}\]
Che è diverso sia da 
\(\frac{x^3}{(x-1)^2}\)
sia da
\(-\frac{x^3}{(x-1)^2}\)


\section{Comportamento asintotico}
\label{sec:comportamentoasintotico}

In questa parte della ricerca vogliamo scoprire come si comporta la funzione 
quando si avvicina ai punti dove non è definita.

\subsection{Comportamento agli estremi del campo di esistenza}
% \label{sec:}

\mbox{ }

\begin{minipage}{.60\linewidth}
Nel primo punto abbiamo visto che il campo di esistenza è composto da due 
intervalli: \(\intervaa{-\infty}{1}\) e \(\intervaa{1}{+\infty}\)quindi 
dobbiamo studiare come si comporta la funzione quando:

\begin{enumerate} [nosep]
 \item \(x\) è un infinito, \(x=M\):
 \[y=\frac{M^3}{(M-1)^2} \sim \frac{M^3}{M^2} = M\]
 quindi se \(x=M\) anche \(y=M\) questo significa che quando x è un infinito 
negativo anche y lo è e quando x è un infinito positivo, anche y lo è.
 \item \(x\) è infinitamente vicino a 1, \(x=1+\epsilon\):\
\[y=\frac{(1+\epsilon)^3}{((1+\epsilon)-1)^2}=
    \frac{(1+\epsilon)^3}{\epsilon^2}= M > 0\]
 quindi se \(x \sim 1\) allora \(y=M>0\) .
\end{enumerate}
\end{minipage}
\hfill
\begin{minipage}{.38\linewidth}
 \begin{center}
\asintoticofunzionea
 \end{center}
\end{minipage}

\subsection{Asintoti}
% \label{sec:02_}

Possiamo concludere che la nostra funzione:

\begin{enumerate} [nosep]
 \item ha come asintoto verticale la retta di equazione: \(x=1\);
 \item non ha asintoti orizzontali;
 \item potrebbe avere asintoti obliqui.
\end{enumerate}

\subsection{Asintoti obliqui}
% \label{sec:02_}

Una funzione ha asintoto obliquo se \(dfrac{f(M)}{M}\) è un numero finito. In 
questo caso il precedente rapporto è proprio il coefficiente angolare 
dell'asintoto. Calcoliamo il rapporto nel caso della nostra funzione:
\[m=\frac{M^3}{M(M-1)^2} = \frac{M^3}{M^3 -2M^2 +M} \sim \frac{M^3}{M^3} = 1\]
Calcolato il coefficiente angolare dobbiamo trovare l'intercetta.

\begin{minipage}{.60\linewidth}
Dall'equazione generica della retta, \(y=mx+q\) esplicitiamo il valore 
dell'intercetta: \(q=y-mx\). 
Ma \(y\) è il valore della nostra funzione, quindi: \(q=f(x)-mx\). 
E con quale valore di \(x\) dobbiamo eseguire questo calcolo? 
Trattandosi di asintoti ci serve un valore infinito:
\begin{align*}
 q &= f(M)-mM = \frac{M^3}{(M-1)^2}-M =\\
   &= \frac{M^3-M^3 +2M^2 -M}{M^2 -2M +1} \sim \frac{+2M^2}{M^2} = 2
\end{align*}

Il nostro asintoto è dunque: \(y=x+2\).
Disegniamolo nel piano cartesiano.
\end{minipage}
\hfill
\begin{minipage}{.38\linewidth}
 \begin{center}
\asintotifunzionea
 \end{center}
\end{minipage}

\section{Andamento}
\label{sec:03_andamento}

Ora vogliamo sapere dove la funzione è crescente, dove è decrescente e dove 
non è né crescente né decrescente. Partiamo da quest'ultimo punto.

\subsection{Punti stazionari}
% \label{sec:03_}

Un punto stazionario è un punto in cui la tangente alla funzione è 
orizzontale, cioè dove la derivata è nulla: \(f'(x)=0\).

Quindi per prima cosa calcoliamo la funzione derivata:

\begin{align*}
f'(x) &= \frac{3x^2(x-1)^2-x^3(2(x-1)\cdot 1)}{(x-1)^4} =\\
      &= \frac{3x^2(x^2-2x+1)-x^3(2x-2)}{(x-1)^4} =\\
      &= \frac{3x^4-6x^3+3x^2-2x^4+2x^3}{(x-1)^4} =\\
      &= \frac{x^4-4x^3+3x^2}{(x-1)^4} =\\
      &= \frac{x^2(x^2-4x+3}{(x-1)^4} =\\
      &= \frac{x^2(x-3)(x-1)}{(x-1)^4} =\\
      &= \frac{x^2(x-3)}{(x-1)^3}
\end{align*}
\begin{minipage}{.60\linewidth}
Come è immediato osservare questa funzione ha tre zeri:
\[\frac{x^2(x-3)(x-1)}{(x-1)^4}=0 \sRarrow x_{1,2}=0,~x_3=3\]
Uno zero doppio in~0 e uno in~3.
Quindi abbiamo due punti stazionari:
\[\punto{0}{0} \text{ e } \punto{3} {\frac{27}{4}}\]
Riportiamo anche questi nel grafico.
\end{minipage}
\hfill
\begin{minipage}{.38\linewidth}
 \begin{center}
\stazionarifunzionea
 \end{center}
\end{minipage}

\subsection{Intervalli di monotonia}
% \label{sec:03_}
Il segno della derivata ci permette di trovare quando la funzione è crescente 
e quando decrescente. Lo studio del segno risulta più semplice se partiamo 
dalla derivata scritta in questo modo:
\[f'(x) = \frac{x^2(x^2-4x+3)}{(x-1)^4}\]
perché abbiamo alcuni fattori di grado pari che non potranno essere negativi.

Il segno di questa espressione corrisponde al segno del trinomio:

\begin{minipage}{.49\linewidth}
\[x^2-4x+3\]
\end{minipage}
\hfill
\begin{minipage}{.49\linewidth}
 \begin{center}
\segnotrinomioa
 \end{center}
\end{minipage}

E riportando anche gli altri zeri del numeratore e del denominatore della 
derivata, si possono ottenere i seguenti intervalli in cui la funzione cresce 
o decresce:
\begin{center}
 \segnoderivataa
\end{center}

Osservando l'andamento della funzione possiamo osservare che in~0 c'è un 
flesso orizzontale e in~3 c'è un punto di minimo.

\begin{comment}

\begin{minipage}{.60\linewidth}
\end{minipage}
\hfill
\begin{minipage}{.38\linewidth}
 \begin{center}
\segnofunzionea
 \end{center}
\end{minipage}

\end{comment}

\section{Concavità}
\label{sec:04_concavita}
Il calcolo della derivata seconda ci permette di ricavare gli intervalli 
della funzione in cui la concavità è rivolta verso l'alto e quelli in cui è 
rivolta verso il basso. Vogliamo derivare la funzione derivata:
\(f'(x)=\frac{x^2(x-3)}{(x-1)^3}=\frac{x^3-3x^2)}{(x-1)^3}\)

\begin{align*}
f''(x) &= \frac{(3x^2 -6x)(x^3-3x^2+3x-1)-(x^3-3x^2)(3(x-1)^2)}{(x-1)^6} =\\
       &= \frac{3x((x -2)(x^3-3x^2+3x-1)-(x^2-3x)(x^2-2x+1))}{(x-1)^6} =\\
       &= \frac{3x(x^4-3x^3+3x^2-x-2x^3+6x^2-6x+2-x^4+2x^3-x^2+3x^3-6x^2+3x)}
               {(x-1)^6} =\\
       &= \frac{3x(2x^2-4x+2)}{(x-1)^6} = \frac{6x(x^2-2x+1)}{(x-1)^6} =
          \frac{6x(x-1)^2}{(x-1)^6} = \frac{6x}{(x-1)^4}
\end{align*}
Da cui possiamo ricavare la concavità:
\begin{center}
 \segnoderivatasecondaa
\end{center}

A questo punto abbiamo tutti gli elementi per disegnare il grafico della 
funzione con una buona approssimazione.

\section{Altre caratteristiche}
\label{sec:altre_caratteristche}
Osservando il grafico possiamo ricavare alcune altre caratteristiche della 
funzione.

\begin{itemize} [nosep]
 \item \emph{Suriettiva}: l'insieme immagine è tutto \(\R\) infatti ogni 
valore di~y è immagine di almeno un valore di~x.
 \item \emph{Non iniettiva}: infatti alcuni valori di~y sono immagini di più 
valori~x.
\end{itemize}

% \newpage

\begin{esempio}
 Descrivi a parole il grafico poi analizza la seguente funzione.
% \vspace{-12pt}
\begin{minipage}{.20\linewidth}
 \begin{center}
\[y=x^4-4x^2+3\]
 \end{center}
% \caption{Funzione~\(y=f(x)\).}
\end{minipage}
\hfill
\begin{minipage}{.80\linewidth}
 \begin{center}
\funzioneb
 \end{center}
% \caption{Grafico della funzione.} \label{fig:funzionea}
\end{minipage}

\end{esempio}

\newpage

\begin{esempio}
 Descrivi a parole il grafico poi analizza la seguente funzione.
 
\begin{minipage}{.20\linewidth}
 \begin{center}
\[y=\frac{x^2+12x+20}{2x+2}\]
 \end{center}
% \caption{Funzione~\(y=f(x)\).}
\end{minipage}
\hfill
\begin{minipage}{.80\linewidth}
 \begin{center}
\funzionec
 \end{center}
% \caption{Grafico della funzione.} \label{fig:funzionea}
\end{minipage}

\end{esempio}

