% (c) 2015 Daniele Zambelli daniele.zambelli@gmail.com

\section{Esercizi}

\subsection{Esercizi dei singoli paragrafi}

\subsubsection*{\numnameref{sec:diff01_differenziale}}
\begin{esercizio}
\label{ese:dif01}
 Calcola il differenziale della variabile $x$ nel punto $x=2$, 
$x=\dfrac{1}{5}$, $x=-\dfrac{2}{3}$ \dots
\begin{multicols}{2}
\begin{enumeratea}
  \item $dx|_{x=2}$
 \item $dx|_{x=\frac{1}{5}}$
 \item $dx|_{x=-\frac{2}{3}}$
 \item $dx|_{x=a}$ 
\item $dy|_{y=a}$
 \item $dz|_{z=k}$
 \item $dx|_{x=\epsilon}$
 \item $dx|_{x=M}$
\end{enumeratea}
\end{multicols}
\end{esercizio}

\begin{esercizio}
\label{ese:dif02}
 Calcola il differenziale della funzione identica $y=x$ per i valori elencati.
\begin{enumeratea}
 \item $dy|_{x=0}$
 \item $dy|_{x=\frac{1}{2k}}$
 \item $dy|_{x=-\frac{2}{3}}$
 \item $dy|_{x=a} $
 \item $dy|_{x=2\epsilon} $
 \end{enumeratea}
\end{esercizio}
 
 \begin{esercizio}\label{ese:dif03}
Calcola il differenziale delle seguenti funzioni per i valori di $x$ assegnati.
 \begin{enumeratea}
  \item  $y=\frac{3}{2}x$, \hspace{5mm}per $x=1$ e per $x=0$;\hfill [$\frac{3}{2}dx$]
  \item  $y=ax$, \hspace{5mm}per $x=9$ e per $x=-\dfrac{3}{2}$; \hfill [$adx$]
  \item  $y=(6-k)x$, \hspace{5mm}per $x=-1$ e per $x=\dfrac{22}{5}$;\hfill [$(6-k)dx$]
  \item  $y=\frac{k^2+2}{5}x$, \hspace{5mm}per $x=3$ e per $x=k$;\hfill [$\frac{k^2+2}{5}dx$]
  \item  $y=5x$, \hspace{5mm}per $x=0$ e per $x=-10$;\hfill [$5dx$]
 \end{enumeratea}
\end{esercizio}
 
\begin{esercizio}\label{ese:dif04}
Calcola il differenziale delle seguenti funzioni per i valori di $x$ assegnati.
 \begin{enumeratea}
  \item  \(f(x) = -5+2x\);\hspace{5mm} $df(x)|_{x=0}=\dots$,
  \hspace{5mm}$df(x)|_{x=-1}=\dots$
  \item  \(f(x) = (a+3)x\);\hspace{5mm} $df(x)|_{x=a}=\dots$
  ;\hspace{5mm}   $df(x)|_{x=-a}=\dots$
  \item  \(f(x) = \dfrac{1}{3}x+9\);\hspace{5mm} $df(x)|_{x=\frac{2}{7}}=\dots$;
  \hspace{5mm}   $df(x)|_{x=0}=\dots$
  \item  \(f(x) = \dfrac{x}{n}\);\hspace{5mm} $df(x)|_{x=x_0}=\dots$;
  \hspace{5mm}   $df(x)|_{x=\epsilon}=\dots$
  \item  \(f(x) = x+k\);\hspace{5mm} $df(x)|_{x=x_0}=\dots$;
  \hspace{5mm}   $df(x)|_{x=k}=\dots$
 \end{enumeratea}
\end{esercizio}

\begin{esercizio}\label{ese:dif05}
Calcola i differenziali per le funzioni date
\begin{enumeratea}
  \item  Data $y=x^2$, calcola:
  $dy|_{x=5}$;\hfill [$10dx+d^2x\sim 10dx$]\\
  $dy|_{x=-5}$\hfill [$-10dx+d^2x\sim -10 dx$]\\
  $dy|_{x=2}$; \hfill [$4dx+d^2x\sim 4dx$]
  \item Data $y=-x^2$, calcola:
  $dy|_{x=0}$;\hspace{3em}$dy|_{x=\frac{1}{2}}$\hspace{3em}$dy|_{x=a}$
  \hfill [$\sim 0$; $\sim -dx$, $\sim -2adx$];
  \item Data $f(x)=3x^2$, calcola:
  $df(x)|_{x=-2}$;\hspace{3em}$f(x)|_{x=\sqrt{2}}$
  \hfill [$\sim -12dx$; $6\sqrt{2}dx$]\\
  $df(x)|_{x=b^2}$; \hfill [$\sim 6b^2dx$]
  \item Data $f(x)=\dfrac{x^2}{4}$, calcola:
  $df(x)|_{x=-4}$;\hspace{3em}$f(x)|_{x=\sqrt{2}}$\hfill [$\sim-2dx$; 
  $\sim \frac{\sqrt{2}}{2}dx$]\\
  $df(x)|_{x=b}$;\hfill [$\sim\frac{b}{2}dx$]
 \end{enumeratea}
\end{esercizio}

\begin{esercizio} \label{ese:dif06}
Differenzia:
\begin{enumeratea}
  \item  $f(x)=\dfrac{3}{x}$ per i valori:\hspace{1.5em} $x=0$;
  \hspace{1.5em}  $x=\frac{1}{2}$\hspace{1.5em}$x=a$; 
  \hfill [imp.; $\sim-12dx$;$\sim -\frac{3dx}{a^2}$]
 \item $y=\dfrac{1}{3-x}$ per:\hspace{1.5em} $x=0$;\hspace{1.5em}
  $x=\frac{1}{2}$\hspace{1.5em}$x=3$;
   \hfill [$\sim \frac{dx}{9}$; $\sim -\frac{4dx}{25}$; imp.]
  \item $f(x)=\dfrac{2}{x}-2$ per:\hspace{1.5em} $x=2$;\hspace{1.5em}
  $x=\frac{a}{3}$;  $x=\frac{1}{2}$ 
  \hfill [$\sim -\frac{dx}{2}$; $\sim -\frac{18dx}{a^2}$; $\sim -8dx$.]
  \item $f(x)=\dfrac{1}{x^2}$ per:\hspace{1.5em} $x=0$;\hspace{1.5em}
  $x=\frac{1}{2}$;\hspace{1.5em}$x=-\sqrt{2}$;
  \hfill [imp.; $\sim-16dx$;$\sim -\frac{\sqrt{2}dx}{2}$]
  \item $y=\dfrac{1}{2-x^2}$ per:\hspace{1.5em} $x=0$;\hspace{1.5em}
  $x=\frac{1}{2}$;\hspace{1.5em}$x=-\sqrt{2}$;
   \hfill [$\sim 0$; $\sim \frac{4dx}{9}$; imp.]
 \end{enumeratea} 
\end{esercizio}

\begin{esercizio}\label{ese:dif06}
Scrivi la funzione che esprime il differenziale delle funzioni date, secondo 
l'esempio:\\
$f(x)=3x^2-2x+9\srarrow df(x)= 3(2xdx+d^2x)-2dx\sim 6xdx-2dx = (6x-2)dx$.
 \begin{enumeratea}
  \item $f(x)=2x^2-5x+1$;\hfill [$\sim (4x-5)dx$]
  \item $f(x)=(2x+3)^2$;\hfill [$\sim (8x+12)dx$]
  \item $\dfrac{(1-5x)^2}{2}$;\hfill [$\sim (-5+25x)dx$]
  \item $f(x)=x^3-1$;\hfill [$\sim 3x^2dx$]
  \item $f(x)=3x^3+9x^2-6x+8$;\hfill [$\sim (3x^2+18x-6)dx$]
  \item $f(x)=5\sqrt{x}$\hfill [$\sim \frac{5}{2\sqrt{x}}dx$]
 \end{enumeratea}
\end{esercizio}

\begin{esercizio}\label{ese:dif07}
Applica le regole di pag.\pageref{subsubsec:diff01_diffsint} per trovare 
l'espressione del differenziale.
 \begin{enumeratea}
  \item $f(x)=x(1-x^2)$;\hfill [$\sim (1-3x^2)dx$]
  \item $f(x)=\dfrac{1}{x}+ 3x^2$;\hfill [$\sim (-\frac{1}{x^2}+6x)dx$]
  \item $f(x)=\dfrac{6-3x}{2x^2}+\sqrt{x}$;\hfill 
  [$\sim \tonda{\frac{3}{x^2}+\frac{1}{2\sqrt{x}}}dx$]
  \item $f(x)=\dfrac{3x^3}{\sqrt{x}}$;\hfill 
  [$\sim (9x\sqrt{x}-\frac{3x^2}{2\sqrt{x}})dx$]
  \item $f(x)=\dfrac{x^3+x^2-x+1}{1-x}$;\hfill 
  [$\sim (\frac{-2x^3+4x^2-2x}{(1-x)^2})dx$]
  \item $f(x)=5x^2\sqrt{x}$;\hfill 
  [$\sim (10x\sqrt{x}+\frac{5x^2}{2\sqrt{x}})dx$]
 \end{enumeratea}
\end{esercizio} 


\subsection{Esercizi sulle derivate}
\begin{enumerate}

\item Calcola la derivata delle seguenti funzioni nel punto c.

\begin{enumerate}
\item \(f(x)= - 5 x^{5} - 5 x^{2} + 2 x,\quad c=-1\) \hfill [\(-13\)]
\item \(f(x)= 3 x^{5} + 3 x^{4} + 5 x,\quad c=-5\) \hfill [\(7880\)]
\item \(f(x)= - 5 x^{3} - 2 x,\quad c=-2\) \hfill [\(-62\)]
\item \(f(x)= 3 x^{5} - 3 x^{4},\quad c=1\) \hfill [\(3\)]
\item \(f(x)= - 4 x^{4} - 2 x^{3},\quad c=-4\) \hfill [\(928\)]
\item \(f(x)= - 3 x^{3} - x,\quad c=-1\) \hfill [\(-10\)]
\item \(f(x)= 2 x,\quad c=2\) \hfill [\(2\)]
\item \(f(x)= x^{4} + 4 x + 2,\quad c=-4\) \hfill [\(-252\)]
\item \(f(x)= - 4 x^{5} - 2 x^{4} - x^{2},\quad c=-1\) \hfill [\(-10\)]
\item \(f(x)= - x^{2},\quad c=-2\) \hfill [\(4\)]
\item \(f(x)= x^{5} + x,\quad c=-4\) \hfill [\(1281\)]
\item \(f(x)= 8 x^{3} + 2 x,\quad c=1\) \hfill [\(26\)]
\item \(f(x)= 2 x^{5} + x^{3},\quad c=0\) \hfill [\(0\)]
\item \(f(x)= - x^{5} + x^{4},\quad c=-2\) \hfill [\(-112\)]
\item \(f(x)= - 3 x^{5} + x^{4} - x^{2},\quad c=-1\) \hfill [\(-17\)]
 \end{enumerate}
% \vspace{0mm}


\item Calcola la retta tangente alla funzione nel punto P.

\begin{enumerate}
\item \(f(x)= 5 x^{3} - 7 x^{2} - 4 x - 8,\quad P(2;~-4)\) \hfill [\(y = 28 x - 
60\)]
\item \(f(x)= x^{3} + x^{2} - 2 x - 7,\quad P(1;~-7)\) \hfill [\(y = 3 x - 10\)]
\item \(f(x)= - 9 x^{3} + 12 x^{2} + 12 x - 10,\quad P(-3;~305)\) \hfill [\(y = 
- 303 x - 604\)]
\item \(f(x)= -5,\quad P(2;~-5)\) \hfill [\(y = -5\)]
\item \(f(x)= 6 x^{4} + 3 x^{3} - 5 x^{2} - 10 x + 7,\quad P(0;~7)\) \hfill [\(y 
= - 10 x + 7\)]
\item \(f(x)= 11 x^{3} - 12 x^{2} - x + 7,\quad P(-4;~-885)\) \hfill [\(y = 623 
x + 1607\)]
\item \(f(x)= 11 x^{3} - 11 x^{2} - 9 x - 6,\quad P(-5;~-1611)\) \hfill [\(y = 
926 x + 3019\)]
\item \(f(x)= - 7 x - 5,\quad P(-3;~16)\) \hfill [\(y = - 7 x - 5\)]
\item \(f(x)= -8,\quad P(2;~-8)\) \hfill [\(y = -8\)]
\item \(f(x)= 5 x^{4} - 4 x^{3} + x^{2} + 12 x + 5,\quad P(3;~347)\) \hfill [\(y 
= 450 x - 1003\)]
\item \(f(x)= 11,\quad P(0;~11)\) \hfill [\(y = 11\)]
\item \(f(x)= - 3 x^{3} - 7 x^{2} + 3 x - 10,\quad P(1;~-17)\) \hfill [\(y = - 
20 x + 3\)]
\item \(f(x)= 4 x + 4,\quad P(-3;~-8)\) \hfill [\(y = 4 x + 4\)]
\item \(f(x)= 5,\quad P(-3;~5)\) \hfill [\(y = 5\)]
\item \(f(x)= 10 x^{4} + 5 x^{3} + 8 x^{2} + 5 x - 7,\quad P(4;~3021)\) \hfill 
[\(y = 2869 x - 8455\)]
 \end{enumerate}
% \vspace{0mm}


\item Deriva le seguenti funzioni del tipo: y=f(x)+g(x).

\begin{enumerate}
\item \(f(x) = - 4 x^{5} + 5 x^{2} + 4 x\) \hfill [\(- 20 x^{4} + 10 x + 4\)]
\item \(f(x) = 5 x^{5} - 4 x^{2} + x\) \hfill [\(25 x^{4} - 8 x + 1\)]
\item \(f(x) = - x^{4} + x + 4\) \hfill [\(- 4 x^{3} + 1\)]
\item \(f(x) = 2 x^{4} + 3 x^{3}\) \hfill [\(8 x^{3} + 9 x^{2}\)]
\item \(f(x) = 2 x^{4} + 2 x^{3} + 1\) \hfill [\(8 x^{3} + 6 x^{2}\)]
\item \(f(x) = - 5 x^{4} + 3\) \hfill [\(- 20 x^{3}\)]
\item \(f(x) = - x^{4} + x^{2} + 1\) \hfill [\(- 4 x^{3} + 2 x\)]
\item \(f(x) = - 2 x^{3} - 3 x^{2}\) \hfill [\(- 6 x^{2} - 6 x\)]
\item \(f(x) = 4 x^{4} - x^{2} - 3 x\) \hfill [\(16 x^{3} - 2 x - 3\)]
\item \(f(x) = - x^{5} + 3 x^{2}\) \hfill [\(- 5 x^{4} + 6 x\)]
\item \(f(x) = 2\) \hfill [\(0\)]
\item \(f(x) = 11 x^{3}\) \hfill [\(33 x^{2}\)]
\item \(f(x) = 5 x^{3}\) \hfill [\(15 x^{2}\)]
\item \(f(x) = x^{4} - 3 x\) \hfill [\(4 x^{3} - 3\)]
\item \(f(x) = 3 x^{4} - x^{2}\) \hfill [\(12 x^{3} - 2 x\)]
 \end{enumerate}
% \vspace{0mm}


\item Deriva le seguenti funzioni del tipo: $y=f(x)\cdot g(x)$.

\begin{enumerate}
\item \(f(x) = \left(9 + \frac{5}{x}\right) \left(- 2 x^{3} + 5\right)\) \hfill 
[\(- 6 x^{2} \left(9 + \frac{5}{x}\right) - \frac{5}{x^{2}} \left(- 2 x^{3} + 
5\right)\)]
\item \(f(x) = \left(1 + \frac{4}{x^{2}}\right) \left(5 x - 3\right)\) \hfill 
[\(5 + \frac{20}{x^{2}} - \frac{8}{x^{3}} \left(5 x - 3\right)\)]
\item \(f(x) = \left(-9 + \frac{2}{x^{4}}\right) \left(-6 + 
\frac{2}{x^{2}}\right)\) \hfill [\(- \frac{4}{x^{3}} \left(-9 + 
\frac{2}{x^{4}}\right) - \frac{8}{x^{5}} \left(-6 + \frac{2}{x^{2}}\right)\)]
\item \(f(x) = \left(-2 + \frac{3}{x}\right) \left(- x - 1\right)\) \hfill [\(2 
- \frac{3}{x} - \frac{3}{x^{2}} \left(- x - 1\right)\)]
\item \(f(x) = \left(3 x - 10\right) \left(2 x^{2} - 10\right)\) \hfill [\(6 
x^{2} + 4 x \left(3 x - 10\right) - 30\)]
\item \(f(x) = - \frac{3}{x} \left(1 + \frac{4}{x^{3}}\right)\) \hfill 
[\(\frac{3}{x^{2}} \left(1 + \frac{4}{x^{3}}\right) + \frac{36}{x^{5}}\)]
\item \(f(x) = \left(-7 - \frac{5}{x^{4}}\right) \left(x^{4} - 6\right)\) \hfill 
[\(4 x^{3} \left(-7 - \frac{5}{x^{4}}\right) + \frac{20}{x^{5}} \left(x^{4} - 
6\right)\)]
\item \(f(x) = \left(-2 - \frac{1}{x}\right) \left(- x^{4} - 2\right)\) \hfill 
[\(- 4 x^{3} \left(-2 - \frac{1}{x}\right) + \frac{1}{x^{2}} \left(- x^{4} - 
2\right)\)]
\item \(f(x) = \left(1 + \frac{1}{x}\right) \left(3 x^{4} + 9\right)\) \hfill 
[\(12 x^{3} \left(1 + \frac{1}{x}\right) - \frac{1}{x^{2}} \left(3 x^{4} + 
9\right)\)]
\item \(f(x) = \left(10 - \frac{5}{x^{2}}\right) \left(- 4 x^{4} - 9\right)\) 
\hfill [\(- 16 x^{3} \left(10 - \frac{5}{x^{2}}\right) + \frac{10}{x^{3}} 
\left(- 4 x^{4} - 9\right)\)]
\item \(f(x) = \left(-3 + \frac{5}{x}\right) \left(- 5 x - 5\right)\) \hfill 
[\(15 - \frac{25}{x} - \frac{5}{x^{2}} \left(- 5 x - 5\right)\)]
\item \(f(x) = - 3 x^{3} \left(8 + \frac{2}{x}\right)\) \hfill [\(- 9 x^{2} 
\left(8 + \frac{2}{x}\right) + 6 x\)]
\item \(f(x) = \left(5 x^{2} + 5\right) \left(x^{3} - 4\right)\) \hfill [\(3 
x^{2} \left(5 x^{2} + 5\right) + 10 x \left(x^{3} - 4\right)\)]
\item \(f(x) = \left(5 + \frac{1}{x}\right) \left(- x^{4} - 8\right)\) \hfill 
[\(- 4 x^{3} \left(5 + \frac{1}{x}\right) - \frac{1}{x^{2}} \left(- x^{4} - 
8\right)\)]
\item \(f(x) = \left(5 - \frac{3}{x^{4}}\right) \left(2 x^{4} + 1\right)\) 
\hfill [\(8 x^{3} \left(5 - \frac{3}{x^{4}}\right) + \frac{12}{x^{5}} \left(2 
x^{4} + 1\right)\)]
 \end{enumerate}
% \vspace{0mm}

\item Deriva le seguenti funzioni del tipo: $y=\dfrac{f(x)}{g(x)}$.

\begin{enumerate}
\item $y= {\frac {10\,x-8}{9\,x+6}} $ \hfill [ $y'= 10\, \left( 9\,x+6 \right) 
^{-1}-9\,{\frac {10\,x-8}{ \left( 9\,x+6 \right) ^{2}}}$ ]
\item $y= {\frac {-5\,x-9}{4\,x+6}} $ \hfill [ $y'= -5\, \left( 4\,x+6 \right) 
^{-1}-4\,{\frac {-5\,x-9}{ \left( 4\,x+6 \right) ^{2}}}$ ]
\item $y= {\frac {-6\,x-8}{-7\,x-9}} $ \hfill [ $y'= -6\, \left( -7\,x-9 \right) 
^{-1}+7\,{\frac {-6\,x-8}{ \left( -7\,x-9 \right) ^{2}}}$ ]
\item $y= {\frac {x-6}{3\,x+4}} $ \hfill [ $y'=  \left( 3\,x+4 \right) 
^{-1}-3\,{\frac {x-6}{ \left( 3\,x+4 \right) ^{2}}}$ ]
\item $y=  \left( 3\,x+4 \right) ^{-1}-3\,{\frac {x-6}{ \left( 3\,x+4 \right) 
^{2}}} $ \hfill [ $y'= -6\, \left( 3\,x+4 \right) ^{-2}+18\,{\frac {x-6}{ \left( 
3\,x+4 \right) ^{3}}}$ ]
\item $y= {\frac {4\,x-3}{-7\,x+3}} $ \hfill [ $y'= 4\, \left( -7\,x+3 \right) 
^{-1}+7\,{\frac {4\,x-3}{ \left( -7\,x+3 \right) ^{2}}}$ ]
\item $y= {\frac {-6\,x+6}{2\,x-2}} $ \hfill [ $y'= -6\, \left( 2\,x-2 \right) 
^{-1}-2\,{\frac {-6\,x+6}{ \left( 2\,x-2 \right) ^{2}}}$ ]
\item $y= {\frac {-3\,x-9}{-3\,x-2}} $ \hfill [ $y'= -3\, \left( -3\,x-2 \right) 
^{-1}+3\,{\frac {-3\,x-9}{ \left( -3\,x-2 \right) ^{2}}}$ ]
\item $y= {\frac {5\,x+3}{4\,x+5}} $ \hfill [ $y'= 5\, \left( 4\,x+5 \right) 
^{-1}-4\,{\frac {5\,x+3}{ \left( 4\,x+5 \right) ^{2}}}$ ]
\item $y= {\frac {-3\,x-8}{-3\,x-3}} $ \hfill [ $y'= -3\, \left( -3\,x-3 \right) 
^{-1}+3\,{\frac {-3\,x-8}{ \left( -3\,x-3 \right) ^{2}}}$ ]
\item $y= {\frac {-9\,x-1}{-5\,x-10}} $ \hfill [ $y'= -9\, \left( -5\,x-10 
\right) ^{-1}+5\,{\frac {-9\,x-1}{ \left( -5\,x-10 \right) ^{2}}}$ ]
\item $y= -{\frac {x+8}{9\,x}} $ \hfill [ $y'= -{\frac {1}{9\,x}}+{\frac 
{x+8}{9\,{x}^{2}}}$ ]
\item $y= {\frac {-4\,x+3}{-9\,x+7}} $ \hfill [ $y'= -4\, \left( -9\,x+7 \right) 
^{-1}+9\,{\frac {-4\,x+3}{ \left( -9\,x+7 \right) ^{2}}}$ ]
\end{enumerate}


\item Deriva le seguenti funzioni del tipo: $y=f(x)\cdot g(x)$.

\begin{enumerate}
\item \(f(x) = - x^{3} \cos{\left (x \right )}\) \hfill [\(x^{3} \sin{\left (x 
\right )} - 3 x^{2} \cos{\left (x \right )}\)]
\item \(f(x) = - 5 x^{4} \sin{\left (x \right )}\) \hfill [\(- 5 x^{4} 
\cos{\left (x \right )} - 20 x^{3} \sin{\left (x \right )}\)]
\item \(f(x) = - 3 x^{3} \tan{\left (x \right )}\) \hfill [\(- 3 x^{3} 
\left(\tan^{2}{\left (x \right )} + 1\right) - 9 x^{2} \tan{\left (x \right 
)}\)]
\item \(f(x) = - 2 x^{3} \sin{\left (x \right )}\) \hfill [\(- 2 x^{3} 
\cos{\left (x \right )} - 6 x^{2} \sin{\left (x \right )}\)]
\item \(f(x) = 2 x^{2} \sin{\left (x \right )}\) \hfill [\(2 x^{2} \cos{\left (x 
\right )} + 4 x \sin{\left (x \right )}\)]
\item \(f(x) = - 4 x \cos{\left (x \right )}\) \hfill [\(4 x \sin{\left (x 
\right )} - 4 \cos{\left (x \right )}\)]
\item \(f(x) = x \cos{\left (x \right )}\) \hfill [\(- x \sin{\left (x \right )} 
+ \cos{\left (x \right )}\)]
\item \(f(x) = 5 x \cos{\left (x \right )}\) \hfill [\(- 5 x \sin{\left (x 
\right )} + 5 \cos{\left (x \right )}\)]
\item \(f(x) = x^{3} \tan{\left (x \right )}\) \hfill [\(x^{3} 
\left(\tan^{2}{\left (x \right )} + 1\right) + 3 x^{2} \tan{\left (x \right 
)}\)]
\item \(f(x) = x^{2} \tan{\left (x \right )}\) \hfill [\(x^{2} 
\left(\tan^{2}{\left (x \right )} + 1\right) + 2 x \tan{\left (x \right )}\)]
\item \(f(x) = - 4 x^{4} \sin{\left (x \right )}\) \hfill [\(- 4 x^{4} 
\cos{\left (x \right )} - 16 x^{3} \sin{\left (x \right )}\)]
\item \(f(x) = 5 x^{4} \sin{\left (x \right )}\) \hfill [\(5 x^{4} \cos{\left (x 
\right )} + 20 x^{3} \sin{\left (x \right )}\)]
\item \(f(x) = 2 x^{3} \cos{\left (x \right )}\) \hfill [\(- 2 x^{3} \sin{\left 
(x \right )} + 6 x^{2} \cos{\left (x \right )}\)]
\item \(f(x) = - 4 x^{3} \sin{\left (x \right )}\) \hfill [\(- 4 x^{3} 
\cos{\left (x \right )} - 12 x^{2} \sin{\left (x \right )}\)]
\item \(f(x) = 2 x \sin{\left (x \right )}\) \hfill [\(2 x \cos{\left (x \right 
)} + 2 \sin{\left (x \right )}\)]
\end{enumerate}
% \vspace{0mm}


\item Deriva le seguenti funzioni composte del tipo: y=f[g(x)].

\begin{enumerate}
\item \(f(x) = - 3 \cos{\left (5 x \right )}\) \hfill [\(15 \sin{\left (5 x 
\right )}\)]
\item \(f(x) = 5 \cos{\left (4 x^{4} \right )}\) \hfill [\(- 80 x^{3} \sin{\left 
(4 x^{4} \right )}\)]
\item \(f(x) = - 3 \cos{\left (2 x^{4} \right )}\) \hfill [\(24 x^{3} \sin{\left 
(2 x^{4} \right )}\)]
\item \(f(x) = - 5 \tan{\left (3 x^{2} \right )}\) \hfill [\(- 30 x 
\left(\tan^{2}{\left (3 x^{2} \right )} + 1\right)\)]
\item \(f(x) = 5 \tan{\left (5 x \right )}\) \hfill [\(25 \tan^{2}{\left (5 x 
\right )} + 25\)]
\item \(f(x) = 4 \sin{\left (2 x^{2} \right )}\) \hfill [\(16 x \cos{\left (2 
x^{2} \right )}\)]
\item \(f(x) = - 3 \cos{\left (x^{2} \right )}\) \hfill [\(6 x \sin{\left (x^{2} 
\right )}\)]
\item \(f(x) = 5 \cos{\left (5 x \right )}\) \hfill [\(- 25 \sin{\left (5 x 
\right )}\)]
\item \(f(x) = \sin{\left (4 x^{3} \right )}\) \hfill [\(12 x^{2} \cos{\left (4 
x^{3} \right )}\)]
\item \(f(x) = - \sin{\left (3 x^{2} \right )}\) \hfill [\(- 6 x \cos{\left (3 
x^{2} \right )}\)]
\item \(f(x) = 5 \sin{\left (5 x \right )}\) \hfill [\(25 \cos{\left (5 x \right 
)}\)]
\item \(f(x) = 5 \sin{\left (2 x \right )}\) \hfill [\(10 \cos{\left (2 x \right 
)}\)]
\item \(f(x) = 3 \sin{\left (5 x^{4} \right )}\) \hfill [\(60 x^{3} \cos{\left 
(5 x^{4} \right )}\)]
\item \(f(x) = 5 \cos{\left (2 x^{3} \right )}\) \hfill [\(- 30 x^{2} \sin{\left 
(2 x^{3} \right )}\)]
\item \(f(x) = - 4 \tan{\left (4 x^{4} \right )}\) \hfill [\(- 64 x^{3} 
\left(\tan^{2}{\left (4 x^{4} \right )} + 1\right)\)]
 \end{enumerate}
% \vspace{0mm}


\item Deriva le seguenti funzioni composte del tipo: $y=f[g(x)]$.

\begin{enumerate}
\item \(f(x) = - 3 e^{\sqrt{x^{2} + 3}}\) \hfill [\(- \frac{3 x e^{\sqrt{x^{2} + 
3}}}{\sqrt{x^{2} + 3}}\)]
\item \(f(x) = 3 e^{\sqrt{3 x^{2} + 3}}\) \hfill [\(\frac{9 x e^{\sqrt{3 x^{2} + 
3}}}{\sqrt{3 x^{2} + 3}}\)]
\item \(f(x) = e^{\sqrt{3 x^{3} - 5}}\) \hfill [\(\frac{9 x^{2} e^{\sqrt{3 x^{3} 
- 5}}}{2 \sqrt{3 x^{3} - 5}}\)]
\item \(f(x) = - e^{\sqrt{5 x^{3} - 3}}\) \hfill [\(- \frac{15 x^{2} e^{\sqrt{5 
x^{3} - 3}}}{2 \sqrt{5 x^{3} - 3}}\)]
\item \(f(x) = - e^{\sqrt{- 5 x^{3} + 1}}\) \hfill [\(\frac{15 x^{2} e^{\sqrt{- 
5 x^{3} + 1}}}{2 \sqrt{- 5 x^{3} + 1}}\)]
\item \(f(x) = 5 e^{\sqrt{2 x^{4} - 5}}\) \hfill [\(\frac{20 x^{3} e^{\sqrt{2 
x^{4} - 5}}}{\sqrt{2 x^{4} - 5}}\)]
\item \(f(x) = - e^{\sqrt{- x + 1}}\) \hfill [\(\frac{e^{\sqrt{- x + 1}}}{2 
\sqrt{- x + 1}}\)]
\item \(f(x) = - 5 e^{\sqrt{3 x^{3} + 1}}\) \hfill [\(- \frac{45 x^{2} 
e^{\sqrt{3 x^{3} + 1}}}{2 \sqrt{3 x^{3} + 1}}\)]
\item \(f(x) = 2 e^{\sqrt{- x^{3} - 5}}\) \hfill [\(- \frac{3 x^{2} e^{\sqrt{- 
x^{3} - 5}}}{\sqrt{- x^{3} - 5}}\)]
\item \(f(x) = 2 e^{\sqrt{- 2 x^{4} + 5}}\) \hfill [\(- \frac{8 x^{3} e^{\sqrt{- 
2 x^{4} + 5}}}{\sqrt{- 2 x^{4} + 5}}\)]
\item \(f(x) = 2 e^{\sqrt{2 x^{3} - 4}}\) \hfill [\(\frac{6 x^{2} e^{\sqrt{2 
x^{3} - 4}}}{\sqrt{2 x^{3} - 4}}\)]
\item \(f(x) = - 4 e^{\sqrt{- 5 x^{3} - 3}}\) \hfill [\(\frac{30 x^{2} 
e^{\sqrt{- 5 x^{3} - 3}}}{\sqrt{- 5 x^{3} - 3}}\)]
\item \(f(x) = 2 e^{\sqrt{- 4 x^{4} - 4}}\) \hfill [\(- \frac{16 x^{3} 
e^{\sqrt{- 4 x^{4} - 4}}}{\sqrt{- 4 x^{4} - 4}}\)]
\item \(f(x) = 2 e^{\sqrt{- 3 x^{2} - 4}}\) \hfill [\(- \frac{6 x e^{\sqrt{- 3 
x^{2} - 4}}}{\sqrt{- 3 x^{2} - 4}}\)]
\item \(f(x) = e^{\sqrt{2 x^{3} + 1}}\) \hfill [\(\frac{3 x^{2} e^{\sqrt{2 x^{3} 
+ 1}}}{\sqrt{2 x^{3} + 1}}\)]
\end{enumerate}
% \vspace{0mm}
\item Deriva le seguenti funzioni contenenti funzioni esponenziali e 
logaritmiche.

\begin{enumerate}
\item $f(x)=4(\log x -1)$\hfill [$\frac{4}{x}$]
\item $f(x)=\log 4x)$\hfill [$\frac{1}{x}$]
\item $f(x)=x(\ln x -1)$\hfill [$\ln x$]
\item $f(x)=e^x(\ln x)$\hfill [$e^x\tonda{\ln x+\frac{1}{x}}$]
\item $f(x)=x^3(\log x)$\hfill [$x^2\tonda{3\log x+1}$]
\item $f(x)=e^{3x}-\ln x^2$\hfill [$3e^{3x}\frac{2}{x}$]
\item $f(x)=\dfrac{\ln x}{x}$\hfill [$\dfrac{1}{x^2}(1-\ln x)$]
\item $f(x)=e^x(\ln x)$\hfill [$e^x\tonda{\ln x+\frac{1}{x}}$]
\item $f(x)=\dfrac{1}{\ln x}$\hfill [$-\dfrac{1}{x\ln^2 x}$]
\item $f(x)=e^{\frac{x-2}{x}}$\hfill [$\dfrac{2}{x^2}e^{\frac{x-2}{x}}$]
\item $f(x)=e^{\frac{-1}{x}}$\hfill [$\dfrac{1}{x^2}e^{\frac{-1}{x}}$]
\item $f(x)=e^{x^2}+e^x+3$\hfill [$2xe^{x^2}+e^x$]
\item $f(x)=e^{2x}\ln(1+x)$\hfill[$e^{2x}\quadra{2\ln (1+x)+\dfrac{1}{1+x}}$]
\item $f(x)=e^{\sen x}+e^{\sqrt{x}}$\hfill 
[$\cos x \cdot e^{\sen x}+ \dfrac{e^{\sqrt{x}}}{2\sqrt{x}}$]
\end{enumerate}

\item Calcola le derivate delle seguenti funzioni.

\begin{enumerate}
\item \(f(x) = \left(x^{2} - 6\right) \left(2 x^{2} + 6\right)\) \hfill [\(4 x 
\left(x^{2} - 6\right) + 2 x \left(2 x^{2} + 6\right)\)]
\item \(f(x) = - 5 \tan{\left (x^{2} \right )}\) \hfill [\(- 10 x 
\left(\tan^{2}{\left (x^{2} \right )} + 1\right)\)]
\item \(f(x) = 4 e^{\sqrt{4 x^{2} - 2}}\) \hfill [\(\frac{16 x e^{\sqrt{4 x^{2} 
- 2}}}{\sqrt{4 x^{2} - 2}}\)]
\item \(f(x) = 4 e^{\sqrt{- 5 x^{2} - 1}}\) \hfill [\(- \frac{20 x e^{\sqrt{- 5 
x^{2} - 1}}}{\sqrt{- 5 x^{2} - 1}}\)]
\item \(f(x)= - 8 x^{3} + 2 x,\quad c=-1\) \hfill [\(-22\)]
\item \(f(x)= - x^{2},\quad c=1\) \hfill [\(-2\)]
\item \(f(x) = - 2 \sin{\left (5 x^{3} \right )}\) \hfill [\(- 30 x^{2} 
\cos{\left (5 x^{3} \right )}\)]
\item \(f(x) = - 5 x^{4} \sin{\left (x \right )}\) \hfill [\(- 5 x^{4} 
\cos{\left (x \right )} - 20 x^{3} \sin{\left (x \right )}\)]
\item \(f(x) = - 5 x^{4} - 5 x^{3} + 5\) \hfill [\(- 20 x^{3} - 15 x^{2}\)]
\item \(f(x) = - 4 \tan{\left (2 x^{2} \right )}\) \hfill [\(- 16 x 
\left(\tan^{2}{\left (2 x^{2} \right )} + 1\right)\)]
\item \(f(x) = - 2 \tan{\left (5 x^{2} \right )}\) \hfill [\(- 20 x 
\left(\tan^{2}{\left (5 x^{2} \right )} + 1\right)\)]
% \item \(f(x)= x^{3} - 2 x^{2} + 8 x + 12,\quad P(-3;~-57)\) \hfill [\(y = 47 x + 
% 84\)]
\item \(f(x) = - x^{4} + 2 x^{3} + x^{2}\) \hfill [\(- 4 x^{3} + 6 x^{2} + 2 
x\)]
\item \(f(x) = - x^{3} \tan{\left (x \right )}\) \hfill [\(- x^{3} 
\left(\tan^{2}{\left (x \right )} + 1\right) - 3 x^{2} \tan{\left (x \right 
)}\)]
\item \(f(x) = - 4 x^{4}\) \hfill [\(- 16 x^{3}\)]
\item \(f(x)= - 7 x + 1,\quad P(0;~1)\) \hfill [\(y = - 7 x + 1\)]
\item \(f(x) = 2 e^{\sqrt{x^{4} + 1}}\) \hfill [\(\frac{4 x^{3} e^{\sqrt{x^{4} + 
1}}}{\sqrt{x^{4} + 1}}\)]
\item \(f(x)= - 3 x^{4} - 5 x^{3} - 8 x^{2} - 3 x - 12,\quad P(-1;~-15)\) \hfill 
[\(y = 10 x - 5\)]
\end{enumerate}

\item Calcola le derivate delle seguenti funzioni.
\begin{enumerate}
\item \(f(x)= 3 x^{5} - 3 x^{4},\quad c=-5\) \hfill [\(10875\)]
\item \(f(x)= - 2 x^{4},\quad c=4\) \hfill [\(-512\)]
\item $y= {\frac {-5\,x+6}{9\,x-6}} $ \hfill [ $y'= -5\, \left( 9\,x-6 \right) 
^{-1}-9\,{\frac {-5\,x+6}{ \left( 9\,x-6 \right) ^{2}}}$ ]
\item $y= {\frac {8\,x-9}{-9\,x+3}} $ \hfill [ $y'= 8\, \left( -9\,x+3 \right) 
^{-1}+9\,{\frac {8\,x-9}{ \left( -9\,x+3 \right) ^{2}}}$ ]
\item $y= {\frac {-7\,x+6}{-5\,x+3}} $ \hfill [ $y'= -7\, \left( -5\,x+3 \right) 
^{-1}+5\,{\frac {-7\,x+6}{ \left( -5\,x+3 \right) ^{2}}}$ ]
\item \(f(x) = - 8 x^{4}\) \hfill [\(- 32 x^{3}\)]
\item \(f(x) = 2 x^{5} - 2 x^{3} - 5 x\) \hfill [\(10 x^{4} - 6 x^{2} - 5\)]
\item $y= {\frac {-4\,x+4}{5\,x+6}} $ \hfill [ $y'= -4\, \left( 5\,x+6 \right) 
^{-1}-5\,{\frac {-4\,x+4}{ \left( 5\,x+6 \right) ^{2}}}$ ]
\item $y= {\frac {-10\,x+10}{-5\,x-8}} $ \hfill [ $y'= -10\, \left( -5\,x-8 
\right) ^{-1}+5\,{\frac {-10\,x+10}{ \left( -5\,x-8 \right) ^{2}}}$ ]
\item $y= {\frac {6\,x-10}{2\,x-9}} $ \hfill [ $y'= 6\, \left( 2\,x-9 \right) 
^{-1}-2\,{\frac {6\,x-10}{ \left( 2\,x-9 \right) ^{2}}}$ ]
\item $y= {\frac {4\,x+6}{-2\,x+10}} $ \hfill [ $y'= 4\, \left( -2\,x+10 \right) 
^{-1}+2\,{\frac {4\,x+6}{ \left( -2\,x+10 \right) ^{2}}}$ ]
\item $y= {\frac {10\,x-6}{-x+2}} $ \hfill [ $y'= 10\, \left( -x+2 \right) 
^{-1}+{\frac {10\,x-6}{ \left( -x+2 \right) ^{2}}}$ ]
\item \(f(x)= - 4 x^{5} - 4 x^{4},\quad c=0\) \hfill [\(0\)]
\item \(f(x) = - 3 x \tan{\left (x \right )}\) \hfill [\(- 3 x 
\left(\tan^{2}{\left (x \right )} + 1\right) - 3 \tan{\left (x \right )}\)]
\item \(f(x) = \sin{\left (3 x^{4} \right )}\) \hfill [\(12 x^{3} \cos{\left (3 
x^{4} \right )}\)]
\item \(f(x) = \left(- 5 x^{2} + 9\right) \left(4 x^{2} - 5\right)\) \hfill [\(8 
x \left(- 5 x^{2} + 9\right) - 10 x \left(4 x^{2} - 5\right)\)]
\item $y= {\frac {8\,x-9}{-x-8}} $ \hfill [ $y'= 8\, \left( -x-8 \right) 
^{-1}+{\frac {8\,x-9}{ \left( -x-8 \right) ^{2}}}$ ]
\item $y= {\frac {-2\,x-9}{3\,x+10}} $ \hfill [ $y'= -2\, \left( 3\,x+10 \right) 
^{-1}-3\,{\frac {-2\,x-9}{ \left( 3\,x+10 \right) ^{2}}}$ ]
\item $y= {\frac {4\,x+10}{-8\,x+1}} $ \hfill [ $y'= 4\, \left( -8\,x+1 \right) 
^{-1}+8\,{\frac {4\,x+10}{ \left( -8\,x+1 \right) ^{2}}}$ ]
\item $y= {\frac {-7\,x+6}{-2\,x+3}} $ \hfill [ $y'= -7\, \left( -2\,x+3 \right) 
^{-1}+2\,{\frac {-7\,x+6}{ \left( -2\,x+3 \right) ^{2}}}$ ]
\end{enumerate}

\end{enumerate}
 
 
\subsection{Problemi che coivolgono l'uso della derivata}
\begin{enumerate}
\item Quale è l'equazione delle rette tangenti al grafico di $y=\sen x$, 
nell'intervallo $\intervcc{0}{2\pi}$, nei punti comuni con l'asse delle $x$?
 \hfill[$y=\pm x\mp k\pi$]

\item Data la curva $y=\dfrac{x+3}{x}$, trova il punto in cui la tangente ha 
la pendenza $m=-2$. Spiega la ragione del doppio risultato.
\hfill [$x=\pm\sqrt{\frac{3}{2}}$]

\item  Trova le equazioni delle tangenti al grafico della parabola 
$y=-x^2+7x-6$ nei punti in cui esse formano rispettivamente un angolo di 
$45^\circ$ e di $135^\circ$ rispetto all'orizzontale e trova la loro 
intersezione.
 \hfill [$y=x+3$; $y=-x+10$; $\punto{\frac{7}{2}}{\frac{13}{2}}$]

\item Se la quantità di carica che attraversa la sezione di un conduttore
segue la legge $q~=~e^{-2t+3}$,  determina 
l'intensità di corrente dopo $5\,s$.
$q$ è la quantità di carica in Coulomb e $t$ è il tempo in secondi.
\hfill [$i=-1,8\times 10^{-3}\, A$]

\item Un triangolo rettangolo elastico ha per base un cateto di $10\, cm$.
Il secondo cateto $c$ all'inizio misura $0\, cm$, ma cresce al ritmo di 
$1\, cm$ ogni secondo. Con quale ritmo cresce l'ipotenusa $i$?
Si tratta di un ritmo costante o variabile?
Controlla le risposte calcolando la velocità di crescita dell'ipotenusa quando 
il secondo cateto misura $10\, cm$, $20\, cm$, $30\, cm$ \dots
\hfill [$\frac{di}{dt}=\frac{c}{\sqrt{100+c^2}}$; variabile]

\item È noto che il rapporto fra la diagonale e il lato di un quadrato è uguale 
a $\sqrt{2}$. Prolungando la diagonale di un infinitesimo, anche il lato 
subisce un allungamento infinitesimo. Che rapporto c'è fra i due allungamenti?
Puoi giustificare la risposta alla luce delle tue conoscenze del calcolo 
infinitesimale?

\item Un recipiente ha la cavità interna a forma di cono equilatero 
(rovesciato), con il diametro di base di $30\, cm$. Il rubinetto che lo riempie
eroga $4$ litri al minuto. Il livello dell'acqua all'interno cresce in modo 
costante? In quanto tempo il recipiente sarà riempito fino a metà altezza?
A quel punto, con quale velocità cresce il livello dell'acqua?
E quando è quasi pieno?
\hfill [No; $13,25\, s$; $2,26\, dm/min$; $0,57\, dm/min$]

\item Una lastra di policarbonato spessa $1 \,cm$ è trasparente  per l'$85\%$,
cioè trattiene il $15\%$ della radiazione luminosa che l'attraversa. Due lastre
uguali non trattengono il doppio, perché la seconda trattiene il $15\%$ di 
quanto le perviene dalla prima lastra: il 15\% dell'85\%. La diminuzione di
intensità luminosa $dI(s)$ è quindi proporzionale alla radiazione in arrivo $I$
e allo spessore $ds$ della lastra: $dI=-0,15\cdot I\cdot ds$. Riscrivi la legge
come derivata: quali funzioni hanno la derivata proporzionale alla funzione 
stessa? Scrivi la legge dell'attenuazione luminosa $I=f(s)$. Quale spessore di 
policarbonato è sufficiente ad attenuare l'intensità luminosa del $40\%$?
\hfill [$I'(s)=-0,15\cdot I(s)$; $I=I_0e^{-0,15s}$; $3,4\,cm$]


\end{enumerate}
