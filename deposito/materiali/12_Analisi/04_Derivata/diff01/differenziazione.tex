% (c) 2015 Daniele Zambelli daniele.zambelli@gmail.com

\chapter{Derivate}
% \begin{wrapfloat}{figure}{r}{0pt}
% \includegraphics[scale=0.35]{img/fig000_.png}
% \caption{...}
% \label{fig:...}
% \end{wrapfloat}
% 
% \begin{center} \input{\folder lbr/fig000_.pgf} \end{center}

Il problema di determinare la velocità istntanea ci ha portati a conoscere 
i numeri infinitesimi e, attraverso questi, l'insieme dei numeri iperreali.
Ora siamo in grado di cercare la risposta alla domanda rimasta in sospeso: 
come si determina la velocità istantanea?\\
La risposta, che conosciamo nelle forme moderne da più di 400 anni,
propone al nostro studio un nuovo potentissimo strumento di calcolo, adatto a 
risolvere problemi in ogni ambito scientifico: la derivata.

\section{Velocità di caduta}
\label{04_diffvelcaduta}
Nel Settecento fiorirono alcune leggende su Galileo Galilei. Una di queste racconta 
che per dimostrare che i gravi cadono con la stessa velocità, gettò dalla Torre
di Pisa due sfere di peso diverso, ma di uguali dimensioni. I due oggetti, come
oggi possiamo immaginare, raggiunsero il suolo contemporaneamente.\\
La Torre di Pisa è alta circa $56m$ e immaginiamo, per semplificare, che la 
distanza percorsa dai due oggetti sia di $56m$ (ti lascio calcolare il 
percorso effettivo: tieni presente che al giorno d'oggi l'inclinazione della 
Torre è di $4,8^\circ$).\\
Oggi sappiamo che un oggetto in caduta libera ha la seguente legge del moto:
\(s=\frac{1}{2}gt^2\). Come al solito, $s$ è lo spazio in metri, $t$ è il 
tempo in secondi, $g=9,81 m/s^2$ è l'accelerazione di gravità, costante nei
pressi della superficie terrestre.\\
Se cerchiamo la velocità media, basta dividere lo spazio percorso per il tempo
impiegato:

\begin{align*}
 s_{tot}= & 56m;\quad \text{da }s=\frac{1}{2}gt^2\ \text{ si ha: }
 t_{tot}=\sqrt{\frac{2s_{tot}}{g}}=\sqrt{\frac{2\times 56}{9,81}}=3,36 s.\\
 v_m= & \frac{s_{tot}}{t_{tot}}=\frac{56 m}{3,36 s}=16,67 m/s,
\end{align*}
che corrispondono a circa $60 km/h$ di media.\\
Ma gli oggetti partono fermi e arrivano velocissimi: è possibile sapere la
progressione della loro velocità e quale è loro velocità finale? È il momento
di usare le quantità infinitesime.\\
Chiamiamo $dt$ un intervallo di tempo infinitesimo, fra due istanti successivi
$t$ e $t+dt$.
Lo spazio percorso nella caduta, in quell'intervallo di tempo, applicando la legge del 
moto, sarà:
\[
 ds=\frac{1}{2}g\tonda{t+dt}^2-\frac{1}{2}gt^2=
 \frac{1}{2}g\tonda{t^2+2tdt+dt^2}-\frac{1}{2}gt^2=
 gtdt+\frac{1}{2}dt^2. 
\]
Dividendo il tutto per $dt$ si ottiene la velocità istantanea:
\[
 v=\frac{ds}{dt}=\frac{gtdt+\frac{1}{2}dt^2}{dt}=gt+\frac{1}{2}dt.
\]
Il numero $gt+\frac{1}{2}dt=9,81t+\frac{1}{2}dt$ è un iperreale, di tipo
inn. Per avere un valore reale, applichiamo la parte standard:
\[
 \pst{9,81t+\frac{1}{2}dt}=9,81t
\]
Questa è la velocità istantanea che cerchiamo: dipende dal tempo $t$,
cioè cresce con il passare dei secondi. $v=0$ all'inizio ($t=0$) e
$v=9,81\times 3,36=32,96 m/s\approx 119 km/h$ alla fine, prima di fermarsi al suolo.\\
La formula ci permette il calcolo della velocità a qualsiasi istante:
in quale tempo intermedio la velocità sarà quella media, calcolata prima? 


\section{Continuità}
\label{04_diffcomtinuità}
La semplicità dei calcoli precedenti lascia intuire la ragione del successo del
calcolo con gli infinitesimi.  Questo tipo di calcolo fiorì per 150 anni a 
partire dall'epoca di Newton e Leibniz. Ma suscitava vivaci polemiche fra
gli specialisti, perché non si era in grado di spiegare come mai alla fine i 
risultati, espressi in numeri infinitesimi, possono diventare numeri standard. 
Oggi i matematici conoscono meglio la materia e queste difficoltà sono superate.
Siamo quindi in grado di procedere nello studio dei questa nuova branca della 
matematica, che si chiama analisi infinitesimale.

\subsubsection{Continuità, intervalli, differenze}
\label{04_diffcontinterv}
C'è un punto critico nei ragionamenti svolti a proposito della caduta dei gravi,
un punto che si dà sempre per scontato in fisica, ma non lo è per i matematici
e per i logici.\\
Tutto il ragionamento vale perché si presuppone che il tempo scorra in modo
uniforme. Se il tempo scorresse a scatti, anche minuscoli, quei calcoli non
sarebbero possibili. Si dice infatti che il tempo $t$ è una variabile
continua, cioè assume tutti i valori, dal minimo al massimo, con regolarità,
senza salti.

\begin{definizione}
 Una variabile continua assume con continuità tutti i valori per cui è definita,
 dal minimo e al massimo.
\end{definizione}

Il più semplice esempio di una variabile continua in matematica è la 
posizione $x$ sull'asse reale dei numeri. Infatti sappiamo che la retta reale non
ha buchi. A maggior ragione, è una variabile continua anche la posizione sull'asse 
degli Iperreali: $x,x\in\IR$. \\
Una variabile discontinua si dice 
\emph{variabile discreta}. Un semplice esempio di variabile discreta è $n, n\in \N$.\\
Nel calcolo precedente, $t$ varia con continuità da $0$ a $3,6$,
assumendone tutti i valori, dal minimo al massimo. In matematica si scrive così:
\(t\in \quadra{0;\ 3,6}\). Le parentesi quadre sono importanti, indicano che gli estremi
dell'intervallo sono valori possibili, sono inclusi.\\
I tipi possibili di intervallo sono:

\begin{center}
\begin{tabular}{ccc}\toprule
intervallo & sigla & significato\\\midrule
chiuso &$\quadra{a,\ b}$  & estremi compresi\\
aperto/chiuso & $(a,\ b]$ & a escluso, b compreso\\
chiuso/aperto & $[a,\ b)$ & a compreso, b escluso\\
aperto & $(a,\ b)$ & estremi esclusi\\\bottomrule
\end{tabular}
\label{tab:diff_tipiinterv}
\end{center}

Tutti i tipi di intervallo precedenti, della retta reale o iperreale, sono
continui, a meno di indicazioni diverse. Se un intervallo $[a,\ b]$ contiene un punto
(o più punti) di discontinuità, per esempio $d$, allora occorre usare indicazioni
diverse: $[a,\ d)\cup(d,\ b]$\\ 
Fra due numeri della retta iperreale $a, b \in \IR$ si può definire la differenza
$a-b$, che può essere positiva, negativa o nulla. Indicheremo con $\Delta$ la 
differenza fra due numeri standard, cioè una differenza finita, mentre, se la 
differenza è infinitesima, sarà indicata con $\delta$ oppure $\epsilon$
o altra lettera minuscola dell'alfabeto greco.\\
In analisi infinitesimale, le differenze infinitesime sono protagoniste.

\subsubsection{Continuità e funzioni}
\label{04_diffcontfunzioni}
Il tema della continuità è vasto e importante e viene trattato nei dettagli
nel prossimo capitolo. Per ora ci limitiamo a considerazioni di carattere intuitivo 
sulle funzioni continue.\\
Se una funzione è continua, ne puoi tracciare il grafico nel piano cartesiano 
senza staccare la matita dal foglio. Se ci fosse un punto (o più punti)
di discontinuità, saresti obbligato a interrompere il disegno e riprenderlo
da punti vicini.
\begin{esempio}
 La funzione $f(x)=x$, che ha per grafico la retta $y=x$ è evidentemente 
 una funzione continua: puoi tracciarne il grafico senza interruzioni
 nell'intervallo $(-M,\ M)$. Sono anche continue tutte le funzioni che hanno
 per grafico una retta, come per esempio $f(x)=-\frac{4}{5}x+9$.
\end{esempio}
\begin{esempio}
 La funzione $f(x)=\frac{1}{x}$ è continua ovunque, tranne che per $x=0$.
 Infatti se $x=0$, $f(x)$ non è calcolabile, quindi nel piano cartesiano 
 non puoi disegnare un punto  che rappresenta il valore standard 
 $(1;\ \frac{1}{0})$. Il punto è comunque visibile nel piano iperreale, con un
 telescopio.\\
 \end{esempio}
\begin{esempio}
 Per ragioni simili, sono discontinue in uno o più punti le funzioni
 (algebriche o trascendenti), per le quali occorra specificare condizioni 
 di esistenza relative a questi punti.
 Così $f(x)=\frac{1}{x^2-1}$ è discontinua per $x=\pm 1$, mentre 
 $f(x)=\frac{1}{x^2+1}$ è continua.
\end{esempio}

Dagli esempi si capisce che \emph{la continuità delle funzioni è una condizione 
di carattere locale}, cioè per punti. Infatti si possono riconoscere dei punti
di discontinuità di una funzione, non degli insiemi di discontinuità.
Se ci si accorge che un punto $(x_0;\ y_0)$ è di discontinuità
per $f(x)$, allora si dice: \emph{$f(x)$ è discontinua per $x=x_0$},
cioè si indica solo la coordinata $x$ che pone questo problema.
\begin{esempio}
 La funzione $f(x)= \tan x$ è discontinua per $x=\frac{\pi}{2}\pm k\pi$.
\end{esempio}

\begin{definizione}
Se una funzione è continua in tutti i punti di un intervallo, allora si dice 
continua in quell'intervallo.
\end{definizione}
\begin{esempio}
 $f(x)=\ln x$ è definita per $x \in (0;\ M)$ ed è ivi continua.
\end{esempio}





\section{Differenziale}
\label{sec:04_differenziale}
Dato un numero iperreale $x\in\IR$, finito oppure infinitesimo, si può scrivere il 
valore del numero a lui infinitamente vicino: si tratta di $x+\epsilon$. La differenza, 
fra i due valori infinitmente vicini è $dx=(x+\epsilon)-x=\epsilon$.
La sigla $dx$ (che si legge \emph{de x}) indica la differenza infinitesimale calcolata
a partire dal punto $x$ e si chiama \emph{differenziale di $x$}.

\begin{esempio}
 Calcola il differenziale della varibile $x$ nel punto $x=-7$.\\
 $dx_{|_7}=(7+\epsilon)-7=\epsilon$
\end{esempio}

\begin{esempio}
 Calcola il differenziale di varibile $\frac{10}{13}x$ nel punto $x=9$.\\
 $dx_{|_9}=\quadra{\frac{10}{13}\cdot(9+\epsilon)}-\frac{10}{13}\cdot 9=\frac{10}{13}\epsilon.$
 \end{esempio}

\begin{osservazione}
 L'infinitesimo $\epsilon$ potrebbe anche essere negativo. Questo non cambierebbe 
 il calcolo.
\end{osservazione}

\begin{definizione}
 Il differenziale è la differenza infinitesimale.
\end{definizione}

Per calcolare il differenziale di una variabile, quindi, bisogna sapere a partire 
da quale valore si vuole svolgere il calcolo e il risultato vale solo a partire 
dal valore richiesto. Se tale valore è indifferente, allora si può evitare di 
indicarlo. In tale caso il risultato vale in tutto l'intervallo continuo per il
quale la variabile è definita (vedi i prossimi esempi).



\subsection{Differenziale e funzioni}
\label{subse:04:difffun}
\begin{osservazione}
 Il differenziale di una funzione è calcolabile solo negli intervalli in cui
 la funzione è continua.
\end{osservazione}

Dato che $dx$ è il differenziale di $x$, allora $dx$ è il differenziale anche 
della funzione identica $f(x)=x$. Il suo grafico nel piano cartesiano è dato 
dalla retta $y=x$. Che significato dobbiamo dunque attribuire al differenziale $dy=dx$?

disegno retta y=x 

L'uguaglianza dei due differenziali ci indica che due punti  infinitamente
vicini sulla retta individuano due uguali differenze infinitesime sugli assi.\\
Succede la stessa cosa anche per altre funzioni che hanno per grafico una retta?
\begin{esempio}
 Proviamo ad esempio a differenziare la funzione $f(x)=\frac{2}{3}x$.\\
\( df(x)=\frac{2}{3}(x+dx)-\frac{2}{3}x=\frac{2}{3}dx\).\\
Questa volta il grafico della funzione $y=\frac{2}{3}x$ mostra che
l'incremento infinitesimo dei valori $x$ provoca un incremento corrispondente 
a $\frac{2}{3}$ sui valori $y$.
\end{esempio}
\begin{esempio}
Proviamo con un'altra funzione che ha per grafico una retta: \(f(x)~=~-5x+2\):
\[
 df(x)= f(x+dx)-f(x)=-5(x+dx)+2-(-5x+2)=-5x-5dx+2+5x-2=-5dx
\]
Quindi $dy=-5dx$. 
\end{esempio}

Come si vede la costante $2$ differenziando sparisce. Infatti
se la funzione è costante, cioè del tipo $f(x)=k$, il suo differenziale è nullo,
perché essendo una funzione costante, i suoi valori non possono cambiare.

\begin{teorema}
Il differenziale di una costante è nullo.
\end{teorema}

\noindent Ipotesi: \(f(x)=k\).\tab Tesi: \(df(x)=0\).

\begin{proof}
\[
 df(x)=f(x+dx)-f(x)=k-k=0
\]
\end{proof}


% \begin{teorema}[Continuità delle costanti]
% Le funzioni costanti sono continue.
% \end{teorema}
% 
% \noindent Ipotesi: \(f(x)=k\).\tab Tesi: \(f(x)\) è continua.
% 
% \begin{proof}
% Per la definizione di continuità vogliamo dimostrare che 
% \[\forall x \text{ se } x_0 \approx x \text{ allora } f(x_0) \approx f(x)\]
% Poniamo \(x_0=x+\epsilon\), essendo la funzione costante, anche 
% \[f(x+\epsilon)=k\] 
% che, ovviamente, è infinitamente vicino a \(k\). In simboli:
% \[f(x_0) = f(x+\epsilon) = k \approx k = f(x)\] 
% \end{proof}
 
\section{Derivata}
\label{sec:02_logaritmiche}


\section{Regole di derivazione}
\label{sec:02_logaritmiche}

