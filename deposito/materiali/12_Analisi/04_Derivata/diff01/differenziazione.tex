% (c) 2015 Daniele Zambelli daniele.zambelli@gmail.com
%  Bruno Stecca

% (c) 2014 Daniele Zambelli - daniele.zambelli@gmail.com
% 
% Tutti i grafici per il capitolo relativo alle parabole
%
% 

\newcommand{\espdueterzi}{% 
    % Esponenziali con basi diverse.
    \disegno{
    \rcom{-10}{+10}{-1}{10}{gray!50, very thin, step=1}
    \begin{scope}[ultra thick, color=Maroon!50!black]
     \tkzInit[xmin=-10.3, xmax=+10.3, ymin=-0.3, ymax=+10.3]
     \tkzFct[domain=-10.3:+6]{(3./2)**x}
     \tkzFct[color=Green!50!black, domain=-6:+10.3]{(2./3)**x}
     \begin{scope}[color=Black!50!black]
      \filldraw (1, 3./2) circle (1.2pt);
      \filldraw (1, 2./3) circle (1.2pt);
     \end{scope}
     \filldraw [color=Red](0, 1) circle (1.2pt);  
    \end{scope}
    \begin{scope}[color=black]
     \draw (-7.3, 7) node{\(f(x)=\tonda{\dfrac{2}{3}}^x\)}; 
     \draw ((7.3, 7) node{\(f(x)=\tonda{\dfrac{3}{2}}^x\)};
    \end{scope}
    }
}

\newcommand{\logduebasi}{% 
    % Esponenziali con basi diverse.
    \disegno{
    \rcom{-1}{+10}{-9}{9}{gray!50, very thin, step=1}
    \begin{scope}[ultra thick, color=Maroon!50!black]
      \tkzInit[xmin=-1.3, xmax=+80, xstep=.5, ymin=-10.3,ymax=+10.3]
      \tkzFct[domain=.01:+10]{log(x)/log(2)}
      \filldraw (2, 1) circle (1.2pt);
      \begin{scope}[color=Green!50!black]
        \tkzFct[domain=-.01:+10]{log(x)/log(1./2)}
        \filldraw (2, -1) circle (1.2pt);
      \end{scope}
    \end{scope}
    \begin{scope}[color=black]
      \draw (9.5, 2.8) node{a=2}; 
      \draw (9.5, -2.8) node{a=0.5};
    \end{scope}
      \filldraw [color=Red] (1,0) circle (1.2pt);
    }
}


\chapter{Derivate}
% \begin{wrapfloat}{figure}{r}{0pt}
% \includegraphics[scale=0.35]{img/fig000_.png}
% \caption{...}
% \label{fig:...}
% \end{wrapfloat}
% 
% \begin{center} \input{\folder lbr/fig000_.pgf} \end{center}

Il problema di determinare la velocità istntanea ci ha portati a conoscere 
i numeri infinitesimi e, attraverso questi, l'insieme dei numeri iperreali.
Ora siamo in grado di cercare la risposta alla domanda rimasta in sospeso: 
come si determina la velocità istantanea?\\
La risposta, che conosciamo nelle forme moderne, da più di 400 anni,
propone al nostro studio un nuovo potentissimo strumento di calcolo, adatto a 
risolvere problemi in ogni ambito scientifico: la derivata.

\section{Velocità di caduta}
\label{04_diffvelcaduta}
Nel Settecento fiorirono alcune leggende su Galileo Galilei. Una di queste racconta 
che per dimostrare che i gravi cadono con la stessa velocità, gettò dalla Torre
di Pisa due sfere di peso diverso, ma di uguali dimensioni. I due oggetti, come
oggi possiamo immaginare, raggiunsero il suolo contemporaneamente.\\
La Torre di Pisa è alta circa $56m$ e immaginiamo, per semplificare, che la 
distanza percorsa dai due oggetti sia di $56m$ (ti lascio calcolare il 
percorso effettivo: tieni presente che al giorno d'oggi l'inclinazione della 
Torre è di $4,8^\circ$).\\
Oggi sappiamo che un oggetto in caduta libera ha la seguente legge del moto:
\(s=\frac{1}{2}gt^2\). Come al solito, $s$ è lo spazio in metri, $t$ è il 
tempo in secondi, $g=9,81 m/s^2$ è l'accelerazione di gravità, costante nei
pressi della superficie terrestre.\\
Se cerchiamo la velocità media, basta dividere lo spazio percorso per il tempo
impiegato:

\begin{align*}
 & s_{tot} = 56m\\
 & s =\frac{1}{2}gt^2 \srarrow t_{tot}= \sqrt{\frac{2s_{tot}}{g}}=
 \sqrt{\frac{2\times 56}{9,81}}=3,36 s.\\
 & v_m= \frac{s_{tot}}{t_{tot}}=\frac{56 m}{3,36 s}=16,67 m/s,
\end{align*}
che corrispondono a circa $60 km/h$ di media.\\
Ma gli oggetti partono fermi e arrivano velocissimi: 
è possibile sapere quale è loro velocità in ogni istante? È il momento
di usare le quantità infinitesime.\\
Chiamiamo $dt$ un intervallo di tempo infinitesimo, fra due istanti successivi
$t$ e $t+dt$.
Lo spazio percorso nella caduta, in quell'intervallo di tempo, applicando la legge del 
moto, sarà:
\[
 ds=\frac{1}{2}g\tonda{t+dt}^2-\frac{1}{2}gt^2=
 \frac{1}{2}g\tonda{t^2+2tdt+(dt)^2}-\frac{1}{2}gt^2=
 gtdt+\frac{1}{2}(dt)^2. 
\]
Dividendo il tutto per $dt$ si ottiene la velocità istantanea, quella che cambia
in ogni istante $t$:
\[
 v(t)=\frac{ds}{dt}=\frac{gtdt+\frac{1}{2}dt^2}{dt}=gt+\frac{1}{2}dt.
\]
L'espressione $gt+\frac{1}{2}dt=9,81t+\frac{1}{2}dt$ diventa un numero ben
preciso per ogni valore di $t$, un iperreale finito che è la
somma di un numero standard e di un numero infinitesimo. 
Per averne il valore reale, applichiamo la parte standard:
\[
 \pst{9,81t+\frac{1}{2}dt}=9,81t
\]
Questa è la velocità istantanea che cerchiamo: dipende dal tempo $t$,
cioè cresce con il passare dei secondi. 

\begin{center}
\begin{tabular}{cc}\toprule
$t$ (in $s$) & $v=9,81\times t$ (in $m/s$) \\\midrule
$0$ & $0$  \\
$1$ & $9,81\times 1 =9,81$ \\
$2$ & $9,81\times 2 =19,62$ \\
... & ... \\
$3,6$ & $9,81\times 3,36= ...$\\\bottomrule
\end{tabular}
\label{tab:diff_velocita}
\end{center}

La formula $v=9,81\times t$ ci permette il calcolo della velocità per ogni valore di $t$.
Per quale valore di \(t\) la velocità sarà uguale a quella media? 


\section{Continuità}
\label{04_diffcomtinuità}
La semplicità dei calcoli precedenti lascia intuire la ragione del successo del
calcolo con gli infinitesimi.  Questo tipo di calcolo fiorì per 150 anni a 
partire dall'epoca di Newton e Leibniz. Ma suscitava vivaci polemiche fra
gli specialisti, perché non si era in grado di spiegare come mai i 
risultati, espressi attraverso numeri infinitesimi, alla fine diventano numeri
``di uso comune``.
Oggi i matematici conoscono meglio la materia e queste difficoltà sono superate.
Siamo quindi in grado di procedere nello studio dei questa nuova branca della 
matematica, che si chiama \emph{Analisi infinitesimale}.

\subsubsection{Continuità, intervalli, differenze}
\label{04_diffcontinterv}
C'è un punto critico nei ragionamenti svolti a proposito della caduta dei gravi,
un punto che si dà sempre per scontato in fisica, ma non lo è per i matematici
e per i logici.\\
Tutto il ragionamento vale perché si presuppone che il tempo scorra in modo
uniforme. Se il tempo scorresse a scatti, anche minuscoli, quei calcoli non
sarebbero possibili. Si dice infatti che il tempo $t$ è una variabile
continua, cioè assume tutti i valori, dal minimo al massimo, con regolarità,
senza salti.

\begin{definizione}
 Una variabile è una grandezza che può assumere valori diversi.
 L'insieme dei valori possibili costituisce il suo insieme di definizione.
\end{definizione}

\begin{definizione}
 Una variabile continua è definita in un intervallo di valori continuo.
 Le variazioni dei suoi valori possono essere arbitrariamente piccole.
\end{definizione}

Il più semplice esempio di una variabile continua in matematica è la 
posizione $x$ sull'asse reale dei numeri. Infatti sappiamo che la retta reale non
ha buchi. A maggior ragione, è una variabile continua anche la posizione sull'asse 
degli Iperreali: $x$, con $x\in\IR$. \\
Viceversa una variabile che pesca i suoi valori in un insieme formato da numeri isolati,
cioè con differenze finite fra l'uno e l'altro, si dice \emph{variabile discreta}. 
\begin{definizione}
 Una variabile discreta assume valori che variano per quantità finite.
\end{definizione}

Un semplice esempio di variabile discreta è $n, n\in \N$.\\
Nel calcolo precedente, $t$ varia con continuità da $0$ a $3,6$,
assumendone tutti i valori, dal minimo al massimo. In matematica si scrive così:
\(t\in \quadra{0;\ 3,6}\). Le parentesi quadre sono importanti, indicano che gli estremi
dell'intervallo sono valori possibili, sono inclusi.\\
I tipi possibili di intervallo sono:

\begin{center}
\begin{tabular}{ccc}\toprule
intervallo & sigla & significato\\\midrule
chiuso &$\intervcc{a}{b}$  & estremi compresi\\
aperto/chiuso & $\intervac{a}{b}$ & a escluso, b compreso\\
chiuso/aperto & $\intervca{a}{b}$ & a compreso, b escluso\\
aperto & $\intervaa{a}{b}$ & estremi esclusi\\\bottomrule
\end{tabular}
\label{tab:diff_tipiinterv}
\end{center}

Tutti i tipi di intervallo precedenti, nella retta reale o iperreale, sono
continui, a meno di indicazioni diverse. Se un intervallo $\intervcc{a}{b}$ contiene un punto
(o più punti) di discontinuità, per esempio $d$, allora occorre usare indicazioni
diverse: $\intervca{a}{d}\cup\intervac{d}{b}$\\ 
La differenza $a-b$ fra due numeri della retta iperreale $a, b \in \IR$, 
può essere positiva, negativa o nulla. Indicheremo con $\Delta$ la 
differenza fra due numeri standard, cioè una differenza finita, mentre, se la 
differenza è infinitesima, sarà indicata con $\delta$ oppure $\epsilon$
o altra lettera minuscola dell'alfabeto greco.\\
In analisi infinitesimale, le differenze infinitesime sono protagoniste.


\section{Differenziale}
\label{sec:diff01_differenziale}

\subsubsection{Parte principale}
\label{subsubsec:diff01_parteprincipale}
I risultati dei calcoli che seguono in molti casi hanno la forma di una somma
fra infinitesimi di ordine diverso, come avviene nel prossimo
esempio sul differenziale della funzione quadratica
e più avanti con le funzioni potenza.\\
In una somma di infinitesimi, gli infinitesimi di
grado superiore (che sono quelli più vicini allo zero) pesano 
sul risultato infinitamente meno degli altri: sono più trascurabili.
Quando in una somma di infinitesimi si trascurano quelli di minor peso,
si dice che si prende la \emph{parte principale della somma}.
Lo si può fare perché la somma esatta e quella approssimata sono 
numeri indistinguibili.

\begin{definizione}
 Il differenziale è la differenza infinitesimale.
\end{definizione}

Dato un numero iperreale $x_0\in\IR$, finito oppure infinitesimo, si può scrivere il 
valore del numero a lui infinitamente vicino: si tratta di $x_0+\epsilon$. La differenza 
fra i due valori infinitamente vicini è $dx|_{x=x_0}=(x_0+\epsilon)-x_0=\epsilon$.
È calcolata a partire dal punto $x_0$ e si chiama \emph{differenziale di $x$ nel 
punto $x_0$}.\\
La sigla $dx$ (che si legge \emph{de x}) indica la differenza infinitesimale.
La sigla $dx|_{x=x_0}$ si legge: \emph{de x, per x uguale a x zero}. 

\begin{esempio}
 Calcola il differenziale della variabile $x$ nel punto $x=-7$. Ripeti poi il 
 calcolo in altri punti.\\
  $dx|_{x=-7} =(-7+\epsilon)-(-7)=\epsilon$\\
  $dx|_{x=7} =(7+\epsilon)-7)=\epsilon$\\
  $dx|_{x=3} =(3+\epsilon)-3=\epsilon$\\
  $dx|_{x=\frac{1}{4}} = \tonda{\frac{1}{4}+\epsilon}-\frac{1}{4}=\epsilon$\\
  $...\quad = \qquad ...$\\
  $dx|_{x=a}  =(a+\epsilon)-a=\epsilon$\\
  $dx|_{x=x_0}  =\ ...\ =\epsilon$\\
  Se il risultato del differenziale è indifferente da $x_0$, allora si evita di 
  indicare $|_{x=x_0}$: $dx=\epsilon$, $\forall x$. 
\end{esempio}

\begin{esempio}
 Calcola il differenziale della variabile $\frac{10}{13}x$ nei punti $x=9$ e 
 $x=-\frac{1}{5}$.\\
 $d\tonda{\frac{10}{13}x}|_{x=9}=\quadra{\frac{10}{13}\cdot(9+\epsilon)}-\frac{10}{13}\cdot 9=
 \frac{10}{13}\epsilon.$\\
 $d\tonda{\frac{10}{13}x}|_{x=-\frac{1}{5}}=\quadra{\frac{10}{13}\cdot\tonda{-\frac{1}{5}+\epsilon}}-
 \frac{10}{13}\cdot \tonda{-\frac{1}{5}}= \frac{10}{13}\epsilon.$\\
 $d\tonda{\frac{10}{13}x}=\frac{10}{13}\epsilon$, $\forall x$.
 \end{esempio}

Anche se due risultati uguali non bastano per fare una prova, e nemmeno i sei 
del primo esempio, si può essere sicuri che mille altri tentativi non
sortirebbero un esito diverso. La prova si ottiene utilizzando $x_0$ 
(oppure una costante analoga) al posto di un valore numerico.

\begin{osservazione}
 L'infinitesimo $\epsilon$ potrebbe anche essere negativo. Questo non cambierebbe 
 il calcolo.\\
\end{osservazione}

L'uso di un valore numerico al posto di $x_0$ è essenziale per precisare il
punto a partire dal quale si vuole svolgere il calcolo. Negli esempi precedenti
tale indicazione è risultata indifferente, ma nella maggior parte dei casi, invece, 
ha un diretto influsso sul risultato.\\

\begin{esempio}
 Calcola $df(x)|_{x=5}$, con $f(x)=x^2$. Calcola poi $df(x)|_{x=-5}$ e infine $df(x)|_{x=2}$.
 \begin{align*} 
  d(x^2)|_{x=5} & =(5+\epsilon)^2-5^2=25+10\epsilon+\epsilon^2-25=10\epsilon+\epsilon^2\\
  d(x^2)|_{x=-5}& =(-5+\epsilon)^2-(-5)^2=25-10\epsilon+\epsilon^2-25=-10\epsilon+\epsilon^2\\
  d(x^2)|_{x=2} & =(2+\epsilon)^2-2^2=4+4\epsilon+\epsilon^2-4=4\epsilon+\epsilon^2\\
 \end{align*}
\end{esempio}


\begin{osservazione}
 Non abbiamo fatto alcuna ipotesi su $\epsilon$. Potrebbe essere un infinitesimo positivo 
 o negativo, potrebbe essere il triplo o il quadrato di un altro infinitesimo. Il risultato 
 non cambia e ha valore per qualsiasi $\epsilon$.
\end{osservazione}



\subsection{Differenziale e funzioni}
\label{subsec:diff01_difffun}
Iniziamo a differenziare le funzioni più semplici, in un generico punto $x_0$.
Se ci accorgeremo che il risultato non dipende da $x_0$, ne trarremo regole
di carattere generale.\\
Ma prima di tutto, una precisazione essenziale:

\begin{osservazione}
 Il differenziale di una funzione è calcolabile solo negli intervalli in cui
 la funzione è continua.
\end{osservazione}

\subsubsection{Funzione costante}
\label{subsec:diff01_diffcostante}
Se $f(x)=k$, si ha:\\
$df(x)|_{x=x_0}=f(x_0+\epsilon)-f(x_0)=k-k=0$\\
Se la funzione è costante il suo differenziale è nullo,
perché, essendo una funzione costante, i suoi valori non possono cambiare. Resta così 
dimostrato il seguente

\begin{teorema}
Il differenziale di una costante è nullo.
\end{teorema}

grafico y=k con differenziale nullo su asse Y\\

Nel piano cartesiano, la funzione  $y=k$ è una retta orizzontale e, come tutte
le rette, è una funzione continua. Quindi il risultato non dipende da $x_0$ e
vale su tutto l'asse iperreale.
\subsubsection{Funzione identica}
\label{diffidentica}
Se $f(x)=x$, allora, banalmente: $df(x)=dx=\epsilon$. Il risultato è
generale, cioè non dipende da $x_0$. Infatti:\\
$df(x)|_{x=x_0}=f(x_0+\epsilon)-f(x_0)=(x_0+\epsilon)-x_o=\epsilon$.\\
È dimostrato così il seguente
\begin{teorema}
Il differenziale della funzione identica è $dx=\epsilon$.
\end{teorema}
D'ora in poi useremo indifferentemente $dx$ oppure $\epsilon$, dato che
sono equivalenti.\\
Il grafico di $f(x)=x$ nel piano cartesiano è dato dalla retta $y=x$. 
Che significato dobbiamo attribuire a $dy=dx$?
L'uguaglianza dei due differenziali indica che due punti infinitamente
vicini sulla retta individuano sugli assi due differenze infinitesime uguali.\\

grafico y=x con uguali differenziali sugli assi\\

Succederebbe la stessa cosa con altre rette, più o meno inclinate passanti o
non passanti dall'origine?

\subsubsection{Funzione lineare}
\label{subsec:diff01_flineare}

\begin{esempio}
 Proviamo a differenziare in $x_0$ la funzione $f(x)=\frac{2}{3}x$.\\
\( df(x)|_{x=x_0}=\frac{2}{3}(x_0+dx)-\frac{2}{3}x_0=\frac{2}{3}dx\).\\
Questa volta il grafico della funzione $y=\frac{2}{3}x$ mostra che
l'incremento infinitesimo dei valori $x$ provoca un incremento corrispondente 
a $\frac{2}{3}$ sui valori $y$. Il risultato è generale, cioè vale $\forall x_0$.
\end{esempio}
\begin{esempio}
Proviamo con un'altra funzione che ha per grafico una retta: \(f(x)~=~-5x+2\):
\[
df(x)|_{x=x_0}= f(x_0+dx)-f(x_0)=[-5(x_0+dx)+2]-(-5x_0+2)=-5x_0-5dx+2+5x_0-2=-5dx
\]
Quindi $dy=-5dx, \forall x_0 \in \IR$. 
\end{esempio}

\begin{teorema}
 Il differenziale di una funzione lineare $f(x)=mx+q$ è $mdx$, $\forall x\in \IR$.
\end{teorema}

\noindent Ipotesi: \(f(x)=mx+q\).\tab Tesi: \(df(x)=mdx\).

\begin{proof}
\[
 df(x)|_{x=x_0}= f(x_0+dx)-f(x_0)=[(mx_0+dx)+q]-(mx_0+q)=mx_0+mdx+q-mx_0-q=mdx
\]
Poiché nel risultato non compare $x_0$, $df(x)$ non dipende dal punto $x_0$.
\end{proof}

\subsubsection{Funzione quadratica}
\label{subsubsec:diff01_diffquad}

\begin{teorema}
 Il differenziale della funzione quadratica $f(x)=x^2$ è $2xdx+(dx)^2$, $\forall x\in \IR$.
\end{teorema}

\noindent Ipotesi: \(f(x)=x^2\).\tab Tesi: \(df(x)=2xdx+(dx)^2\).

\begin{proof}
\[
 df(x)|_{x=x_0}= f(x_0+dx)-f(x_0)=(x_0+dx)^2-x_0^2=x_0^2+x_0dx+(dx)^2-x_0^2=2x_0dx+(dx)^2
\]
Questa volta nel risultato compare $x_0$. Quindi il valore del differenziale 
della funzione cambia al cambiare del punto $x_0$ che viene incrementato.
\end{proof}



\subsubsection{Funzioni potenza}
\label{subsubsec:diff01_diffpot}
Ricaviamo per gradi il differenziale della funzione potenza è $f(x)=x^n$, con
un procedimento per induzione.\\
Iniziamo dai casi già noti $f(x)=x$ e $f(x)=x^2$ e esaminiamo i successivi
aumentando progressivamente l'esponente.

\begin{align*}
  d(x)=x+dx-x & =dx\\
  d(x^2)=(x+dx)^2-x^2 & \sim 2xdx\\
  d(x^3)=(x+dx)^3-x^3 & =[x^3+3x^2dx+3x(dx)^2+(dx)^3]-x^3=
                      3x^2dx+3x(dx)^2+(dx)^3=\\
                      &= 3x^2dx+\delta(x)\sim 3x^2dx\\
  d(x^4)=(x+dx)^4-x^4 & = [x^4+4x^3dx+6x^2(dx)^2+4x(dx)^3+(dx)^4]-x^4=\\
                      &=4x^3dx+6x^2(dx)^2+4x(dx)^3+(dx)^4=
                      4x^3dx+\delta(x)\sim 4x^3dx\\
  ...                 & ...\\
  d(x^7)=(x+dx)^7-x^7 & = x^7+7x^6(dx)+21x^5(dx)^2...+21x^2(dx)^5+7x(dx)^6+(dx)^7 - x^6=\\
  & = x^7+\delta(x)\sim 7x^6dx\\
  ...                 & ...\\
  d(x^{10})=(x+dx)^{10}-x^{10} & =x^{10}+10 x^9dx+\ ...\ -x^{10}=
  \ ...\ =x^{10} +\delta(x)\sim 10 x^9dx      
\end{align*}
L'unico risultato esatto è il primo; gli altri si sintetizzano scrivendo $\delta(x)$
che rappresenta gli infinitesimi di ordine superiore. $\delta(x)$ aggiunge
delle quantità infinitamente meno importanti del primo addendo.
A sinistra e a destra dell'ultimo segno di uguale si trovano quindi 
numeri indistinguibili, cioè le loro differenze non si possono cogliere perché
sono ''più infinitesime`` di quanto lo sono questi numeri.\\

Ora che il meccanismo è chiaro e possiamo ritenere sufficientemente dimostrato
il teorema seguente.

\begin{teorema}
 Il differenziale della funzione potenza $f(x)=x^n$ è $df(x)\sim nx^{n-1}dx$
\end{teorema}

\begin{osservazione}
Anche se abbiamo usato solo esponenti interi, si dimostra che 
la regola vale per qualsiasi esponente reale. Lo puoi verificare nei due casi 
che seguono, riscrivendo le funzioni come potenze.
\end{osservazione}


\subsubsection{Funzione radice quadrata}
\label{subsubsec:diff01_diffradq}

\begin{teorema}
 Il differenziale della funzione radice quadrata $f(x)=\sqrt{x}$ è
 $\sim\frac{dx}{2\sqrt{x_0}}$, $\forall x\in \IR$, $x\neq 0$.
\end{teorema}

\noindent Ipotesi: \(f(x)=\sqrt{x}\).\tab Tesi: \(df(x)\sim\frac{dx}{2\sqrt{x_0}}\).

\begin{proof}
\begin{align*}
 df(x)|_{x=x_0} &= f(x_0+dx)-f(x_0)=\sqrt{x_0+dx}-\sqrt{x_0}=\\
 &=\tonda{\sqrt{x_0+dx}-\sqrt{x_0}}\times\frac{\sqrt{x_0+dx}+
 \sqrt{x_0}}{\sqrt{x_0+dx}+\sqrt{x_0}}=\\
 &=\frac{x_0+dx-x_0}{\sqrt{x_0+dx}+\sqrt{x_0}}\sim\frac{dx}{2\sqrt{x_0}}
\end{align*}
Anche questa volta il risultato dipende da $x_0$. Si tratta di 
un risultato indistinguibile dal risultato esatto. L'approssimazione si ottiene
con le tecniche del capitolo 3.
\end{proof}

\subsubsection{Funzione reciproca}
\label{subsubsec:diff01_diffrecip}
\begin{teorema}
 Il differenziale della funzione reciproca $f(x)=\frac{1}{x}$ è $-\frac{dx}{x^2}$, 
 $\forall x\in \IR$, con $x\neq 0$.
\end{teorema}

\noindent Ipotesi: $f(x)=\frac{1}{x}$.\tab Tesi: $df(x)\sim-\frac{dx}{x^2}$.

\begin{proof}
\[
 df(x)|_{x=x_0}= f(x_0+dx)-f(x_0)=\frac{1}{(x_0+dx)}-\frac{1}{x_0}=
 \frac{x_0-x_0-dx}{x_0(x_0+dx)}=\frac{-dx}{x_0^2+x_0dx}\sim-\frac{dx}{x_0^2}
\]
Anche questa volta il valore del differenziale dipende da $x_0$. 
Per semplicità, teniamo il risultato indistinguibile.
\end{proof}

\subsubsection{Differenziali problematici}
\label{subsubsec:diff01_diffproblemi}

Quest'ultimo calcolo ci porta un punto importante: dato che nel risultato
$x_0$ si trova al denominatore, abbiamo un problema. Che succede se $x_0=0$?
\begin{esempio}
 Calcola $df(x)|_{x=0}$, con $f(x)=\frac{1}{x}$.\\
 $d\tonda{\frac{1}{x}}|_{x=0}=\frac{1}{0+dx}-\frac{1}{0}=$ ?\\
 La funzione è differenziabile $\forall x$, ma non per $x=0$. Se $x\approx 0$
 il differenziale diventa la differenza fra due infiniti, una forma di 
 indecisione che non siamo in grado di risolvere. Il problema viene dal fatto 
 che in $x=0$, $f(x)$ non è definita.
\end{esempio}

\begin{esempio}
Differenzia la funzione $f(x)=\frac{1}{x^2-1}$ per $x_0=1$ e $x_0=-1$.\\
$d\tonda{\frac{1}{x^2-1}}|_{x=1}=\frac{1}{(x+dx)^2-1}-\frac{1}{x^2-1}=
\frac{1}{2dx+(dx)^2}-\frac{1}{0}=$?\\
$d\tonda{\frac{1}{x^2-1}}|_{x=-1}=\frac{1}{(x+dx)^2-1}-\frac{1}{x^2-1}=
\frac{1}{-2dx+(dx)^2}-\frac{1}{0}=$?\\
Questa volta i punti critici sono due. Poiché la funzione non è calcolabile
per $x_0=1$ e $x_0=-1$, non è calcolabile nemmeno il suo differenziale.
\end{esempio}

Nel piano cartesiano tracciamo il grafico delle funzioni degli ultimi 
tre esempi: $y=\sqrt{x}$, $y=\frac{1}{x}$ e $y=\frac{1}{x^2-1}$.

GRAFICI NELLO STESSO PIANO CARTESIANO
\begin{figure}[h]
\begin{inaccessibleblock}[Grafici di funzioni diverse.]
 \begin{center}
\radice
 \end{center}
\end{inaccessibleblock}
\caption{Grafici.} \label{fig:grafici}
\end{figure}

Si può osservare che $y=\sqrt{x}$, essendo definita per i valori $x\ge 0$, 
non può essere calcolata per esempio, se $x=-2$ e quindi nemmeno il suo 
differenziale ha senso in questo punto.\\
Gli altri due grafici mettono in evidenza questo problema: dove la funzione 
non è calcolabile, non esiste il punto che rappresenta la funzione nel piano 
cartesiano reale. Nel piano iperreale, puntando l'infinito con un telescopio,
non riusciamo a focalizzare la differenza infinitesima fra due valori infiniti
della funzione.\\
Consideriamo un tipo diverso di problema.

GRAFICO DI UNA f(x) per casi CON DISCONTINUITA' A SALTO
\[
f(x)=\begin{cases} 
x-1, & \mbox{se }x<2 \\ 
x+1, & \mbox{se }x\ge 2
\end{cases}
\]
$f(x)$ ha due rami e il grafico compie un salto per $x=2$. Le differenze
infinitesime calcolate a destra di tale punto saranno diverse da quelle 
calcolate a sinistra: i differenziali sono calcolabili ma non hanno
uguali valori. Anche in questo caso $f(x)$ non è differenziabile per $x=2$

\subsubsection{Continuità e funzioni}
\label{subsubsec:diff01_diffcontinue}
Il tema della continuità è vasto e importante e viene trattato nei dettagli
nel prossimo capitolo. Per ora ci limitiamo a considerazioni di carattere intuitivo.\\
\emph{Se una funzione è continua, ne puoi tracciare il grafico nel piano cartesiano 
senza staccare la matita dal foglio}. Se ci fosse un punto (o più punti)
di discontinuità, saresti obbligato a interrompere il disegno e riprenderlo
da punti vicini.
\begin{esempio}
 La funzione $f(x)=x$, che ha per grafico la retta $y=x$ è evidentemente 
 una funzione continua: puoi tracciarne il grafico senza interruzioni
 nell'intervallo $(-M,\ M)$. Sono anche continue tutte le funzioni che hanno
 per grafico una retta, come per esempio $f(x)=-\frac{4}{5}x+9$.\\
 Quindi anche la funzione costante $f(x)=k$, che ha per grafico una retta
 orizzontale, è una funzione continua.
\end{esempio}
\begin{esempio}
 La funzione $f(x)=\frac{1}{x}$ è continua ovunque in $\R$que, tranne che per $x=0$.
 Infatti se $x=0$, $f(x)$ non è calcolabile, quindi nel piano cartesiano 
 non puoi disegnare un punto  che rappresenta il valore standard 
 $(0;\ \frac{1}{0})$. Il punto è comunque visibile nel piano iperreale, con un
 telescopio.
 \end{esempio}
\begin{esempio}
 Per ragioni simili, sono discontinue in uno o più punti le funzioni
 (algebriche o trascendenti), per le quali occorra specificare condizioni 
 di esistenza relative a questi punti.
 Così $f(x)=\frac{1}{x^2-1}$ è discontinua per $x=\pm 1$, mentre 
 $f(x)=\frac{1}{x^2+1}$ è continua.
\end{esempio}

Dagli esempi si capisce che \emph{la continuità delle funzioni è una condizione 
di carattere locale}, cioè per punti. Infatti si possono riconoscere dei punti
di discontinuità di una funzione, non degli insiemi di discontinuità.
Se ci si accorge che un punto $(x_0;\ f(x_0)$ è di discontinuità
per $f(x)$, allora si dice: \emph{$f(x)$ è discontinua per $x=x_0$},
cioè si indica solo la coordinata $x$ che pone questo problema 
(non si usa dire: $f(x_0)$ è discontinua).
\begin{esempio}
 La funzione $f(x)= \tan x$ è discontinua per $x=\frac{\pi}{2}\pm k\pi$.
\end{esempio}

\begin{definizione}
Se una funzione è continua in tutti i punti di un intervallo $\intervcc{a}{b}$,
allora si dice continua in $\intervcc{a}{b}$.
\end{definizione}
\begin{osservazione}
 Ovviamente la definizione non cambia se l'intervallo è di tipo diverso.
\end{osservazione}

\begin{esempio}
 $f(x)=\ln x$ è definita per $x \in (0;\ M)$ ed è ivi continua.
\end{esempio}


\subsection{Combinare differenziali}
\label{subsec:diff01_combdiff}
Nella sezione \ref{subsec:diff01_flineare} e in altre ci siamo avvalsi di proprietà
così naturali che non è stato necessario sottolinearle. Ma è meglio non lasciarcele 
sfuggire.

\subsubsection{Differenziale del prodotto per una costante}
\label{}
\begin{teorema}
 Se una funzione è moltiplicata per una costante, anche il suo differenziale risulta 
 moltiplicato per la stessa costante.
\end{teorema}
\noindent Ipotesi: $f(x)=a\cdot g(x)$.\tab Tesi: $df(x)=a\cdot dg(x)$.

\begin{proof}
\[
 df(x)=d\quadra{a\cdot g(x)}= a\cdot g(x+dx)-a\cdot g(x)=a\cdot\quadra{g(x+dx)-g(x)}
 =a\cdot dg(x).
\]
\end{proof}

\subsubsection{Differenziale di una somma di funzioni}
\label{}

\begin{teorema}
 Se una funzione è la somma (la differenza) di due funzioni, anche il suo 
 differenziale sarà la somma (la differenza) dei due differenziali.
\end{teorema}
\noindent Ipotesi: $f(x)=f_1(x)\pm f_2(x)$.\tab Tesi: $df(x)=df_1(x)\pm df_2(x)$.

\begin{proof}
\begin{align*}
 df(x)=d[f_1(x)\pm f_2(x)]=[f_1(x+dx)\pm f_2(x+dx)]-[f_1(x)\pm f_2(x)]=\\
 = [f_1(x+dx)-f_1(x)]\pm [f_2(x+dx)-f_2(x)]= df_1(x)\pm df_2(x)
\end{align*}
\end{proof}

\begin{esempio}
 Un generico polinomio di secondo grado \(f(x)=ax^2+bx+c\) è una
 funzione quadratica composta di tre termini. 
 Con le regole precedenti abbiamo: \(f(x)=f_1+f_2+f_3\) e
 \(df(x)~=~df_1+df_2+df_3\).
 \begin{itemize} [noitemsep]
  \item $f_1=ax^2 \sRarrow df_1=2axdx$; 
  \item $f_2=bx \sRarrow df_2=bdx$
  \item $f_3=c \sRarrow df_3=0$
 \end{itemize}
Quindi $df(x)=2axdx+bdx$. Il grafico della funzione è una parabola generica
e il differenziale ci dice che l'incremento infinitesimo 
$dx$ provoca un incremento (o un decremento) variabile sull'asse $Y$, che 
dipende dal punto $x$ a partire dal quale si calcola $dx$.
\end{esempio}

Completiamo il quadro delle regole di calcolo con l'esame dei differenziali
del prodotto e del rapporto di funzioni. Lo studente smart, che si fida un po' 
troppo delle analogie, potrebbe pensare: ''siccome il differenziale di una 
somma è la somma dei differenziali e lo stesso avviene per la differenza, 
succederà una cosa simile anche per il prodotto e per il rapporto``. 
Per (s)fortuna le cose a volte sono un po' meno smart.

\subsubsection{Differenziale del prodotto di due funzioni}
\label{}
Questa volta, al posto della immarcescibile dimostrazione algebrica, ricorriamo
alla geometria. Immaginiamo che le due funzioni, calcolate in un generico punto $x$,
esprimano la base e l'altezza di un rettangolo:
$b(x)$ sarà la base e $h(x)$ sarà l'altezza. L'area ovviamente
si ottiene da $b(x)\cdot h(x)=\mathit{A}(x)$. 
Differenziare il prodotto $d\quadra{\mathit{A}(x)}$ vuol dire calcolare di
quanto aumenta l'area del rettangolo, se i lati subiscono un incremento 
infinitesimo. 

\begin{osservazione}
Gli incrementi della base e dell'altezza possono essere 
diversi, perché $b(x)$ e $h(x)$ sono funzioni diverse, le quali possono
reagire in modo diverso all'incremento $dx$.
\end{osservazione}

\begin{teorema}
 Se una funzione è il prodotto di due funzioni, il suo  differenziale 
 sarà dato da una somma fra tre prodotti: il differenziale della
 prima funzione per la seconda più la prima funzione per il differenziale 
 della seconda più il prodotto dei due differenziali.
\end{teorema}
\noindent Ipotesi: $\mathit{A}(x)=b(x)\cdot h(x)$.\qquad 
Tesi: $d\mathit{A}(x)=db(x)\cdot h(x)+b(x)\cdot dh(x)+ db(x)\cdot dh(x)$.

\begin{proof}


DISEGNO RETTANGOLO CON INCREMENTO NEI LATI

L'incremento infinitesimo di area è la zona colorata del disegno. È 
formato da tre parti:
\begin{itemize} [noitemsep]
 \item un rettangolo sottile, verticale e sulla destra, di base $db(x)$ e altezza $h(x)$;
 \item un rettangolo orizzontale, in alto, di base $b(x)$ e altezza infinitesima $dh(x)$;
 \item un rettangolino in alto a destra, di area $db(x)\cdot dh(x)$.
\end{itemize}
La descrizione geometrica rappresenta bene la tesi e per i nostri scopi è
una prova sufficiente. 
\end{proof}
Facciamo notare che l'ultimo termine è un infinitesimo di ordine
superiore, quindi il risultato può essere approssimato alla sua parte principale, 
senza gravi danni.

\subsubsection{Differenziale del rapporto fra due funzioni}
\label{}
\begin{teorema}
 Se una funzione è data dal rapporto fra due funzioni, con il denominatore
 non nullo, il suo  differenziale si ottiene calcolando:
 la differenza fra due prodotti (il differenziale del numeratore per il 
 denominatore meno il numeratore per il differenziale del denominatore)
 e individendo il risultato per il quadrato del denominatore.
\end{teorema}
\noindent Ipotesi: $h(x)=\frac{\mathit{A}(x)}{b(x)}$, con $b(x)\neq 0$.\tab 
Tesi: $dh(x)=\frac{b(x)d\mathit{A}(x)-\mathit{A}(x)db(x)}{\tonda{b(x)}^2}$.

\begin{proof}
Ricorriamo alla geometria anche in questo caso.


DISEGNO RETTANGOLO CON INCREMENTO NEI LATI


Questa volta $\mathit{A}(x)$ fornisce l'area della parte bianca del rettangolo, e quindi
$\frac{A(x)}{b(x)}=h(x)$ fornisce l'altezza, fissato $x$. Ovviamente si spera che 
sia $b(x)\neq 0$.\\
Se cerchiamo $dh(x)$, incremento infinitesimo di altezza, siamo costretti a calcolare
$d\quadra{\frac{A(x)}{b(x)}}$, differenziale di un rapporto. Guardando il disegno,
si tratta dell'altezza della fascia superiore colorata, la quale corrisponde a
tutta la zona colorata, escluso il rettangolo verticale destro, di area $h(x)db(x)$.
Dunque:
\begin{align*}
 dh(x)=d\quadra{\frac{\mathit{A}(x)}{b(x)}}=\frac{\quadra{d\mathit{A}(x)-
 \frac{\mathit{A(x)}}{b(x)}db(x)}}{b(x)}=
 \frac{b(x)d\mathit{A}(x)-\mathit{A}(x)db(x)}{\tonda{b(x)}^2}.
\end{align*}
\end{proof}


\subsubsection{Sintesi della sezione}
\label{subsubsec:diff01_diffsint}
Ci siamo limitati a calcolare solo alcuni differenziali elementari, attraverso esempi 
e dimostrazioni. Manca del tutto la trattazione dei differenziali delle
funzioni trascendenti. Avremo modo di vedere anche questi nel corso della prossima
sezione, dove quanto ottenuto fin qui viene utilmente ripreso e ampliato.\\
\newline
I risultati che abbiamo visto valgono sotto le ovvie ipotesi
che si parli di funzioni continue e che i differenziali siano calcolabili
per tutti i possibili $x$ del dominio di tali funzioni. Unificando i simboli e restando
all'essenziale, abbiamo:
\begin{enumerate} [noitemsep]
 \item $f=k \srarrow df=0$;
 \item $f=x \srarrow df=dx$;
 \item $f=x^\alpha \srarrow df\sim\alpha x^{\alpha-1}dx$;\newline
 \item $d(a\cdot f)=adf$ \tab differenziale del prodotto per una costante;
 \item $d\tonda{f\pm g}=df\pm dg$ \tab differenziale di una somma o differenza;
 \item $d(f\cdot g)\sim f\cdot g+f\cdot dg$\tab differenziale del prodotto;
 \item $d\tonda{\frac{f}{g}}\sim\frac{df\cdot g-f\cdot dg}{g^2}$\tab  differenziale 
 del rapporto (\(g\ne0\)).
\end{enumerate}
dove  $k$, $a$, $\alpha$ rappresentano delle costanti, mentre f e g sono funzioni continue. 

\section{Applicazioni}
\label{sec:diff01_applicazioni}

Sulla scia delle applicazioni illustrate al termine del Cap.2, esaminiamo alcuni problemi 
facilmente risolvibili con l'aiuto dei differenziali.

\subsection{Problemi con i differenziali}
\label{subsec:diff01__problemi}

\begin{esempio}
 % Triangolo equilatero: h e area
Un triangolo equilatero ha l'altezza di $8$ cm. Di quanto aumenta il suo perimetro,
man mano che aumenta l'altezza? L'aumento è legato alla misura inziale di $h$?\\
Il perimetro $2p=3l$ e con il Teorema di Pitagora si ha: 
$h=\sqrt{l^2-\tonda{\frac{l}{2}}^2}=\frac{\sqrt{3}}{2}l$. Quindi 
$l=\frac{2}{\sqrt{3}}h$
e $2p=2\sqrt{3}h$. 
Incrementiamo l'altezza a partire da $h_0=8$ e ricaviamo il perimetro corrispondente.\\
$d(2p)|_{h_0=8}=d\tonda{2\sqrt{3}h}|_{h_0=8}=2\sqrt{3}(8+dh)-2\sqrt{3}8=2\sqrt{3}dh$.\\
Per ogni incremento infinitesimo dell'altezza, il perimetro aumenta di $2\sqrt{3}$.
Si tratta di un incremento costante, che non dipende dalla misura iniziale
dell'altezza, nè dall'incremento di questa (positivo, negativo, infinitesimo, finito...),
come si deduce dal fatto che il termine $dh$ non compare nel risultato.

\end{esempio}

\section{Derivata}
\label{sec::diff01_derivata}


\section{Regole di derivazione}
\label{sec:02_logaritmiche}

