% (c) 2015 Daniele Zambelli daniele.zambelli@gmail.com
%  Bruno Stecca

\input{\folder differenziazione_grafici.tex}

\chapter{Derivate}

\section{Introduzione}
\label{}

\footnote{Per scrivere questo capitolo mi sono ispirato 
ai lavori di Giorgio Goldoni ``Il calcolo delle differenze e il calcolo 
differenziale''. 
Chi volesse approfondire l'argomento può acquistare il testo 
all'indirizzo: 
\href{https://www.unilibro.it/libri/f/autore/goldoni\_giorgio}
     {www.unilibro.it/libri/f/autore/goldoni\_giorgio}}
Il problema di determinare la velocità istantanea ci ha portati a conoscere 
i numeri infinitesimi e, attraverso questi, l'insieme dei numeri iperreali.
Ora siamo in grado di cercare la risposta alla domanda rimasta in sospeso: 
come si determina la velocità istantanea?\\
La risposta, che conosciamo nelle forme moderne, da più di 400 anni,
propone al nostro studio un nuovo potentissimo strumento di calcolo, 
adatto a risolvere problemi in ogni ambito scientifico: la derivata.

\section{Velocità di caduta}
\label{04_diffvelcaduta}
Nel Settecento fiorirono alcune leggende su Galileo Galilei. Una di queste 
racconta che per dimostrare che i gravi cadono con la stessa velocità, 
gettò dalla Torre di Pisa due sfere di peso diverso, ma di uguali 
dimensioni. 
I due oggetti, come oggi possiamo immaginare, raggiunsero il suolo 
contemporaneamente.\\
La Torre di Pisa è alta circa $56m$ e immaginiamo, per semplificare, che la 
distanza percorsa dai due oggetti sia di $56m$ (ti lascio calcolare il 
percorso effettivo: tieni presente che al giorno d'oggi l'inclinazione 
della Torre è di $4,8^\circ$).\\
Oggi sappiamo che un oggetto in caduta libera ha la seguente legge del moto:
\(s=\frac{1}{2}gt^2\). Come al solito, $s$ è lo spazio in metri, $t$ è il 
tempo in secondi, $g=9,81 m/s^2$ è l'accelerazione di gravità, costante nei
pressi della superficie terrestre.\\
Se cerchiamo la velocità media, basta dividere lo spazio percorso per il 
tempo impiegato:

\begin{align*}
 & s_{tot} = 56m\\
 & s =\frac{1}{2}gt^2 \srarrow t_{tot}= \sqrt{\frac{2s_{tot}}{g}}=
 \sqrt{\frac{2\times 56}{9,81}}=3,36 s.\\
 & v_m= \frac{s_{tot}}{t_{tot}}=\frac{56 m}{3,36 s}=16,67 m/s,
\end{align*}
che corrispondono a circa $60 km/h$ di media.\\
Ma gli oggetti partono fermi e arrivano velocissimi: 
è possibile sapere quale è loro velocità in ogni istante? È il momento
di usare le quantità infinitesime.\\
Chiamiamo $dt$ un intervallo di tempo infinitesimo, fra due istanti 
successivi
$t$ e $t+dt$.
Lo spazio percorso nella caduta, in quell'intervallo di tempo, applicando 
la legge del moto, sarà:
\[
 ds=\frac{1}{2}g\tonda{t+dt}^2-\frac{1}{2}gt^2=
 \frac{1}{2}g\tonda{t^2+2tdt+(dt)^2}-\frac{1}{2}gt^2=
 gtdt+\frac{1}{2}(dt)^2. 
\]
Dividendo il tutto per $dt$ si ottiene la velocità istantanea, quella che 
cambia istante per istante:
\[
 v(t)=\frac{ds}{dt}=\frac{gtdt+\frac{1}{2}dt^2}{dt}=gt+\frac{1}{2}dt.
\]
L'espressione $gt+\frac{1}{2}dt=9,81t+\frac{1}{2}dt$ diventa un numero ben
preciso per ogni valore di $t$, un iperreale finito che è la
somma di un numero standard e di un numero infinitesimo. 
Per averne il valore reale, applichiamo la parte standard:
\[
 \pst{9,81t+\frac{1}{2}dt}=9,81t
\]
Questa è la velocità istantanea che cerchiamo: dipende dal tempo $t$,
cioè cresce con il passare dei secondi. 

\begin{center}
\begin{tabular}{cc}\toprule
$t$ (in $s$) & $v=9,81\times t$ (in $m/s$) \\\midrule
$0$ & $0$  \\
$1$ & $9,81\times 1 =9,81$ \\
$2$ & $9,81\times 2 =19,62$ \\
... & ... \\
$3,6$ & $9,81\times 3,36= ...$\\\bottomrule
\end{tabular}
\label{tab:diff_velocita}
\end{center}

La formula $v=9,81\times t$ ci permette il calcolo della velocità per ogni 
valore di $t$.
Per quale valore di \(t\) la velocità sarà uguale a quella media? 


\section{Continuità}
\label{04_diffcomtinuità}
La semplicità dei calcoli precedenti lascia intuire la ragione del successo 
del calcolo con gli infinitesimi. 
Questo tipo di calcolo fiorì per 150 anni a partire dall'epoca di Newton e 
Leibniz. 
Ma suscitava vivaci polemiche fra gli specialisti, perché non si era in 
grado di spiegare come mai i risultati, espressi attraverso numeri 
infinitesimi, alla fine diventano numeri ``di uso comune``.
Oggi i matematici conoscono meglio la materia e queste difficoltà sono 
superate.
Siamo quindi in grado di procedere nello studio dei questa nuova branca 
della 
matematica, che si chiama \emph{Analisi infinitesimale}.

\subsubsection{Continuità, intervalli, differenze}
\label{04_diffcontinterv}
C'è un punto critico nei ragionamenti svolti a proposito della caduta dei 
gravi, un punto che si dà sempre per scontato in fisica, ma non lo è per i 
matematici e per i logici.\\
Tutto il ragionamento vale perché si presuppone che il tempo scorra in modo
uniforme. Se il tempo scorresse a scatti, anche minuscoli, quei calcoli non
sarebbero possibili. Si dice infatti che il tempo $t$ è una variabile
continua, cioè assume tutti i valori, dal minimo al massimo, con regolarità,
senza salti.

\begin{definizione}
 Una variabile è una grandezza che può assumere valori diversi.
 L'insieme dei valori possibili costituisce il suo insieme di definizione.
\end{definizione}

\begin{definizione}
 Una variabile continua è definita in un intervallo di valori continuo.
 Le variazioni dei suoi valori possono essere arbitrariamente piccole.
\end{definizione}

Il più semplice esempio di una variabile continua in matematica è la 
posizione $x$ sull'asse reale dei numeri. Infatti sappiamo che la retta 
reale 
non ha buchi. A maggior ragione, è una variabile continua anche la 
posizione sull'asse degli Iperreali: $x$, con $x\in\IR$. \\
Viceversa una variabile che pesca i suoi valori in un insieme formato da 
numeri isolati, cioè con differenze finite fra l'uno e l'altro, si dice 
\emph{variabile discreta}. 
\begin{definizione}
 Una variabile discreta assume valori che variano per quantità finite.
\end{definizione}

Un semplice esempio di variabile discreta è $n, n\in \N$.\\
Nel calcolo precedente, $t$ varia con continuità da $0$ a $3,6$,
assumendone tutti i valori, dal minimo al massimo. In matematica si scrive 
così:
\(t\in \quadra{0;\ 3,6}\). 
Le parentesi quadre sono importanti, indicano che 
gli estremi dell'intervallo sono valori possibili, sono inclusi.\\
I tipi possibili di intervallo sono:

\begin{center}
\begin{tabular}{ccc}\toprule
intervallo & sigla & significato\\\midrule
chiuso &$\intervcc{a}{b}$  & estremi compresi\\
aperto/chiuso & $\intervac{a}{b}$ & a escluso, b compreso\\
chiuso/aperto & $\intervca{a}{b}$ & a compreso, b escluso\\
aperto & $\intervaa{a}{b}$ & estremi esclusi\\\bottomrule
\end{tabular}
\label{tab:diff_tipiinterv}
\end{center}

Tutti i tipi di intervallo precedenti, nella retta reale o iperreale, sono
continui, a meno di indicazioni diverse. Se un intervallo $\intervcc{a}{b}$ 
contiene un punto (o più punti) di discontinuità, per esempio $d$, allora 
occorre usare indicazioni diverse: $\intervca{a}{d}\cup\intervac{d}{b}$\\ 
La differenza $a-b$ fra due numeri della retta iperreale $a, b \in \IR$, 
può essere positiva, negativa o nulla. Indicheremo con $\Delta$ la 
differenza fra due numeri standard, cioè una differenza finita, mentre, se 
la 
differenza è infinitesima, sarà indicata con $\delta$ oppure $\epsilon$
o altra lettera minuscola dell'alfabeto greco.\\
In analisi infinitesimale, le differenze infinitesime sono protagoniste.


\section{Differenziale}
\label{sec:diff01_differenziale}

\subsubsection{Parte principale}
\label{subsubsec:diff01_parteprincipale}
I risultati dei calcoli che seguono in molti casi hanno la forma di una 
somma fra infinitesimi di ordine diverso, come avviene nel prossimo esempio 
sul differenziale della funzione quadratica e più avanti con le funzioni 
potenza.\\
In una somma di infinitesimi, gli infinitesimi di
grado superiore (che sono quelli più vicini allo zero) pesano 
sul risultato infinitamente meno degli altri: sono più trascurabili.
Quando in una somma di infinitesimi si trascurano quelli di minor peso,
si dice che si prende la \emph{parte principale della somma}.
Lo si può fare perché la somma esatta e quella approssimata sono 
numeri indistinguibili.

\subsubsection{Incremento di una funzione}
\label{subsubsec:diff01_incremento}

\begin{definizione}
 L'incremento di una funzione è un valore che indica di quanto aumenta una 
funzione partendo da un certo valore della variabile indipendente (\(x\)) 
quando la variabile indipendente aumenta di una certa quantità:
\[\Delta f(x_0) = f \tonda{x_0 + \Delta} - f(x_0)\]
\end{definizione}

\begin{minipage}{.48 \textwidth}
Vediamo un esempio: 
Vogliamo calcolare l'incremento della 
funzione:~\(f(x) = \dfrac{1}{4} x^2 -x -3\)
Quando \(x\) parte da~\(3\) e aumenta di~\(4\).

Devo calcolare quanto vale la funzione nei due punti~$3$ e~$3+4=7$ e 
calcolare la differenza del secondo valore meno il primo:
\begin{align*}
 \Delta &= f(3+4) - f(3) = \\
        &= f(7) - f(3) = \\
        &= \dfrac{7^2}{4}  -7 -3 - \dfrac{3^2}{4}  +3 +3 =\\
        &= \dfrac{49}{4} -10 - \dfrac{9}{4} +6 =\\
        &= 12,25 -10 - 2,25 +6 = \\
        &= 2,25 + 3,75 = 6\\
\end{align*}

\[\]
\end{minipage}
 \hfill
\begin{minipage}{.48 \textwidth}
 \begin{center}
\incremento
 \end{center}
\end{minipage}

È abbastanza evidente che il valore dell'incremento dipende dalla funzione, 
dal punto di partenza e dall'incremento della $x$.

\ifcoding
Con Python.
\lstinputlisting[firstline=2]{\folder src/01incremento.py} %, lastline=5]
\fi

Quindi la funzione ~ \texttt{incremento} ~ ha~$3$ 
parametri, nel programma precedente l'ho invocata tre volte con argomenti 
diversi e ottenendo: nel primo caso~$6$, nel secondo~$2$ e nel terzo~$0$. 
Ovviamente, utilizzando una funzione diversa, gli incrementi calcolati con 
gli stessi parametri saranno, in generale diversi. (Prova ad esempio 
con la funzione \(f(x)= 2^x\))


\subsubsection{Differenziale di una funzione}
\label{subsubsec:diff01_parteprincipale}

In matematica, e nelle sue applicazioni, sono particolarmente importanti 
gli incrementi che si verificano in una funzione quando la variabile 
indipendente (\(x\)) subisce una variazione infinitesima. Incrementi di 
questo tipo, se esistono, si chiamano ''differenziali``.

\begin{definizione}
 Il differenziale di una funzione è un valore che indica di quanto aumenta 
la funzione partendo da un certo valore della variabile indipendente 
(\(x\)) quando la variabile indipendente aumenta di una quantità 
infinitesima:
\[df(x_0) = f \tonda{x_0 + \epsilon} - f(x_0)\]
\end{definizione}

Parliamo di differenziale di una funzione quando incrementi infinitesimi 
della~$x$ producono incrementi infinitesimi della~$y$.
I differenziali, essendo infinitesimi, sono osservabili solo utilizzando 
microscopi non standard.

\begin{minipage}{.48 \textwidth}
Vediamo un esempio: 

Vogliamo calcolare l'incremento della 
funzione:~\(f(x) = \dfrac{1}{4} x^2 -x -3\)
quando \(x\) parte da~\(7\) e aumenta di~\(\epsilon\).
\begin{align*}
  df(7) &= f(7+\epsilon) - f(7) = \\
        &= \dfrac{\tonda{7 +\epsilon}^2}{4}  -\tonda{7 +\epsilon} -3 - 
           \dfrac{7^2}{4}  +7 +3 =\\
        &= \dfrac{49 +14 \epsilon +\epsilon^2}{4} -10 -\epsilon - 
           \dfrac{49}{4} +10 =\\
        &= \dfrac{14 \epsilon +\epsilon^2}{4} -\epsilon =\\
        &= \dfrac{10 \epsilon +\epsilon^2}{4} =\\
        &= 2,5 \epsilon + \dfrac{\epsilon^2}{4} %\sim 2,5 \epsilon
\end{align*}
\end{minipage}
 \hfill
\begin{minipage}{.48 \textwidth}
 \begin{center}
\differenziale
 \end{center}
\end{minipage}

Possiamo osservare che, in questo caso, il differenziale della funzione è 
un infinitesimo che è la somma di due infinitesimi di ordine diverso.

Per calcolare il differenziale di una funzione devo 
conoscere la funzione~\(f(x)\), 
il punto in cui calcolarlo~\(x_0\) e 
l'incremento infinitesimo~\(\epsilon\).

\subsection{Differenziale della variabile $x$}
\label{subsec:diff01_diffx}

Chiamiamo \(dx\) la differenza fra i due valori 
infinitamente vicini della variabile \(x\) (\(dx\) si legge 
''\emph{de x}``).

Vogliamo calcolare il differenziale della variabile \(x\) partendo dal 
punto \(x_0\) quando l'incremento è \(\epsilon\). In simboli: 
$dx|_{x=x_0}$. 
Questa espressione si legge: ''\emph{de x, per x uguale a x zero}``.

Svolgendo i calcoli si ottiene:
\[dx|_{x=x_0}=(x_0+\epsilon)-x_0=\epsilon\]
Si può osservare che il risultato non dipende dal valore in cui viene 
calcolato il differenziale, ma solo dal valore dell'incremento \(\epsilon\),
infatti, nel risultato, \(x_0\) scompare.

\begin{osservazione}
 L'infinitesimo $\epsilon$ potrebbe anche essere negativo, in questo caso 
sarebbe un ''decremento``. 
Il segno di \(\epsilon\) non cambia comunque il calcolo.

\(x_0 + \epsilon\) indica valori che possono trovarsi a destra di \(x_0\) 
(più grandi) o alla sua sinistra (più piccoli).
\end{osservazione}

\begin{comment}
\begin{esempio}
 Calcola il differenziale della variabile $x$ nel punto $x=-7$. Ripeti poi 
 il  calcolo in altri punti.\\
  $dx|_{x=-7} =(-7+\epsilon)-(-7)=\epsilon$\\
  $dx|_{x=7} =(7+\epsilon)-7)=\epsilon$\\
  $dx|_{x=3} =(3+\epsilon)-3=\epsilon$\\
  $dx|_{x=\frac{1}{4}} = 
          \tonda{\frac{1}{4}+\epsilon}-\frac{1}{4}=\epsilon$\\
  $...\quad = \qquad ...$\\
  $dx|_{x=a}  =(a+\epsilon)-a=\epsilon$\\
  $dx|_{x=x_0}  =\ ...\ =\epsilon$\\
  Se il risultato del differenziale è indifferente da $x_0$, allora si 
 evita di indicare $|_{x=x_0}$: $dx=\epsilon$, $\forall x$. 
\end{esempio}

\begin{esempio}
 Calcola il differenziale della variabile $\frac{10}{13}x$ nei punti $x=9$ 
e 
 $x=-\dfrac{1}{5}$.\\
 
$d\tonda{\dfrac{10}{13}x}\bigg|_{x=9}=\quadra{\dfrac{10}{13}
\cdot(9+\epsilon)}-
  \dfrac{10}{13}\cdot 9 = \dfrac{10}{13}\epsilon.$\\
 
$d\tonda{\dfrac{10}{13}x}\bigg|_{x=-\dfrac{1}{5}}=\quadra{\dfrac{10}{13}
\cdot
  \tonda{-\dfrac{1}{5}+\epsilon}}-
 \dfrac{10}{13}\cdot \tonda{-\dfrac{1}{5}}= \dfrac{10}{13}\epsilon.$\\
 $d\tonda{\dfrac{10}{13}x}=\dfrac{10}{13}\epsilon$, $\forall x$.
 \end{esempio}

Anche se due risultati uguali non bastano per fare una prova, e nemmeno i 
sei del primo esempio, si può essere sicuri che mille altri tentativi non
sortirebbero un esito diverso. La prova si ottiene utilizzando $x_0$ 
(oppure una costante analoga) al posto di un valore numerico.

\begin{osservazione}
 L'infinitesimo $\epsilon$ potrebbe anche essere negativo. Questo non 
cambierebbe il calcolo.\\
\end{osservazione}

L'uso di un valore numerico al posto di $x_0$ è essenziale per precisare il
punto a partire dal quale si vuole svolgere il calcolo. Negli esempi 
precedenti tale indicazione è risultata indifferente, ma nella maggior 
parte dei casi, invece, ha un diretto influsso sul risultato.\\

\begin{esempio}
 Calcola $df(x)|_{x=5}$, con $f(x)=x^2$. Calcola poi $df(x)|_{x=-5}$ 
 e infine $df(x)|_{x=2}$.
 \begin{align*} 
  d(x^2)|_{x=5} & 
=(5+\epsilon)^2-5^2=25+10\epsilon+\epsilon^2-25=10\epsilon+\epsilon^2\\
  d(x^2)|_{x=-5}& 
=(-5+\epsilon)^2-(-5)^2=25-10\epsilon+\epsilon^2-25=-10\epsilon+\epsilon^2\\
  d(x^2)|_{x=2} & 
=(2+\epsilon)^2-2^2=4+4\epsilon+\epsilon^2-4=4\epsilon+\epsilon^2\\
 \end{align*}
\end{esempio}


\begin{osservazione}
 Non abbiamo fatto alcuna ipotesi su $\epsilon$. 
 Potrebbe essere un infinitesimo positivo o negativo, 
 potrebbe essere il triplo o il quadrato di un altro infinitesimo. 
 Il risultato non cambia e ha valore per qualsiasi $\epsilon$.
\end{osservazione}

\end{comment}

\subsection{Differenziale di alcune funzioni}
\label{subsec:diff01_difffun}
Iniziamo a differenziare le funzioni più semplici, in un generico punto 
$x_0$.
% Se ci accorgeremo che il risultato non dipende da $x_0$, ne trarremo regole
% di carattere generale.\\
Ma prima di tutto, una precisazione essenziale:

\begin{osservazione}
 Il differenziale di una funzione è calcolabile solo negli intervalli in cui
 la funzione è continua. 
 Perché solo in questo caso a incrementi \emph{infinitesimi} di $x$ 
 corrispondono incrementi \emph{infinitesimi} di $f(x)$.
\end{osservazione}

\subsubsection{Funzione costante}
\label{subsubsec:diff01_diffcostante}

Una funzione è costante se qualunque sia il valore di $x$ il risultato è 
sempre lo stesso. Possiamo indicare questa funzione in diversi modi:
\[f: x \mapsto k \quad \text{o} \quad f(x)=k \quad \text{o} \quad y = k\]

Il suo differenziale sarà:
\[df(x)|_{x=x_0}=f(x_0+\epsilon)-f(x_0)=k-k=0\]
Quindi, se la funzione è costante, il suo differenziale è nullo.
Infatti, avendo sempre lo stesso valore per qualsiasi \(x\), la differenza 
tra due suoi valori è zero. 

%\begin{figure}[h]
\begin{inaccessibleblock}
  [Differenziale di f costante]
 \begin{center}
 \begin{minipage}[]{.38 \textwidth}
  \diffcostante
%  \caption{$y=k\srarrow dy=0$}
 \end{minipage} 
 \hfill
 \begin{minipage}[]{.58 \textwidth}
Resta così dimostrato il seguente   
\begin{teorema}
Il differenziale di una costante è nullo.
\end{teorema}
Nel piano cartesiano, la funzione  $y=k$ è una retta orizzontale e, come 
tutte le rette, è una funzione continua. Quindi il risultato non dipende da 
$x_0$ e vale su tutto l'asse iperreale.
 \end{minipage}
 \end{center}
\end{inaccessibleblock}

\label{fig:diff01_diffcostante}
%\end{figure}

\subsubsection{Funzione identica}
\label{subsubsec:diff01_diffidentica}

La funzione identica (o identità) è una funzione che riceve un valore e dà 
come risultato lo stesso valore ricevuto. Possiamo indicare questa funzione 
in diversi modi:
\[f: x \mapsto x \quad \text{o} \quad f(x)=x \quad \text{o} \quad y = x\]

Se $f(x)=x$, allora, banalmente: $df(x)=dx=\epsilon$. Il risultato è
generale, cioè non dipende da $x_0$. Infatti:\\
\[df(x)|_{x=x_0}=f(x_0+\epsilon)-f(x_0)=(x_0+\epsilon)-x_o=\epsilon\] 

%\begin{figure}[h!]
\begin{inaccessibleblock}
  [Differenziale di f costante]
 \begin{center}
 \begin{minipage}[]{.38 \textwidth}
  \rettabisettrice
%  \caption{$y=x\srarrow dy=dx$}
 \end{minipage} 
 \hfill
 \begin{minipage}[]{.55 \textwidth}
È dimostrato così il seguente
\begin{teorema}
Il differenziale della funzione identica è $dx=\epsilon$.
\end{teorema}
D'ora in poi useremo indifferentemente $dx$ oppure $\epsilon$, dato che
sono equivalenti.\\
Il grafico di $f(x)=x$ nel piano cartesiano è dato dalla retta $y=x$. 
Che significato dobbiamo attribuire a $dy=dx$?
L'uguaglianza dei due differenziali indica che due punti infinitamente
vicini sulla retta individuano sugli assi due differenze infinitesime 
uguali.
 \end{minipage}
 \end{center}
\end{inaccessibleblock}
\label{fig:diff01_diffcostante}
%\end{figure}
  

Succederebbe la stessa cosa con altre rette, più o meno inclinate passanti o
non passanti dall'origine?

\subsubsection{Funzione lineare}
\label{subsubsec:diff01_flineare}

Una funzione lineare è una funzione espressa da un polinomio di primo grado:
\[f: x \mapsto mx +q \quad \text{o} \quad 
  f(x)=mx +q \quad \text{o} \quad 
  y = mx +q\]

\begin{esempio}
 Iniziamo con un esempio numerico, supponiamo 
 che \(m=\dfrac{2}{3}\) e \(q=4\)
 Proviamo quindi a differenziare in $x_0$ la 
 funzione $f(x)=\dfrac{2}{3}x +4$.
\begin{align*}
df(x)|_{x=x_0} &=f(x_0+dx)-f(x_0)=\\
               &=\frac{2}{3}(x_0+dx)+4-\tonda{\frac{2}{3}x_0+4}=
                 \frac{2}{3}x_0+\frac{2}{3}dx+4-\frac{2}{3}x_0-4=
                 \frac{2}{3}dx
\end{align*}
Questa volta il grafico della funzione $y=\frac{2}{3}x$ mostra che
l'incremento infinitesimo dei valori $x$ provoca un incremento 
corrispondente a $\frac{2}{3}$ sui valori $y$. 
Il risultato è generale, cioè vale $\forall x_0$, il differenziale non 
dipende dal particolare punto in cui lo calcolo.
\end{esempio}

\begin{esempio}
Proviamo con la funzione di un'altra retta: \(f(x) = -5x-2\). 
Ci aspettiamo che anche in questo caso il differenziale sia indipendente 
dal punto in cui lo calcolo:
\begin{align*}
df(x)|_{x=x_0} &=f(x_0+dx)-f(x_0)=\\
               &=-5(x_0+dx)-2-\tonda{-5x_0-2}=
                 -5x_0-5dx-2+5x_0+2=
                 -5dx
\end{align*}
Quindi $df(x)=-5dx, \forall x_0 \in \IR$. 
\end{esempio}

\begin{teorema}
 Il differenziale di una funzione lineare $f(x)=mx+q$ è $mdx$, 
 $\forall x\in \IR$.
\end{teorema}

\noindent Ipotesi: \(f(x)=mx+q\).\tab Tesi: \(df(x)=mdx\).

\begin{proof}
\begin{align*}
df(x)|_{x=x_0} &=f(x_0+dx)-f(x_0)=\\
               &=m(x_0+dx)+q-\tonda{mx_0+q}=
                 mx_0-mdx+q-mx_0-q=
                 mdx
\end{align*}
Poiché nel risultato non compare $x_0$, anche in questo caso $df(x)$ non 
dipende dal punto $x_0$.
\end{proof}

\subsubsection{Funzione quadratica}
\label{subsubsec:diff01_diffquad}

Una funzione quadratica è una funzione che dà come risultato il quadrato 
della variabile indipendente:
\[f: x \mapsto x^2 \quad \text{o} \quad 
  f(x)=x^2 \quad \text{o} \quad 
  y = x^2\]

\begin{teorema}
 Il differenziale della funzione quadratica $f(x)=x^2$ ~ è ~ $2xdx+(dx)^2$, 
~$\forall x\in \IR$.
\end{teorema}

\noindent Ipotesi: \(f(x)=x^2\).\tab Tesi: \(df(x)=2xdx+(dx)^2\).

\begin{proof}
\[
 df(x)|_{x=x_0}= 
f(x_0+dx)-f(x_0)=(x_0+dx)^2-x_0^2=x_0^2+2x_0dx+(dx)^2-x_0^2=2x_0dx+(dx)^2
\]
Questa volta nel risultato compare $x_0$. Quindi il valore del 
differenziale della funzione cambia al cambiare del punto $x_0$ che viene 
incrementato.
Anche in questo caso il differenziale è un infinitesimo, ma questa volta è 
dato dalla somma di due infinitesimi di diverso ordine.
\end{proof}

\subsubsection{Funzioni potenza}
\label{subsubsec:diff01_diffpot}

Una funzione potenza è una funzione che dà come risultato la potenza 
della variabile indipendente:
\[f: x \mapsto x^n \quad \text{o} \quad 
  f(x)=x^n \quad \text{o} \quad 
  y = x^n\]

Ricaviamo per gradi il differenziale della funzione potenza è $f(x)=x^n$, 
con un procedimento per induzione.\\
Iniziamo dai casi già noti $f(x)=x$ e $f(x)=x^2$ e esaminiamo i successivi
aumentando progressivamente l'esponente.

\begin{align*}
  d(x) &=x+dx-x =dx\\
  d(x^2) &=(x+dx)^2-x^2 = x^2 +2xdx +(dx)^2 -x^2 = 2xdx +(dx)^2\\
  d(x^3) &=(x+dx)^3-x^3 =[x^3+3x^2dx+3x(dx)^2+(dx)^3]-x^3=
                      3x^2dx+3x(dx)^2+(dx)^3\\
  d(x^4) &=(x+dx)^4-x^4 = [x^4+4x^3dx+6x^2(dx)^2+4x(dx)^3+(dx)^4]-x^4=\\
                      &=4x^3dx+6x^2(dx)^2+4x(dx)^3+(dx)^4      
\end{align*}

Possiamo osservare che, nel calcolo dell'incremento, il termine non 
infinitesimo si annulla sempre. Quindi, qualunque sia il valore di \(x_0\), 
l'incremento della funzione è infinitesimo. 
Queste funzioni sono quindi continue in tutto \(\IR\) perché a spostamenti 
infinitesimi sull'asse \(x\) corrispondono sempre spostamenti infinitesimi 
sull'asse \(y\).

Possiamo anche osservare che il differenziale è sempre più complesso, ma 
se consideriamo la parte principale dell'infinitesimo cioè se trascuriamo 
gli infinitesimi di ordine superiore il risultato si semplifica e diventa 
facilmente memorizzabile.

Quindi se invece del valore esatto ci accontentiamo della parte principale, 
abbiamo:

\begin{align*}
  d(x) &=x+dx-x =dx\\
  d(x^2) &=2xdx +(dx)^2 \sim 2xdx\\
  d(x^3) &=3x^2dx+3x(dx)^2+(dx)^3 \sim 3x^2dx\\
  d(x^4) &=4x^3dx+6x^2(dx)^2+4x(dx)^3+(dx)^4 \sim 4x^3dx\\
  d(x^5) &=5x^4dx+10x^3(dx)^2+10x^2(dx)^3+5x(dx)^4+(dx)^5 \sim 5x^4dx\\
  d(x^6) &=6x^5dx+\dots+(dx)^6 \sim 6x^5dx\\
  d(x^7) &=7x^6(dx)+\dots+(dx)^7 \sim 7x^6dx\\
  \dots &= \dots\\
  d(x^{10}) &=10x^9(dx)+\dots+(dx)^{10} \sim 10x^9dx\\
  \dots &= \dots\\
  d(x^n) &=nx^{n-1}(dx)+\dots+(dx)^{n} \sim nx^{n-1}dx\\    
\end{align*}
Ora che il meccanismo è chiaro e possiamo ritenere sufficientemente 
dimostrato il teorema seguente.

\begin{teorema}
 Il differenziale della funzione potenza è 
 \[d(x^n) \sim nx^{n-1}dx\]
\end{teorema}

\begin{osservazione}
Anche se abbiamo usato solo esponenti interi, si dimostra che 
la regola vale per qualsiasi esponente reale. Lo puoi verificare nei due 
casi che seguono, riscrivendo le funzioni come potenze.
\end{osservazione}

\subsubsection{Funzione radice quadrata}
\label{subsubsec:diff01_diffradq}

\begin{teorema}
 Il differenziale della funzione radice quadrata è
 \[d(\sqrt{x})\sim\frac{dx}{2\sqrt{x}} \quad \text{se} \quad x \neq 0\]
\end{teorema}

\noindent Ipotesi: \(f(x)=\sqrt{x}\).\tab Tesi: 
\(df(x)\sim\dfrac{dx}{2\sqrt{x}}\).

\begin{proof}
\begin{align*}
 df(x)|_{x=x_0} &= f(x_0+dx)-f(x_0)=\sqrt{x_0+dx}-\sqrt{x_0}=\\
 &=\tonda{\sqrt{x_0+dx}-\sqrt{x_0}}\times\frac{\sqrt{x_0+dx}+
 \sqrt{x_0}}{\sqrt{x_0+dx}+\sqrt{x_0}}=\\
 &=\frac{x_0+dx-x_0}{\sqrt{x_0+dx}+\sqrt{x_0}}
 =\frac{dx}{\sqrt{x_0+dx}+\sqrt{x_0}}\sim\frac{dx}{2\sqrt{x_0}}
\end{align*}
Anche questa volta il risultato dipende da $x_0$. 

Prendendo la parte principale dell'infinitesimo possiamo confrontare il 
risultato con quello ottenuto applicando la regola della funzione potenza:
\[d(\sqrt{x}) = d\tonda{x^{\frac{1}{2}}}
  \sim\frac{1}{2}x^{\tonda{\frac{1}{2}-1}}dx 
  =\frac{1}{2}x^{-\frac{1}{2}}dx
  =\frac{dx}{2x^{\frac{1}{2}}}
  =\frac{dx}{2\sqrt{x}}\]

\end{proof}

\subsubsection{Funzione reciproca}
\label{subsubsec:diff01_diffrecip}
\begin{teorema}
 Il differenziale della funzione reciproca $f(x)=\frac{1}{x}$ è 
 \[d\tonda{\frac{1}{x}}\sim\frac{dx}{x^2} \quad \text{se} \quad x \neq 0\]
\end{teorema}

\noindent Ipotesi: $f(x)=\dfrac{1}{x}$\tab 
Tesi: $df(x)\sim-\dfrac{dx}{x^2}$

\begin{proof}
\[
 df(x)|_{x=x_0}= f(x_0+dx)-f(x_0)=\frac{1}{(x_0+dx)}-\frac{1}{x_0}=
 \frac{x_0-x_0-dx}{x_0(x_0+dx)}=\frac{-dx}{x_0^2+x_0dx}\sim-\frac{dx}{x_0^2}
\]
Anche questa volta il valore del differenziale dipende dal punto in cui si 
calcola $x_0$. 

Anche in questo caso possiamo vedere la funzione reciproca come una 
funzione potenza:
\[d\tonda{\frac{1}{x}} = d\tonda{x^{-1}}
  \sim -x^{\tonda{-1-1}}dx 
  =-x^{-2}dx
  =-\frac{dx}{x^{2}}\]

\end{proof}

\subsubsection{Differenziali problematici}
\label{subsubsec:diff01_diffproblemi}

Quest'ultimo calcolo ci porta un punto importante: dato che nel risultato
$x_0$ si trova al denominatore, abbiamo un problema. Che succede se $x_0=0$?
\begin{esempio}
 Calcola $df(x)|_{x=0}$, con $f(x)=\frac{1}{x}$.\\
 $d\tonda{\frac{1}{x}}|_{x=0}=\frac{1}{0+dx}-\frac{1}{0}=$ ?\\
 La funzione è differenziabile $\forall x$, ma non per $x=0$. Se $x\approx 
0$
 il differenziale diventa la differenza fra due infiniti, una forma di 
 indecisione che non siamo in grado di risolvere. Il problema viene dal 
fatto 
 che in $x=0$, $f(x)$ non è definita.
\end{esempio}

\begin{esempio}
Differenzia la funzione $f(x)=\frac{1}{x^2-1}$ per $x_0=1$ e $x_0=-1$.\\
$d\tonda{\frac{1}{x^2-1}}|_{x=1}=\frac{1}{(x+dx)^2-1}-\frac{1}{x^2-1}=
\frac{1}{2dx+(dx)^2}-\frac{1}{0}=$?\\
$d\tonda{\frac{1}{x^2-1}}|_{x=-1}=\frac{1}{(x+dx)^2-1}-\frac{1}{x^2-1}=
\frac{1}{-2dx+(dx)^2}-\frac{1}{0}=$?\\
Questa volta i punti critici sono due. Poiché la funzione non è calcolabile
per $x_0=1$ e $x_0=-1$, non è calcolabile nemmeno il suo differenziale.
\end{esempio}

\begin{esempio}
Nei due esempi precedenti cercavamo di calcolare il differenziale in un 
punto in cui la funzione non era definita. Ci vuole poca immaginazione per 
capire che se non è definita la funzione non può essere definito neppure il 
differenziale.

Ma che dire del differenziale della radice quadrata in zero? Lì la funzione 
è definita: \(f(0)=\sqrt{0}=0\). 
Mentre non è definito il suo 
differenziale: \(df(0) \sim \dfrac{dx}{2 \sqrt{0}}\)

Calcoliamo il differenziale della radice in zero seguendo la definizione:
\[df(x)|_{x=x_0} = f(0+dx)-f(0)=\sqrt{0+dx}-\sqrt{0}=\sqrt{dx}\]
Intanto possiamo osservare che per poter effettuare i calcoli \(dx\) deve 
essere positivo quindi il resto del ragionamento ha senso soltanto 
assumendo \(dx>0\).

Il differenziale ottenuto, \(\sqrt{dx}\), è un infinitesimo ''molto`` più 
grande di \(dx\). 
Se \(\sqrt{dx} = \delta\) allora \(dx = \delta \cdot \delta\) cioè \(dx\) è 
un infinitesimo di \(\sqrt{dx}\). Quando con uno strumento ottico non 
standard riesco a visualizzare \(dx\), non potrò vedere anche 
\(\sqrt{dx}\), questo sarà fuori dal campo visivo, si troverà all'infinito.

Possiamo concludere che in zero la funzione radice ha un differenziale 
(solo sul lato positivo), ma questo differenziale è infinitamente più 
grande di \(dx\).
\end{esempio}

Nel piano cartesiano tracciamo il grafico delle funzioni degli ultimi 
tre esempi: $y=\sqrt{x}$, $y=\frac{1}{x}$ e $y=\frac{1}{x^2-1}$.
\begin{figure}[h]
\begin{inaccessibleblock}[Grafici di funzioni diverse.]
 \begin{center}
 \begin{minipage}[]{.23 \textwidth}
  \radice
  \vspace*{-5mm} 
  \caption{$y=\sqrt{x}$}
 \end{minipage} 
 \begin{minipage}[]{.37 \textwidth}
  \iperbole
  \caption{$y=\frac{1}{x}$}
 \end{minipage} 
 \begin{minipage}[]{.37 \textwidth}
  \iperbolequad
  \caption{$y=\frac{1}{x^2-1}$}
 \end{minipage}
 \end{center}
\end{inaccessibleblock}
\label{fig:diff01_grafici}
\end{figure}

Guardando i grafici si può osservare che $y=\sqrt{x}$, essendo definita per 
i valori $x\ge 0$, 
non può essere calcolata per esempio, se $x=-2$ e quindi nemmeno il suo 
differenziale ha senso in questo punto.

Gli altri due grafici mettono in evidenza questo problema: dove la funzione 
non è calcolabile, non esiste il punto che rappresenta la funzione nel 
piano cartesiano reale. 

% \begin{figure}[h]
\begin{inaccessibleblock}
  [Discontinuità a salto]
 \begin{center}
 \begin{minipage}[]{.38 \textwidth}
  \salto
%   \caption{Discontinuità a salto.}
 \end{minipage} 
 \hfill
 \begin{minipage}[]{.58 \textwidth}
Consideriamo un tipo diverso di problema.\\
\[f(x)=\begin{cases} 
x-1, & \mbox{se }x<2 \\ 
x+1, & \mbox{se }x\ge 2
\end{cases}
\]
$f(x)$ ha due rami e il grafico compie un salto per $x~=~2$. Le differenze
infinitesime calcolate a destra di tale punto saranno diverse da quelle 
calcolate a sinistra: i differenziali sono calcolabili ma non hanno
uguali valori. Anche in questo caso $f(x)$ non è differenziabile per $x=2$
 \end{minipage}
 \end{center}
\end{inaccessibleblock}
\label{fig:diff01_salto}
% \end{figure}

\subsubsection{Continuità e funzioni}
\label{subsubsec:diff01_diffcontinue}
Il tema della continuità è vasto e importante e viene trattato nei dettagli
nel prossimo capitolo. Per ora ci limitiamo a considerazioni di carattere 
intuitivo.\\
\emph{Se una funzione è continua, ne puoi tracciare il grafico nel piano 
cartesiano senza staccare la matita dal foglio}. 
Se ci fosse un punto (o più punti) di discontinuità, saresti obbligato a 
interrompere il disegno e riprenderlo da punti vicini.
\begin{esempio}
 La funzione $f(x)=x$, che ha per grafico la retta $y=x$ è evidentemente 
 una funzione continua: puoi tracciarne il grafico senza interruzioni
 nell'intervallo $(-M,\ M)$. Sono anche continue tutte le funzioni che hanno
 per grafico una retta, come per esempio $f(x)=-\frac{4}{5}x+9$.\\
 Quindi anche la funzione costante $f(x)=k$, che ha per grafico una retta
 orizzontale, è una funzione continua.
\end{esempio}
\begin{esempio}
 La funzione $f(x)=\frac{1}{x}$ è continua ovunque in $\R$, tranne che per 
$x=0$.
 Infatti se $x=0$, $f(x)$ non è calcolabile, quindi nel piano cartesiano 
 non puoi disegnare un punto  che rappresenta il valore standard 
 $(0;~\frac{1}{0})$. 
 Il punto è comunque visibile nel piano iperreale, con  un telescopio.
 \end{esempio}
\begin{esempio}
 Per ragioni simili, sono discontinue in uno o più punti le funzioni
 (algebriche o trascendenti), per le quali occorra specificare condizioni 
 di esistenza relative a questi punti.
 Così $f(x)=\frac{1}{x^2-1}$ è discontinua per $x=\pm 1$, mentre 
 $f(x)=\frac{1}{x^2+1}$ è continua.
\end{esempio}

Dagli esempi si capisce che \emph{la continuità delle funzioni è una 
condizione di carattere locale}, cioè per punti. Infatti si possono 
riconoscere dei punti di discontinuità di una funzione, non degli 
intervalli di discontinuità.
Se ci si accorge che un punto $(x_0;\ f(x_0))$ è di discontinuità
per $f(x)$, allora si dice: \emph{$f(x)$ è discontinua per $x=x_0$},
cioè si indica solo la coordinata $x$ che pone questo problema 
(non si usa dire: $f(x_0)$ è discontinua).
Una funzione può essere discontinua in infiniti punti.
\begin{esempio}
 La funzione $f(x)= \tan x$ è discontinua per $x=\frac{\pi}{2}\pm k\pi$.
\end{esempio}

\begin{definizione}
Se una funzione è continua in tutti i punti di un intervallo 
$\intervcc{a}{b}$, allora si dice continua in $\intervcc{a}{b}$.
\end{definizione}
\begin{osservazione}
 Ovviamente la definizione non cambia se l'intervallo è di tipo diverso.
\end{osservazione}

\begin{esempio}
 $f(x)=\ln x$ è definita per $x \in (0;\ M)$ ed è ivi continua.
\end{esempio}


\subsection{Combinare differenziali}
\label{subsec:diff01_combdiff}
Nella sezione \ref{subsubsec:diff01_flineare} e in altre ci siamo avvalsi 
di proprietà così naturali che non è stato necessario sottolinearle. 
Ma è meglio non lasciarcele sfuggire.

\subsubsection{Differenziale del prodotto per una costante}
\label{}
\begin{teorema}
 Se una funzione è moltiplicata per una costante, anche il suo 
 differenziale risulta moltiplicato per la stessa costante.
\end{teorema}
\noindent Ipotesi: $f(x)=a\cdot g(x)$.\tab Tesi: $df(x)=a\cdot dg(x)$.

\begin{proof}
\[
 df(x)=d\quadra{a\cdot g(x)}= a\cdot g(x+dx)-a\cdot 
g(x)=a\cdot\quadra{g(x+dx)-g(x)}
 =a\cdot dg(x).
\]
\end{proof}

\subsubsection{Differenziale di una somma di funzioni}
\label{}

\begin{teorema}
 Se una funzione è la somma (la differenza) di due funzioni, anche il suo 
 differenziale sarà la somma (la differenza) dei due differenziali.
\end{teorema}
\noindent Ipotesi: $f(x)=f_1(x)\pm f_2(x)$.\tab Tesi: $df(x)=df_1(x)\pm 
df_2(x)$.

\begin{proof}
\begin{align*}
 df(x)=d[f_1(x)\pm f_2(x)]=[f_1(x+dx)\pm f_2(x+dx)]-[f_1(x)\pm f_2(x)]=\\
 = [f_1(x+dx)-f_1(x)]\pm [f_2(x+dx)-f_2(x)]= df_1(x)\pm df_2(x)
\end{align*}
\end{proof}

\begin{esempio}
 Un generico polinomio di secondo grado \(f(x)=ax^2+bx+c\) è una
 funzione quadratica composta di tre termini. 
 Con le regole precedenti abbiamo: \(f(x)=f_1+f_2+f_3\) e
 \(df(x)~=~df_1+df_2+df_3\).
 \begin{itemize} [noitemsep]
  \item $f_1=ax^2 \sRarrow df_1\sim 2axdx$; 
  \item $f_2=bx \sRarrow df_2=bdx$
  \item $f_3=c \sRarrow df_3=0$
 \end{itemize}
Quindi $df(x)\sim 2axdx+bdx$. Il grafico della funzione è una parabola 
generica
e il differenziale ci dice che l'incremento infinitesimo 
$dx$ provoca un incremento (o un decremento) variabile sull'asse $Y$, che 
dipende dal punto $x$ a partire dal quale si calcola $dx$.
\end{esempio}

Completiamo il quadro delle regole di calcolo con l'esame dei differenziali
del prodotto e del rapporto di funzioni. Lo studente smart, che si fida un 
po' 
troppo delle analogie, potrebbe pensare: ''siccome il differenziale di una 
somma è la somma dei differenziali e lo stesso avviene per la differenza, 
succederà una cosa simile anche per il prodotto e per il rapporto``. 
Per (s)fortuna le cose a volte sono un po' meno smart.

\subsubsection{Differenziale del prodotto di due funzioni}
\label{}
Questa volta, al posto della immarcescibile dimostrazione algebrica, 
ricorriamo
alla geometria. Immaginiamo che le due funzioni, calcolate in un generico 
punto $x$,
esprimano la base e l'altezza di un rettangolo:
$b(x)=b$ sarà la base  e $h(x)=h$ sarà l'altezza . L'area ovviamente
si ottiene da $b(x)\cdot h(x)=\mathit{A}(x)$. 
Differenziare il prodotto $d\quadra{\mathit{A}(x)}$ vuol dire calcolare di
quanto aumenta l'area del rettangolo, se i lati subiscono un incremento 
infinitesimo. 

\begin{osservazione}
Gli incrementi della base e dell'altezza possono essere 
diversi, perché $b(x)$ e $h(x)$ sono funzioni diverse, le quali possono
reagire in modo diverso all'incremento $dx$.
\end{osservazione}

\begin{teorema}
 Se una funzione è il prodotto di due funzioni, il suo  differenziale 
 sarà dato da una somma fra tre prodotti: il differenziale della
 prima funzione per la seconda più la prima funzione per il differenziale 
 della seconda più il prodotto dei due differenziali.
\end{teorema}
\noindent Ipotesi: $\mathit{A}(x)=b(x)\cdot h(x)$.\qquad 
Tesi: $d\mathit{A}(x)=db(x)\cdot h(x)+b(x)\cdot dh(x)+ db(x)\cdot dh(x)$.

\begin{figure}[h]
\begin{inaccessibleblock}
  [Rettangolo con uno gnomone finito e rettangolo con gnomone infinitesimo.]
 \begin{center}
 \begin{minipage}[]{.38 \textwidth}
  \vspace{27mm} \incrementaleprodotto
 \end{minipage} 
 \hfill
 \begin{minipage}[]{.58 \textwidth}
  \differenzialeprodotto
 \end{minipage}
 \end{center}
\end{inaccessibleblock}
\caption{Incrementi finito e infinitesimo dell'area di un rettangolo} 
\label{fig:Incre_prodotto}
\end{figure}

\begin{osservazione}
 Si chiama ''gnomone`` la figura, a forma di L rovesciata, che rappresenta 
la 
crescita dell'area di un rettangolo.
\end{osservazione}

\begin{proof}
L'incremento infinitesimo di area è la zona colorata del disegno, lo 
\emph{gnomone}. È formato da tre parti:
\begin{itemize} [noitemsep]
 \item un rettangolo sottile, verticale e sulla destra, di base
 infinitesima $db(x)$ e altezza $h(x)$;
 \item un rettangolo orizzontale, in alto, di base $b(x)$ e
 altezza infinitesima $dh(x)$;
 \item un rettangolino in alto a destra, di area $db(x) dh(x)$.
\end{itemize}
La descrizione geometrica rappresenta bene la tesi e per i nostri scopi è
una prova sufficiente. 
\end{proof}
Dato che l'ultimo termine è un infinitesimo di ordine
superiore, il risultato può essere approssimato alla sua parte principale, 
senza gravi danni: $d\mathit{A}(x)\sim db(x)\cdot h(x)+b(x)\cdot dh(x)$ .

\subsubsection{Differenziale del rapporto fra due funzioni}
\label{}
\begin{teorema}
 Se una funzione è data dal rapporto fra due funzioni, con il denominatore
 non nullo, il suo  differenziale si ottiene calcolando
 la differenza fra due prodotti (il differenziale del numeratore per il 
 denominatore meno il numeratore per il differenziale del denominatore)
 e dividendo il risultato per il quadrato del denominatore.
\end{teorema}
\noindent Ipotesi: $h(x)=\dfrac{\mathit{A}(x)}{b(x)}$, con $b(x)\neq 
0$.\tab 
Tesi: 
$dh(x) \sim \dfrac{d\mathit{A}(x) \cdot b(x)-\mathit{A}(x) \cdot db(x)}
             {\tonda{b(x)}^2}$

\begin{proof}
Ricorriamo alla geometria anche in questo caso.

% Usare {figure} è una disgrazia!

 \begin{minipage}[]{.38 \textwidth}
 % \vspace{27mm} \incrementaleprodotto
Essendo $\mathit{A}(x)=b(x) \times h(x)$ allora: 
$h(x)=\frac{A(x)}{b(x)}$. 
Ovviamente è necessario che $b(x)\neq 0$.\\
Guardando il disegno, possiamo osservare che $dh(x)$ è l'incremento 
infinitesimo dell'altezza, si tratta dell'altezza della fascia superiore 
colorata.
Questa altezza si può calcolare dividendo il rettangolo 
superiore dello gnomone per la base del rettangolo.
Il rettangolo superiore dello gnomone è uguale a tutto lo gnomone 
infinitesimo, $d\mathit{A}$,
 \end{minipage} 
 \hfill
 \begin{minipage}[]{.58 \textwidth}
 \begin{center}
 \begin{inaccessibleblock}
  [Altezza rettangolo con gnomone infinitesimo.]
  \differenzialerapporto
 \end{inaccessibleblock}
 \end{center}
 \end{minipage}
meno il rettangolo destro infinitesimo, di 
area $h\cdot db$ e meno il rettangolino, sempre infinitesimo, che si trova 
in alto a destra.
Dunque:
\begin{align}
 d\quadra{\frac{\mathit{A}(x)}{b(x)}}&=dh(x)=\\
 &=\frac{\quadra{d\mathit{A}(x) - h \cdot db(x) - db(x) \cdot dh(x)}}
        {b(x)}=\\
 &=\frac{\quadra{d\mathit{A}(x) - \dfrac{\mathit{A(x)}}{b(x)} \cdot db(x) - 
          db(x) \cdot dh(x)}}{b(x)}=\\
 &=\frac{\dfrac{d\mathit{A}(x) \cdot b(x) - \mathit{A(x)} \cdot db(x) -
               b(x) \cdot db(x) \cdot dh(x)}
              {b(x)}}
        {b(x)}=\\
 &=\frac{d\mathit{A}(x) \cdot b(x) - \mathit{A(x)} \cdot db(x) -
              b(x) \cdot db(x) \cdot dh(x)}
        {\tonda{b(x)}^2} \sim \\
 & \sim \frac{d\mathit{A}(x) \cdot b(x) - \mathit{A(x)} \cdot db(x)}
             {\tonda{b(x)}^2}
\end{align}
\end{proof}

\subsubsection{Sintesi della sezione}
\label{subsubsec:diff01_diffsint}
Ci siamo limitati a calcolare solo alcuni differenziali elementari, 
attraverso esempi e dimostrazioni. 
Manca del tutto la trattazione dei differenziali delle funzioni 
trascendenti. 
Avremo modo di vedere anche questi nel corso della prossima sezione, dove, 
quanto ottenuto fin qui, viene utilmente ripreso e ampliato.

I risultati che abbiamo visto valgono sotto le ovvie ipotesi
che si parli di funzioni continue e che i differenziali siano calcolabili
per tutti i possibili $x$ del dominio di tali funzioni. Unificando i 
simboli e restando all'essenziale, abbiamo:
\begin{enumerate} [noitemsep]
 \item $f=k \srarrow df=0$;
 \item $f=x \srarrow df=dx$;
 \item $f=x^\alpha \srarrow df\sim\alpha x^{\alpha-1}dx$;\newline
 \item $d(a\cdot f)=adf$ \tab differenziale del prodotto per una costante;
 \item $d\tonda{f\pm g}=df\pm dg$ \tab differenziale di una somma o 
differenza;
 \item $d(f\cdot g)\sim f\cdot g+f\cdot dg$\tab differenziale del prodotto;
 \item $d\tonda{\dfrac{f}{g}} \sim \dfrac{df\cdot g-f\cdot dg}{g^2}$\tab  
differenziale  del rapporto (\(g \ne 0\)).
\end{enumerate}
dove  $k$, $a$, $\alpha$ rappresentano delle costanti, mentre f e g sono 
funzioni continue. 


Sulla scia delle applicazioni illustrate al termine del Cap.2, esaminiamo 
alcuni problemi 
facilmente risolvibili con l'aiuto dei differenziali.

\subsection{Problemi con i differenziali}
\label{subsec:diff01__problemi}

\begin{esempio}
 % Triangolo equilatero: 2p=f(h) 
Un triangolo equilatero ha l'altezza di $8$ cm. 
Di quanto aumenta il suo perimetro, man mano che aumenta l'altezza? 
L'aumento è legato alla misura iniziale di $h$?\\
Il perimetro è $2p=3l$ e con il Teorema di Pitagora si ha: 
$h=\sqrt{l^2-\tonda{\frac{l}{2}}^2}=\frac{\sqrt{3}}{2}l$. 
Quindi $l=\frac{2}{\sqrt{3}}h$ e $2p=2\sqrt{3}h$. 
Incrementiamo l'altezza a partire da $h_0=8$ e ricaviamo il perimetro 
corrispondente.\\
$d(2p)|_{h_0=8}=
d\tonda{2\sqrt{3}h}|_{h_0=8}=
2 \sqrt{3} \cdot (8+dh)-2\sqrt{3} \cdot 8 = 2 \sqrt{3} \cdot dh$.\\
Per ogni incremento infinitesimo dell'altezza, il perimetro aumenta di 
$2 \sqrt{3}$.
Si tratta di un incremento costante, che non dipende dalla misura iniziale
dell'altezza. Infatti, se si ripete il calcolo scrivendo il simbolo $h_0$ 
al 
posto della 
sua misura $8$, $h_0$ non compare nel risultato.\\
La soluzione può essere ricavata in modo più diretto, applicando le regole
4 e 2 della sintesi a pag.\pageref{subsubsec:diff01_diffsint}.
\end{esempio}

\begin{esempio}
 % Triangolo equilatero: l=f(A)
Di quanto aumenta il lato di un triangolo equilatero,
man mano che aumenta la sua area? L'aumento è legato al valore iniziale del 
lato?\\
Dalla formula dell'area $\mathit{A}=\frac{bh}{2}$ e dall'esempio precedente
($h=\frac{\sqrt{3}}{2}l$), ricaviamo: $\mathit{A}=\frac{\sqrt{3}}{4}l^2$.\\
Differenziando, con l'aiuto delle regole 4 e 3 della sintesi a pag.
\pageref{subsubsec:diff01_diffsint}, abbiamo:\\
$d\tonda{\mathit{A}}=d\tonda{\frac{\sqrt{3}}{4}l^2}=
 \frac{\sqrt{3}}{4}d\tonda{l^2}=
 \frac{\sqrt{3}}{4}\tonda{2l \cdot dl+(dl)^2}=
 \frac{\sqrt{3}}{4}\tonda{2l+dl}dl$.\\
Questa volta la relazione con l'incremento del lato non è elementare: per 
ogni 
incremento infinitesimo del lato si ha un incremento di area pari a 
$\frac{\sqrt{3}}{4}\tonda{2l+dl}$, che dipende dalla misura iniziale del 
lato
e dallo stesso incremento. Per gestire il risultato, occorre approssimare
questo numero all'indistinguibile più vicino:\\
$d\tonda{\mathit{A}}=\frac{\sqrt{3}}{4}\tonda{2l+dl}dl\sim
\frac{\sqrt{3}}{2}ldl$.\\
Da qui, applicando la formula inversa, si ottengono le 
risposte:
$dl\sim\frac{2}{\sqrt{3}}\frac{d\tonda{\mathit{A}}}{l}$.
\begin{osservazione}
 Una via più diretta per giungere alla soluzione potrebbe essere: \\
 $\mathit{A}=\frac{\sqrt{3}}{4}l^2\srarrow 
l=\sqrt{\frac{4}{\sqrt{3}}\mathit{A}}=
 \frac{2}{\sqrt[4]{3}}\sqrt{\mathit{A}}\srarrow dl=
 d\tonda{\frac{2}{\sqrt[4]{3}}\sqrt{\mathit{A}}}=
 \frac{2}{\sqrt[4]{3}}d\tonda{\sqrt{\mathit{A}}}$\\
 A questo punto dobbiamo fermare il calcolo, perché 
 sappiamo calcolare $d\tonda{\sqrt{x}}$, ma 
 non sappiamo ancora come calcolare $d\tonda{\sqrt{f(x)}}$. Per farlo, 
 occorre approfondire le conoscenze.
\end{osservazione}
\end{esempio}

\section{Introduzione alla derivata}
\label{sec:diff01_derivata}
La derivata è un ente matematico conosciuto dalla metà del 1700, che 
da allora si applica utilmente allo studio di fenomeni naturali di ogni 
tipo.\\
Studieremo l'argomento puntando lo sguardo sulle funzioni e sui loro
grafici nel piano cartesiano. Iniziamo dai grafici più semplici.

\subsection{Pendenza di una retta}
\label{subsec:diff01_pendretta}

\begin{figure}[h]
\begin{inaccessibleblock}[pendenza di una retta.]
 \begin{center}
 \begin{minipage}[]{.31 \textwidth}
%  \vspace*{4mm} 
  \rettadueterzi
  \caption{$y=\frac{3}{2}x-1$}
 \end{minipage} 
 \begin{minipage}[]{.31 \textwidth}
  \rettamenounquarto
  \caption{$y=-\frac{1}{4}x+\frac{1}{2}$}
 \end{minipage} 
 \begin{minipage}[]{.31 \textwidth}
  \retteorvert
  \caption{$y=-\frac{3}{2}$ e $x=-2,8$}
 \end{minipage}
 \end{center}
\end{inaccessibleblock}
\label{fig:diff01_ret}
\end{figure}


Sappiamo già calcolare la pendenza di una retta dalla semplice osservazione 
del suo grafico: si fissano sulla retta due punti $A(x_A; y_A)$ e $B(x_B; 
y_B)$
e si calcola il rapporto $m=\frac{y_B-y_A}{x_B-x_A}$.\\
È come se si volesse misurare la distanza verticale
fra i due punti usando la loro distanza orizzontale come unità di misura. 
Nel caso della retta $r$, $m=\frac{3}{2}$ e si potrebbe dire: ''un punto 
che 
si 
muove sulla retta, se si sposta di due quadretti in orizzontale
ne guadagna (o perde) tre in verticale.\\
Un punto che scorre sulla retta orizzontale, non subisce
alcuna variazione lungo l'asse $y$ e per questo $m=0$; al contrario per la
retta verticale le variazioni sono solo verticali e la pendenza è
infinita.\\
Sintetizziamo la formula come rapporto fra differenze:
$m=\frac{y_B-y_A}{x_B-x_A}=\frac{\Delta y}{\Delta x}$. Il simbolo $m$ 
ci riporta all'equazione di una retta generica in forma esplicita
$y=mx+q$, dove $m$ rappresenta appunto il coefficiente angolare, cioè 
l'inclinazione o pendenza.
\begin{osservazione}
Secondo l'uso del capitolo precedente, le indicazioni con la lettera
maiuscola $\Delta$ ($\Delta x$, $\Delta y$) si riferiscono a \emph{quantità 
finite},
cioè a numeri standard.
\end{osservazione}

\subsubsection{Rapporto incrementale}
\label{subsubsec:diff01_rappincr}
C'è un fatto importante: per calcolare la pendenza di una retta, 
la scelta dei due punti è indifferente. Possono essere molto vicini o molto 
lontani, scambiati l'uno con l'altro o presso l'origine, oppure no:
$m=\frac{\Delta y}{\Delta x}$ è sempre lo stesso, come è giusto che sia per 
una retta.
Da $x_B-x_A=\Delta x$ ricaviamo banalmente $x_B=x_A+\Delta x$, cioè nel 
piano 
cartesiano
$B$ si colloca a destra (se $\Delta x\ge 0$) di $A$ di una quantità finita,
grande o piccola che sia.\\
$\Delta x$, $\Delta y$ sono anche chiamati \emph{incrementi} e quindi...
\begin{definizione}
  Si dice \emph{Rapporto Incrementale} (R.I.) il rapporto degli
  incrementi, cioè la quantità R.I.~=~$\frac{\Delta y}{\Delta x}$.
\end{definizione}
Si tratta di una quantità finita, calcolabile se $\Delta x \ne 0$.\\
Il Rapporto Incrementale, calcolato su una retta fornisce la sua pendenza 
ed 
è un valore costante, come abbiamo visto.\\
Ma il calcolo si può applicare a qualsiasi funzione, anche a quelle che nel 
piano cartesiano sono rappresentate da curve. Allora le cose cambiano.\\

\begin{esempio}
I prossimi grafici appartengono alla stessa funzione.

\begin{figure}[h]
\begin{inaccessibleblock}
  [Secanti a una curva]
% \begin{center}
 \begin{minipage}[]{.45\textwidth}
 \curvacubica
 \end{minipage} 
 \hfill
 \begin{minipage}[]{.55\textwidth}
  \secanticubica
 \end{minipage}
% \end{center}
\end{inaccessibleblock}
\caption{Rapporti incrementali in una curva e secanti.} 
\label{}
\end{figure}

Scegliamo alcuni punti sulla curva e mettiamo in evidenza gli intervalli che
consentono il calcolo del rapporto incrementale, in un caso, e la pendenza 
delle secanti nell'altro.\\
Rapporti Incrementali:
\begin{align*}
  \frac{\Delta y}{\Delta x}\bigg\lvert_{AB}= &\frac{y_B-y_A}{x_B-x_A}=
    \frac{2-5}{-2-(-3.5)}=\frac{-3}{1.5}=-2 &
  \frac{\Delta y}{\Delta x}\bigg\rvert_{BC}=\frac{y_C-y_B}{x_C-x_B}=
  \frac{3-2}{0-(-2)}=\frac{1}{2}\\
  \frac{\Delta y}{\Delta x}\bigg\lvert_{CD}= &\frac{y_D-y_C}{x_D-x_C}=
  \frac{3.8-3)}{2.4-0}=\frac{0.8}{2.4}=\frac{1}{3} &
  \frac{\Delta y}{\Delta x}\bigg\rvert_{DE}=\frac{y_E-y_D}{x_E-x_D}=
  \frac{1-3.8}{3.5-2.3)}=\frac{-2.8}{1.2}=-\frac{7}{3}
\end{align*}

Pendenze.\\
$$  m_{AB}=-2 \qquad m_{BC}=\frac{1}{2}\qquad m_{CD}=\frac{1}{3}\qquad  
m_{DE}=-\frac{7}{3}$$
\end{esempio}

I calcoli confermano che se il grafico è una curva, il Rapporto 
Incrementale, 
calcolato fra varie coppie di punti, ha valori diversi. 
Il R.I. cambia a seconda della coppia di punti fissati sulla curva.\\
Se si traccia la retta che unisce la coppia di punti, ne risulta una secante
alla curva.\\
In conclusione, si hanno le seguenti proprietà:

\begin{enumerate}[noitemsep]
\item Il R.I. è un numero finito e esiste solo se $\Delta x\ne 0$.
\item Il R.I fra le coppie di valori di una funzione è 
 a sua volta una funzione, che dipende dalla coppia scelta. 
 \item La funzione è costante se applicata al grafico di una retta. In 
questo 
 caso il R.I calcola  la sua pendenza.
\item  In generale, R.I. calcola la pendenza della retta secante che unisce 
due punti del grafico.
\end{enumerate}


\subsubsection{Rapporto differenziale}
\label{subsubsec:diff01_rappdiff}

\begin{esempio}
Fissiamo su una curva due punti: uno fisso ($A$) e l'altro mobile $P$, cioè 
in 
grado di spostarsi lungo la curva dalla posizione più lontana $P_1$, alla 
più 
prossima ad $A$, cioè oltre $P_7$, fin quasi a sovrapporsi con $A$.

\begin{figure}[h]
\begin{inaccessibleblock}[Verso la tangente a una curva.]
 \begin{center}
\secanticurva
 \end{center}
\end{inaccessibleblock}
\caption{Dalle secanti alla tangente.} \label{fig:diff01_sectang}
\end{figure}
\end{esempio}

Tracciamo le secanti che uniscono $A$ con le varie posizioni di $P_n$. 
Man mano che $P$ si avvicina ad $A$, la secante che li unisce tende 
ad allinearsi alla tangente ideale.\\
Quando $P$ è così vicino ad $A$ che la loro distanza è 
$\overline{AP}<\frac{1}{n}$,
$\forall n$, siamo nel campo degli infinitesimi: cambia la natura del 
Rapporto Incrementale che avevamo imparato a calcolare. Il R.I. si 
trasforma da un rapporto fra quantità finite a un rapporto fra infinitesimi,
quindi non possiamo essere certi su quale sia il tipo del risultato che 
fornisce.\\
Se escludiamo il caso $dx=0$ ($P_n$ coinciderebbe con $A$) e se il rapporto
dà un risultato finito, otterremo la pendenza della secante fra i due 
punti infinitamente vicini $A\punto{x_A}{y_A}$ e 
$P_n\punto{x_A+dx}{f(x_A+dx)}$,
quindi di una retta infinitamente vicina alla tangente, cioè distinta da 
essa 
solo se guardata con il microscopio a ingrandimento infinito.

\begin{definizione}
 Si dice Rapporto Differenziale della funzione $f(x)$, relativo a $x_0$
 il rapporto $\frac{df(x)}{dx}\big|_{x=x_0}$ fra il differenziale della 
funzione  e quello della variabile, calcolati nel punto $x_0$. \\
 $\frac{df(x)}{dx}\big|_{x=x_0}=\frac{f(x_0+dx)-f(x_0)}{dx}$, con $dx\ne 0$.
\end{definizione}

\begin{figure}[h]
\begin{inaccessibleblock}[Secante per P approx A.]
 \begin{center}
\secRD
 \end{center}
\end{inaccessibleblock}
\caption{Secante per due punti infinitamente vicini.} 
\label{fig:diff01_tangente}
\end{figure}

\begin{esempio}
  \label{esempio:diff01_mdiff}
  La curva della Fig.12 rappresenta la parabola di equazione 
  $y=\dfrac{x^2}{5}-\dfrac{3}{5}x+2$. Calcoliamo la pendenza della secante 
che
  passa per $A\punto{5}{4}$ e per un altro punto infinitamente vicino.\\
  La funzione è visibilmente continua nel punto $A$ e il differenziale per 
  $x=x_A$, secondo le regole della sezione precedente, è:\\ 
  $d(f(x)\big|_{x=5}=d\tonda{\dfrac{x^2}{5}-\dfrac{3}{5}x+2}\bigg|_{x=5}=
  d\tonda{\dfrac{x^2}{5}}\bigg|_{x=5}-d\tonda{\dfrac{3}{5}x}\bigg|_{x=5}+
  d(2)\big|_{x=5}=\\
  =\dfrac{1}{5}\tonda{2xdx+(dx)^2}\big|_{x=5}-\dfrac{3}{5}dx\big|_{x=5}+0=
  \dfrac{2}{5}x|_{x=5}dx+\dfrac{1}{5}(dx)^2-\dfrac{3}{5}dx=$\\
  $=\dfrac{2}{5}5dx+\dfrac{1}{5}(dx)^2-\dfrac{3}{5}dx=
  2dx+\dfrac{1}{5}(dx)^2-\dfrac{3}{5}dx=
  \dfrac{7}{5}dx+\dfrac{1}{5}(dx)^2$\\
  Raccogliendo $dx$ nel differenziale della funzione, il rapporto 
differenziale 
  è:\\
  
$\dfrac{d(f(x)}{dx}\bigg|_{x=5}=\dfrac{\tonda{\dfrac{7}{5}+\dfrac{1}{5}dx}dx
}
  {dx}=\dfrac{7}{5}+\dfrac{1}{5}dx$.\\
  Come si vede, la pendenza di questa secante è un numero finito del tipo 
  $a+\epsilon$, che dipende sia dal valore $x_A=5$, sia dall'infinitesimo 
  $dx$ che compare nel risultato. Si tratta dunque di una pendenza 
  infinitamente vicina al valore $m=\dfrac{7}{5}$.
\end{esempio}

\begin{esempio}
 \label{esempio:diff01_m0diff}
 Ripetiamo il calcolo precedente, con riferimento all'ascissa del vertice 
 $x_V=\dfrac{3}{2}$ (per il valore dell'ascissa può essere di aiuto la 
lettura
 del grafico, se per caso nel tempo si fosse attenuato il ricordo della
 regola: $x_V=-b/2a$).\\
 $d(f(x)\big|_{x=3/2}=
  \dfrac{2}{5}x|_{x=3/2} dx+\dfrac{1}{5}(dx)^2-\dfrac{3}{5}dx=
  \dfrac{2}{5}\dfrac{3}{2}dx+\dfrac{1}{5}(dx)^2-\dfrac{3}{5}dx=\\
  =\dfrac{3}{5}dx+\dfrac{1}{5}(dx)^2-\dfrac{3}{5}dx=0+\dfrac{1}{5}(dx)^2$.\\
 Quindi il rapporto differenziale:\\
 $\dfrac{d(f(x)}{dx}\bigg|_{x=3/2}=\dfrac{\dfrac{1}{5}(dx)^2}{dx}
 =\dfrac{1}{5}dx$.\\
 La secante per punti infinitamente vicini al vertice della parabola
 differisce dalla retta orizzontale per un infinitesimo, cioè è
 infinitamente vicina alla retta orizzontale.
\end{esempio}

\begin{osservazione}
La pendenza calcolata nell'esempio \ref{esempio:diff01_mdiff} è 
$m\approx\dfrac{7}{5}$, mentre in quest'ultimo esempio 
\ref{esempio:diff01_m0diff} è
$m\approx 0$. Questo conferma che $m$ cambia a seconda del punto della 
curva: 
$m=m(x)$.
\end{osservazione}

\begin{esempio}
 In quale punto del piano cartesiano la parabola precedente è inclinata 
 di $45^\circ$ rispetto all'orizzontale?\\
 Risposta: poiché solo la retta $y=x$ ha in qualsiasi suo punto
 l'inclinazione richiesta dal problema, occorre cercare in quale punto
 la parabola risulta inclinata come la retta, cioè ha lo stesso 
coefficiente 
 angolare. È chiaro che non si può calcolare il coefficiente angolare
 di una parabola, ma si può immaginare che nel punto desiderato esista una 
retta
 tangente che risponde alle nostre esigenze. Cerchiamo quindi in quale
 punto, almeno approssimativamente, si possa disegnare una retta che ha 
$m=1$ e
 che quasi coincida con la parabola.\\
 Utilizziamo i calcoli precedenti e teniamo incognita $x$, dato che 
conosciamo
 già la pendenza desiderata:\\
 $\dfrac{df(x)}{dx}\bigg|_{x=?}=1 \srarrow 
 \dfrac{2}{5}x+\dfrac{1}{5}dx-\dfrac{3}{5}=1
 \srarrow \dfrac{2}{5}x=1+\dfrac{3}{5}-\dfrac{1}{5}dx
 \srarrow x=\dfrac{8-dx}{5}\dfrac{5}{2}=\dfrac{8-dx}{2}$.\\
 Il punto in questione ha coordinata $x= 4-\dfrac{1}{2}dx\approx 4$.  
\end{esempio}

\section{Derivata: definizione}
\label{sec:diff01_deriv}
Gli esercizi precedenti sono stati risolti con esattezza.
Purtroppo, però, il rapporto differenziale ci dà le soluzioni più semplici 
solo 
in pochi casi, praticamente inutili, cioè quando si applica alle funzioni 
polinomiali di primo grado (le rette nel p8iano cartesiano)).
In tutti gli altri casi il risultato
iperreale contiene infinitesimi che possono essere scomodi da gestire
negli sviluppi successivi.

\begin{esempio}
 Proseguendo con l'esempio \ref{esempio:diff01_mdiff}, calcoliamo in due 
 modi, esatto e approssimato, la coordinata $y$ del punto in questione:\\
 Calcolo esatto: $x=4-\dfrac{1}{2}dx\srarrow 
y=\dfrac{x^2}{5}-\dfrac{3}{5}x+2=
 \dfrac{(4-\dfrac{1}{2}dx)^2}{5}-\dfrac{3}{5}(4-\dfrac{1}{2}dx)+2=\\
 =\dfrac{1}{5}\tonda{16-dx+\dfrac{1}{4}(dx)^2}-\dfrac{12}{5}+\dfrac{3}{10}
 dx+2= \cdots= \dfrac{14}{5}+\dfrac{1}{2}dx+\dfrac{1}{20}(dx)^2\approx 
\dfrac{14}{5} =2,8$\\
 Calcolo approssimato: $x\approx 4\srarrow 
 y\approx\dfrac{4^2}{5}-\dfrac{3}{5}4+2=
 \dfrac{16}{5}-\dfrac{12}{5}+2=\dfrac{14}{5}=2,8$.\\
\end{esempio}

È chiaro che la seconda linea di calcoli è molto più gestibile della prima e
vorremmo poter avere sempre la comodità di una gestione facilitata.\\
Esiste una tecnica da applicare al risultato esatto iperreale per 
trasformarlo nel numero reale più vicino? Se esiste, possiamo guadagnare in 
agilità di calcolo, senza perdere troppo in precisione.

\begin{definizione}
  La \emph{derivata} della funzione $f(x)$ nel punto $\punto{x_0}{f(x_0)}$
  è, se esiste, la parte standard del rapporto differenziale della funzione,
  calcolato nello stesso punto. La derivata si indica con $f'(x_0)$.\\
  $f'(x_0)=\pst{\dfrac{d(f(x)}{dx}\bigg|_{x=x_0}}$.
\end{definizione}

La derivata, cioè l'applicazione della funzione $\pst{}$ al rapporto
differenziale, soddisfa le nostre 
esigenze: fornisce la migliore approssimazione reale del risultato ottenuto 
con il rapporto differenziale. La differenza fra questo e la 
derivata vale uno o più infinitesimi di ordine superiore, che nella 
maggior parte dei casi sono trascurabili.\\
$\dfrac{d(f(x)}{dx}\bigg|_{x=x_0}=f'(x_0)+\epsilon(x_0)\srarrow f'(x_0)
\approx \dfrac{d(f(x)}{dx}\bigg|_{x=x_0}$.

\subsubsection{Significato della derivata}
L'operazione di derivazione ha uno scopo molto più importante 
dell'indubbia comodità di fornire un risultato privo di infinitesimi:
essa consente di calcolare il tasso di variazione di una funzione in un 
dato punto. Per tasso di variazione non si intende semplicemente la 
differenza
fra due valori prossimi della funzione $df(x)$, ma la misura di tale 
differenza, ottenuta usando come unità di misura $dx$, cioè confrontandola
con la variazione della variabile.\\
Dal punto di vista geometrico, se si considera il grafico della funzione 
nel 
piano cartesiano iperreale, la derivata in un punto misura il tasso di 
crescita
della funzione lungo l'asse $Y$ rispetto alla variazione infinitesima lungo 
l'asse $X$, quindi misura la pendenza della tangente al grafico in quel 
punto.
Queste osservazioni sono la conseguenza del fatto che la derivata di una 
funzione 
in un punto è un numero standard.

\begin{osservazione}
 L'operazione di derivazione è conosciuta dai tempi di Leibniz e di Newton, 
 più o meno nei termini che qui sono stati esposti. Il problema attorno al 
 quale i matematici di quell'epoca concentravano i loro sforzi era
 relativo alle variazioni: le variabili erano chiamate 
 \emph{quantità fluenti}  e le variazioni di queste erano dette 
 \emph{flussioni}.
 Calcolare una velocità, per esempio, era calcolare il rapporto
 fra la flussione dello spazio rispetto alla flussione del tempo.
\end{osservazione}

\subsubsection{Nomi per la derivata}
Il nome \emph{derivata} per indicare il calcolo che abbiamo descritto ha 
origini storiche. Si è diffuso  ovunque (derivative, derivada, dérivée, 
...) 
anche se non rende pienamente il significato di ciò che rappresenta. Se ne 
potrà intuire la ragione in un capitolo successivo, quando parleremo anche 
di
funzioni primitive.\\
Sempre per ragioni storiche, si sono diffusi vari simboli che rappresentano 
l'operazione di derivazione:
\begin{enumerate}[noitemsep]
 \item $f'(x_0)$ è il simbolo per il risultato della derivazione di $f$ 
 per $x= x_0$: semplice e sintetico;
 \item $\mathit{D}\quadra{f(x)}$ indica la formula della derivazione di 
$f$, 
 per es. $\mathit{D}\quadra{5x\sqrt[3]{x^2}}=5\sqrt[3]{x^2}+\dfrac{10x}
 {3\sqrt[3]{x}}$;
 \item $\dot{f}$ equivale a $f'$; si usa in alcuni corsi universitari;
 \item $\dfrac{d}{dx}f(x)$ è come $f'(x)$: si pone in evidenza che si 
tratta 
 di una rapporto con $dx$;
 \item $\dfrac{df(x)}{dx}$ si trova spesso nei libri come se fosse 
esattamente 
 uguale a $f'(x)$. Sottolineamo che sono due cose diverse.  
 Nella maggior parte dei casi quest'uguaglianza si può accettare, 
trattandosi 
 di quantità infinitamente vicine, anzi indistinguibili. Per praticità, 
potremo
 anche noi seguire quest'uso, specificando la distinzione solo quando sarà 
 necessario.
\end{enumerate}

\subsubsection{Derivate facili e meno facili}
Nella definizione di derivata troviamo un inciso essenziale: \emph{se 
esiste}.
Significa che la derivata potrebbe anche non esistere, cioè non essere 
calcolabile? Vediamo alcuni esempi di calcoli che si portano a termine 
facilmente 
ed altri più problematici.

\begin{esempio}
  Calcola $f'(4)$ per la funzione $f(x)=1-2\sqrt{x}$.
%\begin{figure}[h]
\begin{inaccessibleblock}
  [Derivate radice]
 \begin{center}
 \begin{minipage}[]{.40 \textwidth}
    \vspace{-5mm} 
  \derivaradice
%  \caption{$f'(4)=-\frac{1}{2}$}
 \end{minipage} 
 \hfill
 \begin{minipage}[]{.58 \textwidth}
   \vspace{5mm}
  Si richiede la derivata di $f(x)=1-2\sqrt{x}$ nel punto 
  $\punto{4}{f(4)}$, che corrisponde, nel grafico, alla pendenza della 
  retta tangente alla curva, per $x=4$. Cioè dobbiamo calcolare:
  \begin{enumerate} [noitemsep]
   \item il differenziale della funzione;
   \item il rapporto fra questo e $dx$ per $x=4$;
   \item la parte standard del risultato precedente.
  \end{enumerate}
\end{minipage}
\end{center}
\end{inaccessibleblock}
\label{}
%\end{figure}

Lo svolgimento dei calcoli:
\begin{enumerate} [noitemsep]
 \item calcolare il differenziale della funzione: dalle regole apprese 
  sui differenziali (pag.\pageref{subsubsec:diff01_diffradq}) sappiamo che
  \begin{enumerate} [noitemsep]
   \item il differenziale di una differenza è la differenza dei 
   differenziali:\\
    $d(1-\sqrt{x})=d(1)-d(2\sqrt{x})$;
   \item il differenziale di una costante è nullo: $d(1)=0$;
   \item il differenziale del prodotto fra una costante e una funzione è
   $d(k(f(x))=kdf(x)$, quindi: $d(2\sqrt{x})=
   2\dfrac{dx}{(\sqrt{x+dx}+\sqrt{x})}\sim \dfrac{dx}{2\sqrt{x}}$. 
  \end{enumerate} 
  Per cui: $d(1-2\sqrt{x})\sim\tonda{0-\frac{dx}{\sqrt{x}}}=
  -\frac{dx}{\sqrt{x}}$.
 \item calcolare il rapporto fra questo e $dx$ nel punto richiesto:\\
  $\tonda{\frac{d(f(x)}{dx}}\bigg|_{x=4}\sim
  \tonda{\frac{-\frac{dx}{\sqrt{x}}}{dx}}\bigg|_{x=4}=-\frac{1}{\sqrt{4}}=
  -\frac{1}{2}$;
 \item calcolare la parte standard del risultato: 
  la parte standard di un numero indistinguibile da $-\frac{1}{2}$ è
  semplicemente: $\pst{-\frac{1}{2}}=-\frac{1}{2}$.
\end{enumerate}
La retta tangente in $\punto{4}{f(4)}$ ha pendenza pari a $-\frac{1}{2}$.
\end{esempio}

Con le regole già date sui differenziali il calcolo è privo di difficoltà, 
non 
sembra che la derivata per questa funzione possa creare problemi.

\begin{esempio}
  \label{esempio:diff01_deriradice}
Calcola $f'(0)$ per la funzione $f(x)=1-2\sqrt{x}$.\\
Riutilizziamo i calcoli precedenti.
\begin{enumerate} [noitemsep]
 \item $d(1-2\sqrt{x})=-\frac{dx}{\sqrt{x}}$;
 \item $\tonda{\frac{d(f(x)}{dx}}\bigg|_{x=0}\sim
  \tonda{\frac{-\frac{dx}{\sqrt{x}}}{dx}}\bigg|_{x=0}=-\frac{1}{\sqrt{0}}=
  \dots$?
 \item ?
\end{enumerate}
Una frazione nulla al denominatore non ha senso, il rapporto differenziale 
non 
è calcolabile e la derivata non esiste. 
\end{esempio}

Cerchiamo allora di capire cosa succede se il radicando è un infinitesimo 
non 
nullo $\epsilon>0$, quindi infinitamente vicino a 0.

\begin{esempio}
Calcolare $f'(\epsilon)$, sempre per  $f(x)=1-2\sqrt{x}$.\\
\begin{enumerate} [noitemsep]
 \item $d(1-2\sqrt{x})=-\frac{dx}{\sqrt{x}}$;
 \item $\tonda{\frac{d(f(x)}{dx}}\bigg|_{x=\epsilon}=
  \tonda{\frac{-\frac{dx}{\sqrt{x}}}{dx}}\bigg|_{x=\epsilon}\sim
  -\frac{1}{\sqrt{\epsilon}}=-M$ (con $\epsilon, M >0$);
 \item $\pst{\frac{d(f(x)}{dx}\bigg|_{x=\epsilon}}=\pst{-M}=$ ?
\end{enumerate} 
\begin{osservazione}
 $-M$ è un infinito negativo perché $\epsilon$ si suppone positivo. Non 
avrebbe 
 senso, comunque, fare un tentativo con $\epsilon$ negativo, perché la 
radice
 quadrata di numeri negativi (reali e iperreali) non è definita.\\
\end{osservazione} 

La parte standard di un numero infinito non esiste. La derivata non esiste, 
quindi la pendenza della tangente per $x=0$ non può essere calcolata.\\
Esiste però la pendenza della retta secante
fra i due punti infinitamente vicini $\punto{0}{f(0)=1}$ e 
$\punto{\epsilon}{f(\epsilon)}$. Infatti il rapporto 
differenziale appena calcolato approssima questa pendenza.\\
Vediamo nel dettaglio l'equazione di questa retta, con la formula della
retta passante per i due punti: $A\punto{0}{1}$ e 
$B\punto{\epsilon}{f(\epsilon)=1-2\sqrt{\epsilon}}$.\\
$\dfrac{x-x_A}{x_B-x_A}=\dfrac{y-y_A}{y_B-y_A}\srarrow
\dfrac{x-0}{\epsilon-0}=\dfrac{y-1}{1-2\sqrt{\epsilon}-1}\srarrow
\dfrac{x}{\epsilon}=\dfrac{y-1}{-2\sqrt{\epsilon}}\srarrow
y=\dfrac{-2\sqrt{\epsilon}}{\epsilon}x+\dfrac{1}{2\sqrt{\epsilon}}$.\\
La pendenza di questa secante è $m=\dfrac{-2\sqrt{\epsilon}}{\epsilon}=
-\dfrac{2}{\sqrt{\epsilon}}$.\\
La frazione ha senso per qualsiasi $\epsilon>0$, quindi si deve pensare
che se $x$ è un infinitesimo sempre più prossimo a $0$, $m$ diventa
sempre più negativa: la retta accentua sempre più la sua inclinazione verso
il basso fino ad assumere una direzione verticale, quando diventerà 
tangente in 
$\punto{0}{1}$.\\
Come possiamo visualizzare la pendenza della secante per $x\approx 0$?

\begin{inaccessibleblock}
  [Derivate radice]
 \begin{center}
 \begin{minipage}[]{.45 \textwidth}
  \derivaradiceinzero
 \end{minipage} 
 \hfill
 \begin{minipage}[]{.50 \textwidth}
\vspace{-2em}
In realtà non possiamo. Infatti se $dx=\epsilon$ abbiamo 
$f(\epsilon)=1-2\sqrt{\epsilon}$.
Realizziamo un primo microscopio con ingrandimento infinito, pari a 
$\dfrac{1}{\sqrt{\epsilon}}$, così $df(x)$ può essere visualizzata con un 
tratto verticale verso il basso, a partire dal punto $\punto{0}{1}$. Ma 
l'ingrandimento non è sufficiente per cogliere $dx=\epsilon$,  
infinitesimo di ordine superiore, quindi troppo piccolo.
Se poi applichiamo un secondo microscopio che visualizza $dx$, allora 
$df(x)$
assume lunghezza infinita e non può essere valutato. Nel punto 
$\punto{0}{1}$
la tangente (linea tratteggiata) è verticale.
\end{minipage}
\end{center}
\end{inaccessibleblock}
\label{}
\end{esempio}

\begin{esempio}
 %discontinua
 Per la funzione $f(x)=\dfrac{1}{x-2}$ calcola le derivate $f'(1)$ e 
$f'(2)$.\\
%\begin{figure}[h]
\begin{inaccessibleblock}
  [Derivate radice]
 \begin{center}
 \begin{minipage}[]{.40 \textwidth}
   \vspace{-.5em}
  \derivaomografica
 % \caption{}
 \end{minipage} 
 \hfill
 \begin{minipage}[]{.58 \textwidth}
   Per le regole che presto approfondiremo, 
$d\quadra{(x-2)^{-1}}~=~d(x^{-1})$
   perciò possiamo fare riferimento al teorema 
pag.\pageref{subsubsec:diff01_diffrecip}.
\begin{enumerate} [noitemsep]
 \item $d(f(x))=d\tonda {\frac{1}{x-2}}=-\dfrac{dx}{(x-2)^2+(x-2)dx}$;
 \item $\tonda{\dfrac{d(f(x)}{dx}}\bigg|_{x=1}=
  \tonda{\dfrac{-\frac{dx}{(x-2)^2+(x-2)dx}}{dx}}\bigg|_{x=1}=\\ 
  =-\dfrac{1}{1-dx}$;
  \item $\pst{-\dfrac{1}{1-dx}}=-1$.
\end{enumerate}
\end{minipage}
\end{center}
\end{inaccessibleblock}
\label{}
%\end{figure} 

Per $x=1$, la tangente ha pendenza $m=-1$.\\
Vediamo ora la seconda risposta.
\begin{enumerate} [noitemsep]
 \item $d(f(x))=d\tonda {\frac{1}{x-2}}=-\dfrac{dx}{(x-2)^2+(x-2)dx}$;
 \item $\tonda{\dfrac{d(f(x)}{dx}}\bigg|_{x=2}=
  \tonda{\dfrac{-\frac{dx}{(x-2)^2+(x-2)dx}}{dx}}\bigg|_{x=2} 
  =-\dfrac{1}{0-0dx}=$?;
  \item è inutile calcolare la parte standard di un numero privo di senso.
\end{enumerate}
Cosa è successo nel secondo caso? Che la funzione è discontinua per $x=2$.
Lo rende evidente il grafico, ma sarebbe stato meglio, prima ancora di
disegnarlo, studiare l'insieme di definizione e evitare calcoli inutili.
Infatti dobbiamo ricordarci che il differenziale è calcolabile solo nei 
punti
di continuità, di conseguenza il discorso vale anche per la derivata.
\end{esempio}

\begin{esempio}
 %valore assoluto
 Per la funzione $f(x)=\dfrac{1}{2}|x-2|+2$ calcola le derivate $f'(0)$, 
$f'(4)$
 e $f'(2)$.\\
 La funzione contiene un valore assoluto e può essere più semplice pensarla 
 come se fosse divisa in due rami:

\begin{figure}[h!]
\begin{inaccessibleblock}
  [Derivate radice]
 \begin{center}
 \begin{minipage}[]{.40 \textwidth}
  \derivavalass
  \caption{}
 \end{minipage} 
 \hfill
 \begin{minipage}[]{.58 \textwidth}
$f(x)= \dfrac{1}{2}|x-2|+2=\\
=\begin{cases}
  \dfrac{x-2}{2}+2 &\text{ per } x-2< 0  \\
 \dfrac{-(x-2)}{2}+2 &\text{ per } x-2 \ge 0
\end{cases} 
\srarrow\\
\srarrow f(x)= \begin{cases}
 \dfrac{x}{2}+1 &\text{ per } x<2 \\
 -\dfrac{x}{2}+3 &\text{ per } x\ge 2
\end{cases}$
\end{minipage}
\end{center}
\end{inaccessibleblock}
\label{}
\end{figure} 

Si tratta di due semirette che si uniscono in $\punto{2}{2}$. L'equazione
di ciascuna di loro è una funzione lineare e calcolare le derivate $f'(0)$
e $f'(4)$ è inutile: ne ricaveremmo comunque la pendenze delle semirette, 
cioè
$f'(0)=\dfrac{1}{2}$ e $f'(4)=-\dfrac{1}{2}$.\\
Il calcolo di $f'(2)$ invece è più interessante:\\
Abbiamo
$f'(x)= \begin{cases}
 \dfrac{1}{2} &\text{ per } x<2 \\
 -\dfrac{1}{2} &\text{ per } x > 2
\end{cases}$.
Quale è la pendenza giusta della tangente per
$x=2$, nel punto cioè dove il grafico cambia pendenza all'improvviso?\\
Tutto dipende dal differenziale e dal rapporto differenziale.
La funzione è continua, perciò $df(x)$ è sempre calcolabile.\\
Immagina $\dfrac{f(2+dx)-f(2)}{dx}$.
Se $dx$ è un qualsiasi infinitesimo positivo, siamo nel ramo destro del 
grafico
e il rapporto risulta negativo. Al contrario, se $dx<0$ siamo nel ramo
sinistro e il rapporto è positivo: la parte standard del rapporto 
differenziale
relativa al punto in cui $x=2$ non è unica, quindi non esiste.
Di conseguenza la derivata non esiste.
\end{esempio}

Da tutti questi esempi impariamo che per poter calcolare la derivata:
\begin{enumerate}[noitemsep]
 \item $f(x)$ deve essere continua nel punto desiderato ed è una condizione
 necessaria per poter derivare (ma non sufficiente);
 \item il rapporto differenziale deve essere un numero finito;
 \item il risultato deve essere indipendente dalla scelta di $dx$, cioè
 deve valere $\forall dx$;
\end{enumerate}
\begin{osservazione}
 Inoltre abbiamo visto un altro fatto importante: la derivata ha un 
risultato
 in genere diverso a seconda del valore $x_0$ per il quale viene calcolata, 
 cioè varia al variare di $x_0$. Poiché se si fissa $x_0$ il risultato, se
 esiste, è unico allora la derivata di una funzione è a sua volta una 
funzione.
 \end{osservazione}

 \begin{definizione}
  Una funzione per la quale la derivata è calcolabile $\forall x_0$ del suo
 dominio si dice funzione derivabile.
\end{definizione}

\begin{osservazione}
 Una funzione derivabile è sicuramente continua, mentre il contrario non 
vale.
\end{osservazione}



\section{Derivare funzioni algebriche}
\label{sec:diff01_derialg}
Sistemate le questioni preliminari, passiamo al calcolo: impariamo a 
derivare.
Nei casi semplici ci avvarremo di quanto visto a proposito dei 
differenziali,
ma, per le funzioni non trattate allora, dovremo calcolare anche questi.
Al termine, raccoglieremo i risultati utili in un prospetto riassuntivo.\\
Immaginiamo che le funzioni da derivare siano derivabili $\forall x$ 
dell'insieme di definizione, per cui la derivata di $f$ nel generico 
punto $\punto{x}{f(x)}$ sarà $f'(x)$.\\
Grazie al capitolo \ref{subsubsec:diff01_diffsint}, sappiamo già come
differenziare alcune funzioni algebriche: da quelle regole e dalla 
definizione di derivata ...deriva direttamente quanto segue.
\begin{teorema}
  La derivata di una funzione costante è $0$: $\mathit{D}\quadra{k}=0$.
\end{teorema}
\noindent Ipotesi: $f(x)=k$.\tab Tesi: $f'(x)=0$.
\begin{proof}
 Infatti $df(x)=0$
\end{proof}

\begin{teorema}
  La derivata della funzione identica è 1: $\mathit{D}\quadra{x}=1$.
\end{teorema}
\noindent Ipotesi: $f(x)=x$.\tab Tesi: $f'(x)=1$.
\begin{proof}
 Infatti $df(x)=\epsilon=dx$, quindi il rapporto differenziale è $1$ e così
 anche la sua parte standard.
\end{proof}
\begin{osservazione}
 $m=1$ è quindi anche la pendenza della bisettrice $y=x$, cosa ormai 
risaputa.
\end{osservazione}

\begin{teorema}
  La derivata della funzione quadratica è: $\mathit{D}\quadra{x^2}=2x$.
\end{teorema}
\noindent Ipotesi: $f(x)=x^2$ .\tab Tesi: $f'(x)=2x$.
\begin{proof}
  Infatti $df(x)=2xdx+(dx)^2$ e il rapporto differenziale è $2x+dx$ da cui,
  applicando la definizione di derivata, ...
\end{proof}

\begin{figure}[h!]
\begin{inaccessibleblock}
  [m tangenti a parabola]
 \begin{center}
 \begin{minipage}[]{.48\textwidth}
 \parabola
 \end{minipage} 
 \hfill
 \begin{minipage}[]{.48\textwidth}
  \tangentiparabola
 \end{minipage}
 \end{center}
\end{inaccessibleblock}
\caption{$y=x^2$ e la pendenza $y=m(x)=2x$ delle sue tangenti.} 
\label{}
\end{figure}

Iniziamo dal ramo sinistro del grafico: al crescere di $x$, la curva e le 
sue 
tangenti, indistinguibili da essa nei punti di tangenza, passano
da un'inclinazione fortemente verso il basso ($m<0$) alla direzione 
orizzontale,
nel vertice. Per $x>0$, poi, l'inclinazione aumenta progressivamente. Il 
progresso della pendenza delle tangenti è costante: per questo motivo il 
grafico di $y=m(x)$ è una retta per l'origine.

\begin{osservazione}
Nota che la funzione derivata di una funzione quadratica è una funzione 
lineare. Detto con eccessiva sintesi: la derivata di una parabola è una 
retta.\\
Con più precisione: la pendenza delle tangenti a una parabola
varia come varia la $y$ rispetto alla $x$ in una retta.
\end{osservazione}

\begin{teorema}
  \label{diff01_teoderpotenza}
  La derivata della generica funzione potenza è: $\mathit{D}\quadra{x^n}=
  nx^{n-1}$.
\end{teorema}
\noindent Ipotesi: $f(x)=x^n$ .\tab Tesi: $f'(x)=nx^{n-1}$.
\begin{proof}
  Infatti il differenziale è $df(x)=nx^{n-1}dx+ \delta(x)$ e,
  applicando la definizione di derivata, ...
\end{proof}

\begin{osservazione}
 Ripetendo l'osservazione a pag.\pageref{subsubsec:diff01_diffpot} relativa 
 a queste funzioni, il teorema \ref{diff01_teoderpotenza} è del tutto 
generale: si
 applica con qualsiasi esponente reale.
Vale quindi anche per le funzioni radicali di qualsiasi indice e per le 
funzioni
razionali fratte, come esemplifichiamo nei prossimi due casi, molto comuni.
\end{osservazione}

\begin{figure}[h!]
\begin{inaccessibleblock}
  [m tangenti a cubica]
 \begin{center}
 \begin{minipage}[]{.48\textwidth}
 \cubica
 \end{minipage} 
 \hfill
 \begin{minipage}[]{.48\textwidth}
  \tangenticubica
 \end{minipage}
 \end{center}
\end{inaccessibleblock}
\caption{$y=x^3$ e la pendenza $y=m(x)=3x^2$ delle sue tangenti.} 
\label{}
\end{figure}
Come esempio di derivata della funzione potenza, consideriamo $f(x)=x^3$
e il suo grafico nel piano cartesiano. I due rami del grafico sono 
simmetrici
rispetto all'origine e quindi lo sono anche le pendenze delle tangenti. 
Considerando le $x$ crescenti, quindi da sinistra verso destra, le pendenze 
delle tangenti sono sempre positive, all'inizio molto accentuate, poi 
diminuiscono
fino a $m=0$. Oltre l'origine, riprendono a crescere, in maniera sempre più 
accentuata. Il grafico di $y=m(x)=3 x^2$ ha infatti la forma di una parabola
simmetrica rispetto all'asse $Y$.

\begin{corollario}
  La derivata della funzione radice quadrata è: 
$\mathit{D}\quadra{\sqrt{x}}=
    \dfrac{1}{2\sqrt{x}}$, con la restrizione $x\ne 0$.
\end{corollario}
\noindent Ipotesi: $f(x)=\sqrt{x}$, con $x\ne 0$ .\tab Tesi: 
    $f'(x)=\dfrac{1}{2\sqrt{x}}$.
\begin{proof}
  Infatti il differenziale è $df(x)=\dfrac{dx}{\sqrt{x+dx}+\sqrt{x}}$ e,
  applicando la definizione di derivata, si ha:\\
  $\pst{\dfrac{dx}{dx\tonda{\sqrt{x+dx}+\sqrt{x}}}}=
  \pst{\dfrac{1}{\sqrt{x+dx}+\sqrt{x}}}=
      \dfrac{1}{\pst{\sqrt{x+dx}+\sqrt{x}}}=
      \dfrac{1}{\pst{\sqrt{x+dx}}+\pst{\sqrt{x}}}=\\
      =\dfrac{1}{\sqrt{x}+\sqrt{x}}=\dfrac{1}{2\sqrt{x}}$.
\end{proof}

\begin{figure}[h!]
\begin{inaccessibleblock}
  [m tangenti a radquad]
 \begin{center}
 \begin{minipage}[]{.48\textwidth}
 \radquad
 \end{minipage} 
 \hfill
 \begin{minipage}[]{.48\textwidth}
  \tangentiradquad
 \end{minipage}
 \end{center}
\end{inaccessibleblock}
\caption{$y=\sqrt{x}$ e la pendenza $y=m(x)=\dfrac{1}{2\sqrt{x}}$ delle sue 
  tangenti.} 
\label{}
\end{figure}

Le rette tangenti ai punti vicini all'origine hanno una pendenza elevata, 
che 
si attenua gradualmente man mano che $x$ aumenta, fino ad assestarsi quasi
orizzontalmente.

\begin{corollario}
  La derivata della funzione reciproca è: $\mathit{D}\quadra{\dfrac{1}{x}}=
    -\dfrac{1}{x^2}$.
\end{corollario}
\noindent Ipotesi: $f(x)=\dfrac{1}{x}$.\tab Tesi: $f'(x)=-\dfrac{1}{x^2}$.
\begin{proof}
  Infatti il differenziale è $df(x)=\dfrac{-dx}{x(x+dx)}$ e,
  applicando la definizione di derivata, si ha:\\
  $\pst{\dfrac{-dx}{dx(x(x+dx))}}=
\dfrac{-1}{\pst{x(x+dx)}}=\dfrac{-1}{\pst{x}\pst{x+dx}}=
      -\dfrac{1}{x\cdot x}=-\dfrac{1}{x^2}$.
\end{proof}

\begin{figure}[h!]
\begin{inaccessibleblock}
  [m tangenti a reciproca]
 \begin{center}
 \begin{minipage}[]{.48\textwidth}
 \recip
 \end{minipage} 
 \hfill
 \begin{minipage}[]{.48\textwidth}
  \tangentirecip
 \end{minipage}
 \end{center}
\end{inaccessibleblock}
\caption{$y=\dfrac{1}{x}$ e la pendenza $y=m(x)=\dfrac{-1}{x^2}$ delle sue 
  tangenti.} 
\label{}
\end{figure}

\begin{osservazione}
 Ovviamente, applicando alla lettera il teorema sulla derivata delle 
funzioni
 potenza si ottengono gli stessi risultati esposti in questi due ultimi 
 corollari.
\end{osservazione}

\section{Regole di derivazione}
\label{sec:diff01_regolederivate}
Possiamo applicare i teoremi precedenti a casi meno elementari, cioè
a funzioni algebriche che contengono somme, prodotti e quozienti di
funzioni elementari. 
\begin{esempio}
  Derivare la funzione $f(x)= 3x-\dfrac{3}{x}$ in $x_0=3$\\
  Si tratta di una funzione nuova, ma è facile riconoscere che è formata 
  dalla somma (algebrica) di due funzioni e ciascuna di queste è data dal 
  prodotto fra la costante $3$ e una funzione appena trattata. Perciò:\\
  $\mathit{D}\quadra{3x}=3\cdot\mathit{D}\quadra{x}=3\cdot 1=3$; \tab
  $\mathit{D}\quadra{\dfrac{3}{x}}=3\cdot\mathit{D}\quadra{\dfrac{1}{x}}
  =3\cdot \dfrac{-1}{x^2}=\dfrac{-3}{x^2}$;\\
  $f'(x)=\mathit{D}\quadra{3x-\dfrac{3}{x}}=\mathit{D}\quadra{3x}-
  \mathit{D}\quadra{\dfrac{3}{x}}= 3-\dfrac{-3}{x^2}= 3+\dfrac{3}{x^2}$;\\
  $f'(3)=3+\dfrac{3}{9}= 10$.\\
\end{esempio}
Senza troppi problemi, abbiamo dato per scontato che 
\begin{enumerate}[noitemsep]
 \item La derivata del prodotto tra una costante e  una funzione è il 
prodotto
 fra la costante e la derivata della funzione:\\ 
 $\mathit{D}\quadra{kf(x)}=k\mathit{D}\quadra{f(x)}$.
 \item La derivata della somma algebrica fra due funzioni è la somma 
algebrica
 delle due derivate:\\
$\mathit{D}\quadra{f(x)+g(x)}=\mathit{D}\quadra{f(x)}+\mathit{D}\quadra{g(x)
}$.
\end{enumerate}
Da dove derivano queste certezze? Basta tornare alle regole di composizione
dei differenziali (pag.\pageref{subsec:diff01_combdiff}) per averne la 
conferma.

\begin{esempio}
Deriva la funzione che nel piano cartesiano è rappresentata dalla retta 
$y=x+9$.\\
$\mathit{D}\quadra{x+9}=1$.
\end{esempio}

Sempre in riferimento a quanto appreso sui differenziali 
e vista la definizione di derivata e le proprietà della parte standard, 
giustifichiamo facilmente anche le regole 3 e 4:
\begin{enumerate}[noitemsep]
\setcounter{enumi}{2}
\item La derivata del prodotto fra due funzioni è la somma fra due prodotti:
la derivata della prima funzione per la seconda (non derivata) più la prima 
funzione (non derivata) per la derivata della seconda:\\
$\mathit{D}\quadra{f(x)\cdot g(x)}=f'(x)\cdot g(x)+f(x)\cdot g'(x)$.
\item La derivata del quoziente fra due funzioni è la frazione che ha per 
denominatore il quadrato del divisore e per numeratore la differenza fra due
prodotti: la derivata della prima funzione per la seconda (non derivata) 
meno
la prima funzione (non derivata) per la derivata della seconda:\\
$\mathit{D}\quadra{\dfrac{f(x)}{g(x)}}=\dfrac{f'(x)\cdot g(x)-f(x)\cdot 
g'(x)}
  {\quadra{g(x)}^2}$.
\end{enumerate} 

\begin{esempio}
  Calcola la derivata del prodotto $f(x)=x\sqrt{x}$.\\
  $f'(x)=1\cdot \sqrt{x}+x\cdot\frac{1}{2\sqrt{x}}=
  \sqrt{x}+\frac{x}{2\sqrt{x}}$.\\
  Fin qui l'applicazione della regola. Ma il risultato si può scrivere 
  in forma più compatta, perché $\sqrt{x}+\frac{x}{2\sqrt{x}}=
  \sqrt{x}+\frac{\sqrt{x}}{2}=\frac{3}{2}\sqrt{x}$.\\
  In realtà per fare questo calcolo non avremmo bisogno della regola del 
  prodotto, poiché $f(x)=x\sqrt{x}=x^{1+\frac{1}{2}}=x^{\frac{3}{2}}$. Puoi 
  quindi applicare il teorema \ref{diff01_teoderpotenza} e controllare il 
  risultato.
\end{esempio}

\begin{esempio}
  Derivare $f(x)=\dfrac{x}{\sqrt{x}}$.\\
  Seguendo la regola n.4: $f'(x)= 
\dfrac{1\cdot\sqrt{x}-x\dfrac{1}{2\sqrt{x}}}
  {\tonda{\sqrt{x}}^2}=\dfrac{\sqrt{x}-\dfrac{\sqrt{x}}{2}}{x}=
  \dfrac{\dfrac{\sqrt{x}}{2}}{x}=\dfrac{1}{2\sqrt{x}}$.\\
  Ma guarda che combinazione: abbiamo ottenuto la derivata della radice! 
Allora 
  la funzione di partenza è equivalente a una radice? (Ad essere precisi, 
non 
  esattamente. Infatti \dots)
\end{esempio}
\begin{esempio}
  Sappiamo già che $\mathit{D}\tonda{\dfrac{1}{x}}=\dfrac{-1}{x^2}$. 
Mettiamo 
  alla prova ancora una volta la regola n.4: $f'(x)=\dfrac{0\cdot x-1\cdot 
1}
  {x^2}= \dots$
\end{esempio}

\begin{esempio}
 Ora finalmente un calcolo che si può svolgere solo con la regola n.4.
 Derivare $f(x)= \dfrac{x+2}{x^3-x+4}$.\\
 $f'(x)=\dfrac{1\cdot(x^3-x+4)-(x+2)(3x^2-1)}{(x^3-x+4)^2}$. Fin qui 
 l'applicazione della regola. \\
 Con ulteriori calcoli:
 $\dots =\dfrac{x^3-x+4-(3x^3-x+6x^2-2)}{(x^3-x+4)^2}=
 \dfrac{-2x^3-6x^2+6}{(x^3-x+4)^2}$.\\
\end{esempio}


\section{Derivare funzioni composte e funzioni inverse}
\label{}
\subsection{Funzioni composte}
\label{subsec:diff01_dericomp}

\begin{esempio}
 $s(t)~=~s_0~+~v_0t~+~\frac{1}{2}at^2$ è la legge oraria del moto rettilineo
 uniformemente accelerato. Anche se nella formula mancano le usuali 
 sigle $f(x)$, $y$, $x$, si tratta di una comunissima funzione polinomiale
 di 2° grado e le si possono applicare le regole che stiamo studiando, senza
 problemi.\\
 Infatti, derivando si ottiene $s'(t)=0+v_0+\frac{1}{2}\cdot 2at=v_0+at$ 
che,
 essendo la derivata dello spazio rispetto al tempo, esprime la velocita 
$v(t)$
 in questo tipo di moto.
\end{esempio}
L'esempio serve a ricordare che le funzioni e le variabili si esprimono con 
sigle qualsiasi, ma questo non cambia le regole dell'analisi o, più in
generale, della matematica. La libertà di uso dei simboli può facilitare i
calcoli, come si vede nel caso della derivata di una funzione composta.
\begin{esempio}
  Deriva la funzione $v(u)=\dfrac{u^2}{8}$. 
  Soluzione: $v'(u)=\dfrac{1}{8}2u=\dfrac{u}{4}$. Infatti il 
  differenziale è $dv(u)=\dfrac{1}{8}[2udu+(du)^2]$ perché 
$df(x)=2xdx+(dx)^2$ 
  e $d(kf(x)=kdf(x)$. Allora  la parte 
 standard del rapporto differenziale fornisce il risultato $\dfrac{u}{4}$.
\end{esempio}

\begin{esempio}
Deriva la funzione $u(t)=3t-2$. Soluzione: $u'(t)=3$. Infatti il 
 differenziale è $du(t)=3dt+ 0$. Allora la parte 
 standard del rapporto differenziale fornisce il risultato $3$.
\end{esempio}

Combiniamo i due esempi: $v=f(u)$ e $u=f(t)$, cioè $v$ è funzione di $u$, 
perché i suoi valori dipendono dai quadrati, divisi per $8$, dei numeri $u$ 
,
invece $u$ è funzione di $t$, nel senso che i suoi valori sono i valori $t$ 
triplicati e poi ridotti di 2. In ``matematichese'' 
$v(u(t))=\dfrac{[u(t)]^2}{8}=\dfrac{(3t-2)^2}{8}$.


\begin{inaccessibleblock}
  [box funzione composta]
 \begin{center}
 \begin{minipage}[]{.48\textwidth}
  \boxfcomposta
 \end{minipage} 
 \hfill
 \begin{minipage}[]{.48\textwidth}
Si tratta di una specie di catena:\\
se si immette il valore $t=6$, la macchina sviluppa $u(6)=3\cdot 6-2=16$ in 
$u$
e infine produce $v(16)=\dfrac{16^2}{8}=32$. Una catena del genere si 
chiama 
\emph{funzionedi funzione} e $v$, che produce il risultato finale, si dice
\emph{funzione composta}: $v(u(t))$.\\
 \end{minipage}
 \end{center}
\end{inaccessibleblock}
\label{}

Come deriviamo  $v$ rispetto a $t$? Dalla definizione di derivata: 
$\mathit{D}\quadra{v(u(t))}=\pst{\dfrac{d(v(u(t)))}{dt}}$, 
quindi il punto è il calcolo dei differenziali.
\begin{inaccessibleblock}
  [differenziale funzione composta]
 \begin{center}
 \begin{minipage}[]{.48\textwidth}
  \diffcomposta
 \end{minipage} 
  \hfill
 \begin{minipage}[]{.48\textwidth}
Dal primo esempio sappiamo che $dv=\dfrac{u}{4}du\ +$ infinitesimi di 
ordine 
superiore.
Poiché $u=3t-2$, $du=3dt$, avremo:\\
$dv\approx \dfrac{u}{4}du$. \hspace{1cm} $du=3dt$. $\srarrow 
dv=\dfrac{3t-2}{4}3dt \srarrow\\ 
\srarrow 
\pst{\dfrac{d(v(u(t)))}{dt}}=\dfrac{(3t-2)3dt}{4dt}=\dfrac{9}{4}t-\dfrac{3}{
2}$.\\
 \end{minipage}
 \end{center}
\end{inaccessibleblock}
\label{}

C'è un modo più semplice? Sì: basta sviluppare il quadrato $(3t-2)^2$, 
dividere
ogni termine per $8$ e poi derivare il polinomio. Ma a volte il modo più 
semplice 
non c'è.
\begin{esempio}
  Calcolare $f'(x)$, con $f(x)=\sqrt{3-x^2}$.\\
  $f(x)$ è composta: si può pensare
  formata così: $g(x)=3-x^2$ e $f(g(x))=\sqrt{g(x)}=\sqrt{3-x^2}$.\\
  In questo modo si vedono meglio i differenziali.
  $df(x)=d\tonda{\sqrt{g(x)}}\approx\frac{1}{2\sqrt{g(x)}}\cdot dg(x)$ e 
  $dg(x)=d(3-x^2)\approx-2xdx$. Per brevità, raccogliamo sotto un'unica
  sigla $\epsilon$ tutti gli infinitesimi 
  di ordine superiore, che poi la parte standard si incaricherà di far 
  scomparire nel momento di calcolare la derivata.\\
  Il differenziale: $df(x)=df(g(x))=\frac{1}{2\sqrt{g(x)}}\cdot 
dg(x)+\delta=
  \frac{1}{2\sqrt{3-x^2}}\cdot (-2x)dx+ \epsilon$.\\
  Da qui la derivata: 
  $f'(x)=\pst{\dfrac{\frac{-2xdx}{2\sqrt{3-x^2}} +\epsilon}{dx}}=
  \dfrac{-x}{\sqrt{3-x^2}}$.
\end{esempio}
Esaminiamo in modo astratto come abbiamo costruito il rapporto 
differenziale della funzione composta nell'esempio precedente: 
$\dfrac{df}{dx}=\dfrac{df}{dg}\dfrac{dg}{dx}$. 
Sembra un'uguaglianza banale, perché semplificando si ottiene
l'identità. In realtà i due differenziali $dg$ hanno un significato
diverso. Quello al denominatore differenzia la variabile indipendente per 
$f$, quello al numeratore differenzia la variabile che dipende da $x$.\\
L'espressione giustifica il teorema seguente.

\begin{teorema}
  \label{teo:diff01_dericomp}
 Se esistono le derivate $g'(x)$ e $f'(g(x))$ per il medesimo valore $x$,
 la funzione composta $f(g(x))$ è derivabile e la sua derivata si calcola
 così: $f'(x)=f'(g(x))=f'(g)g'(x)$, cioè la derivata di una funzione 
composta
 è il prodotto delle derivate delle funzioni componenti, ciascuna rispetto
 alla propria variabile.
\end{teorema}
\noindent Ipotesi: $f(x)=f(g(x))$, $f$,$g$ derivabili .\tab 
Tesi: $f'(x)=f'(g(x))=f'(g(x))g'(x)$.
\begin{proof}
  Tralasciando di specificare la scomparsa degli infinitesimi di ordine 
  superiore e grazie alle proprietà della funzione $\pst{}$, abbiamo:\\
  $f'(g(x))=\pst{\dfrac{df(x)}{dx}}=
  \pst{\dfrac{df(g)}{dg}\dfrac{dg(x)}{dx}}=
  \pst{\dfrac{df(g))}{dg}}\pst{\dfrac{dg(x)}{dx}}=
  f'(g(x))g'(x)$.
\end{proof}
\begin{esempio}
  Derivare $f(x)=\tonda{-\dfrac{3}{2}x^3+2x^2-6}^5$.\\
  Poniamo $g(x)=-\dfrac{2}{3}x^3+2x^2-6$ e $f(g)=g^5$. \\
  Allora: 
  $f'(g)= 5g^4$ e $g'(x)=-2x^2+4x$, \\
  quindi: 
  $f'(x)=f'(g)\cdot g'(x)=5g^4(-2x^2+4x)=
  5\tonda{-\dfrac{2}{3}x^3+2x^2-6}^4(-2x^2+4x)=\\
  =10x\tonda{-\dfrac{2}{3}x^3+2x^2-6}^4(-x+2)$ .
\end{esempio}
\begin{osservazione}
 La regola della funzione composta si estende ai casi in cui le funzioni 
 in gioco sono tre, o più:
 $\mathit{D}\quadra{f(g(h(x)))}=f'(g)\cdot g'(h)\cdot h'(x)$.
\end{osservazione}
\begin{osservazione}
 Lo studente smart si era già accorto che la derivata di un prodotto non 
 è il prodotto delle derivate. Ora arriva la conferma:
 il prodotto delle derivate non è la derivata di un prodotto.
\end{osservazione}


\subsection{Funzioni inverse}
\label{subsec:diff01_derifuninverse}
Non è difficile invertire una funzione nota, partendo dalla sua espressione 
analitica.  Con pochi pochi passaggi elementari, si ricava la formula
inversa.. Ecco alcuni esempi di semplici funzioni algebriche.
\begin{center}
\begin{tabular}{|c|c|c|c|}
$y=f(x)$ & $x=g(y)$ & $f'(x)$ & $g'(y)$\\\hline
$y=x+c$ & $x=y-c$ & 1 & 1\\
$y=kx$ & $x=y/k$ & $k$ & $\frac{1}{k}$\\
$y=x^2$, $x\ge 0$ & $x=\sqrt{y}$ & $2x$ & 
$\frac{1}{2\sqrt{y}}=\frac{1}{2x}$\\
$y=\frac{1}{x}$ & $x=\frac{1}{y}$ & $-\frac{1}{x^2}$ & 
$-\frac{1}{y^2}=-x^2$\\
\hline
\end{tabular}
\end{center}

\begin{osservazione}
La funzione  $y=x^2$, nella terza riga della tabella, è definita $\forall 
x$.
Tuttavia qui si restringe il dominio, in modo da considerare un solo ramo 
della
parabola, perché altrimenti la formula inversa non potrebbe essere una 
funzione.
Occorre sempre porre attenzione al dominio di $f$, quando si vuole 
definirne 
l'inversa. Una buona regola pratica per capire se $f^{-1}$ è definibile, è 
di
tagliare il grafico di $f$ con una retta orizzontale: se la retta incrocia 
il 
grafico di $f$ in più punti, $f^{-1}$ non esiste.\\
\end{osservazione}

Le ultime due colonne riportano le derivate rispettive e insinuano in
noi qualche sospetto.\\
Considera il caso semplice che segue.
\begin{esempio}
  Derivare $f(x)=\sqrt{x^2}$.\\
  Si dirà: non c'è problema, si calcola la funzione e risulta 
  $f(x)=\sqrt{x^2}=x$, perciò  $f'(x)=1$. \\
  Vero. Ma poniamo $g(x)=x^2$ e $f(x)=f(g(x))=\sqrt{g(x)}$.\\
  Con la regola delle funzioni composte si ha:\\ 
  $f'(x)=f'(g)g'(x)= \dfrac{1}{2\sqrt{g}}\cdot 2x=
  \dfrac{1}{2\sqrt{x^2}}\cdot 2x=\dfrac{1}{2x}\cdot 2x=1$.\\
  Conclusione: 
$\mathit{D}\quadra{x^2}=\dfrac{1}{\mathit{D}\quadra{\sqrt{x}}}$
\end{esempio}
Si intuisce che: siccome $f'(g)g'(x)=1$, allora $g'(x)=\dfrac{1}{f'(g)}$.
Se fosse dimostrato, diventerebbe più facile derivare per esempio le 
funzioni 
logaritmiche: basterebbe saper derivare le corrispondenti esponenziali. E 
così
via. Ovviamente occorre qualche cautela: la regola sarebbe applicabile
 \begin{enumerate} [noitemsep]
  \item se esiste l'inversa della funzione da derivare;
  \item se entrambe le funzioni sono derivabili;
  \item se  $f'(g)\ne 0$.
 \end{enumerate}
Infatti, nell'esempio tutto funziona alla perfezione, ma solo se $x\ne0$
(ricorda anche l'esempio \ref{esempio:diff01_deriradice}). \\
Se valgono tutte le condizioni favorevoli, allora esistono la funzione
$f$ e la sua inversa $g=f^{-1}$. $y=f(x)$ e $y=g(x)=f^{-1}(x)$ hanno 
grafici 
simmetrici rispetto alla bisettrice $y=x$.
\begin{osservazione}
 $y=kx$ e $x=\frac{y}{k}$, per esempio, sono formule inverse l'una 
 dell'altra. Ma non sono funzioni inverse, sono due espressioni della 
stessa 
 iperbole equilatera e hanno lo stesso grafico, quindi le stesse tangenti al
 grafico e le stesse derivate rispetto a $x$.\\ La funzione inversa di cui 
 parliamo, nel caso dell'iperbole è $y=\frac{x}{k}$, cioè è la formula 
inversa,
 ma applicata a $x$. Se invece consideriamo la formula inversa 
$x=\frac{y}{k}$
 come funzione $x=f(y)$, allora dovremo derivare rispetto a $y$: 
 $x'=\dfrac{dx}{dy}$.
\end{osservazione}

\begin{inaccessibleblock}
  [differenziale funzione composta]
 \begin{center}
 \begin{minipage}[]{.55\textwidth}
  \diffinversa
 \end{minipage} 
  \hfill
 \begin{minipage}[]{.42\textwidth}
Ogni punto $\punto{x}{f^{-1}(x)}$ sulla curva della funzione inversa ha un
corrispondente $\punto{y}{f(y)}$ sulla curva $y=f(x)$, nella simmetria 
rispetto
alla bisettrice. Guardiamo come si corrispondono i differenziali:
$dx$ e $dy$ sono invertiti in una curva rispetto all'altra. Quindi le 
derivate corrispondenti sono reciproche l'una con l'altra.
 \end{minipage}
 \end{center}
\end{inaccessibleblock}
\label{}

\begin{teorema}
 Le derivate di due funzioni $f$, $g$, inverse l'una dell'altra, se esistono
 e sono diverse da zero, sono reciproche l'una rispetto all'altra.
\end{teorema}
\noindent Ipotesi: $y=f(x)$, $x=g(y)$ $f$,$g$ derivabili, con $f'\ne0$, 
$g'\ne 0$.
\hspace{2cm} Tesi: $f'(x)=\dfrac{1}{g'(y)}$.
\begin{proof}
  Grazie alle proprietà della funzione $\pst{}$, abbiamo:\\
  $f'(x)\cdot g'(y)=\pst{\dfrac{dy}{dx}}\cdot\pst{\dfrac{dx}{dy}}=
  \pst{\dfrac{dy}{dx}\cdot\dfrac{dx}{dy}}=\pst{1}=1$\\
  per cui: $f'(x)=\dfrac{1}{g'(y)}$.
\end{proof}
\begin{osservazione}
 La dimostrazione fa leva su una semplificazione che sembra banale. In 
realtà
 i due rapporti differenziali sono diversi per significato: nel primo la 
 variabile indipendente è $x$, nel secondo è $y$.
\end{osservazione}
\begin{esempio}
  Trova la derivata di $f(x)=\dfrac{1}{\sqrt{5-x^2}}$.
  \begin{enumerate}[noitemsep]
   \item Usando il teorema \ref{teo:diff01_dericomp} e le regole 
precedenti:\\
   $f'(x)=\mathit{D}\quadra{\dfrac{1}{\sqrt{5-x}}}=
   \mathit{D}\quadra{(5-x)^\frac{-1}{2}}=-\dfrac{1}{2}(5-x)
   ^\frac{-3}{2}(-1) = \dfrac{1}{2(\sqrt{5-x})^3}$.
   \item Usando la regola appena appresa:\\
   Costruiamo la formula inversa con pochi passaggi algebrici: riavremo la 
   stessa funzione, in cui $y$ figura come variabile indipendente: 
$x=f(y)$.\\
   Quindi deriviamo: $\mathit{D}\quadra{x}=x'=f'(y)=\dfrac{dx}{dy}$.\\
   $f(x)=y=\dfrac{1}{\sqrt{5-x}}\srarrow y^2=\dfrac{1}{5-x}\srarrow 
   y^{-2}=5-x\srarrow x=5-y^{-2}$  (formula inversa)\\
   $x'=\dfrac{dx}{dy}=-2y^{-3}$ (derivata della funzione inversa)
   $\srarrow \dfrac{dy}{dx}=\dfrac{y^3}{2}=
   \dfrac{1}{2(\sqrt{5-x})^3}$.
  \end{enumerate}
In genere la funzione inversa si costruisce in pochi passaggi semplici, poi 
la derivazione risulta elementare.
\end{esempio}



\section{Derivare funzioni trascendenti}
\label{sec:diff01_deritrasc}
Nel capitolo \ref{sec:diff01_differenziale} abbiamo imparato a 
differenziare 
le funzioni algebriche. In questa parte, quindi, per imparare
a derivare le funzioni trascendenti, dovremo calcolare anche i loro 
differenziali. Si tratta in realtà di un lavoro minimo, perché abbiamo già
discusso (pag.\pageref{subsubsec:insnum_expstar} il comportamento di queste 
funzioni
nell'insieme degli Iperreali e abbiamo ormai un bagaglio di conoscenze 
sulle derivate che agevola il lavoro.
\subsection{Derivata di $f(x)=a^x$}
\label{}
Il grafico di una generica funzione esponenziale $y=a^x$, confrontato con 
quello 
dell'andamento delle tangenti è una sorpresa rispetto ai confronti che 
abbiamo
fatto per altre funzioni. I due grafici praticamente si accompagnano: 
rivelano uguali pendenze in coppie di punti di uguale ordinata.

\begin{inaccessibleblock}
  [esponenziale e derivate]
  \begin{minipage}[]{.45\textwidth}
   \begin{center} \esp \end{center}
 \end{minipage} 
  \hfill
 \begin{minipage}[]{.42\textwidth}
   \begin{center} \tangentiesp \end{center}
 \end{minipage}
\end{inaccessibleblock}
\label{}
\\
Anche se i due grafici non sono esattamente identici, l'andamento delle 
pendenze delle tangenti, cioè l'andamento della derivata della funzione, è 
anch'esso esponenziale.
Sviluppiamo matematicamente questa intuizione, ricordando le proprietà
delle potenze.\\
Differenziale di $y=a^x$: 
$dy=a^{x+dx}-a^x=a^xa^{dx}-a^x=a^x\tonda{a^{dx}-1}$.\\
Rapporto differenziale: $\dfrac{dy}{dx}=\dfrac{\tonda{a^{dx}-1}}{dx}a^x$.\\
Ora dovremmo applicare la parte standard e se vogliamo seguire le 
indicazioni
del grafico, il risultato dovrà essere un'esponenziale. Quindi dobbiamo 
concentrarci sul rapporto $\dfrac{\tonda{a^{dx}-1}}{dx}$, il fattore che 
potrebbe
provocare l'allontanamento del grafico dalla forma esponenziale a cui 
puntiamo.\\
Negli esponenziali succede che $f(0)=1$ e $f(x+dx)=f(x)f(dx)$ e il rapporto 
differenziale in queste funzioni diventa:\\
$\dfrac{df(x)}{dx}=\dfrac{f(x+dx)-f(x)}{dx}=\dfrac{f(x)f(dx)-f(x)}{dx}=
\dfrac{f(dx)-1}{dx}f(x)=\dfrac{f(0+dx)-f(0)}{dx}f(x)\sim\\
\sim f'(0)f(x)\srarrow f'(x)=f'(0)f(x)$.\\
La derivata di una funzione esponenziale generica è proporzionale alla
funzione stessa, attraverso un fattore che corrisponde alla derivata 
calcolata
in $x=0$.\\
Per capire di più cos'è questo fattore, forziamo la situazione e imponiamo 
che
corrisponda a $1$. In questo modo la funzione e la sua derivata saranno 
proprio
identiche e i due grafici saranno sovrapposti: finalmente potremo conoscere
quanto vale la base generica $a$.\\
$f'(0)=1\srarrow \dfrac{\tonda{a^{dx}-1}}{dx}=1\srarrow a^{dx}=dx +1
\srarrow a=(dx+1)^\frac{1}{dx}$.\\
Abbiamo già incontrato un'espressione analoga in 
\ref{subsubsec:insnum_nepero}:
l'espressione individua il Numero di Nepero $e$.
\begin{teorema}
 La derivata della funzione esponenziale $\mathit{D}\quadra{e^x}$ coincide 
con
 la funzione stessa.
\end{teorema}
\noindent Ipotesi: $f(x)=e^x$. \tab $f'(x)=e^x$
\begin{proof}
 La dimostrazione è già stata costruita gradualmente per via intuitiva. 
 Occorrerebbe dimostrare l'unicità della tesi, ma non è essenziale
 per i nostri scopi. Resta comunque stabilito che \emph{se una funzione 
coincide
   con la propria derivata, allora è una funzione esponenziale}.
\end{proof}
A questo punto, l'importanza del numero $e$ risulta ingigantita. Ce ne 
serviamo 
subito.

\begin{teorema}
  La derivata della generica funzione esponenziale è: 
$\mathit{D}\quadra{a^x}=
  a^x\ln{a}$.
\end{teorema}
\noindent Ipotesi: $f(x)=a^x$; \tab Tesi: $f'(x)=a^x\ln{a}$.
\begin{proof}
 Usiamo una trasformazione appresa con lo studio dei logaritmi e 
applichiamo 
 il teorema \pageref{teo:diff01_dericomp}:
 $f(x)~=~a^x~=~e^{\ln a^x}$. Se poniamo $g(x)=\ln a^x=x\ln a$, si ottiene:\\
 $f(g(x))=e^{g(x)}\srarrow f'(g(x))=f'(g)g'(x)= e^{x\ln a}\ln a=
 e^{\ln a^x}\ln a=a^x\ln a$.
\end{proof}
\begin{esempio}
  Calcola la derivata di $f(x)=3e^{x-1}$.\\
  Poniamo $g(x)=x-1$. $f(x)=3e^{g(x)}\srarrow f'(x)=3e^{g(x)}\cdot g'(x)=
  3e^{x-1}\cdot 1=3e^{x-1}$.
\end{esempio}
\begin{esempio}
  Calcola la derivata di $f(x)=e^{x^2}$.\\
  Poniamo $g(x)=x^2\srarrow f(x)=e^{g(x)}\srarrow f'(x)=e^{g(x)}\cdot g'(x)=
  e^{x^2}\cdot 2x=2xe^{x^2}$.
\end{esempio}

\subsection{Derivata di $f(x)=\log_a x$}
\label{}
\begin{esempio}
  Calcola la derivata di $f(x)=e^{\ln x}$.\\
  Poniamo $g(x)=\ln x\srarrow f(x)=e^{g(x)}\srarrow f'(x)=e^{g(x)}\cdot 
g'(x)=
  e^{\ln x}$ \dots ???.\\
  Ragioniamo: dalle proprietà
  dei logaritmi si ha: $e^{\ln x} =x$, che è la funzione identica. Quindi
  \begin{enumerate}[noitemsep]
    \item $e^{\ln x} =x$ e anche $\ln e^x= x\ln e=x$, così come 
    $f(f^{-1}(x))=f^{-1}(f(x))=x$: le due funzioni sono una inversa 
dell'altra,
    il logaritmo naturale $g(x)=\ln x$ è la funzione inversa della
    funzione esponenziale $f(x)=e^x$;
    \item $\mathit{D}\quadra{f^{-1}(x)}=\dfrac{1}{f'(f^{-1}(x))}$;
    \item $\mathit{D}\quadra{\ln x}=\dfrac{1}{\mathit{D}\quadra{e^{g(x)}}}
    =\dfrac{1}{e^{\ln x}}=\dfrac{1}{x}$.
  \end{enumerate}
\end{esempio}

\begin{inaccessibleblock}
  [esponenziale e logaritmo]
  \begin{minipage}[]{.55\textwidth}
   \begin{center} \esplog \end{center}
 \end{minipage} 
  \hfill
 \begin{minipage}[]{.42\textwidth}
 \begin{teorema}
  La derivata della funzione \\
  logaritmo naturale è: $\mathit{D}\quadra{\ln x}=
  \dfrac{1}{x}$.
\end{teorema}
\noindent Ipotesi: $f(x)=\ln x$. \hspace{1cm}b $f'(x)=\dfrac{1}{x}$
\begin{proof}
La dimostrazione è nei ragionamenti dell'esempio precedente, ai quali 
bisogna 
aggiungere le precauzioni perché le due funzioni siano invertibili e 
derivabili: 
poiché $\ln x$ esiste per $x>0$, i ragionamenti valgono solo per $x>0$
\end{proof} 
 \end{minipage}
\end{inaccessibleblock}
\label{}
\\

Vediamo ora il caso generale, quando la base del logaritmo è genericamente 
$a>0$.
\begin{teorema}
  La derivata della funzione logaritmo in base $a$ è: 
  $\mathit{D}\quadra{\log_a x}= \dfrac{1}{x\ln a}$.
\end{teorema}
\noindent Ipotesi: $f(x)=\log_a x$. \tab $f'(x)=\dfrac{1}{x\ln a}$
\begin{proof}
Si ottiene direttamente dalla formula del cambiamento di base:\\
$\log_a x=\dfrac{1}{\ln a}\ln x$.
\end{proof}

\begin{esempio}
    Derivare la funzione $f(x)=Log(x^2+1)^2$.\\
    $g(x)=(x^2+1)^2\srarrow f(x)=Log (g(x))\srarrow f'(x)=\dfrac{1}{\ln 10}
    \dfrac{1}{g(x)}g'(x)=\\
    =\dfrac{1}{(\ln 10)(x^2+1)^2}2(x^2+1)2x=\dfrac{4x}{(\ln 10)(x^2+1)}$.\\
    Nota che $g(x)=(x^2+1)^2$ è a sua volta una funzione composta del tipo 
    $g(x)=\quadra{h(x)}^2$ e quindi è stata applicata la regola della 
derivata
    di più funzioni composte.
\end{esempio}

Abbiamo ora tutti gli strumenti per convalidare l'osservazione al teorema 
\ref{diff01_teoderpotenza}, a proposito delle funzioni potenza.

\begin{teorema}
  La derivata della funzione potenza $f(x)=x^\alpha$ è: \hspace{5mm}
  $\mathit{D}\quadra{x^\alpha}=(\alpha-1)x^\alpha$, $\forall\alpha$.
\end{teorema}
\noindent Ipotesi: $f(x)=x^\alpha$. \tab $f'(x)=(\alpha-1)x^\alpha$, 
$\forall \alpha$.
\begin{proof}
Combinando alcune delle regole precedenti, si ha:\\
$f(x)=x^\alpha=e^{\ln x^\alpha}=e^{\alpha\ln x}$\\
$f'(x)=e^{\alpha\ln x}\alpha\dfrac{1}{x}=x^\alpha\frac{\alpha}{x}=
\alpha x^{\alpha-1}$.\\
Poiché non è stata fatta nessuna particolare ipotesi sull'esponente (intero 
o 
razionale positivo o negativo, irrazionale ...), allora vale per qualsiasi
esponente.
\end{proof}

\begin{esempio}
  Derivare $f(x)=x^{\sqrt{2}}$.\\
  $f'(x)=\sqrt{2}x^{\sqrt{2}-1}$.
\end{esempio}

\subsection{Derivata di funzioni circolari}
\label{}
Anche per queste funzioni dobbiamo dapprima definire il differenziale. 
Per una migliore comprensione, ci affidiamo soprattutto al piano cartesiano.
\subsubsection{Derivata di $f(x)=\sen x$}
Nel capitolo sugli Iperreali abbiamo già visto 
(vedi pag. \pageref{subsubsec:insnum_fseno}) che per angoli infinitesimi il 
seno e 
l'angolo sono indistinguibili: $\st\tonda{\frac{\sen 
\epsilon}{\epsilon}}=1$.
Dall'analisi del disegno ricaviamo l'espressione del differenziale
$df(x)=d(\sen x)= \sen (x+dx) -\sen x$.

\begin{inaccessibleblock}
  [differenziale del seno]
  \begin{minipage}[]{.40\textwidth}
   \dseno 
 \end{minipage} 
  \hfill
 \begin{minipage}[]{.56\textwidth}
Nell'ingrandimento al microscopio nonstandard, l'incremento infinitesimo di
arco $\overset{\frown}{AB}$ (che corrisponde all'incremento di angolo da $x$
a $x+dx$) è racchiuso fra due raggi indistinguibili da segmenti paralleli 
nei
punti $A\equiv \punto{x}{\sen x}$ e $B\equiv \punto{x+dx}{\sen(x+dx)}$. 
L'arco, a sua volta, risulta indistinguibile dal segmento rettilineo $AB$.
I segmenti che uniscono $A$ e $B$ con le loro proiezioni sull'asse $X$ sono
verticali e paralleli, perciò $ABC$ è un triangolo rettangolo infinitesimo, 
simile al triangolo $BOC$. La sua altezza $BC$ corrisponde a $d\sen x$. 
 \end{minipage}
\end{inaccessibleblock}
\label{fig_diff01dseno}\\

Risolviamo il triangolo rettangolo $ABC$ rispetto al lato $BC$:\\
$BC=AB\cdot \cos x \srarrow d(\sen x)= dx\cdot cos x$  

\begin{teorema}
  La derivata della funzione $f(x)=\sen x$ è $\mathit{D}\quadra{\sen 
x}=\cos x$.
\end{teorema}
\noindent Ipotesi: $f(x)=\sen x$. \tab $f'(x)=\cos x$.
\begin{proof}
 Il commento al disegno giustifica la tesi. 
\end{proof}


\begin{osservazione}
  Si potrebbe criticare il metodo per la dimostrazione: chi assicura che 
negli
  altri quadranti le relazioni fra le variabili non cambino? Saremo troppo
  legati al disegno?\\
  Ci sono altri modi per dimostrare la tesi, più vincolati al calcolo
  e meno al disegno.  Per esempio, dalle formule di addizione abbiamo:
  $\sen(x+dx)=\sen x \cos dx + \sen dx \cos x$. Allora:\\
  $\dfrac{\sen(x+dx)-\sen x}{dx}=\dfrac{\sen x \cos dx + \sen dx \cos x -
   \sen x}{dx}=\\
  =\sen x \dfrac{\cos x-1}{dx}+ \cos x\dfrac{\sen dx}{dx}=
  \sen x\cdot 0 + \cos x\cdot 1=\cos x$,\\
  in cui si fa uso delle forme indeterminate discusse a pag. 
  \pageref{subsubsec:insnum_fseno}. Alla fine basta applicare la funzione 
  $\pst{}$.
\end{osservazione}

\begin{osservazione}
Anche il grafico dell'andamento delle tangenti conferma la tesi
in modo assai espressivo. \\
\begin{inaccessibleblock}
  [differenziale del seno]
  \begin{minipage}[]{.47\textwidth}
    \begin{center} \seno \end{center}
 \end{minipage} 
  \hfill
 \begin{minipage}[]{.47\textwidth}
 \begin{center} \tangentiseno \end{center}
 \end{minipage}
\end{inaccessibleblock}
\label{}
\end{osservazione}

\begin{esempio}
Quale pendenza ha il grafico di $y=\sen x$ nell'origine?\\
$f(x)= \sen x \srarrow f'(x)=\cos x\srarrow f'(0)=\cos 0=1$.\\
La tangente al grafico nell'origine è la retta $y=x$.
\end{esempio}

\begin{esempio}
Derivare $f(x)=\sen^2 x$ e $g(x)= \sen x^2$.\\
$f'(x)=2\sen x \cos x$ e $g'(x)=cos x^2\cdot 2x= 2x \cos x^2$.
\end{esempio}

\begin{esempio}
Derivare $f(x)=\sen^2 x$ e $g(x)= \sen 2x$.\\
$f'(x)=2\sen x \cos x$ e $g'(x)=cos 2x\cdot 2= 2 \cos 2x$.
\end{esempio}

\begin{esempio}
  Derivare $f(x)=x^{\sen x}$.\\
  Si tratta di una funzione di tipo nuovo, un misto fra una funzione potenza
  e una funzione esponenziale. Si risolve con una trasformazione che abbiamo
  già visto e con l'uso delle regole della funzione composta e del 
prodotto.\\
  $x^{\sen x}=e^{{(\ln x)}^{\sen x}}=e^{\sen x\ln x}$.\\
  $f'(x)=e^{\sen x\ln x}(\cos x \ln x +\dfrac{\sen x}{x})=
  x^{\sen x}(\cos x \ln x +\dfrac{\sen x}{x})$.
\end{esempio}

 \subsubsection{Derivata di $f(x)=\cos x$}
\begin{teorema}
  La derivata della funzione $f(x)=\cos x$ è $\mathit{D}\quadra{\cos x}=
  -\sin x$.
\end{teorema}
\noindent Ipotesi: $f(x)=\cos x$. \tab $f'(x)=-\sen x$.
\begin{proof}
  Il disegno con cui dimostrare la tesi è uguale a quello di pag.
  \pageref{fig_diff01dseno}. Lo puoi riprodurre, tenendo però l'attenzione
  concentrata sul segmento $AC$.\\
L'unica osservazione importante è che nel passare da $x$ a $ x+dx$, cioè 
mentre
l'angolo cresce, il valore del coseno decresce. Infatti, al contrario di 
quanto
avviene per il seno, nel primo quadrante si ha: $ \cos(x+dx)<\cos x$. 
Questa è 
la ragione del segno meno nel risultato.
\end{proof}

\begin{inaccessibleblock}
  [differenziale del coseno]
  \begin{minipage}[]{.47\textwidth}
    \begin{center} \coseno \end{center}
 \end{minipage} 
  \hfill
 \begin{minipage}[]{.47\textwidth}
 \begin{center} \tangenticoseno \end{center}
 \end{minipage}
\end{inaccessibleblock}
\label{}

\begin{esempio}
Quale pendenza ha il grafico di $y=\cos x$ nell'origine?\\
$f(x)= \cos x \srarrow f'(x)=-\sen x\srarrow f'(0)=-\sen 0=0$.\\
La tangente al grafico nell'origine è orizzontale.
\end{esempio} 

\begin{esempio}
Derivare $f(x)=\cos^2 x$ e $g(x)= \cos x^2$.\\
$f'(x)=2\cos x(-\sen x)=-2\sen x\cos x$ e $g'(x)=-\sen x^2\cdot 2x=-2x\sen 
x^2$.
\end{esempio}

\begin{esempio}
Derivare $f(x)=\cos^2 x +\sen^2 x$.\\
$f'(x)=-2\sen x\cos x + 2\sen x \cos x = 0$.
\end{esempio}


\subsubsection{Derivata di $f(x)=\tg x$}
La funzione $f(x)=\tg x$ è discontinua per $x= \pm\frac{\pi}{2}$. La 
derivata
quindi non può esistere nei punti corrispondenti, come dimostra il grafico.

\begin{inaccessibleblock}
  [differenziale della tangente]
  \begin{minipage}[]{.47\textwidth}
    \begin{center} \tangente \end{center}
 \end{minipage} 
  \hfill
 \begin{minipage}[]{.47\textwidth}
 \begin{center} \tangentitangente \end{center}
 \end{minipage}
\end{inaccessibleblock}
\label{}

\begin{teorema}
   La derivata della funzione $f(x)=\tg x$ è $\mathit{D}\quadra{\tg x}=
   \dfrac{1}{\cos^2 x}=1+tg^2 x$ per $x\ne \pm\frac{\pi}{2}$.
\end{teorema}
\noindent Ipotesi: $f(x)=\tg x$. \tab $f'(x)=\dfrac{1}{\cos^2 x}=1+tg^2 x$,
per $x\ne\pm\frac{\pi}{2}$.
\begin{proof}
Per calcolare la derivata nei punti in cui la funzione è continua, 
ricorriamo
alla seconda relazione fondamentale: $\tg x=\frac{\sin x}{\cos x}$ e
sfruttiamo la regola della derivata di un quoziente (pag. 
\pageref{sec:diff01_regolederivate}).\\
$\mathit{D}\quadra{\tg x}=\mathit{D}\quadra{\dfrac{\sin x}{\cos x}}=
\dfrac{\mathit{D}\quadra{\sen x}\cdot \cos x-\sen x\cdot 
  \mathit{D}\quadra{\cos x}} {\cos^2 x}=
\dfrac{\sen^2 x +\cos^2 x}{\cos^2 x}=\dfrac{1}{\cos^2 x}=\tg^2+1$
\end{proof}

\begin {esempio}
Quale è la pendenza del grafico di $y=\tg x$, per $x=\dfrac{\pi}{4}$? E per
$x=\dfrac{\pi}{2}$?\\
$f'(x)=1+tg^2 x\srarrow f'(\dfrac{\pi}{4})=1+\tg^2\dfrac{\pi}{4}=2$\\
$f'(x)=1+tg^2 x\srarrow f'(\dfrac{\pi}{2})=1+\tg^2\dfrac{\pi}{2}=$ ???\\
Per $x\approx\dfrac{\pi}{2}$ il grafico della funzione cresce verticalmente,
la sua pendenza è un numero infinito e la parte standard di un infinito non
esiste. D'altra parte, se $x$ è esattamente uguale a $\dfrac{\pi}{2}$ ,
la tangente ha un punto di discontinuità.
\end {esempio}


\section{Applicazioni}
\label{sec:diff01_applicazioni}
Si è tanto parlato delle tangenti ai grafici di funzione e delle loro 
pendenze,
senza mai arrivare a definire l'effettiva equazione delle tangenti che
interessano. Ora cercheremo di colmare questa lacuna.

\subsection{Derivata e tangente}
 Hai già incontrato negli anni scorsi dei problemi in cui si chiedeva di 
 calcolare la tangente ad una parabola in un suo punto. Il metodo di 
calcolo 
 algebrico che usavi è efficace ma macchinoso e, sfortunatamente, vale solo 
 per le coniche. Il metodo delle derivate, invece, si rivela
 molto più potente e rapido.\\
 Poiché la tangente è una retta, la sua equazione è del tipo 
$y-y_0=m(x-x_0)$, 
 dove  $\punto{x_0}{y_0}$ è il punto di tangenza 
 e $m$ è la pendenza della retta, sulla quale sappiamo ormai tutto.
 Si ha $y=m(x-x_0)+y_0$ e poiché $m=f'(x_0)$, relativo alla funzione $f(x)$
 di cui si sta studiando il grafico, l'equazione risolvente è:\\
 $y=f'(x_0)(x-x_0)+y_0$.
 
\begin{esempio}
  Trova le equazioni delle tangenti alla parabola $f(x)=x^2$ nei suoi punti
  $V~\equiv~\punto{0}{f(0)}=0$ e $B\equiv\punto{-6}{f(-6)}$.\\
  Soluzione. Nel punto $V$: $f'(0)~=~2\cdot 0~=~0=m$. La tangente è 
  orizzontale e coincide con l'asse $X$: $y=~m(x~-~x_0)~+~y_0~=~0$.\\
  Nel punto $B$: $f'(-6)=2\cdot (-6)=-12$. $m=-12$, la tangente è inclinata
  verso il basso:
  $y=m(x-x_0)+y_0= -12(x-6)+36\srarrow y=-12x+108$.
 \end{esempio}
 
\begin{esempio}
  Trova i punti di intersezione degli assi con la tangente in 
$\punto{2}{f(2)}$
  alla curva $f(x)=2x^3-x$.\\ Soluzione.
  Ricerca della tangente per $x=2$: $f'(x)=6x^2-1$ e $f'(2)=6\cdot 
4-1=23$.\\
  $y_0=f(2)=2\cdot 2^3-2=14$. La tangente: $y=23(x-2)+14=23x-32$.
  Le intersezioni: \\
  Con l'asse $X$: $y=0\srarrow x=\dfrac{32}{23}\srarrow
  \punto{\dfrac{32}{23}}{0}$.\\
  Con l'asse $Y$: $x=0 \srarrow y=-32\srarrow \punto{0}{-32}$.
\end{esempio}

\begin{esempio}
  In quale punto del suo grafico la parabola $y=4x^2-3x+6$ è inclinata di 
  $45^\circ$?\\
  Soluzione. Nel punto che cerchiamo, la parabola avrà un'inclinazione 
  indistinguibile da quella della tangente.
  Le rette inclinate di $45^\circ$ hanno pendenza $m=1$, come la bisettrice 
del
  primo-terzo quadrante.  Dobbiamo quindi imporre alla derivata il valore 
$1$.\\
  $f(x)=4x^2-3x+6\srarrow f'(x)=8x-3$.\\
  $8x-3=1\srarrow x=\dfrac{1}{2}$. Il punto è 
$\punto{\dfrac{1}{2}}{\dfrac{11}{2}}$.
\end{esempio}

\begin{esempio}
  È vero che l'iperbole equilatera di equazione $xy=16$ ha per vertici i 
punti 
  medi del segmento che gli assi staccano sulle tangenti ai vertici? \\
  Risposta. Consideriamo per comodità solo il ramo destro del grafico.
  Il vertice sarà un punto $V$ di coordinate uguali, essendo l'iperbole
  equilatera. Quindi $V=V\punto{4}{4}$.\\
  Poiché la funzione è $y=\dfrac{16}{x}$, la sua derivata in $V$ è 
  $y'|_{x=4}= -\dfrac{16}{x^2}\bigg|_{x=4}=-1$
  e l'equazione della tangente in $V$ è $y=-1(x-4)+4=-x+8$.\\
  La retta $y=-x+8$ interseca gli assi in $\punto{8}{0}$ e $\punto{0}{8}$ 
ed 
  è facile verificare che il punto $V$ è medio fra i due.
  Per ragioni di simmetria accade lo stesso
  con il vertice opposto $\punto{-4}{-4}$.
  \begin{osservazione}
   In realtà si tratta di una proprietà generale dell'iperbole equilatera.
   Qualsiasi retta tangente al grafico stacca sugli assi coordinati dei 
   segmenti che hanno il punto medio coincidente con il punto di tangenza. 
   Non è difficile dimostrarlo usando l'equazione generica $yx=k^2$ e per 
   punto di tangenza le coordinate $\punto{a}{\dfrac{k^2}{a}}$.
  \end{osservazione}
 \end{esempio}

\begin{esempio}
  È vero che è inclinato di $30^\circ$ il raggio  
  della circonferenza $x^2+y^2=20$ che unisce l'origine al suo punto di 
ascissa
  $4$?\\
  Risposta. No, non è vero. \\
  Il modo più elementare per verificarlo è calcolare l'ordinata
  del punto e cercare l'angolo di inclinazione dell'ipotenusa coincidente 
con 
  il raggio.\\
  L'alternativa è calcolare la derivata:
  $x^2+y^2=20\srarrow y=\sqrt{20-x^2}$ (data lo posizione del punto,
  consideriamo solo la semicirconferenza per $y>0$).\\
  $f'(4)= \dfrac{-2x}{2\sqrt{20-x^2}}\bigg|_{x=4}=\dfrac{-4}{\sqrt{20-16}}=
  \dfrac{-4}{2}=-2$.\\
  Dunque la tangente ha una pendenza pari a $-2$. Poiché il raggio e la 
  tangente sono perpendicolari, la retta che contiene questo raggio avrà 
  pendenza $-\dfrac{1}{-2}=\dfrac{1}{2}$.\\
  Possiamo controllare la risposta con la calcolatrice.
  \end{esempio}
 
\subsection{Derivata e normale}
\label{}
Come si vede dall'ultimo esempio, una volta che si sappia come calcolare 
la tangente ad una curva, il calcolo della normale risulta molto facile.
Poiché la tangente e la normale, se passano per lo stesso punto, sono rette
perpendicolari e quindi hanno i coefficienti angolari antireciproci,
l'equazione di una normale ad una curva $y=f(x)$ in un punto 
$\punto{x_0}{y_0}$
sarà:\\
$y=\dfrac{-1}{f'(x_0)}(x-x_0)+y_0$,\\
dove la pendenza della normale $m_n=\frac{-1}{m_t}$ è appunto 
l'antireciproco
della pendenza della tangente.

\begin{esempio}
Scrivi l'equazione della tangente e della normale alla curva di equazione
$y=\dfrac{x^2-1}{\ln x -1}$ nel suo punto di ascissa $1$.\\
Soluzione. $y'|_{x=1}=\dfrac{2x(\ln x-1)-(x^2-1)\dfrac{1}{x}}
{(\ln x -1)^2}\bigg|_{x=1}=\dfrac{2\cdot 1(0-1)-(1-1)\cdot 
1}{(0-1)^2}=-2$.\\
La pendenza della tangente è $m=-2$. Per $x=1$ la funzione vale:
$\dfrac{1^2-1}{\ln 1 -1}=0=y_0$. L'equazione della tangente è quindi:
$y=-2(x-1)=-2x+2x$. Di conseguenza la normale ha equazione 
$y=\dfrac{1}{2}(x-1)
=\dfrac{1}{2}x-\dfrac{1}{2}$.
\end{esempio}


\begin{esempio}
Scrivi l'equazione della tangente e della normale alla curva di equazione
$y=\dfrac{x^2+1}{\ln x +1}$ nel suo punto di ascissa $1$.\\
Soluzione. $y'|_{x=1}=\dfrac{2x(\ln x+1)-(x^2+1)\dfrac{1}{x}}
{(\ln x +1)^2}\bigg|_{x=1}=\dfrac{2\cdot 1(0+1)-(1+1)\cdot 1}{(0+1)^2}=0$.\\
La tangente è quindi una retta orizzontale. Di conseguenza la normale è 
verticale, come si vede subito se si prova a calcolare l'antireciproco di 
$0$.
\end{esempio}

\subsection{Derivata della derivata}
\label{}
Abbiamo già notato che la derivata  di una funzione dipende dal punto in 
cui si 
calcola e che, una volta stabilito questo punto, ha un unico risultato,
se esiste.
Quindi la derivata di una funzione è a sua volta una funzione e,
se ci sono le condizioni, può essere derivata a sua volta.
\begin{definizione}
 Se una funzione $f(x)$ è derivabile, la sua derivata è la funzione $f'(x)$.
 Se anche $f'(x)$ è derivabile, allora esiste la funzione $f''(x)$ ed è
 chiamata \emph{derivata seconda di $f(x)$}.
\end{definizione}
Le regole di calcolo della derivata seconda sono le stesse regole che 
abbiamo 
già visto, quindi la seconda derivazione, se è possibile, non comporta 
problemi 
diversi da quelli conosciuti.\\
Riferendoci a un generico grafico di funzione $y=f(x)$, la derivata prima 
$f'(x)$ ci consente di trovare le pendenze delle tangenti al grafico. La
derivata seconda $f''(x)$ descrive con quanta rapidità (o lentezza) variano 
queste pendenze, perciò ci indica quanto siano aperte o chiuse le concavità 
che
$y=f(x)$ disegna nel piano cartesiano.\\
Se le condizioni sono favorevoli, esistono e sono calcolabili anche le 
derivate 
terze, quarte, ecc. di una funzione, anche se non sono essenziali per i 
nostri 
scopi. Il loro calcolo segue i metodi già visti.

\begin{esempio}
Calcola $f''(1)$ di $f(x)=2x^5-3x^4+x^3+5x^2-6x+9$.\\
Derivata prima: $f'(x)=10 x^4-12 x^3+3x^2+10 x-6$\\
Derivata seconda per $x=1$: $(40x^3-36x^2+6x+10)|_{x=1}=40-36+6+10=20$
\end{esempio}

\begin{esempio}
Calcola $f''(x)$ di $f(x)=\ln x$.\\
$f'(x)=\dfrac{1}{x}$ e $f''(x)=-\dfrac{1}{x^2}$.\\
\begin{osservazione}
 La funzione $\ln x $ esiste per $x>0$. Le derivate prima e seconda esistono
 per $x\ne 0$.
 In generale, l'esistenza di una derivata (prima, seconda, terza. \dots) è
 indipendente dall'esistenza della funzione da derivare (funzione 
primitiva).
\end{osservazione}
\end{esempio}

\begin{esempio}
 Calcola le derivate successive di $f(x)=\sen x$.\\
 $f'(x)=\cos x$ \hspace{1cm}$ f''(x)=-\sen x$ \hspace{1cm}
 $f'''(x)=-\cos x$ \hspace{1cm} $f^{IV}(x)=\sen x$ \dots
\end{esempio}


\subsection{Derivata, differenza e differenziale}
\label{subsec:diff01_deridiff}

\begin{inaccessibleblock}
  [differenziale della tangente]
  \begin{minipage}[]{.47\textwidth}
    \begin{center} \derivata \end{center}
 \end{minipage} 
  \hfill
 \begin{minipage}[]{.47\textwidth} \vspace{2.5em}
Nel punto $\punto{x_0}{f(x_0)}$ il grafico della funzione e la tangente
sono indistinguibili. 
Il campo visivo del primo microscopio mostra ${x_0}$ e $dx$, 
uno fra gli infiniti infinitesimi nella monade di $x_0$. A livello 
microscopico la curvatura del grafico non esiste,  per cui il grafico e la 
tangente sono sovrapposti. Per cogliere la distinzione fra i due occorre 
un secondo microscopio non standard, centrato a distanza infinitesima dal 
punto. Nel suo campo visivo la tangente e il grafico della funzione 
appaiono come rette parallele. Nella rappresentazione doppiamente ingrandita
il punto di coordinate reali più vicino a quello raffigurato  si trova a 
distanza infinita ($\infty^2$).
 \end{minipage}
\end{inaccessibleblock}
\label{}
La figura mostra che la tangente e la secante per due punti infinitamente 
vicini sono distinguibili solo al dettaglio degli infinitesimi. Lo 
stesso avviene per la derivata e il rapporto differenziale.\\
Dalla definizione di derivata $f'(x)=\pst{\dfrac{df(x)}{dx}}$ ricaviamo che
$f'(x)\sim \dfrac{df(x)}{dx}$: la derivata e il rapporto differenziale 
sono quantità quasi, ma non esattamente, uguali. Possiamo esprimere 
meglio questo concetto:\\
$\dfrac{df(x)}{dx}=f'(x)+\epsilon(x)$ e quindi 
$df(x)=f'(x)dx+\epsilon(x)dx$.\\
$\epsilon(x)$ è l'infinitesimo, o l'insieme di infinitesimi, che fa la 
differenza fra la derivata e il rapporto differenziale.  
$\epsilon(x)dx$, un prodotto fra infinitesimi, forma un infinitesimo 
di ordine superiore rispetto a $f'(x)dx$.
Nella maggior parte dei casi pratici si tratta di una differenza 
trascurabile 
e si può accettare l'espressione $f'(x)dx$ al posto dell'espressione
$df(x)$, che può essere meno comoda da calcolare.\\
Nella storia del calcolo infinitesimale l'uso di una formula al posto 
dell'altra
è diventato normale e molti testi definiscono differenziale della 
funzione il prodotto $f'(x)dx$, invece della differenza infinitesimale 
$df(x)$.

Il problema diventa più critico nelle applicazioni pratiche, quando si 
devono 
usare le differenze finite al posto dei differenziali. Si usa allora, per
analogia,:\\
$\Delta f(x)=f'(x_0)\Delta x + \delta(x)\Delta x$.\\
Dato che l'ultimo termine è il meno rilevante, si ha:\\
$\Delta f(x)\cong f'(x_0)\Delta x \srarrow
f(x)-f(x_0)\cong f'(x_0)(x-x_0)\srarrow f(x)\cong f'(x_0)(x-x_0)+ f(x_0)$.\\
Si tratta dell'usuale equazione della tangente per $x=x_0$.\\
La formula è esatta solo per le funzioni rappresentate da rette.
Per le altre funzioni la differenza $\Delta f(x)$ fra due valori della 
funzione 
può essere anche molto diversa da $f'(x_0)\Delta x$, che è in realtà la 
differenza fra due valori $y$, calcolati lungo la tangente.

\begin{inaccessibleblock}
  [differenziale della tangente]
  \begin{minipage}[]{.4\textwidth}
    \begin{center} \falsodifferenziale \end{center}
 \end{minipage} 
  \hfill
 \begin{minipage}[]{.55\textwidth}
 Allontanandosi da $x_0$ di una quantità finita $\Delta x$, le differenze 
della
 funzione $\Delta f(x)$, calcolate a partire da $x_0$, possono 
 essere anche molto diverse dalle differenze $f'(x_0)\Delta x$, calcolate 
lungo
 la tangente.
 \end{minipage}
\end{inaccessibleblock}
\label{}

Nei testi in cui si scrive che $\Delta f(x)=f'(x_0)\Delta x+\delta(x)\Delta 
x$,
$f'(x_0)\Delta x$ viene chiamato differenziale, anche se si tratta di una 
differenza,
una quantità .finita, non infinitesima. In tali testi la differenza $\Delta 
f(x)$ è detta 
incremento e l'equazione\\
$\Delta f(x)\cong f'(x_0)\Delta x$ \hspace{.5cm}(\emph{Equazione alle 
differenze})\\ 
esprime il cosiddetto \emph{teorema dell'incremento}.\\
Ai fini pratici l'Equazione alle differenze è un'equazione utile, 
soprattutto 
quando si studiano i fenomeni naturali, perché le variazioni che si 
misurano in
questi ambiti sono differenze finite. 
Ovviamente i risultati che si ottengono utilizzando il teorema 
dell'incremento
saranno tanto più precisi quanto più piccola è la variazione $\Delta x$, in 
rapporto ai valori $x$.

\begin {esempio}
Fare una stima ragionevole della quantità $\sqrt{25,162}$.\\
Si sta usando la funzione $f(x)=\sqrt{x}$, la cui derivata è: 
$f'(x)=\dfrac{1}{2\sqrt{x}}$.\\
Utilizziamo il teorema dell'incremento, fissando $x_0=25$ e $\Delta 
x=0,162$.\\
$\Delta f(x)=f'(x_0)\Delta x+\delta(x)\Delta x\cong f'(x_0)\Delta x
\srarrow f(x)\cong f'(x_0)\Delta x+ f(x_0)$\\
$f(25,162)\cong f'(25)\cdot 0,162+f(25)\srarrow 
\sqrt{25,162}\cong \dfrac{1}{2\sqrt{25}}\cdot 0,162+ \sqrt{25}=
\dfrac{0,162}{10}+5\cong 5,0162$.\\
Confronta il risultato con quanto propone la calcolatrice.\\
Poiché l'approssimazione è tanto migliore quanto più piccolo è $\Delta x$,
ripeti l'esercizio con $x_0=25,1001$ (la cui radice è $5,01$) e quindi
$\Delta x=0,06199$.
\end {esempio}

\subsection{Sintesi}
\label{subsec:diff01_derisintesi}

{%
\newcommand{\mc}[3]{\multicolumn{#1}{#2}{#3}}
\begin{center}
\begin{tabular}{|c|c||c|c|} 
\mc{4}{c}{Derivate notevoli}\\ \hline
$f(x)$ & $f'(x)$ &   $f(x$) & $f'(x)$\\ \hline\hline
&&&\\[-.5em]
$k$  & 0 &   $\tg x$ & $\dfrac{1}{cos^2 x}=\tg^2 x +1$\\
&&&\\[-.5em]
$x$ & 1 &   $\log_a x$ & $\dfrac{1}{x\ln a}$\\
&&&\\[-.5em]
$x^\alpha$ & $\alpha x^{\alpha-1}$ &   $\ln x$ & $\dfrac{1}{x}$\\
&&&\\[-.5em]
$\sen x$ & $\cos x$ &   $a^x$ & $a^x\ln a$\\
&&&\\[-.5em]
$\cos x$ & $-\sen x$ &   $e^x$ & $e^x$\\[.5em] \hline
\end{tabular}
\end{center}
}%
\label{tab:diff01_derivatefondamentali} 
 
\begin{center}
\begin{tabular}{l}
\vspace{.5em}  Regole di derivazione\\\vspace{.5em}
$\mathit{D}\quadra{f(x)+g(x)}=f'(x)+g'(x)$\\\vspace{.5em}
$\mathit{D}\quadra{kf(x)}=kf'(x)$\\\vspace{.5em}
$\mathit{D}\quadra{f(x)\cdot g(x)}=f'(x)\cdot g(x)+f(x)\cdot 
g'(x)$\\\vspace{.5em}
$\mathit{D}\quadra{\dfrac{1}{f(x)}}=-\dfrac{f'(x)}{f^2(x)}$\\\vspace{.5em}
$\mathit{D}\quadra{\dfrac{f(x)}{g(x)}}=\dfrac{f'(x)\cdot g(x)-f(x)
    \cdot g'(x)}{g^2(x)}$\\\vspace{.5em}
$\mathit{D}\quadra{f(g(x))=f'(g(x))\cdot g'(x)}$
\end{tabular}
\end{center}
\label{tab:diff01_regolederivazione}

\subsection{Applicazioni non solo matematiche}
Il calcolo della derivata è entrato da protagonista nella descrizione 
matematica dei fenomeni naturali da almeno $300$ anni e più recentemente
anche nello studio delle scienze umane e sociali. \\
Gli esempi che seguono si servono di questo calcolo in due modi:
\begin{enumerate}[noitemsep]
 \item per trovare il tasso di variazione:
data una funzione, si deve cercare quanto rapidamente essa varia rispetto
alla sua variabile;
 \item attraverso l'equazione alle differenze, 
nella forma diretta $\Delta f(x)\cong f'(x_0)\Delta x$, o nella forma
inversa $\Delta x \cong \dfrac{\Delta f(x)}{f'(x_0)}$.
\end{enumerate}
L' utilità dell'equazione alle differenze viene dal
fatto che si tratta di un'equazione di primo grado in $\Delta x$, perché i 
termini infinitesimi di grado superiore sono trascurati. Le soluzioni che 
così
si ottengono sono approssimate, ma in genere il grado di imprecisione è 
sopportabile.

\begin{esempio}
 Se lanci verso l'alto una palla alla velocità iniziale $v=20$ m/s, questa 
viene 
 frenata dalla forza di gravità e la sua legge del moto risulta all'incirca 
 $h(t)=20t-5t^2$. Trova a quale altezza $h$ la palla si ferma.\\
 Risposta. Se la palla si ferma, la sua velocità è nulla, quindi:\\
 $v(t)=\dfrac{dh(t)}{dt}= 20-10t=0\srarrow t=2 s \srarrow h(2)=20\cdot 
2-5\cdot 
 2^2=20 m$.\\
 Dopo quanto tempo dal lancio la palla si trova a metà altezza?\\
 Risposta. $\Delta t\cong \dfrac{\Delta h(t)}{v(0)}=\dfrac{10}{20}=0,5$ s.
 Considera l'imprecisione dell'ultima risposta, visto che puoi avere il
 il risultato esatto direttamente dall'equazione del moto, con $h(t)=10$.
\end{esempio}

\begin{esempio}
L'aereo A parte da Milano a mezzogiorno e vola in direzione Ovest 
mediamente a 
$800$ km/h, mentre l'aereo B parte due ore dopo e si dirige a Sud a $800$ 
km/h.
Se volano alla stessa quota, con quale velocità si allontanano l'uno 
dall'altro
dopo 4 ore?\\
Soluzione. Le due equazioni del moto sono $s_A=800t$ e $s_B=800(t-2)$. 
Calcoliamo prima la distanza fra i due, poi la loro velocità relativa.
Si tratta di direzioni perpendicolari e possiamo applicare il teorema di 
Pitagora.\\
$s_{AB}=\sqrt{s_A^2+s_B^2}=\sqrt{(800t)^2+[(800(t-2)]^2}=
800\sqrt{t^2+t^2-2t+4}=800\sqrt{2t^2-2t+4}$.\\
$v_{AB}|_{t=4}=\dfrac{ds_{AB}}{dt}=\dfrac{800(2t-4)}{2\sqrt{2t^2-2t+4}}
\bigg|_{t=4}\cong 358$ km/h.
\end{esempio}

\begin{esempio}
 Un circuito è percorso da corrente variabile. Infatti la carica che 
attraversa
il conduttore ad un certo istante $t$ è data da $q(t)=t^3-24t$. È possibile 
che in
qualche istante le cariche siano ferme?\\
Riposta. Se le cariche sono ferme, la corrente è nulla.\\ 
$i(t)=\dfrac{dq(t)}{dt}=3t^2-24=0\srarrow t=\pm\sqrt{8}=\pm2\sqrt{2}$ s.\\
\end{esempio}

\begin{esempio}
Il biologo Jacques Monod mostrò che lo sviluppo di una colonia di batteri 
di 
 Escherichia Coli segue una crescita esponenziale, se sufficientemente 
nutrita. 
 Ogni microrganismo si scinde in due dopo circa $20$ minuti, per cui la
 popolazione al tempo $t$, misurato in ore, conta $N(t)=N_0e^{\frac{t}{3}}$
 individui. Dopo quante ore il numero di batteri passa da $10^6$ a $10^9$?\\
 Soluzione: $\Delta N=N'(t)\Delta t=\dfrac{N_0}{3}e^{\frac{t}{3}}\Delta t
 \srarrow \Delta t=\dfrac{3\Delta N}{N_0e^{\frac{t}{3}}}=\dfrac{3\Delta N}
 {N(t)}$.\\
 Il numero iniziale di batteri è $10^6=N(0)=N_0e^0$. Perciò:
 $\Delta t=\dfrac{3(10^9-10^6)}{10^9}$, che, calcolato in ore, 
 corrisponde a 3 ore meno $11$ secondi circa.
 \begin{osservazione}
 Si tratta di un problema tipico sulla crescita esponenziale, di quelli già 
 risolti quando ancora non conoscevi l'esistenza delle derivate, riguardanti
 per esempio l'interesse composto o il decadimento radiattivo.
\end{osservazione}

\begin{osservazione}
 Come mai in un caso del genere l'uso delle derivate non è indispensabile? 
Perché
 la funzione esponenziale è l'unica funzione che ha per derivata...
\end{osservazione}

\begin{osservazione}
Dunque, la risposta è che in quasi $3$ ore il numero di batteri passa da un 
milione a un miliardo, che è $1000$ volte tanto. Possiamo pensare che 
occorra
lo stesso tempo per passare da $1$ individuo a $1000$, oppure da $1000$ 
individui a $1$ milione?
\end{osservazione}
\end{esempio}
 
\begin{esempio}
Il costo marginale è l'aumento di costo che si ha quando si vuole produrre 
un'unità in più di un certo bene.\\
Supponi che per produrre un certo numero $n$ di aghi il costo in euro sia
$y=\sqrt{n}$. Calcola il costo marginale per produrne più di $10.000$.\\
Soluzione: $y=\sqrt{n}\srarrow y'=\dfrac{1}{2\sqrt{n}}$.
Se $n=10000$, $\Delta y=\dfrac{1}{2\sqrt{10000}}\Delta n=\dfrac{\Delta 
n}{200}$.
Il costo marginale, cioè per unità in più, è quindi dello $0,5\%$.
\begin{osservazione}
 Anche in questo caso concreto, non è possibile pensare che $\Delta n$ sia 
un
 infinitesimo, dato che non ha senso calcolare il costo per frazioni 
 infinitesime di un ago.
\end{osservazione}
\end{esempio}
 
\begin{esempio}
 Una barra metallica è lunga $10$ m a $T=0 ^\circ C$ e al crescere della 
 temperatura si dilata secondo la legge $l(T)=10(1+0,000024T)$. Di quanti 
gradi 
 occorre aumentare la temperatura perché aumenti la sua lunghezza di $5$ 
cm?\\
 La risposta è la stessa in ogni caso, oppure dipende dal valore iniziale 
di 
 $T$?\\
 Risposta. $\Delta T\cong \dfrac{\Delta l(T)}{l'(T)}=\dfrac{0,05}
 {10\cdot 0,000024}=208,3 ^\circ C$.\\
 Nella formula risolutiva non compare il simbolo $T$, quindi la risposta 
non 
 dipende dalla temperatura iniziale. Ovviamente tutto questo deve avvenire 
nei 
 limiti del fenomeno, cioè finché non si raggiunge la temperatura di 
fusione.
 \begin{osservazione}
  Si tratta di un semplice esercizio di fisica: la legge coinvolta si chiama
  legge della dilatazione lineare, perché il suo grafico nel piano 
cartesiano 
  è una retta. Poiché la legge è espressa da un polinomio di primo grado, 
la 
  soluzione non contiene la variabile $T$ e l'equazione alle differenze è
  esatta: non ci sono infinitesimi da trascurare.\\
  Non è indispensabile coinvolgere il calcolo infintesimale per un problema 
di
  primo grado come questo: avresti potuto risolverlo anche in terza media.
 \end{osservazione}

\end{esempio}
