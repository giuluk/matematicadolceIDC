% (c) 2015 Daniele Zambelli daniele.zambelli@gmail.com

%%%%%%%%%%%%%%%%%%%%%%%%%%%%%%%%%%%%%%%% ATTENZIONE %%%%%%%%%%%%%%%%%%%%%%
% Da riportare nelle definizioni una volta definita definitivamente l'edizione 
% 2016

\renewcommand{\sistema}[1]{\begin{cases}#1\end{cases}}

% (c) 2014 Daniele Zambelli - daniele.zambelli@gmail.com
% 
% Tutti i grafici per il capitolo relativo alle parabole
%
% 

\newcommand{\espdueterzi}{% 
    % Esponenziali con basi diverse.
    \disegno{
    \rcom{-10}{+10}{-1}{10}{gray!50, very thin, step=1}
    \begin{scope}[ultra thick, color=Maroon!50!black]
     \tkzInit[xmin=-10.3, xmax=+10.3, ymin=-0.3, ymax=+10.3]
     \tkzFct[domain=-10.3:+6]{(3./2)**x}
     \tkzFct[color=Green!50!black, domain=-6:+10.3]{(2./3)**x}
     \begin{scope}[color=Black!50!black]
      \filldraw (1, 3./2) circle (1.2pt);
      \filldraw (1, 2./3) circle (1.2pt);
     \end{scope}
     \filldraw [color=Red](0, 1) circle (1.2pt);  
    \end{scope}
    \begin{scope}[color=black]
     \draw (-7.3, 7) node{\(f(x)=\tonda{\dfrac{2}{3}}^x\)}; 
     \draw ((7.3, 7) node{\(f(x)=\tonda{\dfrac{3}{2}}^x\)};
    \end{scope}
    }
}

\newcommand{\logduebasi}{% 
    % Esponenziali con basi diverse.
    \disegno{
    \rcom{-1}{+10}{-9}{9}{gray!50, very thin, step=1}
    \begin{scope}[ultra thick, color=Maroon!50!black]
      \tkzInit[xmin=-1.3, xmax=+80, xstep=.5, ymin=-10.3,ymax=+10.3]
      \tkzFct[domain=.01:+10]{log(x)/log(2)}
      \filldraw (2, 1) circle (1.2pt);
      \begin{scope}[color=Green!50!black]
        \tkzFct[domain=-.01:+10]{log(x)/log(1./2)}
        \filldraw (2, -1) circle (1.2pt);
      \end{scope}
    \end{scope}
    \begin{scope}[color=black]
      \draw (9.5, 2.8) node{a=2}; 
      \draw (9.5, -2.8) node{a=0.5};
    \end{scope}
      \filldraw [color=Red] (1,0) circle (1.2pt);
    }
}


\chapter{La circonferenza nel piano cartesiano}

\section{Circonferenza con il centro nell'origine}
\label{sec:circ_circcentroorigine}

% \subsection{Circonferenza come luogo geometrico}
% \label{subsec:circ_luogo}
% 
% \subsection{Equazione della circonferenza}
% \label{subsec:circ_equazione}

Esistono numerose definizioni di circonferenza, la curva che da sempre è ritenuta esempio di 
perfezione. Vediamone alcune:

\begin{definizione}%[intrinseca]
 La circonferenza è una linea del piano che ha sempre la stessa curvatura.
\end{definizione}

\begin{definizione}%[non standard]
 La circonferenza è un poligono regolare con infiniti lati.
\end{definizione}

\begin{definizione}%[luogo di punti]
 La circonferenza l'insieme dei punti del piano equidistanti da un punto detto 
centro.
\end{definizione}

A seconda del problema che vogliamo risolvere può essere più comodo utilizzare 
una o un'altra delle definizioni precedenti. In questo capitolo vogliamo 
studiare la circonferenza nel piano cartesiano e useremo l'ultima definizione.


\begin{figure}[h]
\centering
\begin{minipage}[]{.48\textwidth}
% Possiamo ottenere una una situazione particolarmente semplice da descrivere 
% con 
% un'equazione scegliendo come centro della circonferenza, l'origine delle 
% coordinate.

 Se prendiamo come centro della circonferenza l'origine delle coordinate, 
otteniamo una situazione particolarmente semplice da descrivere con 
un'equazione.

In questo caso infatti la relazione del teorema di Pitagora lega i tre lati del 
triangolo: \(x,~y \text{ e } r\):
\[x^2 + y^2 = r^2\]
che è l'equazione della circonferenza perché tutti e solo i punti 
della circonferenza sono soluzioni di questa equazione.
\end{minipage}
\hfill
\begin{minipage}[]{.48\textwidth}
\begin{center}
\begin{inaccessibleblock}[Una circonferenza con centro nell'origine degli assi.]
  \circonfO
  \caption{Circonf. con centro nell'origine.} \label{fig:circonfO}
\end{inaccessibleblock}
\end{center}
\end{minipage}
\end{figure}

\begin{esempio}
Calcola l'equazione della circonferenza con centro nell'origine e passante per 
il punto \(P\punto{4}{6}\).

L'equazione sarà del tipo: \(x^2 + y^2 = r^2\) l'unico parametro da individuare 
è il raggio che è la distanza di un punto qualsiasi della circonferenza dal 
centro. L'esercizio ci dà un punto della circonferenza e quindi possiamo usarlo 
per trovare il raggio:
\[r = \sqrt{x_P^2 + y_P^2} = \sqrt{4^2 + 6^2} = \sqrt{16 + 36} = \sqrt{52}\]
l'equazione della circonferenza è allora:
\[x^2 + y^2 = 52\]
\end{esempio}

\begin{esempio}
Calcola le intersezioni tra la circonferenza  \(x^2 + y^2 = 25\) e la 
retta di equazione \(x=-4\).

\noindent\begin{minipage}{.48\textwidth}
La circonferenza ha centro nell'origine e ha \(r^2 = 25\) quindi \(r=5\). 
Disegniamo quindi la circonferenza con centro nell'origine e raggio~5,
poi disegniamo anche la retta formata da tutti i punti che hanno ascissa~\(-4\).

Le intersezioni si ottengono risolvendo il sistema:

\(\sistema{x=-4 \\ x^2 + y^2 = 25}\)

Con la sostituzione otteniamo l'equazione risolvente: 

\(\tonda{-4}^2 + y^2 = 25 \sRarrow 16 + y^2 = 25 \sRarrow\)

\(y^2 = 9 \sRarrow y= \pm 3\)

Le intersezioni tra la retta e la circonferenza sono dunque: 
\[p_0 \punto{-4}{-3} \text { e } p_1 \punto{-4}{+3}\]
\end{minipage}
\hfill
\begin{minipage}{.48\textwidth}
\begin{center}
\begin{inaccessibleblock}[Intersezioni tra una circonferenza e una retta.]
  \circonfretta
\end{inaccessibleblock}
\end{center}
\end{minipage}
\end{esempio}

% \begin{wrapfloat}{figure}{r}{0pt}
% \includegraphics[scale=0.35]{img/fig000_.png}
% \caption{...}
% \label{fig:...}
% \end{wrapfloat}
% 
% \begin{center} \input{\folder lbr/fig000_.pgf} \end{center}

\section{Circonferenza traslata}
\label{sec:circ_circtraslata}

Fin'ora abbiamo trattato circonferenze con il centro nell'origine degli assi, 
vogliamo ora generalizzare l'equazione in modo da ottenere l'equazione di una 
generica circonferenza del piano.

Consideriamo una circonferenza con centro nell'origine:
\[x^2 + y^2 = 52\]
e una generica traslazione:
\[\sistema{x' = x + \alpha \\ y' = y + \beta}\]
Riscriviamo le equazioni della traslazione esplicitando~\(x\) e~\(y\):
\[\sistema{x = x'- \alpha \\ y = y' - \beta}\]

Per traslare la circonferenza, operiamo la sostituzione di variabili
indicata dalla traslazione:
\[\tonda{x'-\alpha}^2 + \tonda{y'-\beta}^2 = r^2\]
questa è l'equazione della circonferenza traslata. Si può osservare che il 
centro della circonferenza traslata è: 
\[C'\punto{\alpha}{\beta}\]
Dato che ci riferiamo sempre allo stesso sistema di riferimento, semplifichiamo 
la scrittura eliminando gli apici ed evidenziando così che quella ottenuta è 
l'equazione di un'altra circonferenza dello stesso piano:
\[\tonda{x-\alpha}^2 + \tonda{y-\beta}^2 = r^2\]
Ora possiamo svolgere i calcoli e riscrivere l'equazione in un altro modo:
\[x^2 -2 \alpha x + \alpha^2 + y^2 -2 \beta y + \beta^2 = r^2\]
\[x^2 + y^2 -2 \alpha x -2 \beta y - r^2 + \alpha^2 + \beta^2 = 0\]
Possiamo osservare che essendo \(\alpha\) un numero, anche \(-2\alpha\) è un 
numero e anche \(-2\beta\) e anche \(- r^2 + \alpha^2 + \beta^2\). 
L'equazione di una circonferenza con centro in un punto qualsiasi del piano 
sarà del tipo:
\[x^2 + y^2 +a x +b y +c = 0\]
dove:
\[\sistema{a = -2\alpha \\ b = -2\beta \\ c = -r^2 +\alpha^2 +\beta^2}\]
In quest'ultimo sistema si possono esplicitare le coordinate del 
centro (\(\alpha\) e \(\beta\)) e il raggio (\(r\)):
\[\sistema{\alpha = -\dfrac{a}{2} \\[7pt] 
           \beta = -\dfrac{b}{2} \\[7pt]
           r = \sqrt{-c +\alpha^2 + \beta^2}}\]
In questo modo possiamo calcolare le coordinate del centro e il raggio della 
circonferenza partendo dai coefficienti dell'equazione scritta in forma 
polinomiale.

% \newpage

\begin{esempio}
~

\noindent\begin{minipage}{.55\textwidth}
Calcola l'equazione polinomiale della circonferenza di cento \(C\punto{-3}{2}\) 
e di raggio \(r=4\).

Possiamo usare l'equazione in forma canonica:
\[\tonda{x+3}^2 + \tonda{y-2}^2 = 4^2\]
e svolgere i calcoli:
\[x^2 +6x+9 +y^2 -4y +4 -16 = 0\]
da cui si ottiene:
\[x^2 +y^2 +6x -4y -3 = 0\]
\end{minipage}
\hfill
\begin{minipage}{.43\textwidth}
\begin{center}
\begin{inaccessibleblock}[Circonferenza con il centro in un punto qualsiasi 
del  piano.]
  \circtraslata
\end{inaccessibleblock}
\end{center}
\end{minipage}
Oppure possiamo partire dal significato dei coefficienti illustrato sopra:
\[\sistema{a = -2(-3)=6 \\ b = -2(+2)=-4 \\ c = -(+4)^2 +(-3)^2 +(+2)^2=-3}\]
\end{esempio}

\begin{esempio}
Calcola le coordinate del centro e il raggio della circonferenza:
\(x^2 +y^2 -6x + 10y -11 = 0\)

Usando il sistema precedente otteniamo:
\[\sistema{\alpha = -\dfrac{-6}{2} = +3\\[7pt] 
           \beta = -\dfrac{10}{2} = -5 \\[7pt] 
           r = \sqrt{-(-11) +(+3)^2 + (-5)^2} = \sqrt{45}= 
               \sqrt{9 \cdot 5} = 3\sqrt{5}}\]
Questa circonferenza ha centro \(C\punto{+3}{-5}\) e raggio \(r=3\sqrt{5}\)
\end{esempio}

\begin{osservazione}
La presenza di una radice quadrata nel calcolo del raggio della circonferenza, 
dovrebbe farci scattare un campanello di allarme: siamo sicuri di poter 
calcolare questa radice? siamo sicuri che il radicando sia positivo?

Il fatto che \(\alpha\) e \(\beta\) siano elevati al quadrato ci assicura che 
questi due addendi siano positivi, ma che dire di \(-c\)? Se \(c\) è negativo 
possiamo essere sicuri che il radicando sia positivo, ma se \(c\) è positivo e 
abbastanza grande, il radicando può essere negativo e in questo caso il raggio 
non potrà essere un numero reale.
\end{osservazione}

Vediamo un esempio.

\begin{esempio}
Calcola le coordinate del centro e il raggio della circonferenza:
\(x^2 +y^2 +2x -4y +9 = 0\)

Usando il sistema precedente otteniamo:
\[\sistema{\alpha = -\dfrac{+2}{2} = -1\\ 
           \beta = -\dfrac{-4}{2} = +2 \\ 
           r = \sqrt{-(+9) +(-1)^2 + (+2)^2} = \sqrt{-4}}\]
           
Questa circonferenza ha centro \(C\punto{-1}{+2}\) ma il suo raggio non è un 
numero reale. È una circonferenza immaginaria!
\end{esempio}

\section{Equazione di una circonferenza}
\label{sec:circ_equazione}

Abbiamo visto come calcolare l'equazione di una circonferenza conoscendo 
il centro e il raggio, ma una circonferenza può essere determinata anche in 
altri modi, conoscendo:

\begin{itemize} [noitemsep]
 \item tre suoi punti;
 \item il centro e un suo punto;
 \item gli estremi di un suo diametro.
\end{itemize}

Vediamo i primi due casi con due esempi.

\begin{esempio}
Calcola l'equazione di una circonferenza conoscendo tre punti di passaggio:
\(A\punto{-2}{2}, B\punto{6}{2}, C\punto{1}{7}\)

\vspace{7pt}

Innanzitutto disegniamo in un piano cartesiano i tre punti dati.
Poiché questi tre punti appartengono alla circonferenza, le loro coordinate 
devono essere soluzioni della sua equazione: \\
\(\sistema{x_A^2 +y_A^2 +ax_A +by_A +c = 0 \\
           x_B^2 +y_B^2 +ax_B +by_B +c = 0 \\
           x_C^2 +y_C^2 +ax_C +by_C +c = 0} \Rightarrow
  \sistema{4 +4 -2a +2b +c = 0 \\
           36 +4 +6a +2b +c = 0 \\
           1 +49 +a +7b +c = 0 } \Rightarrow
  \sistema{-2a +2b +c = -8 \\
           6a +2b +c = -40 \\
           a +7b +c = -50 }\)

\vspace{7pt}
           
Risolvendo il sistema troviamo i valori dei tre coefficienti. Ci sono molti 
metodi per risolvere il sistema, ma visto come è fatto, possiamo cambiare tutti 
i segni ad una delle equazioni e sommarla membro a membro alle altre due 
ottenendo in questo modo un sottosistema con le sole due incognite \(a\) e 
\(b\):
\[\sistema{-2a +2b +c = -8 \\
           6a +2b +c = -40 \\
           -a -7b -c = +50 } \qquad \Rightarrow \qquad
  \sistema{-3a -5b = 42 \\
           5a -5b = 10  }\]

Applicando ancora il metodo di riduzione a quest'ultimo sistema otteniamo: \\
\(-8a = 32 \sRarrow a = -4\)

E sostituendo nell'altra equazione: \\
\(-20 -5b = 10 \sRarrow b = -6\)

Sostituendo infine in una delle equazioni del sistema di partenza: \\
\(+8 -12 +c = -8 \sRarrow c = -4\)

Per cui l'equazione cercata è:
\[x^2 +y^2 -4x -6y -4 = 0\]

\end{esempio}

\begin{esempio}
Calcola l'equazione di una circonferenza sapendo che passa per il punto:
\(P\punto{-1}{4}\) e ha centro in: \(C\punto{2}{1}\).

Conoscendo il centro e un punto della circonferenza, possiamo calcolare il 
raggio che è la distanza \(CP\) e utilizzare quindi il metodo visto 
precedentemente. 
Ma seguiamo un'altra strada. 

Innanzitutto disegniamo in un piano cartesiano i 
due punti dati e, usando il compasso, la circonferenza cercata.
Poi, conoscendo le coordinate del centro, possiamo facilmente calcolare i due 
coefficienti \(a\) e \(b\): \\
\(\sistema{a = -2x_C = -4 \\ b = -2y_C = -2} \sRarrow x^2 +y^2 -4x -2y +c = 0\) 

A questo punto per determinare l'equazione resta da calcolare solo il 
coefficiente \(c\). Per ottenerlo possiamo imporre la condizione di passaggio 
per il punto \(P\): \\
\(x_P^2 +y_P^2 -4x_P -2y_P +c = 0 \sRarrow 1 + 16 +4 -8 +c = 0 \sRarrow 
  c = -13\)

Per cui la circonferenza cercata ha equazione:
\[x^2 +y^2 -4x -2y -13 = 0\]
\end{esempio}

\begin{esempio}
Calcola l'equazione di una circonferenza sapendo che gli estremi di un suo 
diametro sono:
\(A\punto{-3}{-1}\) e \(B\punto{7}{3}\).

Innanzitutto disegniamo in un piano cartesiano i due punti dati.
Il centro della circonferenza è il punto medio del diametro, è 
facile calcolare le sue coordinate: \\[3pt]
\(C\punto{\dfrac{x_A + x_B}{2}}{\dfrac{y_A + y_B}{2}} = 
\punto{\dfrac{-3 + 7}{2}}{\dfrac{-1 + 3}{2}} = \punto{2}{1}\)
\\[3pt]
A questo punto possiamo usare uno dei metodi già visti:\\
\(\sistema{a = -2x_C = -4 \\ b = -2y_C = -2} \sRarrow x^2 +y^2 -4x -2y +c = 0\) 
\\[3pt]
Poi calcoliamo \(c\) imporre la condizione di passaggio per un punto dato: \\[3pt]
\(x_B^2 +y_B^2 -4x_B -2y_B +c = 0 \sRarrow 49 + 9 -28 -6 +c = 0 \sRarrow 
  c = -24\)
\\[3pt]
Per cui la circonferenza cercata ha equazione:
\[x^2 +y^2 -4x -2y -24 = 0\]
\end{esempio}

\section{Circonferenze e rette}
\label{sec:circ_circrette}

Se consideriamo le posizioni reciproche di una circonferenza e di una retta, 
possiamo avere uno di questi tre casi:

\begin{description} %[noitemsep]
 \item [Retta secante:] 
retta e circonferenza hanno due punti distinti in comune. La distanza della 
retta dal centro della circonferenza è minore del raggio.
 \item [Retta tangente:]
retta e circonferenza si intersecano in due punti infinitamente vicini. 
La distanza della retta dal centro della circonferenza è uguale del raggio. Inoltre la retta è
perpendicolare al segmento che unisce il centro con il punto di tangenza.
 \item [Retta esterna:]
retta e circonferenza non hanno punti reali in comune. 
La distanza della retta dal centro della circonferenza è maggiore del 
raggio.
\end{description}

\begin{comment}
\noindent\begin{minipage}{.48\textwidth}
TODO
\end{minipage}
\hfill
\begin{minipage}{.48\textwidth}
\begin{center}
\begin{inaccessibleblock}[Circonferenza con una retta secante, una retta 
tangente e una retta esterna.]
%   \circonfretta TODO
\end{inaccessibleblock}
\end{center}
\end{minipage}
\end{comment}

\begin{esempio}
Disegna la circonferenza di equazione: \(x^2 +y^2 -2x +4y +4 = 0\) e
calcola le intersezioni con la retta \(r:2x-3y+12 = 0\)
\\[7pt]
Mettiamo a sistema le equazioni della circonferenza e della retta: \\[3pt]
\(\sistema{x^2 +y^2 -2x +4y +4 = 0 \\ y = \dfrac{2}{3}x+4}\) 
\\[7pt]
con il metodo di sostituzione otteniamo l'equazione risolutiva: \\[3pt]
\(x^2 +\tonda{\dfrac{2}{3}x+4}^2 -2x +4\tonda{\dfrac{2}{3}x+4}+4 = 0\) \\[3pt]
\(x^2 +\dfrac{4}{9}x^2+\dfrac{16}{3}x+16-2x+\dfrac{8}{3}x+16+4 = 0\) 
\\[4pt]
Moltiplicando l'equazione per 9 e sommando i termini simili otteniamo:\\[3pt]
%\(9x^2+4x^2+48x+144-18x+24x+144+36 = 0\) \\[4pt]
\(13x^2+54x+324=0\) \\[4pt]
\(x_{1,2} = \dfrac{-27 \pm \sqrt{729 -4212}}{13} = 
            \dfrac{-27 \pm \sqrt{-3483}}{13}\) 
\\[4pt]
L'equazione non ha soluzioni reali e quindi retta e circonferenza non si 
intersecano in punti reali, com'è possibile vedere anche dalla Figura \ref{fig:circ_circrette}
\end{esempio}

\begin{figure}
\begin{center}
\begin{inaccessibleblock}[Circonferenza con una retta secante, una retta 
tangente e una retta esterna.]
  \circrette
  \caption{Posizioni reciproche tra retta e circonferenza}
  \label{fig:circ_circrette}
\end{inaccessibleblock}
\end{center}
\end{figure}

\begin{esempio}
Calcola le intersezioni della circonferenza dell'esempio precedente con la 
retta di equazione: \(s:~2x -3y -3 = 0\)
\\[7pt]
Mettiamo a sistema le equazioni della circonferenza e della retta: \\[4pt]
\(\sistema{x^2 +y^2 -2x +4y -8 = 0 \\ y = \dfrac{2}{3}x -1}\) 
\\[7pt]
con il metodo di sostituzione otteniamo l'equazione risolutiva: \\[4pt]
\(x^2 +\tonda{\dfrac{2}{3}x -1}^2 -2x +4\tonda{\dfrac{2}{3}x -1} -8 = 0\) \\[4pt]
\(x^2 +\dfrac{4}{9}x^2 - \dfrac{4}{3}x +1 -2x +\dfrac{8}{3}x -4 -8 = 0\) 
\\[7pt]
Moltiplicando l'equazione per 9 e sommando i termini simili:\\[4pt]
% \(9x^2 +4x^2 -12x +9 -18x +24x -36 -72 = 0\) \\
\(13x^2 -6x -99 = 0\) \\[4pt]
\(x_{1,2} = \dfrac{3 \pm \sqrt{9 +1287}}{13} = 
            \dfrac{3 \pm \sqrt{1296}}{13} = 
            \dfrac{3 \pm 36}{13} \quad \Rightarrow \quad x_1 = -\dfrac{33}{13} \quad x_2 = 3\) \\[7pt] 
e sostituendo nell'equazione della retta otteniamo anche l'ordinata dei 
punti: \\[4pt]
\(I_1 = - \punto{-\dfrac{33}{13}}{-\dfrac{35}{13}} \quad I_2 = \punto{3}{1}\)
\quad (vedi Figura \ref{fig:circ_circrette})

% Mettiamo a sistema la circonferenza  e la retta per vedere se hanno due 
% soluzioni in comune e sono quindi secanti, una soluzione in comune e sono 
% tangenti, nessuna soluzione in comune e, quindi, non si intersecano.
% \begin{center} \begin{tabular}{rl}
% esplicitiamo \(y\) nella retta: & 
% \(\sistema{x^2 +y^2 +2x -y -3 = 0 \\ y = +3x -2}\) \\ [6pt]
% sostituiamo nella prima equazione: &  
% \(x^2 +9x^2 -12x +4 +2x -3x +2 -3 = 0\) \\
% che si riduce a: &    
% \(10x^2 -13x +3 = 0\) \\
% le soluzioni sono: & 
% \(x_1 = 1; \quad x_2 = \dfrac{3}{10}\) \\
% Sostituendoli nell'equazione \\ della retta otteniamo: & 
% \(P_1 \punto{1}{1}; \quad P_2 \punto{\dfrac{3}{10}}{-\dfrac{11}{10}}\)
% \end{tabular} \end{center}
\end{esempio}

\begin{esempio}
Calcola l'equazione della retta tangente alla circonferenza 

\(x^2 +y^2 -2x +4y -8 = 0\) nel suo punto \(P \punto{-1}{+1}\) 

La generica retta per \(P\) è: \\
\(y -1 = m \tonda{x +1} \sRarrow y = mx +m +1\)

Poniamo a sistema le equazioni: \\  
\(\sistema{x^2 + y^2 -2x +4y -8 = 0 \\ y = mx +m +1}\) 

Sostituendo si ottiene: \\    
\(x^2 + \tonda{mx +m +1}^2 -2x +4\tonda{mx +m +1} -8 = 0\) 

Svolgendo i calcoli: \\    
\(x^2 + m^2x^2 +m^2 +1 +2m^2x +2mx +2m -2x +4mx +4m +4 -8 = 0\) 

raccogliendo \(x^2\), \(x\) e i termini noti: \\    
\(\tonda{m^2+1}x^2 -2\tonda{m^2 +3m -1}x +\tonda{m^2 +6m -3} = 0\) 

perché le rette coincidano il discriminante deve essere 
uguale a zero: \(\Delta = 0\): \\ 
\(4\tonda{m^2 +3m -1}^2 -4 \tonda{m^2 +1}\tonda{m^2 +6m -3} = 0\) 

dividendo tutto per 4 ed eseguendo i calcoli: \\ 
\(\cancel{m^4} +9m^2 +1 +\cancel{6m^3} -2m^2 -6m -
\cancel{m^4} -\cancel{6m^3} +3m^2 -m^2 -6m +3 = 0\) 

e semplificando: \\ 
\(9m^2 -12m +4 = 0 \sRarrow \tonda{3m -2}^2 = 0\) 

da cui: \(m_{1,2} = \dfrac{2}{3} \)\\
la tangente è quindi (vedi Figura \ref{fig:circ_circrette}):
\[t:~y = \dfrac{2}{3}x +\dfrac{5}{3}\]
\end{esempio}

% \begin{esempio}
% Calcola l'equazione della retta tangente alla circonferenza 
% 
% \(x^2 + y^2 +10x +2y -42 = 0\) nel suo punto \(P \punto{3}{1}\) \\
% La generica retta per \(P\) è: \\
% \(y -1 = m \tonda{x -3} \sRarrow y = mx -3m +1\) \\
% Poniamo a sistema le equazioni: \\  
% \(\sistema{x^2 + y^2 +10x +2y -42 = 0 \\ y = mx -3m +1}\) \\ 
% Sostituendo si ottiene: \\    
% \(x^2 + \tonda{mx -3m +1}^2 +10x +2\tonda{mx -3m +1} -42 = 0\) \\
% Svolgendo i calcoli: \\    
% \(x^2 + m^2x^2 +9m^2 +1 -6m^2x +2mx -6m +10x +2mx -6m +2 -42 = 0\) \\
% e semplificando: \\    
% \(\tonda{1 +m^2}x^2 -2\tonda{3m^2 -2m -5}x +9m^2 -12m -39 = 0\) \\
% perché le rette coincidano \(\dfrac{\Delta}{4} = 0\): \\ 
% \(\tonda{3m^2 -2m -5}^2 -\tonda{1 +m^2}\tonda{9m^2 -12m -39} = 0\) \\
% eseguendo i calcoli: \\ 
% \(\cancel{9m^4} +4m^2 +25 -\cancel{12m^3} -30m^2 +20m -
% 9m^2 +12m +39 -\cancel{9m^4} +\cancel{12m^3} +39m^2 = 0\) \\
% e semplificando: \\ 
% \(4m^2 +32m +64 = 0 \sRarrow m^2 +8m +16 = 0 \sRarrow \tonda{m +4}^2 = 0\) \\
% da cui: \(m_{1,2} = -4 \)\\
% la tangente è quindi:
% \[y = -4x +13\]
% \end{esempio}

\begin{figure}
\begin{center}
\begin{inaccessibleblock}[Circonferenza con una retta secante, una retta 
tangente e una retta esterna.]
  \circtangenti
  \caption{Posizioni reciproche tra retta e circonferenza}
  \label{fig:circ_circtangenti}
\end{inaccessibleblock}
\end{center}
\end{figure}

\begin{esempio}
Data la circonferenza di centro \(C\punto{-2}{-1}\) e raggio \(r=\sqrt{10}\) 
calcola le equazioni delle tangenti tracciate dal punto \(P\punto{3}{4}\)

L'equazione della circonferenza è: \\
\(\tonda{x +2}^2 + \tonda{y+1}^2 = 10
\sRarrow x^2 +4x +4 +y^2 +2y +1 -10 = 0
\sRarrow x^2 +y^2 +4x +2y -5 = 0\)

Una qualunque retta passante per \(P\) (tranne quella verticale) è rappresentata dall'equazione: \\
\(y -4 = m \tonda{x -3} \sRarrow y = mx -3m +4\)

Mettendo a sistema le due equazioni posso trovare i punti in cui una generica 
retta del fascio interseca la circonferenza: \\
\(\sistema{x^2 +y^2 +4x +2y -5 = 0 \\ y = mx -3m +4}\)

Sostituendo la variabile \(y\) ottengo l'equazione risolutiva del sistema: \\
\(x^2 +\tonda{mx -3m +4}^2 +4x +2\tonda{mx -3m +4} -5 = 0 \)\\
\(x^2 +m^2x^2 +9m^2 +16 -6m^2x +8mx -24m +4x +2mx -6m +8 -5 = 0 \)\\
\(\tonda{m^2 +1}x^2 - 2\tonda{3m^2 -5m -2}x +\tonda{9m^2 -30m +19} = 0 \)

Perché la retta sia tangente bisogna che le intersezioni con la circonferenza 
siano coincidenti quindi imponiamo che il delta sia nullo. Per semplificare i calcoli si può utilizzare: \(\dfrac{\Delta}{4} = 0\): \\
\(\tonda{3m^2 -5m -2}^2 - \tonda{m^2 +1} \tonda{9m^2 -30m +19} = 0 \) \\
\(\cancel{9m^4} +25m^2 +4 -\cancel{30m^3} -12m^2 + 20 m -
\cancel{9m^4} +\cancel{30m^3} -19m^2 -9m^2 +30m -19= 0 \) \\
\(-15m^2 +50 m -15= 0 \sRarrow 3m^2 -10m +3 = 0\) \\

Risolvendo quest'ultima equazione otteniamo due valori di \(m\) che individuano 
due rette del fascio tangenti alla circonferenza: \\
\(m_{0, 1} = \dfrac{5 \pm \sqrt{25-9}}{3} = \dfrac{5 \pm 4}{3} \quad \sRarrow
\quad m_0 = \dfrac{1}{3} \quad \text{ e } \quad m_1 = 3\)

Da cui: 
\[t_0:~y = \dfrac{1}{3}x +3 \quad \text{ e } \quad t_1:~y = 3x -5\]

\end{esempio}

\section{Posizioni reciproche tra circonferenze}
\label{sec:circ_posizionireciproche}

Se vogliamo vedere tutte le posizioni reciproche di due circonferenza  
di raggio \(r_0\) e \(r_1\), possiamo partire dal posizionarle in modo che 
abbiano lo stesso centro, poi muovere una delle due e osservare che cosa 
succede (nel prossimo elenco la distanza tra i centri delle due circonferenze 
viene indicata con \(\overline{C_0 C_1}\)).

\begin{description} %[noitemsep]
 \item [Concentriche]
I due centri coincidono: \(\overline{C_0 C_1} = 0\).
Se i due raggi sono diversi non avranno punti in comune, altrimenti saranno 
coincidenti.
 \item [Una interna all'altra]
In questo caso \(\overline{C_0 C_1} < r_1 - r_0\).
 \item [Una è tangente interna all'altra]
In questo caso \(\overline{C_0 C_1} = r_1 - r_0\).
 \item [Secanti]
In questo caso \(r_1 - r_0 < \overline{C_0 C_1} < r_1 + r_0\).
 \item [Tangenti esterne]
In questo caso \(\overline{C_0 C_1} = r_1 + r_0\).
 \item [Esterne]
In questo caso \(\overline{C_0 C_1} > r_1 + r_0\).
\end{description}

\newpage %-----------------------------------------------------

\begin{esempio}
Trova qual è la posizione reciproca delle due circonferenze:

\(c_0:~x^2 + y^2 +2x +2y -2 = 0\) e \(c_1:~x^2 + y^2 -4x +4y -1 = 0\) 

Ricaviamo innanzitutto i centri e i raggi delle circonferenze: \\
\(C_0 \punto{-1}{-1} ~ r_0 = 2\) \qquad 
\(C_1 \punto{2}{-2} ~ r_1 = 3\) 

e calcoliamo i valori: \quad \(r_1 - r_0 = 1 \quad r_1 + r_0 = 5\) 

e la distanza dei centri: \quad \(\overline{C_0 C_1} = \sqrt{10}\) 

poiché \quad \(1 < \sqrt{10} < 5\) cioè \quad 
\(r_1 - r_0 < \overline{C_0 C_1} < r_1 + r_0\): 

le due circonferenze sono \emph{secanti}.
\end{esempio}

\begin{esempio}
Calcola i punti di intersezione delle due circonferenze:

\(c_0:~x^2 + y^2 +2x = 0\) e \(c_1:~x^2 + y^2 +4x +2y +4 = 0\) 

Mettiamo a sistema le equazioni delle due circonferenze: \\
\(\sistema{x^2 + y^2 +2x = 0 \\x^2 + y^2 +4x +2y +4 = 0}\) 

Con il metodo di riduzione, possiamo ricavare un'equazione di primo grado 
combinando le due equazioni del sistema: \\
\(2x +2y +4 = 0 \sRarrow y = -x -2\)

Mettiamo a sistema l'equazione ottenuta con una delle due 
equazioni di secondo grado: \\
\(\sistema{y = -x -2 \\ x^2 + y^2 +2x = 0}\) 

Possiamo ora risolverlo con il metodo di sostituzione: \\
\(\sistema{y = -x -2 \\ x^2 + x^2 +4x +4 +2x = 0} \sRarrow
  \sistema{y = -x -2 \\ 2x^2 +6x +4 = 0}\) 
  
La seconda è un'equazione di secondo grado con una sola incognita: \\
\(x^2 +3x +2 = 0 \sRarrow \tonda{x +2} \tonda{x +1} = 0\) 

che danno le due soluzioni: \\
\(\sistema{x_1 = -2 \\ y_1 = +2 -2 = 0 } \quad \sand \quad
  \sistema{x_2 = -1 \\ y_2 = +2 -2 = -1 }\)
  
Le intersezioni cercate sono dunque: 
\[\punto{-2}{0} \text{ e } \punto{-1}{-1}\]
\end{esempio}

% \subsection{Fasci di circonferenze}
% \label{subsec:circ_fasci}

% \section{Curve deducibili dall'equazione della circonferenza}
% \label{sec:circ_curve_deducibili}


