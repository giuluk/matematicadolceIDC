% (c) 2014 Daniele Zambelli - daniele.zambelli@gmail.com

\section{Esercizi}

\subsection{Esercizi dei singoli paragrafi}

\subsubsection*{\numnameref{sec:retta_equazionilineari}}

\begin{esercizio}\label{ese:}
 Individua quale tra i seguenti punti appartiene alla retta.
 \begin{enumeratea}
  \item  $y = \dfrac{3}{2} x +\dfrac{33}{2} 
  \hfill (-5;~9),~(4;~-10),~(4;~-9),~(5;~-9),~(-6;~9),~(5;~-10)$
  \item  $y = -\dfrac{13}{2} x -37 
  \hfill (3;~10),~(-4;~-11),~(-5;~-12),~(4;~10),~(-5;~-11),~(-4;~-12)$
  \item  $y = \dfrac{3}{7} x +\dfrac{72}{7} 
  \hfill (-11;~6),~(10;~-7),~(-10;~5),~(-11;~5),~(-10;~6),~(10;~-6)$
  \item  $y = \dfrac{2}{15} x -\dfrac{2}{5} 
  \hfill (-12;~-2),~(11;~1),~(-13;~-3),~(11;~2),~(12;~1),~(12;~2)$
  \item  $y = -\dfrac{13}{5} x +\dfrac{32}{5} 
  \hfill (-1;~9),~(1;~-9),~(-2;~9),~(0;~-9),~(0;~-10),~(1;~-10)$
  \item  $y = -\dfrac{3}{5} x +\dfrac{4}{5} 
  \hfill (-9;~3), (-9;~4), (-8;~4), (8;~-5), (7;~-5), (8;~-4)$
  \item  $y = \dfrac{1}{6} x +\dfrac{7}{6} 
  \hfill (5;~2), (-6;~-2), (-5;~-3), (4;~2), (-5;~-2), (-6;~-3)$
  \item  $y = \dfrac{4}{3} x -3 
  \hfill (5;~5), (-7;~-5), (6;~5), (-7;~-6), (5;~4), (-6;~-6)$
  \item  $y = -5 x +49 
  \hfill (-9;~-10), (-8;~-10), (7;~8), (8;~8), (-9;~-9), (8;~9)$
  \item  $y = \dfrac{2}{9} x +\dfrac{49}{9} 
  \hfill (-11;~3), (10;~-3), (-11;~2), (11;~-3), (-12;~3), (10;~-4)$
  \item  $y = \dfrac{8}{3} x +\dfrac{56}{3} 
  \hfill (4;~-8), (3;~-8), (-4;~7), (-5;~7), (4;~-9), (-4;~8)$
  \item  $x = -9 \hfill (-10;~-9), (-10;~-8), (-9;~-8), (8;~7), (8;~8), (9;~8)$
  \item  $y = \dfrac{3}{2} x +\dfrac{19}{2} 
  \hfill (-2;~8), (0;~-9), (-1;~7), (-2;~7), (-1;~8), (0;~-8)$
  \item  $y = 3 x +10 
  \hfill (4;~4), (5;~4), (-5;~-6), (-6;~-5), (4;~5), (-5;~-5)$
  \item  $y = \dfrac{3}{17} x -\dfrac{135}{17} 
  \hfill (-12;~5), (-12;~6), (10;~-6), (11;~-6), (-11;~5), (11;~-7)$
  \item  $y = \dfrac{10}{3} x +\dfrac{4}{3} 
  \hfill (-2;~-2), (-1;~-2), (0;~1), (1;~2), (-1;~-3), (0;~2)$
  \item  $y = \dfrac{3}{10} x +\dfrac{19}{5} 
  \hfill (-7;~2), (-6;~2), (-7;~1), (5;~-2), (-6;~1), (6;~-2)$
%   \item  $y = \dfrac{4}{7} x -\dfrac{20}{7} 
%   \hfill (-2;~-5), (2;~4), (1;~3), (-2;~-4), (2;~3), (-3;~-4)$
%   \item  $y = \dfrac{1}{12} x -\dfrac{109}{12} 
%   \hfill (-2;~8), (0;~-10), (1;~-10), (-1;~9), (-1;~8), (1;~-9)$
%   \item  $y = -\dfrac{17}{11} x +\dfrac{76}{11} 
%   \hfill (-3;~9), (1;~-11), (-2;~9), (2;~-10), (-3;~10), (-2;~10)$
 \end{enumeratea}
\end{esercizio}

\subsubsection*{\numnameref{sec:retta_equazioni}}

\begin{esercizio}\label{ese:02_01.} %TODO
Riconosci quali delle seguenti è l'equazione di una retta:
\begin{multicols}{2}
 \begin{enumeratea}
  \item  $ y = -3 x +4$
  \item  $ y^2 = x + 3$
  \item  $ y^3 = -x + y^3 -2$
  \item  $ (x +2)(x-2)=(x +3)^2$
  \item  $ (x +y)(x-y) + (y-5)^2 = (x +4)^2$
  \item  $ 0,1 x + 0,2 y = 0,3$
  \item  $ x^2 - y^2 = 0$
  \item  $ 3x^2 - 2x + 5y = 3x^2$
  \item  $ y +4 x = 5$
  \item  $ 5 x -4 y+3 = 0$
  \item  $ (x+3)^2 - (y+2)^2 = 2 x - 2y$
  \item  $ y = 0$
  \item  $ y = 2$
  \item  $ x = y$
  \item  $ 0 x + 0 y = 7$
  \item  $ x^2 -(x+2)^2 = 7 (x -y)$
 \end{enumeratea}
\end{multicols}
\end{esercizio}

\newpage

\begin{esercizio}\label{ese:}
 Trasforma le equazioni implicite in equazioni esplicite.
 \begin{multicols}{3}
 \begin{enumeratea}
  \item  $-1 x - 11 y - 110 = 0$
  \item  $-8 x - 2 y - 20 = 0$
  \item  $-9 x  = 0$
  \item  $8 x + y - 7 = 0$
  \item  $-2 x - 6 y - 54 = 0$
  \item  $-6 x - 9 y - 27 = 0$
  \item  $-7 x - 8 y - 64 = 0$
  \item  $4 x - 5 y - 5 = 0$
  \item  $7 x - y - 9 = 0$
  \item  $4 x + 7 y + 14 = 0$
  \item  $-7 x + 6 y + 0 = 0$
  \item  $-5 x + 10 y - 50 = 0$
  \item  $x = 0$
  \item  $6 x + 4 y - 4 = 0$
  \item  $6 x - 11 y + 99 = 0$
  \item  $11 x - 8 y - 40 = 0$
  \item  $x = -7$
  \item  $-4 x - 9 y + 36 = 0$
  \item  $-9 x - 6 y - 6 = 0$
  \item  $-7 x + 4 y - 40 = 0$
  \item  $8 x + 4 y - 12 = 0$
 \end{enumeratea}
 \end{multicols}
\end{esercizio}


\begin{esercizio}\label{ese:}
 Trasforma le equazioni esplicite in equazioni implicite.
 \begin{multicols}{3}
 \begin{enumeratea}
  \item  $y = \dfrac{2}{5} x +2$
  \item  $y = \dfrac{10}{11} x -6$
  \item  $y = -\dfrac{5}{3} x +10$
  \item  $y = \dfrac{1}{3} x -3$
  \item  $y = - x -5$
  \item  $x = 0$
  \item  $y = \dfrac{8}{7} x +8$
  \item  $y = \dfrac{1}{11} x +9$
  \item  $y = -\dfrac{3}{10} x -7$
  \item  $y = -\dfrac{1}{3} x -7$
  \item  $y = -11 x +8$
  \item  $y = -\dfrac{5}{11} x +10$
  \item  $y = -8$
  \item  $y = -\dfrac{3}{5} x -6$
  \item  $y = 1$
  \item  $y = -\dfrac{9}{5} x -9$
  \item  $y = 8 x $
  \item  $y = \dfrac{2}{5} x +7$
  \item  $y = -\dfrac{11}{10} x +2$
  \item  $y = -\dfrac{1}{2} x +8$
  \item  $y = -\dfrac{3}{5} x +1$
 \end{enumeratea}
 \end{multicols}
\end{esercizio}

% \vspace{-6pt}
\subsubsection*{\numnameref{sec:retta_disegno}}
% \vspace{-4pt}

\begin{esercizio}\label{ese:}
Disegna i seguenti gruppi di rette in diversi piani cartesiani 
calcolandone prima, in una tabella, tre punti.

\begin{tabular}{llll}
1. &
a) \quad $y = \dfrac{6}{5} x +7$ & 
b) \quad $y = \dfrac{11}{10} x -6$ & 
c) \quad $y = -\dfrac{6}{11} x +11$ \\
2. &
a) \quad $y = 3 x +12$ &   
b) \quad $y = -3$ &
c) \quad $y = \dfrac{9}{5} x -2$ \\
3. &
a) \quad $y = -\dfrac{5}{9} x +6$ &   
b) \quad $y = 2 x -9$ &
c) \quad $y = \dfrac{9}{7} x +12$ \\
4. &
a) \quad $x = 0$ &   
b) \quad $y = -\dfrac{10}{3} x -7$ &
c) \quad $y = -\dfrac{12}{11} x -11$ \\
5. &
a) \quad $y = \dfrac{8}{3} x -6$ &   
b) \quad $y = 3 x +10$ &
c) \quad $y = \dfrac{3}{5} x +3$ \\
6. &
a) \quad $y = 7 x +5$ &   
b) \quad $y = 5 x +4$ &
c) \quad $y = 0$ \\
7. &
a) \quad $y = \dfrac{10}{7} x -10$ &   
b) \quad $y = -\dfrac{2}{5} x +4$ &
c) \quad $x = 5$ \\
\end{tabular}
\end{esercizio}

\vspace{-6pt}
\begin{esercizio}\label{ese:}
Disegna i seguenti gruppi di rette in diversi piani cartesiani 
calcolandone prima, in una tabella, tre punti.

\begin{tabular}{llll}
1. &
a) \quad $2 x - 10 y - 30 = 0$ & 
b) \quad $4 x + 10 y - 40 = 0$ & 
c) \quad $-3 x + y + 0 = 0$ \\
2. &
a) \quad $11 x - 3 y - 12 = 0$ &   
b) \quad $7 x = 0$ &
c) \quad $-10 x - 2 y - 16 = 0$ \\
3. &
a) \quad $-7 x - 4 y - 4 = 0$ &   
b) \quad $9 x + 7 y + 42 = 0$ &
c) \quad $-8 x + y + 9 = 0$ \\
4. &
a) \quad $10 x - y + 9 = 0$ &   
b) \quad $6 x - 8 y - 48 = 0$ &
c) \quad $-7 x - y - 11 = 0$ \\
5. &
a) \quad $4 x + 4 y + 36 = 0$ &   
b) \quad $-5 x - 8 y - 48 = 0$ &
c) \quad $-7 x = 0$ \\
6. &
a) \quad $-7 x + 7 y + 63 = 0$ &   
b) \quad $7 x + 6 y + 30 = 0$ &
c) \quad $-11 x - y + 3 = 0$ \\
7. &
a) \quad $-5 x + 5 y - 45 = 0$ &   
b) \quad $8 x - y + 11 = 0$ &
c) \quad $5 x + 6 y - 24 = 0$ \\
\end{tabular}
\end{esercizio}


\subsubsection*{\numnameref{sec:retta_coefficienti}}

\begin{esercizio}\label{ese:}
Disegna i seguenti gruppi di rette in diversi 
piani cartesiani usando il metodo rapido.

\begin{tabular}{llll}
1. &
a) \quad $y = -\dfrac{1}{3} x -4$ & 
b) \quad $y = x -8$ & 
c) \quad $y = -\dfrac{2}{5} x -2$ \\
2. &
a) \quad $y = -\dfrac{1}{2} x +11$ &   
b) \quad $y = -\dfrac{9}{11} x -6$ &
c) \quad $y = 7 x -7$ \\
3. &
a) \quad $y = \dfrac{5}{6} x +3$ &   
b) \quad $y = 2 x -6$ &
c) \quad $y = - x +1$ \\
4. &
a) \quad $y = \dfrac{9}{2} x -4$ &   
b) \quad $y = \dfrac{11}{4} x -3$ &
c) \quad $y = -\dfrac{2}{5} x -5$ \\
5. &
a) \quad $y = -\dfrac{9}{10} x -11$ &   
b) \quad $y = -3 x +12$ &
c) \quad $y = -\dfrac{2}{3} x +12$ \\
6. &
a) \quad $y = 1$ &   
b) \quad $y = \dfrac{1}{8} x +3$ &
c) \quad $y = \dfrac{10}{3} x -4$ \\
7. &
a) \quad $y = -2$ &   
b) \quad $y = x +12$ &
c) \quad $y = 2x -7$ \\
\end{tabular}
\end{esercizio}

\begin{esercizio}\label{ese:}
Disegna i seguenti gruppi di rette in diversi 
piani cartesiani usando il metodo rapido.

\begin{tabular}{llll}
1. &
a) \quad $8 x - 2 y - 18 = 0$ & 
b) \quad $-5 x + 6 y - 18 = 0$ & 
c) \quad $9 x - 45 = 0$ \\
2. &
a) \quad $-7 x + 8 y + 80 = 0$ &   
b) \quad $2 y + 18 = 0$ &
c) \quad $-4 x + 6 y + 12 = 0$ \\
3. &
a) \quad $-7 x - 6 y + 12 = 0$ &   
b) \quad $-6 x - 4 y + 20 = 0$ &
c) \quad $-4 x + y + 6 = 0$ \\
4. &
a) \quad $10 x - 11 y = 0$ &   
b) \quad $- 5 y + 15 = 0$ &
c) \quad $3 x + 11 y + 0 = 0$ \\
5. &
a) \quad $-5 x + 5 y + 50 = 0$ &   
b) \quad $-2 x + 7 = 0$ &
c) \quad $6 x - 5 y + 30 = 0$ \\
6. &
a) \quad $-8 x - y + 12 = 0$ &   
b) \quad $-4 x + 11 y - 11 = 0$ &
c) \quad $-2 x + 7 y - 84 = 0$ \\
7. &
a) \quad $-12 x - 7 y = 0$ &   
b) \quad $8 x - 10 y - 50 = 0$ &
c) \quad $5 x - 10 y - 30 = 0$ \\
\end{tabular}
\end{esercizio}

\subsubsection*{\numnameref{sec:retta_rettaperduepunti}}

\begin{esercizio}\label{ese:}
 Calcola l'equazione della retta: AB.
\begin{multicols}{2}
 \begin{enumeratea}
  \item  $A(3;~2),~B(8;~8)$ \hfill 
   [$y = \dfrac{6}{5} x -\dfrac{8}{5}$]
  \item  $A(-6;~7),~B(-11;~6)$ \hfill 
   [$y = \dfrac{1}{5} x +\dfrac{41}{5}$]
  \item  $A(-9;~1),~B(9;~4)$ \hfill 
   [$y = \dfrac{1}{6} x +\dfrac{5}{2}$]
  \item  $A(0;~-12),~B(-10;~11)$ \hfill 
   [$y = -\dfrac{23}{10} x -12$]
  \item  $A(-5;~1),~B(4;~-2)$ \hfill 
   [$y = -\dfrac{1}{3} x -\dfrac{2}{3}$]
  \item  $A(-3;~-4),~B(4;~-7)$ \hfill 
   [$y = -\dfrac{3}{7} x -\dfrac{37}{7}$]
  \item  $A(6;~-7),~B(-1;~-9)$ \hfill 
   [$y = \dfrac{2}{7} x -\dfrac{61}{7}$]
  \item  $A(-1;~3),~B(-7;~-4)$ \hfill 
   [$y = \dfrac{7}{6} x +\dfrac{25}{6}$]
  \item  $A(10;~1),~B(-11;~-10)$ \hfill 
   [$y = \dfrac{11}{21} x -\dfrac{89}{21}$]
  \item  $A(-8;~-6),~B(-1;~-11)$ \hfill 
   [$y = -\dfrac{5}{7} x -\dfrac{82}{7}$]
  \item  $A(-4;~9),~B(3;~6)$ \hfill 
   [$y = -\dfrac{3}{7} x +\dfrac{51}{7}$]
  \item  $A(1;~8),~B(-1;~-11)$ \hfill 
   [$y = \dfrac{19}{2} x -\dfrac{3}{2}$]
  \item  $A(-6;~1),~B(-12;~6)$ \hfill 
   [$y = -\dfrac{5}{6} x -4$]
  \item  $A(4;~11),~B(2;~-9)$ \hfill 
   [$y = 10 x -29$]
%   \item  $A(-10;~-5),~B(4;~-10)$ \hfill 
%    [$y = -\dfrac{5}{14} x -\dfrac{60}{7}$]
  \item  $A(10;~-6),~B(-12;~7)$ \hfill 
   [$y = -\dfrac{13}{22} x -\dfrac{1}{11}$]
  \item  $A(-6;~-5),~B(-4;~-3)$ \hfill 
   [$y = x +1$]
  \item  $A(-9;~9),~B(9;~10)$ \hfill 
   [$y = \dfrac{1}{18} x +\dfrac{19}{2}$]
  \item  $A(4;~-5),~B(-10;~11)$ \hfill 
   [$y = -\dfrac{8}{7} x -\dfrac{3}{7}$]
  \item  $A(-4;~8),~B(-6;~2)$ \hfill 
   [$y = 3 x +20$]
 \end{enumeratea}
\end{multicols}
\end{esercizio}

\newpage

\subsubsection*{\numnameref{sec:retta_paralleleleperpendicolari}}

\begin{esercizio}\label{ese:}
 Per ciascuna delle seguenti terne di 
punti disegna la retta~AB e le rette parallela e perpendicolare passanti 
per~C.
\begin{multicols}{2}
 \begin{enumeratea}
  \item  $A(10;~7),~B(-9;~-10),~C(3;~-12)$
  \item  $A(-1;~6),~B(-5;~6),~C(-4;~-5)$
  \item  $A(-7;~-2),~B(-9;~-6),~C(5;~-12)$
  \item  $A(-3;~0),~B(-4;~-4),~C(-9;~-9)$
  \item  $A(4;~-3),~B(-10;~9),~C(8;~6)$
  \item  $A(4;~11),~B(-12;~-11),~C(9;~5)$
  \item  $A(6;~-2),~B(-12;~-7),~C(10;~-8)$
  \item  $A(-4;~4),~B(10;~-10),~C(11;~-1)$
  \item  $A(-3;~-10),~B(9;~8),~C(8;~-9)$
  \item  $A(7;~-12),~B(6;~-4),~C(-11;~-3)$
  \item  $A(0;~0),~B(-8;~-3),~C(4;~11)$
  \item  $A(-2;~-2),~B(7;~-7),~C(4;~8)$
  \item  $A(-7;~-9),~B(-4;~8),~C(4;~10)$
  \item  $A(-8;~-5),~B(11;~11),~C(9;~5)$
%   \item  $A(11;~-7),~B(-12;~5),~C(-4;~-7)$
%   \item  $A(11;~3),~B(-1;~-4),~C(-10;~-1)$
%   \item  $A(5;~0),~B(6;~11),~C(3;~-1)$
%   \item  $A(-7;~8),~B(-7;~4),~C(8;~-8)$
%   \item  $A(7;~5),~B(-4;~2),~C(-6;~-5)$
%   \item  $A(7;~-5),~B(2;~-12),~C(-7;~0)$
 \end{enumeratea}
\end{multicols}
\end{esercizio}

% \subsubsection*{\numnameref{sec:retta_fasci}}


\begin{esercizio}\label{ese:}
 Per ciascuna delle seguenti terne di 
punti disegna la retta~AB e le rette parallela e perpendicolare passanti 
per~C. poi calcolane le equazioni.
 \begin{enumeratea}
  \item  $A(3;~-3),~B(-10;~3),~C(-4;~9)$ \hfill 
   [$y = -\dfrac{6}{13} x -\dfrac{21}{13},~y = -\dfrac{6}{13} x 
+\dfrac{93}{13},~y = \dfrac{13}{6} x +\dfrac{53}{3}$]
  \item  $A(4;~-12),~B(9;~-6),~C(6;~-9)$ \hfill 
   [$y = \dfrac{6}{5} x -\dfrac{84}{5},~y = \dfrac{6}{5} x -\dfrac{81}{5},~y = 
-\dfrac{5}{6} x -4$]
  \item  $A(4;~-9),~B(-11;~-5),~C(4;~-10)$ \hfill 
   [$y = -\dfrac{4}{15} x -\dfrac{119}{15},~y = -\dfrac{4}{15} x 
-\dfrac{134}{15},~y = \dfrac{15}{4} x -25$]
  \item  $A(-3;~3),~B(-10;~-7),~C(0;~-3)$ \hfill 
   [$y = \dfrac{10}{7} x +\dfrac{51}{7},~y = \dfrac{10}{7} x -3,~y = 
-\dfrac{7}{10} x -3$]
  \item  $A(6;~-3),~B(9;~-12),~C(10;~8)$ \hfill 
   [$y = -3 x +15,~y = -3 x +38,~y = \dfrac{1}{3} x +\dfrac{14}{3}$]
  \item  $A(-4;~-8),~B(4;~2),~C(-12;~-11)$ \hfill 
   [$y = \dfrac{5}{4} x -3,~y = \dfrac{5}{4} x +4,~y = -\dfrac{4}{5} x 
-\dfrac{103}{5}$]
  \item  $A(10;~-6),~B(9;~7),~C(0;~-5)$ \hfill 
   [$y = -13 x +124,~y = -13 x -5,~y = \dfrac{1}{13} x -5$]
  \item  $A(-2;~0),~B(1;~4),~C(2;~0)$ \hfill 
   [$y = \dfrac{4}{3} x +\dfrac{8}{3},~y = \dfrac{4}{3} x -\dfrac{8}{3},~y = 
-\dfrac{3}{4} x +\dfrac{3}{2}$]
  \item  $A(-10;~7),~B(-6;~-3),~C(11;~5)$ \hfill 
   [$y = -\dfrac{5}{2} x -18,~y = -\dfrac{5}{2} x +\dfrac{65}{2},~y = 
\dfrac{2}{5} x +\dfrac{3}{5}$]
  \item  $A(-6;~8),~B(2;~-5),~C(-11;~-3)$ \hfill 
   [$y = -\dfrac{13}{8} x -\dfrac{7}{4},~y = -\dfrac{13}{8} x -\dfrac{167}{8},~y 
= \dfrac{8}{13} x +\dfrac{49}{13}$]
%   \item  $A(-4;~-4),~B(1;~2),~C(-7;~4)$ \hfill 
%    [$y = \dfrac{6}{5} x +\dfrac{4}{5},~y = \dfrac{6}{5} x +\dfrac{62}{5},~y = 
% -\dfrac{5}{6} x -\dfrac{11}{6}$]
%   \item  $A(-1;~-6),~B(8;~-9),~C(10;~9)$ \hfill 
%    [$y = -\dfrac{1}{3} x -\dfrac{19}{3},~y = -\dfrac{1}{3} x +\dfrac{37}{3},~y = 
% 3 x -21$]
%   \item  $A(10;~-10),~B(-12;~10),~C(-12;~5)$ \hfill 
%    [$y = -\dfrac{10}{11} x -\dfrac{10}{11},~y = -\dfrac{10}{11} x 
% -\dfrac{65}{11},~y = \dfrac{11}{10} x +\dfrac{91}{5}$]
%   \item  $A(-1;~-9),~B(2;~-11),~C(-9;~11)$ \hfill 
%    [$y = -\dfrac{2}{3} x -\dfrac{29}{3},~y = -\dfrac{2}{3} x +5,~y = 
% \dfrac{3}{2} x +\dfrac{49}{2}$]
%   \item  $A(11;~2),~B(-12;~11),~C(7;~9)$ \hfill 
%    [$y = -\dfrac{9}{23} x +\dfrac{145}{23},~y = -\dfrac{9}{23} x 
% +\dfrac{270}{23},~y = \dfrac{23}{9} x -\dfrac{80}{9}$]
%   \item  $A(-8;~-4),~B(5;~-10),~C(-2;~-7)$ \hfill 
%    [$y = -\dfrac{6}{13} x -\dfrac{100}{13},~y = -\dfrac{6}{13} x 
% -\dfrac{103}{13},~y = \dfrac{13}{6} x -\dfrac{8}{3}$]
%   \item  $A(10;~7),~B(1;~5),~C(-10;~1)$ \hfill 
%    [$y = \dfrac{2}{9} x +\dfrac{43}{9},~y = \dfrac{2}{9} x +\dfrac{29}{9},~y = 
% -\dfrac{9}{2} x -44$]
%   \item  $A(0;~11),~B(-1;~9),~C(-4;~-1)$ \hfill 
%    [$y = 2 x +11,~y = 2 x +7,~y = -\dfrac{1}{2} x -3$]
%   \item  $A(8;~-1),~B(2;~-10),~C(-6;~-5)$ \hfill 
%    [$y = \dfrac{3}{2} x -13,~y = \dfrac{3}{2} x +4,~y = -\dfrac{2}{3} x -9$]
%   \item  $A(11;~-8),~B(-11;~-10),~C(4;~-11)$ \hfill 
%    [$y = \dfrac{1}{11} x -9,~y = \dfrac{1}{11} x -\dfrac{125}{11},~y = -11 x 
% +33$]
 \end{enumeratea}
\end{esercizio}

\subsubsection*{\numnameref{sec:retta_distanzapuntoretta}}

\begin{esercizio}\label{ese:}
 Calcola la distanza tra il punto~$P$ e la retta~$r$
 \begin{enumeratea}
  \item  $P(11;~-7),~r:~-6 x + 7 y + 21 = 0$ \hfill 
   [$\dfrac{94}{\sqrt{85}}\approx  10.2$]
  \item  $P(-10;~10),~r:~3 x + 10 y + 10 = 0$ \hfill 
   [$\dfrac{80}{\sqrt{109}}\approx 7.663$]
  \item  $P(8;~-1),~r:~-12 x - 10 y + 40 = 0$ \hfill 
   [$\dfrac{46}{\sqrt{244}}\approx 2.945$]
  \item  $P(-5;~-11),~r:~-6 x + 0 = 0$ \hfill 
   [$\dfrac{30}{\sqrt{36}}\approx   5.0$]
  \item  $P(-1;~-4),~r:~-3 x + 9 y - 81 = 0$ \hfill 
   [$\dfrac{114}{\sqrt{90}}\approx 12.02$]
  \item  $P(-3;~0),~r:~9 x - 6 y + 72 = 0$ \hfill 
   [$\dfrac{45}{\sqrt{117}}\approx  4.16$]
  \item  $P(-10;~-7),~r:~10 x - 9 y + 27 = 0$ \hfill 
   [$\dfrac{10}{\sqrt{181}}\approx0.7433$]
  \item  $P(4;~0),~r:~-9 x + 4 y + 44 = 0$ \hfill 
   [$\dfrac{8}{\sqrt{97}}\approx0.8123$]
  \item  $P(-5;~8),~r:~10 x - 3 y - 27 = 0$ \hfill 
   [$\dfrac{101}{\sqrt{109}}\approx 9.674$]
  \item  $P(-11;~0),~r:~9 x + 11 y - 33 = 0$ \hfill 
   [$\dfrac{132}{\sqrt{202}}\approx 9.287$]
  \item  $P(-9;~-10),~r:~2 x + 4 y + 24 = 0$ \hfill 
   [$\dfrac{34}{\sqrt{20}}\approx 7.603$]
%   \item  $P(5;~7),~r:~3 x + 1 y + 8 = 0$ \hfill 
%    [$\dfrac{30}{\sqrt{10}}\approx 9.487$]
%   \item  $P(8;~7),~r:~-10 x + 6 y + 54 = 0$ \hfill 
%    [$\dfrac{16}{\sqrt{136}}\approx 1.372$]
%   \item  $P(-2;~-6),~r:~-2 x - 6 y - 6 = 0$ \hfill 
%    [$\dfrac{34}{\sqrt{40}}\approx 5.376$]
%   \item  $P(-12;~9),~r:~-1 x + 9 y - 63 = 0$ \hfill 
%    [$\dfrac{30}{\sqrt{82}}\approx 3.313$]
%   \item  $P(-6;~4),~r:~-11 x + 10 y - 70 = 0$ \hfill 
%    [$\dfrac{36}{\sqrt{221}}\approx 2.422$]
%   \item  $P(-6;~-3),~r:~-3 y + 33 = 0$ \hfill 
%    [$\dfrac{42}{\sqrt{9}}\approx  14.0$]
%   \item  $P(7;~5),~r:~-2 x - 7 y - 35 = 0$ \hfill 
%    [$\dfrac{84}{\sqrt{53}}\approx 11.54$]
%   \item  $P(-5;~-6),~r:~-10 x + 7 y + 63 = 0$ \hfill 
%    [$\dfrac{71}{\sqrt{149}}\approx 5.817$]
%   \item  $P(-6;~11),~r:~9 x + 5 y + 55 = 0$ \hfill 
%    [$\dfrac{56}{\sqrt{106}}\approx 5.439$]
 \end{enumeratea}
\end{esercizio}


\begin{esercizio}\label{ese:}
 Calcola la distanza tra il punto~$P$ e la retta~$r$
 \begin{enumeratea}
  \item  $P(-2;~-10),~r:~y = -\dfrac{1}{9} x -11$ \hfill 
   [$\dfrac{7}{\sqrt{82}}\approx 0.773$]
  \item  $P(7;~-9),~r:~y = \dfrac{2}{11} x +1$ \hfill 
   [$\dfrac{124}{\sqrt{125}}\approx 11.09$]
  \item  $P(6;~-2),~r:~y = -\dfrac{3}{4} x -1$ \hfill 
   [$\dfrac{28}{\sqrt{100}}\approx   2.8$]
  \item  $P(-1;~-7),~r:~y = -\dfrac{2}{5} x -6$ \hfill 
   [$\dfrac{7}{\sqrt{29}}\approx   1.3$]
  \item  $P(-4;~0),~r:~y = - x +7$ \hfill 
   [$\dfrac{33}{\sqrt{18}}\approx 7.778$]
  \item  $P(11;~9),~r:~y = \dfrac{10}{11} x +2$ \hfill 
   [$\dfrac{33}{\sqrt{221}}\approx  2.22$]
  \item  $P(8;~0),~r:~y = -\dfrac{1}{10} x -6$ \hfill 
   [$\dfrac{68}{\sqrt{101}}\approx 6.766$]
  \item  $P(-8;~-4),~r:~y = -\dfrac{9}{10} x -6$ \hfill 
   [$\dfrac{52}{\sqrt{181}}\approx 3.865$]
  \item  $P(2;~0),~r:~y = -\dfrac{6}{5} x +2$ \hfill 
   [$\dfrac{2}{\sqrt{61}}\approx0.2561$]
  \item  $P(9;~7),~r:~y = \dfrac{1}{2} x +2$ \hfill 
   [$\dfrac{4}{\sqrt{80}}\approx0.4472$]
  \item  $P(-3;~1),~r:~y = \dfrac{2}{7} x +2$ \hfill 
   [$\dfrac{1}{\sqrt{53}}\approx0.1374$]
%   \item  $P(1;~6),~r:~y = \dfrac{6}{5} x +3$ \hfill 
%    [$\dfrac{9}{\sqrt{61}}\approx 1.152$]
%   \item  $P(3;~-3),~r:~y = -\dfrac{11}{12} x +9$ \hfill 
%    [$\dfrac{111}{\sqrt{265}}\approx 6.819$]
%   \item  $P(-11;~-7),~r:~y = \dfrac{3}{4} x -6$ \hfill 
%    [$\dfrac{29}{\sqrt{25}}\approx   5.8$]
%   \item  $P(1;~5),~r:~y = -\dfrac{6}{5} x -9$ \hfill 
%    [$\dfrac{152}{\sqrt{244}}\approx 9.731$]
%   \item  $P(5;~3),~r:~y = -\dfrac{5}{11} x -11$ \hfill 
%    [$\dfrac{179}{\sqrt{146}}\approx 14.81$]
%   \item  $P(-1;~10),~r:~y = 2 x -11$ \hfill 
%    [$\dfrac{23}{\sqrt{5}}\approx 10.29$]
%   \item  $P(-4;~-11),~r:~y = \dfrac{3}{4} x +6$ \hfill 
%    [$\dfrac{56}{\sqrt{25}}\approx  11.2$]
%   \item  $P(-8;~10),~r:~y = -\dfrac{2}{9} x +4$ \hfill 
%    [$\dfrac{38}{\sqrt{85}}\approx 4.122$]
%   \item  $P(-10;~-7),~r:~y = \dfrac{7}{4} x $ \hfill 
%    [$\dfrac{42}{\sqrt{65}}\approx 5.209$]
 \end{enumeratea}
\end{esercizio}

\begin{esercizio}\label{ese:}
 Per ciascuna delle seguenti terne di 
punti disegna la retta~AB e calcola la sua equazione. 
Calcola la lunghezza del segmento~AB, la distanza del punto~C dalla retta~AB 
e l'area del triangolo~$ABC$
 \begin{enumeratea}
  \item  $A(-4;~10);~B(-3;~0);~C(3;~-9)$ \hfill 
   [$y=-10 x -30;~\sqrt{101};~\dfrac{51}{\sqrt{101}};~25.5$]
  \item  $A(8;~11);~B(6;~-7);~C(7;~-7)$ \hfill 
   [$y=9 x -61;~\sqrt{328};~\dfrac{9}{\sqrt{82}};~9$]
  \item  $A(11;~2);~B(2;~7);~C(11;~-1)$ \hfill 
   [$y=-\dfrac{5}{9} x +\dfrac{73}{9},
     ~\sqrt{106};~\dfrac{27}{\sqrt{106}};~13.5$]
  \item  $A(-5;~9);~B(-8;~4);~C(9;~-5)$ \hfill 
   [$y=\dfrac{5}{3} x+\dfrac{52}{3},
     ~\sqrt{34};~\dfrac{112}{\sqrt{34}};~56$]
  \item  $A(6;~-8);~B(-10;~-6);~C(4;~-10)$ \hfill 
   [$y=-\dfrac{1}{8} x-\dfrac{29}{4},
     ~\sqrt{260};~\dfrac{18}{\sqrt{65}};~18$]
  \item  $A(3;~-6);~B(-5;~-2);~C(10;~-11)$ \hfill 
   [$y=-\dfrac{1}{2} x -\dfrac{9}{2};~\sqrt{80};~\dfrac{3}{\sqrt{5}};~6$]
  \item  $A(1;~-2);~B(3;~-11);~C(7;~-2)$ \hfill 
   [$y=-\dfrac{9}{2} x +\dfrac{5}{2};~\sqrt{85};~\dfrac{54}{\sqrt{85}};~27$]
  \item  $A(-6;~9);~B(11;~11);~C(1;~4)$ \hfill 
   [$y=\dfrac{2}{17} x+\dfrac{165}{17},
     ~\sqrt{293};~\dfrac{99}{\sqrt{293}};~49.5$]
  \item  $A(2;~1);~B(6;~1);~C(-6;~-7)$ \hfill 
   [$y=1;~\sqrt{16};~\dfrac{8}{\sqrt{1}};~16.0$]
  \item  $A(1;~-4);~B(-6;~-10);~C(7;~7)$ \hfill 
   [$y=\dfrac{6}{7} x-\dfrac{34}{7},
     ~\sqrt{85};~\dfrac{41}{\sqrt{85}};~20.5$]
  \item  $A(11;~-8);~B(-8;~9);~C(0;~-8)$ \hfill 
   [$y=-\dfrac{17}{19} x+\dfrac{35}{19},
     ~\sqrt{650};~\dfrac{187}{\sqrt{650}};~93.5$]
%   \item  $A(7;~-1);~B(-3;~-12);~C(-11;~-11)$ \hfill 
%    [$y=\dfrac{11}{10} x-\dfrac{87}{10},
%      ~\sqrt{221};~\dfrac{98}{\sqrt{221}};~49$]
%   \item  $A(-7;~-10);~B(9;~-6);~C(-8;~-3)$ \hfill 
%    [$y=\dfrac{1}{4} x-\dfrac{33}{4},
% ~\sqrt{272};~\dfrac{29}{\sqrt{17}};~58$]
%   \item  $A(-11;~0);~B(4;~-2);~C(11;~5)$ \hfill 
%    [$y=-\dfrac{2}{15} x 
% -\dfrac{22}{15};~\sqrt{229};~\dfrac{119}{\sqrt{229}};~59.5$]
%   \item  $A(-12;~-1);~B(11;~7);~C(-3;~-1)$ \hfill 
%    [$y=\dfrac{8}{23} x+\dfrac{73}{23},
% ~\sqrt{593};~\dfrac{72}{\sqrt{593}};~36$]
%   \item  $A(-10;~11);~B(9;~-5);~C(-12;~2)$ \hfill 
%    [$y=-\dfrac{16}{19} x+\dfrac{49}{19},
% ~\sqrt{617};~\dfrac{203}{\sqrt{617}};~101.5$]
%   \item  $A(6;~-12);~B(-4;~6);~C(10;~-8)$ \hfill 
%    [$y=-\dfrac{9}{5} x-\dfrac{6}{5},
% ~\sqrt{424};~\dfrac{56}{\sqrt{106}};~56$]
%   \item  $A(9;~-10);~B(0;~-6);~C(-5;~-2)$ \hfill 
%    [$y=-\dfrac{4}{9} x -6;~\sqrt{97};~\dfrac{16}{\sqrt{97}};~8$]
%   \item  $A(3;~-11);~B(-6;~4);~C(6;~2)$ \hfill 
%    [$y=-\dfrac{5}{3} x -6;~\sqrt{306};~\dfrac{54}{\sqrt{34}};~81$]
%   \item  $A(3;~-9);~B(-8;~0);~C(-9;~9)$ \hfill 
%    [$y=-\dfrac{9}{11} x-\dfrac{72}{11},
%~\sqrt{202};~\dfrac{90}{\sqrt{202}};~45$]
 \end{enumeratea}
\end{esercizio}

%\newpage
\begin{esercizio}\label{ese:}
 Disegna le due rette, individua le 
coordinate dell'intersezione, verifica che queste sono soluzioni di 
entrambe le equazioni.
 \begin{multicols}{2}
 \begin{enumeratea}
  \item  $r:~y = 4 x -8; \quad s:~y = -\dfrac{1}{4} x +9$
  \item  $r:~y = \dfrac{5}{8} x -4; \quad s:~y = \dfrac{1}{4} x -7$
  \item  $r:~y = -\dfrac{20}{3} x +9; \quad s:~y = -\dfrac{17}{3} x +6$
  \item  $r:~y = 1;\quad s:~y = 3 x -11$
  \item  $r:~y = -\dfrac{5}{2} x -4; \quad s:~y = -8 x +7$
  \item  $r:~y = 4 x -3; \quad s:~y = \dfrac{17}{3} x -8$
  \item  $r:~y = 4 x ; \quad s:~y = 16 x -12$
  \item  $r:~y = \dfrac{10}{11} x -11; \quad s:~y = -\dfrac{5}{11} x +4$
  \item  $r:~y = \dfrac{1}{2} x +7; \quad s:~y = \dfrac{11}{4} x -11$
  \item  $r:~y = \dfrac{3}{2} x +7; \quad s:~y = 4$
  \item  $r:~y = 9; \quad s:~y = \dfrac{10}{11} x -1$
  \item  $r:~y = \dfrac{5}{4} x -11; \quad s:~y = \dfrac{11}{8} x -12$
%   \item  $r:~y = \dfrac{9}{8} x ; \quad s:~y = -\dfrac{1}{4} x +11$
%   \item  $r:~y = \dfrac{9}{4} x +6; \quad s:~y = \dfrac{5}{2} x +8$
%   \item  $r:~y = -\dfrac{4}{5} x -10; \quad s:~y = \dfrac{3}{10} x +1$
%   \item  $r:~y = 2 x +9; \quad s:~y = \dfrac{17}{9} x +8$
%   \item  $r:~x = 0; \quad s:~x = 0$
%   \item  $r:~y = \dfrac{9}{5} x -7; \quad s:~y = \dfrac{11}{10} x $
%   \item  $r:~y = -\dfrac{2}{3} x ; \quad s:~y = -\dfrac{7}{9} x +1$
%   \item  $r:~y = \dfrac{1}{9} x -11; \quad s:~y = \dfrac{1}{9} x -11$
 \end{enumeratea}
 \end{multicols}
\end{esercizio}

\begin{esercizio}\label{ese:}
 Disegna le due rette e calcola le coordinate dell'intersezione.
 \begin{enumeratea}
  \item  $r:~y = -\dfrac{3}{11} x -6;\quad s:~y = -\dfrac{11}{8} x -6$ 
\hfill    [$\left(0;~-6\right)$]
  \item  $r:~y = \dfrac{5}{7} x -1;\quad s:~y = \dfrac{1}{5} x -7$ 
    \hfill    [$\left(-\dfrac{35}{3};~-\dfrac{28}{3}\right)$]
  \item  $r:~y = \dfrac{5}{12} x +6;\quad s:~y = -\dfrac{6}{7} x -8$ 
    \hfill    [$\left(-\dfrac{1176}{107};~\dfrac{152}{107}\right)$]
  \item  $r:~y = -\dfrac{11}{2} x +1;\quad s:~y = -\dfrac{5}{2} x +6$ 
\hfill    [$\left(-\dfrac{5}{3};~\dfrac{61}{6}\right)$]
  \item  $r:~y = 11 x -8;\quad s:~y = -\dfrac{7}{11} x -1$ 
    \hfill    [$\left(\dfrac{77}{128};~-\dfrac{177}{128}\right)$]
  \item  $r:~y = 4;\quad s:~y = \dfrac{6}{7} x $ \hfill 
   [$\left(\dfrac{14}{3};~4\right)$]
  \item  $r:~y = \dfrac{5}{2} x -11;\quad s:~y = -\dfrac{11}{7} x +9$ 
\hfill    [$\left(\dfrac{280}{57};~\dfrac{73}{57}\right)$]
  \item  $r:~y = -\dfrac{9}{2} x -5;\quad s:~y = \dfrac{5}{6} x +11$ 
    \hfill    [$\left(-3;~\dfrac{17}{2}\right)$]
  \item  $r:~y = \dfrac{7}{9} x +1;\quad s:~y = \dfrac{8}{3} x +12$ 
    \hfill    [$\left(-\dfrac{99}{17};~-\dfrac{60}{17}\right)$]
  \item  $r:~y = -\dfrac{1}{3} x +2;\quad s:~y = 2 x +12$ 
    \hfill    [$\left(-\dfrac{30}{7};~\dfrac{24}{7}\right)$]
  \item  $r:~y = -7 x +1;\quad s:~y = -5 x +11$ 
    \hfill    [$\left(-5;~36\right)$]
%   \item  $r:~y = 6 x -5;\quad s:~y = 8 x -1$ 
%     \hfill 
%    [$\left(-2;~-17\right)$]
%   \item  $r:~y = 3 x +11;\quad s:~y = -\dfrac{7}{2} x +1$ 
%     \hfill 
%    [$\left(-\dfrac{20}{13};~\dfrac{83}{13}\right)$]
%   \item  $r:~y = \dfrac{2}{9} x -4;\quad s:~y = \dfrac{1}{3} x -4$ 
%     \hfill 
%    [$\left(0;~-4\right)$]
%   \item  $r:~y = -\dfrac{2}{5} x -9;\quad s:~y = \dfrac{4}{5} x +7$ 
%     \hfill 
%    [$\left(-\dfrac{40}{3};~-\dfrac{11}{3}\right)$]
%   \item  $r:~y = \dfrac{12}{7} x +7;\quad s:~y = \dfrac{1}{9} x $ 
%     \hfill 
%    [$\left(-\dfrac{441}{101};~-\dfrac{49}{101}\right)$]
%   \item  $r:~y = 2 x +8;\quad s:~y = \dfrac{1}{2} x -9$ 
%     \hfill 
%    [$\left(-\dfrac{34}{3};~-\dfrac{44}{3}\right)$]
%   \item  $r:~y = -9 x ;\quad s:~y = -\dfrac{11}{10} x +5$ 
%     \hfill 
%    [$\left(-\dfrac{50}{79};~\dfrac{450}{79}\right)$]
%   \item  $r:~y = \dfrac{5}{8} x -10;\quad s:~y = -\dfrac{1}{5} x -2$ 
%     \hfill 
%    [$\left(\dfrac{320}{33};~-\dfrac{130}{33}\right)$]
%   \item  $r:~y = -\dfrac{8}{11} x +4;\quad s:~y = - x +4$ 
%     \hfill 
%    [$\left(0;~4\right)$]
 \end{enumeratea}
\end{esercizio}

\begin{esercizio}\label{ese:}
 Disegna le due rette e calcola le coordinate dell'intersezione.
 \begin{enumeratea}
  \item  $r:~4 x + 7 y - 63 = 0; s:~-9 x - 10 y + 110 = 0$ \hfill 
   [$\left(\dfrac{140}{23};~\dfrac{127}{23}\right)$]
  \item  $r:~-7 x + 0 = 0; s:~-6 x + 6 y + 60 = 0$ \hfill 
   [$\left(0;~-10\right)$]
  \item  $r:~8 x + 0 = 0; s:~-10 x - 12 y + 72 = 0$ \hfill 
   [$\left(0;~6\right)$]
  \item  $r:~-12 x - 10 y - 60 = 0; s:~6 x - 5 y + 10 = 0$ \hfill 
   [$\left(-\dfrac{10}{3};~-2\right)$]
  \item  $r:~4 x - 8 y + 24 = 0; s:~-1 x + 10 y + 50 = 0$ \hfill 
   [$\left(-20;~-7\right)$]
  \item  $r:~-9 x + 2 y - 2 = 0; s:~-1 x + 10 y + 30 = 0$ \hfill 
   [$\left(-\dfrac{10}{11};~-\dfrac{34}{11}\right)$]
  \item  $r:~7 x + 0 = 0; s:~-3 x + 10 y - 20 = 0$ \hfill 
   [$\left(0;~2\right)$]
  \item  $r:~-1 x - 12 y + 12 = 0; s:~-2 y + 18 = 0$ \hfill 
   [$\left(-96;~9\right)$]
  \item  $r:~-1 x + 3 y + 30 = 0; s:~11 x - 9 y - 72 = 0$ \hfill 
   [$\left(-\dfrac{9}{4};~-\dfrac{43}{4}\right)$]
  \item  $r:~11 x - 1 y + 11 = 0; s:~-7 x - 8 y - 8 = 0$ \hfill 
   [$\left(-\dfrac{96}{95};~-\dfrac{11}{95}\right)$]
%   \item  $r:~-6 x - 10 y + 30 = 0; s:~5 x + 9 y - 45 = 0$ \hfill 
%    [$\left(-45;~30\right)$]
%   \item  $r:~7 x - 9 y + 63 = 0; s:~-2 x - 12 y - 120 = 0$ \hfill 
%    [$\left(-18;~-7\right)$]
%   \item  $r:~-10 x + 9 y + 72 = 0; s:~7 x + 1 y + 6 = 0$ \hfill 
%    [$\left(\dfrac{18}{73};~-\dfrac{564}{73}\right)$]
%   \item  $r:~-5 x + 2 y + 12 = 0; s:~-7 x - 4 y - 24 = 0$ \hfill 
%    [$\left(0;~-6\right)$]
%   \item  $r:~-10 x - 10 y - 40 = 0; s:~-10 x - 4 y - 20 = 0$ \hfill 
%    [$\left(-\dfrac{2}{3};~-\dfrac{10}{3}\right)$]
%   \item  $r:~-11 y - 99 = 0; s:~-10 x + 1 y + 5 = 0$ \hfill 
%    [$\left(-\dfrac{2}{5};~-9\right)$]
%   \item  $r:~8 x - 7 y - 77 = 0; s:~5 x - 9 y - 99 = 0$ \hfill 
%    [$\left(0;~-11\right)$]
% %   \item  $r:~9 x + 9 y + 54 = 0; s:~10 x + 3 y + 33 = 0$ \hfill 
% %    [$\left(-\dfrac{15}{7};~-\dfrac{27}{7}\right)$]
%   \item  $r:~-1 x + 10 y - 100 = 0; s:~-6 y + 66 = 0$ \hfill 
%    [$\left(10;~11\right)$]
%   \item  $r:~-11 x - 9 y + 0 = 0; s:~-6 x - 5 y + 10 = 0$ \hfill 
%    [$\left(-90;~110\right)$]
 \end{enumeratea}
\end{esercizio}


\subsection{Esercizi riepilogativi}

% Da Vincenzo Gentile
\begin{esercizio}\label{ese:02_01.}
Determina il circocentro, l'ortocentro, il baricentro, il perimetro e l'area 
del triangolo avente per vertici i punti~$A(-1;~-1), B(2;~-1 ), C(0;~3$).  
\end{esercizio}

\begin{esercizio}\label{ese:02_01.}
Determina la proiezione ortogonale del punto~$P(-1;~-4)$ sulla 
retta~$y = - \dfrac{1}{5} x - 1$ 
\end{esercizio}

\begin{esercizio}\label{ese:02_01.}
Dati i tre punti~$A(1;~3)$, $B(-1;~6)$, $C(-4;~4)$ 
determina il punto~$D$ in modo tale che il quadrilatero~$ABCD$ risulti essere 
un quadrato. 
(Suggerimento: ci sono due metodi per risolvere l'esercizio, uno è molto 
veloce...) 
\end{esercizio}

\begin{esercizio}\label{ese:02_01.}
Verifica che il triangolo di vertici~$A(3;~2)$, $B(2;~5)$, $C(-4;~3)$ è 
rettangolo e calcola l'area. \hfill[10]
\end{esercizio}

\begin{esercizio}\label{ese:02_01.}
Nel fascio di rette di centro~$A(-2;~1)$ determinare la retta~$r$ 
perpendicolare alla retta di equazione~$2x - 2y - 3 = 0$ \hfill[x + y + 1 = 0]
\end{esercizio}

\begin{esercizio}\label{ese:02_01.}
Nel fascio di rette parallele a~$y = -2x$ determinare la retta r 
passante per~$A(0;~-3)$ \hfill[2x + y + 3 = 0]
\end{esercizio}

\begin{esercizio}\label{ese:02_01.}
Dati i tre vertici di un triangolo~$A(5;~0)$ $B(1;~2)$ e $C(-3;~2)$, 
scriverne le equazioni dei lati.
 \hfill[$x + 2y - 5 = 0$  $x + 4 y - 5 = 0$ $y = 2$]
\end{esercizio}

\begin{esercizio}\label{ese:02_01.}
Scrivere l'equazione di una retta passante per~$A(4;~2)$ 
e per il punto comune alle rette~$r:~x + y = 3$ e~$s:~x - y + 1 = 0$
\hfill[$y = 2$]
\end{esercizio}

\begin{esercizio}\label{ese:02_01.}
Scrivere l'equazione della retta congiungente il punto d'intersezione delle 
rette~$a:~x + y = 3$ e~$b:~x - y + 1 = 0$, con quello d'intersezione delle 
rette~$c:~x - y = 1$ e~$d~x = -1$
 \hfill[$y = 2x$]
\end{esercizio}

\begin{esercizio}\label{ese:02_01.}
Scrivere l'equazione della retta passante per~$A(-5;~-1)$ parallela alla retta 
congiungente l'origine delle coordinate con~$B(1;~2)$
 \hfill[$2x - y + 9 = 0$]
\end{esercizio}

\begin{esercizio}\label{ese:02_01.}
La retta passante per~$A(2;~3)$ e~$B(-1;~-6)$ e quella per~$C(6;~-1)$
e~$D(-3;~2)$ come sono fra loro?
 \hfill[perpendicolari]
\end{esercizio}

% \begin{esercizio}\label{ese:02_01.}
% problema 30 pag 18 \hfill[]
% \end{esercizio}
% 
% \begin{esercizio}\label{ese:02_01.}
%  \hfill[]
% \end{esercizio}


