% (c) 2014 Daniele Zambelli - daniele.zambelli@gmail.com
% 
% Tutti i grafici per il capitolo relativo alle parabole
%


\newcommand{\parabolafuocoedirettrice}{% 
    % Parabola come luogo di punti.
  \disegno{
    \rcom{-3}{+12}{-5}{+7}{gray!50, very thin, step=1}
    \tkzInit[xmin=-3.3,xmax=+12.3,ymin=-5.3,ymax=+7.3]
    \tkzFct[domain=-5.3:+12.3, ultra thick, color=Maroon!50!black]
           {.25*x*x-2*x+1}
    \foreach \pi/\pl in {(+4, -3)/V, (+4, -2)/F, 
                         (+8, +1)/P_1, (+8, -4)/H_1, 
                         (+10, +6)/P_2, (+10, -4)/H_2}
    \filldraw [blue!50!black] \pi circle (2pt) 
              node [below right] {\(\pl\)};
    \draw [green!50!black, ultra thick] (-3.3, -4) 
          node [below right] {\(d\)}
          -- (+12.3, -4);
    \draw [blue!50!black, thick, dashed] (+4, -2) -- (+8, +1) -- (+8, -4);
    \draw [blue!50!black, thick, dashed] (+4, -2) -- (+10, +6) -- (+10, -4);
  }
}

\newcommand{\parabolina}[2]{% 
  \def \eq{#1}
  \def \paracaption{#2}
    % Parabola con a positivo.
  \disegno{
    \rcom{-5}{+5}{-5}{5}{gray!50, very thin, step=1}
    \tkzInit[xmin=-5.3,xmax=+5.3,ymin=-5.3,ymax=+5.3]
    \tkzFct[domain=-5:+5, ultra thick, color=Maroon!50!black]{\eq}
    \node [black] at (0, -6) {\paracaption};
  }
}


\newcommand{\puntia}{% 
  % Alcuni punti di una parabola.
  \disegno{
    \rcom{-4}{+6}{-4}{13}{gray!50, very thin, step=1}
    \foreach \pi in {
    % (-4, 22), 
    (-3, 13), (-2, 6), (-1, 1), 
    (0, -2), (1, -3), (2, -2), (3, 1), (4, 6), (5, 13)}
    \filldraw [Maroon!50!black] \pi circle (1.5pt);
  }
}

\newcommand{\puntib}{% 
    % Altri punti di una parabola.
    \disegno{
    \rcom{-4}{+6}{-4}{13}{gray!50, very thin, step=1}
    \foreach \pi in {
    (-3, 13), (-2, 6), (-1, 1), (0, -2), (1, -3), (2, -2), (3, 1), 
    (4, 6), (5, 13),
    (-2.5, 9.25), (-1.5, 3.25), (-0.5, -0.75), (0.5, -2.75), 
    (1.5, -2.75), (2.5, -0.75), (3.5, 3.25), (4.5, 9.25) 
    }
    \filldraw [Maroon!50!black] \pi circle (1.5pt);
    }
}

\newcommand{\graficotrinomio}{% 
    % Grafico di un trinomio di secondo grado.
    \disegno{
    \rcom{-4}{+6}{-4}{13}{gray!50, very thin, step=1}
    \tkzInit[xmin=-4.3,xmax=+6.3,ymin=-4.3,ymax=+13.3]
    \tkzFct[domain=-4:+6, ultra thick, color=Maroon!50!black]{x*x-2*x-2}
    }
}

\newcommand{\coefficientea}{% 
    % Parabole con diversi coefficienti del termine di 2° grado.
    \disegno{
    \rcom{-10}{+10}{-10}{10}{gray!50, very thin, step=1}
    \begin{scope}[ultra thick, color=Maroon!50!black]
    \tkzInit[xmin=-10.3,xmax=+10.3,ymin=-10.3,ymax=+10.3]
    \tkzFct[domain=-10:+10]{2*x*x+x+2}
    \tkzFct[domain=-10:+10]{0.5*x*x+x+2}
    \tkzFct[domain=-10:+10]{0.1*x*x+x+2}
    \tkzFct[domain=-10:+10]{0.01*x*x+x+2}
    \begin{scope}[color=Green!50!black]
    \tkzFct[domain=-10:+10]{-0.01*x*x+x+2}
    \tkzFct[domain=-10:+10]{-0.1*x*x+x+2}
    \tkzFct[domain=-10:+10]{-0.5*x*x+x+2}
    \tkzFct[domain=-10:+10]{-2*x*x+x+2}
    \end{scope}
    \end{scope}
    \begin{scope}[color=black]
    \draw (-1.5, 9.5) node {a} (-4, 8.5) node {b} (-9, 2) node {c} 
        (-9, -5.5) node {d} (-8, -7.5) node {e} (-5.6, -8.5) node {f}
        (-3.5, -9.5) node {g} (-1.5, -9.5) node {h};
    \end{scope}
    }
}

\newcommand{\quadrati}{% 
    % Numeri quadrati
    \disegno{
    \draw[gray!50, very thin, step=1] (-1, 1) grid (27, 7); % Griglia
    \draw (-.5, 0) node{0};
    \foreach \n in {1, ..., 4} {
        \pgfmathparse{\n * \n / 2 + \n * 3 / 2}
        \xdef\corner{\pgfmathresult}
        \fill[fill=green,fill opacity=0.3] (\corner, \n + 1) --
                                            (\corner - \n, \n + 1) --
                                            (\corner - \n, \n) --
                                            (\corner - 1, \n) --
                                            (\corner - 1, 1) --
                                            (\corner, 1) -- cycle; 
        \foreach \j in {1, ..., \n} {
            \foreach \i in {1, ..., \n} { 
                \fill[blue!40!white] (\n * \n / 2 + \n / 2 -.5 + \i, \j+.5) 
                                    circle (.4);
            }
        }
    }
    \foreach \n in {1, ..., 6} {
        \pgfmathparse{int(\n * \n)}
        \xdef\square{\pgfmathresult}
        \draw (\n * \n / 2 + \n  , 0) node{\square};
    }
    \draw (.5, -1) node{1} (2.6  , -1) node{3} (5.9  , -1) node{5}
        (10, -1.3) node{\ldots} (15, -1.3) node{\ldots}
        (21, -1.3) node{\ldots} (26, -1.3) node{\ldots};
    }
}

\newcommand{\disrapido}{% 
    % Metodo rapido per il disegno di parabole.
    \disegno{
    \tkzInit[xmin=-5.5,xmax=+5.5,ymin=-5.5,ymax=+15.5]

    \clip (-5.3, -1.3) rectangle (5.7, 13.8);
    \rcom{-5}{+5}{-1}{13}{gray!50, very thin, step=1}

    \coordinate (a0) at (0, 0); \coordinate (a0p) at (1, 0); 
    \coordinate (a1) at (1, 1); \coordinate (a1p) at (2, 1);
    \coordinate (a2) at (2, 4); \coordinate (a2p) at (3, 4);
    \coordinate (a3) at (3, 9);
    \coordinate (i0m) at (-1, 0);
    \coordinate (i1) at (-1, 1); \coordinate (i1m) at (-2, 1);
    \coordinate (i2) at (-2, 4); \coordinate (i2m) at (-3, 4);
    \coordinate (i3) at (-3, 9);

    \begin{scope}[red!50!black]
    % Senza etichetta
    \foreach \p in {a0, a1, a2, a3, i1, i2, i3}
    \filldraw (\p) circle (1.5pt);

    \tkzFct[domain=-5:+5, ultra thick]{x*x}

    \draw (i3) 
    .. controls +(-.5, 0) and +(-.5, 0) .. ++(0, -1)
    .. controls +(-.5, 0) and +(-.5, 0) .. ++(0, -1)
    .. controls +(-.5, 0) and +(-.5, 0) .. ++(0, -1)
    .. controls +(-.5, 0) and +(-.5, 0) .. ++(0, -1)
    .. controls +(-.5, 0) and +(-.5, 0) .. ++(0, -1)
    .. controls +(0, -.5) and +(0, -.5) .. ++(+1, 0)
    .. controls +(-.5, 0) and +(-.5, 0) .. ++(0, -1)
    .. controls +(-.5, 0) and +(-.5, 0) .. ++(0, -1)
    .. controls +(-.5, 0) and +(-.5, 0) .. ++(0, -1)
    .. controls +(0, -.5) and +(0, -.5) .. ++(+1, 0)
    .. controls +(-.5, 0) and +(-.5, 0) .. ++(0, -1)
    .. controls +(0, -.5) and +(0, -.5) .. ++(+1, 0)
    
    .. controls +(0, -.5) and +(0, -.5) .. ++(+1, 0)
    .. controls +(+.5, 0) and +(+.5, 0) .. ++(0, +1)
    .. controls +(0, -.5) and +(0, -.5) .. ++(+1, 0)
    .. controls +(+.5, 0) and +(+.5, 0) .. ++(0, +1)
    .. controls +(+.5, 0) and +(+.5, 0) .. ++(0, +1)
    .. controls +(+.5, 0) and +(+.5, 0) .. ++(0, +1)
    .. controls +(0, -.5) and +(0, -.5) .. ++(+1, 0)
    .. controls +(+.5, 0) and +(+.5, 0) .. ++(0, +1)
    .. controls +(+.5, 0) and +(+.5, 0) .. ++(0, +1)
    .. controls +(+.5, 0) and +(+.5, 0) .. ++(0, +1)
    .. controls +(+.5, 0) and +(+.5, 0) .. ++(0, +1)
    .. controls +(+.5, 0) and +(+.5, 0) .. ++(0, +1);

    \draw node [below, scale=.7, yshift=-7pt] at ($ (a0)!.5!(i0m) $) 
{$-1$};
    \draw node [left, scale=.7, xshift=+10pt] at ($ (i1)!.5!(i0m) $) 
{$+1$};
    \draw node [below, scale=.7, yshift=-7pt] at ($ (i2m)!.5!(i2m) $) 
{$-1$};
    \draw node [left, scale=.7, xshift=+10pt] at ($ (i2)!.5!(i1m) $) 
{$+3$};
    \draw node [below, scale=.7, yshift=-7pt] at ($ (i1m)!.5!(i1m) $) 
{$-1$};
    \draw node [left, scale=.7, xshift=+10pt] at ($ (i3)!.5!(i2m) $) 
{$+5$};

    \draw node [below, scale=.7, yshift=-7pt] at ($ (a0)!.5!(a0p) $) 
{$+1$};
    \draw node [left, scale=.7, xshift=+9pt] at ($ (a1)!.5!(a0p) $) 
{$+1$};
    \draw node [below, scale=.7, yshift=-7pt] at ($ (a2p)!.5!(a2p) $) 
{$+1$};
    \draw node [left, scale=.7, xshift=+9pt] at ($ (a2)!.5!(a1p) $) 
{$+3$};
    \draw node [below, scale=.7, yshift=-7pt] at ($ (a1p)!.5!(a1p) $) 
{$+1$};
    \draw node [left, scale=.7, xshift=+9pt] at ($ (a3)!.5!(a2p) $) 
{$+5$};

    \end{scope}
    }
}

% \newcommand{\disrapido}{% 
%     % Metodo rapido per il disegno di parabole.
%     \disegno{
%     \tkzInit[xmin=-5.5,xmax=+5.5,ymin=-5.5,ymax=+15.5]
% 
%     \clip (-5.3, -1.3) rectangle (5.7, 13.8);
%     \rcom{-5}{+5}{-1}{13}{gray!50, very thin, step=1}
% 
%     \coordinate (a0) at (0, 0); \coordinate (a0p) at (1, 0); 
%     \coordinate (a1) at (1, 1); \coordinate (a1p) at (2, 1);
%     \coordinate (a2) at (2, 4); \coordinate (a2p) at (3, 4);
%     \coordinate (a3) at (3, 9);
%     \coordinate (i0m) at (-1, 0);
%     \coordinate (i1) at (-1, 1); \coordinate (i1m) at (-2, 1);
%     \coordinate (i2) at (-2, 4); \coordinate (i2m) at (-3, 4);
%     \coordinate (i3) at (-3, 9);
% 
%     \begin{scope}[red!50!black]
%     % Senza etichetta
%     \foreach \p in {a0, a1, a2, a3, i1, i2, i3}
%     \filldraw (\p) circle (1.5pt);
% 
%     \tkzFct[domain=-5:+5, ultra thick]{x*x}
% 
%     \draw (i3) 
%     .. controls +(-.5, 0) and +(-.5, 0) .. ++(0, -1)
%     .. controls +(-.5, 0) and +(-.5, 0) .. ++(0, -1)
%     .. controls +(-.5, 0) and +(-.5, 0) .. ++(0, -1)
%     .. controls +(-.5, 0) and +(-.5, 0) .. ++(0, -1)
%     .. controls +(-.5, 0) and +(-.5, 0) .. ++(0, -1)
%     .. controls +(0, -.5) and +(0, -.5) .. ++(+1, 0)
%     .. controls +(-.5, 0) and +(-.5, 0) .. ++(0, -1)
%     .. controls +(-.5, 0) and +(-.5, 0) .. ++(0, -1)
%     .. controls +(-.5, 0) and +(-.5, 0) .. ++(0, -1)
%     .. controls +(0, -.5) and +(0, -.5) .. ++(+1, 0)
%     .. controls +(-.5, 0) and +(-.5, 0) .. ++(0, -1)
%     .. controls +(0, -.5) and +(0, -.5) .. ++(+1, 0)
%     
%     .. controls +(0, -.5) and +(0, -.5) .. ++(+1, 0)
%     .. controls +(+.5, 0) and +(+.5, 0) .. ++(0, +1)
%     .. controls +(0, -.5) and +(0, -.5) .. ++(+1, 0)
%     .. controls +(+.5, 0) and +(+.5, 0) .. ++(0, +1)
%     .. controls +(+.5, 0) and +(+.5, 0) .. ++(0, +1)
%     .. controls +(+.5, 0) and +(+.5, 0) .. ++(0, +1)
%     .. controls +(0, -.5) and +(0, -.5) .. ++(+1, 0)
%     .. controls +(+.5, 0) and +(+.5, 0) .. ++(0, +1)
%     .. controls +(+.5, 0) and +(+.5, 0) .. ++(0, +1)
%     .. controls +(+.5, 0) and +(+.5, 0) .. ++(0, +1)
%     .. controls +(+.5, 0) and +(+.5, 0) .. ++(0, +1)
%     .. controls +(+.5, 0) and +(+.5, 0) .. ++(0, +1);
% 
%     \draw node [below, scale=.7, yshift=-7pt] at (\( (a0)!.5!(i0m) \)) 
% {\(-1\)};
%     \draw node [left, scale=.7, xshift=+10pt] at (\( (i1)!.5!(i0m) \)) 
% {\(+1\)};
%     \draw node [below, scale=.7, yshift=-7pt] at (\( (i2m)!.5!(i2m) \)) 
% {\(-1\)};
%     \draw node [left, scale=.7, xshift=+10pt] at (\( (i2)!.5!(i1m) \)) 
% {\(+3\)};
%     \draw node [below, scale=.7, yshift=-7pt] at (\( (i1m)!.5!(i1m) \)) 
% {\(-1\)};
%     \draw node [left, scale=.7, xshift=+10pt] at (\( (i3)!.5!(i2m) \)) 
% {\(+5\)};
% 
%     \draw node [below, scale=.7, yshift=-7pt] at (\( (a0)!.5!(a0p) \)) 
% {\(+1\)};
%     \draw node [left, scale=.7, xshift=+9pt] at (\( (a1)!.5!(a0p) \)) 
% {\(+1\)};
%     \draw node [below, scale=.7, yshift=-7pt] at (\( (a2p)!.5!(a2p) \)) 
% {\(+1\)};
%     \draw node [left, scale=.7, xshift=+9pt] at (\( (a2)!.5!(a1p) \)) 
% {\(+3\)};
%     \draw node [below, scale=.7, yshift=-7pt] at (\( (a1p)!.5!(a1p) \)) 
% {\(+1\)};
%     \draw node [left, scale=.7, xshift=+9pt] at (\( (a3)!.5!(a2p) \)) 
% {\(+5\)};
% 
%     \end{scope}
%     }
% }

\newcommand{\piancart}{% 
    % Piano cartesiano  per il disegno di alcune parabole
    \disegno{
    \rcom{-10}{+10}{-10}{+10}{gray!50, very thin, step=1}

    % \tkzInit[xmin=-10.5,xmax=+10.5,ymin=-10.5,ymax=+10.5]
    % \tkzFct[domain=-10:+10, ultra thick, color=Maroon!50!black]{-x*x}
    % \tkzFct[domain=-10:+10, ultra thick, color=Maroon!50!black]{-0.5*x*x+2*x+8}
    % \tkzFct[domain=-10:+10, ultra thick, color=Maroon!50!black]{2*x*x+24*x+64}
    }
}

\newcommand{\parabolaerette}{% 
    % Una parabola e tre rette nel piano cartesiano.
    \disegno{
    \rcom{-7}{+5}{-7}{+10}{gray!50, very thin, step=1}

    \tkzInit[xmin=-7.3,xmax=+5.7,ymin=-7.5,ymax=+10.5]

    \begin{scope}[color=Red!50!black]
    \tkzFct[domain=-10:+10, ultra thick]{-x*x+4}
    \node at (2.5, -5.3) {p};
    \end{scope}
    \begin{scope}[color=Blue!50!black]
    \tkzFct[domain=-10:+10, ultra thick]{2*x+9}
    \tkzFct[domain=-10:+10, ultra thick]{2*x+5}
    \tkzFct[domain=-10:+10, ultra thick]{2*x+1}
    \node at (-6.2, -2.3) {r};
    \node at (-5.5, -5.3) {t};
    \node at (-4.2, -6.5) {s};
    \end{scope}
    }
}

\newcommand{\parabolaetangenti}{% 
    % Parabola e tangenti.
    \disegno{
    \rcom{-5}{+10}{-6}{+11}{gray!50, very thin, step=1}
    \tkzInit[xmin=-5.3,xmax=+10.3,ymin=-6.3,ymax=+11.3]
    \tkzFct[domain=-10:+10.3, ultra thick, color=Blue!50!black]{-1./4.*x*x+2*x+1}
    \tkzFct[domain=-10:+10.3, ultra thick, color=Green!50!black]{x+2}
    \tkzFct[domain=-10:+10.3, ultra thick, color=Green!50!black]{3*x+2}
    \tkzFct[domain=-10:+10.3, ultra thick, color=Red!50!black]{-2.*x+17}

    \begin{scope}[color=Green!50!black]
    \coordinate (a) at (0, +2);
    \filldraw  (a) circle (1.5pt); 
    \node at (a) [xshift=-9pt] {\(A\)};
    \end{scope}

    \begin{scope}[color=Red!50!black]
    \coordinate (b) at (+8, +1);
    \filldraw (b) circle (1.5pt); 
    \node at (b) [xshift=+7pt] {\(B\)};
    \end{scope}

    \begin{scope}[color=Blue!50!black]
    \coordinate (b) at (+5, -2);
    \filldraw (b) circle (1.5pt); 
    \node at (b) [xshift=+7pt] {\(C\)};
    \filldraw (+2, +4) circle (1.5pt); 
    \filldraw (-2, -4) circle (1.5pt); 
    \end{scope}
    }
}

\newcommand{\intersezioniparabole}{% 
    % Intersezione tra due parabole.
    \disegno{
    \rcom{-3}{+5}{-6}{+10}{gray!50, very thin, step=1}
    \tkzInit[xmin=-3.3,xmax=+5.3,ymin=-6.3,ymax=+10.3]
    \tkzFct[domain=-3.3:+5.3, ultra thick, color=Blue!50!black]{-x*x+2*x+8}
    \tkzFct[domain=-3.3:+5.3, ultra thick, color=Red!50!black]{5./4.*x*x-19./4.*x-1}
    \filldraw (-1, +5) circle (1.5pt) node [xshift=-9pt] {\(I_0\)};
    \filldraw (+4, 0) circle (1.5pt) 
        node [xshift=9pt, yshift=6pt] {\(I_1\)};
    }
}

\newcommand{\parabolapertrepunti}{% 
    % Parabola per tre punti.
    \disegno{
    \rcom{-5}{+5}{-9}{+10}{gray!50, very thin, step=1}
    \tkzInit[xmin=-5.3,xmax=+5.3,ymin=-9.3,ymax=+10.3]
    \tkzFct[domain=-10:+10, ultra thick, color=Blue!50!black]{1./2.*x*x+2*x-7}
    \filldraw (-4, -7) circle (1.5pt) node [xshift=-9pt] {\(P_0\)};
    \filldraw (+2, -1) circle (1.5pt) node [xshift=9pt] {\(P_1\)};
    \filldraw (+4, +9) circle (1.5pt) node [xshift=9pt] {\(P_2\)};
    }
}

\newcommand{\parabolaverticepunto}{% 
    % Equazione della parabola dati vertice e punto.
    \disegno{
    \rcom{-1}{+10}{-6}{+8}{gray!50, very thin, step=1}
    \tkzInit[xmin=-1.3,xmax=+10.3,ymin=-6.3,ymax=+8.3]
    \tkzFct[domain=-10:+10, ultra thick, color=Blue!50!black]{-1./3.*x*x+2*x+4}
    \filldraw (+3, +7) circle (1.5pt) node [yshift=9pt] {\(V\)};
    \filldraw (+9, -5) circle (1.5pt) node [xshift=9pt] {\(P\)};
    }
}
