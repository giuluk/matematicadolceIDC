% Created by Rigo Diego

\newcommand{\vettcol}[3]{\begin{pmatrix}
#1\\ #2\\ #3
\end{pmatrix}}


\chapter{Geometria cartesiana dello spazio}

\section{Punti e vettori}
\label{sec:Punti_e_vettori}

Abbiamo già imparato in precedenza quali tecniche e formule si possono utilizzare per manipolare oggetti nel piano cartesiano bidimensionale. Andiamo ora a studiare l'analogo caso tridimensionale. 

\vspace{7pt}

Il piano cartesiano è formato da 3 assi tra loro perpendicolari, che indicano le 3 direzioni possibili di movimento per un oggetto nello spazio. Un punto viene perciò individuato dalle 3 coordinate spaziali, e lo indicheremo perciò con la terna ordinata P\,$(x,y,z)$. La visualizzazione dei punti non è semplice, poiché deve rendere conto della profondità e, in certi casi, della prospettiva. Quindi ci affideremo per quanto possibile all'intuizione geometrica, senza legarci troppo ai disegni.

\vspace{7pt}

Dato un insieme di punti distinti, è possibile innanzitutto calcolarne le rispettive distanze. La formula per calcolare la \textbf{distanza tra 2 punti} non è altro che una generalizzazione del teorema di Pitagora, ovvero:
\[\overline{AB} = \sqrt{(x_A-x_B)^2+(y_A-y_B)^2+(z_A-z_B)^2}\]
Un'oggetto interessante, che ci permetterà poi di ricavare l'equazione di una retta e di un piano, è il vettore. Chiamiamo \textbf{vettore} una freccia che connette due punti. In genere, dati 2 punti A e B, indichiamo con $\vect{\text{AB}}$ il vettore che congiunge A con B, ovvero la freccia che parte da A e arriva a B, con la punta rivolta verso B. Chiamiamo \emph{coda} e \emph{punta} rispettivamente i punti A e B. 

\vspace{7pt}

Spesso si pensa un vettore come lo spostamento di un oggetto. Si può pensare per esempio che il punto B sia dato dallo spostamento di A lungo la freccia $\vect{\text{AB}}$. Ovvero:
\[\text{B} = \text{A}+\vect{\text{AB}} \quad \Rightarrow \quad \vect{\text{AB}} = \text{B}-\text{A}\]
Quindi per "costruire" un vettore di coda A e punta B è sufficiente calcolare la differenza tra le coordinate dei punti.
\begin{esempio}
\emph{Calcola il vettore $\vect{CF}$, dove $C\,(4,-2)$ e $F \, (-6,1)$}
\[\vect{CF} = F-C = (-6,1)-(4,-2) = (-10,3)\]
\end{esempio}

Notiamo che due vettori con stessa direzione e stesso verso (ovvero sovrapponibili mediante traslazione) sono indistinguibili:
\begin{esempio}
\emph{Calcola i vettori $\vect{AB}$ e $\vect{CD}$, dove $A\,(0,0)$, $B \, (3,5)$, $C \, (4,0)$ e $D \, (7,5)$}
\[\vect{AB} = (3,5)-(0,0) = (3,5) \qquad;\qquad \vect{CD} = (7,5)-(4,0) = (3,5)\]
\end{esempio}
Questa proprietà è detta \textbf{equipollenza}. Due vettori sono equipollenti se hanno stessa direzione e verso. Tale proprietà ci permette di pensare un vettore applicato in un qualunque punto (ovvero con la coda posizionata in tale punto) ed in particolare possiamo sempre pensarlo applicato nell'origine.

\section{Operazioni con i vettori}

Per le principali operazioni tra vettori rimandiamo al relativo capitolo del libro 3. Richiamiamo qui invece le definizioni di parallelismo e perpendicolarità:
\begin{itemize}
 \item Due vettori $ \vect{a} = (a_x,a_y,a_z) \;,\; \vect{b} = (b_x,b_y,b_z)$ sono \textbf{paralleli} se il rapporto tra i rispettivi coefficienti è sempre lo uguale (nel caso in cui una coordinata del primo vettore sia nulla, dev'esserlo anche nel secondo vettore). In formule:
\[\vect{a} \parallel \vect{b} \quad (a_j\,, b_j \neq 0) \quad \Longleftrightarrow \quad \frac{a_x}{b_x} = \frac{a_y}{b_y} = \frac{a_z}{b_z}\]
\item Due vettori sono \textbf{perpendicolari} se il corrispondente prodotto scalare è nullo. In formule:
\[ \vect{a} \perp \vect{b} \quad \Longleftrightarrow\quad \vect{a} \cdot \vect{b} = 0 \quad \text{ovvero:} \quad a_xb_x+a_yb_y+a_zb_z=0\]
\end{itemize}
\begin{esempio}
 \emph{Riconoscere quali vettori sono tra loro paralleli e quali perpendicolari:}
 \[\vect{a} = (5,2,1) \quad;\quad \vect{b} = (2,-3,-4) \quad;\quad \vect{c} = (1,-2,-1)  \quad;\quad \vect{d} = (10,4,2)\]
 Prendiamo i primi due vettori: $\vect{a}$ e $\vect{b}$. Verifichiamo se sono paralleli:
 \[\frac{5}{2} \neq \frac{2}{-3} \neq \frac{1}{-4} \qquad \Rightarrow \qquad \text{Non sono paralleli}\]
 Controlliamo ora se sono perpendicolari:
 \[\vect{a} \cdot \vect{b} = (5,2,1)\times (2,-3,-4) = 5\cdot 2+2\cdot (-3)+1 \cdot (-4) = 10-6-4=0\]
 Quindi $\vect{a}$ e $\vect{b}$ sono tra loro perpendicolari. \'E facile vedere ora che $\vect{d}$ è parallelo ad $\vect{a}$ e che $\vect{c}$ è perpendicolare a $\vect{a}$ ma non parallelo a $\vect{b}$.
\end{esempio}


\section{Retta (forma parametrica)}

Come si definisce una retta nel piano cartesiano? Ci sono vari metodi, più o meno complicati. Uno dei modi più semplici per definirla si basa su un approccio pratico: per disegnare una qualunque retta si segna un punto, col righello si imposta l'inclinazione e quindi si traccia la linea. Quindi una retta è definita come l'insieme di tutti i quei punti che si ottengono spostandosi da un punto fissato P\,$(x_P,y_P,z_P)$ lungo la direzione $\vect{d}\,(d_x,d_y,d_z)$. In altre parole:
\[\boxed{\vect{r} = P+t \, \vect{d}} \quad \Rightarrow \quad \begin{cases}
x = x_P+t d_x \\
y = y_P+t d_y \\
z = z_P+t d_z
\end{cases}  \quad (t \in \mathbb{R})\]
\begin{esempio}
\label{rettaPuntoDirez}
\emph{Determina la retta passante per $P\,(4,-2,1)$, avente direzione $\vect{d} \,(-1,3,5)$}
\[\vettcol{x}{y}{z} = \vettcol{4}{-2}{1}+t \vettcol{-1}{3}{5} \qquad \longrightarrow \qquad\begin{cases}
x = 4-t \\
y = -2+3t \\
z = 1+5t
\end{cases}\]
\end{esempio}
\begin{esempio}
 \emph{Determina la retta passante per i punti $A\,(3,5,2)$ e $B\,(1,5,4)$}\\[5pt]
 Innanzitutto calcoliamo la direzione della retta, come vettore $\vect{AB}$:
\[\vect{AB} = B-A = (1,5,4)-(3,5,2) = (-2,0,2)\]
Ora che abbiamo la direzione e un punto di passaggio (scegliamo A), calcoliamo:
\[\vettcol{x}{y}{z} = \vettcol{3}{5}{2}+t \vettcol{-2}{0}{2} \qquad \longrightarrow \qquad\begin{cases}
x = 3-2t \\
y = 5 \\
z = 2+2t
\end{cases}\]
\end{esempio}

\section{Piano (forma parametrica)}

Anche in questo caso non esiste un solo modo di definire un piano. Vediamo innanzitutto il modo più semplice: si costruisce un piano prendendo un punto P\,$(x_P,y_P,z_P)$ e spostandosi lungo due direzioni non parallele $\vect{a}\,(a_x,a_y,a_z)$ e $\vect{b}\,(b_x,b_y,b_z)$. Cioé:
\[\boxed{\vect{r} = P+u \vect{a}+v \vect{b}} \quad \Rightarrow \quad \begin{cases}
x = x_P+u a_x+v b_x \\
y = y_P+u a_y+v b_y \\
z = z_P+u a_z+v b_z
\end{cases} \quad (u \in \mathbb{R}\;,\, v \in \mathbb{R})\]
\begin{esempio}
\emph{Determina il piano passante per $P\,(1,4,3)$, determinato dai vettori $\vect{a} \,(1,3,6)$ e $\vect{b} \, (-3,-2,0)$}
\[\vettcol{x}{y}{z} = \vettcol{1}{4}{3}+u \vettcol{1}{3}{6}+v \vettcol{-3}{-2}{0} \qquad \longrightarrow \qquad\begin{cases}
x = 1+u-3v \\
y = 4+3u-2v \\
z = 3+6u
\end{cases}\]
\end{esempio}
\begin{esempio}
\label{pianoPuntoDirez}
\emph{Determina il piano passante per i punti $A\,(1,2,7)$, $B\,(4,2,2)$} e $C\,(5,-4,-1)$\\[5pt]
 Innanzitutto calcoliamo i vettori generatori del piano, scegliendo per esempio $\vect{AB}$ e $\vect{AC}$:
\[\vect{AB} = (4,2,2)-(1,2,7) = (3,0,-5) \qquad, \qquad \vect{AC} = (5,-4,-1)-(1,2,7) = (4,-6,-8) \]
Ora che abbiamo i vettori generatori e un punto di passaggio (scegliamo A, visto che è il punto da cui ``partono`` i due vettori ricavati), calcoliamo:
\[\vettcol{x}{y}{z} = \vettcol{1}{2}{7}+u \vettcol{3}{0}{-5}+v \vettcol{4}{-6}{-8} \qquad \longrightarrow \qquad\begin{cases}
x = 1+3u+4v \\
y = 2-6v \\
z = 7-5u-8v
\end{cases}\]
\end{esempio}

\section{Piano (forma cartesiana)}
Il piano può anche essere definito in un altro modo, con una definizione più tecnica. Dato un punto $P_0\,(x_0,y_0,z_0)$ e un vettore $\vect{v}\,(a,b,c)$ ad esso applicato, è possibile definire un piano come l'insieme di tutti quei punti $P\,(x,y,z)$ tali per cui il vettore $\vect{P_0P}$ sia perpendicolare al vettore $\vect{v}$. Poiché due vettori sono perpendicolari se il corrispondente prodotto scalare è nullo, possiamo scrivere:
\[\vect{P_0P} \cdot \vect{v} =0 \quad \Rightarrow \quad (P-P_0)\cdot \vect{v}=0 \quad \Rightarrow \quad P \cdot \vect{v}-P_0\cdot \vect{v}=0\]
Svolgendo i calcoli otteniamo:
\[ a x+by+cz + \underbrace{(-a x_0-by_0-bz_0)}_{d}=0 \quad \Rightarrow \quad \boxed{ax+by+cz+d=0}\]
Quindi un piano è rappresentato da un'equazione di primo grado in 3 variabili. Notare che i coefficienti delle 3 variabili $(a,b,c)$ determinano anche il \emph{vettore normale} (perpendicolare) al piano stesso, che definisce la \textbf{direzione} stessa del piano. 
\begin{esempio}
 \emph{Determina l'equazione del piano perpendicolare al vettore $\vect{d} \, (4,2,1)$ e passante per il punto $P\, (5,6,9)$}\\[5pt]
Poiché il vettore ortogonale al piano ci dà i coefficienti delle 3 variabili abbiamo già una struttura per il nostro piano: \(4x+2y+z +d=0\)\\[3pt]
Rimane da calcolare il valore del parametro $d$, che possiamo ricavare semplicemente imponendo il passaggio per il punto $P$, ovvero sostituendo le coordinate del punto:
\[4x+2y+z+d=0 \quad \longrightarrow \quad 4 (5)+2 (6)+(9)+d=0 \quad \rightarrow d = -41\]
E quindi l'equazione della retta è data da \(4x+2y+z-41=0\)
\end{esempio}

\section{Retta (forma cartesiana)}

Una volta visto come si definisce un piano in forma cartesiana, è possibile vedere un secondo modo di definire una retta, in forma cartesiana, come intersezione di 2 piani non paralleli.
\[\begin{cases}
a_1 x+b_1y+c_1 z+d_1 = 0 \\
a_2 x+b_2y+c_2 z+d_2 = 0
\end{cases} \quad \text{tali che } (a_1,b_1,c_1) \;\; \cancel{\parallel}\;\; (a_2,b_2,c_2)\]
Da notare che il sistema non è determinato perché è costituito da 2 equazioni in 3 incognite. Il restante grado di libertà del sistema è proprio quello necessario a descrivere un insieme infinito di punti allineati, ovvero una retta.

\section{Da forma cartesiana a implicita (e viceversa)}
Come si passa da una forma all'altra? Vediamo come fare attraverso alcuni esempi:
\begin{esempio}
\emph{Descrivere la retta dell'esercizio \ref{rettaPuntoDirez} in forma cartesiana}\\[5pt]
Per trasformarla in forma cartesiana è sufficiente ricavare ''t" da una delle 3 equazioni e sostituirla nelle altre, eliminando infine l'equazione utilizzata:
\[\begin{cases}
x = 4-t \\
y = -2+3t \\
z = 1+5t
\end{cases} \quad \rightarrow \quad \begin{cases}
t = 4-x \\
y = -2+3(4-x) \\
z = 1+5(4-x)
\end{cases} \quad \rightarrow \quad \begin{cases}
3x+y-10 = 0 \\
5x+z-21 = 0
\end{cases} \]
\end{esempio}
\begin{esempio}
\emph{Descrivere il piano dell'esercizio \ref{pianoPuntoDirez} in forma cartesiana}\\[5pt]
Come nell'esempio precedente, questa volta ricaviamo sia ``u'' che ''v``:
\[\begin{cases}
x = 1+3u+4v \\
y = 2-6v \\
z = 7-5u-8v
\end{cases} \quad \rightarrow \quad \begin{cases}
x = 1+3u+4 \tonda{\frac{2-y}{6}} \\
v = \frac{2-y}{6} \\
z = 7-5u-8\tonda{\frac{2-y}{6}}
\end{cases} \quad \rightarrow \quad \begin{cases}
u = \frac{3x+2y-7}{9} \\
v = \frac{2-y}{6} \\
z = 7-5 \tonda{\frac{3x+2y-7}{9}}-8\tonda{\frac{2-y}{6}}
\end{cases} \]
Eliminando le prime due equazioni, otteniamo:
\[18z=126-30x-20y+70-48+24y \quad \rightarrow \quad -15x+2y-9z+74=0\]
\end{esempio}
\begin{esempio}
\emph{Descrivere il piano $3x-2y+z-5=0$ in forma parametrica}\\[5pt]
Qui il discorso è più semplice: è sufficiente infatti assegnare a due incognite i parametri ''u`` e ''v``, e ricavare la terza incognita:
\[3x-2y+z-5=0 \quad \rightarrow \quad \begin{cases}
x = u \\
v = v \\
z = -3u+2v+5
\end{cases} \quad \rightarrow \quad \vettcol{x}{y}{z} =  \vettcol{0}{0}{5}+u \vettcol{1}{0}{-3}+v \vettcol{0}{1}{2} \]
\end{esempio}
Lo stesso vale anche per la retta, con l'unica differenza di dover assegnare solamente il parametro ''t`` ad una delle incognite.

\section{Posizioni reciproche tra rette e piani}

Ricordiamo innanzitutto che due rette nello spazio si possono trovare in 5 diverse configurazioni. Possono essere tra loro:
\begin{itemize}
\item \textbf{Incidenti}: se si intersecano in un punto;
\item \textbf{Coincidenti}: se sono identiche;
\item \textbf{Parallele}: se esiste un piano che le contiene entrambe, ma non sono incidenti;
\item \textbf{Perpendicolari}: se sono incidenti e formano tra esse 4 angoli retti;
\item \textbf{Sghembe}: Se non sono nè parallele nè incidenti;
\end{itemize}

\vspace{7pt}

Per parlare di parallelismo e perpendicolarità tra rette o piani ci si affida semplicemente a considerazioni sui vettori direttori. 

\vspace{7pt}

\textbf{\underline{Rette} :} \;La direzione di una retta, come definita in precedenza, è data semplicemente dal vettore $\vect{d}$, che può essere ricavato dalla forma parametrica della retta stessa. Per verificare dunque se 2 rette sono tra loro parallele, bisogna soltanto verificare che le due direzioni $\vect{d_1}$ e $\vect{d_2}$ siano a loro volta parallele. Per la perpendicolarità non basta invece. \'E necessario anche verificare che le rette assegnate siano incidenti.

\vspace{7pt}

\textbf{\underline{Piani} :} \; La direzione di un piano è data dal corrispondente vettore normale $\vect{n}$. Anche qui dunque non si deve far altro che basarsi sulla nozione stessa di parallelismo/perpendicolarità tra i vettori $\vect{n_1}$ e $\vect{n_2}$ dei due piani.

\section{Distanza punto-piano}

Nel piano cartesiano avevamo visto che era possibile calcolare la distanza punto-retta con una formula ''magica``, in cui compariva un valore assoluto al numeratore e una radice a denominatore. Anche nello spazio esiste una formula analoga, per calcolare però la \textbf{distanza punto-piano}: dato un punto $P\,\tonda{x_P,y_P,z_P}$ e un piano $\alpha$ in forma cartesiana $ax+by+cz+d=0$, la distanza del punto P dal piano è data da:
\[d(P,\alpha) = \frac{|ax_P+by_P+cz_P+d|}{\sqrt{a^2+b^2+c^2}}\]
\begin{esempio}
 \emph{Calcolare la distanza del punto $P\,(1,5,-3)$ dal piano $3x-5y+2z-4=0$}\\[5pt]
 \[d(P,\alpha)=\frac{|3 (1)-5(5)+2(-3)-4|}{\sqrt{(3)^2+(-5)^2+(2)^2}} = \frac{32}{\sqrt{38}} = \frac{32 \sqrt{38}}{38}\]
\end{esempio}



\section{Sfere}

Un altro oggetto interessante da studiare per familiarizzare con lo spazio tridimensionale è la sfera. La definizione è molto semplice: la sfera è il luogo dei punti dello spazio equidistanti da un punto, detto \textbf{centro} della sfera. Perciò, se $C\,(x_C,y_C,z_C)$ è il centro della sfera e $P\,(x,y,z)$ un generico punto su essa, allora P deve soddisfare l'equazione:
\[\overline{PC}=r \quad \Rightarrow \quad \sqrt{(x-x_C)^2+(y-y_C)^2+(z-z_C)^2} = r\]
Elevando ambo i membri al quadrato: \quad \fbox{\((x-x_C)^2+(y-y_C)^2+(z-z_C)^2=r^2\)}\\[8pt]
Notiamo che l'equazione ricorda molto quella della circonferenza, con la sola differenza di avere una coordinata in più. Calcolando, è possibile ottenere anche una seconda formula:
\[x^2-2x_C x +x_C^2 +y^2-2y_C y +y_C^2+z^2-2z_C z +z_C^2-r^2=0\]
Riorganizzando i termini:
\[x^2+y^2+z^2\underbrace{-2x_C}_a x\underbrace{-2y_C}_b y\underbrace{-2z_C}_c z+\underbrace{x_C^2+y_C^2+z_C^2-r^2}_d=0\]
Si ottiene così l'equazione: \quad \fbox{\(x^2+y^2+z^2+ax+by+cz+d=0\)}\\[8pt]
Data una sfera in questa forma, è possibile ricavare centro e raggio con le formule che si ottengono dalla definizione stessa dei 4 parametri $a,b,c,d$, ovvero:
\[C\; \tonda{-\frac{a}{2};-\frac{b}{2}; -\frac{c}{2}} \qquad ;\qquad r = \sqrt{\tonda{-\frac{a}{2}}^2+\tonda{-\frac{b}{2}}^2+\tonda{-\frac{c}{2}}^2-d}\]
\begin{esempio}
 \emph{Ricava l'equazione della sfera di raggio 7 e centro $C\, \tonda{3,1,0}$}
 \[(x-3)^2+(y-1)^2+(z-0)^2=7^2 \quad \rightarrow \quad x^2+y^2+z^2-6x-2y-39=0\]
\end{esempio}
\begin{esempio}
 \emph{Data la sfera $x^2+y^2+z^2-2x+6y-10z+1=0$, ricava centro e raggio}
 \[x_C = -\frac{-2}{2}=1 \quad;\quad y_C = -\frac{6}{2}=-3 \quad;\quad z_C = -\frac{-10}{2}=5\]
 \[\text{quindi:}\quad C = \tonda{1;-3,5} \quad \text{mentre}\quad r = \sqrt{(1)^2+(-3)^2+(5)^2-1} = \sqrt{36} = 6\]
\end{esempio}

\begin{osservazione}
 Anche per la sfera esiste una forma parametrica, ma solitamente non viene molto usata. Ne scriviamo la formula per completezza. Se $C\,(x_C,y_C,z_C)$ è il centro e $R$ il raggio:
\[\begin{cases}
   x = x_C+R \cos u+R \cos v \\
   y = y_C+R \cos u+R \sin v \\
   z = z_C+R \sin u
  \end{cases} \qquad \text{con} \quad u \in \quadra{-\frac{\pi}{2};\frac{\pi}{2}} \; v \in 
\quadra{-\pi;\pi} \]
\end{osservazione}



\section{Esempi particolari}

Vediamo in questa sezione come si calcolano rette, piani e sfere in situazioni particolari, in cui serve ragionare prima di applicare aride formule.

\begin{esempio}
 \emph{Sfera di centro $C\,\tonda{-2,1,2}$ tangente al piano \,$3y+4z+4=0$}\\[7pt]
 Qui è sufficiente notare che il raggio della sfera dovrà essere uguale alla distanza del centro stesso dal piano. Quindi possiamo calcolare:
 \[r = d(C,\alpha) = \frac{|3+8+4|}{\sqrt{9+16}} = \frac{15}{5} = 3\]
 E quindi la sfera è data da:
 \[(x+2)^2+(y-1)^2+(z-2)^2=3^2 \quad \longrightarrow \quad x^2+y^2+z^2+4x-2y-4z=0\]
\end{esempio}
\begin{esempio}
 \emph{Distanza del punto $P\,(4,-2,1)$ dalla retta di direzione $\vect{d} \, (1,4,2)$ e passante per $A\,(1,1,0)$}\\[7pt]
\end{esempio}

