% (c) 2017 Diego Rigo  rig0gt@gmail.com


\section{Esercizi}

\subsection{Esercizi dei singoli paragrafi}

\begin{esercizio}
Calcola il perimetro del triangolo ABC:
 \begin{enumeratea}
  \item  \(A\,\tonda{1;4;3} \quad B\,\tonda{3;7;2}\quad C\,\tonda{1;4;1}\) \hfill [\(2(1+\sqrt{14}) \)]
  \item  \(A\,\tonda{4;1;0} \quad B\,\tonda{4;3;-2}\quad C\,\tonda{1;7;-2}\) \hfill [\(2(6+\sqrt{2}) \)]
  \item  \(A\,\tonda{3;3;2} \quad B\,\tonda{5;1;2}\quad C\,\tonda{3;1;4}\) \hfill [\(6\sqrt{2} \)]
   \item  \(A\,\tonda{4\sqrt{3};5\sqrt{2};1} \quad B\,\tonda{\sqrt{3};-\sqrt{2};0}\quad C\,\tonda{3\sqrt{3};-2\sqrt{2};3}\) \hfill [\(10+\sqrt{23}+\sqrt{105} \)]
 \end{enumeratea}
\end{esercizio}

\begin{esercizio}
Determina se le seguenti coppie di vettori sono parallele/perpendicolari:
 \begin{enumeratea}
  \item  \(\vect{a} = \tonda{3;1;0} \quad \vect{b} = \tonda{-1;3;5}\)   \hfill [\(\text{perpendicolari} \)]
  \item  \(\vect{a} = \tonda{4;2;-8} \quad \vect{b} = \tonda{-2;-1;4}\)   \hfill [\(\text{paralleli} \)]
  \item  \(\vect{a} = \tonda{-\frac{1}{2};\frac{4}{3};\frac{3}{5}} \quad \vect{b} = \tonda{-10;-6;5}\)   \hfill [\(\text{perpendicolari} \)]
  \item  \(\vect{a} = \tonda{\frac{1}{3};\frac{2}{5};\frac{3}{2}} \quad \vect{b} = \tonda{\frac{1}{6};\frac{1}{5};\frac{3}{4}}\)   \hfill [\(\text{paralleli} \)]
  \item  \(\vect{a} = \tonda{0;8;2} \quad \vect{b} = \tonda{0;-2;-\frac{1}{2}}\)   \hfill [\(\text{paralleli} \)]
 \end{enumeratea}
\end{esercizio}

\begin{esercizio}
Determina la direzione della retta, dopo averla trasformata in forma parametrica:
\begin{multicols}{2}
 \begin{enumeratea}
  \item  \(\begin{cases}
x+2y-11=0 \\
5y+z+20=0
\end{cases}\)    \hfill [\((-2;1;5)\)]
\item\(\begin{cases}
x-3y+24=0 \\
3y+z-27=0
\end{cases}\)    \hfill [\((-3;-1;3)\)]
\item\(\begin{cases}
x-2z=0 \\
y-4z-1=0
\end{cases}\)    \hfill [\((2;4;1)\)]
\item\(\begin{cases}
x-6z-2=0 \\
3y-16z+12=0
\end{cases}\)    \hfill [\((18;16;3)\)]
 \end{enumeratea}
 \end{multicols}
\end{esercizio}

\begin{esercizio}
Determina l'equazione del piano di direzione $\vect{d}$, passante per $P$:
 \begin{enumeratea}
  \item  \(\vect{d} = \tonda{3;4;2} \quad P\,\tonda{3;0;2}\)    \hfill [\(3x+4y+2z-13=0\)]
  \item  \(\vect{d} = \tonda{1;2;8} \quad P\,\tonda{0;-4;3}\)    \hfill [\(x+2y+8z-16=0\)]
  \item  \(\vect{d} = \tonda{\frac{1}{3};\frac{3}{2};\frac{5}{3}} \quad P\,\tonda{2;-1;3}\)    \hfill [\(2x+9y+10z-25=0\)]
 \end{enumeratea}
\end{esercizio}

\begin{esercizio}
Calcola la distanza del punto $P$ dal piano $\alpha$
 \begin{enumeratea}
  \item  \(P\,\tonda{3;3;0} \qquad \alpha\,:\;x+2y+2z-3=0\)    \hfill [\(2\)]
  \item  \(P\,\tonda{8;-4;0} \qquad \alpha\,:\;2x+3y+6z+3=0\)    \hfill [\(1\)]
  \item  \(P\,\tonda{-3;-1;1} \qquad \alpha\,:\;-x+4y-8z=0\)    \hfill [\(1\)]
  \item  \(P\,\tonda{\sqrt{3};-2\sqrt{3};3} \qquad \alpha\,:\;x-4y+8z+3=0\)    \hfill [\(3+\sqrt{3}\)]
  \item  \(P\,\tonda{1;-2;0} \qquad \alpha\,:\;2x+6y-9z=0\)    \hfill [\(1\)]
 \end{enumeratea}
\end{esercizio}


\begin{esercizio}
Calcola l'equazione della sfera di centro $C$ e raggio $r$
 \begin{enumeratea}
  \item  \(C\,\tonda{3;1;2} \qquad r = 5\)    \hfill [\(x^2+y^2+z^2-6x-2y-4z-11=0\)]
  \item  \(C\,\tonda{1;-3;4} \qquad r = 2\)    \hfill [\(x^2+y^2+z^2-2x+6y-8z+22=0\)]
  \item  \(C\,\tonda{-5;2;1} \qquad r = 3\)    \hfill [\(x^2+y^2+z^2+10x-4y-2z+21=0\)]
  \item  \(C\,\tonda{1;1;1} \qquad r = 6\)    \hfill [\(x^2+y^2+z^2-2x-2y-2z-33=0\)]
 \end{enumeratea}
\end{esercizio}

% \subsection{Esercizi riepilogativi}
% 
% \begin{esercizio}
% \label{ese:D.19}
% testo esercizio
% \end{esercizio}
% 
% \begin{esercizio}\label{ese:03.1}
% Consegna:
%  \begin{enumeratea}
%   \item  
%  \end{enumeratea}
% \end{esercizio}
