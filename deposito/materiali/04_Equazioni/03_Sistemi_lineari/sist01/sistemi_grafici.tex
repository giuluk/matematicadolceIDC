% (c) 2017 Daniele Zambelli - daniele.zambelli@gmail.com
%
% Tutti i grafici per il capitolo relativo ai sitemi lineari
%

\newcommand{\rettaconpunti}[3]{% Retta con alcuni punti evidenziati
  % \rettaconpunti{-2./3*x+2}{-2./3*\x+2}
  %               {-3, -2.5, ..., 5}
  \def \funcf{#1}
  \def \funcn{#2}
  \def \xpunti{#3}
  \disegno{
    \rcom{-3}{+5}{-2}{+4}{gray!50, very thin, step=1}
    \tkzInit[xmin=-3.3,xmax=+5.3,ymin=-3.3,ymax=+5.3]
    \tkzFct[domain=-3.3:+5.3, ultra thick, color=red!50!black]{\funcf}
    \foreach \x in \xpunti{
      \draw[fill=blue] (\x, \funcn) circle (1.5pt);
    }
  }
}

\newcommand{\funzionicolorate}[5]{% Più funzioni in un piano cartesiano
  % \funzionicolorate{-3}{+8}{-2}{+6}
  %                  {{-x+5}/red!50!black, {1./3*x+1}/blue!50!black}
  \def \fcxmi{#1}
  \def \fcxma{#2}
  \def \fcymi{#3}
  \def \fcyma{#4}
  \def \funccolors{#5}
  \rcom{\fcxmi}{\fcxma}{\fcymi}{\fcyma}{gray!50, very thin, step=1}
  \tkzInit[xmin=\fcxmi-.3, xmax=\fcxma+.3, ymin=\fcymi-.3, ymax=\fcyma+.3]
  \foreach \func/\col in \funccolors{
    \tkzFct[domain=\fcxmi-.3:\fcxma+.3, ultra thick, color=\col]{\func}
  }
}

\newcommand{\intersezionediduerette}{% Due rette con evidenziata l'inters.
  \disegno{
    \funzionicolorate{-3}{+8}{-2}{+6}
                     {{-x+5}/red!50!black, {1./3*x+1}/green!50!black}
    \draw[fill=blue] (3, 2) circle (1.5pt) node [below, xshift=-1mm] {\(I\)};
  }
}

\newcommand{\rettapuntietichette}[3]{% Retta con alcuni punti evidenziati
% \rettapuntietichette{-2./3*x+2}{-2./3*\x+2}
%                     {0/\(P\punto{0}{2}\), 3/\(A\punto{3}{0}\)}
  \def \funcf{#1}
  \def \funcn{#2}
  \def \xpunti{#3}
  \disegno{
    \rcom{-3}{+5}{-2}{+4}{gray!50, very thin, step=1}
    \tkzInit[xmin=-3.3,xmax=+5.3,ymin=-3.3,ymax=+5.3]
    \tkzFct[domain=-3.3:+5.3, ultra thick, color=red!50!black]{\funcf}
    \foreach \x/\lp in \xpunti{
      \draw[fill=blue] (\x, \funcn) circle (1.5pt)
                                    node [below left, black] {\lp};
    }
  }
}

\newcommand{\rettangolo}[2]{%Disegna il rettangolo ABCD date le dimensioni
% \rettangolo{5}{3}
  \def \base{#1}
  \def \altezza{#2}
  \disegno{
    \draw (0,0) rectangle  (\base, \altezza);
    \node [below left=-.2] at (0, 0) {$A$};
    \node[below right=-.2]  at (\base, 0) {$B$};
    \node[above right=-.2]  at (\base, \altezza) {$C$};
    \node[above left=-.2]  at (0, \altezza) {$D$};
  }
}

\newcommand{\segmentoconpunto}[2]{% Disegna un segmento con un punto
% \segmentoconpunto{8}{1.5}
  \def \lung{#1}
  \def \ppos{#2}
  \disegno{
	  \filldraw [fill=blue] (0, 0) circle (1.5pt) node [below] {\(A\)} --
	        (\ppos, 0) circle (1.5pt) node [below] {\(P\)} --
	        (\lung, 0) circle (1.5pt) node [below] {\(B\)};
  }
}
