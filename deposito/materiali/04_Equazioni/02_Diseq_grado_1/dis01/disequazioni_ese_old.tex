% (c) 2012 -2014 Dimitrios Vrettos - d.vrettos@gmail.com
% (c) 2014 Daniele Zambelli - daniele.zambelli@gmail.com

\section{Esercizi}

\subsection{Esercizi dei singoli paragrafi}

%\subsubsection*{21.1 - Intervalli sulla retta reale}
\subsubsection*{\numnameref{sec:intervalli}}

\begin{esercizio}
 \label{ese:21.1}
 Determina la scrittura corretta per il seguente grafico.
 \begin{center}
  % (c) 2012 Dimitrios Vrettos - d.vrettos@gmail.com
\begin{tikzpicture}[font=\small,x=10mm, y=5mm]

\draw[->] (0,0) -- (8,0) node [below right] () {$r$};
\node[above]  at (4,0) {$-3$};
\begin{scope}[blue,thick]
\draw (0,0) -- (4,0);
\draw[fill=white] (4,0)circle (1.5pt);
\end{scope}

\end{tikzpicture}

 \boxA\quad~\(x<-3\) \quad\boxB\quad~\(x>-3\)\quad\boxC\quad~\(x\le -3\)\quad\boxD\quad~\(x\le -3\)
 \end{center}

\end{esercizio}

\begin{esercizio}
 \label{ese:21.2}
 Determina la scrittura corretta per il seguente grafico.
  \begin{center}
  % (c) 2012 Dimitrios Vrettos - d.vrettos@gmail.com
\begin{tikzpicture}[font=\small,x=10mm, y=5mm]

\draw[->] (0,0) -- (8,0) node [below right] () {$r$};
\node[above]  at (4,0) {$2$};
\begin{scope}[blue,thick,->]
\draw (4,0) -- (8,0);
\draw[fill=blue] (4,0)circle (1.5pt);
\end{scope}

\end{tikzpicture}

\boxA\quad~\(x<2\)\quad\boxB\quad~\(x>2\)\quad\boxC\quad~\(x\ge~2\)\quad\boxD\quad~\(x\le~2\)
 \end{center}
\end{esercizio}

\begin{esercizio}
 \label{ese:21.3}
 Determina la scrittura corretta per il seguente grafico.
  \begin{center}
  % (c) 2012 Dimitrios Vrettos - d.vrettos@gmail.com
\begin{tikzpicture}[font=\small,x=10mm, y=5mm]

\draw[->] (0,0) -- (8,0) node [below right] () {$r$};
\node[above]  at (2,0) {$-2$};
\node[above]  at (6,0) {2};
\begin{scope}[blue,thick]
\draw (2,0) -- (6,0);
\foreach \x in {2,6}
\draw[fill=white] (\x,0)circle (1.5pt);
\end{scope}

\end{tikzpicture}

\boxA\quad~\(x<+2\)\quad\boxB\quad~\(x>-2\)\quad\boxC\quad~\(-2\le x\le~2\)\quad\boxD\quad~\(-2<x<2\)
 \end{center}
\end{esercizio}

\begin{esercizio}
 \label{ese:21.4}
 Determina la scrittura corretta per il seguente grafico.
  \begin{center}
  % (c) 2012 Dimitrios Vrettos - d.vrettos@gmail.com
\begin{tikzpicture}[font=\small,x=10mm, y=5mm]

\draw[->] (0,0) -- (8,0) node [below right] () {$r$};
\node[above]  at (2,0) {$3$};
\node[above]  at (6,0) {5};
\begin{scope}[blue,thick]
\draw (2,0) -- (6,0);
\draw[fill=white] (2,0)circle (1.5pt);
\draw[fill=blue] (6,0)circle (1.5pt);
\end{scope}

\end{tikzpicture}

  \boxA\quad~\(x<5;x>3\)\quad\boxB\quad~\(3>x\ge~5\)\quad\boxC\quad~\(3\le x<5\)\quad\boxD\quad~\(3<x\le~5\)
 \end{center}
  \end{esercizio}

\begin{esercizio}
 \label{ese:21.5}
Determina la scrittura corretta per il seguente grafico.
 \begin{center}
  % (c) 2012 Dimitrios Vrettos - d.vrettos@gmail.com
\begin{tikzpicture}[font=\small,x=10mm, y=5mm]

\draw[->] (0,0) -- (8,0) node [below right] () {$r$};
\node[above]  at (2,0) {$-1$};
\node[above]  at (6,0) {0};
\begin{scope}[blue,thick]
\draw (2,0) -- (6,0);
\draw[fill=blue] (2,0)circle (1.5pt);
\draw[fill=blue] (6,0)circle (1.5pt);
\end{scope}

\end{tikzpicture}

  \boxA\quad~\(\insR^{-}-\{-1\}\)\quad\boxB\quad~\(-1\ge x\ge~0\)\quad\boxC\quad~\(-1\le x\le~0\)\quad\boxD\quad~\(0<x<-1\)
 \end{center}
 \end{esercizio}

\begin{esercizio}
 \label{ese:21.6}
Determina la scrittura corretta per il seguente grafico.
 \begin{center}
  % (c) 2012 Dimitrios Vrettos - d.vrettos@gmail.com
\begin{tikzpicture}[font=\small,x=10mm, y=5mm]

\draw[->] (0,0) -- (8,0) node [below right] () {$r$};
\node[above]  at (4,0) {0};
\begin{scope}[blue,thick,->]
\draw (4,0) -- (8,0);
\draw[fill=white] (4,0)circle (1.5pt);
\end{scope}

\end{tikzpicture}

  \boxA\quad~\(x>0\)\quad\boxB\quad~\(x>-\infty \)\quad\boxC\quad~\(x\le~0\)\quad\boxD\quad~\(0<x\le~0\)
 \end{center}
  \end{esercizio}

\begin{esercizio}
 \label{ese:21.7}
Determina la scrittura corretta per il seguente grafico.
 \begin{center}
  % (c) 2012 Dimitrios Vrettos - d.vrettos@gmail.com
\begin{tikzpicture}[font=\small,x=10mm, y=5mm]

\draw[->] (0,0) -- (8,0) node [below right] () {$r$};
\node[above]  at (2,0) {$1$};
\node[above]  at (6,0) {2};
\begin{scope}[blue,thick]
\draw (2,0) -- (6,0);
\draw[fill=blue] (2,0)circle (1.5pt);
\draw[fill=white] (6,0)circle (1.5pt);
\end{scope}

\end{tikzpicture}

  \boxA\quad~\(x\ge~1;x<2\)\quad\boxB\quad~\(1\le x<2\)\quad\boxC\quad~\(x\le~1\text{ e }x>2\)\quad\boxD\quad~\(2\ge~1\)
 \end{center}
  \end{esercizio}

%\subsubsection*{21.2 - Disequazioni numeriche}
\subsubsection*{\numnameref{sec:disequzioninumeriche}}

\begin{esercizio}
 \label{ese:21.8}
Completa la seguente tabella indicando con una crocetta il tipo di
disuguaglianza o disequazione:

 \begin{tabularx}{.9\textwidth}{X|c|c|c|}
 \toprule
 Proposizione&\multicolumn{2}{c}{Disuguaglianza}& Disequazione\\
  & Vera & Falsa & \\
 \midrule
 Il doppio di un numero reale è minore del suo triplo aumentato di~1: & & & \\
 \midrule
 La somma del quadrato di~4 con~3 è maggiore della somma del quadrato di~3 con~4: & & &\\
 \midrule
 Il quadrato della somma di~4 con~3 è minore o uguale a~49: & & & \\
 \midrule
 In~\(\insZ:(5+8)-(2)^{4}>0\): & & & \\
 \midrule
 \(-x^{2}>0\): & & & \\
 \midrule
 \((x+6)^{2}\cdot (1-9)\cdot (x+3-9)<0\): & & & \\
 \bottomrule
 \end{tabularx}
\end{esercizio}

\begin{esercizio}
 \label{ese:21.9}
 Rappresenta graficamente l'insieme delle soluzioni
delle seguenti disequazioni.
\begin{multicols}{3}
 \begin{enumeratea}
 \item \(x-2>0\)
\item \(x+5>0\)
\item \(x-4>0\)
\item \(x-5\ge~0\)
\item \(x+3\le~0\)
\item \(x>0\)
\item \(x\ge~0\)
\item \(-1\le x\)
\item \(3>x\)
 \end{enumeratea}
\end{multicols}
\end{esercizio}

\begin{esercizio}[\Ast]
 \label{ese:21.10}
Trova l'Insieme Soluzione delle seguenti disequazioni.
\begin{multicols}{2}
 \begin{enumeratea}
 \item \(3-x>x\)
\item \(2x>3\)
\item \(3x\le~4\)
\item \(5x\ge -4\)
\item \(x^{2}+x^{4}+10>0\)
\item \(x^{2}+x^{4}+100<0\)
\item \(-x+3>0\)
\item \(-x-3\le~0\)
\end{enumeratea}
\end{multicols}
\end{esercizio}

\begin{esercizio}[\Ast]
 \label{ese:21.11}
Trova l'Insieme Soluzione delle seguenti disequazioni.
 \begin{multicols}{2}
 \begin{enumeratea}
 \item \(3+2x\ge~3x+2\)
\item \(5x-4\ge~6x-4\)
\item \(-3x+2\ge -x-8\)
\item \(4x+4\ge~2(2x+8)\)
\item \(4x+4\ge~2(2x+1)\)
\item \(4x+4\ge~2(2x+2)\)
\item \(4x+4<2(2x+3)\)
\item \(4x+4>2(2x+2)\)
\end{enumeratea}
\end{multicols}
\end{esercizio}

\begin{esercizio}[\Ast]
 \label{ese:21.12}
Trova l'Insieme Soluzione delle seguenti disequazioni.
 \begin{multicols}{2}
 \begin{enumeratea}
 \item \(4x+4<2(2x+2)\)
\item \(x^{2}+4>3\)
\item \(x^{2}+3<-1\)
\item \(-3x-8\ge~2\)
\item \(-3x>0\)
\item \(-3x\le~0\)
\item \(-3x+5\ge~0\)
\item \(-3x-8\ge~0\)
\end{enumeratea}
\end{multicols}
\end{esercizio}

\newpage
\begin{esercizio}[\Ast]
 \label{ese:21.13}
Trova l'Insieme Soluzione delle seguenti disequazioni.
 \begin{multicols}{2}
 \begin{enumeratea}
 \item \(4x+4\ge~3\left(x+\frac{4}{3}\right)\)
\item \(-{\dfrac{4}{3}}x\ge~1\)
\item \(-{\dfrac{4}{3}}x\ge~0\)
\item \(-{\dfrac{4}{3}}x\ge \dfrac{2}{3}\)
\item \(-{\dfrac{2}{3}}x\le \dfrac{1}{9}\)
\item \(-{\dfrac{2}{3}}x\le~9\)
\item \(\dfrac{x+5}{2}>-{\dfrac{1}{5}}\)
\item \(x^2+1\ge\dfrac{x^2+4x-1}{2}+3x\)
\end{enumeratea}
\end{multicols}
\end{esercizio}

\begin{esercizio}[\Ast]
 \label{ese:21.14}
Trova l'Insieme Soluzione delle seguenti disequazioni.
 \begin{multicols}{2}
 \begin{enumeratea}
 \item \(x+\dfrac{1}{2}<\dfrac{(x+3)}{3}-1\)
\item \(\dfrac{(x+5)}{3}+3+2\dfrac{(x-1)}{3}\le x+4\)
\item \((x+3)^{2}\ge (x-2)(x+2)\)
\item \(\dfrac{3}{2}x+\dfrac{1}{4}<5\left(\dfrac{2}{3}x-\dfrac{1}{2}\right)\)
\item \(1-(2x-4)^{2}>-x\cdot (4x+1)+2\)
\item \((x+1)^{2}\ge (x-1)^{2}\)
\item \(\dfrac{3}{2}\cdot (x+1)-\dfrac{1}{3}\cdot (1-x)<x+2\)
\item \(\dfrac{x+0,25}{2}<1,75+0,25x\)
\end{enumeratea}
\end{multicols}
\end{esercizio}

\begin{esercizio}[\Ast]
 \label{ese:21.15}
Trova l'Insieme Soluzione delle seguenti disequazioni.
 \begin{enumeratea}
 \item \(\dfrac{1}{2}\left(3x-\dfrac{1}{3}\right)-\dfrac{1}{3}(1+x)(1-x)+3\left(\dfrac{1}{3}x-1\right)^{2}\ge~0\)
\item \(3\dfrac{(x+1)}{2}-\dfrac{x+1}{3}-\dfrac{1}{9}>-5x+\dfrac{1}{2}\)
\item \(\left(\dfrac{x}{2}-1\right)\left(1+\dfrac{x}{2}\right)+x-\dfrac{1}{2}>x\dfrac{(x-1)}{4}+\dfrac{5x-6}{4}\)
\item \(\dfrac{1}{2}\left(x-\dfrac{1}{2}\right)+\dfrac{1}{3}\left(x+\dfrac{1}{3}\right)>\dfrac{x-\dfrac{1}{2}}{3}+\dfrac{x-\dfrac{1}{3}}{2}\)
\end{enumeratea}
\end{esercizio}
\begin{multicols}{2}
\begin{esercizio}[\Ast]
 \label{ese:21.16}
 Sommando un numero con il doppio del suo successivo si deve ottenere
un numero maggiore di~17. Quali numeri verificano questa
condizione?
\end{esercizio}

 \begin{esercizio}[\Ast]
 \label{ese:21.17}
 Sommando due numeri pari consecutivi si deve ottenere un numero che
non supera la metà del numero più grande. Quali valori può
assumere il primo numero pari?
 \end{esercizio}

 \begin{esercizio}[\Ast]
 \label{ese:21.18}
 Il noleggio di una automobile costa \officialeuro\ 55,00 al giorno, più
\officialeuro\ 0,085 per ogni chilometro percorso. Qual è il massimo di
chilometri da percorrere giornalmente, per spendere non più di \officialeuro\ 80,00 al giorno?
 \end{esercizio}

 \begin{esercizio}
 \label{ese:21.19}
 In una fabbrica, per produrre una certa merce, si ha
una spesa fissa settimanale di \officialeuro\ 413, ed un costo di produzione di \officialeuro\ 2,00 per ogni
kg di merce. Sapendo che la merce viene venduta a \officialeuro\ 4,00 al~kg, determinare la quantità minima da produrre
alla settimana perché l'impresa non sia in perdita.
 \end{esercizio}

 \begin{esercizio}[\Ast]
 \label{ese:21.20}
 Per telefonare in alcuni paesi esteri, una compagnia telefonica
propone due alternative di contratto:
\begin{enumeratea}
 \item \officialeuro\ 1,20 per il primo minuto di conversazione, \officialeuro\ 0,90 per ogni minuto successivo;
\item \officialeuro\ 1,00 per ogni minuto di conversazione.
\end{enumeratea}
Quanti minuti deve durare una telefonata perché convenga la seconda
alternativa?
 \end{esercizio}

\begin{esercizio}[\Ast]
 \label{ese:21.21}
 Il prezzo di un abbonamento mensile ferroviario è di \officialeuro\ 125,00.
 Sapendo che il prezzo di un singolo biglietto sulla stessa
tratta è di \officialeuro\ 9,50, trovare il numero minimo di viaggi per
cui l'abbonamento mensile risulta conveniente, e
rappresentare grafica-mente la soluzione.
 \end{esercizio}

 \begin{esercizio}
 \label{ese:21.22}
 Al circolo tennis i soci pagano \officialeuro\ 12 a ora di gioco, i non
soci pagano \officialeuro\ 15. Sapendo che la tessera annuale costa
\officialeuro\ 150, dopo quante partite all'anno conviene
fare la tessera di socio?
 \end{esercizio}

 \begin{esercizio}[\Ast]
 \label{ese:21.23}
 \ In montagna l'abbonamento per due settimane allo
skipass costa \officialeuro\ 220 mentre il biglietto giornaliero costa
\officialeuro\ 20. Andando a sciare ogni giorno, dopo quanti giorni
conviene fare l'abbonamento?
 \end{esercizio}

 \begin{esercizio}[\Ast]
 \label{ese:21.24}
 Marco ha preso alle prime tre prove di matematica i seguenti voti: 5;
5,5; 4,5. Quanto deve prendere alla quarta e ultima prova per avere almeno~6
di media?
 \end{esercizio}

 \begin{esercizio}
 \label{ese:21.25}
 Per produrre un tipo di frullatore un'azienda ha dei
costi fissi per \officialeuro\ 12\,000 a settimana e riesce a produrre~850
frullatori a settimana, ognuno dei quali ha un costo di produzione pari
a \officialeuro\ 34. L'azienda concorrente riesce a
vendere un frullatore analogo a \officialeuro\ 79. A quanto devono essere
venduti i frullatori in modo che l'azienda abbia un
utile e che il prezzo di vendita non sia superiore a quello del
prodotto concorrente?
 \end{esercizio}

 \begin{esercizio}[\Ast]
 \label{ese:21.26}
 Per noleggiare un'auto una compagnia propone
un'auto di tipo citycar al costo di \officialeuro\ 0,20 per km percorso e una quota fissa giornaliera
di \officialeuro\ 15,00,
un'auto di tipo economy al costo di \officialeuro\ 0,15
per km e una quota fissa giornaliera di \officialeuro\ 20,00. Dovendo
noleggiare l'auto per~3 giorni quanti km occorre fare
perché sia più conveniente l'auto di tipo economy?
 \end{esercizio}

 \begin{esercizio}
 \label{ese:21.27}
 Alle~9.00 di mattina sono in autostrada e devo raggiungere una città
che dista~\(740\unit{km}\) entro le~17.00 poiché ho un appuntamento di lavoro.
Prevedendo una sosta di mezzora per mangiare un panino, a quale
velocità devo viaggiare per arrivare in orario?
 \end{esercizio}

 \begin{esercizio}[\Ast]
 \label{ese:21.28}
 Quanto deve essere lungo il lato di un triangolo equilatero il cui
perimetro deve superare di~\(900\unit{cm}\) il perimetro di un triangolo
equilatero che ha il lato di~\(10\unit{cm}\)?
 \end{esercizio}

 \begin{esercizio}[\Ast]
 \label{ese:21.29}
 I lati di un triangolo sono tali che il secondo è doppio del primo e
il terzo è più lungo del secondo di~\(3\unit{cm}\) Se il perimetro deve
essere compreso tra~\(10\unit{cm}\) e~\(20\unit{cm}\), tra quali valori può variare il lato
più piccolo?
 \end{esercizio}

 \begin{esercizio}[\Ast]
 \label{ese:21.30}
 In un triangolo isoscele l'angolo
alla base deve essere minore della metà dell'angolo
al vertice. Tra quali valori deve essere compresa la misura
dell'angolo alla base?
 \end{esercizio}

 \begin{esercizio}[\Ast]
 \label{ese:21.31}
 Un trapezio rettangolo l'altezza che è il triplo
della base minore, mentre la base maggiore è~5 volte la base minore.
Se il perimetro del trapezio non deve superare i~\(100\unit{m}\), quali valori
può assumere la lunghezza dell'altezza del
trapezio?
 \end{esercizio}

 \begin{esercizio}[\Ast]
 \label{ese:21.32}
 Un rettangolo ha le dimensioni una doppia dell'altra.
Si sa che il perimetro non deve superare~\(600\unit{m}\) e che
l'area non deve essere inferiore a~\(200\unit{m^2}\) Tra quali
valori possono variare le dimensioni del rettangolo?
 \end{esercizio}
\end{multicols}

%\subsubsection*{21.3 - Sistemi di disequazioni}
\subsubsection*{\numnameref{sec:sistemi}}

\begin{esercizio}
 \label{ese:21.33}
Sulla retta reale rappresenta l'insieme soluzione~\(S_{1}\)
dell'equazione:
\[\frac{1}{6}+\frac{1}{4}\cdot (5x+3)=2+\frac{2}{3}\cdot (x+1)\]
\newpage
e l'insieme soluzione~\(S_{2}\) della disequazione:
\[\frac{1}{2}-2\cdot\left(\frac{1-x}{4}\right)\ge~3-\frac{6-2x}{3}-\frac{x}{2}.\]

È vero che~\(S_{1}\subset S_{2}\)?
\end{esercizio}

\begin{esercizio}[\Ast]
 \label{ese:21.34}
 Determina i numeri reali che verificano il sistema:
 \(\left\{%
  \begin{array}{l}
  x^{2}\le~0
  \\2-3x\ge~0
 \end{array}\right..\)
 \end{esercizio}

\begin{esercizio}
 \label{ese:21.35}
 L'insieme soluzione del sistema:
\(\left\{\begin{array}{l}
  (x+3)^{3}-(x+3)\cdot (9x-2)>x^{3}+27\\
  \dfrac{x+5}{3}+3+\dfrac{2\cdot (x-1)}{3}<x+1
 \end{array}\right.\) è:
\begin{multicols}{2}
\boxA\quad~\(\left\{x\in \insR/x>3\right\}\)

\boxB\quad~\(\left\{x\in \insR/x>-3\right\}\)

\boxC\quad~\(\left\{x\in \insR/x<-3\right\}\)

\boxD\quad~\(\IS=\emptyset \)

\boxE\quad~\(\left\{x\in\insR/x<3\right\}\)
\end{multicols}

\end{esercizio}

\begin{esercizio}
 \label{ese:21.36}
 Attribuire il valore di verità alle seguenti proposizioni:

\begin{enumeratea}
\item il quadrato di un numero reale è sempre positivo;
\item l'insieme complementare di~\(A=\{x\in\insR/x>-8\}\text{ è }B=\{x\in\insR/x<-8\}\)
\item il monomio~\(-6x^{3}y^{2}\) assume valore positivo per tutte le coppie dell'insieme~\(\insR^{+}\times\insR^{+}\)
\item nell'insieme~\(\insZ\) degli interi relativi il sistema~\(\left\{\begin{array}{l}x+1>0\\8x<0\end{array}\right.\) non ha soluzione;
\item l'intervallo~\(\left[-1,\left.-{\dfrac{1}{2}}\right)\right.\) rappresenta l'\(\IS\) del sistema~\(\left\{\begin{array}{l}1+2x<0 \\\dfrac{x+3}{2}\le x+1\end{array}\right.\)
\end{enumeratea}
\end{esercizio}

\begin{esercizio}[\Ast]
 \label{ese:21.37}
 Risolvi i seguenti sistemi di disequazioni.
 \begin{multicols}{2}
 \begin{enumeratea}
 \item \(\left\{\begin{array}{l}
  3-x>x\\
  2x>3
        \end{array}\right.;\)
\item \(\left\{\begin{array}{l}
  3x\le~4\\
  5x\ge -4
   \end{array}\right.;\)
\item \(\left\{\begin{array}{l}
  2x>3\\
  3x\le~4
        \end{array}\right.;\)
\item \(\left\{\begin{array}{l}
  3x-5<2\\
  x+7<-2x
   \end{array}\right..\)
 \end{enumeratea}
\end{multicols}
\end{esercizio}

\begin{esercizio}[\Ast]
 \label{ese:21.38}
 Risolvi i seguenti sistemi di disequazioni.
 \begin{multicols}{2}
 \begin{enumeratea}
 \item \(\left\{\begin{array}{l}
  3-x\ge x-3\\
  -x+3\ge~0
        \end{array}\right.;\)
\item \(\left\{\begin{array}{l}
  -x-3\le~3\\
  3+2x\ge~3x+2
   \end{array}\right.;\)
\item \(\left\{\begin{array}{l}
  2x-1>2x \\
  3x+3\le~3
        \end{array}\right.;\)
\item \(\left\{\begin{array}{l}
  2x+2<2x+3\\
  2(x+3)>2x+5
        \end{array}\right..\)
\end{enumeratea}
\end{multicols}
\end{esercizio}

\begin{esercizio}[\Ast]
 \label{ese:21.39}
 Risolvi i seguenti sistemi di disequazioni.
 \begin{multicols}{2}
 \begin{enumeratea}
 \item \(\left\{\begin{array}{l}
  -3x>0\\
  -3x+5\ge~0\\
  -3x\ge-2x
        \end{array}\right.;\)
\item {\longarray \(\left\{\begin{array}{l}
  -{\dfrac{4}{3}}x\ge\dfrac{2}{3}\\
  -{\dfrac{2}{3}}x\le\dfrac{1}{9}
        \end{array}\right.;\)}
\item \(\left\{\begin{array}{l}
  3+2x>3x+2 \\
  5x-4\le~6x-4\\
  -3x+2\ge -x-8
        \end{array}\right.;\)
\item \(\left\{\begin{array}{l}
  4x+4\ge~3\cdot\left(x+\dfrac{4}{3}\right)\\
  4x+4\ge~2\cdot (2x+2)
        \end{array}\right..\)
\end{enumeratea}
\end{multicols}
\end{esercizio}

\begin{esercizio}[\Ast]
 \label{ese:21.40}
 Risolvi i seguenti sistemi di disequazioni.
 \begin{multicols}{2}
 \begin{enumeratea}
 \item \(\left\{\begin{array}{l}
  3(x-1)<2(x+1)\\
  x-\dfrac{1}{2}+\dfrac{x+1}{2}>0
        \end{array}\right.;\)
\item {\longarray \(\left\{\begin{array}{l}
  16(x+1)-2+(x-3)^{2}\le(x+5)^{2}\\
        \dfrac{x+5}{3}+3+2\cdot\dfrac{x-1}{3}\le x+4
        \end{array}\right.;\)}
\item \(\left\{\begin{array}{l}
  x+\dfrac{1}{2}<\dfrac{1}{3}(x+3)-1\\
  (x+3)^{2}\ge (x-2)(x+2)
        \end{array}\right.;\)
\item {\longarray \(\left\{\begin{array}{l}
        \dfrac{2x+3}{3}>x-1\\
        \dfrac{x-4}{5}<\dfrac{2x+1}{3}
        \end{array}\right..\)}
\end{enumeratea}
\end{multicols}
\end{esercizio}

\begin{esercizio}[\Ast]
 \label{ese:21.41}
 Risolvi i seguenti sistemi di disequazioni.

 \begin{enumeratea}
 \item {\longarray \(\left\{\begin{array}{l}
  2\left(x-\dfrac{1}{3}\right)+x>3x-2\\
        \dfrac{x}{3}-\dfrac{1}{2}\ge \dfrac{x}{4}-\dfrac{x}{6}
   \end{array}\right.;\)}
\item \(\left\{\begin{array}{l}
    \dfrac{3}{2}x+\dfrac{1}{4}<5\cdot\left(\dfrac{2}{3}x-\dfrac{1}{2}\right)\\
    x^2-2x+1\ge~0
   \end{array}\right.;\)
\item {\longarray \(\left\{\begin{array}{l}
  3\left(x-\dfrac{4}{3}\right)+\dfrac{2-x}{3}+x-\dfrac{x-1}{3}>0\\
        \left[1-\dfrac{1}{6}(2x+1)\right]+\left(x-\dfrac{1}{2}\right)^{2}<(x+1)^{2}+\dfrac{1}{3}(1+2x)
   \end{array}\right.;\)}
\item {\longarray \(\left\{\begin{array}{l}
        \left(x-\dfrac{1}{2}\right)\left(x+\dfrac{1}{2}\right)>\left(x-\dfrac{1}{2}\right)^{2}\\
  2\left(x-\dfrac{1}{2}\right)\left(x+\dfrac{1}{2}\right)<\left(x-\dfrac{1}{2}\right)^{2}+\left(x+\dfrac{1}{2}\right)^{2}
   \end{array}\right..\)}
 \end{enumeratea}
\end{esercizio}

%\subsubsection*{21.4 - Disequazioni polinomiali di grado superiore al primo}
\subsubsection*{\numnameref{sec:fratta}}

\begin{esercizio}
 \label{ese:21.42}
 Risolvi le seguenti disequazioni.

\begin{enumeratea}
\item \((x+3)\cdot \left(\frac{1}{5}x+\frac{3}{2}\right)<0\) e~\(\left(-{\frac{6}{11}}+2x\right)\cdot\left(-x+\frac{9}{2}\right)\)
\item \(\left(x+\frac{3}{2}\right)\cdot \left(5x+\frac{1}{5}\right)<0\) e~\(\left(-{\frac{1}{10}}x+2\right)\cdot \left(-3x+9\right)\ge~0\)
\end{enumeratea}

\end{esercizio}

\begin{esercizio}
 \label{ese:21.43}
 \((x-3)\cdot (2x-9)\cdot (4-5x)>0.\)
\end{esercizio}

\begin{esercizio}[\Ast]
 \label{ese:21.44}
Trovare l'Insieme Soluzione delle seguenti disequazioni.
\begin{multicols}{2}
 \begin{enumeratea}
 \item \((x+2)(3-x)\le~0\)
\item \(x(x-2)>0\)
\item \((3x+2)(2-3x)<0\)
\item \(-3x(2-x)(3-x)\ge~0\)
\end{enumeratea}
\end{multicols}
\end{esercizio}
\newpage
\begin{esercizio}[\Ast]
 \label{ese:21.45}
Trovare l'Insieme Soluzione delle seguenti disequazioni.
\begin{multicols}{2}
 \begin{enumeratea}
 \item \((x+1)(1-x)\left(\frac{1}{2}x-2\right)\ge~0\)
\item \((x-1)(x-2)(x-3)(x-4)<0\)
\item \(x^{2}-16\le~0\)
\item \(4x^{2}-2x<0\)
\end{enumeratea}
\end{multicols}
\end{esercizio}

\begin{esercizio}[\Ast]
 \label{ese:21.46}
Trovare l'Insieme Soluzione delle seguenti disequazioni.
\begin{multicols}{2}
 \begin{enumeratea}
 \item \(x^{4}-81\ge~0\)
\item \(x^{2}+17x+16\le~0\)
\item \(16-x^{4}\le~0\)
\item \(x^{2}+2x+1<0\)
\end{enumeratea}
\end{multicols}
\end{esercizio}

\begin{esercizio}[\Ast]
 \label{ese:21.47}
Trovare l'Insieme Soluzione delle seguenti disequazioni.
\begin{multicols}{2}
 \begin{enumeratea}
 \item \(x^{2}+6x+9\ge~0\)
\item \(x^{2}-5x+6<0\)
\item \(x^{2}+3x-4\le~0\)
\item \(x^{3}>x^{2}\)
\end{enumeratea}
\end{multicols}
\end{esercizio}

\begin{esercizio}[\Ast]
 \label{ese:21.48}
Trovare l'Insieme Soluzione delle seguenti disequazioni.
\begin{multicols}{2}
 \begin{enumeratea}
 \item \(x^{2}(2x^{2}-x)-(2x^{2}-x)<0\)
\item \(x^{2}-2x+1+x(x^{2}-2x+1)<0\)
\item \(x^{3}-2x^{2}-x+2\ge~0\)
\item \(x^{4}+4x^{3}+3x^{2}>0\)
\end{enumeratea}
\end{multicols}
\end{esercizio}

\begin{esercizio}[\Ast]
 \label{ese:21.49}
Trovare l'Insieme Soluzione delle seguenti disequazioni.
\begin{multicols}{2}
 \begin{enumeratea}
 \item \((6x^{2}-24x)(x^{2}-6x+9)<0\)
\item \((x^{3}-8)(x+2)<(2-x)(x^{3}+8)\)
\item \((2a+1)(a^{4}-2a^{2}+1)<0\)
\item \(x^{3}-6x^{2}+11>1-3x\)
\item \(x^{6}-x^{2}+x^{5}-6x^{4}-x+6<0\)
\end{enumeratea}
\end{multicols}
\end{esercizio}

\begin{esercizio}[\Ast]
 \label{ese:21.50}
 Determinare i valori che attribuiti alla variabile~\(y\) rendono positivi
entrambi i polinomi
seguenti:~\(p_{1}=y^{4}-13y^{2}+36;\quad p_{2}=y^{3}-y^{2}-4y+4.\)
\end{esercizio}

\begin{esercizio}[\Ast]
 \label{ese:21.51}
 Determinare i valori di~\(a\) che rendono~\(p=a^{2}+1\) minore di~5.
\end{esercizio}

\begin{esercizio}[\Ast]
 \label{ese:21.52}
 Determina~\(\IS\) dei seguenti sistemi di disequazioni.
 \begin{multicols}{3}
 \begin{enumeratea}
 \item \(\left\{\begin{array}{l}
  x^{2}-9\ge~0\\
  x^{2}-7x+10<0
           \end{array}\right.;\)
\item \(\left\{\begin{array}{l}
  x^{2}+3x-18\ge~0\\
  12x^{2}+12x+3>0
           \end{array}\right.;\)
\item \(\left\{\begin{array}{l}
  16x^{4}-1<0 \\
  16x^{3}+8x^{2}\ge~0 \end{array}\right.. \)
 \end{enumeratea}
 \end{multicols}
\end{esercizio}

\begin{esercizio}[\Ast]
 \label{ese:21.53}
 Determina~\(\IS\) dei seguenti sistemi di disequazioni.
 \begin{multicols}{2}
 \begin{enumeratea}
 \item \(\left\{\begin{array}{l}
  49a^{2}-1\ge~0\\
  9a^{2}<1\\
  1-a>0
           \end{array}\right.;\)

\item \(\left\{\begin{array}{l}
          2x^{2}-13x+6<0\\
          (2x^{2}-5x-3)(1-3x)>0\\
          x^{2}+7>1
           \end{array}\right..\)
 \end{enumeratea}
 \end{multicols}
\end{esercizio}

\begin{esercizio}
\label{ese:21.54}
Studia il segno della frazione
\[f=\dfrac{x^{3}+11x^{2}+35x+25}{x^{2}-25}.\]
\emph{Suggerimento}: scomponi in fattori numeratore e denominatore, otterrai
\[ f=\frac{(x+5)^{2}(x+1)}{(x+5)(x-5)}.\]
% Poniamo le~\(\CE\) e semplifica la frazione: \dotfill
% 
% Studia separatamente il segno di tutti i fattori che vi compaiono. Verifica che la tabella dei segni sia:
% \begin{center}
% % (c) 2012 Dimitrios Vrettos - d.vrettos@gmail.com
\begin{tikzpicture}[font=\small,x=10mm, y=10mm]

\draw[->] (0,0) -- (8,0) node [below right] () {$r$};

\foreach \x in {1.5,3.5,6.5}{
\draw(\x,3pt)--(\x,-3pt);
\begin{scope}[dotted]
\draw (\x,0) -- (\x,-2);
\draw (0,-.5) -- (1.5,-.5);
\draw (0,-1) -- (3.5,-1);
\draw (0,-1.5) -- (6.5,-1.5);
\end{scope}}


\node[above] at (1.5,0) {$-5$};
\node[above]  at (3.5,0) {$-1$};
\node[above]  at (6.5,0) {$5$};

\begin{scope}[blue,thick]
\draw (1.5,-.5) -- (8,-.5);
\draw (3.5,-1) -- (8,-1);
\draw (6.5,-1.5) -- (8,-1.5);

\draw[fill=white] (1.5,-.5)circle (1.5pt);
\draw[fill=blue] (3.5,-1)circle (1.5pt);
\draw[fill=white] (6.5,-1.5)circle (1.5pt);
\end{scope}

\foreach \x in {-1.5}{
\node (n1) at (\x,-.25) {segno di $n_1$:};
\node(n2)  at (\x,-.75) {segno di $n_2$:};
\node   at (\x,-1.25) {segno di $D$:};
\node (d3) at (\x,-1.75) {segno di $f$:};
}

 \draw[decorate, decoration={brace, mirror}]  let \p1=(n1.north west), \p2=(n2.south west) in(\p1 ) -- (\p2) node[midway, left=2pt] {$N:$};

\foreach \z in {2.5,5,7.25}
\node  at (\z,-.25) {$+$};

 \foreach \zi in {.75,2.5}
 \node  at (\zi,-.75) {$-$};

\foreach \zii in {5,7.25}
 \node  at (\zii,-.75) {$+$};

 \foreach \ziii in {.75,2.5,5}
\node  at (\ziii,-1.25) {$-$};

\node  at (.75,-.25) {$-$};
\node  at (7.25,-1.25) {$+$};

\begin{scope}[red]
\foreach \y in {-1.75}{
\foreach \ziv in {.75,5}
	\node at (\ziv,\y) {$-$};
\foreach \zv in {2.5,7.25}
\node at (\zv,\y) {$+$};
}
\end{scope}
\end{tikzpicture}
% \end{center}
La frazione assegnata, con la~\(\CE: x\neq -5\text{ e }x\neq~5\), si annulla se~\(x=-1\)
è positiva nell'insieme~\(A^{+}=\left\{x\in \insR/-5<x<-1\vee x>5\right\}\), è negativa in
\(A^{-}=\left\{x\in\insR/x<-5\vee -1<x<5\right\}\)
\end{esercizio}

\begin{esercizio}[\Ast]
\label{ese:21.55}
Determinate~\(\IS\) delle seguenti disequazioni fratte.
\begin{multicols}{2}
\begin{enumeratea}
\spazielenx
\item \(\dfrac{x-2}{3x-9}>0\)
\item \(\dfrac{3x+12}{(x-4)(6-3x)}\geqslant~0\)
\item \(\dfrac{x+2}{x-1}<2\)
\item \(\dfrac{4-3x}{6-5x}\geqslant -3\)
\end{enumeratea}
\end{multicols}
\end{esercizio}

\begin{esercizio}[\Ast]
\label{ese:21.56}
Determinate~\(\IS\) delle seguenti disequazioni fratte.
\begin{multicols}{2}
\begin{enumeratea}
\spazielenx
 \item \(\dfrac{x+8}{x-2}\ge~0\)
\item \(\dfrac{3x+4}{x^{2}+1}\ge~2\)
\item \(\dfrac{4}{x+4}+\dfrac{2}{x-3}\leqslant~0\)
\item \(\dfrac{7}{x+3}-\dfrac{6}{x+9}\geqslant~0\)
\end{enumeratea}
\end{multicols}
\end{esercizio}

\begin{esercizio}[\Ast]
\label{ese:21.57}
Determinate~\(\IS\) delle seguenti disequazioni fratte.
\begin{multicols}{2}
\begin{enumeratea}
\spazielenx
 \item \(\dfrac{3}{2-x}\leqslant \dfrac{1}{x-4}\)
\item \(\dfrac{2}{4x-16}<\dfrac{2-6x}{x^{2}-8x+16}\)
\item \(\dfrac{x-3}{x^{2}-4x+4}-1<\dfrac{3x-3}{6-3x}\)
\item \(\dfrac{2}{x-2}>\dfrac{2x-2}{(x-2)(x+3)}\)
\end{enumeratea}
\end{multicols}
\end{esercizio}

\begin{esercizio}[\Ast]
\label{ese:21.58}
Determinate~\(\IS\) delle seguenti disequazioni fratte.
\begin{multicols}{2}
\begin{enumeratea}
\spazielenx
 \item \(\dfrac{5}{2x+6}\geqslant \dfrac{5x+4}{x^{2}+6x+9}\)
\item \(\dfrac{x}{x+1}-\dfrac{1}{x^{3}+1}\le~0\)
\item \(\dfrac{(x+3)(10x-5)}{x-2}<0\)
\item \(\dfrac{4-3x}{x-2}<\dfrac{3x+1}{x-2}\)
\end{enumeratea}
\end{multicols}
\end{esercizio}

\begin{esercizio}[\Ast]
\label{ese:21.59}
Determinate~\(\IS\) delle seguenti disequazioni fratte.
\begin{multicols}{2}
\begin{enumeratea}
\spazielenx
 \item \(\dfrac{5x-4}{3x-12}\ge \dfrac{x-4}{4-x}\)
\item \(\dfrac{2-x}{5x-15}\le \dfrac{5x-1}{2x-6}\)
\item \(\dfrac{(3x-12)(6-x)}{(24-8x)(36-18x)}\leqslant~0\)
\item \(\dfrac{(x-2)(5-2x)}{(5x-15)(24-6x)}\geqslant~0\)
\end{enumeratea}
\end{multicols}
\end{esercizio}

% \newpage

\begin{esercizio}[\Ast]
\label{ese:21.60}
Determinate~\(\IS\) delle seguenti disequazioni fratte.
\begin{multicols}{2}
\begin{enumeratea}
\spazielenx
\item \(\dfrac{(x-2)(x+4)(x+1)}{(x-1)(3x-9)(10-2x)}\leqslant~0\)
\item \(\dfrac{(5-x)(3x+6)(x+3)}{(4-2x)(x-6)x}\leqslant~0\)
\item \(\dfrac{(x-5)(3x-6)(x-3)}{(4-2x)(x+6)x}\leqslant~0\)
\item \(\dfrac{(x-3)(x+2)(15+5x)}{x^{2}-5x+4}\geqslant~0\)
\end{enumeratea}
\end{multicols}
\end{esercizio}

\begin{esercizio}[\Ast]
\label{ese:21.61}
Determinate~\(\IS\) delle seguenti disequazioni fratte.
\begin{multicols}{2}
\begin{enumeratea}
\spazielenx
\item \(\dfrac{\left(x-4\right)^{2}(x+3)}{x^{2}+5x+6}\geqslant~0\)
\item \(\dfrac{x}{1-x^{2}}>\dfrac{1}{2x+2}-\dfrac{2}{4x-4}\)
\item \(\dfrac{3-x}{x-2}<\dfrac{x-1}{x+3}+\dfrac{2}{x^{2}+x-6}\)
\item \(\dfrac{2}{x+2}-\dfrac{1}{x+1}\ge \dfrac{3}{2x+2}\)
\end{enumeratea}
\end{multicols}
\end{esercizio}

\begin{esercizio}[\Ast]
\label{ese:21.62}
Determinate~\(\IS\) delle seguenti disequazioni fratte.
\begin{multicols}{2}
\begin{enumeratea}
\spazielenx
 \item \(\dfrac{3}{2x-1}\le \dfrac{2x^{2}}{2x^{2}-x}-\dfrac{x+1}{x}\)
\item \(\dfrac{2x^{2}}{2x^{2}-x}>1\)
\item \(\dfrac{2x}{2x-1}+\dfrac{x+2}{2x+1}>\dfrac{3}{2}\)
\item \(\dfrac{x^{2}-5x+6}{x^{2}-7x+12}\le~1\)
\end{enumeratea}
\end{multicols}
\end{esercizio}

\begin{esercizio}[\Ast]
\label{ese:21.63}
Determinate~\(\IS\) delle seguenti disequazioni fratte.
\begin{multicols}{2}
\begin{enumeratea}
\spazielenx
 \item \(\dfrac{\dfrac{2}{x+1}}{x^{2}-1}<0\)
\item \(\dfrac{x}{x+1}-\dfrac{4-x}{x+2}\ge \dfrac{2x+1}{x^{2}+3x+2}\)
\item \(\dfrac{3}{2x^{2}-4x-6}-\dfrac{x-2}{3x+3}<\dfrac{x-1}{2x-6}\)
\item \(\dfrac{1}{2-2x}\cdot \left(\dfrac{x(x-2)}{x-1}-\dfrac{3}{3-3x}\right)>-1\)
\end{enumeratea}
\end{multicols}
\end{esercizio}

\begin{esercizio}[\Ast]
\label{ese:21.64}
Determinate~\(\IS\) delle seguenti disequazioni fratte.

\begin{enumeratea}
 \item \(-{\dfrac{2}{27-3x^{2}}}-\dfrac{x+1}{2x-6}+\dfrac{3-2x}{6x-18}<-{\dfrac{3}{x^{2}-9}}+4\dfrac{x-3}{18-2x^{2}}\)
\item \(\dfrac{2}{x^{2}-3x+2}-\dfrac{x}{x-2}<\dfrac{x-1}{x-1}-\dfrac{1}{3x-x^{2}-2}+\dfrac{2-x}{4x-4}\)
\item \(\dfrac{(x-2)(x+4)(x^{2}+5x+6)}{(x^{2}-9)(-4-7x^{2})(x^{2}-6x+8)(x^{2}+4)}<0\)
\end{enumeratea}
\end{esercizio}

\begin{esercizio}
\label{ese:21.65}
Dopo aver ridotto ai minimi termini la frazione
\(f=\dfrac{3x^{4}-2x^{3}+3x^{2}-2x}{6x^{2}-x-7}\), completa;

 \begin{enumeratea}
 \item \(f>0\) per~\(x<-1\) oppure \dotfill
 \item \(f=0\) per \dotfill
 \item \(f<0\) per \dotfill
 \end{enumeratea}
\end{esercizio}

\begin{esercizio}
\label{ese:21.66}
Determinate il segno delle frazioni, dopo averle ridotte ai minimi termini.
\[f_{1}=\dfrac{1-a^{2}}{2+3a};\quad f_{2}=\dfrac{a^{3}-5a^{2}-3+7a}{9-6a+a^{2}};\quad f_{3}=\dfrac{11m-m^{2}+26a}{(39-3m)(m^{2}+4m+4)}.\]
\end{esercizio}

\begin{esercizio}[\Ast]
\label{ese:21.67}
Determinate~\(\IS\) delle seguenti disequazioni fratte.
\begin{multicols}{2}
\begin{enumeratea}{\longarray
 \item \(\left\{\begin{array}{l}
  \dfrac{2-x}{3x^{2}+x}\ge~0\\
  x^{2}-x-6\ge~0\\
  x^{2}-4\le~0
        \end{array}\right.;\)
\item \(\left\{\begin{array}{l}
        \dfrac{x^{2}-4x+4}{9-x^{2}>0}\\
        x^{2}-3x\le~0
       \end{array}\right.;\)
\item \(\left\{\begin{array}{l}
           \dfrac{1}{x-2}+\dfrac{3}{x+2}<0\\
           \dfrac{2-x}{5x-15}\le\dfrac{5x-1}{2x-6}
           \end{array}\right.;\)
\item \(\left\{\begin{array}{l}
           \dfrac{4}{8-4x}-\dfrac{6}{2x-4}<0\\
           \dfrac{x}{x-2}-\dfrac{6}{x^{3}-8}>1
           \end{array}\right.;\)
\item \(\left\{\begin{array}{l}
           \left(1+\dfrac{2}{x-2}\right)\left(1-\dfrac{2}{x-2}\right)<\dfrac{x-4}{2-x}\\
           \left(\dfrac{2-x}{x^{2}-6x+9}+\dfrac{2+x}{x^{2}-9}\right)\cdot{\dfrac{x^{3}-27}{2x}}>0
           \end{array}\right..\)}
\end{enumeratea}
\end{multicols}
\end{esercizio}

\begin{esercizio}[\Ast]
\label{ese:21.68}
Determinate~\(\IS\) delle seguenti disequazioni fratte.
\begin{multicols}{2}
\begin{enumeratea}{\longarray
 \item \(\left\{\begin{array}{l}
  \left(1-\dfrac{1}{x}\right)+3\left(\dfrac{2}{x}+1\right)>\dfrac{13}{2}\\
  \dfrac{7+x}{2x}>\dfrac{2-x}{1-2x}
   \end{array}\right.;\)
\item \(\left\{\begin{array}{l}
  \dfrac{x^{2}-2x-3}{2x^{2}-x-1}\ge~0\\
  \dfrac{4x-1-3x^{2}}{x^{2}-4}\le~0
        \end{array}\right.;\)
\item \(\left\{\begin{array}{l}
  x^{2}-3x+2\le0\\
  \dfrac{6}{2+x}-\dfrac{x+2}{x-2}>\dfrac{x^{2}}{4-x^{2}}
        \end{array}\right.;\)
\item \(\left\{\begin{array}{l}
  x^{2}+1\le -2x\\
  3x-1<2\left(x-\dfrac{1}{2}\right)
  \end{array}\right..\)}
\end{enumeratea}
\end{multicols}
\end{esercizio}

\begin{esercizio}
\label{ese:21.69}
Motivare la verità o la falsità delle seguenti
proposizioni riferite alle frazioni.
\begin{multicols}{3}
\noindent\[f_{1}=\frac{a^{3}-81a}{81-a^{2}},\]
\[f_{2}=\frac{7a^{2}+7}{3+3a^{4}+6a^{2}},\]
\[f_{3}=\frac{20a-50a^{2}-2}{4a-20a^{2}},\]
\[f_{4}=\frac{a^{4}}{2a^{4}+a^{2}},\]
\[f_{5}=\frac{1-4a^{2}}{2-8a+8a^{2}},\]
\[f_{6}=\frac{2a^{2}+a^{3}+a}{2a^{2}-a^{3}-a}.\]
\end{multicols}
\begin{enumeratea}
\TabPositions{11cm}
\item \(f_{1}\) per qualunque valore positivo della variabile è negativa \tab\boxV\quad\boxF
\item \(f_{2}\) è definita per qualunque valore attribuito alla variabile \tab\boxV\quad\boxF
\item \(f_{3}\) è positiva nell'insieme~\(\IS=\left\{a\in \insR/a<0\vee a>\frac{1}{5}\right\}\) \tab\boxV\quad\boxF
\item \(f_{4}\) è positiva per qualunque valore reale attribuito alla variabile \tab\boxV\quad\boxF
\item nell'intervallo~\({[}-\frac{1}{2},\frac{1}{2}{[}\), \(f_{5}\) non si annulla \tab\boxV\quad\boxF
\item \(f_{6}\) è negativa per qualunque valore dell'insieme~\(K=\insR-\{-1,0,1\}\) \tab\boxV\quad\boxF
\end{enumeratea}
\end{esercizio}

\subsection{Risposte}

\begin{multicols}{2}
 \paragraph{\ref{ese:21.10}} a)~\(x<\frac{3}{2}\),\quad b)~\(x>\frac{3}{2}\),\quad
c)~\(x\le \frac{4}{3}\),\quad d)~\(x\ge -{\frac{4}{5}}\),\quad
e)~\(\insR\),\quad f)~\(\emptyset \),\quad
g)~\(x<3\),\quad \protect\\ h)~\(x\ge -3\)

\paragraph{\ref{ese:21.11}} a)~\(x\le~1\),\quad b)~\(x\le~0\),\quad
c)~\(x\le~5\),\quad d)~\(\emptyset \),\quad
e)~\(\insR\),\quad f)~\(\insR\),\quad
g)~\(\insR \),\quad h)~\(\emptyset \)

\paragraph{\ref{ese:21.12}} a)~\(\emptyset \),\quad b)~\(\insR\),\quad
c)~\(\emptyset \),\quad d)~\(x\le -{\frac{10}{3}}\),\quad
e)~\(x<0\),\quad f)~\(x\ge~0\),\quad
g)~\(x\le \frac{5}{3}\),\quad h)~\(x\le -{\frac{8}{3}}\)

\paragraph{\ref{ese:21.13}} a)~\(x\ge~0\),\quad b)~\(x\le -{\frac{3}{4}}\),\quad
c)~\(x\le~0\),\quad d)~\(x\le -{\frac{1}{2}}\),\quad
e)~\(x\ge -{\frac{1}{6}}\),\quad f)~\(x\ge -{\frac{27}{2}}\),\quad
\protect\\g)~\(x>-{\frac{27}{5}}\),\quad h)~\(\insR\)

\paragraph{\ref{ese:21.14}} a)~\(x<-{\frac{3}{4}}\),\quad b)~\(\insR\),\quad
c)~\(x\ge -{\frac{13}{6}}\),\quad d)~\(x>\frac{3}{2}\),\quad
e)~\(x>1\),\quad f)~\(x\ge~0\),\quad\protect\\
g)~\(\{x\in\insR/x<1\}=(-\infty,1)\),\quad h)~\(x<\frac{13}{2}\)

\paragraph{\ref{ese:21.15}} a)~\(\insR\),\quad b)~\(x>-{\frac{10}{111}}\),\quad
c)~\(\emptyset \),\quad d)~\(\insR\)

\paragraph{\ref{ese:21.16}} \(x>5\)

\paragraph{\ref{ese:21.17}} \(x\le -2/3\)

\paragraph{\ref{ese:21.18}} Massimo~\(294\unit{km}\)

\paragraph{\ref{ese:21.20}} Meno di~3 minuti.

\paragraph{\ref{ese:21.21}} 14

\paragraph{\ref{ese:21.23}} \(x>11\)

\paragraph{\ref{ese:21.24}} Almeno~9.

\paragraph{\ref{ese:21.26}} Più di~\(300\unit{km}\)

\paragraph{\ref{ese:21.28}} \(x>310\unit{cm}\)

\paragraph{\ref{ese:21.29}} \(\frac{7}{5}\unit{cm}<x<\frac{17}{5}\unit{cm}\)

\paragraph{\ref{ese:21.30}} \(0^{\circ}<\alpha<45^{\circ}\)

\paragraph{\ref{ese:21.31}} \(h\le \frac{150}{7}m\)

\paragraph{\ref{ese:21.32}} Il lato minore tra~\(10\unit{m}\) e~\(100\unit{m}\), il lato maggiore tra~\(20\unit{m}\) e~\(200\unit{m}\)

\paragraph{\ref{ese:21.34}} \(x = 0\)

\paragraph{\ref{ese:21.37}} a)~\(\emptyset \),\quad b)~\(-{\frac{4}{5}}\le x\le\frac{4}{3}\),\quad c)~\(\emptyset \),\quad
\protect\\ d)~\(x<-{\frac{7}{3}}\)

\paragraph{\ref{ese:21.38}} a)~\(x\le~3\),\quad b)~\(-6\le x\le~1\),\quad c)~\(\emptyset \),\quad d)~\(\insR\)

\paragraph{\ref{ese:21.39}} a)~\(x<0\),\quad b)~\(\emptyset \),\quad c)~\(0\le x<1\),\quad d)~\(x\ge~0\)

\paragraph{\ref{ese:21.40}} a)~\(0<x<5\),\quad b)~\(\insR\),\quad
\protect\\ c)~\(-{\frac{13}{6}}\le x<-{\frac{3}{4}}\),\quad d)~\(-{\frac{17}{7}}<x<6\)

\paragraph{\ref{ese:21.41}} a)~\(x\ge~2\),\quad b)~\(x>\frac{3}{2}\),\quad c)~\(x>\frac{9}{10}\),\quad d)~\(x>\frac{1}{2}\)

\paragraph{\ref{ese:21.44}} a)~\(x\le -2\vee x\ge~3\),\quad b)~\(x<0\vee x>2\),\quad
c)~\(x<-{\frac{2}{3}}\vee x>\frac{2}{3}\),\quad d)~\(x\ge~0\vee~2\le x\le~3\)

\paragraph{\ref{ese:21.45}} a)~\(x\le -1\vee~1\le x\le4\),\quad b)~\(1<x<2\vee~3<x<4\),\quad
c)~\(-4\le x\le~4\),\quad d)~\(0<x<\frac{1}{2}\)

\paragraph{\ref{ese:21.46}} a)~\(x\le -3\vee x\ge~3\),\quad b)~\(-16\le x\le -1\),\quad
c)~\(x\le -2\vee x\ge~2\),\quad d)~\(\emptyset \)

\paragraph{\ref{ese:21.47}} a)~\(\insR\),\quad b)~\(2<x<3\),\quad
c)~\(-4\le x\le~1\),\quad d)~\(x>1\)

\paragraph{\ref{ese:21.48}} a)~\(-1<x<0\vee \frac{1}{2}<x<1\),\quad \protect\\ b)~\(x<-1\),\quad
c)~\(-1\le x\le~1\vee x\ge~2\),\quad \protect\\ d)~\(x<-3\vee x>-1\wedge x\neq~0\)

\paragraph{\ref{ese:21.49}} a)~\(0<x<4\wedge x\neq~3\),\quad b)~\(-2<x<2\),\quad
c)~\(a<-{\frac{1}{2}}\wedge a\neq -1\),\quad d)~\(-1<x<2\vee x>5\),\quad e)~\(-3<x<-1\vee~1<x<2\)

\paragraph{\ref{ese:21.50}} \(-2<y<1\vee y>3\)

\paragraph{\ref{ese:21.51}} \(-2<a<2\)

\paragraph{\ref{ese:21.52}} a)~\(3\le x<5\),\quad b)~\(x\le -6\vee x\ge~3\),\quad
c)~\(-{\frac{1}{2}}<x<\frac{1}{2}\)

\paragraph{\ref{ese:21.53}} a)~\(-{\frac{1}{3}}<a\le -{\frac{1}{7}}\vee \frac{1}{7}\le a<\frac{1}{3}\),\quad
b)~\(\frac{1}{2}<x<3\)

\paragraph{\ref{ese:21.55}} a)~\(x<2\vee x>3\),\quad b)~\(x\le -4\vee~2<x<4\),\quad
c)~\(x<1\vee x>4\),\quad d)~\(x<\frac{6}{5}\vee x\ge\frac{11}{9}\)

\paragraph{\ref{ese:21.56}} a)~\(x\le -8\vee x>2\),\quad b)~\(-{\frac{1}{2}}\le x\le~2\),\quad
c)~\(x<-4\vee\frac{2}{3}\le x<3\),\quad \protect\\ d)~\(-45\le x<-9\vee x>-3\)

\paragraph{\ref{ese:21.57}} a)~\(2<x\le \frac{7}{2}\vee x>4\),\quad
b)~\(x<\frac{8}{13}\),\quad
c)~\(x<2\vee~2<x<\frac{5}{2}\),\quad d)~\(x<-3\vee x>2\)

\paragraph{\ref{ese:21.58}} a)~\(x\le \frac{7}{5}\wedge x\neq-3\),\quad
b)~\(-1<x\le~1\),\quad
\protect\\ c)~\(x<-3\vee\frac{1}{2}<x<2\),\quad d)~\(x<\frac{1}{2}\vee x>2\)

\paragraph{\ref{ese:21.59}} a)~\(x\le~2\vee x>4\),\quad b)~\(x\le \frac{1}{3}\vee x>3\),\quad
c)~\(x<2\vee~3<x\le~4\vee x\ge~6\),\quad \protect\\ d)~\(x\le~2\vee \frac{5}{2}\le x<3\vee x>4\)
\end{multicols}
\paragraph{\ref{ese:21.60}} a)~\(x\le -4\vee -1\le x<1\vee~2\le x<3\vee x>5\),\quad
\protect\\ b)~\(-3\le x\le -2\vee~0<x<2\vee~5\le x<6\),\quad
c)~\(x<-6\vee~0<x\le3\vee x\ge~5\text{ con }x\neq~2\),\quad d)~\(-3\le x\le -2\vee~1<x\le~3\vee x>4\)

\paragraph{\ref{ese:21.61}} a)~\(x>-2\),\quad b)~\(x<-1\),\quad
c)~\(x<-3\vee -1<x<2\vee x>\frac{5}{2}\),\quad
\protect\\ d)~\(x\le -6\vee -2<x<-1\)

\paragraph{\ref{ese:21.62}} a)~\(x<0\vee\frac{1}{4}\le x<\frac{1}{2}\),\quad b)~\(x<\frac{1}{2}\wedge x\neq~0\),\quad
c)~\(-\frac{1}{2}<x<\frac{1}{10}\vee x>\frac{1}{2}\),\quad d)~\(x<4\wedge x\neq~3\)

\paragraph{\ref{ese:21.63}} a)~\(x<-1\vee -1<x<1\),\quad b)~\(x<-2\vee x\ge \frac{5}{2}\),\quad
c)~\(x<-1\vee~0<x<2\vee x>3\),\quad d)~\(\insR-\{1\}\)

\paragraph{\ref{ese:21.64}} a)~\(x<-3\vee x>3\),\quad b)~\(x<0\vee~1<x<\frac{12}{7}\vee x>2\),\quad
\protect\\ c)~\(x<-4\vee -2<x<3\vee x>4\text{ con }x\neq2\)

\paragraph{\ref{ese:21.67}} a)~\(\left\{x\in\insR/x=-2\right\}\),\quad b)~\(\left\{x\in \insR/0\le x<3\text{ con }x\neq~2\right\}\),\quad
c)~,\(x<-2\)\quad d)~\(x>2\),\quad
\protect\\ e)~\(1<x<3\wedge x\neq~2\)

\paragraph{\ref{ese:21.68}} a)~\(0<x<\frac{7}{17}\vee\frac{1}{2}<x<2\),\quad b)~\(x<-2\vee \frac{1}{3}\le x<1\vee x\ge~3\),\quad
\protect\\ c)~\(1\le x<2\),\quad d)~\(\emptyset \)
