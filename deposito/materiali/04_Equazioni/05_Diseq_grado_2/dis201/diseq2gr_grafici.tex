% (c) 2017 Daniele Zambelli - daniele.zambelli@gmail.com
% 
% Tutti i grafici per il capitolo relativo alle disequazioni di secondo grado
%

\newcommand{\asseretta}[5]{% asse x e retta
  \def \slung{#1}
  \def \zero{#2}
  \def \segnoa{#3}
  \def \segnob{#4}
  \def \emme{#5}
  \disegno{      
    \assex{-\slung}{+\slung}{0}
    \tkzInit[xmin=-5,xmax=+5,ymin=-1.3,ymax=+1.3]
    \tkzFct[domain=-5:5,thick,color=Maroon!50!black]{\emme*x};
    \foreach \x in {-3, -1} 
      \node at (\x, 0) [above, yshift=-3pt] {\(\segnoa\)};
    \foreach \x in {1, 3} 
      \node at (\x, 0) [above, yshift=-3pt] {\(\segnob\)};
    \draw[fill=orange] (0,0) circle (1.5pt) node[below] {\(\zero\)};
  }
}

\newcommand{\rettammi}[2]{% retta m<0
  \asseretta{#1}{#2}{+}{-}{-.5}
}

\newcommand{\rettamma}[2]{% retta m>0
  \asseretta{#1}{#2}{-}{+}{+.5}
}

\newcommand{\parabolaamidmi}{% Parabola a<0 e Delta<0
  \disegno{      
    \assex{-5}{+5}{0}
    \tkzInit[xmin=-5,xmax=+5,ymin=-1.3,ymax=+1.3]
    \tkzFct[domain=-5:5,thick,color=Maroon!50!black]{-.125*x*x-.5};
    \foreach \x in {-4,...,4}
      \node at (\x, 0) [above, yshift=-3pt] {\(-\)};
  }
}

\newcommand{\parabolaamidz}[1]{% Parabola a<0 e Delta=0
  \def \zero{#1}
  \disegno{      
    \assex{-5}{+5}{0}
    \tkzInit[xmin=-5,xmax=+5,ymin=-1.3,ymax=+1.3]
    \tkzFct[domain=-5:5,thick,color=Maroon!50!black]{-.125*x*x};
    \foreach \x in {-4, -2, 2, 4} 
      \node at (\x, 0) [above, yshift=-3pt] {\(-\)};
    \draw[fill=orange] (0,0) circle (1.5pt) node[below] {\(\zero\)};
  }
}

\newcommand{\parabolaamidma}[2]{% Parabola a<0 e Delta>0
  \def \zeroa{#1}
  \def \zerob{#2}
  \disegno{      
    \assex{-5}{+5}{0}
    \tkzInit[xmin=-5,xmax=+5,ymin=-1.3,ymax=+1.3]
    \tkzFct[domain=-5:5,thick,color=Maroon!50!black]{-.25*x*x+1};
    \foreach \x in {-3, +3} 
      \node at (\x, 0) [above, yshift=-3pt] {\(-\)};
    \node at (0, 0) [above, yshift=-3pt] {\(+\)};
    \draw[fill=orange] (-2,0) circle (1.5pt) node[below] {\(\zeroa\)};
    \draw[fill=orange] (+2,0) circle (1.5pt) node[below] {\(\zerob\)};
  }
}

\newcommand{\parabolaamadmi}{% Parabola a>0 e Delta<0
  \disegno{      
    \assex{-5}{+5}{0}
    \tkzInit[xmin=-5,xmax=+5,ymin=-1.3,ymax=+1.3]
    \tkzFct[domain=-5:5,thick,color=Maroon!50!black]{+.125*x*x+.5};
    \foreach \x in {-4,...,4}
      \node at (\x, 0) [above, yshift=-3pt] {\(+\)};
  }
}

\newcommand{\parabolaamadz}[1]{% Parabola a>0 e Delta=0
  \def \zero{#1}
  \disegno{      
    \assex{-5}{+5}{0}
    \tkzInit[xmin=-5,xmax=+5,ymin=-1.3,ymax=+1.3]
    \tkzFct[domain=-5:5,thick,color=Maroon!50!black]{+.125*x*x};
    \foreach \x in {-4, -2, 2, 4} 
      \node at (\x, 0) [above, yshift=-3pt] {\(+\)};
    \draw[fill=orange] (0,0) circle (1.5pt) node[below] {\(\zero\)};
  }
}

\newcommand{\parabolaamadma}[2]{% Parabola a>0 e Delta>0
  \def \zeroa{#1}
  \def \zerob{#2}
  \disegno{      
    \assex{-5}{+5}{0}
    \tkzInit[xmin=-5,xmax=+5,ymin=-1.3,ymax=+1.3]
    \tkzFct[domain=-5:5,thick,color=Maroon!50!black]{+.25*x*x-1};
    \foreach \x in {-3, +3} 
      \node at (\x, 0) [above, yshift=-3] {\(+\)};
    \node at (0, 0) [above, yshift=-3pt] {\(-\)};
    \draw[fill=orange] (-2,0) circle (1.5pt) node[below] {\(\zeroa\)};
    \draw[fill=orange] (+2,0) circle (1.5pt) node[below] {\(\zerob\)};
  }
}

% \newcommand{\inti}[7]{% Intervallo interno
%   % \disegno{\inti{0}{-2}{+2}{a}{b}{white}{black}}
%   \def \posy{#1}
%   \def \posa{#2}
%   \def \posb{#3}
%   \def \labela{#4}
%   \def \labelb{#5}
%   \def \inta{#6}
%   \def \intb{#7}
%   \coordinate (a) at (\posa, \posy);
%   \coordinate (b) at (\posb, \posy);
%   \draw [blue, thick, decorate, decoration=snake] (a) -- (b);
%   \draw[fill=\inta] (a) circle (2pt) node [above] {\(\labela\)};
%   \draw[fill=\intb] (b) circle (2pt) node [above] {\(\labelb\)};
% }
% 
% \newcommand{\rayl}[5]{% semiretta sinistra
%   % \disegno{\rayl{0}{5}{-2}{p}{white}}
%   \def \posy{#1}
%   \def \lung{#2}
%   \def \posp{#3}
%   \def \labelp{#4}
%   \def \intp{#5}
%     \coordinate (p) at (\posp, \posy);
%     \draw [blue, thick, decorate, decoration=snake] (-\lung, \posy) -- (p);
%     \draw[fill=\intp] (p) circle (2pt) node [above] {\(\labelp\)};
% }
% 
% \newcommand{\rayr}[5]{% semiretta destra
%   % \disegno{\rayr{0}{5}{-2}{p}{white}}
%   \def \posy{#1}
%   \def \lung{#2}
%   \def \posp{#3}
%   \def \labelp{#4}
%   \def \intp{#5}
%     \coordinate (p) at (\posp, \posy);
%     \draw [blue, thick, decorate, decoration=snake] (p) -- (+\lung, \posy);
%     \draw[fill=\intp] (p) circle (2pt) node [above] {\(\labelp\)};
% }
% 
% \newcommand{\inte}[8]{% Intervallo esterno
%   % \disegno{\inte{0}{5}{-2}{+2}{a}{b}{white}{black}}
% %   \def \posy{#1}
% %   \def \lung{#2}
% %   \def \posa{#3}
% %   \def \posb{#4}
% %   \def \labela{#5}
% %   \def \labelb{#6}
% %   \def \inta{#7}
% %   \def \intb{#8}
%   \rayl{#1}{#2}{#3}{#5}{#7}
%   \rayr{#1}{#2}{#4}{#6}{#8}
% }

\newcommand{\intpunto}[4]{% Punto
  \def \slung{#1}
  \def \posp{#2}
  \def \labelp{#3}
  \def \intp{#4}
  \disegno{
    \assex{-\slung}{+\slung}{0}
    \draw[fill=\intp] (\posp, 0) circle (2pt) node [above] {\(\labelp\)};
  }
}

\newcommand{\intid}[7]{% Intervallo ]a; b[ con disegno e asse
  % \intid{5}{-2}{+2}{a}{b}{white}{black}
  \disegno{
    \assex{-#1}{+#1}{0}
    \inti{0}{#2}{#3}{#4}{#5}{#6}{#7}
  }
}

\newcommand{\inted}[7]{%  ]-infty; a[ u ]b; +infty[ con disegno e asse
  % \inted{5}{-2}{+2}{a}{b}{white}{black}
  \disegno{
    \assex{-#1}{+#1}{0}
    \inte{0}{#1}{#2}{#3}{#4}{#5}{#6}{#7}
  }
}

\newcommand{\intiaa}[5]{% Intervallo ]a; b[
  % \intiaa{5}{-3}{3}{a}{b}
  \intid{#1}{#2}{#3}{#4}{#5}{white}{white}
}

\newcommand{\intica}[5]{% Intervallo ]a; b[
  \intid{#1}{#2}{#3}{#4}{#5}{black}{white}
}

\newcommand{\intiac}[5]{% Intervallo ]a; b[
  \intid{#1}{#2}{#3}{#4}{#5}{white}{black}
}

\newcommand{\inticc}[5]{% Intervallo ]a; b[
  \intid{#1}{#2}{#3}{#4}{#5}{black}{black}
}

\newcommand{\inteaa}[5]{% Intervallo ]a; b[
  % \inteaa{5}{-3}{3}{a}{b}
  \inted{#1}{#2}{#3}{#4}{#5}{white}{white}
}

\newcommand{\inteca}[5]{% Intervallo ]a; b[
  \inted{#1}{#2}{#3}{#4}{#5}{black}{white}
}

\newcommand{\inteac}[5]{% Intervallo ]a; b[
  \inted{#1}{#2}{#3}{#4}{#5}{white}{black}
}

\newcommand{\intecc}[5]{% Intervallo ]a; b[
  \inted{#1}{#2}{#3}{#4}{#5}{black}{black}
}

\newcommand{\puntoa}[3]{% Intervallo ]a; b[
  \intpunto{#1}{#2}{#3}{white}
}

\newcommand{\puntoc}[3]{% Intervallo [a; a]
  \intpunto{#1}{#2}{#3}{black}
}

\newcommand{\intv}[1]{% Intervallo vuoto
  \def \lung{#1}
  \disegno{
    \assex{-\lung / 2}{+\lung / 2}{0}
  }
}

\newcommand{\intr}[1]{% Intervallo vuoto
  \def \lung{#1}
  \disegno{
    \assex{-\lung / 2}{+\lung / 2}{0}
    \draw [blue, thick, decorate, decoration=snake] 
          (-\lung / 2, 0) -- (+\lung / 2, 0);
  }
}

\newcommand{\segnoprodottoa}{% Segno del prodotto di due funzioni
  \disegno{
    \trespolosegni{5}{1.2}{3}{3}
    \node at (-2.5, 0) [above] {\(-2\)};
    \node at (-0, 0) [above] {\(\frac{3}{2}\)};
    \node at (+2.5, 0) [above] {\(+2\)};
    \node at (-5, -1.2) [above left] {\(x^2-4\)};
    \node at (-5, -2.4) [above left] {\(-2x+3\)};
    \node at (-5, -3.6) [above left] {\(f(x)\)};
    \foreach \x/\y in {-2.5/-.7, +2.5/-.7, 0/-1.9, 
                       -2.5/-3.1, +2.5/-3.1, 0/-3.1}
      \cerchietto{\x}{\y}
    \foreach \x/\s in {-3.5/+, -1.25/-, 1.25/-, 3.5/+}
      \node at (\x, -1.2) [above] {\(\s\)};
    \foreach \x/\s in {-3.5/+, -1.25/+, 1.25/-, 3.5/-}
      \node at (\x, -2.4) [above] {\(\s\)};
    \foreach \x/\s in {-3.5/+, -1.25/-, 1.25/+, 3.5/-}
      \node at (\x, -3.6) [above] {\(\s\)};
  }
}

\newcommand{\solprodottoa}{% Soluzione disequazione prodotto di funzioni
  \disegno{
    \assex{-4}{+4}{0}
    \inti{0}{-2.5}{0}{-2}{\frac{3}{2}}{black}{black}
    \rayr{0}{4}{2}{+2}{black}
  }
}

\newcommand{\segnoprodottob}{% Segno del prodotto di due funzioni
  \disegno{
    \trespolosegni{5}{1.2}{3}{3}
    \node at (-5, -1.2) [above left] {\(3x^2-8x-3\)};
    \node at (-5, -2.4) [above left] {\(x\)};
    \node at (-5, -3.6) [above left] {\(f(x)\)};
  }
}

\newcommand{\segnofrazionea}{% Segno del prodotto di due funzioni
  \disegno{
    \trespolosegni{5}{1.2}{3}{3}
    \node at (-2.5, 0) [above] {\(-4\)};
    \node at (-0, 0) [above] {\(7\)};
    \node at (+2.5, 0) [above] {\(29\)};
    \node at (-5, -1.2) [above left] {\(-x +29\)};
    \node at (-5, -2.4) [above left] {\(x^2-3x-28\)};
    \node at (-5, -3.6) [above left] {\(f(x)\)};
    \foreach \x/\y in {2.5/-.7, 2.5/-3.1} \cerchietto{\x}{\y}
    \foreach \x/\y in {-2.5/-1.9, 0/-1.9, 
                       -2.5/-3.1, 0/-3.1}
      \crocetta{\x}{\y}
    \foreach \x/\s in {-3.5/+, -1.25/+, 1.25/+, 3.5/-}
      \node at (\x, -1.2) [above] {\(\s\)};
    \foreach \x/\s in {-3.5/+, -1.25/-, 1.25/+, 3.5/+}
      \node at (\x, -2.4) [above] {\(\s\)};
    \foreach \x/\s in {-3.5/+, -1.25/-, 1.25/+, 3.5/-}
      \node at (\x, -3.6) [above] {\(\s\)};
  }
}

\newcommand{\solfrazionea}{% Soluzione disequazione prodotto di funzioni
  \disegno{
    \assex{-4}{+4}{0}
    \rayl{0}{4}{-2}{-4}{white}
    \inti{0}{0}{2}{7}{29}{white}{black}
  }
}

\newcommand{\segnofrazioneb}{% Segno del quoziente di due funzioni
  \disegno{
    \trespolosegni{6}{1.2}{3}{4}
    \node at (-6, -1.2) [above left] {\(-2x^2+x+3\)};
    \node at (-6, -2.4) [above left] {\(4x^2-1\)};
    \node at (-6, -3.6) [above left] {\(f(x)\)};
  }
}

\newcommand{\segnofrazionec}{% Segno del quoziente di due funzioni
  \disegno{
    \trespolosegni{6}{1.2}{3}{4}
    \node at (-6, -1.2) [above left] {\(-6x^2 +2x +1\)};
    \node at (-6, -2.4) [above left] {\(-2x^2 +x +1\)};
    \node at (-6, -3.6) [above left] {\(f(x)\)};
  }
}

\newcommand{\segnosistemaaa}{% Segno prima disequazione del primo sistema
  \disegno{
    \trespolosegni{5}{1.2}{3}{3}
    \node at (-2.5, 0) [above] {\(-3\)};
    \node at (-0, 0) [above] {\(-\frac{3}{2}\)};
    \node at (+2.5, 0) [above] {\(+3\)};
    \node at (-5, -1.2) [above left] {\(x^2-9\)};
    \node at (-5, -2.4) [above left] {\(2x+3\)};
    \node at (-5, -3.6) [above left] {\(f(x)\)};
    \foreach \x/\y in {2.5/-.7, 2.5/-3.1} \cerchietto{\x}{\y}
    \foreach \x/\y in {-2.5/-.7,          +2.5/-.7,
                                  0/-1.9, 
                       -2.5/-3.1, 0/-3.1, +2.5/-3.1}
      \cerchietto{\x}{\y}
    \foreach \x/\s in {-3.5/+, -1.25/-, 1.25/-, 3.5/+}
      \node at (\x, -1.2) [above] {\(\s\)};
    \foreach \x/\s in {-3.5/-, -1.25/-, 1.25/+, 3.5/+}
      \node at (\x, -2.4) [above] {\(\s\)};
    \foreach \x/\s in {-3.5/-, -1.25/+, 1.25/-, 3.5/+}
      \node at (\x, -3.6) [above] {\(\s\)};
    \rayl{-3.6}{5}{-2.5}{}{white}
    \inti{-3.6}{0}{2.5}{}{}{white}{white}
      
  }
}

\newcommand{\segnosistemaab}{% Segno del prodotto di due funzioni
  \disegno{
    \trespolosegni{6}{1.2}{3}{4}
    \node at (-3.6, 0) [above] {\(-5\)};
    \node at (-1.2, 0) [above] {\(-\frac{3}{2}\)};
    \node at (+1.2, 0) [above] {\(+\frac{3}{2}\)};
    \node at (+3.6, 0) [above] {\(+3\)};
    \node at (-6, -1.2) [above left] {\(-x^2 -2x+15\)};
    \node at (-6, -2.4) [above left] {\(4x^2-9\)};
    \node at (-6, -3.6) [above left] {\(f(x)\)};
    \foreach \x/\y in {-3.6/-.7, +3.6/-.7, 
                       -3.6/-3.1, +3.6/-3.1} \cerchietto{\x}{\y}
    \foreach \x/\y in {-1.2/-1.9, +1.2/-1.9, 
                       -1.2/-3.1, +1.2/-3.1}
      \crocetta{\x}{\y}
    \foreach \x/\s in {-4.7/-, -2.3/+, 0/+, 2.3/+, 4.7/-}
      \node at (\x, -1.2) [above] {\(\s\)};
    \foreach \x/\s in {-4.7/+, -2.3/+, 0/-, 2.3/+, 4.7/+}
      \node at (\x, -2.4) [above] {\(\s\)};
    \foreach \x/\s in {-4.7/-, -2.3/+, 0/-, 2.3/+, 4.7/-}
      \node at (\x, -3.6) [above] {\(\s\)};
    \rayl{-3.6}{6}{-3.6}{}{black}
    \inti{-3.6}{-1.2}{+1.2}{}{}{white}{white}
    \rayr{-3.6}{6}{+3.6}{}{black}
  }
}

\newcommand{\sistemaa}{% Intersezione soluzioni sistema a
  \disegno{
    \trespolosegni{6}{1.2}{3}{5}
    \node at (-4, 0) [above] {\(-5\)};
    \node at (-2, 0) [above] {\(-3\)};
    \node at (0, 0) [above] {\(-\frac{3}{2}\)};
    \node at (+2, 0) [above] {\(+\frac{3}{2}\)};
    \node at (+4, 0) [above] {\(+3\)};
    \node at (-6, -1.2) [left] {\(D_1\)};
    \node at (-6, -2.4) [left] {\(D_2\)};
    \node at (-6, -3.6) [left] {\(S\)};
    \rayl{-1.2}{6}{-2}{}{white}
    \inti{-1.2}{0}{4}{}{}{white}{white}
    \rayl{-2.4}{6}{-4}{}{black}
    \inti{-2.4}{0}{2}{}{}{white}{white}
    \rayr{-2.4}{6}{+4}{}{black}
    \rayl{-3.6}{6}{-4}{}{black}
    \inti{-3.6}{0}{2}{}{}{white}{white}
  }
}

\newcommand{\segnosistemaba}{% Segno prima disequazione del primo sistema
  \disegno{
    \trespolosegni{3}{1.2}{3}{2}
    \node at (-1, 0) [above] {\(-4\)};
    \node at (+1, 0) [above] {\(+7\)};
    \node at (-3, -1.2) [above left] {\(-x^2 +3x +21\)};
    \node at (-3, -2.4) [above left] {\(x-7\)};
    \node at (-3, -3.6) [above left] {\(f(x)\)};
    \foreach \x/\y in {-1/-.7, +1/-.7, -1/-3.1} \cerchietto{\x}{\y}
    \foreach \x/\y in {+1/-1.9, +1/-3.1} \crocetta{\x}{\y}
    \foreach \x/\s in {-2/-, 0/+, +2/-}
      \node at (\x, -1.2) [above] {\(\s\)};
    \foreach \x/\s in {-2/-, 0/-, +2/+}
      \node at (\x, -2.4) [above] {\(\s\)};
    \foreach \x/\s in {-2/+, 0/-, +2/-}
      \node at (\x, -3.6) [above] {\(\s\)};
    \rayl{-3.6}{3}{-1}{}{black}
      
  }
}

\newcommand{\segnosistemabb}{% Segno del prodotto di due funzioni
  \disegno{
    \trespolosegni{4}{1.2}{3}{3}
    \node at (-2, 0) [above] {\(-5\)};
    \node at (-0, 0) [above] {\(-4\)};
    \node at (+2, 0) [above] {\(+4\)};
    \node at (-4, -1.2) [above left] {\(-x -5\)};
    \node at (-4, -2.4) [above left] {\(-x^2 +16\)};
    \node at (-4, -3.6) [above left] {\(f(x)\)};
    \foreach \x/\y in {-2/-.7, -2/-3.1} \cerchietto{\x}{\y}
    \foreach \x/\y in {0/-1.9, +2/-1.9, 
                       0/-3.1, +2/-3.1}
      \crocetta{\x}{\y}
    \foreach \x/\s in {-3/+, -1/-, 1/-, 3/-}
      \node at (\x, -1.2) [above] {\(\s\)};
    \foreach \x/\s in {-3/-, -1/-, 1/+, 3/-}
      \node at (\x, -2.4) [above] {\(\s\)};
    \foreach \x/\s in {-3/-, -1/+, 1/-, 3/+}
      \node at (\x, -3.6) [above] {\(\s\)};
    \inti{-3.6}{-2}{0}{}{}{black}{white}
    \rayr{-3.6}{4}{+2}{}{white}
  }
}

\newcommand{\sistemab}{% Intersezione soluzioni sistema a
  \disegno{
    \trespolosegni{4}{1.2}{3}{3}
    \node at (-2, 0) [above] {\(-5\)};
    \node at (-0, 0) [above] {\(-4\)};
    \node at (+2, 0) [above] {\(+4\)};
    \node at (-4, -1.2) [left] {\(C_1\)};
    \node at (-4, -2.4) [left] {\(C_2\)};
    \node at (-4, -3.6) [left] {\(S\)};
    \rayl{-1.2}{4}{0}{}{black}
    \inti{-2.4}{-2}{0}{}{}{black}{white}
    \rayr{-2.4}{4}{+2}{}{white}
    \inti{-3.6}{-2}{0}{}{}{black}{white}
  }
}


\newcommand{\parabole}{% Varie parabole nel piano cartesiano
  \disegnod{8}{
    \rcom{-5}{+5}{-4}{+4}{gray!50, very thin, step=1}
    \tkzInit[xmin=-5.3,xmax=+5.3,ymin=-4.3,ymax=+4.3]
    \tkzFct[domain=-1.5:1.5,thick,color=darkgray]{2*x*x-1}
    \node [inner sep=0pt, circle, fill=gray!20] (a) at (-1.1, 2.7) {B};
    \tkzFct[domain=0:3,thick,color=blue]{-2*x*x+6*x-4};
    \node [inner sep=0pt, circle, fill=gray!20] (a) at (.6, -3) {D};
    \tkzFct[domain=-3.3:-.7,thick,color=RedOrange]{-2*x*x-8*x-8.5};
    \node [inner sep=0pt, circle, fill=gray!20] (a) at (-3, -3.5) {E};
    \tkzFct[domain=1.5:4.5,thick,color=purple]{2*x*x-12*x+18};
    \node [inner sep=0pt, circle, fill=gray!20] (a) at (2, 3.7) {C};
    \tkzFct[domain=-4.5:-1.5,thick,color=olive]{2*x*x+12*x+19};
    \node [inner sep=0pt, circle, fill=gray!20] (a) at (-3.8, 3.5) {A};
    
  }
}
