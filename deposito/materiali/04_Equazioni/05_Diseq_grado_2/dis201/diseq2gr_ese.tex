% (c)~2014 Claudio Carboncini - claudio.carboncini@gmail.com
% (c)~2014 Dimitrios Vrettos - d.vrettos@gmail.com
% (c) 2015 Daniele Zambelli daniele.zambelli@gmail.com

% \section{Esercizi}
% 
% \subsection{Esercizi dei singoli paragrafi}
% 
% \subsubsection*{\numnameref{sec:01_}}
% 
% \begin{esercizio}
% \label{ese:D.19}
% testo esercizio
% \end{esercizio}
% 
% \begin{esercizio}\label{ese:03.1}
% Consegna:
%  \begin{enumeratea}
%   \item  
%  \end{enumeratea}
% \end{esercizio}
% 
% \subsection{Esercizi riepilogativi}
% 
% \begin{esercizio}
% \label{ese:D.19}
% testo esercizio
% \end{esercizio}
% 
% \begin{esercizio}\label{ese:03.1}
% Consegna:
%  \begin{enumeratea}
%   \item  
%  \end{enumeratea}
% \end{esercizio}

\def \dy{-.5em}

\section{Esercizi}
\subsection{Esercizi dei singoli paragrafi}
% \subsection*{4.1 - Risoluzione delle disequazioni di secondo grado}
\numnameref{sec:diseq_secondo_grado}

\begin{esercizio}
 \label{ese:4.7}
Rappresentare nel riferimento cartesiano ortogonale le seguenti parabole.
\vspace{\dy}
\begin{multicols}{2}
 \begin{enumeratea}
 \item~\( y=-3x^2+x \)
 \item~\( y=\frac 1 2x-2x+\frac 3 2 \)
 \item~\( y=x^2+x-1 \)
 \item~\( y=x^2-x+1 \)
 \end{enumeratea}
 \end{multicols}
\end{esercizio}

\begin{esercizio}
 \label{ese:4.8}
Rappresentare nel riferimento cartesiano ortogonale le seguenti parabole.
\vspace{\dy}
\begin{multicols}{2}
 \begin{enumeratea}
 \item~\( y=-3x^2+3 \)
 \item~\( y=x^2+4x+3 \)
 \item~\( y=x^2+\frac 3 5 \)
 \item~\( y=-\frac 2 5x^2+4x-\frac 1 5 \)
 \item~\( y=-\frac 1 2 x^2-4x-1 \)
 \end{enumeratea}
 \end{multicols}
\end{esercizio}

\vspace{\dy}
\begin{minipage}{.39 \textwidth}
\begin{esercizio}
 \label{ese:4.9}
Per ciascun grafico di parabola \(y=ax^2+bx+c\) indica il segno del primo 
coefficiente e del discriminante, la natura dei suoi zeri (reali distinti, 
reali coincidenti, non reali), il segno della funzione.
\end{esercizio}
\end{minipage}
\hfill
\begin{minipage}{.59 \textwidth}
\begin{center}
 \scalebox{.7}{\parabole}
\end{center}
\end{minipage}

\begin{esercizio}[\Ast]
 \label{ese:4.1}
Risolvi le seguenti disequazioni di secondo grado.
\vspace{\dy}
\begin{multicols}{2}
 \begin{enumeratea}
 \item~\(x^2-6x\le 0\) \hfill \(\left[0\le x\le 6\right]\)
 \item~\(5x^2>0\) \hfill \(\left[x\neq 0\right]\)
 \item~\(x^2+x>0\) \hfill \(\left[x<-1\vee x>0\right]\)
 \item~\(x^2\le 0\) \hfill \(\left[x=0\right]\)
 \item~\(3x^2\le -1\) \hfill \(\left[\emptyset \right]\)
 \item~\(x^2-9>0\) \hfill \(\left[x_1<-3\vee x>3\right]\)
 \item~\(2x^2-3x+1>0\) \hfill \(\left[x<\frac 1 2\vee x>1\right]\)
 \item~\(-x^2+3x\ge 0\) \hfill \(\left[0\le x\le 3\right]\)
 \item~\(3x^2+x-2>0\) \hfill \(\left[x_1<-1\vee x>\frac 2 3\right]\)
 \item~\(x^2-4>0\) \hfill \(\left[x_1<-2\vee x>2\right]\)
 \item~\(\frac 4 3x^2-\frac 1 3x-1<0\) \hfill \(\left[-\frac 3 4<x<1\right]\)
 \item~\(x^2-8\le 0\) \hfill \(\left[-2\sqrt 2\le x\le 2\sqrt 2\right]\)
%  \item~\(x^2-5x+3\ge 0                          \) ZERI IRRAZIONALI
%  \hfill \(\left[x\le \frac{5-\sqrt{13}} 2\vee x\ge \frac{5+\sqrt{13}} 2\right]\)
 \item~\(x^2-4x+9>0\) \hfill \(\left[\insR\right]\)
 \item~\(x^2-6x+8\le 0\) \hfill \(\left[2\le x\le 4\right]\)
 \item~\(x^2+3x-4\ge 0\) \hfill \(\left[x\le -4\vee x\ge 1\right]\)
 \item~\(x^2-4x-9\le 0\) \hfill \(\left[2-\sqrt{13}\le x\le 2+\sqrt{13}\right]\)
 \item~\(x^2-9x+18<0\) \hfill \(\left[3<x<6\right]\)
 \end{enumeratea}
 \end{multicols}
\end{esercizio}

\begin{esercizio}[\Ast]
 \label{ese:4.4}
Risolvi le seguenti disequazioni di secondo grado.
\vspace{\dy}
\begin{multicols}{2}
 \begin{enumeratea}
 \item~\(x^2-8x+15\ge 0\) \hfill \(\left[x\le 3\vee x\ge 5\right]\)
 \item~\(-2x^2\ge 0\) \hfill \(\left[x=0\right]\)
 \item~\(3x^2-\frac 2 3x-1\le 0\) 
  \hfill \(\left[\frac{1-2\sqrt 7} 9\le x\le \frac{1+2\sqrt 7} 9\right]\)
 \item~\(x^2+5>0\) \hfill \(\left[\insR\right]\)
%  \item~\(x^2+6x-2>0\) \hfill \(\left[x<-3-\sqrt{11}\vee x>-3+\sqrt{11}\right]\)
 \item~\(2x^2+5x+4\le 0\) \hfill \(\left[\emptyset\right]\)
 \item~\(x^2-3x-\frac 5 2<0\) 
  \hfill \(\left[\frac{3-\sqrt{19}} 2<x<\frac{3+\sqrt{19}} 2\right]\)
 \item~\(x^2+1>0\) \hfill \(\left[\insR\right]\)
 \item~\(-x^2+5\le 0\) \hfill \(\left[x\le -\sqrt 5\vee x\ge \sqrt 5\right]\)
 \item~\(x^2+x\ge 0\) \hfill \(\left[x\le -1\vee x\ge 0\right]\)
 \item~\((x+1)^2\ge 0\) \hfill \(\left[\insR\right]\)
 \item~\(x^2>1\) \hfill \(\left[x<-1\vee x>1\right]\)
 \item~\(2x^2-6<0\) \hfill \(\left[-\sqrt 3<x<\sqrt 3\right]\)
 \item~\(-x^2-1\le 0\) \hfill \(\left[\insR\right]\)
 \item~\(9-4x^2\le 0\) \hfill \(\left[\right]\)
 \item~\(3x-2x^2>0\) \hfill \(\left[\right]\)
 \item~ \(x^2\ge 0\) \hfill \(\left[\right]\)
 \item~\(2x^2+4>0\) \hfill \(\left[\right]\)
 \item~\(x^2-x-2>0\) \hfill \(\left[\right]\)
 \item~\(x^2+11x+30\le 0\) \hfill \(\left[\right]\)
 \end{enumeratea}
 \end{multicols}
\end{esercizio}

% \subsection*{4.2 - Risoluzione grafica di una disequazione di secondo grado}
% 
% \begin{esercizio}
%  \label{ese:4.10}
% Risolvere graficamente le seguenti disequazioni di secondo grado.
% \begin{multicols}{2}
%  \begin{enumeratea}
%  \item~\( 2x^2+3x-1<0 \)
%  \item~\( x^2-5x+6\le 0 \)
%  \item~\( x^2-3x-4>0 \)
%  \item~\( x^2-6x+5\ge 0 \)
%  \end{enumeratea}
%  \end{multicols}
% \end{esercizio}
% 
% \begin{esercizio}
%  \label{ese:4.11}
% Risolvere graficamente le seguenti disequazioni di secondo grado.
% \begin{multicols}{2}
%  \begin{enumeratea}
%  \item~\( 6x^2+x-2>0 \)
%  \item~\( 15x^2+x-6\le 0 \)
%  \item~\( -x^2+1\ge 0 \)
%  \item~\( x^2-\frac 1 4>0 \)
%  \end{enumeratea}
%  \end{multicols}
% \end{esercizio}
% 
% \begin{esercizio}
%  \label{ese:4.12}
% Risolvere graficamente le seguenti disequazioni di secondo grado.
% \begin{multicols}{2}
%  \begin{enumeratea}
%  \item~\( x^2-\frac 1 4x\le 0 \)
%  \item~\( x^2+2x\le 0 \)
%  \item~\( x^2+2x+1\le 0 \)
%  \item~\( x^2+x+1<0 \)
%  \end{enumeratea}
%  \end{multicols}
% \end{esercizio}
% 
% \begin{esercizio}[\Ast]
%  \label{ese:4.13}
% Risolvi le disequazioni di secondo grado con il metodo algebrico o con quello 
% grafico.
% \begin{multicols}{3}
%  \begin{enumeratea}
%  \end{enumeratea}
%  \end{multicols}
% \end{esercizio}
% 
% \begin{esercizio}[\Ast]
%  \label{ese:4.14}
% Risolvi le disequazioni di secondo grado con il metodo algebrico o con quello 
% grafico.
% \begin{multicols}{3}
%  \begin{enumeratea}
%  \item~\(-x^2+4x+3>0\)
%  \item~\(x^2+4x+4<0\)
%  \item~\(x^2-x+1<0\)
%  \item~\(x^2-\frac 1 9\ge 0\)
%  \item~\(9x^2+3x-2\le 0\)
%  \item~\(2x^2+5<0\)
%  \end{enumeratea}
%  \end{multicols}
% \end{esercizio}
% 
% \begin{esercizio}[\Ast]
%  \label{ese:4.15}
% Risolvi le disequazioni di secondo grado con il metodo algebrico o con quello 
% grafico.
% \begin{multicols}{2}
%  \begin{enumeratea}
%  \item~\(4x-x^2\ge 0\)
%  \item~\(9x^2+10x+1\le 0\)
%  \item~\(0,01x^2-1>0\)
%  \item~\(1,\bar 6x^2-2x\le 0\)
%  \end{enumeratea}
%  \end{multicols}
% \end{esercizio}
% 
% \begin{esercizio}[\Ast]
%  \label{ese:4.16}
% Risolvi le disequazioni di secondo grado con il metodo algebrico o con quello 
% grafico.
% \begin{multicols}{2}
%  \begin{enumeratea}
%  \item~\(\frac 1 2x^2-\frac 1 8>0\)
%  \item~\(4x^2+\frac 5 3x-1\le 0\)
%  \item~\(x^2+x+\sqrt 2>0\)
%  \item~\(x^2+2\sqrt 2x+2>0\)
%  \end{enumeratea}
%  \end{multicols}
% \end{esercizio}

% \paragraph{4.13.} a)~\(x\le -\frac 3 2\vee x\ge \frac 3 2\),\quad b)~\(0<x<\frac 3 
% 2\),\quad c)~ \(\insR\),\quad d)~ \(\insR\),\; e)~ \(x<-1\vee x>2\),\; f)~ \(-6\le x\le 
% -5\)
% 
% \paragraph{4.14.} a)~\(2-\sqrt 7<x<2+\sqrt 7\),\quad b)~\(\emptyset \) ,\quad 
% c)~\(\emptyset \) ,\quad d)~\(x\le -\frac 1 3\vee x\ge \frac 1 3\),\quad e)~\(-\frac 
% 2 3\le x\le \frac 1 3\),\quad f)~\(\emptyset \)
% 
% \paragraph{4.15.} a)~\(0\le x\le 4\),\quad b)~ \(-1\le x<-\frac 1 9\),\quad 
% c)~\(x<-10\vee x>10\),\quad d)~\(0\le x<\frac 6 5\)
% 
% \paragraph{4.16.} a)~\(x<-\frac 1 2\vee x>\frac 1 2\),\quad b)~\(-\frac 3 4\le x\le 
% \frac 1 3\),\quad c)~\(\insR\),\quad d)~\(\insR-\{\sqrt 2\}\)

\begin{esercizio}[\Ast]
 \label{ese:4.17}
Risolvi le disequazioni di secondo grado.
\vspace{\dy}
 \begin{enumeratea}
 \item~\(12x^2-3\ge 4x(2x-1)\) 
  \hfill \(\left[x\le -\frac 3 2\vee x\ge \frac 1 2\right]\)
 \item~\(2x^2-11x-6\ge 0\) 
  \hfill \(\left[x\le -\frac 1 2\vee x\ge 6\right]\)
 \item~\((3x+1)^2>(2x-1)^2\) 
  \hfill \(\left[x<-2\vee x>0\right]\)
 \item~\((x+1)(x-1)^2>x^3\) 
  \hfill \(\left[-\frac{\sqrt 5+1} 2<x<\frac{\sqrt 5-1} 2\right]\)
 \item~\((x+3)(x+2)<-(x+2)^2\)
  \hfill \(\left[-\frac 5 2<x<-2\right]\)
 \item~\(\frac{x+1} 2+\frac{(x+1)(x-1)} 4>x^2-1\)
  \hfill \(\left[-1<x<\frac 5 3\right]\)
 \item~\((x+1)^3-(x+2)^2>\frac{2x^3-1} 2\)
  \hfill \(\left[x<\frac{1-\sqrt{21}} 4\vee x>\frac{1+\sqrt{21}} 4\right]\)
 \item~\((x-2)(3-2x)\ge x-2\)
  \hfill \(\left[1\le x\le 2\right]\)
 \item~\((3x+1)\left(\frac 5 2+x\right)\le 2x-1\)
  \hfill \(\left[-\frac 7 6\le x\le -1\right]\)
 \item~\(\frac{x^2+16} 4+x-1<\frac{x-3} 2\)
  \hfill \(\left[\emptyset\right]\)
 \item~\(\frac{3x-2} 2<x^2-2\)
  \hfill \(\left[x<-\frac 1 2\vee x>2\right]\)
 \item~\(\frac{x-3} 2-\frac{x^2+2} 3<1+x\)
  \hfill \(\left[\insR\right]\)
 \end{enumeratea}
\end{esercizio}

\begin{esercizio}[\Ast]
\label{ese:4.20}
Risolvi le disequazioni di secondo grado.
\vspace{\dy}
 \begin{enumeratea}
 \item~\((x+4)^2+8\ge \frac{x-1} 3\)
  \hfill \(\left[\insR\right]\)
 \item~\(\left(\frac{x-1} 3-\frac x 6\right)^2\le (x+1)^2\)
  \hfill \(\left[x\le -\frac 8 5\vee x\ge -\frac 4 7\right]\)
 \item~\(\frac 1 2\left(x-\frac 2 3\right)^2+x\left(x-\frac 2 3\right)
  \left(x+\frac 2 3\right)>x^3-\frac x 2\left(x-\frac 2 3\right)-\frac 8{27}\)
  \hfill \(\left[x<\frac 2 3\vee x>\frac 7 9\right]\)
 \item~\(3x-5+(1-3x)^2>(x-2)(x+2)\)
  \hfill \(\left[x<0\vee x>\frac 3 8\right]\)
 \item~\(\frac{x-2} 3-(3x+3)^2>x\)
  \hfill \(\left[-\frac{29}{27}<x<-1\right]\)
 \item~\((x-4)^2+(2-x)^2-2(2x+17)>4(x+5)(3-x)+(x+1)^2\)
  \hfill \(\left[x<-3\vee x>5\right]\)
 \item~\((x-2)^3-x^3>x^2-4\)
  \hfill \(\left[\frac{6-2\sqrt 2} 7<x<\frac{6+2\sqrt 2} 7\right]\)
 \item~\((2-x)^3-(2-x)^2<\frac{3-4x^3} 4\)
  \hfill \(\left[\emptyset\right]\)
 \item~\((x+2000)^2+x+2000<2\)
  \hfill \(\left[-202<x<-199\right]\)
 \item~\(\frac{\left(2x-1\right)^3-8x} 2-\frac{\left(2x+1\right)^2-15} 4\le 
 4x\left(x-1\right)^2-6\)
  \hfill \(\left[\right]\)
 \item~\(\frac{(3-x)^2} 2-1\ge -\frac{x^2-4} 4\) 
  \hfill \(\left[x\le 2-\frac{\sqrt 6} 3\vee x\ge 2+\frac{\sqrt 6} 3\right]\)
 \item~\(\left(\frac x 2+1\right)^2-2x>\frac 5 4\left(\frac x 2-1\right)\)
  \hfill \(\left[\right]\)
 \item~\((x+1)^2>(x-1)^2+(x+2)^2+4x\)
  \hfill \(\left[\emptyset \right]\)
 \item~\(\frac{x^2} 4+x<\frac{x+3} 4+\frac x 2-\frac{1-\frac x 2} 2\)
  \hfill \(\left[-1<x<1\right]\)
 \end{enumeratea}
\end{esercizio}

\begin{esercizio}
\label{ese:4.23}
Il monomio \(16x^2\) risulta positivo per:

\boxA\quad \(x>16\)\qquad \boxB\quad \(x>\frac 1{16}\)\qquad\boxC\quad \(x<-4\vee 
x>16\)\qquad\boxD\quad \(x\in \insR\)\qquad\boxE\quad \(x\in \insR_0\)

\end{esercizio}

\begin{esercizio}
\label{ese:4.24}
Il binomio \(16+x^2\) risulta positivo per:

\boxA\; \(x>-16\)\quad \boxB\; \(-4<x<4\) \quad\boxC\; \(x\in \insR-\{-4,4\}\) 
\quad\boxD\; \(x\in \insR\) \quad\boxE\; \(x<-4\vee x>4\)
\end{esercizio}

\begin{esercizio}
\label{ese:4.25}
Il binomio \(16-x^2\) risulta positivo per:

\boxA\; \(x>-16\)\quad \boxB\; \(-4<x<4\) \quad\boxC\; \(x\in \insR-\{-4,4\}\) 
\quad\boxD\; \(x\in \insR\) \quad\boxE\; \(x<-4\vee x>4\)
\end{esercizio}

% \begin{esercizio}
%  \label{ese:4.26}
% Spiegate sfruttando il metodo grafico la verità della proposizione: ``nessun 
% valore della variabile \(a\) rende il polinomio \((3+a)^2-(2a+1)\cdot 
% (2a-1)-(a^2+2a+35)\) positivo''.
% \end{esercizio}

% \subsection*{4.3 - Segno del trinomio a coefficienti letterali}
% 
% \begin{esercizio}[\Ast]
%  \label{ese:4.27}
% Risolvi e discuti le seguenti disequazioni.
% \begin{multicols}{2}
%  \begin{enumeratea}
%  \item~\(x^2-2{kx}+k^2-1>0\)
%  \item~\(3x^2-5{ax}-2a^2<0\)
%  \end{enumeratea}
%  \end{multicols}
% \end{esercizio}
% 
% \begin{esercizio}[\Ast]
%  \label{ese:4.28}
% Risolvi e discuti le seguenti disequazioni.
% \begin{multicols}{2}
%  \begin{enumeratea}
%  \item~\(4x^2-4x+1-9m^2<0\)
%  \item~\(2x^2-3{ax}<0\)
%  \end{enumeratea}
%  \end{multicols}
% \end{esercizio}
% 
% \begin{esercizio}[\Ast]
%  \label{ese:4.29}
% Risolvi e discuti le seguenti disequazioni.
% \begin{multicols}{2}
%  \begin{enumeratea}
%  \item~\(x^2-2{tx}-8t^2>0\)
%  \item~\((1-s)x^2+9>0\)
%  \end{enumeratea}
%  \end{multicols}
% \end{esercizio}
% 
% \begin{esercizio}[\Ast]
%  \label{ese:4.30}
% Risolvi e discuti le seguenti disequazioni.
% \begin{multicols}{2}
%  \begin{enumeratea}
%  \item~\((m-1)x^2-{mx}>0\)
%  \item~\({kx}^2-(k+1)x-3\ge 0\)
%  \end{enumeratea}
%  \end{multicols}
% \end{esercizio}
% 
% \paragraph{4.27.} a)~\(x<k-1\vee x>k+1\),\, b)~\(a=0\to \emptyset\) \(a>0\to -\frac 1 
% 3a<x<2a\) \(a<0\to 2a<x<-\frac 1 3a\)
% 
% \paragraph{4.28.} a)~\(m=0\to \emptyset\) \(m>0\to \frac{1-3m}2<x<\frac{1+3m} 2\) 
% \(m<0\to \frac{1+3m} 2<x<\frac{1-3m} 2\),\quad b)~\(a=0\to \emptyset\) \(a>0\to 
% 0<x<\frac 3 2a\) \(a<0\to \frac 3 2a<x<0\)
% 
% \paragraph{4.29.} a)~\(t=0\to x\neq 0\) \(t>0\to -2t<x<4t\) \(t<0\to 4t<x<-2t\),\quad 
% b)~\(s\le 1\to \insR\) \(s>1\to \frac{-3}{\sqrt{k-1}}<x<\frac 3{\sqrt{k-1}}\)
% 
% \paragraph{4.30.} a)~\(m=0\to \emptyset\) \(m=1\to x<0\) \(0<m<1\to \frac m{m-1}<x<0\) 
% \(m<0\to 0<x<\frac m{m-1}\) \(m>1\to x<0\vee x>\frac m{m-1}\)

% \begin{esercizio}
%  \label{ese:4.31}
% Trovare il segno del trinomio \(t=(1-m)x^2-2{mx}-m+3\) al variare del 
% parametro~\( m \)
% \end{esercizio}

\newpage %--------------------------------------

% \subsection*{4.4 - Disequazioni polinomiali di grado superiore al secondo}
\numnameref{sec:diseq_grado_superiore}

\begin{esercizio}
 \label{ese:4.32}
Data la disequazione \(\left(x^2-x\right)\cdot \left(2x^2+13x+20\right)<0\) 
verificare che nessun numero naturale appartiene all'insieme soluzione. C'è 
qualche numero intero nell'\( \IS \)? È vero che l'\( \IS \) è formato dall'unione 
di due intervalli aperti di numeri reali?
\end{esercizio}

\begin{esercizio}
 \label{ese:4.33}
Dopo aver scomposto in fattori il polinomio \(p(x)=2x^4-5x^3+5x-2\) determinare il 
suo segno.
\end{esercizio}

\begin{esercizio}
 \label{ese:4.34}
Dato il trinomio \(p(x)=9x^2+x^4-10\) stabilire se esiste almeno un numero 
naturale che lo renda negativo.
\end{esercizio}

\begin{esercizio}
\label{ese:4.35}
Nell'insieme dei valori reali che rendono positivo il trinomio 
\(p(x)=2x^5-12x^3-14x\) vi sono solo due numeri interi negativi?
\end{esercizio}

\begin{esercizio}
 \label{ese:4.36}
\(x\in (-1;+\infty )\Rightarrow p(x)=x^5-2x^2-x+2>0\) Vero o falso?
\end{esercizio}

\begin{esercizio}
\label{ese:4.37}
Nell'insieme dei valori reali che rendono negativo \(p(x)=(2x-1)^3-(3-6x)^2\) 
appartiene un valore razionale che lo annulla. Vero o falso?
\end{esercizio}

\begin{esercizio}[\Ast]
\label{ese:4.38}
Risolvi le seguenti disequazioni di grado superiore al secondo.
\begin{enumeratea}
\item \((1-x)(2-x)(3-x)>0\) \hfill \(\left[x<1\vee 2<x<3\right]\)
\item \((2x-1)(3x-2)(4x-3)\le 0\) 
 \hfill \(\left[\frac 2 3\le x\le \frac 3 4 \vee x\le \frac 1 2\right]\)
\item \(-2x(x-1)(x+2)>0\) \hfill \(\left[x<-2\vee 0<x<1\right]\)
\item \( \left(x^4-4x^2-45\right)\cdot \left(4x^2-4x+1\right)>0 \) 
 \hfill \(\left[\right]\)
\item \(3x(x-2)(x+3)(2x-1)\le 0\) 
 \hfill \(\left[\frac 1 2\le x\le 2\vee -3\le x\le 0\right]\)
\item \(\left(x^2+1\right)(x-1)(x+2)>0\) \hfill \(\left[x<-2\vee x>1\right]\)
\item \(\left(1-9x^2\right)\left(9x^2-3x\right)2x>0\) 
 \hfill \(\left[x<-1/3\right]\)
\item \(\left(16x^2-1\right)\left(x^2-x-12\right)>0\) 
 \hfill \(\left[-\frac 1 4<x<\frac 1 4\vee x<-3\vee x>4\right]\)
\item \(-x\left(x^2-3x-10\right)\left(x^2-9x+18\right)\le 0\)
 \hfill \(\left[3\le x\le 5\vee -2\le x\le 0\vee x\ge 6\right]\)
\item \(x^2(x-1)\left(2x^2-x\right)\left(x^2-3x+3\right)>0\)
 \hfill \(\left[0<x<\frac 1 2\vee x>1\right]\)
\item \(\left(x^2-1\right)\left(x^2-2\right)\left(x^2-3x\right)>0\)
 \hfill \(\left[x<-\sqrt 2\vee 1<x<\sqrt 2\vee -1<x<0\vee x>3\right]\)
\item \(x^3-x^2+x-1>0\) \hfill \(\left[x>1\right]\)
\item \(x^3-5x^2+6<0\) \hfill \(\left[3-\sqrt 3<x<3+\sqrt 3\vee x<-1\right]\)
\item \(\left(5x^3-2x^2\right)\left(3x^2-5x\right)\ge 0\) 
 \hfill \(\left[0\le x\le \frac 2 5\vee x\ge \frac 5 3\right]\)
\item \(x^4-2x^3-x+2>0\) \hfill \(\left[x<1\vee x>2\right]\)
\item \(x^4+x^2-9x^2-9\le 0\) \hfill \(\left[-3\le x\le 3\right]\)
\item \(25x^4-9>0\) 
 \hfill \(\left[x<-\frac{\sqrt{15}} 5\vee x>\frac{\sqrt{15}} 5\right]\)
% \item \(x^3-1\ge 2x(x-1)\) \hfill \(\left[x\ge 1\right]\)
% \item \(x^4-1>x^2+1\) \hfill \(\left[x<-\sqrt 2\vee x>\sqrt 2\right]\)
\item \(\left(x^2+x\right)^2+2\left(x+1\right)^2\ge 0\) 
 \hfill \(\left[\insR\right]\)
% \item \((x+1)\left(x-\frac 1 2\right)(x+2)<0\) 
%  \hfill \(\left[-1<x<\frac 1 2\vee x<-2\right]\)
% \item \(\left(x^2-4\right)(x-2)\ge 0\) 
%  \hfill \(\left[x\ge -2\right]\)
% \item \((x-7)\left(x^2-7x+10\right)<0\) 
%  \hfill \(\left[5<x<7\vee x<2\right]\)
% \item \(\left(x^2-4\right)\left(x^2-9\right)\ge 0\) 
%  \hfill \(\left[x\le -3\vee -2\le x\le 2\vee x\ge 3\right]\)
\end{enumeratea}
\end{esercizio}
% 
% \begin{esercizio}[\Ast]
%  \label{ese:4.44}
% Risolvi le seguenti disequazioni di grado superiore al secondo.
% \begin{multicols}{2}
% \begin{enumeratea}
% \item \(\left(x^4+4x^3-12x^2\right)\left(x+3\right)\ge 0\)
% \item \((x-4)^3-(x-4)^2-2x+10>2\)
% \item \(x^3-1\ge 0\)
% \item \(\left(x^4+4x^3-12x^2\right)\left(x+3\right)\ge 0\)
% \end{enumeratea}
% \end{multicols}
% \end{esercizio}
% 
% \begin{esercizio}[\Ast]
%  \label{ese:4.45}
% Risolvi le seguenti disequazioni di grado superiore al secondo.
% \begin{enumeratea}
% \item \((x+3)(x+4)(x+5)(5-x)(4-x)(3-x)>0\)
% \item \((x^2-2x)(x^2+1)>0\)
% \item \((8-2x^2)(3x-x^2+4)<0\)
% \item \((6x^2-6)(100x^2+100x)<0\)
% \end{enumeratea}
% \end{esercizio}
% 
% \paragraph{4.44.} a)~\(x=0\vee -6\le x\le -3\vee x\ge 2\),\quad b)~\(3<x<4\vee 
% x>6\),\quad c)~\(x\ge 1\),\protect\\
% d)~\(-9<x<-6\vee -\frac 1 2<x<3\)
% 
% \paragraph{4.45.} a)~ \(-5<x<-4\vee -3<x<3\vee 4<x<5\),\quad b)~\(x<0\vee 
% x>2\),\protect\\
% c)~\(-2<x<-1\vee 2<x<4\),\quad d)~\(0<x<1\)
% 
% \begin{esercizio}[\Ast]
%  \label{ese:4.46}
% Risolvi le seguenti disequazioni di grado superiore al secondo.
% \begin{multicols}{2}
% \begin{enumeratea}
% \item \((1+x^2)(3x^2+x)<0\)
% \item \((x^2+3x+3)(4x^2+3)>0\)
% \item \((125+4x^2)(128+2x^2)<0\)
% \item \((x^2+4x+4)(x^2-4x+3)>0\)
% \end{enumeratea}
% \end{multicols}
% \end{esercizio}
% \newpage
% \begin{esercizio}[\Ast]
%  \label{ese:4.47}
% Risolvi le seguenti disequazioni di grado superiore al secondo.
% \begin{multicols}{2}
% \begin{enumeratea}
% \item \((x^2-5x+8)(x^2-2x+1)>0\)
% \item \((-2x+1)(3x-x^2)>0\)
% \item \((4x^2-3x)(x^2-2x-8)<0\)
% \item \((4x-x^2+5)(x^2-9x+20)<0\)
% \end{enumeratea}
% \end{multicols}
% \end{esercizio}
% 
% \begin{esercizio}[\Ast]
%  \label{ese:4.48}
% Risolvi le seguenti disequazioni di grado superiore al secondo.
% \begin{multicols}{2}
% \begin{enumeratea}
% \item \((5+2x)(-2x^2+14x+16)<0\)
% \item \((5x-2x^2-10)(x^2+3x-28)>0\)
% \item \((x^2-6x+9)(8x-7x^2)>0\)
% \item \((3x^2+2x-8)(6x^2+19x+15)<0\)
% \end{enumeratea}
% \end{multicols}
% \end{esercizio}
% 
% \begin{esercizio}[\Ast]
%  \label{ese:4.49}
% Risolvi le seguenti disequazioni di grado superiore al secondo.
% \begin{multicols}{2}
% \begin{enumeratea}
% \item \((3x^2-5x-2)(4x^2+8x-5)>0\)
% \item \((4x-4)(2x^2-3x+2)<0\)
% \item \((2x-4)(2x^2-3x-14)>0\)
% \item \((-7x+6)(x^2+10x+25)<0\)
% \end{enumeratea}
% \end{multicols}
% \end{esercizio}
% 
% \begin{esercizio}[\Ast]
%  \label{ese:4.50}
% Risolvi le seguenti disequazioni di grado superiore al secondo.
% \begin{multicols}{2}
% \begin{enumeratea}
% \item \((-3+3x)(x^3-4x^2)>0\)
% \item \(\left(x^2+1\right)\left(x^2-1\right)>0\)
% \item \((1-x)(2-x)^2\le 0\)
% \item \(-x\left(x^2+1\right)(x+1)\ge 0\)
% \item \((x+1)^2\left(x^2-1\right)<0\)
% \item \((x^2-4)(2x-50x^2)\ge 0\)
% \end{enumeratea}
% \end{multicols}
% \end{esercizio}
% 
% \begin{esercizio}[\Ast]
%  \label{ese:4.51}
% Risolvi le seguenti disequazioni di grado superiore al secondo.
% \begin{multicols}{2}
% \begin{enumeratea}
% \item \((x-4)(2x^2+x-1)\ge 0\)
% \item \(-3x^3+27>0\)
% \item \(3x^3+27>0\)
% \item \(x^3+3x^2+3x+1\le 0\)
% \item \(x^3-6x+9<0\)
% \item \(x^5+1>x\left(x^3+1\right)>0\)
% \end{enumeratea}
% \end{multicols}
% \end{esercizio}
% 
% \paragraph{4.46.} a)~\(-\frac 1 3<x<0\),\quad b)~\(\IS=\insR\),\quad 
% c)~\(\IS=\emptyset \),\quad d)~\(x<-2\vee -2<x<1\vee x>3\)
% 
% \paragraph{4.47.} a)~\(x<1\vee x>1\),\quad b)~\(0<x<\frac 1 2\vee x>3\),\quad 
% c)~\(-2<x<0\vee \frac 3 4<x<4\),\protect\\
% d)~\(x<-1\vee 4<x<5\vee x>5\)
% 
% \paragraph{4.48.} a)~\(-\frac 5 2<x<-1\vee x>8\),\quad b)~\(-7<x<4\),\; 
% c)~\(0<x<\frac 8 7\),\; d)~\(-2<x<-\frac 5 3\vee -\frac 3 2<x<\frac 4 3\)
% 
% \paragraph{4.49.} a)~\(x<-\frac 5 2\vee -\frac 1 3<x<\frac 1 2\vee x>2\),\quad 
% b)~\(x<1\),\quad c)~\(-2<x<2\vee x>\frac 7 2\),\quad d)~\(x>6/7\)
% 
% \paragraph{4.50.} a)~\(\IS=x\in \insR| x<0\vee 0<x<1\vee x>4\)
% 
% \paragraph{4.51.} d)~\(x\le -1\),\quad e)~\(x<-3\)

% \begin{esercizio}
%  \label{ese:4.52}
% Risolvi le seguenti disequazioni di grado superiore al secondo.
% \begin{multicols}{2}
% \begin{enumeratea}
% \item \(x^3-7x^2+4x+12\ge 0\)
% \item \(x^3+5x^2-2x-24<0\)
% \item \(6x^3+23x^2+11x-12\le 0\)
% \item \(4x^3+4x^2-4x-4\ge 0\)
% \item \(-6x^3-30x^2+192x-216<0\)
% \item \(81x^4-1\le 0\)
% \end{enumeratea}
% \end{multicols}
% \end{esercizio}
% 
% \begin{esercizio}
%  \label{ese:4.53}
% Risolvi le seguenti disequazioni di grado superiore al secondo.
% \begin{multicols}{2}
% \begin{enumeratea}
% \item \(3x^5+96<0\)
% \item \(x^4-13x^2+36<0\)
% \item \(9x^4-37x^2+4\ge 0\)
% \item \(-4x^4+65x^2-16<0\)
% \item \(x^6-4x^3+3\ge 0\)
% \item \(x^8-x^4-2<0\)
% \end{enumeratea}
% \end{multicols}
% \end{esercizio}
% 
% \begin{esercizio}
%  \label{ese:4.54}
% Risolvi le seguenti disequazioni di grado superiore al secondo.
% \begin{enumeratea}
% \item \(\frac 2 3x^3>\frac 9 4\)
% \item \( (2x-1)^2\ge x^2\left(4x^2-4x+1\right) \)
% \item \( (x+1)\left(x^2-1\right)>\left(x^2-x\right)(x-1)^2 \)
% \item \( -4x\left(x^2+7x+12\right)\left(x^2-25\right)(4-x)>0 \)
% \item \( (x-5x^2)(x^4-3x^3+5x^2)\ge 0 \)
% \item \( (4+7x^2)\left[x^2-(\sqrt 2+\sqrt 3)x+\sqrt 6\right]<0 \)
% \end{enumeratea}
% \end{esercizio}
% \newpage
% \begin{esercizio}
%  \label{ese:4.55}
% Risolvi le seguenti disequazioni di grado superiore al secondo.
% \begin{multicols}{2}
% \begin{enumeratea}
% \item \( (x^3-9x)(x-x^2)(4x-4-x^2)>0 \)
% \item \( x\left|x+1\right|\cdot (x^2-2x+1)\ge 0 \)
% \item \(16x^4-1\ge 0\)
% \item \(16x^4+1\le 0\)
% \item \(-16x^4-1>0\)
% \item \(-16x^4+1>0\)
% \end{enumeratea}
% \end{multicols}
% \end{esercizio}
% 
% \begin{esercizio}
%  \label{ese:4.56}
% Risolvi le seguenti disequazioni di grado superiore al secondo.
% \begin{multicols}{3}
% \begin{enumeratea}
% \item \(1-16x^4<0\)
% \item \(27x^3-8\ge 0\)
% \item \(8x^3+27<0\)
% \item \(4x^4+1\ge 0\)
% \item \(4x^4-1\ge 0\)
% \item \(1000x^3+27>0\)
% \end{enumeratea}
% \end{multicols}
% \end{esercizio}
% 
% \begin{esercizio}
%  \label{ese:4.57}
% Risolvi le seguenti disequazioni di grado superiore al secondo.
% \begin{multicols}{3}
% \begin{enumeratea}
% \item \(10000x^4-1\ge 0\)
% \item \(x^7+7<0\)
% \item \(x^3-8\ge 0\)
% \item \(9x^4-4\ge 0\)
% \item \(x^6+\sqrt 6\le 0\)
% \item \(0,1x^4-1000\ge 0\)
% \item \(x^4-9\ge 0\)
% \item \(x^4+9\le 0\)
% \item \(-x^4+9\le 0\)
% \end{enumeratea}
% \end{multicols}
% \end{esercizio}

% \subsection*{4.5 - Disequazioni fratte}
\numnameref{sec:diseq_fratte}

\begin{esercizio}[\Ast]
 \label{ese:4.58}
Determinare l'Insieme Soluzione delle seguenti disequazioni fratte.
\begin{multicols}{2}
\begin{enumeratea}
\item \(\frac{x+2}{x-1}>0\) \hfill \(\left[x<2\vee x>1\right]\)
% \item \(\frac{x+3}{4-x}>0\) \hfill \(\left[-3<x<4\right]\)
% \item \(\frac{x+5}{x-7}>0\) \hfill \(\left[x<-5\vee x>7\right]\)
\item \(\frac{2-4x}{3x+1}\ge 0\) \hfill \(\left[-\frac 1 3<x\le \frac 1 2\right]\)
\item \(\frac{x^2-4x+3}{4-7x}\ge 0\) 
 \hfill \(\left[x<\frac 4 7\vee 1\le x\le 3\right]\)
\item \(\frac{x+5}{x^2-25}>0\) \hfill \(\left[x>5\right]\)
\item \(\frac{x^2-1}{x-2}>0\) \hfill \(\left[-1<x<1\vee x>2\right]\)
\item \(\frac{x^2-4x+3}{x+5}<0\) \hfill \(\left[x<-5\vee 1<x<3\right]\)
\item \(\frac{-x^2+4x-3}{x+5}>0\) \hfill \(\left[x<-5\vee 1<x<3\right]\)
\item \(\frac{x^2+1}{x^2-2x}>0\) \hfill \(\left[x<0\vee x>2\right]\)
\item \(\frac{9-x^2}{2x^2-x-15}>0\) \hfill \(\left[-3<x<-\frac 5 2\right]\)
\item \(\frac{x^2-7x}{-x^2-8}>0\) \hfill \(\left[0<x<7\right]\)
\item \(\frac{x+2}{x-1}\le 0\) \hfill \(\left[1< x\le 2\right]\)
\item \(\frac 1{x^2+2x+1}>0\) \hfill \(\left[\insR-\{-1\}\}\right]\)
\item \(\frac{-3}{-x^2-4x-8}>0\) \hfill \(\left[\insR\right]\)
\item \(\frac{x^2+2x+3}{-x^2-4}>0\) \hfill \(\left[\emptyset\right]\)
\item \(\frac{3x-12}{x^2-9}>0\) \hfill \(\left[-3<x<3\vee x>4\right]\)
\item \(\frac{5-x}{x^2-4}>0\) \hfill \(\left[x<-2\vee 2<x<5\right]\)
\end{enumeratea}
\end{multicols}
\end{esercizio}


\begin{esercizio}[\Ast]
 \label{ese:4.61}
Determinare l'Insieme Soluzione delle seguenti disequazioni fratte.
\begin{multicols}{2}
\begin{enumeratea}
\item \(\frac{3x-x^2-2}{2x^2+5x+3}>0\) \hfill \(\left[-\frac 3 2<x<-1\vee 1<x<2\right]\)
\item \(\frac{4-2x}{x^2-2x-8}>0\) \hfill \(\left[x<-2\vee 2<x<4\right]\)
\item \(\frac{x^2-4x+3}{5-10x}>0\) \hfill \(\left[x<\frac 1 2\vee 1<x<3\right]\)
\item \(\frac{x^2+3x+10}{4-x^2}>0\) \hfill \(\left[-2<x<2\right]\)
\item \(\frac{x^2-3x+2}{4x-x^2-5}>0\) \hfill \(\left[1<x<2\right]\)
\item \(\frac{x^2+2}{25-x^2}>0\) \hfill \(\left[-5<x<5\right]\)
\item \(\frac{3x^2-2x-1}{4-2x}>0\) \hfill \(\left[x<-\frac 1 3\vee 1<x<2\right]\)
\item \(\frac{x+2}{x^2+4x+4}>0\) \hfill \(\left[x>-2\right]\)
\item \(\frac{x+2}{x^2+4x+2}>0\) \hfill \(\left[x<1\vee 3<x<5\right]\)
\item \(\frac{-x^2+2x+8}{-x-1}<0\) \hfill \(\left[x<-2\vee -1<x<4\right]\)
\item \(\frac{x^2+3x+2}{25-x^2}>0\) \hfill \(\left[-5<x<-2\vee -1<x<5\right]\)
\item \(\frac{x^2+4x+3}{3x-6}>0\) \hfill \(\left[-3<x<-1\vee x>2\right]\)
\item \(\frac{5-x}{x^2-4x+3}>0\) \hfill \(\left[x<-\frac 3 4\vee 1<x<4\right]\)
\item \(\frac{1-x^2}{x^2+2x+3}<0\) \hfill \(\left[x<-1\vee x>1\right]\)
% \item \(\frac{x^2-9}{x^2-5x}>0\) \hfill \(\left[x<-3\vee 0<x<3\vee x>5\right]\)
% \item \(\frac{x^2-x-2}{x-x^2+6}>0\) \hfill \(\left[-2<x<-1\vee 2<x<3\right]\)
% \item \(\frac{x^2-5x+6}{-3x+7}<0\) \hfill \(\left[2<x<\frac 7 3 \vee x>3\right]\)
% \item \(\frac{2x+8}{x^2+4x-12}>0\) \hfill \(\left[-6<x<-4\vee x>2\right]\)
\end{enumeratea}
\end{multicols}
\end{esercizio}

\begin{esercizio}[\Ast]
 \label{ese:4.64}
Determinare l'Insieme Soluzione delle seguenti disequazioni fratte.
\begin{enumeratea}
% \item \(\frac{x^2-2x-63}{4x+5-x^2}>0\) \hfill \(\left[-7<x<-1\vee 5<x<9\right]\)
% \item \(\frac{4-x^2+3x}{x^2-x}>0\) \hfill \(\left[-1<x<0\vee 1<x<4\right]\)
% \item \(\frac{x^2-2x}{5-x^2}>0\) 
%  \hfill \(\left[-\sqrt 5<x<0\vee 2<x<\sqrt 5\right]\)
\item \(\frac{x^2-x-2}{-3x^2+3x+18}\le 0\) 
 \hfill \(\left[x<-2\vee -1\le x\le 2\vee x>3\right]\)
\item \(\frac{x^2-8x+15}{x^2+3x+2}>0\) 
 \hfill \(\left[x<-2\vee -1<x<3\vee x>5\right]\)
\item \(\frac{4x+7}{3x^2-x-2}>0\) 
 \hfill \(\left[-\frac 7 4<x<-\frac 2 3\vee x>1\right]\)
\item \(\frac{-x^2-4x-3}{6x-x^2}>0\) 
 \hfill \(\left[x<-3\vee -1<x<0\vee x>6\right]\)
\item \(\frac{5x+x^2+4}{6x^2-6x}>0\) 
 \hfill \(\left[x<-4\vee -1<x<0\vee x>1\right]\)
\item \(\frac{9-x^2}{x^2+5x+6}\cdot \frac{6x-2x^2}{4-x^2}>0\) 
 \hfill \(\left[0<x<2\right]\)
\item \(\frac{2x-4x^2}{x^2+x-12}\cdot \frac{16-x^2}{5x-x^2}\le 0\) 
 \hfill \(\left[x\le \frac 1 2\vee 3<x\le 4\vee x>5 \wedge 
 x\neq 0 \wedge x\neq -4\right]\)
\item \(\frac{1-x^2}{x^2}\le \frac 1{x^2}-x^2-\frac 1 2\) 
 \hfill \(\left[-\frac{\sqrt 2} 2\le x\le \frac{\sqrt 2} 2 \wedge 
 x\neq 0\right]\)
\item \(\frac{x+2}{x-1}\ge \frac{24}{x+1}-\frac x{x^2-1}\) 
 \hfill \(\left[x<-1\vee 1<x\le 10-\sqrt{74}\vee x\ge 10+\sqrt{74}\right]\)
\item \(\frac 1 x+\frac 1{x-1}+\frac 1{x+1}<\frac{2x+1}{x^2-1}\) 
 \hfill \(\left[-\frac{\sqrt 2} 2\le x\le \frac{\sqrt 2} 2\right]\)
\item \(\frac x{x+2}\ge \frac{x-4}{x^2-4}\) 
 \hfill \(\left[x<-2\vee x>2\right]\)
\item \(\frac{4x+1}{x^2-9}+\frac{1-x}{x+3}<6-\frac x{x-3}\) 
 \hfill \(\left[x<-3\vee -\frac{13} 6<x<3\vee x>4\right]\)
\item \(\frac{x+1}{2x-1}+\frac 3{4x+10}\ge 1-\frac{2x+2}{4x^2+8x-5}\) 
 \hfill \(\left[-\frac 5 2<x\le -\frac 3 2\vee \frac 1 2<x\le \frac 7 2\right]\)
\item \(\frac{2x+5}{(2x+4)^2}\ge \frac 2{2x+4}\) 
 \hfill \(\left[x\le -\frac 3 2 \wedge x \neq -2\right]\)
\item \(\frac{10x^2}{x^2+x-6}+\frac x{2-x}-1\le \frac 5{x+3}\) 
 \hfill \(\left[-3<x<2\right]\)
\item \(\frac{5x+20}{5x+5}+\frac{2x-8}{2x-2}\ge 2\) 
 \hfill \(\left[-1<x<1\right]\)
\item \(\frac 8{8x^2-8x-70}-\frac 4{4x^2-4x-35}>\frac{8x+8}{4x^2-20x+21}\)
 \hfill \(\left[\frac 3 2<x<\frac 7 2\vee x<-1\right]\)
\item \(\frac{4x^2-8x+19}{8x^2-36x+28}-\frac{2x-5}{4x-4}\ge \frac{8x+12}{8x-28}\)
 \hfill \(\left[\frac 1 2\le x<\frac 7 2 \wedge x \neq 1\right]\)
\end{enumeratea}
\end{esercizio}

\begin{esercizio}[\Ast]
 \label{ese:4.68}
Assegnate le due funzioni \(f_1=\frac{x^2+1}{2x-x^2}\) e \(f_2=\frac 1 x+\frac 
1{x-2}\) stabilire per quali valori della variabile indipendente si ha \(f_1\ge 
f_2\)
 \hfill \(\left[-1-\sqrt 2\le x<0\vee -1+\sqrt 2\le x<2\right]\)
\end{esercizio}

\begin{esercizio}
 \label{ese:4.69}
Spiegare perché l'espressione letterale 
\(E=\frac{1-\frac{x^2}{x^2-1}}{2+\frac{3x-1}{1-x}}\) è sempre positiva nel suo 
dominio.
\end{esercizio}

\begin{esercizio}[\Ast]
 \label{ese:4.70}
Per quali valori di x la funzione \(y=\frac{(x-1)\cdot x-2}{5x^2-x-4}\) è maggiore 
o uguale a 1.
 \hfill \(\left[-\frac 3 2\le x<-\frac 4 5\right]\)
\end{esercizio}

\begin{esercizio}[\Ast]
 \label{ese:4.71}
\( x \), \( x+2 \), \( x+4 \) sono tre numeri naturali. Determinate in \( \insN \) il 
più piccolo numero che rende vera la proposizione: ``il doppio del primo 
aumentato del prodotto degli altri due è maggiore della differenza tra il doppio 
del terzo e il quadrato del secondo''
 \hfill \(\left[5\right]\)
\end{esercizio}

\begin{esercizio}
 \label{ese:4.72}
Date chiare e sintetiche motivazioni alla verità della seguente proposizione: 
``il segno della frazione \(f=\frac{9-x^2+3x}{2+x^2}\) non è mai positivo e la 
frazione non ha zeri reali''.
\end{esercizio}

\begin{esercizio}
 \label{ese:4.73}
Stabilire se basta la condizione \(x\neq 1\wedge x\neq -1\) per rendere positiva 
la frazione \(~f=~\frac{x^3-1}{x^4-2x^2+1}\)
\end{esercizio}

\begin{esercizio}
 \label{ese:4.74}
Determinare per quali valori reali la frazione \(f=\frac{(x+1)^2}{4x^2-12x+9}\) 
risulta non superiore a 1.
\end{esercizio}

% \subsection*{4.6 - Sistemi di disequazioni}
\numnameref{sec:diseq_sistemi}

\begin{esercizio}
 \label{ese:21.33}
Sulla retta reale rappresenta l'insieme soluzione~\(S_{1}\)
dell'equazione:
\[\frac{1}{6}+\frac{1}{4}\cdot (5x+3)=2+\frac{2}{3}\cdot (x+1)\]

e l'insieme soluzione~\(S_{2}\) della disequazione:
\[\frac{1}{2}-2\cdot\left(\frac{1-x}{4}\right)\ge~3-\frac{6-2x}{3}-\frac{x}{2}.\]

È vero che~\(S_{1}\subset S_{2}\)?
\end{esercizio}

\begin{esercizio}[\Ast]
 \label{ese:21.34}
 Determina i numeri reali che verificano il sistema:
 \(\left\{%
  \begin{array}{l}
  x^{2}\le~0
  \\2-3x\ge~0
 \end{array}\right.\)
\hfill \(\left[x = 0\right]\)
 \end{esercizio}

\begin{esercizio}
 \label{ese:21.35}
 L'insieme soluzione del sistema:
\(\left\{\begin{array}{l}
  (x+3)^{3}-(x+3)\cdot (9x-2)>x^{3}+27\\
  \dfrac{x+5}{3}+3+\dfrac{2\cdot (x-1)}{3}<x+1
 \end{array}\right.\) è:
\begin{multicols}{2}
\boxA\quad~\(\left\{x\in \insR/x>3\right\}\)

\boxB\quad~\(\left\{x\in \insR/x>-3\right\}\)

\boxC\quad~\(\left\{x\in \insR/x<-3\right\}\)

\boxD\quad~\(\IS=\emptyset \)

\boxE\quad~\(\left\{x\in\insR/x<3\right\}\)
\end{multicols}

\end{esercizio}

\begin{esercizio}
 \label{ese:21.36}
 Attribuire il valore di verità alle seguenti proposizioni:

\begin{enumeratea}
\item il quadrato di un numero reale è sempre positivo;
\item l'insieme complementare di~\(A=\{x\in\insR/x>-8\}\text{ è }B=\{x\in\insR/x<-8\}\)
\item il monomio~\(-6x^{3}y^{2}\) assume valore positivo per tutte le coppie dell'insieme~\(\insR^{+}\times\insR^{+}\)
\item nell'insieme~\(\insZ\) degli interi relativi il sistema~\(\left\{\begin{array}{l}x+1>0\\8x<0\end{array}\right.\) non ha soluzione;
\item l'intervallo~\(\left[-1,\left.-{\dfrac{1}{2}}\right)\right.\) rappresenta l'\(\IS\) del sistema~\(\left\{\begin{array}{l}1+2x<0 \\\dfrac{x+3}{2}\le x+1\end{array}\right.\)
\end{enumeratea}
\end{esercizio}

\begin{esercizio}[\Ast]
 \label{ese:21.37}
 Risolvi i seguenti sistemi di disequazioni.
 \begin{multicols}{2}
 \begin{enumeratea}
 \item \(\left\{\begin{array}{l}
3-x>x\\
2x>3
        \end{array}\right.\)
\hfill \(\left[\emptyset\right]\)
\item \(\left\{\begin{array}{l}
3x\le~4\\
5x\ge -4
\end{array}\right.\)
\hfill \(\left[-{\frac{4}{5}}\le x\le\frac{4}{3}\right]\)
\item \(\left\{\begin{array}{l}
2x>3\\
3x\le~4
        \end{array}\right.\)
\hfill \(\left[\emptyset\right]\)
\item \(\left\{\begin{array}{l}
3x-5<2\\
x+7<-2x
\end{array}\right.\)
\hfill \(\left[x<-{\frac{7}{3}}\right]\)
 \item \(\left\{\begin{array}{l}
3-x\ge x-3\\
-x+3\ge~0
\end{array}\right.\)
\hfill \(\left[x\le~3\right]\)
\item \(\left\{\begin{array}{l}
-x-3\le~3\\
3+2x\ge~3x+2
\end{array}\right.\)
\hfill \(\left[-6\le x\le~1\right]\)
\item \(\left\{\begin{array}{l}
2x-1>2x \\
3x+3\le~3
\end{array}\right.\)
\hfill \(\left[\emptyset\right]\)
\item \(\left\{\begin{array}{l}
2x+2<2x+3\\
2(x+3)>2x+5
\end{array}\right.\)
\hfill \(\left[\insR\right]\)
 \item \(\left\{\begin{array}{l}
-3x>0\\
-3x+5\ge~0\\
-3x\ge-2x
\end{array}\right.\)
\hfill \(\left[x<0\right]\)
\item {\longarray \(\left\{\begin{array}{l}
-{\dfrac{4}{3}}x\ge\dfrac{2}{3}\\
-{\dfrac{2}{3}}x\le\dfrac{1}{9}
\end{array}\right.\)}
\hfill \(\left[\emptyset\right]\)
\item \(\left\{\begin{array}{l}
3+2x>3x+2 \\
5x-4\le~6x-4\\
-3x+2\ge -x-8
\end{array}\right.\)
\hfill \(\left[0\le x<1\right]\)
\item \(\left\{\begin{array}{l}
4x+4\ge~3\cdot\left(x+\dfrac{4}{3}\right)\\
4x+4\ge~2\cdot (2x+2)
\end{array}\right.\)
\hfill \(\left[x\ge~0\right]\)
 \item \(\left\{\begin{array}{l}
3(x-1)<2(x+1)\\
x-\dfrac{1}{2}+\dfrac{x+1}{2}>0
\end{array}\right.\)
\hfill \(\left[0<x<5\right]\)
\end{enumeratea}
\end{multicols}
\end{esercizio}

\begin{esercizio}[\Ast]
 \label{ese:4.75}
Risolvere i seguenti sistemi di disequazioni.
\begin{multicols}{2}
\begin{enumeratea}
\item \(\left\{\begin{array}{l}x^2-4>0\\x-5\le 0\end{array}\right.\) 
 \hfill \(\left[x<-2\vee 2<x\right]\)
\item \(\left\{\begin{array}{l}x^2-4x+3\le 0\\x-2x^2<-10\end{array}\right.\)
 \hfill \(\left[\frac 5 2<x\le 3\right]\)
\item \(\left\{\begin{array}{l}4x-x^2>0\\3x^2(x-3)>0\end{array}\right.\)
 \hfill \(\left[3<x<4\right]\)
\item \(\left\{\begin{array}{l}x^2+5x+6\le 0\\2x+5\le 0\end{array}\right.\)
 \hfill \(\left[-3\le x\le -\frac 5 2\right]\)
\item \(\left\{\begin{array}{l}3x-x^2-2\le 0\\x^2>49\end{array}\right.\)
 \hfill \(\left[x<-7\vee x>7\right]\)
\item \(\left\{\begin{array}{l}3x-2>0\\x^2-1>0\\2x-x^2<0\end{array}\right.\)
 \hfill \(\left[x>2\right]\)
\item \(\left\{\begin{array}{l}x^2-4x+4\ge 0\\x<6\end{array}\right.\)
 \hfill \(\left[x<6\right]\)
\item \(\left\{\begin{array}{l}x^2+6x+9<0\\x<2\\x^2+1>0\end{array}\right.\)
 \hfill \(\left[\emptyset\right]\)
\item \(\left\{\begin{array}{l}x^2+6x+9\le 0\\x<2\end{array}\right.\)
 \hfill \(\left[x=-3\right]\)
\item \(\left\{\begin{array}{l}4x-x^2-3<0\\3x\ge 2\end{array}\right.\)
 \hfill \(\left[\frac 2 3\le x<1\vee x>3\right]\)
\item \(\left\{\begin{array}{l}2x^2<8\\-x^2+5x>-6\\x^2(9-x^2)\le 0
              \end{array}\right.\)
 \hfill \(\left[x=0\right]\)
\item \(\left\{\begin{array}{l}(x^2-4x+3)(2x-4)>0\\2x-x^2\le 1
              \end{array}\right.\)
 \hfill \(\left[1<x<2\vee x>3\right]\)
\item \(\left\{\begin{array}{l}(3-x)(x^2-4)(x^2-2x-8)<0\\x^2-64\le 0
              \end{array}\right.\)
 \hfill \(\left[2<x<3\vee 4<x\le 8\right]\)
\item \(\left\{\begin{array}{l}2x^2-x-1\le 0\\3x+7>0\\x^2-10x+9\le 0
              \end{array}\right.\)
 \hfill \(\left[x=1\right]\)
\item \(\left\{\begin{array}{l}2x^2-x-1<0\\3x+7>0\\x^2-10x+9\le 0
              \end{array}\right.\)
 \hfill \(\left[\emptyset\right]\)
\item \(\left\{\begin{array}{l}x^2-10x+25>0\\x<7\end{array}\right.\)
 \hfill \(\left[x<5\vee 5<x<7\right]\)
\item \(\left\{\begin{array}{l}x^2-10x+25\ge 0\\x<7\end{array}\right.\)
 \hfill \(\left[x<7\right]\)
\item \(\left\{\begin{array}{l}x^4-8\ge 1\\\frac{5-x} x<\frac 1 2\\x^3-1<0
              \end{array}\right.\)
 \hfill \(\left[x\le -\sqrt 3\right]\)
\item \(\left\{\begin{array}{l}x^2-4x+3\le 0\\x^2-4>0 \\x^2+1>0 \\x-1>0 
              \end{array}\right.\)
 \hfill \(\left[2<x\le 3\right]\)
\item \(\left\{\begin{array}{l}x^2-5x+6\le 0\\x^2-1>0 \\x^2+1<0 \\x-1>0 
              \end{array}\right.\)
 \hfill \(\left[\emptyset\right]\)
% \item \(\left\{\begin{array}{l}
%            x^2-2x+1\ge 0 \\
%            x^2+5x\ge 0 \\
%            x^2+1>0 \\
%            x^2-2x+7>0\end{array}\right.\)
%  \hfill \(\left[x\le -5\vee x\ge 0\right]\)
% \item \(\left\{\begin{array}{l}x^2-3x+2>0\\x^2-3x+2<0\\2x^2-x-1>0\\x^2-2x>0 
%               \end{array}\right.\)
%  \hfill \(\left[\emptyset\right]\)
% \item \(\left\{\begin{array}{l}
%            x^2-3x+2\le 0\\
%            x^2-4x+4\le 0\\
%            x^2-3x+2\ge 0\\
%            x^2-4x+4\ge 0\end{array}\right.\)
%  \hfill \(\left[x=2\right]\)
\end{enumeratea}
\end{multicols}
\end{esercizio}

\begin{esercizio}[\Ast]
 \label{ese:21.41}
 Risolvi i seguenti sistemi di disequazioni.
 \begin{enumeratea}
\item \(\left\{\begin{array}{l}x^2-4x+4>0\\x\le 6\\1-x^2\le 0\end{array}\right.\)
 \hfill \(\left[x\le -1\vee 1\le x<2\vee 2<x\le 6\right]\)
\item {\longarray \(\left\{\begin{array}{l}
\dfrac{2x+3}{3}>x-1\\
\dfrac{x-4}{5}<\dfrac{2x+1}{3}
\end{array}\right.\)}
 \hfill \(\left[-{\frac{17}{7}}<x<6\right]\)
\item {\longarray \(\left\{\begin{array}{l}
16(x+1)-2+(x-3)^{2}\le(x+5)^{2}\\
\dfrac{x+5}{3}+3+2\cdot\dfrac{x-1}{3}\le x+4
\end{array}\right.\)}
 \hfill \(\left[\insR\right]\)
\item \(\left\{\begin{array}{l}
x+\dfrac{1}{2}<\dfrac{1}{3}(x+3)-1\\
(x+3)^{2}\ge (x-2)(x+2)
\end{array}\right.\)
 \hfill \(\left[-{\frac{13}{6}}\le x<-{\frac{3}{4}}\right]\)
\item {\longarray \(\left\{\begin{array}{l}
2\left(x-\dfrac{1}{3}\right)+x>3x-2\\
\dfrac{x}{3}-\dfrac{1}{2}\ge \dfrac{x}{4}-\dfrac{x}{6}
  \end{array}\right.\)}
 \hfill \(\left[x\ge~2\right]\)
\item \(\left\{\begin{array}{l}
  \dfrac{3}{2}x+\dfrac{1}{4}<5\cdot\left(\dfrac{2}{3}x-\dfrac{1}{2}\right)\\
  x^2-2x+1\ge~0
\end{array}\right.\)
 \hfill \(\left[x>\frac{3}{2}\right]\)
\item {\longarray \(\left\{\begin{array}{l}
3\left(x-\dfrac{4}{3}\right)+\dfrac{2-x}{3}+x-\dfrac{x-1}{3}>0\\
\left[1-\dfrac{1}{6}(2x+1)\right]+\left(x-\dfrac{1}{2}\right)^{2}<(x+1)^{2}+\dfrac{1}{3}(1+2x)
   \end{array}\right.\)}
 \hfill \(\left[x>\frac{9}{10}\right]\)
% \item {\longarray \(\left\{\begin{array}{l}
% \left(x-\dfrac{1}{2}\right)\left(x+\dfrac{1}{2}\right)>\left(x-\dfrac{1}{2}
% \right)^{2}\\
% \left(x-\dfrac{1}{2}\right)\left(x+\dfrac{1}{2}\right)<\left(x-\dfrac{1}{2}
% \right)^{2}+\left(x+\dfrac{1}{2}\right)^{2}
% \end{array}\right.\)}
%  \hfill \(\left[x>\frac{1}{2}\right]\)
\item \(\left\{\begin{array}{l}
           x^2-3x+2\le 0\\
           x^2-4x+4\le 0\\
           x^2-x+10>0\\
           x^2-2x\le 0 \end{array}\right.\)
 \hfill \(\left[x=2\right]\)
\item \(\left\{\begin{array}{l}
           \frac{4-x^2+3x}{x^2-x}>0 \\
           \frac{x^2-x-2}{-3x^2+3x+18} \le 0 \end{array}\right.\)
 \hfill \(\left[3<x<4\vee -1<x<0\vee 1<x\le 2\right]\)
\item \(\left\{\begin{array}{l}
           x^3-5x^2-14x\ge 0 \\ 
           \frac{2x+1}{2x} > \frac 3{x+1}\end{array}\right.\)
 \hfill \(\left[2\le x<-1\vee x\ge 7\right]\)
\item \(\left\{\begin{array}{l}
           \frac 1 x>\frac 1{x-3}\\3x-1-2x^2<0\\
           \frac{x^2-6x+5}{2-x}>0 \end{array}\right.\)
 \hfill \(\left[0<x<\frac 1 2\vee 2<x<3\right]\)
% \item \(\left\{\begin{array}{l}x^2-2x+1>0\\
%                               x^2+5x\ge 0 \\
%                               x^2+x+23>0\\
%                               x^2-2x+7>0
%               \end{array}\right.\)
%  \hfill \(\left[x\le -5\vee 0\le x<1\vee x>1\right]\)
 \end{enumeratea}
\end{esercizio}

\begin{esercizio}[\Ast]
\label{ese:4.82}
Dato il sistema \(\left\{\begin{array}{l}{x(x-3)>3\left(\frac{x^2} 
2-2x\right)}\\{2+x\cdot \frac{3x-7} 3\ge 5-\frac 1 3x}\end{array}\right.\) 
determina i numeri naturali che lo risolvono. \hfill \(\left[3, 4, 5\right]\)
\end{esercizio}

\begin{esercizio}[\Ast]
 \label{ese:4.83}
Per quali valori di \( x \) le due funzioni \(f_1=x^4-x^3+x-1\) e \(f_2=x^4-8x\) 
assumono contemporaneamente valore positivo? \hfill \(\left[x<-1\vee x>2\right]\)
\end{esercizio}
