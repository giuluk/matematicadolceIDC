% (c)~2014 Claudio Carboncini - claudio.carboncini@gmail.com
% (c)~2014 Dimitrios Vrettos - d.vrettos@gmail.com
% (c) 2015 Daniele Zambelli daniele.zambelli@gmail.com

\section{TODO}

% \section{Esercizi}
% 
% \subsection{Esercizi dei singoli paragrafi}
% 
% \subsubsection*{\numnameref{sec:01_}}
% 
% \begin{esercizio}
% \label{ese:D.19}
% testo esercizio
% \end{esercizio}
% 
% \begin{esercizio}\label{ese:03.1}
% Consegna:
%  \begin{enumeratea}
%   \item  
%  \end{enumeratea}
% \end{esercizio}
% 
% \subsection{Esercizi riepilogativi}
% 
% \begin{esercizio}
% \label{ese:D.19}
% testo esercizio
% \end{esercizio}
% 
% \begin{esercizio}\label{ese:03.1}
% Consegna:
%  \begin{enumeratea}
%   \item  
%  \end{enumeratea}
% \end{esercizio}

\section{Esercizi}
\subsection{Esercizi dei singoli paragrafi}
\subsection*{6.1 - Sistemi di secondo grado}

\begin{esercizio}[\Ast]
 \label{ese:6.1}
Risolvere i seguenti sistemi di secondo grado.
\begin{multicols}{2}
 \begin{enumeratea}
 \item~$\left\{\begin{array}{l}x^2+2y^2=3\\x+y=2\end{array}\right.$
 \item~$\left\{\begin{array}{l}3x^2-4y^2-x=0\\x-2y=1\end{array}\right.$
 \item~$\left\{\begin{array}{l}4x^2+2y^2-6=0\\x=y\end{array}\right.$
 \item~$\left\{\begin{array}{l}2x^2-6xy=x\\3x+5y=-2\end{array}\right.$
 \end{enumeratea}
 \end{multicols}
\end{esercizio}

\begin{esercizio}[\Ast]
\label{ese:6.2}
Risolvere i seguenti sistemi di secondo grado.
\begin{multicols}{2}
 \begin{enumeratea}
 \item~$\left\{\begin{array}{l}y^2-3y=2xy\\y=x-3\end{array}\right.$
 \item~$\left\{\begin{array}{l}xy-x^2+2y^2=y-2x\\x+y=0\end{array}\right.$
 \item~$\left\{\begin{array}{l}{x+2y=-1}\\{x+5y^2=23}\end{array}\right.$
 \item~$\left\{\begin{array}{l}{x-5y=2}\\{x^2+2y^2=4}\end{array}\right.$
 \end{enumeratea}
 \end{multicols}
\end{esercizio}

\begin{esercizio}[\Ast]
 \label{ese:6.3}
Risolvere i seguenti sistemi di secondo grado.
\begin{multicols}{2}
 \begin{enumeratea}
 \item~$\left\{\begin{array}{l}3x-y=2\\x^2+2xy+y^2=0\end{array}\right.$
 \item~$\left\{\begin{array}{l}x^2-4xy+4y^2-1=0\\x=y+2\end{array}\right.$
 \item~$\left\{\begin{array}{l}x^2+y^2=1\\x+3y=10\end{array}\right.$
 \item~$\left\{\begin{array}{l}x^2+y^2=2\\x+y=2\end{array}\right.$
 \end{enumeratea}
 \end{multicols}
\end{esercizio}

\begin{esercizio}[\Ast]
 \label{ese:6.4}
Risolvere i seguenti sistemi di secondo grado.
\begin{multicols}{2}
 \begin{enumeratea}
 \item~$\left\{\begin{array}{l}x+y=1\\x^2+y^2-3x+2y=3\end{array}\right.$
 \item~$\left\{\begin{array}{l}3x+y=2\\x^2-y^2=1\end{array}\right.$
 \item~$\left\{\begin{array}{l}5x^2-y^2+4y-2x+2=0\\x-y=1\end{array}\right.$
 \item~$\left\{\begin{array}{l}x^2+y^2=25\\4x-3y+7=0\end{array}\right.$
 \end{enumeratea}
 \end{multicols}
\end{esercizio}

\begin{esercizio}[\Ast]
\label{ese:6.5}
Risolvere i seguenti sistemi di secondo grado.
\begin{multicols}{2}
 \begin{enumeratea}
 \item~$\left\{\begin{array}{l}x+2y=3\\x^2-4xy+2y^2+x+y-1=0\end{array}\right.$
 \item~$\left\{\begin{array}{l}x^2-4xy+4y^2-1=0\\x=y+2\end{array}\right.$
 \item~$\left\{\begin{array}{l}2x^2+xy-7x-2y=-6\\2x+y=3\end{array}\right.$
 \item~$\left\{\begin{array}{l}x-2y-7=0\\x^2-xy=4\end{array}\right.$
 \end{enumeratea}
 \end{multicols}
\end{esercizio}

\begin{esercizio}[\Ast]
\label{ese:6.6}
Risolvere i seguenti sistemi di secondo grado.
\begin{multicols}{2}
 \begin{enumeratea}
 \item~$\left\{\begin{array}{l}x+y=0\\x^2+y^2-x-10=0\end{array}\right.$
 \item~$\left\{\begin{array}{l}x^2+2y^2-3xy-x+2y-4=0\\2x-3y+4=0\end{array}\right.$
 \item~$\left\{\begin{array}{l}x^2-4y^2=0\\4x-7y=2\end{array}\right.$
 \item~$\left\{\begin{array}{l}x-2y=1\\x^2+y^2-2x=1\end{array}\right.$
 \end{enumeratea}
 \end{multicols}
\end{esercizio}
\newpage
\begin{esercizio}[\Ast]
 \label{ese:6.7}
Risolvere i seguenti sistemi di secondo grado.
 \begin{enumeratea}
 \item~$\left\{\begin{array}{l}x+y=1\\x^2+y^2-2xy-2y-2=0\end{array}\right.$
 \item~$\left\{\begin{array}{l}9x^2-12xy+4y^2-2x+6y=8\\x-2y=2\end{array}\right.$
 \item~$\left\{\begin{array}{l}3x+y=4\\x^2-y^2=1\end{array}\right.$
 \item~$\left\{\begin{array}{l}\frac 1 2(2y-x)(y+x)-(x+y)^2+\frac 3 2x(x+y+1)+2(y-1)=0\\\frac 2 3(x-3)^2+4\left(x-\frac 3 2\right)=2({xy}+1)\end{array}\right.$
 \end{enumeratea}
\end{esercizio}

\begin{esercizio}[\Ast]
 \label{ese:6.8}
Risolvere i seguenti sistemi, dopo aver eseguito la discussione sul parametro.
\begin{multicols}{2}
 \begin{enumeratea}
 \item~$\left\{\begin{array}{l}x+y=3 \\x^2+y^2=k \end{array}\right.$
 \item~$ \left\{\begin{array}{l}ky+2x=4\\xy=2\end{array}\right. $
 \item~$ \left\{\begin{array}{l}y=kx-1 \\y^2-kx^2+1=0\end{array}\right. $
 \item~$\left\{\begin{array}{l}y=kx-2k \\x^2-2y-x=2\end{array}\right.$
 \end{enumeratea}
 \end{multicols}
\end{esercizio}

\begin{esercizio}[\Ast]
 \label{ese:6.9}
Risolvere i seguenti sistemi, dopo aver eseguito la discussione sul parametro.
\begin{multicols}{2}
 \begin{enumeratea}
 \item~$\left\{\begin{array}{l}y=x+k \\y=3x^2+2x\end{array}\right.$
 \item~$ \left\{\begin{array}{l}y=-x+k \\x^2-y^2-1=0\end{array}\right. $
 \item~$\left\{\begin{array}{l}y+x-k=0 \\{xy}+2{kx}-3{ky}-6k^2=0\end{array}\right.$
 \item~$ \left\{\begin{array}{l}y-x+k=0 \\y-x^2+4x-3=0\end{array}\right. $
 \end{enumeratea}
 \end{multicols}
\end{esercizio}

\begin{esercizio}[\Ast]
 \label{ese:6.10}
Trovare le soluzioni dei seguenti sistemi frazionari.
\begin{multicols}{2}
 \begin{enumeratea}
 \item~$\left\{\begin{array}{l}x^2+y^2=4\\\frac{x+2y}{x-1}=2\end{array}\right.$
 \item~$\left\{\begin{array}{l}\frac{x+2y}{x-y}=4\\x^2+y^2+3x-2y=1\end{array}\right.$
 \item~$\left\{\begin{array}{l}\frac{2x+y}{x+2y}=3\\xy+3y=1\end{array}\right.$
 \item~$\left\{\begin{array}{l}\frac{3x-2y} x=\frac{1-x}{y-1}\\2x-y=1\end{array}\right.$
 \end{enumeratea}
 \end{multicols}
\end{esercizio}

\begin{esercizio}[\Ast]
 \label{ese:6.11}
Trovare le soluzioni dei seguenti sistemi frazionari.
\begin{multicols}{2}
 \begin{enumeratea}
 \item~$\left\{\begin{array}{l}\frac{x+y}{x-2}=y+\frac 1 3\\y=2x+2\end{array}\right.$
 \item~$\left\{\begin{array}{l}\frac{2x+1}{y-2}=\frac{y-1}{x+1}\\2x+2y=3\end{array}\right.$
 \item~$\left\{\begin{array}{l}\frac{y-1}{x+y}=x\\x-y=0\end{array}\right.$
 \item~$\left\{\begin{array}{l}\frac{x+1}{2y-1}=y\\2y-x=-4\end{array}\right.$
 \end{enumeratea}
 \end{multicols}
\end{esercizio}

\begin{esercizio}
 \label{ese:6.12}
Risolvere i seguenti sistemi di secondo grado in tre incognite.
\begin{multicols}{2}
 \begin{enumeratea}
 \item~$\left\{\begin{array}{l}x-3y-z=-4\\3x+2y+z=6\\4x^2+2xz+y^2=6\end{array}\right.$
 \item~$\left\{\begin{array}{l}x+y=5\\2x-y+3z=9\\x^2-y+z^2=1\end{array}\right.$
 \item~$\left\{\begin{array}{l}x-y+z=1\\2x-y+z=0\\x^2-y+z=3\end{array}\right.$
 \item~$\left\{\begin{array}{l}x-y+2z=3\\2x-2y+z=1\\x^2-y^2+z=12\end{array}\right.$
 \end{enumeratea}
 \end{multicols}
\end{esercizio}

\begin{esercizio}[\Ast]
 \label{ese:6.13}
Risolvere i seguenti sistemi di secondo grado in tre incognite.
\begin{multicols}{2}
 \begin{enumeratea}
 \item~$\left\{\begin{array}{l}2x-3y=-3\\5y+2z=1\\x^2+y^2+z^2=1\end{array}\right.$
 \item~$\left\{\begin{array}{l}x-2y+z=3\\x+2y+z=3\\x^2+y^2+z^2=29\end{array}\right.$
 \item~$\left\{\begin{array}{l}x+y-z=0\\x-y+3z=9\\x^2-y+z=12\end{array}\right.$
 \item~$\left\{\begin{array}{l}x-y=1\\x+y+z=0\\x^2+xy-z=0\end{array}\right.$
 \item~$\left\{\begin{array}{l}x-y-z=-1\\x+y+z=1\\x+y^2+z^2=32\end{array}\right.$
 \end{enumeratea}
 \end{multicols}
\end{esercizio}

\subsection*{6.2 - Sistemi simmetrici}

\begin{esercizio}[\Ast]
 \label{ese:6.14}
Risolvere i seguenti sistemi simmetrici di secondo grado.
\begin{multicols}{2}
 \begin{enumeratea}
 \item~$\left\{\begin{array}{l}x+y=4\\{xy}=3\end{array}\right.$
 \item~$\left\{\begin{array}{l}x+y=1\\{xy}=7 \end{array}\right.$
 \item~$\left\{\begin{array}{l}x+y=5\\{xy}=6 \end{array}\right.$
 \item~$\left\{\begin{array}{l}x+y=-5\\{xy}=-6 \end{array}\right.$
 \end{enumeratea}
 \end{multicols}
\end{esercizio}

\begin{esercizio}[\Ast]
 \label{ese:6.15}
Risolvere i seguenti sistemi simmetrici di secondo grado.
\begin{multicols}{2}
 \begin{enumeratea}
 \item~$\left\{\begin{array}{l}x+y=3\\{xy}=2 \end{array}\right.$
 \item~$\left\{\begin{array}{l}x+y=3\\{xy}=-4\end{array}\right.$
 \item~$\left\{\begin{array}{l}x+y=-4\\{xy}=4 \end{array}\right.$
 \item~$\left\{\begin{array}{l}x+y=6\\{xy}=9 \end{array}\right.$
 \end{enumeratea}
 \end{multicols}
\end{esercizio}

\begin{esercizio}[\Ast]
 \label{ese:6.16}
Risolvere i seguenti sistemi simmetrici di secondo grado.
\begin{multicols}{3}
 \begin{enumeratea}
 \item~$\left\{\begin{array}{l}x+y=2\\{xy}=10 \end{array}\right.$
 \item~$\left\{\begin{array}{l}x+y=7\\{xy}=12 \end{array}\right.$
 \item~$\left\{\begin{array}{l}x+y=-1\\{xy}=2 \end{array}\right.$
 \item~$\left\{\begin{array}{l}x+y=12\\{xy}=-13 \end{array}\right.$
 \item~$\left\{\begin{array}{l}x+y=\frac 6 5\\{xy}=\frac 9{25}\end{array}\right.$
 \item~$\left\{\begin{array}{l}x+y=4\\{xy}=50 \end{array}\right.$
 \end{enumeratea}
 \end{multicols}
\end{esercizio}

\begin{esercizio}[\Ast]
 \label{ese:6.17}
Risolvere i seguenti sistemi simmetrici di secondo grado.
\begin{multicols}{2}
 \begin{enumeratea}
 \item~$\left\{\begin{array}{l}x+y=-5\\{xy}=-14 \end{array}\right.$
 \item~$\left\{\begin{array}{l}x+y=5\\{xy}=-14\end{array}\right.$
 \item~$\left\{\begin{array}{l}x+y=\frac 1 4\\{xy}=-\frac 3 8\end{array}\right.$
 \item~$\left\{\begin{array}{l}x+y=2\\{xy}=-10\end{array}\right.$
 \end{enumeratea}
 \end{multicols}
\end{esercizio}

\begin{esercizio}[\Ast]
 \label{ese:6.18}
Risolvere i seguenti sistemi simmetrici di secondo grado.
\begin{multicols}{2}
 \begin{enumeratea}
 \item~$\left\{\begin{array}{l}x+y=4\\{xy}=0 \end{array}\right.$
 \item~$\left\{\begin{array}{l}x+y=\frac 5 2\\{xy}=-\frac 7 2\end{array}\right.$
 \item~$\left\{\begin{array}{l}x+y=-5\\{xy}=2 \end{array}\right.$
 \item~$\left\{\begin{array}{l}x+y=\frac 4 3\\{xy}=-\frac 1 2 \end{array}\right.$
 \end{enumeratea}
 \end{multicols}
\end{esercizio}

\begin{esercizio}[\Ast]
 \label{ese:6.19}
Risolvere i seguenti sistemi simmetrici di secondo grado.
\begin{multicols}{2}
 \begin{enumeratea}
 \item~$\left\{\begin{array}{l}x+y=\frac 5 2\\{xy}=-\frac 9 2\end{array}\right.$
 \item~$\left\{\begin{array}{l}x+y=2\\{xy}=-\frac 1 3\end{array}\right.$
 \item~$\left\{\begin{array}{l}x+y=1\\{xy}=-3\end{array}\right.$
 \item~$\left\{\begin{array}{l}x+y=4\\{xy}=-50\end{array}\right.$
 \end{enumeratea}
 \end{multicols}
\end{esercizio}

\begin{esercizio}[\Ast]
\label{ese:6.20}
Risolvere i seguenti sistemi riconducibili al sistema simmetrico fondamentale.
\begin{multicols}{2}
 \begin{enumeratea}
 \item~$\left\{\begin{array}{l}x+y=1 \\x^2+y^2=1\end{array}\right.$
 \item~$\left\{\begin{array}{l}x+y=2 \\x^2+y^2=2\end{array}\right.$
 \item~$\left\{\begin{array}{l}x+y=3 \\x^2+y^2=5\end{array}\right.$
 \item~$\left\{\begin{array}{l}x+y=2 \\x^2+y^2+x+y=1\end{array}\right.$
 \item~$\left\{\begin{array}{l}x+y=4\\x^2+y^2=8\end{array}\right.$
 \item~$\left\{\begin{array}{l}x+y=2 \\x^2+y^2-3xy=4\end{array}\right.$
 \end{enumeratea}
 \end{multicols}
\end{esercizio}

\begin{esercizio}[\Ast]
\label{ese:6.21}
Risolvere i seguenti sistemi riconducibili al sistema simmetrico fondamentale.
\begin{multicols}{2}
 \begin{enumeratea}
 \item~$\left\{\begin{array}{l}{x+y=-12}\\{x^2+y^2=72}\end{array}\right.$
 \item~$\left\{\begin{array}{l}{2x+2y=-2}\\{(y-x)^2-{xy}=101}\end{array}\right.$
 \item~$\left\{\begin{array}{l}{-4x-4y=-44}\\{2x^2+2y^2-3{xy}=74}\end{array}\right.$
 \item~$\left\{\begin{array}{l}x+y=3 \\x^2+y^2-4x-4y=5\end{array}\right.$
 \end{enumeratea}
 \end{multicols}
\end{esercizio}

\begin{esercizio}[\Ast]
 \label{ese:6.22}
Risolvere i seguenti sistemi riconducibili al sistema simmetrico fondamentale.
\begin{multicols}{2}
 \begin{enumeratea}
 \item~$\left\{\begin{array}{l}x+y=7 \\x^2+y^2=29\end{array}\right.$
 \item~$\left\{\begin{array}{l}2x+2y=-2\\4x^2+4y^2=52\end{array}\right.$
 \item~$\left\{\begin{array}{l}\frac{x+y} 2=\frac 3 4\\3x^2+3y^2=\frac{15} 4\end{array}\right.$
 \item~$\left\{\begin{array}{l}x+y=-3 \\x^2+y^2-5xy=37\end{array}\right.$
 \end{enumeratea}
 \end{multicols}
\end{esercizio}

\begin{esercizio}[\Ast]
\label{ese:6.23}
Risolvere i seguenti sistemi riconducibili al sistema simmetrico fondamentale.
\begin{multicols}{2}
 \begin{enumeratea}
 \item~$\left\{\begin{array}{l}x+y=-6 \\x^2+y^2-xy=84\end{array}\right.$
 \item~$\left\{\begin{array}{l}x+y=-5 \\x^2+y^2-4xy+5x+5y=36\end{array}\right.$
 \item~$\left\{\begin{array}{l}x+y=-7 \\x^2+y^2-6xy-3x-3y=44\end{array}\right.$
 \item~$\left\{\begin{array}{l}x^2+y^2=-1\\x+y=6 \end{array}\right.$
 \end{enumeratea}
 \end{multicols}
\end{esercizio}

\begin{esercizio}[\Ast]
\label{ese:6.24}
Risolvere i seguenti sistemi riconducibili al sistema simmetrico fondamentale.
\begin{multicols}{2}
 \begin{enumeratea}
 \item~$\left\{\begin{array}{l}x^2+y^2=1\\x+y=-7\end{array}\right.$
 \item~$\left\{\begin{array}{l}x^2+y^2=18\\x+y=6 \end{array}\right.$
 \item~$\left\{\begin{array}{l}x^2+y^2-4xy-6x-6y=1\\x+y=1 \end{array}\right.$
 \item~$\left\{\begin{array}{l}x^2+y^2=8\\x+y=3\end{array}\right.$
 \end{enumeratea}
 \end{multicols}
\end{esercizio}
\newpage
\begin{esercizio}[\Ast]
\label{ese:6.25}
Risolvere i seguenti sistemi riconducibili a sistemi simmetrici.
\begin{multicols}{3}
 \begin{enumeratea}
 \item~$\left\{\begin{array}{l}{x-y=1}\\{x^2+y^2=5}\end{array}\right.$
 \item~$\left\{\begin{array}{l}{\frac 1 x+\frac 1 y=-12}\\{{xy}=\frac 1{35}}\end{array}\right.$
 \item~$\left\{\begin{array}{l}{-2x+y=3}\\{{xy}=1}\end{array}\right.$
 \end{enumeratea}
 \end{multicols}
\end{esercizio}

\begin{esercizio}[\Ast]
 \label{ese:6.26}
Risolvere i seguenti sistemi di grado superiore al secondo.
\begin{multicols}{2}
 \begin{enumeratea}
 \item~$\left\{\begin{array}{l}{x+y=-1}\\{x^3+y^3=-1}\end{array}\right.$
 \item~$\left\{\begin{array}{l}{{xy}=-2}\\{x^2+y^2=13}\end{array}\right.$
 \item~$\left\{\begin{array}{l}{x+y=-2}\\{x^3+y^3-{xy}=-5}\end{array}\right.$
 \item~$\left\{\begin{array}{l}{x+y=8}\\{x^3+y^3=152}\end{array}\right.$
 \end{enumeratea}
 \end{multicols}
\end{esercizio}

\begin{esercizio}[\Ast]
 \label{ese:6.27}
Risolvere i seguenti sistemi di grado superiore al secondo.
\begin{multicols}{2}
 \begin{enumeratea}
 \item~$\left\{\begin{array}{l}x^3+y^3=9\\x+y=3\end{array}\right.$
 \item~$\left\{\begin{array}{l}x^3+y^3=-342\\x+y=-6\end{array}\right.$
 \item~$\left\{\begin{array}{l}{x^3-y^3=351}{{xy}=-14}\end{array}\right.$
 \item~$\left\{\begin{array}{l}x^3+y^3=35\\x+y=5\end{array}\right.$
 \end{enumeratea}
 \end{multicols}
\end{esercizio}

\begin{esercizio}[\Ast]
 \label{ese:6.28}
Risolvere i seguenti sistemi di grado superiore al secondo.
\begin{multicols}{2}
 \begin{enumeratea}
 \item~$\left\{\begin{array}{l}x^4+y^4=2\\x+y=0\end{array}\right.$
 \item~$\left\{\begin{array}{l}x^4+y^4=17\\x+y=-3\end{array}\right.$
 \item~$\left\{\begin{array}{l}x^3+y^3=-35\\xy=6\end{array}\right.$
 \item~$\left\{\begin{array}{l}x^3+y^3=-26\\xy=-3\end{array}\right.$
 \end{enumeratea}
 \end{multicols}
\end{esercizio}

\begin{esercizio}[\Ast]
 \label{ese:6.29}
Risolvere i seguenti sistemi di grado superiore al secondo.
\begin{multicols}{2}
 \begin{enumeratea}
 \item~$\left\{\begin{array}{l}{x+y=3}\\{x^4+y^4=17}\end{array}\right.$
 \item~$\left\{\begin{array}{l}{x+y=-1}\\{8x^4+8y^4=41}\end{array}\right.$
 \item~$\left\{\begin{array}{l}{x+y=3}\\{x^4+y^4=2}\end{array}\right.$
 \item~$\left\{\begin{array}{l}{x+y=5}\\{x^4+y^4=257}\end{array}\right.$
 \end{enumeratea}
 \end{multicols}
\end{esercizio}

\begin{esercizio}[\Ast]
 \label{ese:6.30}
Risolvere i seguenti sistemi di grado superiore al secondo.
\begin{multicols}{2}
 \begin{enumeratea}
 \item~$\left\{\begin{array}{l}x^4+y^4=2\\xy=1\end{array}\right.$
 \item~$\left\{\begin{array}{l}x^4+y^4=17\\xy=-2\end{array}\right.$
 \item~$\left\{\begin{array}{l}{x+y=-1}\\{x^5+y^5=-211}\end{array}\right.$
 \item~$\left\{\begin{array}{l}x^5+y^5=64\\x+y=4\end{array}\right.$
 \end{enumeratea}
 \end{multicols}
\end{esercizio}

\begin{esercizio}[\Ast]
 \label{ese:6.31}
Risolvere i seguenti sistemi di grado superiore al secondo.
\begin{multicols}{2}
 \begin{enumeratea}
 \item~$\left\{\begin{array}{l}x^5+y^5=-2882\\x+y=-2\end{array}\right.$
 \item~$\left\{\begin{array}{l}x^5+y^5=2\\x+y=0\end{array}\right.$
 \item~$\left\{\begin{array}{l}x^5+y^5=31\\xy=-2\end{array}\right.$
 \item~$\left\{\begin{array}{l}x^4+y^4=337\\xy=12\end{array}\right.$
 \end{enumeratea}
 \end{multicols}
\end{esercizio}
\newpage
\begin{esercizio}[\Ast]
 \label{ese:6.32}
Risolvere i seguenti sistemi di grado superiore al secondo.
\begin{multicols}{2}
 \begin{enumeratea}
 \item~$\left\{\begin{array}{l}x^3+y^3=\frac{511} 8\\xy=-2\end{array}\right.$
 \item~$\left\{\begin{array}{l}x^2+y^2=5\\xy=2 \end{array}\right.$
 \item~$\left\{\begin{array}{l}x^2+y^2=34\\xy=15 \end{array}\right.$
 \item~$\left\{\begin{array}{l}xy=1 \\x^2+y^2+3xy=5\end{array}\right.$
 \end{enumeratea}
 \end{multicols}
\end{esercizio}

\begin{esercizio}[\Ast]
 \label{ese:6.33}
Risolvere i seguenti sistemi di grado superiore al secondo.
\begin{multicols}{2}
 \begin{enumeratea}
 \item~$\left\{\begin{array}{l}xy=12 \\x^2+y^2=25\end{array}\right.$
 \item~$\left\{\begin{array}{l}xy=1 \\x^2+y^2-4xy=-2\end{array}\right.$
 \item~$\left\{\begin{array}{l}x^2+y^2=5\\xy=3 \end{array}\right.$
 \item~$\left\{\begin{array}{l}x^2+y^2=18\\xy=9 \end{array}\right.$
 \end{enumeratea}
 \end{multicols}
\end{esercizio}

\begin{esercizio}[\Ast]
 \label{ese:6.34}
Risolvere i seguenti sistemi di grado superiore al secondo.
\begin{multicols}{2}
 \begin{enumeratea}
 \item~$\left\{\begin{array}{l}x^2+y^2+3xy=10\\xy=6 \end{array}\right.$
 \item~$\left\{\begin{array}{l}x^2+y^2+5xy-2x-2y=3\\xy=1 \end{array}\right.$
 \item~$\left\{\begin{array}{l}x^2+y^2-6xy+3x+3y=2\\xy=2 \end{array}\right.$
 \item~$\left\{\begin{array}{l}x^2+y^2=8\\xy=-3\end{array}\right.$
 \end{enumeratea}
 \end{multicols}
\end{esercizio}

\begin{esercizio}[\Ast]
\label{ese:6.35}
Risolvere i seguenti sistemi di grado superiore al secondo.
 \begin{enumeratea}
 \item~$\left\{\begin{array}{l}x^2+y^2+5xy+x+y=-6\\xy=-2 \end{array}\right.$
 \item~$\left\{\begin{array}{l}{x+y=-\frac 1 3}\\{x^5+y^5=-\frac{31}{243}}\end{array}\right.$
 \item~$\left\{\begin{array}{l}x^2+y^2+5xy+x+y=-\frac{25} 4\\xy=-2 \end{array}\right.$
 \item~$\left\{\begin{array}{l}{x+y=1}\\{x^5+y^5=-2}\end{array}\right.$
 \item~$\left\{\begin{array}{l}{x+y=1}\\{x^5+y^5+7{xy}=17}\end{array}\right.$
 \end{enumeratea}
\end{esercizio}

\subsection*{6.3 - Sistemi omogenei di quarto grado}

\begin{esercizio}[\Ast]
 \label{ese:6.36}
Risolvi i seguenti sistemi omogenei.
\begin{multicols}{2}
 \begin{enumeratea}
 \item~$\left\{\begin{array}{l}x^2-2xy+y^2=0\\x^2+3xy-2y^2=0\end{array}\right.$
 \item~$\left\{\begin{array}{l}3x^2-2xy-y^2=0\\2x^2+xy-3y^2=0\end{array}\right.$
 \item~$\left\{\begin{array}{l}x^2-6{xy}+8y^2=0 \\x^2+4{xy}-5y^2=0 \end{array}\right.$
 \item~$\left\{\begin{array}{l}2x^2+xy-y^2=0\\4x^2-2xy-6y^2=0\end{array}\right.$
 \end{enumeratea}
\end{multicols}
\end{esercizio}

\begin{esercizio}[\Ast]
\label{ese:6.37}
Risolvi i seguenti sistemi omogenei.
\begin{multicols}{2}
 \begin{enumeratea}
 \item~$\left\{\begin{array}{l}x^2-5xy+6y^2=0\\x^2-4xy+4y^2=0\end{array}\right.$
 \item~$\left\{\begin{array}{l}x^2-5xy+6y^2=0\\x^2+2xy-8y^2=0\end{array}\right.$
 \item~$\left\{\begin{array}{l}x^2+xy-2y^2=0\\x^2+5xy+6y^2=0\end{array}\right.$
 \item~$\left\{\begin{array}{l}x^2+7xy+12y^2=0\\2x^2+xy+6y^2=0\end{array}\right.$
 \end{enumeratea}
\end{multicols}
\end{esercizio}

\begin{esercizio}[\Ast]
\label{ese:6.38}
Risolvi i seguenti sistemi omogenei.
\begin{multicols}{2}
 \begin{enumeratea}
 \item~$\left\{\begin{array}{l}x^2+6xy+8y^2=0\\2x^2+12xy+16y^2=0\end{array}\right.$
 \item~$\left\{\begin{array}{l}-4x^2-7{xy}+2y^2=0 \\12x^2+21{xy}-6y^2=0 \end{array}\right.$
 \item~$\left\{\begin{array}{l}x^2+2xy+y^2=0\\x^2+3xy+2y^2=0\end{array}\right.$
 \item~$\left\{\begin{array}{l}x^2+4xy=0\\x^2+2xy-4y^2-4=0\end{array}\right.$
 \end{enumeratea}
\end{multicols}
\end{esercizio}

\begin{esercizio}[\Ast]
\label{ese:6.39}
Risolvi i seguenti sistemi omogenei.
\begin{multicols}{2}
 \begin{enumeratea}
 \item~$\left\{\begin{array}{l}x^2-8xy+15y^2=0\\x^2-2xy+y^2=1\end{array}\right.$
 \item~$\left\{\begin{array}{l}4x^2-y^2=0\\x^2-y^2=-3\end{array}\right.$
 \item~$\left\{\begin{array}{l}x^2+3xy+2y^2=0\\x^2-3xy-y^2=3\end{array}\right.$
 \item~$\left\{\begin{array}{l}x^2-4xy+4y^2=0\\2x^2-y^2=-1\end{array}\right.$
 \end{enumeratea}
\end{multicols}
\end{esercizio}

\begin{esercizio}[\Ast]
\label{ese:6.40}
Risolvi i seguenti sistemi omogenei.
\begin{multicols}{2}
 \begin{enumeratea}
 \item~$\left\{\begin{array}{l}6x^2+5xy+y^2=12\\x^2+4xy+y^2=6\end{array}\right.$
 \item~$\left\{\begin{array}{l}x^2-xy-2y^2=0\\x^2-4xy+y^2=6\end{array}\right.$
 \item~$\left\{\begin{array}{l}x^2+y^2=3\\x^2-xy+y^2=3\end{array}\right.$
 \item~$\left\{\begin{array}{l}x^2-3xy+5y^2=1\\x^2+xy+y^2=1\end{array}\right.$
 \end{enumeratea}
\end{multicols}
\end{esercizio}

 \begin{esercizio}[\Ast]
\label{ese:6.41}
Risolvi i seguenti sistemi omogenei.
\begin{multicols}{2}
 \begin{enumeratea}
 \item~$\left\{\begin{array}{l}x^2+y^2=5\\x^2-3xy+y^2=11\end{array}\right.$
 \item~$\left\{\begin{array}{l}x^2+5xy+4y^2=10\\x^2-2xy-3y^2=-11\end{array}\right.$
 \item~$\left\{\begin{array}{l}4x^2-xy-y^2=-\frac 1 2\\x^2+2xy-y^2=\frac 1 4\end{array}\right.$
 \item~$\left\{\begin{array}{l}x^2-xy-8y^2=-8\\x^2-2y^2-xy=16\end{array}\right.$
 \end{enumeratea}
\end{multicols}
 \end{esercizio}

\begin{esercizio}
\label{ese:6.42}
Risolvi i seguenti sistemi omogenei.
\begin{multicols}{2}
 \begin{enumeratea}
 \item~$\left\{\begin{array}{l}x^2-6xy-y^2=10\\x^2+xy=-2\end{array}\right.$
 \item~$\left\{\begin{array}{l}4x^2-3xy+y^2=32\\x^2+3y^2-9xy=85\end{array}\right.$
 \item~$\left\{\begin{array}{l}x^2+3xy+2y^2=8\\3x^2-y^2+xy=-4\end{array}\right.$
 \item~$\left\{\begin{array}{l}x^2+5xy-7y^2=-121\\3xy-3x^2-y^2=-7\end{array}\right.$
 \end{enumeratea}
\end{multicols}
\end{esercizio}

\begin{esercizio}[\Ast]
\label{ese:6.43}
Risolvi i seguenti sistemi particolari.
\begin{multicols}{2}
 \begin{enumeratea}
 \item~$\left\{\begin{array}{l}x^2-5xy-3y^2=27\\-2x^2-2y^2+4xy=-50\end{array}\right.$
 \item~$\left\{\begin{array}{l}9x^2+5y^2=-3\\x^2+4xy-3y^2=8\end{array}\right.$
 \item~$\left\{\begin{array}{l}2x^2-4xy-3y^2=18\\xy-2x^2+3y^2=-18\end{array}\right.$
 \item~$\left\{\begin{array}{l}x^2+2xy=-\frac 7 4\\x^2-4xy+4y^2=\frac{81} 4\end{array}\right.$
 \end{enumeratea}
\end{multicols}
\end{esercizio}
\newpage
\begin{esercizio}[\Ast]
\label{ese:6.44}
Risolvi i seguenti sistemi particolari.
\begin{multicols}{2}
 \begin{enumeratea}
 \item~$\left\{\begin{array}{l}x^2+4xy+4y^2-16=0\\x^2-xy+4y^2-6=0\end{array}\right.$
 \item~$\left\{\begin{array}{l}x^2-2xy+y^2-1=0\\x^2-2xy-y^2=1\end{array}\right.$
 \end{enumeratea}
\end{multicols}
\end{esercizio}

\begin{esercizio}
\label{ese:6.45}
Risolvi i seguenti sistemi particolari.
\begin{multicols}{2}
 \begin{enumeratea}
 \item~$\left\{\begin{array}{l}x^2-y^2=0\\2x+y=3\end{array}\right.$
 \item~$\left\{\begin{array}{l}(x-2y)(x+y-2)=0\\3x+6y=3\end{array}\right.$
 \item~$\left\{\begin{array}{l}(x+y-1)(x-y+1)=0\\x-2y=1\end{array}\right.$
 \item~$\left\{\begin{array}{l}(x-3y)(x+5y-2)=0\\(x-2)(x-y+4)=0\end{array}\right.$
 \end{enumeratea}
\end{multicols}
\end{esercizio}

\begin{esercizio}[\Ast]
\label{ese:6.46}
Risolvi i seguenti sistemi particolari.
\begin{multicols}{2}
 \begin{enumeratea}
 \item~$\left\{\begin{array}{l}(x^2-3x+2)(x+y)=0\\x-y=2\end{array}\right.$
 \item~$\left\{\begin{array}{l}(x-y)(x+y+1)(2x-y-1)=0\\(x-3y-3)(x+y-2)=0\end{array}\right.$
 \item~$\left\{\begin{array}{l}(4x^2-9y^2)(x^2-2xy+y^2-9)=0 \\2x-y=2 \end{array}\right.$
 \item~$\left\{\begin{array}{l}x^2+6xy+9y^2-4=0\\(x^2-y^2)(2x-y-4)=0\end{array}\right.$
 \end{enumeratea}
\end{multicols}
\end{esercizio}

\begin{esercizio}[\Ast]
 \label{ese:6.47}
Risolvi i seguenti sistemi particolari.
 \begin{enumeratea}
 \item~$\left\{\begin{array}{l}x^2-2xy-8y^2=0\\(x+y)(x-3)=0\end{array}\right.$
 \item~$\left\{\begin{array}{l}(2x^2-3xy+y^2)(x-y-1)=0 \\(x^2-4xy+3y^2)(12x^2-xy-y^2)=0 \end{array}\right.$
 \item~$\left\{\begin{array}{l}(x-2y-2)(x^2-9y^2)=0 \\(4x^2-4xy+y^2)(y+2)(x-y)=0 \end{array}\right.$
 \item~$\left\{\begin{array}{l}x^4-y^4=0 \\x^2-(y^2-6y+9)=0 \end{array}\right.$
 \end{enumeratea}
\end{esercizio}

\begin{esercizio}[\Ast]
 \label{ese:6.48}
Risolvi i seguenti sistemi particolari.
 \begin{enumeratea}
 \item~$\left\{\begin{array}{l}(y^2-4y+3)(x^2+2x-15)=0 \\(x^2-3xy+2y^2)(9x^2-6xy+y^2)=0 \end{array}\right.$
 \item~$\left\{\begin{array}{l}(x-y)(x+4y-4)(x+y-1)(3x-5y-2)=0 \\(3x+y-3)(x^2-4y^2)=0 \end{array}\right.$
 \end{enumeratea}
\end{esercizio}

\subsection*{6.4 - Problemi che si risolvono con sistemi di grado superiore al primo}
\begin{multicols}{2}
\begin{esercizio}[\Ast]
 \label{ese:6.49}
La differenza tra due numeri è $\frac {11} 4$ e il loro prodotto $\frac {21} 8$ Trova i due numeri.
\end{esercizio}

\begin{esercizio}[\Ast]
 \label{ese:6.50}
Trovare due numeri positivi sapendo che la metà del primo supera di $ 1 $ il secondo e che il quadrato del secondo supera di $ 1 $ la sesta parte del quadrato del primo.
\end{esercizio}

\begin{esercizio}[\Ast]
 \label{ese:6.51}
Data una proporzione tra numeri naturali conosciamo i due medi che sono $ 5 $ e $ 16 $ Sappiamo anche che il rapporto tra il prodotto degli estremi e la loro somma è uguale a $\frac {10} 3 $ Trovare i due estremi.
\end{esercizio}

\begin{esercizio}[\Ast]
 \label{ese:6.52}
La differenza tra un numero di due cifre con quello che si ottiene scambiando le cifre è uguale a $ 36 $ La differenza tra il prodotto delle cifre e la loro somma è uguale a $ 11 $ Trovare il numero.
\end{esercizio}

\begin{esercizio}[\Ast]
 \label{ese:6.53}
Oggi la differenza delle età tra un padre e sua figlia è $ 26 $ anni, mentre due anni fa il prodotto delle loro età era $ 56 $ Determina l'età del padre e della figlia.
\end{esercizio}

\begin{esercizio}[\Ast]
 \label{ese:6.54}
La somma delle età di due fratelli oggi è $ 46 $ anni, mentre fra due anni la somma dei quadrati delle loro età sarà $ 1250 $ Trova l'età dei due fratelli.
\end{esercizio}

\begin{esercizio}[\Ast]
 \label{ese:6.55}
Nella produzione di un oggetto la macchina A impiega 5 minuti in più rispetto alla macchina B. Determinare il numero di oggetti che produce ciascuna macchina in 8 ore se in questo periodo la macchina A ha prodotto 16 oggetti in meno rispetto alla macchina B.
\end{esercizio}

\begin{esercizio}[\Ast]
 \label{ese:6.56}
In un rettangolo la differenza tra i due lati è uguale a $2\unit{cm}$ Se si diminuiscono entrambi i lati di $ 1\unit{cm} $ si ottiene un'area di $0,1224\unit{m^2}$ Calcolare il perimetro del rettangolo.
\end{esercizio}

\begin{esercizio}[\Ast]
 \label{ese:6.57}
Trova due numeri sapendo che la somma tra i loro quadrati è $ 100 $ e il loro rapporto $ \frac 3 4 $
\end{esercizio}

\begin{esercizio}[\Ast]
 \label{ese:6.58}
Ho comprato due tipi di vino. In tutto 30 bottiglie. Per il primo tipo ho speso 54 € e per il secondo 36 €. Il prezzo di una bottiglia del secondo tipo costa 2,5 € in meno di una bottiglia del primo tipo. Trova il numero delle bottiglie di ciascun tipo che ho acquistato e il loro prezzo unitario.
\end{esercizio}

\begin{esercizio}[\Ast]
 \label{ese:6.59}
In un triangolo rettangolo di area $630\unit{m^2}$, l'ipotenusa misura $53\unit{m}$ Determinare il perimetro.
\end{esercizio}

\begin{esercizio}[\Ast]
 \label{ese:6.60}
Un segmento di $35\unit{cm}$ viene diviso in due parti. La somma dei quadrati costruiti su ciascuna delle due parti è $625\unit{{cm}^2}$ Quanto misura ciascuna parte?
\end{esercizio}

\begin{esercizio}[\Ast]
 \label{ese:6.61}
Se in un rettangolo il perimetro misura $ 16,8\unit{m} $ e l'area $ 17,28\unit{m^2} $, quanto misura la sua diagonale?
\end{esercizio}

\begin{esercizio}[\Ast]
 \label{ese:6.62}
In un triangolo rettangolo la somma dei cateti misura $ 10,5\unit{cm} $, mentre l'ipotenusa è $ 7,5\unit{cm} $ Trovare l'area.
\end{esercizio}

\begin{esercizio}[\Ast]
 \label{ese:6.63}
Quanto misura un segmento diviso in due parti, tali che una parte è $ \frac 3 4 $ dell'altra, sapendo che la somma dei quadrati costruiti su ognuna delle due parti è uguale a $121\unit{{cm}^2}$?
\end{esercizio}

\begin{esercizio}[\Ast]
 \label{ese:6.64}
In un trapezio rettangolo con area di $81\unit{m^2}$ la somma della base minore e dell'altezza è $12\unit{m}$ mentre la base minore è $\frac 1 5$ della base maggiore. Trovare il perimetro del rettangolo.
\end{esercizio}

\begin{esercizio}[\Ast]
 \label{ese:6.65}
La differenza tra le diagonali di un rombo è $8\unit{cm}$, mentre la sua area è $24\unit{{cm}^2}$ Determinare il lato del rombo.
\end{esercizio}

\begin{esercizio}[\Ast]
 \label{ese:6.66}
Sappiamo che in un trapezio rettangolo con area di $40\unit{{cm}^2}$ la base minore è $7\unit{cm}$, mentre la somma della base maggiore e dell'altezza è $17\unit{cm}$ Trovare il perimetro del rettangolo.
\end{esercizio}

\begin{esercizio}[\Ast]
 \label{ese:6.67}
Un rettangolo ha l'area uguale a quella di un quadrato. L'altezza del rettangolo è $16\unit{cm}$, mentre la sua base è di $5\unit{cm}$ maggiore del lato del quadrato. Determinare il lato del quadrato.
\end{esercizio}

\begin{esercizio}[\Ast]
 \label{ese:6.68}
La differenza tra i cateti di un triangolo rettangolo è $7k$, mentre la sua area è $60 k^2$ Calcola il perimetro. ($k>0$)
\end{esercizio}

\begin{esercizio}[\Ast]
 \label{ese:6.69}
L'area di un rettangolo che ha come lati le diagonali di due quadrati misura $90 k^2$ La somma dei lati dei due quadrati misura $14k$ Determinare i lati dei due quadrati. ($ k>0 $)
\end{esercizio}

\begin{esercizio}[\Ast]
 \label{ese:6.70}
Nel rettangolo ABCD la differenza tra altezza e base è $4k$ Se prolunghiamo la base AB dalla parte di B di $2k$ fissiamo il punto E e congiungiamo B con E. Trovare il perimetro del trapezio AECD sapendo che la sua area è $28k^2$ con $k>0$
\end{esercizio}

\begin{esercizio}[\Ast]
 \label{ese:6.71}
In un triangolo isoscele la base è $ \frac 2 3 $ dell'altezza e l'area è $12k^2$ Trova il perimetro del triangolo.
\end{esercizio}
\end{multicols}

\subsection{Risposte}
\paragraph{6.1.} a)~$\left(1;1\right)\vee \left(\frac 5 3;\frac 1 3\right)$,\quad b)~$\left(-1;-1\right)\vee \left(\frac 1 2;-\frac 1 4\right)$,\quad c)~$\left(1;1\right)\vee \left(-1;-1\right)$,\quad d)~$\left(0;-\frac 2 5\right)\vee \left(-\frac 1 4;-\frac 1 4\right)$

\paragraph{6.2.} a)~$\left(3;0\right)\vee \left(-6;-9\right)$,\quad b)~$\left(0;0\right)$,\quad c)~ $\left(-\frac{29} 5,\frac{12} 5\right)\vee (3,-2)$,\quad d)~$\left(-\frac{46}{27},-\frac{20}{27}\right)\vee (2,0)$

\paragraph{6.3.} a)~$\left(\frac 1 2;-\frac 1 2\right)$,\quad b)~$\left(3;1\right)\vee \left(5;3\right)$,\quad c)~$\emptyset $,\quad d)~$\left(1;1\right)$

\paragraph{6.4.} a)~$\left(0;1\right)\vee \left(\frac 7 2;-\frac 5 2\right)$,\quad b)~$\emptyset $,\quad c)~$\left(-\frac 3 2;-\frac 5 2\right)\vee \left(\frac 1 2;-\frac 1 2\right)$,\quad d)~$\left(-4;-3\right)\vee \left(\frac{44}{25};\frac{117}{25}\right)$

\paragraph{6.5.} a)~$\left(1;1\right)\vee \left(\frac{10} 7;\frac{11}{14}\right)$,\quad b)~$\left(3;1\right)\vee \left(5;-3\right)$,\quad c)~$ \forall (x,y)\in \insR \times \insR:\,y=-2x+3$,\quad d)~ $\left(1;-3\right)\vee \left(-8;-\frac{15} 2\right)$

\paragraph{6.6.} a)~ $\left(-2;2\right)\vee \left(\frac 5 2;-\frac 5 2\right)$,\quad b)~$\left(4;4\right)\vee \left(-5;-2\right)$,\quad c)~$\left(4;2\right)\vee \left(\frac 4{15};-\frac 2{15}\right)$,\quad d)~$\left(1+\frac{2\sqrt{10}} 5;\frac{\sqrt{10}} 5\right)\vee \left(1-\frac{2\sqrt{10}} 5;-\frac{\sqrt{10}} 5\right)$

\paragraph{6.7.} a)~$\left(\frac{1+\sqrt{13}} 4;\frac{3-\sqrt{13}} 4\right)\vee \left(\frac{1-\sqrt{13}} 4;\frac{3+\sqrt{13}} 4\right)$,\; b)~$\left(\frac{-9+\sqrt{241}} 8;\frac{-25+\sqrt{241}}{16}\right)\vee \left(\frac{9-\sqrt{241}} 8;\frac{-25-\sqrt{241}}{16}\right)$,\protect\\
c)~$\left(\frac{6-\sqrt 2} 4;\frac{-2+3\sqrt 2} 4\right)\vee \left(\frac{6+\sqrt 2} 4;\frac{-2-3\sqrt 2} 4\right)$,\quad d)~$\left(\frac{6-8\sqrt 3}{13};\frac{17+12\sqrt 3}{26}\right)\vee\left(\frac{6+ 8\sqrt 3}{13};\frac{17-12\sqrt 3}{26}\right)$

\paragraph{6.8.} a)~$k\ge \frac 9 2.\, \left(\frac{3-\sqrt{2k-9}} 2; \frac{3+\sqrt{2k-9}} 2\right)\vee \left(\frac{3+\sqrt{2k-9}} 2; \frac{3-\sqrt{2k-9}} 2\right)$,\protect\\
\quad d)~$\forall k\in \insR:\, (2; 0)\vee (2k-1; 2k^2-3k)$

\paragraph{6.9.} a)~$k\ge -\frac 1{12}:\, \left(\frac{-1-\sqrt{12k+1}} 6; \frac{6k-1-\sqrt{12k+1}} 6\right) \vee \left(\frac{-1+\sqrt{12k+1}} 6; \frac{6k-1+\sqrt{12k+1}} 6\right)$,\protect\\
\quad d)~$\forall k\in \insR:\, (3k; -2k)$

\paragraph{6.10.} a)~$x\neq 1:\, \left(2;0\right)\vee \left(-\frac 6 5;-\frac 8 5\right)$,\quad b)~$x\neq y:\, \left(\frac 2 5;\frac 1 5\right)\vee \left(-2;-1\right)$,\quad c)~$x\neq -2y:\, \emptyset $,\protect\\
\quad d)~$x\neq 0\wedge y\neq 1:\, (4;7)$

\paragraph{6.11.} a)~$x\neq 2:\, \left(-1;0\right)\vee \left(\frac{10} 3;\frac{26} 3\right)$,\quad b)~$x\neq -1\wedge y\neq 2:\, \left(-\frac 5 2;4\right)$,\quad c)~$x\neq -y:\, \emptyset $,\protect\\
\quad d)~$y\neq \frac 1 2:\, \left(2;-1\right)\vee \left(9;\frac 5 2\right)$

\paragraph{6.12.} a)~$\left(1;2;-1\right)$,\quad b)~$\emptyset $,\quad c)~$\forall z \in \insR (-1;z-2;z)$,\quad d)~$\left(-\frac{47} 3;-\frac{46} 3;\frac 5 3\right)$

\paragraph{6.13.} a)~$\emptyset $,\quad b)~$\left(5;0;-2\right)\vee \left(-2;0;5\right)$,\quad c)~$\left(-4;\frac{25} 2;\frac{17} 2\right)\vee \left(3;-\frac 3 2;\frac 3 2\right)$,\quad d)~$\left(-1;-2;3\right)\vee \left(\frac 1 2;-\frac 1 2;0\right)$,\quad e)~$\left(0;\frac{3\sqrt 7+1} 2;-\frac{3\sqrt 7-1} 2\right)\vee \left(0;-\frac{3\sqrt 7-1} 2;\frac{3\sqrt 7+1} 2\right)$

\paragraph{6.14.} a)~$(3;1)\vee(1;3)$,\quad b)~$\emptyset $,\quad c)~$(3;2)\vee(2;3)$,\quad d)~$(1;-6)\vee(-6;1)$

\paragraph{6.15.} a)~$(2;1)\vee(1;2)$,\quad b)~$(4;-1)\vee(-1;4)$,\quad c)~$(-2;-2)$,\quad d)~$(3;3)$

\paragraph{6.16.} a)~$\emptyset $,\quad b)~$(4;3)\vee(3;4)$,\quad c)~$\emptyset $,\quad d)~$(13;-1)\vee(-1;13)$,\quad e)~$\left(\frac 3 5;\frac 3 5\right)$,\quad f)~$\emptyset $

\paragraph{6.17.} a)~$(2;-7)\vee(-7;2)$,\quad b)~$(7;-2)\vee(-2;7)$,\quad c)~$\left(\frac 3 4;-\frac 1 2\right)\vee\left(-\frac 1 2;\frac 3 4\right)$,\protect\\
\quad d)~$\left(1+\sqrt{11};1-\sqrt{11}\right)\vee\left(1-\sqrt{11};1+\sqrt{11}\right)$

\paragraph{6.18.} a)~$(0;4)\vee(4;0)$,\; b)~$\left(\frac 7 2;-1\right)\vee\left(-1;\frac 7 2\right)$,\; c)~$\left(\frac{-5+\sqrt{17}} 2;\frac{-5-\sqrt{17}} 2\right)\vee \left(\frac{-5-\sqrt{17}} 2\\y=\frac{-5+\sqrt{17}} 2\right)$,\quad d)~$\left(\frac{4+\sqrt{34}} 6;\frac{4-\sqrt{34}} 6\right)\vee \left(\frac{4-\sqrt{34}} 6\\y=\frac{4+\sqrt{34}} 6\right)$

\paragraph{6.19.} a)~$\left(\frac{5+\sqrt{97}} 4\\y=\frac{5-\sqrt{97}} 4\right)\vee \left(\frac{5-\sqrt{97}} 4\\y=\frac{5+\sqrt{97}} 4\right)$,\quad b)~$\left(\frac{3+{2\sqrt 3}} 3;\frac {3-{2\sqrt 3}} 3\right)\vee \left(\frac {3-{2\sqrt 3}} 3;\frac{3+{2\sqrt 3}} 3\right)$,\quad c)~$\left(\frac{1+\sqrt{13}} 2;\frac{1-\sqrt{13}} 2\right)\vee \left(\frac{1-\sqrt{13}} 2;\frac{1+\sqrt{13}} 2\right)$,\quad d)~$\left(2+3\sqrt 6;2-3\sqrt 6\right)\vee \left(2-3\sqrt 6;2+3\sqrt 6\right)$

\paragraph{6.20.} a)~$(1,0)\vee(0,1)$,\quad b)~$(1,1)$,\quad c)~$(1,2)\vee(2,1)$,\quad d)~$\emptyset$,\quad e)~$(2,2)$,\quad f)~$(0,2)\vee(2,0)$

\paragraph{6.21.} a)~$(-6,-6)$,\quad b)~$(-5,4)\vee(4,-5)$,\quad c)~$(3,8)\vee(8,3)$,\quad d)~$(-1,4)\vee(4,-1)$

\paragraph{6.22.} a)~$(2,5)\vee(5,2)$,\quad b)~$(-3,2)\vee(2,-3)$,\quad c)~$(\frac 1 2;1)\vee(1;\frac 1 2)$,\quad d)~$(-4;1)\vee(1;-4)$

\paragraph{6.23.} a)~$(-8;2)\vee(2;-8)$,\quad b)~$(-6;1)\vee(1;-6)$,\quad c)~$\left(-\frac 1 2;-\frac{13} 2\right)\vee \left(-\frac{13} 2;-\frac 1 2\right)$,\quad d)~$\emptyset $

\paragraph{6.24.} a)~$\emptyset $,\; b)~$(3;3)$,\; c)~$\left(\frac{1+\sqrt 5} 2;\frac{1-\sqrt 5} 2\right)\vee \left(\frac{1-\sqrt 5} 2;\frac{1+\sqrt 5} 2\right)$,\; d)~$\left(\frac{3+\sqrt 7} 2;\frac{3-\sqrt 7} 2\right)\vee \left(\frac{3-\sqrt 7} 2;\frac{3+\sqrt 7} 2\right)$

\paragraph{6.25.} a)~$ (-1; -2)\vee(2;1) $,\, b)~$\left(-\frac 1 7;-\frac 1 5\right)\vee\left(-\frac 1 5;-\frac 1 7\right)$,\, c)~$\left(\frac{-3-\sqrt{17}} 4;{\frac{3-\sqrt{17}} 2}\right)\vee \left(\frac{-3+\sqrt{17}} 4;{\frac{3+\sqrt{17}} 2}\right)$

\paragraph{6.26.} a)~$(-1;0)\vee(0;-1)$,\quad b)~$\left(\frac{-3-\sqrt{17}} 2;\frac{-3+\sqrt{17}} 2\right)\vee\left(\frac{-3+\sqrt{17}} 2;\frac{-3-\sqrt{17}} 2\right)$,\protect\\
\quad c)~$\left(\frac{-5-\sqrt{10}} 5;\frac{-5+\sqrt{10}} 5\right)\vee \left(\frac{-5+\sqrt{10}} 5;\frac{-5-\sqrt{10}} 5\right)$,\quad d)~$(3;5)\vee(5;3)$

\paragraph{6.27.} a)~$(1;2)\vee(2;1)$,\quad b)~$(-7;1)\vee(1;-7)$,\quad c)~$(2;-7)\vee(7;-2)$,\quad d)~$(2;3)\vee(3;2)$

\paragraph{6.28.} a)~$(-1;1)\vee(1;-1)$,\;b)~$(-2;-1)\vee(-1;-2)$,\; c)~$(-3;-2)\vee(-2;-3)$,\; d)~$(-3;1)\vee(1;-3)$

\paragraph{6.29.} a)~$(1;2)\vee(2;1)$,\quad b)~$\left(-\frac 3 2;\frac 1 2\right)\vee\left(\frac 1 2;-\frac 3 2\right)$,\quad c)~$\emptyset$,\quad d)~$(1;4)\vee(4;1)$

\paragraph{6.30.} a)~$(1;1)$,\quad b)~$(1;-2)\vee(-2;1)\vee(-1;2)\vee(2;-1)$,\quad c)~$(-3;2)\vee(2;-3)$,\quad d)~$(2;2)$

\paragraph{6.31.} a)~$(-5;3)\vee(3;-5)$,\quad b)~$\emptyset$,\quad c)~$(-1;2)\vee(2;-1)$,\quad d)~$(-4;-3)\vee(-3;-4)\vee(3;4)\vee(4;3)$

\paragraph{6.32.} a)~$\left(-\frac 1 2;4\right)\vee\left(4;-\frac 1 2\right)$,\quad b)~$(-2;-1)\vee(-1;-2)\vee(1;2)\vee(2;1)$,\quad c)~$(-5;-3)\vee(-3;-5)\vee(3;5)\vee(5;3)$,\quad d)~$(-1;-1)\vee(1;1)$

\paragraph{6.33.} a)~$(-4;-3)\vee(-3;-4)\vee(3;4)\vee(4;3)$,\quad b)~$(-1;-1)\vee(1;1)$,\quad c)~$\emptyset$,\quad d)~$(-3;-3)\vee(3;3)$

\paragraph{6.34.} a)~$\emptyset$,\quad b)~$(1;1)$,\quad c)~$(-3-\sqrt 7;-3+\sqrt 7)\vee(-3+\sqrt 7;-3-\sqrt 7)\vee(1;2)\vee(2;1)$,\protect\\
\quad d)~$\left(\frac{\sqrt{14}+\sqrt 2} 2;\frac{\sqrt 2-\sqrt{14}} 2\right)\vee\left(\frac{\sqrt{14}-\sqrt 2} 2;\frac{\sqrt 2+\sqrt{14}} 2\right)\vee\left(\frac{\sqrt{14}-\sqrt 2} 2;-\frac{\sqrt 2+\sqrt{14}} 2\right)\vee\left(\frac{\sqrt{14}+\sqrt 2} 2;-\frac{\sqrt 2-\sqrt{14}} 2\right)$

\paragraph{6.35.} a)~$(-2;1)\vee(1;-2)\vee(-\sqrt {2};\sqrt{2})\vee(\sqrt {2};-\sqrt{2})$,\quad b)~$\left(-\frac 2 3;\frac 1 3\right)\vee\left(\frac 1 3;-\frac 2 3\right)$,\protect\\
\quad c)~$\left(\frac{-1+\sqrt{33}} 4;\frac{-1-\sqrt{33}} 4\right)\vee \left(\frac{-1-\sqrt{33}} 4;\frac{-1+\sqrt{33}} 4\right)$,\quad d)~$\emptyset$,\quad e)~$(-1;2)\vee(2;-1)$

\paragraph{6.36.} a)~$(0;0)$,\quad b)~$(t;t)$,\quad c)~$(0;0)$,\quad d)~$(t;-t)$

\paragraph{6.37.} a)~$(2t;t)$,\quad b)~$(2t;t)$,\quad c)~$(-2t;t)$,\quad d)~$(0;0)$

\paragraph{6.38.} a)~$(-4t;t)\vee(-2t;t)$,\quad b)~$(k;4k)\vee(k;-\frac{1 2}k)$,\quad c)~$(-t;t)$,\quad d)~$(-4;1)\vee(4;-1)$

\paragraph{6.39.} a)~$\left(-\frac 3 2;-\frac 1 2\right)\vee\left(\frac 3 2;\frac 1 2\right)\vee\left(-\frac 5 4;-\frac 1 4\right)\vee\left(\frac 5 4;\frac 1 4\right)$,\quad b)~$(1;2)\vee(-1;-2)\vee (-1;2)\vee (1;-2)$,\protect\\
\quad c)~$(-1;1)\vee (1;-1)\left(-\frac{2\sqrt 3} 3;\frac{\sqrt 3} 3\right)\vee\left(\frac{2\sqrt 3} 3;-\frac{\sqrt 3} 3\right)$,\quad d)~$\emptyset$

\paragraph{6.40.} a)~$(1;1)\vee(-1;-1)\vee\left(\sqrt 6;-4\sqrt 6\right)\vee\left(-\sqrt 6;4\sqrt 6\right)$,\quad b)~$(1;-1)\vee(-1;1)$,\protect\\
\quad c)~$(\sqrt 3;0)\vee (-\sqrt 3;0)\vee (0;\sqrt 3)\vee (0;-\sqrt 3)$,\quad d)~$(1;0)\vee (-1;0)\vee\left(\frac{\sqrt 3} 3;\frac{\sqrt 3} 3\right)\vee\left(-\frac{\sqrt 3} 3;-\frac{\sqrt 3} 3\right)$

\paragraph{6.41.} a)~$(1;-2)\vee(-1;2)\vee (-2;1)\vee (2;-1)$,\quad b)~$(2;-3)\vee(-2;3)$,\protect\\
\quad c)~$\left(\frac 1 2;1\right)\vee \left(-\frac 1 2;-1\right)$,\quad d)~$(4;-2)\vee(-4;2)\vee (6;2)\vee (-6;-2)$

\paragraph{6.42.} a)~$(-1;3)\vee(1;-3)$,\quad b)~$(1;-4)\vee(-1;4)\vee (-1;-7)\vee (1;7)$,\protect\\
\quad c)~$(0;2)\vee (0;-2)\vee \left(\frac{10} 3;-\frac{14} 3\right)\vee \left(-\frac{10} 3;\frac{14} 3\right)$,\quad d)~$(2;5)\vee (-2;-5)\vee \left(-\frac{18} 7;-\frac{37} 7\right)\vee \left(\frac{18} 7;\frac{37} 7\right)$

\paragraph{6.43.} a)~$(3;-2)\vee(-3;2)\vee\left(\frac{34} 7;-\frac 1 7\right)\vee \left(-\frac{34} 7;\frac 1 7\right)$,\quad b)~$\emptyset$,\quad c)~$(-3;0)\vee(3;0)$,\protect\\
\quad d)~$\left(\frac 1 2;-2\right)\vee\left(-\frac 1 2;2\right)\vee\left(\frac 7 4;-\frac{11} 8\right)\vee\left(-\frac 7 4;\frac{11} 8\right)$

\paragraph{6.44.} a)~$(-2;-1)\vee(2;1)$,\quad b)~$(-1;0)\vee(1;0)$

\paragraph{6.45.} a)~$(1;1)\vee(3;-3)$,\quad b)~$(3;-1)\vee\left(\frac 1 2;\frac 1 4\right)$,\quad c)~$(1;0)\vee(-3;-2)$,\protect\\
\quad d)~$(2;0)\vee\left(2;\frac 2 3\right)\vee(-6;-2)\vee(-3;1)$

\paragraph{6.46.} a)~$(1;-1)_\text{doppia} \vee(2;0)$,\quad b)~$(0;-1)_\text{doppia}\vee\left(-\frac 3 2;-\frac 3 2\right)\vee(1;1)_\text{doppia}$,\quad c)~$(5;8)\vee\left(\frac 3 2;1\right)\vee(-1;-4)\vee\left(\frac 3 4;-\frac 1 2\right)$,\quad d)~$(1;-1)\vee(2;0)\vee(-1;1)\vee\left(\frac 1 2;\frac 1 2\right)\vee\left(-\frac 1 2;-\frac 1 2\right)\vee\left(\frac{10} 7;-\frac 8 7\right)$

\paragraph{6.47.}a)~$(0;0)_\text{doppia}\vee\left(3;-\frac 3 2\right)\vee(3;\frac 3 4)$,\quad b)~$(t;t)\vee\left(\frac 3 2;\frac 1 2\right)\vee\left(\frac 1 5;-\frac 4 5\right)\vee\left(-\frac 1 2;-\frac 3 2\right)$,\protect\\
\quad c)~$(0;0)_\text{tripla}\vee(-2;-2)_\text{doppia}\vee\left(-\frac 2 3,-\frac 4 3\right)_\text{doppia}\vee(6;-2)\vee(-6;-2)$,\quad d)~$\left(\frac 3 2;\frac 3 2\right)\vee\left(-\frac 3 2;\frac 3 2\right)$

\paragraph{6.48.} a)~$(1;1)\vee(2;1)\vee(3;3)_\text{doppia}\vee(6;3)\vee\left(\frac 1 3;1\right)\vee(1;3)\vee(-5;-5)\vee\left(-5;-\frac 5 2\right)\vee\left(3;\frac 3 2\right)\vee(-5;-15)\vee(3;9)$,\quad b)~$(0;0)_\text{doppia}\vee\left(\frac 3 4;\frac 3 4\right)\vee(1;0)\vee\left(\frac 8{11,}\frac 9{11}\right)\vee\left(\frac{17}{18};\frac 1 6\right)\vee\left(\frac 4 3;\frac 2 3\right)\vee\protect\\
(-4;2)(2;-1)\vee\left(\frac 2 3;\frac 1 3\right)\vee\left(\frac 4{11};-\frac 2{11}\right)\vee(4;2)$
\begin{multicols}{2}

\paragraph{6.49.} $\left(-\frac 3 4;-\frac 7 2\right)\vee \left(\frac 7 2;-\frac 3 4\right)$

\paragraph{6.50.} $(12;5)$

\paragraph{6.51.} $(4;20)\vee (20,4)$

\paragraph{6.52.} $73$

\paragraph{6.53.} $(30;4)$

\paragraph{6.54.} $(23;23)$

\paragraph{6.55.} $(32;48)$

\paragraph{6.56.} $2p=144\unit{cm}$

\paragraph{6.57.} $(-6;-8)\vee (6,8)$

\paragraph{6.58.} $(12;18)$

\paragraph{6.59.} $2p=126\unit{m}$

\paragraph{6.60.} $[15\unit{cm}$ e $20\unit{cm}]$

\paragraph{6.61.} $\text{Diagonale }=6\unit{m}$

\paragraph{6.62.} $\Area=13,5\unit{{cm}^2}$

\paragraph{6.63.} $15,4\unit{cm}$

\paragraph{6.64.} $2p_1=42\vee 2p_2=57+3\sqrt{145}$

\paragraph{6.65.} $2\sqrt{10}\unit{cm}$

\paragraph{6.66.} $2p=24+2\sqrt{13}$

\paragraph{6.67.} $20\unit{cm}$

\paragraph{6.68.} $2p=40k$

\paragraph{6.69.} $l_1=5k\vee l_2=9k$

\paragraph{6.70.} $2p=15+k\sqrt{53}$

\paragraph{6.71.} $2p=4k(1+\sqrt{10})$
\end{multicols}
