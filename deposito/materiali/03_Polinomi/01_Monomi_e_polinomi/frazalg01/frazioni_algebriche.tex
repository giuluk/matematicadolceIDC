% (c) 2012-2013 Claudio Carboncini - claudio.carboncini@gmail.com
% (c) 2012-2013 Dimitrios Vrettos - d.vrettos@gmail.com
% (c) 2015 Daniele Zambelli daniele.zambelli@gmail.com

\chapter{Frazioni algebriche}

% \begin{wrapfloat}{figure}{r}{0pt}
% \includegraphics[scale=0.35]{img/fig000_.png}
% \caption{...}
% \label{fig:...}
% \end{wrapfloat}
% 
% \begin{center} \input{\folder lbr/fig000_.pgf} \end{center}

I polinomi, rispetto alle operazioni si comportano come i numeri interi, 
in particolare la divisione tra due polinomi spesso dà un resto. Così, proprio 
come abbiamo fatto con i numeri interi, anche con i polinomi dovremo utilizzare 
le frazioni per poter eseguire sempre le divisioni. Per poter operare con le 
frazioni dovremo imparar a calcolare il minimo comune multiplo tra polinomi.

\section{Divisore comune e multiplo comune}
\label{sec:frazalg_MCDemcm}

Per determinare~$\mcd$ (\emph{massimo comune divisore}) 
e~$\mcm$ (\emph{minimo comune multiplo}) di due o più polinomi occorre prima 
di tutto scomporli in fattori irriducibili.

\osservazione La cosa non è semplice poiché non si può essere sicuri di aver 
trovato il massimo comune divisore o il minimo comune multiplo
per la difficoltà di decidere se un polinomio è irriducibile: 
prudentemente si dovrebbe parlare di divisore comune e di multiplo comune.

Un polinomio~$A$ si dice multiplo di un polinomio~$B$ se esiste un 
polinomio~$C$ per il quale~$A=B\cdot C$ in questo caso diremo
anche che~$B$ è divisore del polinomio~$A$.

\subsection{Massimo Comun Divisore}
Dopo aver scomposto ciascun polinomio in fattori irriducibili, il massimo 
comune divisore tra due o più polinomi è il prodotto di tutti i fattori comuni 
ai polinomi, presi ciascuno una sola volta, con il minimo esponente.
Sia i coefficienti numerici, sia i monomi possono essere considerati polinomi.

\begin{procedura}
Calcolare il~$\mcd$ tra polinomi:
\begin{enumeratea}
\item scomponiamo in fattori irriducibili ogni polinomio;
\item prendiamo i fattori comuni a tutti i polinomi una sola volta con 
l'esponente più piccolo;
\item se non ci sono fattori comuni a tutti i polinomi il~$\mcd$ è~$1$.
\end{enumeratea}
\end{procedura}

% \begin{exrig}
 \begin{esempio}
Determinare il~$\mcd\left(3a^{2}b^{3}-3b^{3};\,6a^{3}b^{2}-6b^{2}; \quad
                          2a^{2}b^{2}-24ab^{2}+12b^{2}\right)$.
 \begin{itemize*}
 \item Scomponiamo in fattori i singoli polinomi;
  \begin{itemize*}
  \item $3a^{2}b^{3}-3b^{3}=3b^{3}\left(a^{2}-1\right)=3b^{3}(a-1)(a+1)$
  \item $6a^{3}b^{2}-6b^{2}=6b^{2}\left(a^{3}-1\right)=
         6b^{2}(a-1)\left(a^{2}+a+1\right)$
  \item $12a^{2}b^{2}-24ab^{2}+12b^{2}=12b^{2}\left(a^{2}-2a+1\right)=
         12b^{2}(a-1)^{2}$.
  \end{itemize*}
 \item i fattori comuni a tutti i polinomi presi con l'esponente più piccolo 
  sono:
  \begin{itemize*}
  \item tra i numeri il~$3$
  \item tra i monomi~$b^{2}$
  \item tra i polinomi~$a-1$.
  \end{itemize*}
 \item quindi il~$\mcd=3b^{2}(a-1)$.
 \end{itemize*}
 \end{esempio}
% \end{exrig}

\subsection{Minimo comune multiplo}
Dopo aver scomposto ciascun polinomio in fattori, il minimo comune multiplo 
tra due o più polinomi è il prodotto dei fattori comuni e non comuni di tutti 
i polinomi, quelli comuni presi una sola volta, con il massimo esponente.

\begin{procedura}
Calcolare il~$\mcm$ tra polinomi:
\begin{enumeratea}
\item scomponiamo in fattori irriducibili ogni polinomio;
\item prendiamo tutti i fattori comuni e non comuni dei polinomi, i fattori 
 comuni presi una sola volta con il massimo esponente.
\end{enumeratea}
\end{procedura}

% \begin{exrig}
 \begin{esempio}
Determinare il~$\mcm\left(3a^{2}b^{3}-3b^{3};\,6a^{3}b^{2}-6b^{2};\quad
                          2a^{2}b^{2}-24ab^{2}+12b^{2}\right)$.
 \begin{itemize*}
 \item Scomponiamo in fattori i singoli polinomi;
  \begin{itemize*}
  \item $3a^{2}b^{3}-3b^{3}=3b^{3}\left(a^{2}-1\right)=3b^{3}(a-1)(a+1)$
  \item $6a^{3}b^{2}-6b^{2}=6b^{2}\left(a^{3}-1\right)=
         6b^{2}(a-1)\left(a^{2}+a+1\right)$
  \item $12a^{2}b^{2}-24ab^{2}+12b^{2}=12b^{2}\left(a^{2}-2a+1\right)=
         12b^{2}(a-1)^{2}$.
  \end{itemize*}
 \item i fattori comuni e non comuni presi con il massimo esponente sono:
  \begin{itemize*}
  \item tra i coefficienti numerici il~$12$
  \item tra i monomi~$b^{3}$
  \item tra i polinomi~$(a-1)^{2}\cdot (a+1)\cdot \left(a^{2}+a+1\right)$.
  \end{itemize*}
 \item quindi il~$\mcm=12b^{3}(a-1)^{2}(a+1)\left(a^{2}+a+1\right)$.
 \end{itemize*}
 \end{esempio}
% \end{exrig}

% \ovalbox{\risolvii \ref{ese:18.1}, \ref{ese:18.2}, \ref{ese:18.3}, 
% \ref{ese:18.4}, \ref{ese:18.5}, \ref{ese:18.6}, \ref{ese:18.7}, 
% \ref{ese:18.8}, \ref{ese:18.9}, \ref{ese:18.10}, \ref{ese:18.11}}

\section{Definizione di frazione algebrica}
\label{sec:frazalg_definizione}

% \begin{wrapfloat}{figure}{r}{0pt}
% \includegraphics[scale=0.35]{img/fig000_.png}
% \caption{...}
% \label{fig:...}
% \end{wrapfloat}
% 
% \begin{center} \input{\folder lbr/fig000_.pgf} \end{center}

Diamo la seguente definizione:
\begin{definizione}
Si definisce \emph{frazione algebrica} una espressione del tipo~$\frac{A}{B}$ 
dove~$A$ e~$B$ sono polinomi.
\end{definizione}

% ???????????????????????????
% Osserviamo che un'espressione di questo tipo si ottiene talvolta quando ci si 
% propone di ottenere il quoziente di due monomi.

Di seguito vediamo come calcolare il quoziente tra espressioni letterali.

% \begin{exrig}
 \begin{esempio}
Determinare il quoziente tra~$m_{1}=5a^{3}b^{2}c^{5}$ e~$m_{2}=-3a^{2}bc^{5}$.

Questa operazione si esegue applicando, sulla parte letterale, le proprietà 
delle potenze e sul coefficiente la divisione tra numeri
razionali:~$q=5a^{3}b^{2}c^{5}:\left(-3a^{2}bc^{5}\right)=
              \frac{5a^{3}b^{2}c^{5}}{-3a^{2}bc^{5}}=
              -{\frac{5}{3}}ab$.
Il quoziente è quindi un monomio.
 \end{esempio}

 \begin{esempio}
Determinare il quoziente tra~$m_{1}=5a^{3}b^{2}c^{5}$ e~$m_{2}=-3a^{7}bc^{5}$.

In questo caso l'esponente della~$a$ nel dividendo è minore dell'esponente 
della stessa variabile nel divisore quindi
si ottiene~$q_{1}=5a^{3}b^{2}c^{5}:\left(-3a^{7}bc^{5}\right)=
                  \frac{5a^{3}b^{2}c^{5}}{-3a^{7}bc^{5}}= 
                  -{\frac{5b}{3a^{4}}}
                  -{\frac{5}{3}}a^{-4}b$. 

% Questo non è un monomio per la presenza dell'esponente
% negativo alla variabile~$a$. %Sappiamo che~$a^{-4}=\frac{1}{a^{4}}$ e
% Quindi:~$q_{1}=5a^{3}b^{2}c^{5}:\left(-3a^{7}bc^{5}\right)=
%                {\frac{5b}{3a^{4}}}$. Il quoziente è una frazione algebrica.
 \end{esempio}
% \end{exrig}

Quando vogliamo determinare il quoziente di una divisione tra un monomio e un 
polinomio e tra polinomi si presentano diversi casi.

\paragraph{Caso I}Monomio diviso un polinomio.
\begin{itemize*}
 \item Determinare il quoziente tra:~$D=2a^{3}b$ e~$d=a^{2}+b$.
\end{itemize*}
Il dividendo è un monomio e il divisore un polinomio.
Questa operazione non ha come risultato un polinomio ma una
frazione. $q=2a^{3}b:\left(a^{2}+b\right)=\frac{2a^{3}b}{a^{2}+b}$.

\paragraph{Caso II}Un polinomio diviso un monomio.
\begin{itemize*}
 \item Determinare il quoziente tra:~$D=2a^{3}b+a^{5}b^{3}-3ab^{2}$ 
  e~$d=\frac{1}{2}ab$.
\end{itemize*}
$q=\left(2a^{3}b+a^{5}b^{3}-3ab^{2}\right):\left(\frac{1}{2}ab\right)=
   4a^{2}+2a^{4}b^{2}-6b$.
Il quoziente è un polinomio.
\begin{itemize*}
 \item Determinare il quoziente tra:~$D=2a^{3}b+a^{5}b^{3}-3ab^{2}$ 
  e~$d=\frac{1}{2}a^{5}b$.
\end{itemize*}
Dividiamo ciascun termine del polinomio per il monomio assegnato: il quoziente 
sarà
$q=\left(2a^{3}b+a^{5}b^{3}-3ab^{2}\right):\left(\frac{1}{2}a^{5}b\right)=
   \frac{4}{a^{2}}+2b^{2}-\frac{6b}{a^{4}}$.
Il quoziente è una somma di frazioni algebriche.

\paragraph{Caso III}Un polinomio diviso un altro polinomio.
\begin{itemize*}
 \item Determinare il quoziente tra:~$D=x-3$ e~$d=x^{2}+1$.
\end{itemize*}
La divisione tra polinomi in una sola variabile è possibile, quando il grado 
del dividendo è maggiore o uguale al grado del divisore;
questa condizione non si verifica nel caso proposto.
Il quoziente è la frazione algebrica~$q=\frac{x-3}{x^{2}+1}$.

\paragraph{Conclusione}Una frazione algebrica può essere considerata come il 
quoziente indicato tra due polinomi.
Ogni frazione algebrica è dunque un'espressione letterale fratta.

\section{Condizioni di esistenza per una frazione algebrica}
\label{sec:frazalg_condizioniesistenza}

Per ``discussione di una frazione algebrica'' intendiamo la ricerca dei valori 
che, attribuiti alle variabili, non la rendano priva di significato. 
Poiché non è possibile dividere per~$0$, una frazione algebrica perde di 
significato per quei valori che attribuiti alle variabili rendono il 
denominatore uguale a zero. 
Quando abbiamo una frazione algebrica tipo~$\frac{A}{B}$ poniamo sempre la 
condizione di esistenza (abbreviato con~$\CE$):~$B\neq~0$.

% \begin{exrig}
 \begin{esempio}
Determinare le condizioni di esistenza di~$\frac{1+x}{x}$.

Questa frazione perde di significato quando il denominatore si 
annulla:~$\CE x\neq~0$.
 \end{esempio}

 \begin{esempio}
Determinare le condizioni di esistenza di~$\frac{x}{x+3}$.

Questa frazione perde di significato quando il denominatore si 
annulla:~$\CE x\neq -3$.
 \end{esempio}

 \begin{esempio}
Determinare le condizioni di esistenza di~$\frac{3a+5b-7}{ab}$.

$\CE ab\neq~0$. Sappiamo che un prodotto è nullo quando almeno uno dei suoi 
fattori è nullo, dunque affinché il denominatore non si annulli non si deve 
annullare né $a$ né $b$,
quindi~$a\neq~0$ e~$b\neq~0$. Concludendo~$\CE a\neq~0\wedge b\neq~0$.
 \end{esempio}

 \begin{esempio}
Determinare le condizioni di esistenza di~$\frac{-6}{2x+5}$.

$\CE~2x+5\neq~0$, per risolvere questa disuguaglianza si procede come per le 
usuali 
equazioni:~$2x+5\neq~0 \Rightarrow~2x\neq -5\Rightarrow x\neq -{\frac{5}{2}}$ 
si può concludere~$\CE x\neq -{\frac{5}{2}}$.
 \end{esempio}

 \begin{esempio}
Determinare le condizioni di esistenza di~$\frac{-x^{3}-8x}{x^{2}+2}$.

$\CE x^{2}+2\neq~0$, il binomio è sempre maggiore di~$0$ perché somma di due 
grandezze positive.
Pertanto la condizione~$x^{2}+2\neq~0$ è sempre verificata e la frazione 
esiste sempre. Scriveremo~$\CE \forall x\in \insR$.
 \end{esempio}

 \begin{esempio}
Determinare le condizioni di esistenza di~$\frac{2x}{x^{2}-4}$.

$\CE x^{2}-4\neq~0$ per rendere nullo il denominatore si dovrebbe 
avere~$x^2 = 4$ e questo si verifica se~$x = +2$
oppure se~$x = -2$ possiamo anche osservare che il denominatore è una 
differenza di quadrati e che quindi la
condizione di esistenza si può scrivere come~$\CE (x-2)(x+2)\neq~0$, essendo 
un prodotto possiamo
scrivere~$\CE x-2\neq~0\wedge x+2\neq~0$ e 
concludere:~$\CE x\neq~2\wedge x\neq -2$.
 \end{esempio}
% \end{exrig}

\begin{procedura}
Determinare la condizione di esistenza di una frazione algebrica:
\begin{enumeratea}
% \item porre il denominatore della frazione diverso da zero;
\item scomporre in fattori il denominatore;
\item porre ciascun fattore del denominatore diverso da zero;
\item escludere i valori che annullano il denominatore.
\end{enumeratea}
\end{procedura}

% \ovalbox{\risolvii \ref{ese:19.1}, \ref{ese:19.2}, \ref{ese:19.3}, 
% \ref{ese:19.4}}

\section{Semplificazione di una frazione algebrica}
\label{sec:frazalg_semplificazione}

Semplificare una frazione algebrica significa dividere numeratore e 
denominatore per uno stesso fattore diverso da zero, in questo modo infatti la 
proprietà invariantiva della divisione garantisce che la frazione non cambia 
di valore.
Quando semplifichiamo una frazione numerica dividiamo il numeratore e il 
denominatore per il loro~$\mcd$
che è sempre un numero diverso da zero, ottenendo una frazione ridotta ai 
minimi termini equivalente a quella assegnata.
Quando ci poniamo lo stesso problema su una frazione algebrica, dobbiamo porre 
attenzione a escludere quei valori che attribuiti alle variabili rendono nullo 
il~$\mcd$.

% \begin{exrig}
 \begin{esempio}
Semplificare~$\dfrac{16x^{3}y^{2}z}{10xy^{2}}$.

$\CE xy^{2}\neq~0 \rightarrow x\neq~0 \wedge y\neq~0$.
Puoi semplificare la parte numerica.
Per semplificare la parte letterale applica la proprietà delle potenze 
relativa al quoziente di potenze con la stessa
base:~$x^{3}:x=x^{3-1}=x^{2}$ e~$y^{2}:y^{2}=1$. Quindi:
\[\frac{16x^{3}y^{2}z}{10xy^{2}}=\frac{8x^{2}z}{5}=\frac{8}{5}x^{2}z.\]
 \end{esempio}

 \begin{esempio}
Ridurre ai minimi termini la frazione:~$\dfrac{a^{2}-6a+9}{a^{4}-81}$.
\begin{itemize*}
 \item Scomponiamo in fattori
  \begin{itemize*}
  \item il numeratore:~$a^2 - 6a +9 = (a - 3 )^2$
  \item il denominatore:~$a^4 - 81 = \left(a^2 - 9\right)\left(a^2 + 9\right)= 
                                     (a - 3)(a + 3)\left(a^2 + 9\right)$
  \end{itemize*}
 \item riscriviamo la 
 frazione~$\frac{\left(a-3\right)^{2}}
                {(a-3)\cdot(a+3)\cdot \left(a^{2}+9\right)}$
 \item $\CE (a-3)\cdot(a+3)\cdot\left(a^{2}+9\right)\neq~0$ da 
  cui~$\CE a \neq +3$ e~$a \neq -3$, il terzo fattore non si annulla mai 
  perché somma di un numero positivo e un quadrato;
 \item 
  semplifichiamo:~$\frac{(a-3)^{\cancel{2}}}
                        {\cancel{(a-3)}\cdot (a+3)\cdot \left(a^{2}+9\right)}=
                   \frac{a-3}{\left(a+3)(a^{2}+9\right)}$.
\end{itemize*}
 \end{esempio}

 \begin{esempio}
Ridurre ai minimi termini la frazione in due 
variabili:~$\dfrac{x^{4}+x^{2}y^{2}-x^{3}y-xy^{3}}
                  {x^{4}-x^{2}y^{2}+x^{3}y-xy^{3}}$.
\begin{itemize*}
 \item Scomponiamo in fattori
  \begin{itemize*}
  \item $x^{4}+x^{2}y^{2}-x^{3}y-xy^{3}=
   x^{2} \left(x^{2}+y^{2}\right)-xy \left(x^{2}+y^{2}\right)=
   x \left(x^{2}+y^{2}\right) (x-y)$
  \item $x^{4}-x^{2}y^{2}+x^{3}y-xy^{3}=
   x^{2} \left(x^{2}-y^{2}\right)+xy \left(x^{2}-y^{2}\right)=
   x (x+y)^{2} (x-y)$
  \end{itemize*}
 \item la frazione diventa:~$\frac{x^{4}+x^{2}y^{2}-x^{3}y-xy^{3}}
                                  {x^{4}-x^{2}y^{2}+x^{3}y-xy^{3}}=
                             \frac{x \left(x^{2}+y^{2}\right) (x-y)}
                                  {x(x+y)^{2} (x-y)}$
 \item $\CE x\cdot (x+y)^{2}\cdot (x^{2}+y^{2})\neq~0$ 
  cioè~$\CE x\neq~0\wedge x\neq -y$
 \item semplifichiamo i fattori 
  uguali:~$\frac{\cancel{x} \left(x^{2}+y^{2}\right) \cancel{(x-y)}}
                {\cancel{x}(x+y)^{2} \cancel{(x-y)}}=\frac{x^2+y^2}{(x+y)^2}$.
\end{itemize*}
 \end{esempio}
% \end{exrig}

Le seguenti semplificazioni sono errate.
\begin{itemize*}
 \item $\frac{\cancel{a}+b}{\cancel{a}}$ 
  questa semplificazione è errata perché $a$ e~$b$ sono addendi, non sono 
  fattori;
 \item $\frac{\cancel{x^2}+x+4}{\cancel{x^2}+2}$ 
  questa semplificazione è errata perché $x^2$ è un addendo, non un fattore;
 \item $\frac{x^{\cancel{2}}+y^{\cancel{2}}}{(x+y)^{\cancel{2}}}=1$,
  \quad$\frac{\cancel{3a}(a-2)}{\cancel{3a}x-7}=\frac{a-2}{x-7}$,
  \quad~$\frac{\cancel{\left(x-y^2\right)}\cancel{(a-b)}}
              {\cancel{\left(y^2-x\right)}\cancel{(a-b)}}=1$
 \item $\frac{\cancel{(2x-3y)}}{(3y-2x)^{\cancel{2}}}=\frac{1}{3y-2x}$,
  \quad~$\frac{a^2+ab}{a^3}=\frac{\cancel a(a+b)}{a^{\cancel{3}2}}=
         \frac{\cancel{a}+b}{a^{\cancel{2}}}=\frac{1+b}{a}$.
\end{itemize*}

% \ovalbox{\risolvii \ref{ese:19.5},\ref{ese:19.6}, \ref{ese:19.7}, 
% \ref{ese:19.8}, \ref{ese:19.9}, \ref{ese:19.10}, \ref{ese:19.11}, 
% \ref{ese:19.12}}


\section{Moltiplicazione di frazioni algebriche}
\label{sec:frazalg_moltiplicazione}

Il prodotto di due frazioni è una frazione avente per numeratore il prodotto 
dei numeratori e per denominatore il prodotto dei denominatori.

Si vuole determinare il prodotto~$p=\frac{7}{15}\cdot \frac{20}{21}$ possiamo 
scrivere prima il risultato dei prodotti dei numeratori e dei denominatori e 
poi ridurre ai minimi termini la frazione
ottenuta:~$p=\frac{7}{15}\cdot \frac{20}{21}=
\frac{\cancel{140}^4}{\cancel{315}^9}=\frac{4}{9}$,
oppure prima semplificare i termini delle frazioni e poi
moltiplicare:~$p=\frac{7}{15}\cdot \frac{20}{21}=
\frac{\cancel{7}^1}{\cancel{15}^3}\cdot \frac{\cancel{20}^4}{\cancel{21}^3}=
\frac{4}{9}$.

% \begin{exrig}
 \begin{esempio}
Prodotto delle frazioni 
algebriche~$f_{1}=-{\frac{3a^{2}}{10b^{3}c^{4}}}$ 
e~$f_{2}=\frac{25ab^{2}c^{7}}{ab}$.

Poniamo le~$\CE$ per ciascuna frazione assegnata ricordando che tutti i 
fattori letterali dei denominatori devono essere diversi da zero,
quindi~$\CE a\neq~0\wedge b\neq~0\wedge c\neq~0$.
Il prodotto è la 
frazione~$f=-{\frac{3a^{2}}{10b^{3}c^{4}}}\cdot\frac{25ab^{2}c^{7}}{ab}=
-{\frac{15a^{2}c^{3}}{2b^{2}}}$.
 \end{esempio}

 \begin{esempio}
Prodotto delle frazioni algebriche~$f_{1}=-{\frac{3a}{2b+1}}$ 
e~$f_{2}=\frac{10b}{a-3}$.

L'espressione è in due variabili, i denominatori sono polinomi di primo grado 
irriducibili; poniamo le condizioni di
esistenza:~$\CE~2b+1\neq~0\wedge a-3\neq~0$ 
dunque~$\CE b\neq -{\frac{1}{2}}\wedge a\neq~3$.
Il prodotto è la frazione~$f=-{\frac{3a}{2b+1}}\cdot\frac{10b}{a-3}=
-{\frac{30ab}{(2b+1)(a-3)}}$ in cui non è possibile alcuna semplificazione.

\osservazione
$f=-{\frac{3\cancel{a}}{2\cancel{b}+1}}\cdot\frac{10\cancel{b}}{\cancel{a}-3}$. 
Questa semplificazione contiene errori in quanto la variabile~$a$ è un fattore 
del numeratore ma è un addendo nel denominatore; analogamente la variabile~$b$.
 \end{esempio}

 \begin{esempio}
Prodotto delle frazioni algebriche in cui numeratori e denominatori sono 
polinomi~$f_{{1}}=\frac{2x^{2}-x}{x^{2}-3x+2}$
e~$f_{2}=\frac{5x-5}{x-4x^{2}+4x^{3}}$.
\begin{itemize*}
 \item Scomponiamo in fattori tutti i denominatori (servirà per la 
 determinazione delle~$\CE$) e
    tutti i numeratori (servirà per le eventuali semplificazioni),
  \begin{itemize*}
  \item $f_{1}=\frac{2x^{2}-x}{x^{2}-3x+2}=
               \frac{x\cdot(2x-1)}{(x-1)\cdot(x-2)}$,
  \item $f_{2}=\frac{5x-5}{x-4x^{2}+4x^{3}}=
               \frac{5\cdot(x-1)}{x\cdot(2x-1)^{2}}$
  \end{itemize*}
 \item poniamo le~$\CE$ ricordando che tutti i fattori dei denominatori devono 
 essere diversi da
  zero:~$\CE x-1\neq~0\wedge x-2\neq~0\wedge x\neq~0\wedge~2x-1\neq~0$ 
  da cui~$\CE x\neq~1\wedge x\neq~2\wedge x\neq~0\wedge x\neq \frac{1}{2}$
 \item determiniamo la frazione prodotto, effettuando le eventuali 
  semplificazioni:
  \begin{itemize*}
  \item $f=\frac{\cancel{x}\cdot\cancel{(2x-1)}}{\cancel{(x-1)}\cdot
  {(x-2)}}\cdot\frac{5\cdot\cancel{(x-1)}}{\cancel{x}\cdot(2x-1)^{\cancel{2}}}=
     \frac{5}{(x-2)(2x-1)}$.
  \end{itemize*}
\end{itemize*}
\end{esempio}
% \end{exrig}

% \ovalbox{\risolvii \ref{ese:19.13}, \ref{ese:19.14}, \ref{ese:19.15}, 
% \ref{ese:19.16}, \ref{ese:19.17}}

\section{Divisione di frazioni algebriche}
\label{sec:frazalg_divisione}

L'introduzione delle frazioni algebriche permette di trasformare ogni 
divisione in una moltiplicazione.
Il quoziente di due frazioni è la frazione che si ottiene moltiplicando la 
prima con il reciproco della seconda.
Lo schema di calcolo può essere illustrato nel modo seguente, come del resto 
abbiamo visto nell'insieme dei numeri razionali:

\[\frac{m}{n}:\frac{p}{q}=\frac{m}{n}\cdot {\frac{q}{p}}=
  \frac{m\cdot q}{n\cdot p}.\]

Si vuole determinare il quoziente~$q=\frac{5}{12}:\frac{7}{4}$. 
Il reciproco di~$\frac{7}{4}$ è la frazione~$\frac{4}{7}$
dunque,

\[q=\frac{5}{12}:\frac{7}{4}=
    \frac{5}{\cancel{12}^3}\cdot\frac{\cancel{4}^1}{7}=\frac{5}{21}.\]

% \begin{exrig}
 \begin{esempio}
Determinare il quoziente delle frazioni 
algebriche~$f_{1}=\frac{3a-3b}{2a^{2}b}$ e~$f_{2}=\frac{a^{2}-ab}{b^{2}}$.
\begin{itemize*}
 \item Scomponiamo in fattori le due frazioni 
  algebriche:~$f_{1}=\frac{3a-3b}{2a^{2}b}=\frac{3\cdot(a-b)}{2a^{2}b}$ e
    $f_{2}=\frac{a^{2}-ab}{b^{2}}=\frac{a\cdot (a-b)}{b^{2}};$
 \item poniamo le condizioni d'esistenza dei 
  denominatori:~$2a^{2}b\neq~0\wedge b^{2}\neq~0$
  da cui~$\CE$ $a\neq~0\wedge b\neq~0$
 \item determiniamo la frazione inversa di~$f_2$. Per poter determinare 
  l'inverso dobbiamo porre le condizioni perché la frazione non sia nulla.
  Poniamo il numeratore diverso da zero, $C_0: a\neq~0\wedge a-b\neq~0$ 
  da cui~$C_0: a\neq~0\wedge a\neq b$
 \item aggiorniamo le condizioni~$\CE a\neq~0\wedge b\neq~0\wedge a\neq b$
 \item cambiamo la divisione in moltiplicazione e semplifichiamo:
\begin{equation*}
\frac{3\cdot(a-b)}{2a^{2}b}:\frac{a\cdot(a-b)}{b^2}=
\frac{3\cdot\cancel{(a-b)}}{2a^{2}\cancel{b}}\cdot
\frac{b^{\cancel{2}}}{a\cdot\cancel{(a-b)}}=\frac{3b}{2a^3}.
\end{equation*}
\end{itemize*}
 \end{esempio}
% \end{exrig}

% \ovalbox{\risolvii \ref{ese:19.19}, \ref{ese:19.20}, \ref{ese:19.21}}

\section{Potenza di una frazione algebrica}
\label{sec:frazalg_potenza}

La potenza di esponente~$n$, naturale diverso da zero, della frazione 
algebrica~$\frac{A}{B}$ con~$B{\neq}0$ ($\CE$) è la frazione
avente per numeratore la potenza di esponente~$n$ del numeratore e per 
denominatore la potenza di esponente~$n$
del denominatore:~$\left(\frac{A}{B}\right)^{n}=\frac{A^{n}}{B^{n}}$.

% \begin{exrig}
 \begin{esempio}
Calcoliamo  $\left(\frac{x-2}{x^{2}-1}\right)^{3}$.

Innanzi tutto determiniamo le~$\CE$ per la frazione
assegnata

\[\frac{x-2}{x^{2}-1}=\frac{x-2}{(x-1)\cdot(x+1)}(x-1)(x+1)\neq~0,\] 

da cui~$\CE x\neq~1\wedge x\neq -1$.
Dunque si ha

\[\left(\frac{x-2}{x^{2}-1}\right)^{3}=
  \frac{(x-2)^{3}}{(x-1)^{3}\cdot (x+1)^{3}}.\]
  
 \end{esempio}
% \end{exrig}

\subsection{Casi particolari dell'esponente}

Se~$n = 0$ sappiamo che qualsiasi numero diverso da zero elevato a zero è 
uguale a~$1$ lo stesso si può dire se la base è una frazione algebrica, 
purché essa non sia nulla.
$\left(\frac{A}{B}\right)^{0}=1$ con~$A\neq~0$ e~$B\neq~0$.
% \begin{exrig}
 \begin{esempio}
Quali condizioni deve rispettare la variabile~$a$ per 
avere~$\left(\dfrac{3a-2}{5a^{2}+10a}\right)^{0}=1$?
\begin{itemize*}
 \item Scomponiamo in fattori numeratore e denominatore della 
  frazione:~$\left(\frac{3a-2}{5a\cdot (a+2)}\right)^{0}$
 \item determiniamo le~$\CE$ del denominatore:~$a\neq~0\wedge a+2\neq~0$ 
  da cui, $\CE a\neq~0\wedge a\neq -2$.
  Poniamo poi la condizione, affinché la frazione non sia nulla, che anche il 
  numeratore sia diverso da zero.
  Indichiamo con~$C_0$ questa condizione, dunque~$C_0$:~$3a-2\neq~0$, 
  da cui~$a\neq \frac{2}{3}$
 \item le condizioni di esistenza sono 
  allora~$a\neq -2\wedge a\neq~0\wedge a\neq \frac{2}{3}$.
\end{itemize*}
 \end{esempio}
% \end{exrig}

Se~$n$ è intero negativo la potenza con base diversa da zero è uguale alla 
potenza che ha per base l'inverso della base e per esponente l'opposto
dell'esponente. 
$\left(\frac{A}{B}\right)^{-n}=
 \left(\frac{B}{A}\right)^{+n}$ con~$A\neq~0$ e~$B\neq~0$.
% \begin{exrig}
 \begin{esempio}
Determinare~$\left(\dfrac{x^{2}+5x+6}{x^{3}+x}\right)^{-2}$.
\begin{itemize*}
 \item Scomponiamo in fattori numeratore e 
  denominatore:~$\left(\frac{\left(x+2\right)\cdot \left(x+3\right)}
                            {x\cdot \left(x^{2}+1\right)}\right)^{-2}$
 \item $\CE$ del denominatore~$x\neq~0$ e~$x^2+1\neq~0$ da cui~$\CE x\neq~0$ 
  essendo l'altro fattore sempre diverso da~0.
  Per poter determinare la frazione inversa dobbiamo porre le condizioni 
  perché la frazione non sia nulla e cioè che anche il numeratore sia diverso 
  da zero, quindi si deve 
  avere~$C_0: (x +2)(x+3 )\neq~0$ da cui~$C_0: x\neq -2$ e $x \neq -3$
 \item quindi se~$x\neq~0$, $x\neq-2$ 
  e~$x\neq-3$ si ha~$\left(\frac{(x+2)\cdot (x+3)}
                                {x\cdot \left(x^{2}+1\right)}\right)^{-2}=
    \frac{x^{2}\cdot \left(x^{2}+1\right)^{2}}{(x+2)^{2}\cdot (x+3)^{2}}$.
\end{itemize*}
 \end{esempio}
% \end{exrig}

% \ovalbox{\risolvi \ref{ese:19.18}}

\section{Addizione di frazioni algebriche}
\label{sec:frazalg_addizione}

% \subsection{Proprietà della addizione tra frazioni algebriche}
% Nell'insieme delle frazioni algebriche la somma:
% \begin{itemize*}
% \item è commutativa:~$f_1+ f_2 = f_2 + f_1$
% \item è associativa:~$(f_1+ f_2) + f_3 = f_1 + (f_2 + f_3) = f_1 + f_2 + f_3$
% \item possiede l'elemento neutro, cioè esiste una frazione~$F^0$ tale che per 
%  qualunque frazione~$f$ si abbia~$F^{0} + f = f + F^0= f$ e~$F^0 = 0$
% \item ogni frazione algebrica~$f$, possiede la frazione opposta~$(-f )$ 
%  tale che
% \[(-f) + f = f + (- f) = F^0 = 0.\]
% \end{itemize*}
% Quest'ultima proprietà ci permette di trattare contemporaneamente 
% l'operazione 
% di addizione e di sottrazione, perché ogni sottrazione può essere trasformata 
% nell'addizione tra il primo termine e l'opposto del secondo.
% Come per i numeri relativi, quando si parlerà di somma di frazioni si 
% intenderà ``somma algebrica''.

Come per le frazioni numeriche, anche nelle frazioni algebriche l'addizione
è l'operazione più complicata. Infatti per addizionare due frazioni algebriche 
bisogna:

\begin{procedura}
 Per addizionare frazioni algebriche:
\begin{enumeratea}
\item scomporre in fattori i denominatori;
\item determinare il $\mcm$;
\item riscrivere le frazioni in modo che abbiano lo stesso denominatore;
\item addizionare i numeratori;
\item semplificare il risultato ottenuto ponendo le condizioni di esistenza.
\end{enumeratea}
\end{procedura}

% \begin{exrig}
 \begin{esempio}
$\dfrac{x+2}{x^{2}-2x}-\dfrac{x-2}{2x+x^{2}}+\dfrac{-4x}{x^{2}-4}$
\begin{enumeratea}
 \item \emph{scomporre in fattori i denominatori}
 
  \[\frac{x+2}{x(x-2)}-\frac{x-2}{x(2+x)}+\frac{-4x}{(x+2)(x-2)}\]
  
 \item \emph{determinare il $\mcm$}
 
  \[\mcm = x\cdot (x+2) \cdot (x-2)\]
  
%  \item poniamo le~$\CE$ $x(x+2)(x-2){\neq}0$ da cui~$\CE$ $x{\neq}0$, 
%   ${x\neq}2$ e~${x\neq}-2$

 \item \emph{riscrivere le frazioni in modo che abbiano lo stesso denominatore}
 
  dividiamo il~$\mcm$ per ciascun denominatore e moltiplichiamo il 
  quoziente ottenuto per il relativo numeratore:
  
    \[\frac{(x+2)^{2}-(x-2)^{2}-4x^{2}}{x\cdot(x+2)\cdot(x-2)}\]
    
 \item \emph{addizionare i numeratori} 
 
  eseguiamo le operazioni al numeratore:
    \begin{equation*}
     \frac{x^{2}+4x+4-x^{2}+4x-4-4x^{2}}{x\cdot(x+2)\cdot(x-2)}=
     \frac{8x-4x^{2}}{x\cdot(x+2)\cdot(x-2)}
     \end{equation*}
     
 \item \emph{semplificare il risultato ottenuto ponendo le condizioni di 
  esistenza}
 
  per semplificare la frazione dobbiamo scomporre il numeratore:
    \begin{equation*}
    \frac{-4\cancel{x}\cdot\cancel{(x-2)}}
         {\cancel{x}\cdot(x+2)\cdot\cancel{(x-2)}}=
    \frac{-4}{x+2} \quad \wedge \quad x \neq 0 \quad \wedge \quad x \neq 2.
    \end{equation*}
\end{enumeratea}
 \end{esempio}

 \begin{esempio}
$\dfrac{x}{x-2}-\dfrac{2x}{x+1}+
\dfrac{x}{x-1}-\dfrac{5x^{2}-7}{x^{3}-2x^{2}+2-x}=$
\begin{enumeratea}
 \item Scomponiamo in fattori~$x^{3}-2x^{2}+2-x$, 
  essendo gli altri denominatori irriducibili: 
  $x^{3}-2x^{2}+2-x=x^2(x-2)-1(x-2)=(x-2)\left(x^2-1\right)=
   (x-2)(x+1)(x-1)$ che è anche il~$\mcm$ dei denominatori;

 \item Calcoliamo il $\mcm$ dei denominatori: 
  $\mcm=(x-2)(x+1)(x-1)$;
  
%  \item poniamo le~$\CE$ $(x-2)(x+1)(x-1){\neq}0$ da cui~$\CE$ $x{\neq}2$, 
${x\neq}-1$ e~${x\neq}1$
 \item dividiamo il~$\mcm$ per ciascun denominatore e moltiplichiamo il 
  quoziente ottenuto per il relativo numeratore:
  
 \[=\frac{x(x+1)(x-1)-2x(x-2)(x-1)+x(x-2)(x+1)-(5x^{2}-7)}{(x-2)(x+1)(x-1)}=\]
 
 
 \item eseguiamo le operazioni al numeratore:
 
  \[=\frac{x(x^2-1)-2x(x^2-3x+2)+x(x^2-x-2)-(5x^{2}-7)}{(x-2)(x+1)(x-1)} = \]
  
  \[=\frac{x^3-x-2x^3+6x^2-4x+x^3-x^2-2x-5x^2+7}{(x-2)(x+1)(x-1)}=\]
 
  \[=\frac{-x-4x-2x+7}{(x-2)(x+1)(x-1)}=\frac{-7x+7}{(x-2)(x+1)(x-1)}=\]
 
 \item scomponiamo il numeratore e semplifichiamo la frazione ottenuta, 
  ponendo le condizioni. La frazione somma è:
 
  \[=\frac{-7\cancel{(x-1}}{(x-2)(x+1)\cancel{(x-1)}}=
    -{\frac{7}{(x-2)(x+1)}} \quad \wedge \quad x \neq +1\]
 
 \end{enumeratea}
 \end{esempio}

% \end{exrig}
% 
% A volte la situazione è più semplice:
% 
% \begin{itemize*}
%  \item se i denominatori sono dei monomi, sono già scomposti in fattori;
%  \item a volte le frazioni hanno già lo stesso denominatore.
% \end{itemize*}
% 
%  In questi casi si può saltare una parte della procedura.
% 
% % \begin{exrig}
%  \begin{esempio}
% $\dfrac{x+y}{3x^{2}y}-\dfrac{2y-x}{2xy^{3}}$.
% 
% Dobbiamo trasformare le frazioni in modo che abbiano lo stesso denominatore:
% \begin{itemize*}
%  \item calcoliamo il~$\mcm(3x^{2}y, 2xy^{3}) = 6x^{2}y^{3}$
%  \item poniamo le~$\CE$ $6x^{2}y^{3}{\neq}0$ da cui~$\CE x{\neq}0$ 
e~$y{\neq}0$
%  \item dividiamo il~$\mcm$ per ciascun denominatore e moltiplichiamo il 
%  quoziente ottenuto per il relativo numeratore:
%     \[\frac{2y^{2}\cdot(x+y)}{6x^{2}y^{3}}-\frac{3x\cdot(2y-x)}{6x^{2}y^{3}}\]
%  \item la frazione somma ha come denominatore lo stesso denominatore e come 
%  numeratore la somma dei numeratori:
%     \begin{equation*}
%     \frac{2y^{2}\cdot(x+y)}{6x^{2}y^{3}}-\frac{3x\cdot(2y-x)}{6x^{2}y^{3}}=
%     \frac{2xy^{2}+2y^{3}+2x^{2}y-6xy+3x^{2}}{6x^{2}y^{3}}.
%     \end{equation*}
% \end{itemize*}
%  \end{esempio}
% 
%  \begin{esempio}
% $\frac{2x-3y}{x+y}+\frac{x+2y}{x+y}$
% 
% Poniamo le~$\CE$ $x + y{\neq}0$ da cui~$\CE x{\neq}-y$ allora
% \begin{equation*}
% 
% \frac{2x-3y}{x+y}+\frac{x+2y}{x+y}=\frac{(2x-3y)+(x+2y)}{x+y}=\frac{3x-y}{x+y}.
% \end{equation*}
%  \end{esempio}
%  
% % \osservazione A questo caso ci si può sempre ricondurre trasformando le 
% % frazioni allo stesso denominatore. 
% % Si potrebbe scegliere un qualunque denominatore comune, ad esempio il 
% prodotto 
% % di tutti i denominatori ma, scegliamo il~$\mcm$ dei denominatori delle 
% % frazioni addendi per semplificare i calcoli.
% 
% % \end{exrig}

% \ovalbox{\risolvii \ref{ese:19.22}, \ref{ese:19.23}, \ref{ese:19.24}, 
% \ref{ese:19.25}, \ref{ese:19.26}, \ref{ese:19.27}, \ref{ese:19.28}, 
% \ref{ese:19.29}, \ref{ese:19.30}}
