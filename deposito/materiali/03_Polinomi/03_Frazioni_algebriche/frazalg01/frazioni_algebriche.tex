% (c) 2012-2013 Claudio Carboncini - claudio.carboncini@gmail.com
% (c) 2012-2013 Dimitrios Vrettos - d.vrettos@gmail.com
% (c) 2015 Daniele Zambelli daniele.zambelli@gmail.com

\input{\folder frazioni_algebriche_grafici}

\chapter{Frazioni algebriche}

% \begin{wrapfloat}{figure}{r}{0pt}
% \includegraphics[scale=0.35]{img/fig000_.png}
% \caption{...}
% \label{fig:...}
% \end{wrapfloat}
% 
% \begin{center} \input{\folder lbr/fig000_.pgf} \end{center}

I polinomi, rispetto alle operazioni si comportano come i numeri interi, 
in particolare la divisione tra due polinomi spesso dà un resto. Così, proprio 
come abbiamo fatto con i numeri interi, anche con i polinomi dovremo utilizzare 
le frazioni per poter eseguire sempre le divisioni. Per poter operare con le 
frazioni dovremo imparar a calcolare il minimo comune multiplo tra polinomi.

\section{Divisore comune e multiplo comune}
\label{sec:frazalg_MCDemcm}

Per determinare~\(\mcd\) (\emph{massimo comune divisore}) 
e~\(\mcm\) (\emph{minimo comune multiplo}) di due o più polinomi occorre prima 
di tutto scomporli in fattori irriducibili.

\osservazione La cosa non è semplice poiché non si può essere sicuri di aver 
trovato il massimo comune divisore o il minimo comune multiplo
per la difficoltà di decidere se un polinomio è irriducibile: 
prudentemente si dovrebbe parlare di divisore comune e di multiplo comune.

Un polinomio~\(A\) si dice multiplo di un polinomio~\(B\) se esiste un 
polinomio~\(C\) per il quale~\(A=B\cdot C\) in questo caso diremo
anche che~\(B\) è divisore del polinomio~\(A\).

\subsection{Massimo Comun Divisore}
Dopo aver scomposto ciascun polinomio in fattori irriducibili, il massimo 
comune divisore tra due o più polinomi è il prodotto di tutti i fattori comuni 
ai polinomi, presi ciascuno una sola volta, con il minimo esponente.
Sia i coefficienti numerici, sia i monomi possono essere considerati polinomi.

\begin{procedura}
Calcolare il~\(\mcd\) tra polinomi:
\begin{enumeratea}
\item scomponiamo in fattori irriducibili ogni polinomio;
\item prendiamo i fattori comuni a tutti i polinomi una sola volta con 
l'esponente più piccolo;
\item se non ci sono fattori comuni a tutti i polinomi il~\(\mcd\) è~\(1\).
\end{enumeratea}
\end{procedura}

% \begin{exrig}
 \begin{esempio}
Determinare il~\(\mcd \tonda{3a^{2}b^{3}-3b^{3};\quad 6a^{3}b^{2}-6b^{2}; 
                      \quad 2a^{2}b^{2}-24ab^{2}+12b^{2}} \).
 \begin{itemize*}
 \item Scomponiamo in fattori i singoli polinomi;
  \begin{itemize*}
  \item \(3a^{2}b^{3}-3b^{3}=3b^{3} \tonda{a^{2}-1} =3b^{3}(a-1)(a+1)\)
  \item \(6a^{3}b^{2}-6b^{2}=6b^{2} \tonda{a^{3}-1} =
         6b^{2}(a-1) \tonda{a^{2}+a+1} \)
  \item \(12a^{2}b^{2}-24ab^{2}+12b^{2}=12b^{2} \tonda{a^{2}-2a+1} =
         12b^{2}(a-1)^{2}\).
  \end{itemize*}
 \item i fattori comuni a tutti i polinomi presi con l'esponente più piccolo 
  sono:
  \begin{itemize*}
  \item tra i numeri il~\(3\)
  \item tra i monomi~\(b^{2}\)
  \item tra i polinomi~\(a-1\).
  \end{itemize*}
 \item quindi il~\(\mcd=3b^{2}(a-1)\).
 \end{itemize*}
 \end{esempio}
% \end{exrig}

\subsection{Minimo comune multiplo}
Dopo aver scomposto ciascun polinomio in fattori, il minimo comune multiplo 
tra due o più polinomi è il prodotto dei fattori comuni e non comuni di tutti 
i polinomi, quelli comuni presi una sola volta, con il massimo esponente.

\begin{procedura}
Calcolare il~\(\mcm\) tra polinomi:
\begin{enumeratea}
\item scomponiamo in fattori irriducibili ogni polinomio;
\item prendiamo tutti i fattori comuni e non comuni dei polinomi, i fattori 
 comuni presi una sola volta con il massimo esponente.
\end{enumeratea}
\end{procedura}

% \begin{exrig}
 \begin{esempio}
Determinare il~\(\mcm \tonda{3a^{2}b^{3}-3b^{3};\quad 6a^{3}b^{2}-6b^{2};
                      \quad 2a^{2}b^{2}-24ab^{2}+12b^{2}} \).
 \begin{itemize*}
 \item Scomponiamo in fattori i singoli polinomi;
  \begin{itemize*}
  \item \(3a^{2}b^{3}-3b^{3}=3b^{3} \tonda{a^{2}-1} =3b^{3}(a-1)(a+1)\)
  \item \(6a^{3}b^{2}-6b^{2}=6b^{2} \tonda{a^{3}-1} =
         6b^{2}(a-1) \tonda{a^{2}+a+1} \)
  \item \(12a^{2}b^{2}-24ab^{2}+12b^{2}=12b^{2} \tonda{a^{2}-2a+1} =
         12b^{2}(a-1)^{2}\).
  \end{itemize*}
 \item i fattori comuni e non comuni presi con il massimo esponente sono:
  \begin{itemize*}
  \item tra i coefficienti numerici il~\(12\)
  \item tra i monomi~\(b^{3}\)
  \item tra i polinomi~\((a-1)^{2}\cdot (a+1)\cdot  \tonda{a^{2}+a+1} \).
  \end{itemize*}
 \item quindi il~\(\mcm=12b^{3}(a-1)^{2}(a+1) \tonda{a^{2}+a+1} \).
 \end{itemize*}
 \end{esempio}
% \end{exrig}

% \ovalbox{\risolvii \ref{ese:18.1}, \ref{ese:18.2}, \ref{ese:18.3}, 
% \ref{ese:18.4}, \ref{ese:18.5}, \ref{ese:18.6}, \ref{ese:18.7}, 
% \ref{ese:18.8}, \ref{ese:18.9}, \ref{ese:18.10}, \ref{ese:18.11}}

\section{Definizione di frazione algebrica}
\label{sec:frazalg_definizione}

Abbiamo visto, nel capitolo precedente che, a volte il quoziente di due 
polinomi è un polinomio, altre volte la divisione tra polinomi dà un resto.
In questo caso il quoziente esatto dei due polinomi non è un polinomio, ma 
una frazione algebrica, cioè una frazione che ha un polinomio al numeratore e 
uno al denominatore.

% \begin{wrapfloat}{figure}{r}{0pt}
% \includegraphics[scale=0.35]{img/fig000_.png}
% \caption{...}
% \label{fig:...}
% \end{wrapfloat}
% 
% \begin{center} \input{\folder lbr/fig000_.pgf} \end{center}

Diamo la seguente definizione:
\begin{definizione}
Si definisce \emph{frazione algebrica} una espressione del 
tipo~\(\dfrac{A}{B}\) dove~\(A\) e~\(B\) sono polinomi.
\end{definizione}

Una frazione algebrica può essere vista come una funzione: quando alle 
variabili viene assegnato un valore, la frazione algebrica dà un risultato.
I valori delle variabili sono gli argomenti della funzione, il valore assunto 
dalla frazione algebrica è il risultato della funzione.

\affiancati{.58}{.38}
{
Quando l'espressione contiene una sola variabile, può essere vista come una 
funzione ``reale'' cioè una funzione che associa ad ogni numero reale al 
massimo un altro numero reale.
In questo caso può essere rappresentata su un piano cartesiano.

Ad esempio la funzione: ~~ \(f(x) = \dfrac{x^2 -3x -4}{2x^2 +4x -16}\)\\
produce il grafico a fianco.

Si può osservare che la funzione presenta due zeri per i valori: 
\(x_1=-1 \stext{e} x_2=+4\) e per due valori sembra non essere definita:
\(x_3=-4 \stext{e} x_4=+2\).
}
{\hspace{5mm} \graficofrazionea}

% Di seguito vediamo come calcolare il quoziente tra espressioni letterali.
% Ricordiamo che la divisione è definita solo se il divisore è diverso da 
% zero.
% 
% Quando nel semplificare una frazione, scompaiono dei fattori presenti a 
% denominatore, dobbiamo specificare che quei fattori devono essere diversi 
% da zero.
% 
% Vediamo alcuni esempi.
% 
% % \paragraph{Caso I}Quoziente tra monomi.
% % \begin{exrig}
%  \begin{esempio} 
% Eseguiamo le seguenti divisioni ricordando di indicare quali espressioni 
% devono essere diverse da zero.
% \begin{enumerate}
% \item 
% \(5a^{3}b^{2}c^{5}: \tonda{-3a^{2}bc^{5}} =
%     \dfrac{5a^{\cancel{3}}b^{\cancel{2}}{\cancel{c^5}}}
%           {-3\cancel{a^{2}}\cancel{b}\cancel{c^{5}}}=
%     -\dfrac{5}{3}ab 
%       \quad \text{ con } a \ne 0 \stext{e} b \ne 0 \stext{e} c \ne 0\)
% \item 
% \(4a^{3}b^{2}c^{5} : \tonda{-7a^{7}bc^{5}} =
%     \dfrac{4\cancel{a^{3}}b^{\cancel{2}}\cancel{c^{5}}}
%           {-7a^{\cancel{7}}\cancel{b}\cancel{c^{5}}}= 
%     -{\dfrac{4b}{7a^{4}}}=
%     -{\dfrac{4~}{7~}}a^{-4}b
%        \quad \text{ con } a \ne 0 \stext{e} b \ne 0 \stext{e} c \ne 0\)
% \item 
% \(q=2a^{3}b: \tonda{a^{2}+b} =\dfrac{2a^{3}b}{a^{2}+b}
%        \quad \text{ con } a^{2}+b \ne 0\)
% \item 
% \(\tonda{2a^{3}b+a^{5}b^{3}-3ab^{2}} : \tonda{\dfrac{1}{2}ab} =
%   4a^{2}+2a^{4}b^{2}-6b
%        \quad \text{ con } a \ne 0 \stext{e} b \ne 0\)
% \item 
% \(\tonda{2a^{3}b+a^{5}b^{3}-3ab^{2}} : \tonda{\dfrac{1}{2}a^{5}b} =
%   \dfrac{4}{a^{2}}+2b^{2}-\dfrac{6b}{a^{4}}
%        \quad \text{ con } a \ne 0 \stext{e} b \ne 0\)
% \item 
% \(\tonda{x-3} : \tonda{x^{2}+1}=\dfrac{x-3}{x^{2}+1}
%        \quad \text{ con } x^{2}+1 \ne 0\)
% \end{enumerate}
% 
%  \end{esempio}
% 
% \paragraph{Conclusione}
% Il quoziente tra due polinomi è un polinomio se il dividendo è divisibile 
% per il divisore, altrimenti il risultato della divisione può essere 
% espresso da una \emph{frazione algebrica}.
% Una frazione algebrica può essere considerata come il quoziente tra due 
% polinomi.

\section{Condizioni di esistenza per una frazione algebrica}
\label{sec:frazalg_condizioniesistenza}

Poiché non è mai possibile dividere per~\(0\), una frazione algebrica perde 
di significato per quei valori che attribuiti alle variabili rendono il 
denominatore uguale a zero. 
Per ``discussione di una frazione algebrica'' intendiamo la ricerca dei 
valori che, attribuiti alle variabili, la rendano calcolabile. 
Quando abbiamo una frazione algebrica tipo~\(\dfrac{A}{B}\) possiamo 
studiarne la condizione di esistenza (abbreviato con~\(\CE\)):~\(B\neq~0\).

La funzione precedente non dà quindi un risultato quando: \(2x^2 +4x -16=0\).
Scomponendo in fattori il polinomio a primo membro: 
\(2 \tonda{\tonda{x+4}\tonda{x-2}}=0\),
vediamo che il denominatore si annulla quando \(x=-4\) o \(x=+2\).
Per questi due valori la funzione non è definita.
% \begin{exrig}
 \begin{esempio}
Determinare le condizioni di esistenza di esistenza delle seguenti funzioni, 
cioè studiare per quali valori delle variabili il denominatore è diverso da 
zero.

Vediamo alcuni esempi:

\begin{enumerate}
\item \(f(x)=\dfrac{1+x}{x} \srarrow \CE:~ x\neq~0\)
\item \(f(x)=\dfrac{x}{x+3} \srarrow \CE:~ x+3 \neq 0 \sRarrow x \neq -3\)
\item \(f(x)=\dfrac{3a+5b-7}{ab} \srarrow 
\CE:~ ab \neq 0 \sRarrow a\neq~0 \stext{e} b\neq~0\)
\item \(f(x)=\dfrac{-6}{2x+5} \srarrow 
\CE:~ 2x+5\neq~0 \sRarrow 2x \neq -5 \sRarrow x\neq -\dfrac{5}{2}\)
\item \(f(x)=\dfrac{-x^{3}-8x}{x^{2}+2} \srarrow 
\text{ il denominatore è sempre maggiore di zero } \CE:~\text{qualunque 
numero.} \)
\end{enumerate}
\end{esempio}

Ci sono dei casi più laboriosi da risolvere, quando il denominatore si 
presenta nella forma di un polinomio non lineare.

Negli insiemi numerici che stiamo usando, il risultato di una moltiplicazione 
è zero se almeno uno dei fattori è zero:

\begin{definizione}
Si dice \emph{Legge di annullamento del prodotto}, l'osservazione che nei 
numeri razionali il prodotto di più fattori è nullo se almeno uno dei fattori 
è nullo.
\end{definizione}

Quando è possibile, dobbiamo scomporre in fattori il denominatore e studiare 
ogni singolo fattore ponendolo diverso da zero.

 \begin{esempio}
Determinare le condizioni di esistenza di~\(\dfrac{2x}{x^{2}-4}\).

Scomponendo in fattori il denominatore otteniamo:
\(\dfrac{2x}{x^{2}-4} = \dfrac{2x}{\tonda{x-2}\tonda{x+2}}\)

e per la legge di annullamento del prodotto ricaviamo:

\(\CE:~ \tonda{x-2}\tonda{x+2} \neq~0 \sRarrow 
\tonda{x-2} \neq 0 \stext{e} \tonda{x+2} \neq 0 \sRarrow 
x \neq -2 \stext{e} x \neq +2\)

La condizione \(x \neq -2 \stext{e} x \neq +2\) può anche essere sintetizzata 
con l'espressione: \(x \neq \mp 2\).
 \end{esempio}
% \end{exrig}

\begin{procedura}
Determinare la condizione di esistenza di una frazione algebrica:
\begin{enumeratea}
\item scomporre in fattori il denominatore;
\item porre ciascun fattore del denominatore diverso da zero;
\item escludere i valori che annullano il denominatore.
\end{enumeratea}
\end{procedura}

% \ovalbox{\risolvii \ref{ese:19.1}, \ref{ese:19.2}, \ref{ese:19.3}, 
% \ref{ese:19.4}}

\section{Semplificazione di una frazione algebrica}
\label{sec:frazalg_semplificazione}

Semplificare una frazione algebrica significa dividere numeratore e 
denominatore per uno stesso fattore diverso da zero, in questo modo infatti 
la proprietà invariantiva della divisione garantisce che la frazione non 
cambia di valore.
Quando riduciamo ai minimi termini una frazione numerica, dividiamo il 
numeratore e il denominatore per il loro~\(\mcd\) che è sempre un numero
diverso da zero, ottenendo una frazione equivalente a quella assegnata.

Quando semplifichiamo una frazione algebrica, dobbiamo 
porre attenzione a escludere quei valori che, attribuiti alle variabili, 
rendono nullo il~\(\mcd\).

% \begin{exrig}
 \begin{esempio}
Semplificare~\(\dfrac{16x^{3}y^{2}z}{10xy^{2}}\)
\[\dfrac{\cancel{16}x^{\cancel{3}}\cancel{y^{2}}z}
        {\cancel{10}\cancel{x}\cancel{y^{2}}}=
\dfrac{8x^{2}z}{5}=\dfrac{8}{5}x^{2}z
\quad \text{ con } x \ne 0 \stext{e} y \ne 0
\]
 \end{esempio}

 \osservazione In una frazione non possiamo semplificare singoli addendi 
 ma solo fattori. 
 
 Vediamo un esempio numerico: \quad \(\dfrac{4+6}{2+12}\). 
 
Se operiamo correttamente otteniamo: \quad
\(\dfrac{2+6}{2+12}=\dfrac{8}{14}=\dfrac{4}{7}\)

Se ci avventuriamo in ``semplificazioni'' fantasiose, possiamo 
semplificare il \(2\) con il \(2\) e il \(6\) con il 
\(12\), ottenendo: \quad
\(\dfrac{\cancel{2} +\cancel{6}}{\cancel{2} +\cancel{12}}=
\dfrac{1+1}{1+2}=\dfrac{2}{3}\)

È chiaro che questo tipo di ``semplificazioni'' non è accettabile.

\newpage % ------------------------------------------------

Di seguito possiamo vedere alcuni esempi di ``semplificazioni'' errate:
\begin{multicols}{3}
\begin{itemize}
 \item \(\dfrac{\cancel{a}+b}{\cancel{a}}\)
 \item \(\dfrac{\cancel{x^2}+x+4}{\cancel{x^2}+2}\) 
 \item \(\dfrac{x^{\cancel{2}}+y^{\cancel{2}}}{(x+y)^{\cancel{2}}}=1\) 
 \item \(\dfrac{\cancel{3a}(a-2)}{\cancel{3a}x-7}=\dfrac{a-2}{x-7}\) 
 \item \(\dfrac{\cancel{ \tonda{x-y^2}}}
              {\cancel{ \tonda{y^2-x}}}=1\)
%  \item \(\dfrac{\cancel{(2x-3y)}}{(3y-2x)^{\cancel{2}}}=\dfrac{1}{3y-2x}\) 
 \item \(\dfrac{\cancel{a}+b}{a^{\cancel{2}}}=\dfrac{1+b}{a}\)
\end{itemize}
\end{multicols}

La semplificazione di una funzione razionale produce un'altra funzione che si 
comporta \emph{quasi} sempre come la funzione di partenza.

\affiancati{.48}{.48}
{\centering\graficofrazioneb}
{\centering\graficofrazionec}

Le funzioni \(f_1\) e \(f_2\) sono uguali dappertutto tranne che in \(+3\) 
dove \(f_1\) non è definita mentre \(f_2\) vale \(\frac{2}{5}\).
% : 
% \(f_1\) non è definita in \(-2\) e \(+3\) (zeri del denominatore), 
% \(f_2\) non è definita in solo in \(-2\). 

 \begin{esempio}
Ridurre ai minimi termini la frazione:~\(\dfrac{a^{2}-6a+9}{a^{4}-81}\).
\begin{enumerate}
 \item Scomponiamo in fattori
  \begin{itemize*}
  \item il numeratore:~\(a^2 - 6a +9 = (a - 3 )^2\)
  \item il denominatore:~\(a^4 - 81 = \tonda{a^2 - 9} \tonda{a^2 + 9} = 
                           \tonda{a - 3}\tonda{a + 3} \tonda{a^2 + 9} \)
  \end{itemize*}
 \[\dfrac{a^{2}-6a+9}{a^{4}-81} = 
   \dfrac{\tonda{a-3}^2}
         {\tonda{a-3} \tonda{a+3} \tonda{a^{2}+9}}\]
 \item Semplifichiamo:
 \[\dfrac{(a-3)^{\cancel{2}}}
         {\cancel{(a-3)}\cdot (a+3)\cdot  \tonda{a^{2}+9} }=
   \dfrac{a-3}{\tonda{a+3)(a^{2}+9}}\]
 \item Indichiamo quali sono le condizioni che si sono perse nella 
semplificazione:
 \[a-3 \neq 0 \sRarrow a \neq +3\]
\end{enumerate}
Il tutto può essere scritto in un'unica espressione:
 \[\dfrac{a^{2}-6a+9}{a^{4}-81} \stackrel{1}{=} 
   \dfrac{(a-3)^{\cancel{2}}}
         {\cancel{(a-3)}\cdot (a+3)\cdot \tonda{a^{2}+9}}\stackrel{2}{=} 
   \dfrac{a-3}{ \tonda{a+3)(a^{2}+9}} \stackrel{3}{\stext{quando}} 
   a-3 \neq 0 \sRarrow a \neq +3\]
 \end{esempio}

 \begin{esempio}
Ridurre ai minimi termini la frazione in due 
variabili:~\(\dfrac{x^{4}+x^{2}y^{2}-x^{3}y-xy^{3}}
                   {x^{4}-x^{2}y^{2}+x^{3}y-xy^{3}}\).
\begin{itemize*}
 \item Scomponiamo in fattori con il raccoglimento parziale
  \begin{itemize*}
  \item \(x^{4}+x^{2}y^{2}-x^{3}y-xy^{3}=
   x^{2}  \tonda{x^{2}+y^{2}} -xy  \tonda{x^{2}+y^{2}} =
   x  \tonda{x^{2}+y^{2}}  (x-y)\)
  \item \(x^{4}-x^{2}y^{2}+x^{3}y-xy^{3}=
   x^{2}  \tonda{x^{2}-y^{2}} +xy  \tonda{x^{2}-y^{2}} =
   x (x+y)^{2} (x-y)\)
  \end{itemize*}
 \item Semplifichiamo:
\[\dfrac{x^{4}+x^{2}y^{2}-x^{3}y-xy^{3}}
        {x^{4}-x^{2}y^{2}+x^{3}y-xy^{3}}=
  \dfrac{\cancel{x}  \tonda{x^{2}+y^{2}}  \cancel{(x-y)}}
        {\cancel{x}(x+y)^{2} \cancel{(x-y)}}=\dfrac{x^2+y^2}{(x+y)^2} 
%   \textcon x-y \neq 0 \sRarrow x \neq y \texte x \neq 0\]
  \stext{quando} x-y \neq 0 \sRarrow x \neq y \stext{e} x \neq 0\]
\end{itemize*}
 \end{esempio}
% \end{exrig}

% \ovalbox{\risolvii \ref{ese:19.5},\ref{ese:19.6}, \ref{ese:19.7}, 
% \ref{ese:19.8}, \ref{ese:19.9}, \ref{ese:19.10}, \ref{ese:19.11}, 
% \ref{ese:19.12}}


\section{Moltiplicazione di frazioni algebriche}
\label{sec:frazalg_moltiplicazione}

Il prodotto di due frazioni è una frazione avente per numeratore il prodotto 
dei numeratori e per denominatore il prodotto dei denominatori.

% Si vuole determinare il prodotto~\(p=\dfrac{7}{15}\cdot \dfrac{20}{21}\) 
% possiamo 
% % scrivere prima il risultato dei prodotti dei numeratori e dei 
% % denominatori e poi ridurre ai minimi termini la frazione
% % ottenuta:~\(p=\dfrac{7}{15}\cdot \dfrac{20}{21}=
% % \dfrac{\cancel{140}^4}{\cancel{315}^9}=\dfrac{4}{9}\),
% % oppure 
% prima operare la semplificazione incrociata e poi
% moltiplicare:~\(p=\dfrac{7}{15}\cdot \dfrac{20}{21}=
% \dfrac{\cancel{7}^1}{\cancel{15}^3}\cdot \dfrac{\cancel{20}^4}{\cancel{21}
% ^3} = \dfrac{4}{9}\).

% \begin{exrig}
 \begin{esempio}
Calcoliamo il prodotto delle frazioni algebriche: ~~ 
\(f_{1}=-\dfrac{3a^{2}}{10b^{3}c^{4}}\) ~~e~~ 
\(f_{2}=\dfrac{25bc^{7}}{9}\)

Il prodotto si ottiene semplificando e ponendo le condizioni per ogni fattore 
semplificato: 
\[f_1 \cdot f_2 =
-\dfrac{\cancel{3}a^{2}}{\cancel{10}b^{\cancel{3}}\cancel{c^{4}}}
\cdot \dfrac{\cancel{25}\cancel{b}c^{\cancel{7}}}{\cancel{9}}=
-{\dfrac{5a^{2}c^{3}}{6b^{2}}} \stext{quando} b \neq 0 \stext{e} c \neq 0\]
 \end{esempio}

 \begin{esempio}
Calcoliamo il prodotto delle frazioni algebriche: ~~
\(f_{1}=-{\dfrac{3a}{2b+1}}\) ~~e~~ \(f_{2}=\dfrac{10b}{a-3}\)

In questo caso non è possibile alcuna semplificazione, il prodotto si ottiene 
moltiplicando fra loro i numeratori e i denominatori:
\[f_1 \cdot f_2 = 
-{\dfrac{3a}{2b+1}}\cdot\dfrac{10b}{a-3}=
-{\dfrac{30ab}{(2b+1)(a-3)}}\] 
\osservazione
Come detto sopra, ``semplificazioni'' di questo tipo:~~
\(f=-{\dfrac{3\cancel{a}}{2\cancel{b}+1}}\cdot
      \dfrac{10\cancel{b}}{\cancel{a}-3}\), 
sono proprio sbagliate!
 \end{esempio}

 Quando al numeratore e al denominatore abbiamo polinomi, per prima cosa 
dobbiamo scomporli in fattori in modo da poter semplificare le frazioni.
 
 \begin{esempio}
Calcoliamo il prodotto delle frazioni: ~~
\(f_{1}=\dfrac{-x+2x^2}{x^2-3x+2}\) ~~e~~ \(f_{2}=\dfrac{5x-5}{x-4x^2+4x^3}\)

\begin{enumerate}
 \item Scomponiamo in fattori tutti i denominatori (servirà per la 
 determinazione delle~\(\CE\)) e
    tutti i numeratori (servirà per le eventuali semplificazioni),
  \begin{itemize*}
  \item \(f_1=\dfrac{2x^2-x}{x^2-3x+2}=
              \dfrac{x \cdot (2x-1)}{(x-1) \cdot (x-2)}\)
  \item \(f_2=\dfrac{5x-5}{4x^3-4x^2+x}=
              \dfrac{5 \cdot (x-1)}{x \cdot \tonda{4x^2-4x+1}}=
              \dfrac{5 \cdot (x-1)}{x \cdot (2x-1)^2}\)
  \end{itemize*}
 \item determiniamo la frazione prodotto, effettuando le eventuali 
  semplificazioni:
\[f_1 \cdot f_2 = 
\dfrac{\cancel{x} \cdot \cancel{(2x-1)}}{\cancel{(x-1)} \cdot {(x-2)}} \cdot 
\dfrac{5\cdot\cancel{(x-1)}}{\cancel{x}\cdot(2x-1)^{\cancel{2}}}=
\dfrac{5}{(x-2)(2x-1)}
\stext{quando} x \neq 0 \stext{e} x \neq +1 \stext{e} x \neq +\dfrac{1}{2}\]
%  \item esplicitiamo le~\(\CE\) relative ai fattori che sono stati semplificati:
% \[\CE \sistema{x \neq 0 \\
%       x-1 \neq 0 \\
%       2x-1 \neq 0}
% \sRarrow
% \sistema{x \neq 0 \\
%          x \neq +1 \\
%          x \neq +\dfrac{1}{2}}
%       \]

% \[\CE x \neq 0 \stext{e} 
%       x-1 \neq 0 \sRarrow x\neq +1 \stext{e} 
%       2x-1 \neq 0 \sRarrow x \neq +\dfrac{1}{2}\]
\end{enumerate}
\end{esempio}
% \end{exrig}

% \ovalbox{\risolvii \ref{ese:19.13}, \ref{ese:19.14}, \ref{ese:19.15}, 
% \ref{ese:19.16}, \ref{ese:19.17}}

\section{Divisione di frazioni algebriche}
\label{sec:frazalg_divisione}

L'introduzione delle frazioni algebriche permette di trasformare ogni 
divisione in una moltiplicazione quindi si può sempre eseguire la divisione, 
a patto che il divisore sia diverso da zero.
L'insieme delle frazioni algebriche è chiuso anche per la divisione.

Il quoziente di due frazioni è la frazione che si ottiene moltiplicando la 
prima con il reciproco della seconda.
Lo schema di calcolo può essere illustrato nel modo seguente, come del resto 
abbiamo visto nell'insieme dei numeri razionali:

\[\dfrac{m}{n}:\dfrac{p}{q}=\dfrac{m}{n} \cdot \dfrac{q}{p}=
  \dfrac{m \cdot q}{n \cdot p}\]

% Per determinare il quoziente di due frazioni 
% numeriche si procede nel seguente modo:
% \[\dfrac{5}{12}:\dfrac{7}{4}=
%     \dfrac{5}{\cancel{12}^3}\cdot\dfrac{\cancel{4}^1}{7}=\dfrac{5}{21}\]

Quando eseguiamo il reciproco di una frazione algebrica perdiamo 
un'informazione, può darsi che il reciproco di una frazione si possa 
calcolare per determinati valori delle variabili, ma che non si possa 
calcolare la frazione di partenza:
\(\dfrac{5}{0}\) non si può calcolare, ma il suo reciproco sì: 
\(\dfrac{0}{5}\).
Quindi prima di operare il reciproco di una frazione dobbiamo porre le
\(\CE\) di quella frazione:

\begin{esempio}
Calcoliamo il reciproco di \(f=\dfrac{2x^2-12x+18}{3x^2-75}\).
\begin{enumerate}
\item Scomponiamo in fattori i polinomi
\begin{itemize*}
 \item \(f=\dfrac{2x^2-12x+18}{3x^2-75}=
           \dfrac{2 \tonda{x^2-6x+9}}{3 \tonda{x^2-25}}=
           \dfrac{2\cdot \tonda{x-3}^2}{3\tonda{x-5} \cdot \tonda{x+5}}\)
\end{itemize*}
\item Condizione di esistenza della frazione di partenza: \quad
\(x \neq -5 \stext{e} x \neq +5\)
\end{enumerate}
In un'unica espressione:
\[\stext{Il reciproco di} \dfrac{2x^2-12x+18}{x^2-25} =
  \dfrac{2\cdot \tonda{x-3}^2}{3\tonda{x-5} \cdot \tonda{x+5}}
\stext{è}
\dfrac{\tonda{x-5} \cdot \tonda{x+5}}{2\cdot \tonda{x-3}^2} \stext{quando}
x \neq \mp 5\]

\end{esempio}


% \begin{exrig}
 \begin{esempio}
Calcoliamo il quoziente delle frazioni algebriche
\(f_{1}=\dfrac{3a-3b}{2a^{2}}\) e~\(f_{2}=\dfrac{a^{2}-ab}{b^{2}}\).

\begin{enumerate}
\item Esplicitiamo la condizione di esistenza del divisore: \(b \neq 0\)
\item Scomponiamo in fattori i polinomi
\begin{itemize*}
 \item \(f_{1}=\dfrac{3a-3b}{2a^{2}}=\dfrac{3\cdot(a-b)}{2a^{2}}\)
 \item \(f_{2}=\dfrac{a^{2}-ab}{b^{2}}=\dfrac{a\cdot (a-b)}{b^{2}};\)
\end{itemize*}
% \item Cambiamo la divisione in una moltiplicazione e semplifichiamo:
% \[
% \dfrac{3\cdot(a-b)}{2a^{2}}:\dfrac{a\cdot(a-b)}{b^2}=
% \dfrac{3\cdot\cancel{(a-b)}}{2a^{2}}\cdot
% \dfrac{b^2}{a\cdot\cancel{(a-b)}}=\dfrac{3b^2}{2a^3}
% \]
%  \item Esplicitiamo tutte le condizioni d'esistenza:
% \(\sistema{b\neq~0\\a \neq b}\)
\end{enumerate}
In un'unica espressione:
\[f_1 : f_2 = 
  \dfrac{3\cdot(a-b)}{2a^{2}}:\dfrac{a\cdot(a-b)}{b^2}=
  \dfrac{3\cdot\cancel{(a-b)}}{2a^{2}}\cdot
  \dfrac{b^2}{a\cdot\cancel{(a-b)}}=\dfrac{3b^2}{2a^3} \stext{quando}
  b \neq 0 \stext{e} a \neq b\]
 \end{esempio}
% \end{exrig}

% \ovalbox{\risolvii \ref{ese:19.19}, \ref{ese:19.20}, \ref{ese:19.21}}

\section{Potenza di una frazione algebrica}
\label{sec:frazalg_potenza}

La potenza di esponente~\(n\), naturale diverso da zero, della frazione 
algebrica~\(\dfrac{A}{B}\) con~\(B{\neq}0\) (\(\CE\)) è la frazione
avente per numeratore la potenza di esponente~\(n\) del numeratore e per 
denominatore la potenza di esponente~\(n\)
del denominatore:~\( \tonda{\dfrac{A}{B}} ^{n}=\dfrac{A^{n}}{B^{n}}\).

% \begin{exrig}
 \begin{esempio}
Calcoliamo  \( \tonda{\dfrac{x-2}{x^{2}-1}} ^{3}\)
\[\tonda{\dfrac{x-2}{x^{2}-1}} ^{3} =
\dfrac{\tonda{x-2}^3}{\quadra{(x-1)\cdot(x+1)}^3} = 
  \dfrac{\tonda{x-2}^3}{\tonda{x-1}^3 \cdot \tonda{x+1}^3}\]
 \end{esempio}
% \end{exrig}

\subsection{Casi particolari dell'esponente}

\subsubsection{Esponente nullo}
Se~\(n = 0\) sappiamo che qualsiasi numero \emph{diverso da zero}, elevato a 
zero è uguale a~\(1\). 
Lo stesso si può dire se la base è una frazione algebrica.
\(\tonda{\dfrac{A}{B}}^0=1\) se la frazione è definita: 
\(B\neq~0\) e è diversa da zero: \(A\neq~0\).
% \begin{exrig}
 \begin{esempio}
Quali valori può avere la variabile~\(a\) perché sia: 
\( \tonda{\dfrac{3a-2}{5a^{2}+10a}} ^{0}=1\)?
\begin{itemize*}
 \item Scomponiamo in fattori numeratore e denominatore della 
  frazione:~\( \tonda{\dfrac{3a-2}{5a\cdot (a+2)}} ^{0}\)
 \item determiniamo le~\(\CE\) della frazione:
 \(a\neq~0 \stext{e} a+2\neq~0\) 
  da cui, \(\CE a\neq~0 \stext{e} a\neq -2\).
\item Una frazione è nulla se è definita e il suo denominatore è nullo, 
quindi:
\(3a-2 \neq 0 \sRarrow a\neq \dfrac{2}{3}\)
\end{itemize*}
In definitiva: \(a\neq -2 \stext{e} a\neq~0 \stext{e} a\neq \dfrac{2}{3}\).
 \end{esempio}
% \end{exrig}

\subsubsection{Esponente negativo}
Se~\(n\) è intero negativo la potenza con base diversa da zero è uguale alla 
potenza che ha per base l'inverso della base e per esponente l'opposto
dell'esponente. 
\( \tonda{\dfrac{A}{B}} ^{-n}=
  \tonda{\dfrac{B}{A}} ^{+n}\) con~\(A\neq~0\) e~\(B\neq~0\).

Quindi deve esistere sia la frazione di partenza, sia il suo reciproco, cioè 
dovranno essere diversi da zero sia il denominatore sia il numeratore.

% \begin{exrig}
 \begin{esempio}
Determinare~\(f = \tonda{\dfrac{x^{2}+5x+6}{x^{3}+x}} ^{-2}\).
\begin{itemize*}
 \item Scomponiamo in fattori numeratore e 
  denominatore:~\( \tonda{\dfrac{ \tonda{x+2} \cdot  \tonda{x+3} }
                            {x\cdot  \tonda{x^{2}+1} }} ^{-2}\)
 \item \(\CE\) della frazione: \(x\neq~0 \stext{e} x^2+1\neq~0\) da cui
\(x \neq 0\) essendo l'altro fattore sempre positivo.
 \item \(\CE\) del reciproco: \(x+2 \neq 0 \stext{e} x+3 \neq 0\) da cui 
\(x \neq -2 \stext{e} x \neq -3\)
\end{itemize*}
Quindi:
\(\tonda{\dfrac{(x+2)\cdot (x+3)}{x\cdot  \tonda{x^2+1} }}^{-2}=
  \dfrac{x^2 \cdot \tonda{x^2+1}^2}{\tonda{x+2}^{2} \cdot \tonda{x+3}^2}\),
\quad se~\(x \neq~0;~x \neq -2 \stext{e} x\neq-3\).
 \end{esempio}
% \end{exrig}

% \ovalbox{\risolvi \ref{ese:19.18}}

\section{Addizione di frazioni algebriche}
\label{sec:frazalg_addizione}

% \subsection{Proprietà della addizione tra frazioni algebriche}
% Nell'insieme delle frazioni algebriche la somma:
% \begin{itemize*}
% \item è commutativa:~\(f_1+ f_2 = f_2 + f_1\)
% \item è associativa:~\((f_1+ f_2) + f_3 = f_1 + (f_2 + f_3) = f_1 + f_2 + 
% f_3\)
% \item possiede l'elemento neutro, cioè esiste una frazione~\(F^0\) tale che 
% per 
%  qualunque frazione~\(f\) si abbia~\(F^{0} + f = f + F^0= f\) e~\(F^0 = 0\)
% \item ogni frazione algebrica~\(f\), possiede la frazione opposta~\((-f )\) 
%  tale che
% \[(-f) + f = f + (- f) = F^0 = 0.\]
% \end{itemize*}
% Quest'ultima proprietà ci permette di trattare contemporaneamente 
% l'operazione 
% di addizione e di sottrazione, perché ogni sottrazione può essere 
% trasformata 
% nell'addizione tra il primo termine e l'opposto del secondo.
% Come per i numeri relativi, quando si parlerà di somma di frazioni si 
% intenderà ``somma algebrica''.

Come per le frazioni numeriche, anche nelle frazioni algebriche l'addizione
è l'operazione più complicata. Infatti per addizionare due frazioni 
algebriche bisogna:

\begin{procedura}
 Per addizionare frazioni algebriche:
\begin{enumeratea}
\item scomporre in fattori i denominatori;
\item determinare il \(\mcm\);
\item riscrivere le frazioni in modo che abbiano lo stesso denominatore;
\item addizionare i numeratori;
\item semplificare il risultato ottenuto ponendo le condizioni di esistenza.
\end{enumeratea}
\end{procedura}

% \begin{exrig}
 \begin{esempio}
\(\dfrac{x+2}{x^{2}-2x}-\dfrac{x-2}{2x+x^{2}}+\dfrac{-4x}{x^{2}-4}\)
\begin{enumeratea}
 \item \emph{scomporre in fattori i denominatori}
 
  \[\dfrac{x+2}{x(x-2)}-\dfrac{x-2}{x(2+x)}+\dfrac{-4x}{(x+2)(x-2)}\]
  
 \item \emph{determinare il \(\mcm\)}
 
  \[\mcm = x\cdot (x+2) \cdot (x-2)\]
  
%  \item poniamo le~\(\CE\) \(x(x+2)(x-2){\neq}0\) da cui~\(\CE\) \(x{\neq}
% 0\), 
%   \({x\neq}2\) e~\({x\neq}-2\)

 \item \emph{riscrivere le frazioni in modo che abbiano lo stesso
 denominatore}

dividiamo il~\(\mcm\) per ciascun denominatore e moltiplichiamo il 
quoziente ottenuto per il relativo numeratore:
\[\dfrac{(x+2)^{2}-(x-2)^{2}-4x^{2}}{x\cdot(x+2)\cdot(x-2)}\]
 \item \emph{addizionare i numeratori} eseguiamo le operazioni al numeratore:
    \begin{equation*}
     \dfrac{x^{2}+4x+4-x^{2}+4x-4-4x^{2}}{x\cdot(x+2)\cdot(x-2)}=
     \dfrac{8x-4x^{2}}{x\cdot(x+2)\cdot(x-2)}
     \end{equation*}
 \item \emph{semplificare il risultato ottenuto ponendo le condizioni di 
  esistenza}
 
  per semplificare la frazione dobbiamo scomporre il numeratore:
    \begin{equation*}
    \dfrac{-4\cancel{x}\cdot\cancel{(x-2)}}
         {\cancel{x}\cdot(x+2)\cdot\cancel{(x-2)}}=
    \dfrac{-4}{x+2} \quad  \stext{e} \quad x \neq 0 \quad  \stext{e} \quad x 
\neq 2.
    \end{equation*}
\end{enumeratea}
 \end{esempio}

\newpage % ------------------------------------------------

 \begin{esempio}
\(\dfrac{x}{x-2}-\dfrac{2x}{x+1}+
  \dfrac{x}{x-1}-\dfrac{5x^{2}-7}{x^{3}-2x^{2}+2-x}\)
\begin{enumeratea}
 \item Scomponiamo in fattori~\(x^{3}-2x^{2}+2-x\), 
  essendo gli altri denominatori irriducibili: 
  \(x^{3}-2x^{2}+2-x=x^2(x-2)-1(x-2)=(x-2) \tonda{x^2-1} =
   (x-2)(x+1)(x-1)\)% che è anche il~\(\mcm\) dei denominatori;

 \item Calcoliamo il \(\mcm\) dei denominatori: 
  \(\mcm=(x-2)(x+1)(x-1)\);
  
%  \item poniamo le~\(\CE\) \((x-2)(x+1)(x-1){\neq}0\) da cui~\(\CE\) \
% (x{\neq}2\), \({x\neq}-1\) e~\({x\neq}1\)
 \item dividiamo il~\(\mcm\) per ciascun denominatore e moltiplichiamo il 
  quoziente ottenuto per il relativo numeratore:
 \[=\dfrac{x(x+1)(x-1)-2x(x-2)(x-1)+x(x-2)(x+1)-(5x^{2}-7)}
         {(x-2)(x+1)(x-1)}=\]
 \item eseguiamo le operazioni al numeratore:
\[=\dfrac{x(x^2-1)-2x(x^2-3x+2)+x(x^2-x-2)-(5x^{2}-7)}{(x-2)(x+1)(x-1)} = \]
\[=\dfrac{\cancel{x^3}-x~\cancel{-2x^3}~\cancel{+6x^2}-4x~\cancel{+x^3}
          ~\cancel{-x^2}-2x~\cancel{-5x^2}+7}
         {(x-2)(x+1)(x-1)}=\]
\[=\dfrac{-x-4x-2x+7}{(x-2)(x+1)(x-1)}=\dfrac{-7x+7}{(x-2)(x+1)(x-1)}=\]
 \item scomponiamo il numeratore e semplifichiamo la frazione ottenuta, 
  ponendo le condizioni. La frazione somma è:
\[=\dfrac{-7\cancel{(x-1)}}{(x-2)(x+1)\cancel{(x-1)}}=
   -{\dfrac{7}{(x-2)(x+1)}} \quad  \stext{quando} \quad x \neq +1\]
 \end{enumeratea}
 \end{esempio}

% \end{exrig}
% 
% A volte la situazione è più semplice:
% 
% \begin{itemize*}
%  \item se i denominatori sono dei monomi, sono già scomposti in fattori;
%  \item a volte le frazioni hanno già lo stesso denominatore.
% \end{itemize*}
% 
%  In questi casi si può saltare una parte della procedura.
% 
% % \begin{exrig}
%  \begin{esempio}
% \(\dfrac{x+y}{3x^{2}y}-\dfrac{2y-x}{2xy^{3}}\).
% 
% Dobbiamo trasformare le frazioni in modo che abbiano lo stesso 
% denominatore:
% \begin{itemize*}
%  \item calcoliamo il~\(\mcm(3x^{2}y, 2xy^{3}) = 6x^{2}y^{3}\)
%  \item poniamo le~\(\CE\) \(6x^{2}y^{3}{\neq}0\) da cui~\(\CE x{\neq}0\) 
% e~\(y{\neq}0\)
%  \item dividiamo il~\(\mcm\) per ciascun denominatore e moltiplichiamo il 
%  quoziente ottenuto per il relativo numeratore:
%  \[\dfrac{2y^{2}\cdot(x+y)}{6x^{2}y^{3}}-\dfrac{3x\cdot(2y-x)}{6x^{2}y^{3}}\]
%  \item la frazione somma ha come denominatore lo stesso denominatore e come 
%  numeratore la somma dei numeratori:
%     \begin{equation*}
%     \dfrac{2y^{2}\cdot(x+y)}{6x^{2}y^{3}}-\dfrac{3x\cdot(2y-x)}{6x^{2}y^{3}}=
%     \dfrac{2xy^{2}+2y^{3}+2x^{2}y-6xy+3x^{2}}{6x^{2}y^{3}}.
%     \end{equation*}
% \end{itemize*}
%  \end{esempio}
% 
%  \begin{esempio}
% \(\dfrac{2x-3y}{x+y}+\dfrac{x+2y}{x+y}\)
% 
% Poniamo le~\(\CE\) \(x + y{\neq}0\) da cui~\(\CE x{\neq}-y\) allora
% \begin{equation*}
% 
% \dfrac{2x-3y}{x+y}+\dfrac{x+2y}{x+y}=\dfrac{(2x-3y)+(x+2y)}{x+y}=\dfrac{3x-y}
% {x+y}.
% \end{equation*}
%  \end{esempio}
%  
% % \osservazione A questo caso ci si può sempre ricondurre trasformando le 
% % frazioni allo stesso denominatore. 
% % Si potrebbe scegliere un qualunque denominatore comune, ad esempio il 
% prodotto 
% % di tutti i denominatori ma, scegliamo il~\(\mcm\) dei denominatori delle 
% % frazioni addendi per semplificare i calcoli.
% 
% % \end{exrig}

% \ovalbox{\risolvii \ref{ese:19.22}, \ref{ese:19.23}, \ref{ese:19.24}, 
% \ref{ese:19.25}, \ref{ese:19.26}, \ref{ese:19.27}, \ref{ese:19.28}, 
% \ref{ese:19.29}, \ref{ese:19.30}}
