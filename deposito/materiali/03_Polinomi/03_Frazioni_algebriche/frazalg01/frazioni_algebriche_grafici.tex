% (c) 2017 Daniele Zambelli - daniele.zambelli@gmail.com
% 
% Tutti i grafici per il capitolo relativo alle frazioni algebriche
%

\newcommand{\graficofrazionea}{% 
  \def \funzione{(x**2-3*x-4)/(2*x**2+4*x-16)}
  \disegno[4]{
    \rcom{-6}{+6}{-6}{+6}{gray!50, very thin, step=1}
    \tkzInit[xmin=-6.3,xmax=+6.3,ymin=-6.3,ymax=+6.3]
    \tkzFct[domain=-6.3:-4, ultra thick, color=Maroon!50!black]
           {\funzione}
    \tkzFct[domain=-4:+2, ultra thick, color=Maroon!50!black]
           {\funzione}
    \tkzFct[domain=+2:+6.3, ultra thick, color=Maroon!50!black]
           {\funzione}
  }
}

\newcommand{\funzioneb}{% 
  \def \funzione{((x-3)*(x-1))/((x-3)*(x+2))}
  \rcom{-6}{+6}{-6}{+6}{gray!50, very thin, step=1}
  \tkzInit[xmin=-6.3,xmax=+6.3,ymin=-6.3,ymax=+6.3]
  \tkzFct[domain=-6.3:-2, ultra thick, color=Maroon!50!black]
         {\funzione}
  \tkzFct[domain=-2:+6.3, ultra thick, color=Maroon!50!black]
         {\funzione}
}

\newcommand{\graficofrazioneb}{% 
  \disegno[4]{\funzioneb
  \draw (3, .4) circle (2pt) [fill=white];
  \node at (0, -7.3) {\(f_1(x)=\dfrac{(x-3)(x-1)}{(x-3)(x+2)}\)};
  }
}

\newcommand{\graficofrazionec}{% 
  \disegno[4]{\funzioneb
  
  \node at (0, -7.3) {\(f_2(x)=\dfrac{(x-1)}{(x+2)}\)};
  }
}

\begin{comment}

\newcommand{\nomefunzione}{% 
  \disegno{
    }
}

\end{comment}

