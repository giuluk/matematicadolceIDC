% (c)~2014 Claudio Carboncini - claudio.carboncini@gmail.com
% (c)~2014 Dimitrios Vrettos - d.vrettos@gmail.com
% (c) 2014 Daniele Zambelli - daniele.zambelli@gmail.com

\section{Esercizi}

\subsection{Esercizi dei singoli paragrafi}
% \subsubsection*{2.1 - Radici}
\subsubsection*{\numnameref{sec:radici_definizioni}}

% % Radici quadrate, cubiche, n-esime

\begin{esercizio}
 \label{ese:2.01}
Determina le seguenti radici quadrate razionali (quando è possibile calcolarle).
\begin{multicols}{4}
 \begin{enumeratea}
 \item~$\sqrt 9$
 \item~$\sqrt{36}$
 \item~$\sqrt{-49}$
 \item~$\sqrt{\frac{16}{25}}$
 \item~$\sqrt{64}$
 \item~$\sqrt{-81}$
 \item~$\sqrt{0,09}$
 \item~$\sqrt{\frac{49}{81}}$
 \item~$\sqrt{0,04}$
 \item~$\sqrt{0,0001}$
 \item~$\sqrt{625}$
 \item~$\sqrt{\frac{121}{100}}$
 \item~$\sqrt{-4}$
 \item~$\sqrt{144}$
 \item~$\sqrt{0,16}$
 \item~$\sqrt{\frac{144}{36}}$
 \end{enumeratea}
 \end{multicols}
\begin{flushright}
\vspace*{-8pt}
 $[3 \quad 6 \quad nan \quad 9 \quad nan \quad 
 \frac{4}{5} \quad \frac{7}{9} \quad \frac{11}{10} \quad 2 \quad 
 nan \quad 0,2 \quad 0,3 \quad 0.1 \quad 25 \quad 0,4 \quad 12]$
\end{flushright}
\end{esercizio}

\begin{esercizio}
 \label{ese:2.02}
Senza usare la calcolatrice determina il valore delle seguenti radici 
quadrate razionali.
\begin{multicols}{2}
 \begin{enumeratea}
 \item~$\sqrt{-0,09}$;
 \item~$\sqrt{25\cdot 16}$;
 \item~$\sqrt{36\cdot 49}$;
 \item~$\sqrt{0,04\cdot 0,0121}$;
 \item~$\sqrt{13+\sqrt{7+\sqrt{1+\sqrt{6+\sqrt 9}}}}$;
 \item~$\sqrt{5+\sqrt{14+\sqrt{2+\sqrt 4}}}$.
 \end{enumeratea}
 \end{multicols}
\begin{flushright}
\vspace*{-8pt}
$[nan \quad 20 \quad 42 \quad 0,22 \quad 4 \quad 3]$
\end{flushright}
\end{esercizio}

\begin{esercizio}
 \label{ese:2.03}
Con l'aiuto della calcolatrice determina il valore delle seguenti radici 
quadrate approssimato a $10^{-4}$ (presta attenzione all'approssimazione:
 \begin{multicols}{4}
 \begin{enumeratea}
 \item $\sqrt 3$
 \item $\sqrt 5$
 \item $\sqrt 7$
 \item $\sqrt{\frac 1 2}$
 \item $\sqrt[3]{25}$
 \item $\sqrt{11}$
 \item $\sqrt[3]3$
 \item $\sqrt{\frac{17} 4}$
 \item $\sqrt 69$
 \item $\sqrt[3]4$
 \item $\sqrt[3]7$
 \item $\sqrt[3]{\frac{7}{3}}$
 \item $\sqrt[3]{100}$
 \item $\sqrt{96}$
 \item $\sqrt[3]{250}$
 \item $\sqrt{\frac{1}{10}}$
 \end{enumeratea}
 \end{multicols}
\begin{flushright}
\vspace*{-8pt}
$[1,7321 \quad 2,6458 \quad 3,3166 \quad 2,2361 \quad 0,7071 \quad 
  2,0616 \quad 0,3162 \quad 1,3264]$
  
$[1,4422 \quad 1,5874 \quad 1,9129 \quad
  4,6416 \quad 2,9240 \quad 6,2996 \quad 8,3066 \quad 9,7980]$
\end{flushright}
\end{esercizio}

% \begin{esercizio}
%  \label{ese:2.04}
% Estrai le seguenti radici di espressioni letterali, facendo attenzione al 
% valore assoluto.
% \begin{multicols}{3}
% \begin{enumeratea}
%  \item $\sqrt{a^2+2a+1}$;
%  \item $\sqrt{4x^2+8x+4}$;
%  \item $\sqrt{9-12a+4a^2}$.
% \end{enumeratea}
% \end{multicols}
% \end{esercizio}

\begin{esercizio}[\Ast]
 \label{ese:2.06}
Determina le seguenti radici se esistono.
 \begin{multicols}{3}
 \begin{enumeratea}
 \item $\sqrt[3]{27}$
 \item $\sqrt[3]{64}$
 \item $\sqrt[3]{\frac 8{27}}$
 \item $\sqrt[3]{-1}$
 \item $\sqrt[3]{1000}$
 \item $\sqrt[3]{-\frac{64}{125}}$
 \item $\sqrt[3]{125}$
 \item $\sqrt[3]{-216}$
 \item $\sqrt[3]{\frac{1000}{27}}$
 \end{enumeratea}
 \end{multicols}
\begin{flushright}
\vspace*{-8pt}
$[3 \quad 4 \quad -1 \quad 10 \quad 5 \quad -6 \quad \frac{2}{3} \quad
  -\frac{4}{5} \quad \frac{10}{3}]$
\end{flushright}
\end{esercizio}

% \begin{esercizio}[\Ast]
%  \label{ese:2.07}
% Determina le seguenti radici se esistono.
%  \begin{multicols}{2}
%  \begin{enumeratea}
%  \item $\sqrt[3]{0,001}$;
%  \item $\sqrt[3]{\frac 1 8}$;
%  \item $\sqrt[3]{-0,008}$;
%  \item $\sqrt[3]{4+\sqrt[3]{61+\sqrt[3]{25+\sqrt[3]8}}}$;
%  \item $\sqrt[3]{25+\sqrt[3]{3+\sqrt[3]{122+\sqrt[3]{27}}}}$;
%  \item $\sqrt[3]{27\cdot \sqrt{64}}$;
%  \item $\sqrt[9]0$;
%  \item $\sqrt[8]{-1}$;
%  \item $\sqrt[5]{-100000}$.
%  \end{enumeratea}
%  \end{multicols}
% \end{esercizio}

% \looseness=-1
\begin{esercizio}
 \label{ese:2.08}
Senza usare la calcolatrice determina il valore delle seguenti radici.
 \begin{multicols}{3}
 \begin{enumeratea}
 \item $\sqrt[4]{0,0001}$
 \item $\sqrt[4]{81}$
 \item $\sqrt[5]{\frac{32}{243}}$
 \item $\sqrt[6]{64}$
 \item $\sqrt[4]{-4}$
 \item $\sqrt[5]{34-\sqrt[4]{14+\sqrt{2+\sqrt[3]8}}}$
 \item $\sqrt[10]0$
 \item $\sqrt[4]{0,0081}$
 \item $\sqrt{20+\sqrt[3]{121+\sqrt[4]{256}}}$
 \end{enumeratea}
 \end{multicols}
% \looseness=-1
% \nopagebreak[4]
% \enlargethispage*{20 \baselineskip}
% \enlargethispage{20 \baselineskip}
% \vspace*{-40pt}
\begin{flushright}
\vspace*{-8pt}
$[2 \quad 3 \quad nan \quad 0,03 \quad \frac{2}{3} \quad 2 \quad 0 
\quad 0,1 \quad 5]$
\end{flushright}
\end{esercizio}

\begin{esercizio}[\Ast]
\label{ese:2.09}
Senza usare la calcolatrice determina il valore delle seguenti radici.
 \begin{multicols}{3}
 \begin{enumeratea}
 \item $\sqrt{21+\sqrt{16}}$
 \item $\sqrt[5]{31+\sqrt[4]1}$
 \item $\sqrt[5]{240+\sqrt 9}$
 \item $\sqrt{\sqrt{0,16}}$
 \item $\sqrt[5]{32\cdot 10^{-5}}$
 \item $\sqrt{3 (37-4\sqrt{81})\cdot 27}$
 \item $\sqrt{72+\sqrt{80+\sqrt 1}}$
 \item $\sqrt{24336}$
 \item $\sqrt[4]{620+\sqrt[4]{625}}$
 \end{enumeratea}
 \end{multicols}
\begin{flushright}
\vspace*{-8pt}
$[3 \quad 5 \quad 45 \quad 2 \quad 0,4 \quad 5 \quad 156 \quad 9 \quad 0,1]$
\end{flushright}
\end{esercizio}

% \begin{esercizio}[\Ast]
%  \label{ese:2.10}
% Determina le seguenti radici se esistono.
%  \begin{multicols}{3}
%  \begin{enumeratea}
%  \item $\sqrt{\frac{25a^4} 9}$
%  \item $\sqrt[5]{243}$
%  \item $\sqrt[4]{600+\sqrt{25}\cdot \sqrt{25}}$
%  \item $\sqrt[3]{8a^3+12a^2+6a+1}$
%  \item $\sqrt[3]{a^6+9a^4+27a^2+27}$
%  \item $\sqrt[3]{1-6x+12x^2-8x^3}$
%  \end{enumeratea}
%  \end{multicols}
% \end{esercizio}

% \paragraph{\ref{ese:2.06}}
% b)~$4$,\quad h)~$-\frac{4}{5}$,\quad i)~$\frac{10}{3}$
% 
% % \paragraph{\ref{ese:2.07}}
% % e)~$3$,\quad h)~$\emptyset$.
% 
% \paragraph{\ref{ese:2.08}}
% b)~$3$,\quad d)~$\frac{2}{3}$,\quad h)~$2$
% 
% \paragraph{\ref{ese:2.09}}
% c)~$3$,\quad e)~$0,2$,\quad i)~$5$

% \paragraph{\ref{ese:2.10}}
% d)~$2a+1$,\quad e)~$a^2+3$,\quad f)~$1-2x$.

% \subsubsection*{2.2 - Condizioni di esistenza}
\subsubsection*{\numnameref{sec:radici_condizioni_esistenza}}

\begin{esercizio}
 \label{ese:2.11}
Determina le condizioni di esistenza dei seguenti radicali.
 \begin{multicols}{4}
 \begin{enumeratea}
 \item $\sqrt{x+1}$
 \item $\sqrt[3]{1-x}$
 \item $\sqrt{x{x+1}}$
 \item $\sqrt[3]{3x^2y}$
 \item $\sqrt{3xy}$
 \item $\sqrt[6]{-2x^2y^2}$
 \item $\sqrt[4]{x^2(x-1)}$
 \item $\sqrt[5]{x^3(x^2-x)}$
%  \item $\sqrt{\dfrac{4-x}{x-3}}$
 \end{enumeratea}
 \end{multicols}
\vspace*{-8pt}
$[\forall x\in \insR,\quad x\le 1,\quad x>-1,\quad y\ge 0,\quad x>1]$
\end{esercizio}

\begin{esercizio}[\Ast]
 \label{ese:2.12}
Determina le condizioni di esistenza dei seguenti radicali.
 \begin{multicols}{3}
 \begin{enumeratea}
 \item $\sqrt{x^2(x+1)}$
 \item $\sqrt[3]{1+a^2}$
 \item $\sqrt[6]{2x-1}$
 \item $\sqrt{1+\valass{x}}$
 \item $\sqrt{(a-1)(a-2)}$
 \item $\sqrt{\valass{x}+1}\cdot \sqrt[3]{x+1}$
 \item $\sqrt{1-x}+2\sqrt{\dfrac 1{x-1}}$
 \item $\sqrt{\dfrac{5-x}{x+2}}$
 \item $\sqrt{\dfrac{2y}{(2y+1)^2}}$
 \item $\sqrt{\dfrac{x-3}{1-x}}$
 \item $\sqrt{\dfrac{a}{(a+1)(a-2)}}$
 \item $\sqrt{\dfrac{1}{(b-2)(b+2)}}$
 \item $\sqrt{\dfrac{(x-1)^2}{(x-3)(x+2)}}$
 \item $\sqrt{\dfrac{4+x^2}{2x}}$
 \item $\sqrt[6]{\dfrac{x-1}{\left|x\right|}}$
% \item $\sqrt[4]{\dfrac{4x^2+4+8x} 9}$
 \item $\sqrt[6]{\dfrac{\left(b^2+1+2b\right)^3}{729b^6}}$
 \item $\sqrt{\dfrac{x(x-1)}{x-4}}$
% \item $\sqrt{\dfrac 1{x^2}+\dfrac 1{y^2}+\dfrac 2{xy}}$
 \item $\sqrt[4]{\dfrac{m+1}{m-1}}$
% \item $\sqrt[3]{x(x+2)^2}$
 \item $\sqrt{\dfrac{1+a}{a^2}}$
 \item $\sqrt{\dfrac{a+2}{a(a-4)}}$
 \item $\sqrt{\dfrac 1{b^2-4}}$
% \item $\sqrt{\dfrac{a^3}{a^2+6a+9}}$
 \end{enumeratea}
 \end{multicols}
\end{esercizio}

% \paragraph{\ref{ese:2.12}}
% a)~$x\ge -1$,\quad d)~$\emptyset$,\quad i)~$-12$.
% 
% \paragraph{\ref{ese:2.13}}
% a)~$-2<x\le 5$,\quad e)~$b<-2\vee b>2$.
% 
% \paragraph{\ref{ese:2.14}}
% b)~$0\le x\le 1\vee x>4$,\quad e)~$-2<a<0\;\vee \;a>4$.
% 
% \begin{esercizio}[\Ast]
%  \label{ese:2.15}
% Determina le condizioni di esistenza dei seguenti radicali.
%  \begin{multicols}{3}
%  \begin{enumeratea}
%  \item $\sqrt{\dfrac{x^2}{x^2+1}}$;
%  \item $\sqrt{\dfrac{x^2-4}{x-2}}$;
%  \item $\sqrt{\dfrac x{x^2+1}}$;
%  \item $\sqrt[3]{\dfrac{x^3}{x^3+1}}$;
%  \item $\sqrt{2x+3}$;
%  \item $\sqrt[3]{a^2-1}$;
%  \item $\sqrt{x(x+1)(x+2)}$;
%  \item $\sqrt{\left|x\right|+1}$;
%  \item $\sqrt{\dfrac x{\left|x+1\right|}}$;
%  \item $\sqrt{\dfrac 1{-x^2-1}}$.
%  \end{enumeratea}
%  \end{multicols}
% \end{esercizio}

% \paragraph{\ref{ese:2.15}}
% a)~$\forall x\in \insR$,\quad d)~$\forall x\in \insR$,\quad g)~$-2<x<-1 \vee x>0$,\quad i)~$x>0$,\quad f)$\emptyset$.
% 
%\clearpage

% \subsubsection*{2.3 - Potenze a esponente razionale}
\subsubsection*{\numnameref{sec:radici_esp_razionale}}

\begin{esercizio}
 \label{ese:2.16}
Calcola le seguenti potenze con esponente razionale.
 \begin{multicols}{4}
 \begin{enumeratea}
 \item $4^{\frac 3 2}$
 \item $8^{\frac 2 3}$
 \item $9^{-\frac 1 2}$
 \item $16^{\frac 3 4}$
 \item $16^{\frac 5 4}$
 \item $32)^{\frac 4 5}$
 \item $125^{-\frac 2 3}$
 \item $81^{-\frac 3 4}$
 \item $25^{-\frac 3 2}$
 \item $27^{\frac 4 3}$
 \end{enumeratea}
 \end{multicols}
\end{esercizio}

% \begin{esercizio}[\Ast]
%  \label{ese:2.17}
% Calcola le seguenti potenze con esponente razionale.
%  \begin{multicols}{3}
%  \begin{enumeratea}
%  \item $32^{\frac 2 5}$;
%  \item $49^{-\frac 1 2}$;
%  \item $\left(\dfrac 1 4\right)^{-\frac 1 2}$;
%  \item $\left(-\dfrac 1{27}\right)^{-\frac 2 3}$;
%  \item $\left(\dfrac 4 9\right)^{-\frac 5 2}$;
%  \item $\left(0,008\right)^{-\frac 2 3}$;
%  \item $4^{0,5}$;
%  \item $16^{0,25}$;
%  \item $32^{0,2}$;
%  \item $100^{0,5}$.
%  \end{enumeratea}
%  \end{multicols}
% \end{esercizio}

% \paragraph{\ref{ese:2.17}}
% a)~$4$,\quad f)~$25$,\quad i)~$2$.

\begin{esercizio}[\Ast]
 \label{ese:2.18}
Trasforma le seguenti espressioni in forma di potenza con esponente 
frazionario.
 \begin{multicols}{3}
 \begin{enumeratea}
 \item $\sqrt 2$
 \item $\sqrt[3]{8^2}$
 \item $\sqrt[7]{5^3}$
 \item $\sqrt{3^3}$
 \item $\sqrt{\left(\dfrac 1{3^3}\right)}$
 \item $\sqrt[3]{\dfrac 1{3^2}}$
 \item $\sqrt[3]{\dfrac 1{25}}$
 \item $\sqrt[5]{\dfrac{4^2}{3^2}}$
 \end{enumeratea}
 \end{multicols}
\end{esercizio}

% \paragraph{\ref{ese:2.18}}
% c)~$5^{\frac 3 7}$,\quad g)~$25^{-\frac 1 3}$

% \newpage
% 
% \begin{esercizio}[\Ast]
% \label{ese:2.19}
%  Trasforma nella forma radicale le seguenti espressioni.
%  \begin{multicols}{2}
%  \begin{enumeratea}
%  \item $\left(\left(a^2+1\right)^{\frac 2 3}+1\right)^{\frac 1 4}$;
%  \item $\left(1+\left(1+a^{\frac 2 3}\right)^{\frac 1 5}\right)^{\frac 2 3}$.
%  \end{enumeratea}
%  \end{multicols}
% \end{esercizio}

% \paragraph{\ref{ese:2.19}}
% a)~$\sqrt[4]{\sqrt[3]{(a^2+1)^2+1}}$.

\begin{esercizio}
 \label{ese:2.20}
Scrivi in ordine crescente i seguenti numeri:
 \[0,00000001,\quad (0,1)^{10},\quad (0,1)^{0,1},\quad 10^{-10},
 \quad \sqrt{0,0000000001}.\]
\end{esercizio}

% \subsubsection*{2.4 - Semplificazione di radici}
\subsubsection*{\numnameref{sec:radici_semplificazione}}

\begin{esercizio}
 \label{ese:2.21}
Trasforma i seguenti radicali applicando la proprietà invariantiva.
 \begin{multicols}{3}
 \begin{enumeratea}
 \item $\sqrt[4]4=\sqrt[8]{\text{   }}$
 \item $\sqrt[3]9=\sqrt[6]{\quad}$
 \item $\sqrt[5]5=\sqrt[15]{\quad}$
 \item $\sqrt 2=\sqrt[6]{\quad}$
 \item $\sqrt 2=\sqrt[\dots]{16}$
 \item $\sqrt[3]3=\sqrt[\dots]{81}$
 \item $\sqrt[3]{-5}=-\sqrt[\ldots]{25}$
 \item $\sqrt[4]{\frac 3 2}=\sqrt[\ldots]{\frac{27} 8}$
 \item $\sqrt[21]{a^7}=\sqrt[6]{\ldots}, a>0$
 \item $\sqrt[8]{a^{24}}=\sqrt[5]{\ldots}, a>0$
 \item $\sqrt[3]{27}=\frac 1{\sqrt{\ldots}}$
 \item $\sqrt{x^4+2x^2+1}=\sqrt[7]{\ldots}$
 \end{enumeratea}
 \end{multicols}
\end{esercizio}

% \begin{esercizio}[\Ast]
% \label{ese:2.23}
% Semplifica i seguenti radicali.
%  \begin{multicols}{4}
%  \begin{enumeratea}
%  \item $\sqrt[4]{25}$
%  \item $\sqrt[6]8$
%  \item $\sqrt[8]{16}$
%  \item $\sqrt[9]{27}$
%  \item $\sqrt[4]{100}$
%  \item $\sqrt[6]{144}$
%  \item $\sqrt[4]{169}$
%  \item $\sqrt[6]{121}$
% %  \item $\sqrt[6]{125}$.
%  \end{enumeratea}
%  \end{multicols}
% \end{esercizio}

% \paragraph{\ref{ese:2.23}}
% c)~$\sqrt 2$,\quad e)~$\sqrt{10}$,\quad i)~$\sqrt 5$

\begin{esercizio}[\Ast]
 \label{ese:2.24}
Semplifica i seguenti radicali.
 \begin{multicols}{3}
 \begin{enumeratea}
 \item $\sqrt[4]{49}$
  \hfill $\left[...\right]$
 \item $\sqrt[6]{64}$
  \hfill $\left[2\right]$
 \item $\sqrt[12]{16}$
  \hfill $\left[...\right]$
%  \item $\sqrt[6]{\frac{16}{121}}$
%   \hfill $\left[\sqrt[3]{\frac 4{11}}\right]$
 \item $\sqrt[4]{\frac 1{16}}$
  \hfill $\left[...\right]$
 \item $\sqrt[10]{\frac{25}{81}}$
  \hfill $\left[...\right]$
 \item $\sqrt[15]{\frac{64}{27}}$
  \hfill $\left[...\right]$
 \item $\sqrt[9]{-3^3}$
  \hfill $\left[\sqrt[3]{-3}\right]$
 \item $\sqrt[6]{(-2)^4}$
  \hfill $\left[...\right]$
 \item $\sqrt[12]{-4^6}$
  \hfill $\left[\emptyset\right]$
 \item $\sqrt[10]{-32}$
  \hfill $\left[...\right]$
 \item $\sqrt[6]{5^2-4^2}$
  \hfill $\left[...\right]$
 \item $\sqrt[4]{12^2+5^2}$
  \hfill $\left[...\right]$
 \item $\sqrt[10]{3^2+4^2}$
  \hfill $\left[\sqrt[5]5\right]$
 \item $\sqrt[4]{10^2-8^2}$
  \hfill $\left[...\right]$
%  \item $\sqrt[3]{2^6\cdot 5^{15}}$
%   \hfill $\left[12.500\right]$
%  \item $\sqrt[4]{3^4\cdot 4^6}$
%   \hfill $\left[...\right]$
%  \item $\sqrt[5]{5^5\cdot 4^{10}\cdot 2^{15}}$
%   \hfill $\left[...\right]$
%  \item $\sqrt[9]{27\cdot 8\cdot 125}$
%   \hfill $\left[...\right]$
%  \item $\sqrt[4]{625}$
%   \hfill $\left[5\right]$
%  \item $\sqrt[6]{1000}$
%   \hfill $\left[...\right]$
%  \item $\sqrt[4]{2+\frac{17}{16}}$
%   \hfill $\left[...\right]$
 \item $\sqrt[6]{\left(\frac{13} 4+\frac 1 8\right)^4}$
  \hfill $\left[...\right]$
 \item $\sqrt[6]{\left(1+\frac{21} 4\right)^3}$
  \hfill $\left[\frac 9 4\right]$
 \item $\sqrt[16]{(-16)^4}$
  \hfill $\left[...\right]$
%  \item $\sqrt[10]{2^{10}\cdot 3^{20}}$
%   \hfill $\left[2\right]$
%  \item $\sqrt[6]{2^8\cdot 3^6}$
%   \hfill $\left[...\right]$
 \end{enumeratea}
 \end{multicols}
\end{esercizio}

\begin{esercizio}[\Ast]
 \label{ese:2.27}
Semplifica i seguenti radicali.
 \begin{multicols}{2}
 \begin{enumeratea}
 \item $\sqrt[12]{3^6\cdot 4^{12}}$
  \hfill $\left[4\cdot \sqrt 3\right]$
 \item $\sqrt[4]{2^{10}\cdot 3^{15}\cdot 12^5}$
  \hfill $\left[2^5 \cdot 3^5\right]$
 \item $\sqrt[5]{\frac{32a^{10}}{b^{20}}}$
  \hfill $\left[\frac{2 a^2}{b^4}\right]$
 \item $\sqrt[6]{3^9\cdot 8^2}$
  \hfill $\left[2 \cdot 3^\frac{3}{2}\right]$
 \item $\sqrt[4]{9x^2y^4}$
  \hfill $\left[4a^2b^3\right]$
 \item $\sqrt[4]{\frac{20a^6}{125b^{10}}}$
  \hfill $\left[\sqrt{\frac2 a^3}{5 b^5}\right]$
 \item $\sqrt[3]{64a^6b^9}$
  \hfill $\left[...\right]$
 \item $\sqrt[3]{x^6y^9(x-y)^{12}}$
  \hfill $\left[...\right]$
 \item $\sqrt[8]{\frac{16x^5y^8}{81x}}$
  \hfill $\left[\valass y \cdot \sqrt{\frac{2\cdot \valass x} 3}\right]$
 \item $\left(\sqrt{a+1}\right)^6$
  \hfill $\left[\sqrt{(2x+3)}\right]$
 \item $\sqrt[9]{27a^6b^{12}}$
  \hfill $\left[...\right]$
%  \item $\sqrt[6]{\frac{0,008x^{15}y^9}{8a^{18}}}$
%   \hfill $\left[...\right]$
%  \item $\sqrt[12]{(2x+3)^3}$
%   \hfill $\left[...\right]$
%  \item $\sqrt[6]{a^2+2a+1}$
%   \hfill $\left[\sqrt[5]{\frac{11a^2} b}\right]$
%  \item $\sqrt{\frac{25a^4b^8c^7}{c(a+2b)^6}}$
%   \hfill $\left[...\right]$
%  \item $\sqrt[9]{a^3+3a^2+3a+1}$
%   \hfill $\left[...\right]$
%  \item $\sqrt{3a^2+\sqrt{a^4}}$
%   \hfill $\left[...\right]$
 \item $\sqrt[10]{\frac{121a^5}{ab^2}}$
  \hfill $\left[\sqrt[5]{\frac{11a^2}{b}}\right]$
 \item $\sqrt[4]{x^4+2x^2+1}$
  \hfill $\left[...\right]$
 \item $\sqrt[10]{a^4+6a^2x+9x^2}$
  \hfill $\left[\sqrt[5]{\left|a^2+3x\right|}\right]$
 \item $\sqrt[4]{\frac{16a^4b^6}{25x^2}}$
  \hfill $\left[...\right]$
 \item $\sqrt[6]{8a^3-24a^2+24a-8}$
  \hfill $\left[...\right]$
% %  \item $\sqrt[6]{\frac{9x^2}{y^6}}$
%   \hfill $\left[...\right]$
 \item $\sqrt{\frac{2x^2-2}{8x^2-8}}$
  \hfill $\left[\frac 1 2\right]$
% %  \item $\sqrt[8]{a^4+2a^2x^2+x^4}$
%   \hfill $\left[...\right]$
 \item $\sqrt[9]{x^6+3x^5+3x^4+x^3}$
  \hfill $\left[\frac{5a^2\valass b^3}{a^2+2}\right]$
 \item $\sqrt{\frac{25a^4b^6}{a^4+4+4a^2}}$
  \hfill $\left[...\right]$
 \end{enumeratea}
 \end{multicols}
\end{esercizio}

% \begin{esercizio}[\Ast]
%  \label{ese:2.30}
% Semplifica i seguenti radicali.
%  \begin{multicols}{2}
%  \begin{enumeratea}
%  \item $\sqrt[4]{a^2+6a+9}$;
%  \item $\sqrt[9]{8x^3-12x^2+6x+x^3}$;
%  \item $\sqrt[4]{a^4(a^2-2a+1)}$;
%  \item $\sqrt[4]{(x^2-6x+9)^2}$;
%  \item $\sqrt[12]{(x^2+6x+9)^3}$;
%  \item $\sqrt{a^2+2a+1}-\sqrt{a^2-2a+1}$;
%  \item $\sqrt[18]{\frac{a^9+3a^8+3a^7+a^6}{9a^7+9a^5+18a^6}}$;
%  \item $\sqrt[6]{\frac{(x^2+1-2x)^3b}{b^7\left(x^3+3x^2+3x+1\right)^2}}$;
%  \item $\sqrt{\frac{\left(x^3+x^2y\right)(a+2)}{2x+2y+ax+ay}}$.
%  \end{enumeratea}
%  \end{multicols}
% \end{esercizio}

% \begin{esercizio}[\Ast]
%  \label{ese:2.31}
% Semplifica i seguenti radicali.
%  \begin{multicols}{3}
%  \begin{enumeratea}
%  \item $\sqrt[2n]{16^n}$
%  \item $\sqrt[4n]{\frac{2^{3n}}{3^{2n}}}$
%  \item $\sqrt[n^2]{\frac{6^{2n}}{5^{3n}}}$
%  \item $\sqrt[3n]{27^n\cdot 64^{2n}}$
%  \item $\sqrt[2n^2]{16^{2n}\cdot 81^{2n}}$
%  \item $\sqrt[n+1]{16^{2n+2}}$
%  \item $\sqrt[5]{25x^3y^4}$
%  \item $\sqrt[12]{81a^6b^{12}}$
%  \item $\sqrt[5]{32x^{10}}$
%  \end{enumeratea}
%  \end{multicols}
% \end{esercizio}

% \paragraph{\ref{ese:2.31}}
% b)~$\sqrt[4]{\frac 8 9}$,\quad e)~$\sqrt[n]{6^4}$,\quad i)$2x^2$


% \subsubsection*{2.5 - Moltiplicazione e divisione di radici}
\subsubsection*{\numnameref{sec:radici_moltiplicazione}}

\begin{esercizio}[\Ast]
 \label{ese:2.32}
Esegui le seguenti moltiplicazioni e divisioni di radicali.
 \begin{multicols}{3}
 \begin{enumeratea}
 \item $\sqrt{45}\cdot \sqrt 5$
  \hfill $\left[15\right]$
 \item $\sqrt 2\cdot \sqrt{18}$
  \hfill $\left[...\right]$
 \item $\sqrt[3]{16}\cdot \sqrt[3]4$
  \hfill $\left[...\right]$
 \item $\sqrt{75}\cdot \sqrt{12}$
  \hfill $\left[30\right]$
 \item $\sqrt[3]{20}\cdot \sqrt{50}$
  \hfill $\left[...\right]$
 \item $\sqrt{40}\div(\sqrt{2}\cdot \sqrt{5})$
  \hfill $\left[...\right]$
 \item $\sqrt{5} \div \sqrt{45}$
  \hfill $\left[...\right]$
 \item $\sqrt[3]3:\sqrt[3]9$
  \hfill $\left[...\right]$
 \item $\sqrt[5]2\cdot \sqrt[5]6:\sqrt[5]{12}$
  \hfill $\left[1\right]$
 \item $\sqrt[6]{81}\cdot \sqrt[6]{81}:\sqrt[6]9$
  \hfill $\left[...\right]$
%  \item $\sqrt[4]{1+\frac 1 2}\cdot 
%         \sqrt[4]{2-\frac 1 2}\cdot \sqrt[4]{1+\frac 5 4}$
%   \hfill $\left[...\right]$
 \item $\sqrt 3\cdot \sqrt[3]9$
  \hfill $\left[\sqrt[6]{3^7}\right]$
 \item $\sqrt[3]2\cdot \sqrt 8$
  \hfill $\left[...\right]$
 \item $\sqrt[6]{81}\cdot \sqrt 3$
  \hfill $\left[\sqrt[6]{3^7}\right]$
 \item $\sqrt 2\cdot \sqrt 2\cdot\sqrt 2$
  \hfill $\left[...\right]$
%  \item $\sqrt{\frac{10} 2}\cdot \sqrt[3]{\frac 6 3}:\sqrt[6]{\frac 4 9}$
%   \hfill $\left[...\right]$
%  \item $\sqrt{2^3\cdot 3}\cdot \sqrt 2\cdot \sqrt{3^3}$
%   \hfill $\left[\sqrt[6]{\frac{3^2\cdot 5^3}{4^2}}\right]$
 \end{enumeratea}
 \end{multicols}
\end{esercizio}

% \begin{esercizio}[\Ast]
%  \label{ese:2.34}
% Esegui le seguenti moltiplicazioni e divisioni di radicali.
%  \begin{multicols}{3}
%  \begin{enumeratea}
%  \item $\left(\sqrt[3]{\frac{42}{13}}:\sqrt[3]{\frac{91}{36}}\right):
%         \sqrt{13}$;
%  \item $\sqrt[3]{\frac 3 4}\cdot \sqrt[3]{\frac{25}{24}}\cdot 
%         \sqrt[3]{\frac 5 2}$;
%  \item $\sqrt[3]{5+\frac 1 3}\cdot \sqrt[3]{\frac 4 3}$;
%  \item $\sqrt[5]{2^3}\cdot \sqrt[10]{2^4}$;
%  \item e)$\sqrt{15}\cdot \sqrt{30}\cdot \sqrt 8$;
%  \item $\sqrt 2\cdot \sqrt 3$;
%  \item $\sqrt[3]{-1-\frac 1 2}:\sqrt{1-\frac 1 2}\cdot \sqrt[6]{12}$;
%  \item $\sqrt[3]{1+\frac 1 2}\cdot \sqrt[4]{2+\frac 1 4}$.
%  \end{enumeratea}
%  \end{multicols}
% \end{esercizio}

% \clearpage

% \paragraph{\ref{ese:2.30}}
% c)~$\left|a\right|\sqrt{\left|a-1\right|}$,\quad d)~$\valass{x-3}$,\quad h)$\frac{\valass{x-1}}{\valass b \valass{x+1}}$.

% \paragraph{\ref{ese:2.34}}
% b)~$\frac 5 4$,\quad d)~$2$,\quad e)~$60$,\quad h)~$\sqrt[6]{\frac{3^5}{2^5}}$.

\begin{esercizio}
 \label{ese:2.35}[\Ast]
Esegui le seguenti operazioni (le lettere rappresentano numeri reali positivi).
 \begin{multicols}{2}
 \begin{enumeratea}
 \item $\sqrt[3]{4a}\cdot \sqrt[3]{9a}\cdot \sqrt[3]{12a}$
  \hfill $\left[...\right]$
 \item $\sqrt{3a}:\sqrt{\frac 1 5a}$
  \hfill $\left[\sqrt{15}\right]$
 \item $\sqrt[3]{2ab}\cdot \sqrt[3]{4a^2b^2}$
  \hfill $\left[2ab\right]$
 \item $\sqrt{\frac 1{a^4}}\cdot \sqrt{\frac{a^6b} 2}:\sqrt{\frac{2b} a}$
  \hfill $\left[...\right]$
 \item $\sqrt x\cdot \sqrt[3]{x^2}:\sqrt[6]x$
  \hfill $\left[\sqrt[6]{\frac{2^3a^2}{3^4}}\right]$
 \item $\sqrt{\frac 4 9}\cdot \sqrt{\frac 3 2a}:\sqrt[6]{3a}$
  \hfill $\left[...\right]$
 \item $\sqrt{\frac{a^2+2a+1}{2a}}\cdot \sqrt{\frac{1+a}{a^2}}:
        \sqrt{\frac 2 a}$
  \hfill $\left[...\right]$
 \item $\sqrt{\frac{a+1}{a-3}}\cdot \sqrt[3]{\frac{a^2-9}{a^2-1}}$
  \hfill $\left[\sqrt[6]{\frac{(a+1)(a+3)^2}{(a-3)(a-1)^2}}\right]$
%  \item $\sqrt{\frac{x+1}{x-2}}\cdot \sqrt{\frac{x-1}{x+3}}:
%         \sqrt[3]{\frac{x^2-1}{x^2+x-6}}$
%   \hfill $\left[\sqrt[6]{\frac{(x-1)(x+1)}{(x-2)(x+3)}}\right]$
 \item $\sqrt{a^4b}\cdot \sqrt[6]{\frac{a^2} b}$
  \hfill $\left[...\right]$
 \item $\sqrt[3]{\frac{a^2-2}{a+3}}\cdot \sqrt[4]{\frac{a+3}{a-2}}$
  \hfill $\left[...\right]$
 \item $\sqrt{\frac x y-\frac y x}:\sqrt{x+y}$
  \hfill $\left[\sqrt{\frac{x-y}{xy}}\right]$
 \end{enumeratea}
 \end{multicols}
\end{esercizio}

% \begin{esercizio}[\Ast]
%  \label{ese:2.36}
% Esegui le seguenti operazioni (le lettere rappresentano numeri reali 
% positivi).
%  \begin{multicols}{3}
%  \begin{enumeratea}
%  \item $\sqrt[3]{ax}\cdot \sqrt{xy}\cdot \sqrt[5]{ay}$;
%  \item $\sqrt[3]{(x+1)^2}:\sqrt{x-1}$;
%  \item $\sqrt{a^2-b^2}:\sqrt{a+b}$;
%  \item $\sqrt{a^2-3a}\cdot \sqrt[3]{a^2}\cdot \sqrt[6]{a^5}$;
%  \item $\sqrt{\frac{1-x}{1+x}}\cdot \sqrt[3]{\frac{1-x^2}{1+x^2}}$;
%  \item $\sqrt{\frac{a+b}{a-b}}:\sqrt[3]{\frac{a+b}{a-b}}$.
%  \end{enumeratea}
%  \end{multicols}
% \end{esercizio}

% \paragraph{\ref{ese:2.36}}
% b)~$\sqrt[6]{\frac{(x+1)^4}{(x-1)^3}}$,\quad c)~$\sqrt{a-b}$,\quad e)~$\sqrt[6]{\frac{(1-x)^4}{(1+x)(1+x^2)^2}}$.

% \begin{esercizio}[\Ast]
%  \label{ese:2.38}
% Esegui le seguenti operazioni 
% (le lettere rappresentano numeri reali positivi).
%  \begin{multicols}{2}
%  \begin{enumeratea}
%  \item $\sqrt{\frac 1{b^2}-\frac 1{a^2}}:\sqrt{\frac 1 b-\frac 1 a}$;
%  \item $\frac{\sqrt{4a^2-9}\cdot \sqrt{2a-3}}{\sqrt[3]{2a+3}}$;
%  \item $\sqrt{\frac{9-a^2}{(a+3)^2}}\cdot \sqrt{\frac{27+9a}{3-a}}$;
%  \item $\sqrt{\frac{a+2}{a-1}}:\sqrt[3]{\frac{(a-1)^2}{a^2+4a+4}}$;
%  \item $\sqrt{\frac{x^2-4}{x+1}}\cdot \sqrt[3]{\frac 1{x^3-2x^2}}$;
%  \item $\sqrt[4]{\frac{a+b}{a^2-b^2}}\cdot \sqrt[3]{\frac{a-2b}{a+2b}}\cdot 
%         \sqrt[6]{a^2-4b^2}$.
%  \end{enumeratea}
%  \end{multicols}
% \end{esercizio}

% \begin{esercizio}[\Ast]
%  \label{ese:2.39}
% Esegui le seguenti operazioni 
% (le lettere rappresentano numeri reali positivi).
%  \begin{enumeratea}
%  \item $\sqrt{\frac{a^2b+ab^2}{xy}}\cdot 
%         \sqrt[6]{\frac{(a+b)^2}{x^2}}\cdot 
%         \sqrt[6]{\frac{x^2y^3}{(a+b)^2}}\cdot 
%         \sqrt[4]{\frac x{a^3b^2+a^2b^3}}$;
%  \item $\frac{\sqrt{\frac x y+\frac y x}:\sqrt[3]{\frac x y-\frac 1 x}}
%              {\sqrt{\frac{xy}{x+y}}}$.
%  \end{enumeratea}
% \end{esercizio}

% \paragraph{\ref{ese:2.38}}
% a)~$\sqrt{\frac{a+b}{ab}}$,\quad d)~$\sqrt[6]{\frac{(a+2)^7}{(a-1)^7}}$,\quad e)~$\sqrt[6]{\frac{x+2}{x^2(x+1)}}$.
% 
% \paragraph{\ref{ese:2.39}}
% a)~$\sqrt[4]{\frac{a+b} x}$.

% \subsubsection*{2.6 - Portare un fattore sotto il segno di radice}
\subsubsection*{\numnameref{sec:radici_portare_dentro}}

\begin{esercizio}[\Ast]
 \label{ese:2.40}
Trasporta dentro la radice i fattori esterni.
 \begin{multicols}{4}
 \begin{enumeratea}
 \item $2\sqrt 2$
  \hfill $\left[\sqrt{2^3}\right]$
 \item $3\sqrt 3$
  \hfill $\left[...\right]$
 \item $2\sqrt 3$
  \hfill $\left[...\right]$
 \item $3\sqrt 2$
  \hfill $\left[...\right]$
 \item $\frac 1 2\sqrt 2$
  \hfill $\left[...\right]$
 \item $\frac 1 3\sqrt 3$
  \hfill $\left[...\right]$
 \item $\frac 1 2\sqrt 6$
  \hfill $\left[\sqrt{\frac 3 2}\right]$
 \item $\frac 2 3\sqrt 6$
  \hfill $\left[...\right]$
%  \item $\frac 3 4\sqrt{\frac 3 2}$
%   \hfill $\left[...\right]$
 \item $2\sqrt[3]2$
  \hfill $\left[...\right]$
 \item $\frac 1 3\sqrt[3]3$
  \hfill $\left[...\right]$
%  \item $4\sqrt[3]{\frac 1 2}$
%   \hfill $\left[...\right]$
 \item $-3\sqrt 3$
  \hfill $\left[...\right]$
%  \item $-2\sqrt[3]2$
%   \hfill $\left[...\right]$
%  \item $\frac{-1} 2\sqrt[3]4$
%   \hfill $\left[-\sqrt[3]{\frac 1 2}\right]$
%  \item $\frac{-1} 5\sqrt 5$
%   \hfill $\left[...\right]$
%  \item $-\frac 1 3\sqrt[3]9$
%   \hfill $\left[...\right]$
%  \item $\left(1+\frac 1 2\right)\sqrt 2$
%   \hfill $\left[...\right]$
 \end{enumeratea}
 \end{multicols}
\end{esercizio}

\begin{esercizio}[\Ast]
 \label{ese:2.41}
Trasporta dentro la radice i fattori esterni, discutendo i casi letterali.
 \begin{multicols}{3}
 \begin{enumeratea}
 \item $x\sqrt{12}$
  \hfill $\left[...\right]$
 \item $x^2\sqrt[3]x$
  \hfill $\left[\sqrt[3]{x^7}\right]$
 \item $a\sqrt 2$
  \hfill $\left[...\right]$
 \item $x^2\sqrt[3]3$
  \hfill $\left[...\right]$
 \item $2a\sqrt 5$
  \hfill $\left[...\right]$
 \item $a\sqrt{-a}$
  \hfill $\left[...\right]$
 \item $(a-1)\sqrt a$
  \hfill $\left[\sqrt{(a-1)^2a}\right]$
 \item $(x-2)\sqrt{\frac 1{2x-4}}$
  \hfill $\left[...\right]$
%  \item $x\sqrt{\frac 1{x^2+x}}$
%   \hfill $\left[...\right]$
%  \item $\frac{a+1}{a+2}\sqrt{\frac{a^2+3a+2}{a^2+4a+3}}$
%   \hfill $\left[...\right]$
%  \item $\frac 2 x\sqrt{\frac{x^2+x}{x-1}-x}$
%   \hfill $\left[...\right]$
%  \item $\frac 1{x-1}\sqrt{x^2-1}$
%   \hfill $\left[...\right]$
 \end{enumeratea}
 \end{multicols}
\end{esercizio}

% \subsubsection*{2.7 - Portare un fattore fuori dal segno di radice}
\subsubsection*{\numnameref{sec:radici_portare_fuori}}

\begin{esercizio}[\Ast]
 \label{ese:2.42}
Semplifica i radicali portando fuori i fattori possibili.
 \begin{multicols}{4}
 \begin{enumeratea}
 \item $\sqrt{250}$
  \hfill $\left[5\sqrt{10}\right]$
 \item $\sqrt{486}$
  \hfill $\left[9\sqrt 6\right]$
 \item $\sqrt{864}$
  \hfill $\left[12\sqrt 6\right]$
 \item $\sqrt{1-\frac 9{25}}$
  \hfill $\left[...\right]$
%  \item $\sqrt{20}$
%   \hfill $\left[...\right]$
%  \item $\sqrt{0,12}$
%   \hfill $\left[...\right]$
%  \item $\sqrt{45}$
%   \hfill $\left[...\right]$
%  \item $\sqrt{48}$
%   \hfill $\left[...\right]$
%  \item $\sqrt{98}$
%   \hfill $\left[...\right]$
%  \item $\sqrt{50}$
%   \hfill $\left[...\right]$
 \item $\sqrt{300}$
  \hfill $\left[10\sqrt 3\right]$
 \item $\sqrt{27}$
  \hfill $\left[...\right]$
 \item $\sqrt{75}$
  \hfill $\left[...\right]$
 \item $\frac 2 5\sqrt{\frac{50} 4}$
  \hfill $\left[...\right]$
 \item $\sqrt{40}$
  \hfill $\left[...\right]$
 \item $\sqrt{12}$
  \hfill $\left[...\right]$
 \item $\sqrt{80}$
  \hfill $\left[...\right]$
 \item $\sqrt{\frac{18}{80}}$
  \hfill $\left[...\right]$
%  \item $\sqrt{\frac 9 4+\frac 4 9}$
%   \hfill $\left[\frac 1 6\sqrt{97}\right]$
%  \item $\frac 3 2\sqrt{\frac 8{27}}$
%   \hfill $\left[...\right]$
%  \item $\frac 5 7\sqrt{\frac{98}{75}}$
%   \hfill $\left[\sqrt{\frac 2 3}\right]$
%  \item $\frac 1 5\sqrt{\frac{1000}{81}}$
%   \hfill $\left[...\right]$
 \item $\sqrt{3456}$
  \hfill $\left[24\sqrt 6\right]$
 \item $\sqrt[3]{250}$
  \hfill $\left[...\right]$
 \item $\sqrt[3]{24}$
  \hfill $\left[...\right]$
 \item $\sqrt{\frac{10} 3+\frac 2 9}$
  \hfill $\left[...\right]$
%  \item $\sqrt[3]{108}$
%   \hfill $\left[...\right]$
%  \item $\sqrt[4]{32}$
%   \hfill $\left[...\right]$
%  \item $\sqrt[4]{48}$
%   \hfill $\left[...\right]$
%  \item $\sqrt[4]{250}$
%   \hfill $\left[...\right]$
%  \item $\sqrt[5]{96}$
%   \hfill $\left[...\right]$
%  \item $\sqrt[5]{160}$
%   \hfill $\left[...\right]$
 \end{enumeratea}
 \end{multicols}
\end{esercizio}

\begin{esercizio}[\Ast]
 \label{ese:2.44}
Semplifica i radicali portando fuori i fattori possibili 
(attenzione al valore assoluto).
 \begin{multicols}{3}
 \begin{enumeratea}
 \item $\sqrt{x^2y}$
  \hfill $\left[...\right]$
 \item $\sqrt{\frac{a^5}{b^2}}$
  \hfill $\left[...\right]$
 \item $\sqrt{\frac{a^2b^3c^3}{d^9}}$
  \hfill $\left[...\right]$
 \item $\sqrt{4ax^2}$
  \hfill $\left[...\right]$
 \item $\sqrt{9a^2b}$
  \hfill $\left[3\valass a\sqrt{b}\right]$
 \item $\sqrt{2a^2x}$
  \hfill $\left[...\right]$
 \item $\sqrt{x^3}$
  \hfill $\left[...\right]$
 \item $\sqrt{a^7}$
  \hfill $\left[...\right]$
 \item $\sqrt[3]{16a^3x^4}$
  \hfill $\left[...\right]$
%  \item $\sqrt[3]{4a^4b^5}$
%   \hfill $\left[...\right]$
%  \item $\sqrt[3]{27a^7b^8}$
%   \hfill $\left[...\right]$
%  \item $\sqrt{18a^6b^5c^7}$
%   \hfill $\left[...\right]$
 \end{enumeratea}
 \end{multicols}
\end{esercizio}

% \newpage

% \begin{esercizio}[\Ast]
%  \label{ese:2.45}
% Semplifica i radicali portando fuori i fattori possibili 
% (attenzione al valore assoluto).
%  \begin{multicols}{3}
%  \begin{enumeratea}
%  \item $\sqrt{a^2+a^3}$;
%  \item $\sqrt{4x^4-4x^2}$;
%  \item $\sqrt{25x^7-25x^5}$;
%  \item $\sqrt[3]{3a^5b^2c^9}$;
%  \item $\sqrt[4]{16a^4b^5c^7x^6}$;
%  \item $\sqrt[5]{64a^4b^5c^6d^7}$;
%  \item $\sqrt[6]{a^{42}b^{57}}$;
%  \item $\sqrt[7]{a^{71}b^{82}}$;
%  \item $\sqrt{a^3}+\sqrt{a^5}+\sqrt{a^7}$.
%  \end{enumeratea}
%  \end{multicols}
% \end{esercizio}

% \paragraph{\ref{ese:2.45}}
% b)~$\valass{2x}\sqrt{x^2-1}, \CE x\le 1\vee x\ge 1$,\quad i)~$(a+a^2+a^3)\sqrt a$.

% \subsubsection*{2.8 - Potenza di radice e radice di radice}
\subsubsection*{\numnameref{sec:radici_potenza}}

\begin{esercizio}[\Ast]
 \label{ese:2.46}
Esegui le seguenti potenze di radici.
 \begin{multicols}{4}
 \begin{enumeratea}
 \item $\left(\sqrt 3\right)^2$
  \hfill $\left[...\right]$
 \item $\left(\sqrt[3]2\right)^3$
  \hfill $\left[...\right]$
 \item $\left(\sqrt 4\right)^2$
  \hfill $\left[...\right]$
 \item $\left(\sqrt[4]2\right)^6$
  \hfill $\left[\sqrt{2^3}\right]$
%  \item $\left(2\sqrt 3\right)^2$
%   \hfill $\left[...\right]$
%  \item $\left(3\sqrt 5\right)^2$
%   \hfill $\left[...\right]$
%  \item $\left(5\sqrt 2\right)^2$
%   \hfill $\left[...\right]$
%  \item $\left(-2\sqrt 5\right)^2$
%   \hfill $\left[...\right]$
%  \item $\left(\frac 1 2\sqrt 2\right)^2$
%   \hfill $\left[...\right]$
%  \item $\left(\frac 2 3\sqrt[4]{\frac 2 3}\right)^2$
%   \hfill $\left[...\right]$
%  \item $\left(a\sqrt{2a}\right)^2$
%   \hfill $\left[...\right]$
 \item $\left(\frac 1 a\sqrt a\right)^2$
  \hfill $\left[2a^3\right]$
 \item $\left(2\sqrt[3]3\right)^3$
  \hfill $\left[...\right]$
 \item $\left(3\sqrt[3]3\right)^3$
  \hfill $\left[...\right]$
%  \item $\left(\frac 1 3\sqrt[3]3\right)^3$
%   \hfill $\left[...\right]$
%  \item $\left(\frac 1 9\sqrt[3]9\right)^3$
%   \hfill $\left[\frac 1 9\right]$
%  \item $\left(\sqrt 3\right)^3$
%   \hfill $\left[...\right]$
%  \item $\left(2\sqrt 5\right)^3$
%   \hfill $\left[...\right]$
%  \item $\left(3\sqrt 2\right)^3$
%   \hfill $\left[...\right]$
%  \item $\left(\sqrt[3]2\right)^6$
%   \hfill $\left[...\right]$
%  \item $\left(\sqrt[3]3\right)^6$
%   \hfill $\left[...\right]$
%  \item $\left(\sqrt[3]5\right)^5$
%   \hfill $\left[...\right]$
%  \item $\left(\sqrt[3]2\right)^6$
%   \hfill $\left[...\right]$
%  \item $\left(\sqrt[6]3\right)^4$
%   \hfill $\left[...\right]$
%  \item $\left(\sqrt[6]{3ab^2}\right)^4$
%   \hfill $\left[...\right]$
%  \item $\left(\sqrt[4]{16a^2b^3}\right)^2$
%   \hfill $\left[\sqrt{2^4a^2\left|b^3\right|}\right]$
%  \item $\left(\sqrt[3]{6a^3b^2}\right)^4$
%   \hfill $\left[...\right]$
%  \item $\left(\sqrt[3]{81ab^4}\right)^4$
%   \hfill $\left[...\right]$
 \end{enumeratea}
 \end{multicols}
\end{esercizio}

\begin{esercizio}[\Ast]
 \label{ese:2.48}
Esegui le seguenti radici di radici.
 \begin{multicols}{4}
 \begin{enumeratea}
 \item $\sqrt[3]{\sqrt 2}$
  \hfill $\left[...\right]$
 \item $\sqrt[3]{\sqrt[3]{16}}$
  \hfill $\left[...\right]$
 \item $\sqrt[3]{\sqrt[4]{15}}$
  \hfill $\left[...\right]$
 \item $\sqrt[5]{\sqrt{a^5}}$
  \hfill $\left[...\right]$
 \item $\sqrt{\sqrt{16}}$
  \hfill $\left[...\right]$
 \item $\sqrt{\sqrt{\sqrt 3}}$
  \hfill $\left[...\right]$
 \item $\sqrt{\sqrt[4]{16}}$
  \hfill $\left[...\right]$
 \item $\sqrt[5]{\sqrt{\sqrt 15}}$
  \hfill $\left[...\right]$
 \item $\sqrt[5]{\sqrt{a^{10}}}$
  \hfill $\left[...\right]$
%  \item $\sqrt[3]{\sqrt{\sqrt[3]{a^{12}}}}$
%   \hfill $\left[\sqrt[3]{a^2}\right]$
 \item $\sqrt{\sqrt[3]{3a}}$
  \hfill $\left[...\right]$
 \item $\sqrt{\sqrt[4]{3ab}}$
  \hfill $\left[...\right]$
%  \item $\sqrt[3]{\sqrt{(a+1)^5}}$
%   \hfill $\left[...\right]$
 \item $\sqrt[4]{\sqrt{(2a)^5}}$
  \hfill $\left[...\right]$
%  \item $\sqrt{2(a-b)}\cdot \sqrt{\sqrt[3]{\frac 1{4a-4b}}}$
%   \hfill $\left[...\right]$
%  \item $\sqrt{3(a+b)}\cdot \sqrt{\sqrt[3]{\frac 1{3a+3b}}}$
%   \hfill $\left[\sqrt[3]{3(a+b)}, \CE a>b\right]$
 \end{enumeratea}
 \end{multicols}
\end{esercizio}

% \subsubsection*{2.9 - Somma di radicali}
\subsubsection*{\numnameref{sec:radici_somma}}

\begin{esercizio}[\Ast]
 \label{ese:2.50}
Esegui le seguenti operazioni con i radicali.
 \begin{multicols}{2}
 \begin{enumeratea}
%  \item $3\sqrt 2+\sqrt 2$
%   \hfill $\left[...\right]$
 \item $\sqrt 3-3\sqrt 3$
  \hfill $\left[-2 \sqrt{3}\right]$
 \item $8\sqrt 6-3\sqrt 6$
  \hfill $\left[5\sqrt 6\right]$
 \item $\sqrt 5-3\sqrt 5+7\sqrt 5$
  \hfill $\left[5 \sqrt{5}\right]$
 \item $3\sqrt 2+2\sqrt 2-3\sqrt 2$
  \hfill $\left[...\right]$
 \item $2\sqrt 7-7\sqrt 7+4\sqrt 7$
  \hfill $\left[-\sqrt 7\right]$
 \item $11\sqrt 5+6\sqrt 2-(8\sqrt 5+3\sqrt 2)$
  \hfill $\left[3(\sqrt 5+3\sqrt 2)\right]$
%  \item $5\sqrt 3+3\sqrt 7-[2\sqrt 3-(4\sqrt 7-3\sqrt 3)]$
%   \hfill $\left[7\sqrt 7\right]$
%  \item $\sqrt 2+\frac 1 2\sqrt 2-\frac 3 4\sqrt 2$
%   \hfill $\left[...\right]$
%  \item $\frac{\sqrt 3} 2-\frac{\sqrt 3} 3+\frac{\sqrt 3} 4$
%   \hfill $\left[...\right]$
 \item $3\sqrt 5+\frac 2 3\sqrt 2-\frac 5 6\sqrt 2$
  \hfill $\left[\sqrt 5-\frac 1 6\sqrt 2\right]$
 \item $5\sqrt{10}-\left(6+4\sqrt{19}\right)+2-\sqrt{10}$
  \hfill $\left[...\right]$
%  \item $\sqrt 5+\sqrt 2+3\sqrt 2-2\sqrt 2$
%   \hfill $\left[...\right]$
 \item $-3\sqrt 7+4\sqrt 2+\sqrt 3-5\sqrt 7+8\sqrt 3$
  \hfill $\left[...\right]$
%  \item $3\sqrt 3+5\sqrt 5+6\sqrt 6-7\sqrt 3-8\sqrt 5-9\sqrt 6$
%   \hfill $\left[...\right]$
 \item $\sqrt[3]2+3\sqrt[3]2-2\sqrt 2+3\sqrt 2$
  \hfill $\left[...\right]$
 \item $5\sqrt 6+3\sqrt[4]6-2\sqrt[4]6+3\sqrt[3]6-2\sqrt 6$
  \hfill $\left[...\right]$
 \item $\sqrt{75}+3\sqrt{18}-2\sqrt{12}-2\sqrt{50}$
  \hfill $\left[\sqrt 3-\sqrt 2\right]$
 \end{enumeratea}
 \end{multicols}
\end{esercizio}

\begin{esercizio}[\Ast]
 \label{ese:2.52}
Esegui le seguenti operazioni con i radicali.
 \begin{enumeratea}
 \item $3\sqrt{128}-2\sqrt{72}-(2\sqrt{50}+\sqrt 8)$
  \hfill $\left[0\right]$
 \item $3\sqrt{48}+2\sqrt{32}+\sqrt{98}-(4\sqrt{27}+\sqrt{450})$
  \hfill $\left[0\right]$
 \item $\sqrt[4]{162}-\sqrt[4]{32}+5\sqrt[3]{16}-\sqrt[3]{54}+\sqrt[3]{250}$
  \hfill $\left[\sqrt[4]2+12\sqrt[3]2\right]$
 \item $2\sqrt[3]{54}-\sqrt[4]{243}+3\sqrt[4]{48}-\sqrt[3]{250}$
  \hfill $\left[\sqrt[3]2+3\sqrt[4]3\right]$
 \item $\sqrt{\frac{32}{25}}-\sqrt{\frac{108}{25}}+\sqrt{\frac{27}{49}}+
        \frac 2 5\sqrt{\frac 3 4}-\sqrt{\frac 8 9}$
  \hfill $\left[\frac 2{15}\sqrt 2-\frac 4 7\sqrt 3\right]$
 \item $2\sqrt{\frac{27} 8}+5\sqrt{\frac 3{50}}+7\sqrt{\frac{27}{98}}-
        5\sqrt{\frac{147}{50}}$
  \hfill $\left[0\right]$
 \item $\frac 1 2\sqrt a-\frac 4 5\sqrt b-\sqrt a+0,4\sqrt b$
  \hfill $\left[-\frac 1 2\sqrt a-\frac 2 5\sqrt b\right]$
 \item $\sqrt[3]{a-b}+\sqrt[3]{a^4-a^3b}-\sqrt[3]{{ab}^3-b^4}$
  \hfill $\left[(1+a-b)\sqrt[3]{a-b}\right]$
 \item $3\sqrt x-5\sqrt x$
  \hfill $\left[...\right]$
 \item $2\sqrt[3]{x^2}+3\sqrt x+3\sqrt[3]{x^2}-2\sqrt x$
  \hfill $\left[...\right]$
 \item $\sqrt{a-b}+\sqrt{a+b}-\sqrt{a-b}+2\sqrt{a+b}$
  \hfill $\left[...\right]$
%  \item $\frac 1 3\sqrt x-\frac 4 5\sqrt x+0,4\sqrt a-\frac 1 2\sqrt a$
%   \hfill $\left[...\right]$
%  \item $2a\sqrt{2a}-7a\sqrt{2a}+3a\sqrt{2a}-\frac 1 2\sqrt a$
%   \hfill $\left[...\right]$
 \item $6\sqrt{{ab}}-3\sqrt a-7\sqrt{{ab}}+2\sqrt a+9\sqrt b+\sqrt a$
  \hfill $\left[9\sqrt b-\sqrt{ab}\right]$
 \item $3\sqrt{xy}+3\sqrt x-3\sqrt y+2\sqrt{xy}-3(\sqrt x+\sqrt y)$
  \hfill $\left[...\right]$
 \end{enumeratea}
\end{esercizio}

\begin{esercizio}[\Ast]
 \label{ese:2.55}
Esegui le seguenti operazioni con i radicali.
 \begin{multicols}{2}
 \begin{enumeratea}
%  \item $(\sqrt 2+1)(\sqrt 2+2)$
%   \hfill $\left[...\right]$
%  \item $(3\sqrt 2-1)(2\sqrt 2-3)$
%   \hfill $\left[...\right]$
 \item $(\sqrt 2-1)(\sqrt 2+1)$
  \hfill $\left[...\right]$
 \item $(\sqrt 2-3\sqrt 3)(3\sqrt 3-\sqrt 2)$
  \hfill $\left[...\right]$
 \item $(\sqrt 3+1)^2$
  \hfill $\left[4+2\sqrt 3\right]$
 \item $(\sqrt 3-2)^2$
  \hfill $\left[7-4\sqrt 3\right]$
 \item $(2+\sqrt 5)^2$
  \hfill $\left[9+4\sqrt 5\right]$
 \item $(4-\sqrt 3)^2$
  \hfill $\left[19-8\sqrt 3\right]$
 \item $(6+2\sqrt 3)^2$
  \hfill $\left[48+24\sqrt 3\right]$
 \item $(\sqrt 6-\frac 1 2\sqrt 3)^2$;
  \hfill $\left[\frac{27} 4-\sqrt{18}\right]$
%  \item $(\sqrt 2-1)^2$;
%   \hfill $\left[...\right]$
%  \item $(2\sqrt 2-1)^2$.
%   \hfill $\left[...\right]$
%  \item $(\sqrt 3+1)^2$
%   \hfill $\left[...\right]$
%  \item $(\sqrt 3-3)^2$
%   \hfill $\left[...\right]$
%  \item $(\sqrt 5-2)^2$
%   \hfill $\left[...\right]$
%  \item $(2\sqrt 5+3)^2$
%   \hfill $\left[...\right]$
%  \item $(2\sqrt 7-\sqrt 5)^2$
%   \hfill $\left[...\right]$
 \item $(3\sqrt 2-2\sqrt 3)^2$
  \hfill $\left[...\right]$
 \item $(\sqrt 2-3\sqrt 3)^2$
  \hfill $\left[...\right]$
%  \item $(1+\sqrt 2+\sqrt 3)^2$
%   \hfill $\left[...\right]$
 \item $(\sqrt 2-1-\sqrt 5)^2$
  \hfill $\left[8-2\sqrt 2-2\sqrt{10}+2\sqrt 5\right]$
 \item $(\sqrt 3-2\sqrt 2+1)^2$
  \hfill $\left[1-3\sqrt[3]4+3\sqrt[3]2\right]$
%  \item $(\sqrt 2+\sqrt 3+\sqrt 6)^2$
%   \hfill $\left[...\right]$
%  \item $(\sqrt[3]2-1)^3$
%   \hfill $\left[...\right]$
 \item $(\sqrt 2+\sqrt 3)^2$
  \hfill $\left[...\right]$
%  \item $(2\sqrt 2-1)^2$
%   \hfill $\left[...\right]$
%  \item $(3\sqrt 3+2\sqrt 2)^2$
%   \hfill $\left[...\right]$
%  \item $\left(\sqrt 3-2\sqrt 2\right)^2$
%   \hfill $\left[...\right]$
%  \item $(4\sqrt 3-3\sqrt 7)^2$
%   \hfill $\left[...\right]$
%  \item $(2\sqrt 2-3\sqrt 3)^2$
%   \hfill $\left[-19-12\sqrt 6\right]$
%  \item $(\sqrt x-1)^2$
%   \hfill $\left[...\right]$
%  \item $(2x+\sqrt x)^2$
%   \hfill $\left[...\right]$
%  \item $(x+\sqrt[3]x)^3$
%   \hfill $\left[...\right]$
%  \item $(2x+\sqrt x)(2x-\sqrt x)$
%   \hfill $\left[...\right]$
%  \item $\left(\sqrt a+\frac 1{\sqrt a}\right)^2$
%   \hfill $\left[a+2+\frac 1 a\right]$
%  \item $\left(\sqrt a+\frac 1 a\right)\left(\sqrt a-\frac 1 a\right)$
%   \hfill $\left[...\right]$
 \end{enumeratea}
 \end{multicols}
\end{esercizio}

% \begin{esercizio}[\Ast]
%  \label{ese:2.57}
% Esegui le seguenti operazioni con i radicali.
%  \begin{multicols}{2}
%  \begin{enumeratea}
%  \item $(\sqrt[3]3+1)^3$;
%  \item $(\sqrt[3]2-2)^3$;
%  \item $(\sqrt[3]3+\sqrt[3]2)^3$;
%  \item $(\sqrt[3]3+\sqrt[3]2)(\sqrt[3]9-\sqrt[3]4)$;
%  \item $\left[(\sqrt[4]2+1)(\sqrt[4]2-1)\right]^2$;
%  \item $(\sqrt[3]2+\sqrt[3]3)(\sqrt[3]4-\sqrt[3]6+\sqrt[3]9)$;
%  \item $(\sqrt 3+\sqrt 3)\sqrt 3 \sqrt 3$;
%  \item $3\sqrt 3+\sqrt 3:\sqrt 3-(1+\sqrt 3)^2$;
%  \item $6\sqrt 5+2\sqrt 5\cdot \sqrt{20}-3\sqrt 5+\sqrt{25}$;
%  \item $(\sqrt[3]a-\sqrt[3]2)(\sqrt[3]{a^2}+\sqrt[3]{2a}+\sqrt[3]4)$;
%  \item $(1+\sqrt 2)^2$;
%  \item $(2-\sqrt 2)^2$.
%  \end{enumeratea}
%  \end{multicols}
% \end{esercizio}

% \paragraph{\ref{ese:2.57}}
% i)~$3\sqrt 5+25$.

% \clearpage

\begin{esercizio}[\Ast]
 \label{ese:2.59}
Esegui le seguenti operazioni con i radicali.
 \begin{enumeratea}
 \item $(\sqrt x+\sqrt y)(\sqrt x-\sqrt y)$
  \hfill $\left[x-y\right]$
 \item $(\sqrt 2-1)^2-(2\sqrt 2-1)^2+(\sqrt 2-1)(\sqrt 2+1)$
  \hfill $\left[...\right]$
 \item $(\sqrt 3+1)^2+\sqrt 3(\sqrt 3-3)-2(\sqrt 3+3)(\sqrt 3-3)$
  \hfill $\left[...\right]$
 \item $(\sqrt 3-3)^2+(\sqrt 3-3)^3+2\sqrt{27}-\sqrt 3(2\sqrt 3-2)$
  \hfill $\left[...\right]$
 \item $(\sqrt 5-2)^2-(2\sqrt 5+3)^2+\left[(\sqrt 5-\sqrt 2)^2+1\right]
        (\sqrt 5+\sqrt 2)$
 \item $(2\sqrt 7-\sqrt 5)^2+2(\sqrt 7+\sqrt 5+1)^2-\sqrt{35}$
  \hfill $\left[...\right]$
 \item $(\sqrt 2+1)^2+(\sqrt 2-1)^2$
  \hfill $\left[...\right]$
 \item $(2\sqrt 2-3\sqrt 3)(3\sqrt 2+2\sqrt 3)$
  \hfill $\left[6\right]$
 \end{enumeratea}
\end{esercizio}

% \begin{esercizio}
%  \label{ese:2.60}
% Esegui le seguenti operazioni con i radicali.
%  \begin{multicols}{2}
%  \begin{enumeratea}
%  \item $(\sqrt x-1)^2+(2\sqrt x+1)(\sqrt x-2)$;
%  \item $(\sqrt 2-1)^3+(\sqrt 2-1)^2\sqrt 2-1$;
%  \item $2\sqrt{54}-\sqrt[4]{243}+3\sqrt[4]{48}-\sqrt[3]{250}$;
%  \item $(\sqrt{10}-\sqrt 7)(2\sqrt{10}+3\sqrt 7)$;
%  \item $\sqrt{48x^2y}+5x\sqrt{27y}$;
%  \item $\sqrt 5\sqrt{15}-4\sqrt 3$;
%  \item $(\sqrt 7-\sqrt 5)(2\sqrt 7+3\sqrt 5)$;
%  \item $\sqrt{27ax^4}+5x^2\sqrt{75a}$.
%  \end{enumeratea}
%  \end{multicols}
% \end{esercizio}

% \begin{esercizio}[\Ast]
%  \label{ese:2.61}
% Esegui le seguenti operazioni con i radicali.
%  \begin{enumeratea}
%  \item $\sqrt{125}+3\sqrt[6]{27}-\sqrt{45}-2\sqrt[4]9+\sqrt{20}+7\sqrt[8]{81}$;
%  \item $\sqrt[3]{a\sqrt a}\cdot \sqrt{a\sqrt[3]a}\cdot \sqrt[3]{a\sqrt[3]a}\cdot \sqrt[3]{a\sqrt a}\cdot \sqrt[9]{a^8}$;
%  \item $\sqrt[5]{b\sqrt[3]{b^2}}\cdot \sqrt{b^2\sqrt{b\sqrt{b^2}}}:\sqrt[5]{b^4\sqrt[3]{b^2}}\cdot \sqrt b$;
%  \item $\sqrt[3]{\frac x{y^3}-\frac 1{y^2}}+\sqrt[3]{xy^3-y^4}-\sqrt[3]{8x-8y}$;
%  \item $(\sqrt 2+3)\cdot (1-\sqrt 3)^2$;
%  \item $(\sqrt[3]2+3)\cdot (1-\sqrt[3]3)^2$;
%  \item $\frac{\sqrt a}{\sqrt a+1}\cdot \frac{\sqrt a}{\sqrt a-1}$;
%  \item $\sqrt[5]{b\sqrt[3]{b^2}}\cdot \sqrt{b\sqrt{b\sqrt{b^2}}}:\left(\sqrt[5]{b\sqrt[3]{b^2}}\cdot \sqrt b\right)$.
%  \end{enumeratea}
% \end{esercizio}
% 
% \begin{esercizio}[\Ast]
%  \label{ese:2.62}
% Esegui le seguenti operazioni con i radicali.
%  \begin{multicols}{2}
%  \begin{enumeratea}
%  \item $\sqrt{\frac{4a^2-b^2}{a^2-b^2}}\sqrt{\frac{a-b}{2a+b}}$
%  \item $\sqrt{\frac {9a}{b}}\sqrt{\frac{b^2-2b}{3ab-6a}}$
%  \item $\sqrt{\frac{9a^2-6ab+b^2}{a^2-b^2}}\sqrt{\frac{a+b}{3a-b}}$
%  \item $\sqrt{\frac{x-y}{x+y}}\sqrt{\frac{x^2+2xy+y^2}{x^2-y^2}}$
% %  \item $\sqrt[3]{\frac a{a+3}\sqrt{\frac a{a+3}\sqrt{\frac a{a+3}}}}:\sqrt{\frac a{a+3}}$;
% %  \item $\sqrt{\frac{x-1}{x+1}\sqrt{\frac{x-1}{x+1}\sqrt{\frac 1{x-1}}}}\cdot \sqrt[4]{x+1}$.
%  \end{enumeratea}
%  \end{multicols}
% \end{esercizio}
% 
% \begin{esercizio}[\Ast]
%  \label{ese:2.63}
% Esegui le seguenti operazioni con i radicali.
%  \begin{enumeratea}
%  \item $\sqrt{\frac{a^2-2a+1}{a(a+1)^3}}\cdot \sqrt[4]{\frac{a^2}{(a+1)^2}}\cdot \sqrt[3]{\frac{(a+1)^3}{(a-1)^2}}$
%  \item $\left(\sqrt{\frac 1{b^4}+\frac 1{b^2}}+\sqrt{\frac{ab^5+ab^4} a}-2\sqrt{b+1}\right)\cdot \frac{b^2}{(b+1)^2}$
%  \item $\left(\sqrt[3x]{y^x\sqrt[4x]y}+\sqrt[6]{y^2\sqrt[2x^2]y}\right)\cdot \sqrt[3]{y\sqrt[4x^2]{\frac 1 y}}$
%  \item $\sqrt[4]{\frac{b^2-1} b}\cdot \sqrt[3]{\frac{3b-3}{6b^2}}:\sqrt[6]{\frac{(b-1)^4}{4b^5}}$
%  \item $\sqrt[3]{\frac{a^2+2a+1}{ab-b}}\cdot \sqrt[6]{\frac{a^2-2a+1}{ab+b}}\cdot \sqrt[4]{\frac{b^2(a-1)^2}{2a^2+4a+2}}$
%  \item $\sqrt[3]{\frac{x^2+2xy+y^2}{x+3}}\cdot \sqrt[3]{\frac{5x}{x^2+6x+9}}\cdot \sqrt[3]{\frac{x+y}{5x}}$
%  \end{enumeratea}
% \end{esercizio}
% 
% \begin{esercizio}[\Ast]
%  \label{ese:2.64}
% Esegui le seguenti operazioni con i radicali.
%  \begin{enumeratea}
%  \item $\sqrt[3]{\frac{x^2-x}{x+1}}\cdot \sqrt[15]{\frac{x^2+2x+1}{x^2-2x+1}}:\sqrt[5]{\frac{x-1}{x+1}}$
%  \item $\sqrt{\frac{25x^3+25x^2}{y^3-y^2}}+\sqrt{\frac{x^3+x^2}{y^3-y^2}}-x\sqrt{\frac{4x+4}{y^3-y^2}}$
%  \item $\left(\sqrt{\frac 1{y^4}+\frac 1{y^3}}+\sqrt{\frac{xy^5+xy^4} x}-2\sqrt{y+1}\right):\frac{(y+1)^2}{y^2}$
%  \item $\sqrt[4]{\frac{a^2-a}{(a+1)^2}}\cdot \sqrt[12]{\frac{a^2-2a+1}{(a-1)^7}}:\sqrt[3]{\frac{2a^2-2a+1}{a^3-a^2}-\frac 1{a-1}}$
%  \item $\sqrt{\frac{a^2b+ab^2}{xy}}\cdot \sqrt[6]{\frac{(a+b)^2}{x^2}}\cdot \sqrt[6]{\frac{x^2y^3}{(a+b)^2}\cdot \sqrt[4]{\frac x{a^3b^2+a^2b^3}}}$
%  \item $\sqrt[6]{\frac 1 x+4x-4}\cdot \sqrt[3]{\frac 1 x+4x+4}\cdot \sqrt{\frac x{4x^2-1}}$
%  \end{enumeratea}
% \end{esercizio}
% 
% \begin{esercizio}
%  \label{ese:2.65}[\Ast]
% Esegui le seguenti operazioni con i radicali.
%  \begin{enumeratea}
%  \item $\sqrt{\frac{a^2-2a+1}{a(a+1)^3}}\cdot \sqrt[4]{\frac{a^2}{(a+1)^2}}\cdot \sqrt[3]{\frac{(a+1)^2}{(a-1)^2}}$
%  \item $\left(\sqrt[3]{\frac a 3-2+\frac 3 a}\cdot \sqrt[6]{\frac{9a^2(a+3)^3}{(a-3)^2}}\right):\sqrt{\frac{a^2-9}{3a}}$
%  \item $\sqrt[4]{\frac{a^3-a^2}{(a+1)^2}}\cdot \sqrt[12]{\frac{a^2-2a+1}{(a-1)^7}}\cdot \sqrt[3]{\frac{2a^2-2a+1}{a^3-a^2}-\frac 1{a-1}}$
%  \item $\sqrt{1-\frac 1 y+\frac 1{4y^2}}:\left(\sqrt[6]{\frac 1{8y^3+12y^2+6y+1}}\cdot \sqrt{1-\frac 1{4y^2}}\right)$
%  \item $\sqrt[3]{1-\frac 1 a+\frac 1{4a^2}}:\left(\sqrt{1-\frac 1{4a^2}}\cdot \sqrt[6]{\frac 1{8a^3+12a^2+6a+1}}\right)$
%  \item $\sqrt{\frac 1{5a}+\frac 1{25a^2}}+\sqrt{\frac{25a^2-1}{20a^3-4a^2}}-\sqrt{\frac{5a+1}{100a^2}}$
%  \end{enumeratea}
% \end{esercizio}

% \begin{esercizio}
%  \label{ese:2.66}[\Ast]
% Esegui le seguenti operazioni con i radicali.
%  \begin{enumeratea}
%  \item $\sqrt[3]{\frac x{y^3}-\frac 1{y^2}}+\sqrt[3]{xy^3-y^4}-\sqrt[3]{8x-8y}$;
%  \item $\sqrt{\frac{x^2+xy+y^2}{4x^2}}+\sqrt{\frac{4x^3-4y^3}{x-y}}+\sqrt{4x^4+4x^3y+4x^2y^2}$;
%  \item $\sqrt{\frac{a^3+2a^2+a}{a^2+6a+9}}+\sqrt{\frac{a^3+4a^2+4a}{a^2+6a+9}}-\sqrt{\frac{a^3}{a^2+6a+9}}$;
%  \item $\sqrt{4x-12y}+\sqrt{\frac{x^3-3x^2y}{y^2}}+\sqrt{\frac{xy^2-3y^3}{x^2}}$;
%  \item $\left(\sqrt[6]{\frac 1{x^2-2x+1}}+\sqrt[6]{\frac{64a^6}{x^2-2x+1}}+\sqrt[6]{\frac{a^{12}}{x^2-2x+1}}\right)\cdot \sqrt[3]{x-1}$;
%  \item $\left(\sqrt[3x]{y^x\sqrt[4x]y}+\sqrt[6]{y^2\sqrt[2x^2]y}\right)\cdot \sqrt[4x^2]{\frac 1 y}$.
%  \end{enumeratea}
% \end{esercizio}

%\paragraph{\ref{ese:2.60}}
%a)~,\quad d)~,\quad g)~,\quad i)~,\quad f).

% \paragraph{\ref{ese:2.61}}
% c)~$\sqrt[5]{b^7}$,\quad h)~$\sqrt b$.
% %-----------------------2.61
% %(218) a \ (219) b \ (220) c R. $\sqrt[5]{b^7}$ \ (221) d \ (222) e\ (223) f \ (224) g \ (225) h R. $\sqrt b$

% \paragraph{\ref{ese:2.62}}
% e)~$\sqrt[12]{\frac a{a+3}}$,\quad f)~$\sqrt[8]{\left(\frac{x-1}{x+1}\right)^5}$.
%-----------------------2.62
%(226) a \ (227) b \ (228) c \ (229) d \ (230) e R. $\sqrt[12]{\frac a{a+3}}$ \ (231) f R $\sqrt[8]{\left(\frac{x-1}{x+1}\right)^5}$

% \end{multicols}

% \paragraph{\ref{ese:2.63}}
% a)~$\sqrt[3]{\frac{a-1}{(a+1)^3}}$,\quad b)~$(b-1)^2\sqrt{b+1}$,\quad c)~$2\sqrt[3]{y^2}$,\quad d)~$\sqrt[12]{\frac{(b+1)^3}{b(b-1)}}$,\quad e)~$\sqrt[4]{\frac{(a-1)^2} 2}$,\quad f)~$\frac{x+y}{x+3}$
%-----------------------2.63
%(232) a R. $\sqrt[3]{\frac{a-1}{(a+1)^3}}$
%(233) b R. $(b-1)^2\sqrt{b+1}$
%(234) c R. $2\sqrt[3]{y^2}$
%(235) d R. $\sqrt[12]{\frac{(b+1)^3}{b(b-1)}}$
%(236) e R. $\sqrt[4]{\frac{(a-1)^2} 2}$
%(237) f R. $\frac{x+y}{x+3}$

% \paragraph{\ref{ese:2.64}}
% a)~$\sqrt[3]x$,\quad c)~$(y-1)^2\sqrt{y+1}$,\quad d)~$\sqrt[12]{\frac{a^{11}}{(a^2-1)^6}}$,\quad e)~$\sqrt[24]{\frac{a^{10}b^{10}(a+b)^{11}}{x^{11}}}$,\quad f)~$\sqrt[6]{\frac{2x+1}{2x-1}}$
%-----------------------2.64
%(238) a R. $\sqrt[3]x$
%(239) b
%(240) c R. $(y-1)^2\sqrt{y+1}$
%(241) d R. $\sqrt[12]{\frac{a^{11}}{(a^2-1)^6}}$
%(242) e R. $\sqrt[24]{\frac{a^{10}b^{10}(a+b)^{11}}{x^{11}}}$
%(243) f R. $\sqrt[6]{\frac{2x+1}{2x-1}}$

% \paragraph{\ref{ese:2.65}}
% a)~$\sqrt[3]{\frac{a-1}{(a+1)^2}}$,\; b)~$\sqrt[6]{\frac{27a^3}{a-3}}$,\; c)~$\sqrt[6]{\frac{a-1}{a(a+1)^3}}$,\; d)~$\sqrt{2y-1}$,\; e)~$\sqrt[6]{4a^2(2a-1)}$,\; f)~$\frac 3{5a}\sqrt{5a+1}$
%-----------------------2.65
%(244) a R. $\sqrt[3]{\frac{a-1}{(a+1)^2}}$
%(245) b R. $\sqrt[6]{\frac{27a^3}{a-3}}$
%(246) c R. $\sqrt[6]{\frac{a-1}{a(a+1)^3}}$
%(247) d R. $\sqrt{2y-1}$
%(248) e R. $\sqrt[6]{4a^2(2a-1)}$
%(249) f R. $\frac 3{5a}\sqrt{5a+1}$

% \paragraph{\ref{ese:2.66}}
% a)~$\frac{(1-y)^2} y\sqrt[3]{x-y}$,\; b)~$\frac{(1+2x)^2}{2x}\sqrt{x^2+xy+y^2}$,\; c)~$\sqrt a$,\; d)~$\frac{(x+y)^2}{xy}\sqrt{x-3y}$,\; e)~$(1+a)^2$.
% %-----------------------2.66
% %(250) a R. $\frac{(1-y)^2} y\sqrt[3]{x-y}$
% %(251) b R. $\frac{(1+2x)^2}{2x}\sqrt{x^2+\mathit{xy}+y^2}$
% %(252) c R. $\sqrt a$ \vspazio\ovalbox{\risolvii \ref{ese:2.68}, \ref{ese:2.69}, \ref{ese:2.70}, \ref{ese:2.71}, \ref{ese:2.72}, \ref{ese:2.73}, \ref{ese:2.74}, \ref{ese:2.75}, \ref{ese:2.76}}
% %(253) d R. $\frac{(x+y)^2}{\mathit xy}\sqrt{x-3y}$
% %(254) e R. $(1+a)^2$
% %(255) f

\begin{esercizio}[\Ast]
 \label{ese:2.67}
Esegui trasformando i radicali in potenze con esponente frazionario.
 \begin{enumeratea}
 \item $\sqrt{a\sqrt[3]{a\sqrt[3]{a^2}}}\cdot 
        \sqrt[3]{a\sqrt[3]{\frac 1 a}}:\sqrt{\frac 1 a}$
  \hfill $\left[\sqrt{a^3}\right]$
 \item $\sqrt[5]{a\sqrt{a^3}}\cdot 
        \sqrt{a\sqrt[7]{\frac 1{a^2}}}:\sqrt[7]{a^4\sqrt a}$
  \hfill $\left[\sqrt[14]{a^3}\right]$
 \item $\sqrt[3]{a\sqrt a}\cdot \sqrt[3]{a\sqrt[3]a}\cdot 
        \sqrt{a\sqrt[3]a}\cdot \sqrt[3]{a\sqrt a}$
  \hfill $\left[\sqrt[9]{a^{19}}\right]$
 \item $\sqrt[5]{b\sqrt[3]{b^2}}\cdot \sqrt{b^2\sqrt{b\sqrt{b^2}}}:
        \sqrt[5]{b^4\sqrt[3]{b^2}}\cdot \sqrt b$
  \hfill $\left[\sqrt[5]{b^7}\right]$
 \end{enumeratea}
\end{esercizio}

% \subsubsection*{2.10 - Razionalizzazione del denominatore di una frazione}
\subsubsection*{\numnameref{sec:radici_razionalizzazione}}

\begin{esercizio}[\Ast]
 \label{ese:2.68}
Razionalizza i seguenti radicali.
 \begin{multicols}{4}
 \begin{enumeratea}
 \item $\frac 1{\sqrt 3}$
  \hfill $\left[...\right]$
 \item $\frac 2{\sqrt 2}$
  \hfill $\left[...\right]$
 \item $\frac 5{\sqrt{10}}$
  \hfill $\left[...\right]$
 \item $\frac{10}{\sqrt 5}$
  \hfill $\left[2\sqrt 5\right]$
 \item $-\frac 2{\sqrt 3}$
  \hfill $\left[...\right]$
 \item $\frac 4{2\sqrt 2}$
  \hfill $\left[...\right]$
 \item $\frac 3{\sqrt{27}}$
  \hfill $\left[...\right]$
 \item $\frac 4{\sqrt 8}$
  \hfill $\left[\sqrt 2\right]$
%  \item $-\frac{10}{5\sqrt 5}$
%   \hfill $\left[...\right]$
%  \item $\frac 2{3\sqrt 6}$
%   \hfill $\left[\frac{\sqrt 6} 9\right]$
%  \item $-\frac 3{4\sqrt 5}$
%   \hfill $\left[...\right]$
%  \item $\frac 1{\sqrt{50}}$
%   \hfill $\left[...\right]$
 \item $\frac 9{\sqrt{18}}$
 \item $\frac 7{\sqrt{48}}$
 \item $\frac 3{\sqrt{45}}$
 \item $\frac 5{\sqrt{125}}$
 \item $\frac 6{5\sqrt{120}}$
 \item $\frac 1{3\sqrt{20}}$
 \item $\frac{\sqrt 2}{5\sqrt{50}}$
 \item $3\frac{\sqrt 3}{2\sqrt{324}}$
%  \item $\frac 2{\sqrt{2\sqrt 2}}$
%  \item $\frac a{\sqrt a}$
%  \item $\frac x{\sqrt x}$
%  \item $\frac{ax}{\sqrt{2a}}$
 \item $\frac{2a}{\sqrt 2}$
  \hfill $\left[...\right]$
 \item $\frac a{2\sqrt a}$
  \hfill $\left[...\right]$
 \item $\frac x{3\sqrt{2x}}$
  \hfill $\left[\frac{\sqrt{2x}} 6\right]$
 \item $\frac{x^2}{a\sqrt x}$
  \hfill $\left[...\right]$
%  \item $\frac{3x}{\sqrt{12x}}$
%   \hfill $\left[...\right]$
%  \item $\frac{1+\sqrt 2}{\sqrt 2}$
%   \hfill $\left[...\right]$
%  \item $\frac{2-\sqrt 2}{\sqrt 2}$
%   \hfill $\left[...\right]$
%  \item $\frac{\sqrt 2+\sqrt 3}{\sqrt 3}$
%   \hfill $\left[...\right]$
%  \item $\frac{\sqrt 2-\sqrt 3}{\sqrt 6}$
%   \hfill $\left[...\right]$
%  \item $\frac{\sqrt 3+2}{2\sqrt 3}$
%   \hfill $\left[...\right]$
%  \item $\frac{\sqrt 3-1}{3\sqrt 3}$
%   \hfill $\left[...\right]$
%  \item $\frac{\sqrt 6+2\sqrt 3}{\sqrt 3}$
%   \hfill $\left[...\right]$
 \end{enumeratea}
 \end{multicols}
\end{esercizio}

% \paragraph{\ref{ese:2.71}}
% c)~$\frac{\sqrt 2+2} 2$%,\quad l)~$\frac 2 3\sqrt[3]{36}$.

% \begin{esercizio}[\Ast]
%  \label{ese:2.71}
% Razionalizza i seguenti radicali.
%  \begin{multicols}{4}
%  \begin{enumeratea}
%  \item $\frac{\sqrt 5-5\sqrt 2}{\sqrt{10}}$;
%  \item $\frac{\sqrt{16}+\sqrt{40}}{\sqrt 8}$;
%  \item $\frac{\sqrt{10}+\sqrt{20}}{2\sqrt 5}$;
%  \item $\frac{9-\sqrt 2}{\sqrt 2}$;
%  \item $\frac{3a-\sqrt 3}{2\sqrt 5}$;
%  \item $\frac{a^2-b^2}{\sqrt{a+b}}$;
%  \item $\frac{\sqrt{x-y}}{\sqrt{x^2-y^2}}$;
%  \item $\frac x{\sqrt{2x+1}}$;
% %  \item $\frac 1{\sqrt[3]2}$;
% %  \item $\frac 2{\sqrt[3]4}$;
% %  \item $\frac 3{\sqrt[3]5}$;
% %  \item $\frac 4{\sqrt[3]6}$.
%  \end{enumeratea}
%  \end{multicols}
% \end{esercizio}

\begin{esercizio}
 \label{ese:2.72}
Razionalizza i seguenti radicali.
 \begin{multicols}{4}
 \begin{enumeratea}
%  \item $\frac 1{\sqrt[3]2}$
%   \hfill $\left[...\right]$
%  \item $\frac 2{\sqrt[3]4}$
%   \hfill $\left[...\right]$
%  \item $\frac 3{\sqrt[3]5}$
%   \hfill $\left[...\right]$
%  \item $\frac 4{\sqrt[3]6}$
%   \hfill $\left[...\right]$
%  \item $\frac 2{3\sqrt[3]2}$
%   \hfill $\left[...\right]$
%  \item $\frac 6{5\sqrt[3]{100}}$
%   \hfill $\left[...\right]$
%  \item $\frac 2{\sqrt[5]9}$
%   \hfill $\left[...\right]$
%  \item $\frac 3{2\sqrt[6]{27}}$
%   \hfill $\left[...\right]$
%  \item $\frac{10}{\sqrt[5]{125}}$
%  \item $\frac{16}{\sqrt[3]{36}}$
%  \item $\frac 9{\sqrt[4]{2025}}$
%  \item $\frac 1{\sqrt[5]{144}}$
%  \item $\frac{ab}{\sqrt[3]{a^2b}}$
%   \hfill $\left[\sqrt[3]{a^2b}\right]$
%  \item $\frac{ab^2}{\sqrt[3]{ab^2}}$
%   \hfill $\left[...\right]$
%  \item $\frac{3a^2b}{\sqrt[4]{9ab^3}}$
%   \hfill $\left[...\right]$
%  \item $\frac{2\sqrt a}{\sqrt[4]{27ab^2c^5}}$
%   \hfill $\left[...\right]$
%  \item $\frac{5x}{\sqrt[3]{x\sqrt 5}}$
%   \hfill $\left[...\right]$
%  \item $\frac{2\sqrt 2}{\sqrt[5]{16a^2b^3c^4}}$
%   \hfill $\left[...\right]$
%  \item $\frac{\sqrt[3]{x^2y}+\sqrt[3]{xy^2}}{\sqrt[3]{xy}}$
%   \hfill $\left[...\right]$
%  \item $\frac{3-a\sqrt[3]9}{\sqrt[3]{9a}}$
%   \hfill $\left[...\right]$
 \item $\frac{1-\sqrt[3]a}{\sqrt[3]{4a^2x}}$
  \hfill $\left[...\right]$
 \item $\frac 1{\sqrt 3+\sqrt 2}$
  \hfill $\left[...\right]$
 \item $\frac 1{\sqrt 2-\sqrt 3}$
  \hfill $\left[...\right]$
 \item $\frac 2{\sqrt 3+\sqrt 5}$
  \hfill $\left[...\right]$
 \item $\frac{2\sqrt 2}{\sqrt 5+\sqrt 7}$
  \hfill $\left[...\right]$
 \item $\frac 3{\sqrt 2+1}$
  \hfill $\left[...\right]$
 \item $\frac 2{\sqrt 2-1}$
  \hfill $\left[...\right]$
 \item $\frac{\sqrt 3+1}{\sqrt 3-1}$
  \hfill $\left[...\right]$
 \item $\frac{2+\sqrt 3}{\sqrt 3+\sqrt 2}$
  \hfill $\left[...\right]$
 \item $\frac 3{2+3\sqrt 3}$
  \hfill $\left[...\right]$
 \item $\frac x{\sqrt x+1}$
  \hfill $\left[...\right]$
 \item $\frac 1{\sqrt x+\sqrt y}$
  \hfill $\left[...\right]$
 \item $\frac{\sqrt x}{\sqrt x-\sqrt y}$
  \hfill $\left[279\right]$
 \item $\frac{a+b}{\sqrt a+\sqrt{ab}}$
  \hfill $\left[...\right]$
%   \hfill $\left[3-2\sqrt 2+2\sqrt 3-\sqrt 6\right]$
 \item $\frac x{\sqrt y-\sqrt{x+y}}$
  \hfill $\left[...\right]$
 \item $\frac{\sqrt 2-1}{\sqrt{3-\sqrt 3}}$
  \hfill $\left[...\right]$
 \end{enumeratea}
 \end{multicols}
\end{esercizio}

% \begin{esercizio}
%  \label{ese:2.75}
% Razionalizza i seguenti radicali.
%  \begin{multicols}{4}
%  \begin{enumeratea}
%  \item $\frac 1{\sqrt{\sqrt 2}+1}$;
%  \item $\frac 7{\sqrt{7+2\sqrt 6}}$;
%  \item $\frac{a-2}{\sqrt a-2}$;
%  \item $\frac{a-x}{\sqrt a-2\sqrt x}$;
%  \item $\frac{x+1}{\sqrt{x(x+1)}}$;
%  \item $\frac 4{\sqrt 5+\sqrt 3-\sqrt 2}$;
%  \item $\frac{-3}{\sqrt 2-\sqrt 3+1}$;
%  \item $\frac 2{2\sqrt 3-3\sqrt 2+2}$;
%  \item $\frac{(a+b)^2}{\sqrt a+\sqrt b-\sqrt{ab}}$;
%  \item $\frac 3{\sqrt[3]2+\sqrt[3]9}$;
%  \item $\frac 6{\sqrt[3]3-\sqrt[3]5}$;
%  \item $\frac{\sqrt 6}{\sqrt[3]4+\sqrt[3]9}$.
%  \end{enumeratea}
%  \end{multicols}
% \end{esercizio}

% \begin{esercizio}[\Ast]
%  \label{ese:2.76}
% Razionalizza i seguenti radicali.
%  \begin{multicols}{3}
%  \begin{enumeratea}
%  \item $\frac{\sqrt 2}{2\sqrt[3]2-3\sqrt[3]3}$;
%  \item $\frac{\sqrt 2+1}{\sqrt[3]2-1}$;
%  \item $\frac 3{\sqrt[3]4-\sqrt[3]2}$;
%  \item $\frac{a-4b^2}{\sqrt a-2b}$;
%  \item $\frac 2{\sqrt[3]2-1}$;
%  \item $\frac{\sqrt a}{\sqrt a+1}$;
%  \item $\frac{a-b}{\sqrt a+\sqrt b}$;
%  \item $\frac 1{\sqrt a-\sqrt b}+\frac{3\sqrt a-\sqrt b}{a-b}$;
%  \item $\frac{\sqrt 5}{\sqrt 5+\sqrt 2+\sqrt 3}$;
%  \item $\frac{1-\sqrt 2}{1+\sqrt 2-\sqrt 3}$;
%  \item $\frac{\sqrt 2+\sqrt 3+\sqrt 5}{\sqrt 5-\sqrt 2+\sqrt 3}$;
%  \item $\frac{a+2\sqrt{\mathit{ab}}+b}{\sqrt a+\sqrt b}$.
%  \end{enumeratea}
%  \end{multicols}
% \end{esercizio}

% % \subsubsection*{2.11 - Radicali doppi}
% \subsubsection*{\numnameref{sec:02_radicali_doppi}}
% 
% \begin{esercizio}[\Ast]
%  \label{ese:2.77}
% $a^2-b$ deve essere un quadrato perfetto per applicare la formula di 
% trasformazione.
%  \begin{multicols}{4}
%  \begin{enumeratea}
%  \item $\sqrt{12-\sqrt{23}}$;
%  \item $\sqrt{12+2\sqrt 5}$;
%  \item $\sqrt{15+\sqrt{29}}$;
%  \item $\sqrt{3+\sqrt 5}$;
%  \item $\sqrt{3-\sqrt 8}$;
%  \item $\sqrt{4+2\sqrt 3}$;
%  \item $\sqrt{4-\sqrt 7}$;
%  \item $\sqrt{5+\sqrt{21}}$;
%  \item $\sqrt{6+4\sqrt 2}$;
%  \item $\sqrt{6-3\sqrt 3}$;
%  \item $\sqrt{6+2\sqrt 5}$;
%  \item $\sqrt{6-\sqrt{11}}$.
%  \end{enumeratea}
%  \end{multicols}
% \end{esercizio}
% 
% \begin{esercizio}[\Ast]
%  \label{ese:2.78}
% $a^2-b$ deve essere un quadrato perfetto per applicare la formula di 
% trasformazione.
%  \begin{multicols}{4}
%  \begin{enumeratea}
%  \item $\sqrt{7+3\sqrt 5}$;
%  \item $\sqrt{7+2\sqrt{10}}$;
%  \item $\sqrt{7-\sqrt{33}}$;
%  \item $\sqrt{7+2\sqrt 6}$;
%  \item $\sqrt{7-\sqrt{13}}$;
%  \item $\sqrt{8+2\sqrt{15}}$;
%  \item $\sqrt{8-\sqrt{55}}$;
%  \item $\sqrt{8+4\sqrt 3}$.
%  \end{enumeratea}
%  \end{multicols}
% \end{esercizio}
% \newpage
% \begin{esercizio}


% \paragraph{\ref{ese:2.77}}
% d)~$\frac{\sqrt{10}} 2+\frac{\sqrt 2} 2$.
% %----------------2.77
% %(287) a \ \ \ \ b \ \ \ \ c \ \ \ \ d R. $\frac{\sqrt{10}} 2+\frac{\sqrt 2} 2$ (288) e \ \ \ \ \ \ f \ \ \ \ \ \ g \ \ \ \ h
% %(289) i \ \ \ \ j \ \ \ \ \ \ k \ \ \ \ l
% 
% \paragraph{\ref{ese:2.78}}
% d)~$\sqrt 6+1$.
% %----------------2.78
% %(290) a \ \ \ \ b \ \ \ \ c \ \ \ \ d R. $\sqrt 6+1$ (291) e \ \ \ \ f \ \ \ \ \ \ g \ \ \ \ h

%  \label{ese:2.79}
% $a^2-b$ deve essere un quadrato perfetto per applicare la formula di 
% trasformazione.
%  \begin{multicols}{4}
%  \begin{enumeratea}
%  \item $\sqrt{8-\sqrt{39}}$;
%  \item $\sqrt{8-4\sqrt 7}$;
%  \item $\sqrt{8+\sqrt{15}}$;
%  \item $\sqrt{5+2\sqrt 6}$;
%  \item $\sqrt{\frac{15} 2-\sqrt{\frac{86} 9}}$;
%  \item $\sqrt{\frac 5 2-\sqrt 6}$;
%  \item $\sqrt{\frac 8 5-\sqrt{\frac 7 4}}$;
%  \item $\sqrt{10+\sqrt{19}}$.
%  \end{enumeratea}
%  \end{multicols}
% \end{esercizio}

% \subsubsection*{2.12 - Equazioni, disequazioni, sistemi}
\subsubsection*{\numnameref{sec:radici_equazioni}}

\begin{esercizio}[\Ast]
 \label{ese:2.80}
Risolvi le seguenti equazioni a coefficienti irrazionali.
 \begin{multicols}{2}
 \begin{enumeratea}
 \item $\sqrt 2x=2$
  \hfill $\left[\sqrt 2 \right]$
 \item $\sqrt 2x=\sqrt{12}$
  \hfill $\left[\sqrt 6\right]$
 \item $2x=\sqrt 6$
  \hfill $\left[\frac{\sqrt{6}}{2}\right]$
 \item $\sqrt 2x=\sqrt 6+\sqrt{14}$
  \hfill $\left[...\right]$
 \item $x-\sqrt 3=2\left(x-\sqrt 3\right)$
  \hfill $\left[...\right]$
 \item $2\sqrt 3x-\sqrt 2=\sqrt 2$
  \hfill $\left[\frac{\sqrt 6} 3\right]$
 \item $2x+\sqrt 5=\sqrt 5x+2$
  \hfill $\left[1\right]$
 \item $(1+\sqrt 2)x=\sqrt 2(1-\sqrt 2)$
  \hfill $\left[4-3\sqrt 2\right]$
 \item $\frac{1-x}{\sqrt 2}-\frac x{\sqrt 8}=x-\sqrt 2$
  \hfill $\left[18-12\sqrt 2\right]$
%  \item $2x-\left(x+\sqrt 3\right)\sqrt 2=2x+3\sqrt 5$
%   \hfill $\left[-\frac{2\sqrt 3+3\sqrt{10}} 2\right]$
 \item $\frac{x+1}{\sqrt 2}+\frac{x+\sqrt 2}{\sqrt 2}=\frac{x-1} 2$
  \hfill $\left[-(1+\sqrt 2)\right]$
%  \item $\frac{x+\sqrt 2}{x-\sqrt 2}+\frac{x-\sqrt 2}{x+\sqrt 2}=2$
%   \hfill $\left[...\right]$
 \item $(x+\sqrt 2)^2-(x+\sqrt 3)^2=6$
  \hfill $\left[...\right]$
 \item $2(x-1)^2-\sqrt 2x=1+2x(x-2)$
  \hfill $\left[\emptyset\right]$
 \item $\frac{x-\sqrt 3} 2-\frac{\sqrt 2-3x} 4=2x$
  \hfill $\left[\frac{-7(\sqrt 2+\sqrt 3)} 2\right]$
 \end{enumeratea}
 \end{multicols}
\end{esercizio}

% \begin{esercizio}[\Ast]
%  \label{ese:2.82}
% Risolvi le seguenti equazioni a coefficienti irrazionali.
%  \begin{multicols}{2}
%  \begin{enumeratea}
%  \item $\frac{\sqrt 3}{3x-6}-\frac 1{20-10x}=\sqrt 3+2$;
%  \item $\frac{3x-2}{\sqrt 8x-\sqrt{32}}+\frac{5x}{4\sqrt 3x-8\sqrt 3}=0$.
%  \end{enumeratea}
%  \end{multicols}
% \end{esercizio}

% \begin{esercizio}[\Ast]
%  \label{ese:2.83}
% Risolvi le seguenti disequazioni a coefficienti irrazionali.
%  \begin{multicols}{2}
%  \begin{enumeratea}
%  \item $4x+\sqrt 2<2x-\sqrt 2$;
%  \item $(\sqrt 3+1)-(\sqrt 3+\sqrt 2x)<3\sqrt 2$;
%  \item $x\sqrt 2+\sqrt 5>\sqrt{10}$;
%  \item $3(x-\sqrt 3)<2(x+\sqrt 3)-\sqrt 6$;
%  \item $\frac{x-\sqrt 2} 2\le \frac{2x-\sqrt 3}{\sqrt 2}$.
%  \end{enumeratea}
%  \end{multicols}
% \end{esercizio}

% \paragraph{\ref{ese:2.82}}
% a)~$-\frac{\sqrt 2+2\sqrt 3} 3$,\quad b)~$\frac{\sqrt 2} 2$,\quad c)~$\frac{36+17\sqrt 3}{30}$,\quad d)~$\frac{36-10\sqrt 6}{29}$.
% %-------------2.81
% %(301) a R. $-\frac{\sqrt 2+2\sqrt 3} 3$ (302) b R. $\frac{\sqrt 2} 2$ (303) c R. $\frac{36+17\sqrt 3}{30}$ (304) d R. $\frac{36-10\sqrt 6}{29}$

% \paragraph{\ref{ese:2.83}}
% a)~$x<-\sqrt 2$,\, b)~$x>\frac{\sqrt 2-6} 2$,\, c)~$x>\frac{\sqrt{10}(\sqrt 2-1)} 2$,\, d)~$x<5\sqrt 3-\sqrt 6$,\, e)~$x\ge \frac{4\sqrt 3-4+\sqrt 6-\sqrt 2} 7$.
% ---------------2.83
% (305) a R. $x<-\sqrt 2$ (306) b R. $x>\frac{\sqrt 2-6} 2$ (307) c R. $x>\frac{\sqrt{10}(\sqrt 2-1)} 2$ (308) d R. $x<5\sqrt 3-\sqrt 6$
% (309) e R. $x\ge \frac{4\sqrt 3-4+\sqrt 6-\sqrt 2} 7$

% \newpage

\begin{esercizio}[\Ast]
 \label{ese:2.84}
Risolvi i seguenti sistemi di disequazioni a coefficienti irrazionali.
 \begin{multicols}{2}
 \begin{enumeratea}
 \item $\left\{\begin{array}{l}\sqrt 2x\ge 2\\
 (3-\sqrt 2)x<\sqrt 2 \end{array}\right.$ 
  \hfill $\left[\emptyset\right]$
 \item $\left\{\begin{array}{l}2(x-\sqrt 2)>3x-\sqrt 3\\
 (x-\sqrt 2)^2>(x-\sqrt 3)^2-\sqrt 3 \end{array}\right.$ 
  \hfill $\left[\frac{\sqrt 3-3+\sqrt 2-\sqrt 6} 2<x<\sqrt 3-2\sqrt 2\right]$
 \item $\left\{\begin{array}{l}{\sqrt 2x+\sqrt 3y=5}\\
 {\sqrt 3x+\sqrt 2y=2\sqrt 6} \end{array}\right.$ 
  \hfill $\left[(\sqrt 2;\sqrt 3)\right]$
 \item $\left\{\begin{array}{l}{x-\sqrt 3=2-y}\\
 {x+2=y+\sqrt 3} \end{array}\right.$ 
  \hfill $\left[(\sqrt 3;2)\right]$
%  \item $\left\{\begin{array}{l}{x+2y=\sqrt 2-1}\\
%  {2x-2y=2\sqrt 2} \end{array}\right.$ 
%   \hfill $\left[\left(\sqrt 2+\frac 1 3;-\frac 1 3\right)\right]$
%  \item $\left\{\begin{array}{l}
%         {\frac{2\left(x+\sqrt 3\right)}{\sqrt 2+2\sqrt 3}=\frac y{\sqrt 2}}\\
%         {\frac{2x-y}{2\sqrt 6}=\frac{\sqrt 2} 2} \end{array}\right.$ 
%   \hfill $\left[\left(\sqrt 2+\sqrt 3;2\sqrt 2\right)\right]$
 \item $\left\{\begin{array}{l}x+\sqrt 3y=2\\
 \sqrt 3x-4y=1 \end{array}\right.$ 
  \hfill $\left[(\frac{\sqrt 3+8} 7;~\frac{2\sqrt 3-1} 7)\right]$
 \item $\left\{\begin{array}{l}\sqrt 2x-y=1\\
 2x+\sqrt 2y=0 \end{array}\right.$ 
  \hfill $\left[(\frac{\sqrt 2} 4;~-\frac 1 2)\right]$
 \item $\left\{\begin{array}{l}4x-2\sqrt 5y=\sqrt 2\\
 \sqrt 2x+y=-2 \end{array}\right.$ 
  \hfill $\left[(\frac{5\sqrt 5-11\sqrt 2} 6\frac{10-5\sqrt{10}} 6\right]$
 \item $\left\{\begin{array}{l}\sqrt 3x+4\sqrt 2y=4\\
 \sqrt{12}x+8\sqrt 2y=8 \end{array}\right.$ 
  \hfill $\left[\insR\right]$
 \item $\left\{\begin{array}{l}2x+3\sqrt 2y=2\\
 \sqrt 3x-y=-\sqrt 8 \end{array}\right.$ 
  \hfill $\left[(\frac{2-3\sqrt 6} 5;~\frac{\sqrt 2+2\sqrt 3} 5)\right]$
 \end{enumeratea}
 \end{multicols}
\end{esercizio}

% \begin{esercizio}[\Ast]
%  \label{ese:2.87}
% Risolvi i seguenti sistemi di equazioni a coefficienti irrazionali.
%  \begin{multicols}{2}
%  \begin{enumeratea}
%  \item $\left\{\begin{array}{l}x+y=3\sqrt 5\\
%  \sqrt 8x+2\sqrt 2y=-5\sqrt{11} \end{array}\right.;$
%  \item $\left\{\begin{array}{l}x-3\sqrt 3y=\sqrt{27}\\
%  -\sqrt 3x+\sqrt{243}y=0 \end{array}\right.;$
%  \item $\left\{\begin{array}{l}\sqrt 2x+2y=4\\
%  2x+\sqrt{32}y=-1 \end{array}\right.;$
%  \item $\left\{\begin{array}{l}x-y\sqrt 3=2\\
%  x\sqrt 3-y=1 \end{array}\right..$
%  \end{enumeratea}
%  \end{multicols}
% \end{esercizio}
% 
% \begin{esercizio}[\Ast]
%  \label{ese:2.88}
% Risolvi i seguenti sistemi di equazioni a coefficienti irrazionali.
%  \begin{multicols}{2}
%  \begin{enumeratea}
%  \item $\left\{\begin{array}{l}x-2y\sqrt 2=\sqrt 2\\
%  x\sqrt 2+y=\sqrt 2 \end{array}\right.;$
%  \item $\left\{\begin{array}{l}x\sqrt 2+y=1\\
%  x+y\sqrt 2=0 \end{array}\right.;$
%  \item $\left\{\begin{array}{l}2x+3y\sqrt 2=0\\
%  x+y=\sqrt 8 \end{array}\right.;$
%  \item $\left\{\begin{array}{l}x\sqrt 3+4y\sqrt 2=4\\
%  x\sqrt{12}+8y\sqrt 2=-4 \end{array}\right..$
%  \end{enumeratea}
%  \end{multicols}
% \end{esercizio}

% \begin{esercizio}[\Ast]
%  \label{ese:2.89}
% Risolvi i seguenti sistemi di equazioni a coefficienti irrazionali.
%  \begin{multicols}{2}
%  \begin{enumeratea}
%  \item $\left\{\begin{array}{l}x-3y\sqrt 3=0\\
%  -x\sqrt 3+9y=0 \end{array}\right.;$
%  \item $\left\{\begin{array}{l}x+y=3\sqrt 5\\
%  2x-y=\sqrt 5 \end{array}\right.;$
%  \item $\left\{\begin{array}{l}x\sqrt 2-2y=-1\\
%  x\sqrt 8-y=0 \end{array}\right..$
%  \end{enumeratea}
%  \end{multicols}
% \end{esercizio}

% \paragraph{\ref{ese:2.87}}
% a)~$\emptyset$,\quad b)~$(\frac{9+9\sqrt 3} 2;\frac{1+\sqrt 3} 2)$,\quad c)~$(\frac 1 2+4\sqrt 2;-2-\frac{\sqrt 2} 4)$,\quad d)~$(\frac{\sqrt 3} 2-1;\frac 1 2-\sqrt 3)$.
% %---------------2.87
% %(318) a impossibile
% %(319) b R. $\left(\frac{9+9\sqrt 3} 2;\frac{1+\sqrt 3} 2\right)$
% %(320) c R. $\left(\frac 1 2+4\sqrt 2;-2-\frac{\sqrt 2} 4\right)$
% %(321) d R. $\left(\frac{\sqrt 3} 2-1;\frac 1 2-\sqrt 3\right)$

% \paragraph{\ref{ese:2.88}}
% a)~$(\frac{\sqrt 2+4} 5;\frac{\sqrt 2-2} 5)$,\quad b)~$(\sqrt 2;-1)$,\quad c)~$(-\frac{4\sqrt 2+12} 7;\frac{18\sqrt 2+12} 7)$,\quad d)~$\emptyset$.
% %---------------2.88
% %(322) a R. $\left(\frac{\sqrt 2+4} 5;\frac{\sqrt 2-2} 5\right)$
% %(323) b R. $\left(\sqrt 2;-1\right)$
% %(324) c R. $\left(-\frac{4\sqrt 2+12} 7;\frac{18\sqrt 2+12} 7\right)$
% %(325) d R. impossibile

% \paragraph{\ref{ese:2.89}}
% a)~$\insR$,\quad b)~$(\frac{4\sqrt 5} 3;\frac{5\sqrt 5} 3)$,\quad c)~$(\frac{\sqrt 2} 6;\frac 2 3)$.
% %---------------2.89
% %(326) a R. indeterminato
% %(327) b R. $\left(\frac{4\sqrt 5} 3;\frac{5\sqrt 5} 3\right)$
% %(328) c R. $\left(\frac{\sqrt 2} 6;\frac 2 3\right)$
% \paragraph{\ref{ese:2.106}}
% a)~$-1$,\quad b)~$2\cdot (3\sqrt 2-2\sqrt 3)$.
%--------------2.106
%Risolvi le equazioni
%(345) a \ \ \ \ $\mathit{R.}[-1]$
%(346) b \ \ \ \ $\mathit{R.}[2\cdot (3\sqrt 2-2\sqrt 3)]$

\subsection*{Esercizi di riepilogo}

\begin{esercizio}
 \label{ese:2.90}
Vero o Falso? È dato un quadrato di lato $3\sqrt 2$.

\TabPositions{11.5cm}
 \begin{enumeratea}
 \item Il suo perimetro è in numero irrazionale \tab\boxV\quad\boxF
 \item La sua area è un numero irrazionale\tab\boxV\quad\boxF
 \end{enumeratea}
\end{esercizio}

\begin{esercizio}
 \label{ese:2.91}
Vero o Falso? È dato un rettangolo di base $\sqrt{12}$ e altezza $14$

\TabPositions{11.5cm}
 \begin{enumeratea}
 \item il suo perimetro è un numero irrazionale \tab\boxV\quad\boxF
 \item la sua area è un numero razionale \tab\boxV\quad\boxF
 \item il perimetro non esiste perché non si sommano razionali con 
  irrazionali \tab\boxV\quad\boxF
 \item la misura del perimetro è un numero sia razionale che 
  irrazionale \tab\boxV\quad\boxF
 \end{enumeratea}
\end{esercizio}

\begin{esercizio}%2.92
Vero o Falso? Un triangolo rettangolo ha i cateti lunghi 
rispettivamente $\sqrt 3\unit{cm}$ e $\sqrt{13}\unit{cm}$

\TabPositions{11.5cm}
 \begin{enumeratea}
 \item l'ipotenusa ha come misura un numero razionale \tab\boxV\quad\boxF
 \item il perimetro è un numero irrazionale \tab\boxV\quad\boxF
 \item l'area è un numero irrazionale \tab\boxV\quad\boxF
 \end{enumeratea}
\end{esercizio}

\begin{esercizio}%2.93
Vero o Falso? È dato un quadrato di lato $1+\sqrt 5$

\TabPositions{11.5cm}
 \begin{enumeratea}
 \item la misura della diagonale è un numero irrazionale \tab\boxV\quad\boxF
 \item l'area è un numero irrazionale \tab\boxV\quad\boxF
 \end{enumeratea}
\end{esercizio}

\begin{esercizio}%2.94
Vero o Falso? È dato un rettangolo di base $\sqrt{12}$ e altezza $\sqrt 3$

\TabPositions{11.5cm}
 \begin{enumeratea}
 \item il perimetro è un numero irrazionale \tab\boxV\quad\boxF
 \item l'area è un numero irrazionale \tab\boxV\quad\boxF
 \item la misura della diagonale è un numero irrazionale \tab\boxV\quad\boxF
 \item il quadrato della misura del perimetro è un numero 
  irrazionale \tab\boxV\quad\boxF
 \end{enumeratea}
\end{esercizio}

\begin{esercizio}%2.95
Un triangolo rettangolo ha un cateto lungo $7\unit{cm}$ Determina, se esiste, 
una possibile misura dell'altro cateto in modo che questa sia un numero irrazionale e che l'ipotenusa sia, invece, un numero razionale.
\end{esercizio}

\begin{esercizio}%2.96
Perché l'uguaglianza $\sqrt{(-5)^2}=-5$ è falsa?
\end{esercizio}

\begin{esercizio}%2.97
Determina il valore di verità delle seguenti affermazioni.
\begin{enumeratea}
 \item la radice terza del triplo di $a$ è uguale ad $a$
 \item dati due numeri reali positivi, il quoziente delle loro radici 
  quadrate è uguale alla radice quadrata del quoziente;
 \item il doppio della radice quadrata di $a$ è uguale alla radice quadrata 
  del quadruplo di $a$
 \item dati due numeri reali positivi, la somma delle loro radici cubiche è 
  uguale alla radice cubica della loro somma;
 \item la radice cubica di $2$ è la metà della radice cubica di $8$
 \item dati un numero reale positivo, la radice quadrata della sua radice 
  cubica è uguale alla radice cubica della sua radice quadrata;
 \item sommando due radicali letterali simili si ottiene un radicale che ha la 
  stessa parte letterale dei radicali dati.
\end{enumeratea}
\end{esercizio}

\begin{esercizio}%2.98
Riscrivi in ordine crescente i radicali $\sqrt 5$, $4\sqrt 2$, $2\sqrt 3$,
\end{esercizio}

% \begin{esercizio}%2.99
% Verifica che il numero irrazionale $\sqrt{7-2\sqrt 6}$ appartiene 
% all'intervallo $(1; 2)$ e rappresentalo sull'asse dei numeri reali.
% \end{esercizio}
% 
% \begin{esercizio}%2.100
% Dati i numeri\quad $\alpha =\sqrt[3]{(\sqrt{30}-\sqrt 3)\cdot 
% (\sqrt{30}+\sqrt 3)}+\sqrt[4]{(7\sqrt 2-\sqrt{17})\cdot 
% (7\sqrt 2-\sqrt{17})}$\quad e $\beta =(3+\sqrt 5)\cdot 
% (3-\sqrt 5)-\frac 3{2+\sqrt 5}$, quali affermazioni sono vere?
% \begin{multicols}{2}
% \begin{enumeratea}
%  \item sono entrambi irrazionali;
%  \item solo $\alpha$ è irrazionale;
%  \item $\alpha$ è minore di $\beta$
%  \item $\alpha$ è maggiore di $\beta$
%  \item $\beta$ è irrazionale negativo.
% \end{enumeratea}
% \end{multicols}
% \end{esercizio}

% \begin{esercizio}%2.101
% Le misure rispetto al cm dei lati di un rettangolo sono i numeri 
% reali $l_1=\sqrt[3]{1-\frac 1 8}\cdot \sqrt[3]{1-\frac 2 7}\cdot 
% \sqrt[3]{25}$ e 
% $l_2=\sqrt{\sqrt 2}\cdot \sqrt[4]3\cdot 
% (\sqrt[8]6)^3:\sqrt[4]{\sqrt 6}$. 
% Determinare la misura del perimetro e della diagonale del rettangolo.
% \end{esercizio}
% 
% \begin{esercizio}%2.102
% Se $x$ è positivo e diverso da $1$, l'espressione 
% $E=\sqrt[4]{\frac 4{\sqrt x-1}-\frac 4{\sqrt x+1}}:
% \sqrt[4]{\frac 4{\sqrt x-1}+\frac 4{\sqrt x+1}}$ è uguale a:
% \begin{multicols}{5}
% \begin{enumeratea}
%  \item $\sqrt[4]{\frac 1 x}$;
%  \item $\sqrt[8]{\frac 1 x}$;
%  \item $\frac 1 x$;
%  \item $\sqrt[8]x$;
%  \item $0$.
% \end{enumeratea}
% \end{multicols}
% \end{esercizio}
% 
% \begin{esercizio}%2.103
% Stabilire se la seguente affermazione è vera o falsa. Per tutte le coppie 
% $(a,b)$ di numeri reali positivi con $a=3b$, l'espressione 
% $E=\frac{\sqrt a+\sqrt b}{\sqrt a-\sqrt b}+
% \frac{\sqrt a-\sqrt b}{\sqrt a+\sqrt b}-\frac{a+b}{a-b}$ 
% ha il numeratore doppio del denominatore.
% \end{esercizio}

\begin{esercizio}%2.104
Calcola il valore delle seguenti espressioni letterali per i valori indicati 
delle lettere.
\begin{multicols}{3}
\begin{enumeratea}
\item $x+2\sqrt 3$ per $x=\sqrt 3$
\item $\sqrt 2x+3\sqrt 6$ per $x=\sqrt{3}$
\item $x^2+x-1$ per $x=\sqrt 2$
\item $x^2+\sqrt 5x-1$ per $x=\sqrt 5$
\item $(x+2\sqrt 2)^2$ per $x=\sqrt 2$
\end{enumeratea}
\end{multicols}
\end{esercizio}

% \begin{esercizio}%2.105
% Trasforma in un radicale di indice $9$ il seguente radicale 
% $\sqrt[3]{\frac{\sqrt{\frac a b-\frac b a}}{\sqrt{\frac a b+\frac b a+2}}:
% \sqrt{\frac{a+b}{a-b}}+1}$.
% \end{esercizio}

% \begin{esercizio}[\Ast]
% \label{ese:2.106}
% Risolvi le seguenti equazioni.
% \begin{multicols}{2}
%  \begin{enumeratea}
%  \item $\frac{x\sqrt 2-\sqrt 3}{\sqrt 2+\sqrt 3}+
% \frac{x\sqrt 2+\sqrt 3}{\sqrt 3-\sqrt 2}=\frac{3x+3}{\sqrt 3}$;
%  \item $\frac{\sqrt 3+x}{x-\sqrt 3}+\frac{x+\sqrt 2}{x-\sqrt 2}=2$.
%  \end{enumeratea}
% \end{multicols}
% \end{esercizio}

\begin{esercizio}%2.107
Per quale valore di $k$ il sistema lineare è determinato?
$\left\{\begin{array}{l}{x\sqrt 3+(k-\sqrt 3)y=1}\\
 {-2x+y\sqrt 6=-k} \end{array}\right..$
\end{esercizio}

\begin{esercizio}%2.108
L'insieme di soluzioni della disequazione $(\sqrt 2-\sqrt 3)x<0$ è:
\begin{multicols}{5}
 \begin{enumeratea}
 \item $x\ge 0$
 \item $x\le 0$
 \item $x>0$
 \item $x<0$
 \item $\insR$
 \end{enumeratea}
\end{multicols}
\end{esercizio}

\begin{esercizio}%2.109
Data l'espressione 
$E=\frac{2a-2\sqrt 2}{\sqrt 2}+\frac{(a+2)\cdot \sqrt 2} 2+\frac 4{\sqrt 2}-1$, 
stabilire se esistono valori di $a$ che la rendono positiva.
\end{esercizio}

\begin{esercizio}%2.110
Data la funzione $f(x)=\frac{\sqrt{x+1}}{\sqrt{x+1}-\sqrt{x-1}}$
 \begin{enumeratea}
 \item determina il suo dominio;
 \item riscrivi la funzione razionalizzando il denominatore;
 \item calcola $f(2)$
 \item per quali valori di $x$ si ha $f(x)>0$?;
 \item risolvi l'equazione $f(x)=0$
 \end{enumeratea}
\end{esercizio}
