% (c) 2015 Daniele Zambelli daniele.zambelli@gmail.com

\section{Esercizi}

\subsection{Esercizi dei singoli paragrafi}

\subsubsection*{\numnameref{sec:coniche_e_retta}}

\begin{esercizio}
  \label{ese:div.003}
  Considera la conica data e stabilisci se le rette al suo fianco 
sono secanti, tangenti od esterne.
  \begin{enumeratea}

\item $y=3 x^{2} +6x-4;~r:~y=2x+3,~s:~y=\dfrac{1}{4}x-8,~t:~y=-3x-1$\\ 
\hfill  $\left[r~secante,~s ~esterna,~ t~secante\right]$
\item $y=-x^{2}+2x+4;~r:~y=4x+5,~s:~y=3x+1,~t:~y=-2x+8$\\
\hfill $\left[r~tangente,~s~esterna,~t~tangente\right]$
\item $\dfrac{x^{2}}{18} 
+\dfrac{y^{2}}{36}=1;~r:~y=3x-2,~s:~y=-6,~t:~y=-2x+8$\\
\hfill $\left[r~secante,~s~tangente,~t~esterna\right]$
\item $\dfrac{x^{2}}{25}+\dfrac{y^{2}}{4}=1;~r:~y=-2x+1,~s:~y=3x,~t:~y= 
\dfrac{x}{2} +6$\\
\hfill $\left[r~secante,~s~secante,~t~esterna\right]$
\item $ x^{2}+y^{2}=4;~r:~y=-x-1,~s:~y=x+3,~t:~x+\sqrt{3}y=4$\\
\hfill $\left[r~secante,~s~esterna,~t~tangente\right]$
\item $4 x^{2}-5 y^{2}=20;~r:~y=~-x-1,~s:~y=3x,~t:~y=3x+7$\\
\hfill $\left[r~tangente,~s~esterna,~t~secante\right]$
\item $\dfrac{x^{2}}{9}-\dfrac{y^{2}}{4}=1;~r:~5x-6y-9=0,~s:~x=4,~t:~ 
x-3y-3=0$\\
\hfill $\left[r~tangente,~s~secante,~t~secante\right]$
\item $x^{2}+y^{2}-4x+2y=0;~r:~x+2y-5=0,~s:~x+4y-5=0,~t:~-x+2y-8=0$\\
\hfill $\left[r~tangente,~s~secante,~t~esterna\right]$
\end{enumeratea}
\end{esercizio}

\begin{esercizio}
  \label{ese:div.003}
  Determina i punti di intersezione tra le coniche e le rette 
sottostanti, dopo aver verificato che la retta è tangente alla conica.
  \begin{enumeratea}
  \item $ x^{2}+4y^{2}=1 $, $y=x+1$
  \hfill$\left[P_{1}\left( -\dfrac{3}{5};~\dfrac{2}{5} \right);~ 
P_{2}\left(-1;~0\right)\right]$
  \item  $4x^{2}+y^{2}=4 $, $y=x+2$
  \hfill$\left[P_{1}\left( -\dfrac{4}{5};~\dfrac{6}{5} \right);~ 
P_{2}\left(0;~2\right)\right]$
  \item $5x^{2}-y^{2}=11 $, $y=2$
  \hfill$\left[P_{1}\left( \sqrt{3};~2 \right);~ 
P_{2}\left(-\sqrt{3};~2\right)\right]$
  \item $9x^{2}-25y^{2}=225 $, $y=\dfrac{2}{5}x+2$
  \hfill $\left[P_{1}\left(13;~\dfrac{36}{5} \right);~ 
P_{2}\left(-5;~0\right)\right]$
  \end{enumeratea}
\end{esercizio}

\subsubsection*{\numnameref{sec:coniche_tangenti}}

\begin{esercizio}
  \label{ese:div.003}
  Determina le rette tangenti alla conica indicata passanti per il 
punto A, ad essa esterno.
  \begin{enumeratea}

\item $9x^{2}+4y^{2}=36,~A(2;~5)$  
\hfill $\left[y=\dfrac{4}{5}x+\dfrac{17}{5};~x=2\right]$
\item $x^{2}+2y^{2}=2,~A(2;~1)$
\hfill $\left[y=1;~y=2x+3\right]$
\item $\dfrac{x^{2}}{4}+\dfrac{y^{2}}{5}=1,~A(3;~0)$
\hfill $\left[y=-x+3;~y=x-3\right]$
\item $\dfrac{x^{2}}{16}+\dfrac{y^{2}}{9}=1,~A(6;~-1)$
\hfill $\left[y=-x+5;~y=\dfrac{2}{5}x-\dfrac{17}{5}\right]$
\item $\dfrac{x^{2}}{9}-\dfrac{y^{2}}{4}=1,~A(0;~3)$
\hfill $\left[y=-\dfrac{6}{5}x+3;~y=\dfrac{6}{5}x+3\right]$
\item $\dfrac{x^{2}}{25}-\dfrac{y^{2}}{16}=1,~A(-5;~-2)$ 
\hfill $\left[x=-5;~y=x+3\right]$
\item $x^{2}-2y^{2}=2,~A(1;~2)$
\hfill $\left[y=-5x+7;~y=x+1\right]$
\item $4x^{2}-9y^{2}=144,~A(0;~2)$
\hfill $\left[y=\dfrac{3}{4}x+2;~y=-\dfrac{3}{4}x+2\right]$
\end{enumeratea}
\end{esercizio}

\begin{esercizio}
  \label{ese:div.003}
  Applicando la formula dello sdoppiamento determina la tangente alla 
conica data passante per il suo punto A.
  \begin{enumeratea}
\item $\dfrac{x^{2}}{4}+\dfrac{y^{2}}{9}=1,~A\left(-\dfrac{8}{5};~- 
\dfrac{9}{5}\right)$  
\hfill $\left[y=-2x-5\right]$
\item $\dfrac{x^{2}}{25}+\dfrac{y^{2}}{16}=1,~A\left(-3;~\dfrac{16}{5} 
\right)$
\hfill $\left[y=\dfrac{3}{5}x+5\right]$
\item $x^{2}+3y^{2}=3,~A\left(\dfrac{3}{2};~\dfrac{1}{2} \right)$
\hfill $\left[y=-x+2\right]$
\item $x^{2}+9y^{2}=9,~A(0;~-1)$
\hfill $\left[y=-1\right]$
\item $\dfrac{x^{2}}{9}-\dfrac{y^{2}}{4}=1,~A\left(3\sqrt{2};~2\right)$
\hfill $\left[y=\dfrac{2\sqrt{2}}{3}x-2\right]$
\item $2x^{2}-y^{2}=2,~A(3;~4)$
\hfill $\left[y=\dfrac{3}{2}x-\dfrac{1}{2}\right]$
\item $4x^{2}-3y^{2}=4,~A(2;~2)$
\hfill $\left[y=\dfrac{4}{3}x-\dfrac{2}{3}\right]$
\item $\dfrac{5x^{2}}{16}-\dfrac{y^{2}}{4}=1,~A(2;~1)$
\hfill$\left[y=\dfrac{5}{2}x-4\right]$
\end{enumeratea}
\end{esercizio}


\subsubsection*{\numnameref{sec:coniche_curve_deducibili}}

\begin{esercizio}
  \label{ese:div.003}
  Date le seguenti funzioni irrazionali, identificane la conica che 
ne consente di determinare il grafico e, dopo aver impostato il sistema che 
le definisce, disegnale. 
  \begin{enumeratea}
    \item $ y=\sqrt{9-x} $
    \item $y=\sqrt{4-x^{2}}  $
    \item $ y=\sqrt{9-4x^{2}} $
    \item $ y=\sqrt{4x^{2}-25} $
    \item $ y=3+\sqrt{4x-x^{2}} $
    \item $ y=\sqrt{4-\dfrac{x^{2}}{4}} $
  \end{enumeratea}
\end{esercizio}


% \subsection{Esercizi riepilogativi} TODO
% 
% \begin{esercizio}
% \label{ese:D.19}
% testo esercizio
% \end{esercizio}
% 
% \begin{esercizio}\label{ese:03.1}
% Consegna:
%  \begin{enumeratea}
%   \item  
%  \end{enumeratea}
% \end{esercizio}
