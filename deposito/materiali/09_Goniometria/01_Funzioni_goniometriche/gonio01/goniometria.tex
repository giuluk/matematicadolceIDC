% (c) 2015 Daniele Zambelli daniele.zambelli@gmail.com

\chapter{Goniometria}

\section{Angoli e archi}
\label{sec:gonio_angoli_archi}

% \begin{inaccessibleblock}[Figura: TODO]
%  \begin{figure}[b]
%  \begin{minipage}[t]{.45\textwidth}
% \centering
%  % (c) 2013 Claudio Carboncini - claudio.carboncini@gmail.com
\begin{tikzpicture}[x=8mm, y=8mm ]
\draw[step=0.8cm,color=gray!30] (-3.5,-2.5) grid (3.5,4.5);
  \tkzInit[xmin=-3,xmax=3,ymin=-2.5,ymax=3.5]
  \clip (-3.5,-2.5) rectangle (4,5.5);
  \begin{scope}[font=\small]
    \tkzAxeY[orig = false, label options={left = 1pt}]
    \tkzAxeX[orig = true, label options={below = 1pt}]
  \end{scope}
  \tkzFct[domain=-3:2,thick,color=Maroon]{x*x+x-2};
  \draw[fill=orange] (-2,0)circle (1.5pt);
  \draw[fill=orange] (1,0)circle (1.5pt);

\end{tikzpicture}

% \caption{Esempio~\ref{ex:4.9}.}\label{fig:4.1}
%  \end{minipage}\hfil
%  \begin{minipage}[t]{.45\textwidth}
% \centering
%  % (c) 2013 Claudio Carboncini - claudio.carboncini@gmail.com
% (c) 2014 Dimitrios Vrettos - d.vrettos@gmail.com
\begin{tikzpicture}[x=8mm, y=8mm]

\draw[step=0.8cm,color=gray!30] (-1.5,-.5) grid (5.5,6.5);
  \tkzInit[xmin=-1,xmax=5,ymin=0,ymax=5.5]
  \clip (-1.5,-.5) rectangle (6,7);
  \begin{scope}[font=\small]
    \tkzAxeY[orig = false, label options={left = 1pt}]
    \tkzAxeX[orig = true, label options={below = 1pt}]
  \end{scope}
  \tkzFct[domain=-1:5,thick,color=Maroon]{x*x-4*x+4};
  \draw[fill=orange] (2,0)circle (1.5pt);

\end{tikzpicture}

% \caption{Esempio~\ref{ex:4.10}.}\label{fig:4.2}
%  \end{minipage}
% \end{figure}
% \end{inaccessibleblock}
% 
%  \begin{figure}[!h]
% \begin{inaccessibleblock}[]
% \centering
%  \input{\folder lbr/00.pgf}
%  \caption{}
%  \label{fig:trigo_}
% \end{inaccessibleblock}
% \end{figure}

La goniometria è la parte della matematica che si interessa degli angoli. 
Di angoli ne abbiamo già parlato all'inizio dello studio della geometria, 
la definizione imparata allora diceva all'incirca:

``Angolo è ciascuna delle due parti di piano delimitate da due semirette
che hanno l'origine in comune''.

Questa definizione ci ha permesso di affrontare molti argomenti, ma presenta 
dei limiti, in particolare, con questa definizione non ha senso andare oltre
un angolo giro, o non ha senso parlare di angoli negativi, ma entrambe queste 
possibilità possono esserci utili in molte occasioni. 
Proviamo a vedere gli angoli da un altro punto di vista:

\begin{definizione}
 \emph{Angolo} è la rotazione che porta una semiretta a coincidere con un'altra 
 semiretta avente l'origine in comune.
\end{definizione}

 \begin{figure}[!h]
 \begin{minipage}[t]{.45\textwidth}
\begin{inaccessibleblock}[Rotazione che porta la semiretta a sulla semiretta b]
\centering
 % (c) 2014 Daniele Zambelli - daniele.zambelli@gmail.com

\begin{tikzpicture}[x=10mm,y=10mm, >=latex]

\draw (0, 0) -- (1, 0) node [below right] {$a$} --(2, 0);
\draw (0,0) -- (70:1.2) node [above left] {$b$} --(70:1.8);
\draw[->, ultra thick, orange] (1, 0) arc(0:70:1);

\end{tikzpicture}
 \caption{Angolo $\widehat{ab}$}
 \label{fig:trigo_angolo01}
\end{inaccessibleblock}
 \end{minipage}
 \begin{minipage}[t]{.45\textwidth}
\begin{inaccessibleblock}[Rotazione che porta la semiretta a sulla semiretta b]
\centering
 % (c) 2012 Dimitrios Vrettos - d.vrettos@gmail.com

\begin{tikzpicture}[x=10mm,y=10mm, >=latex]

\draw (0, 0) -- (1, 0) node [below right] {$a$} --(2, 0);
\draw (0,0) -- (70:1.2) node [above left] {$b$} --(70:1.8);
\draw[->, ultra thick, red!50!black] (1, 0) arc(0:-290:1);
 
% \foreach \x/\xtext in {-1/-1,1/1}
% \node[below] at(\x,0) {$\xtext$};
% 
% \foreach \xi in {-1,1}
% \draw (\xi,1.5pt) -- (\xi,-1.5pt);
% 
% \foreach \yi in {-1,1}
% \draw(1.5pt,\yi)--(-1.5pt,\yi);
% 
% \foreach \y/\ytext in {-1/-1,1/1}
% \node[right] at(0,\y) {$\ytext$};
% 
% \begin{scope}[Maroon, thick, ->]
% \foreach \z in {0,35,65,90,130,180,205,270,310}
% \draw[rotate=\z] (0,0)--(1,0 );
% \end{scope}
% \draw[dotted] (0,0) circle (1);
% 
% \node [above left] at (0,1) {$B_1$};
% \node [above left] at (-1,0) {$B_2$};
% \node [below left] at (0,-1) {$B_3$};
% \node [above right] at (1,0) {$B_4$};
% 
% % \begin{scope}[fill=CornflowerBlue, draw=black]
% \filldraw (1,0) circle (1.5pt);
% \filldraw (0,1) circle (1.5pt);
% \filldraw (-1,0) circle (1.5pt);
% \filldraw (0,-1) circle (1.5pt);
% \end{scope}
\end{tikzpicture}

 \caption{Angolo $\widehat{ab}$}
 \label{fig:trigo_angolo02}
\end{inaccessibleblock}
 \end{minipage}
\end{figure}

La rotazione che porta $a$ su $b$ vedi figura~\ref{fig:trigo_angolo01} non è 
l'unica rotazione: $a$ potrebbe andare su $b$ anche girando dall'altra 
parte, figura~\ref{fig:trigo_angolo02}. Dobbiamo quindi distinguere le due 
rotazioni.
Per definire gli angoli in questo nuovo modo dobbiamo quindi stabilire quale è 
il verso positivo e quale quello negativo di una rotazione. 
Ci si è accordati di definire come verso positivo quello antiorario e di 
definire come verso negativo quello orario.

\begin{wrapfigure}{r}{0.5\textwidth} 
 \vspace{-6pt}
  \begin{center}
  \begin{minipage}{.3\textwidth}
\begin{inaccessibleblock}[Una freccia curva in verso antiorario con un segno 
 più, una in verso orario con un segno meno.]
    % (c) 2012 Dimitrios Vrettos - d.vrettos@gmail.com

\begin{tikzpicture}[x=20mm,y=20mm, >=latex]

\begin{scope}[->, very thick, orange!50!black]
\draw (+1, 0) arc (20:70:1.3);
\draw (-1, 0) arc (160:110:1.3);
\end{scope}

\draw [black] (.8, .6) node {$+$};
\draw [black] (-.8, .6) node {$-$};

\end{tikzpicture}

    \caption{Verso di rotazione}
    \label{fig:trigo_verso_rotazioni}
\end{inaccessibleblock}
  \end{minipage}
  \end{center}
   \vspace{-12pt}
  \vspace{1pt}
\end{wrapfigure} 

Stabilito il verso, dobbiamo poter misurare l'ampiezza della rotazione. 
Da diversi millenni si è usato il grado e i suoi sottomultipli. 
Questa unità di misura è legata al tempo (un grado è all'incirca la rotazione 
che la Terra compie attorno al Sole in un giorno) e come le ore ha dei 
sottomultipli che sono sessantesimi dell'unità di misura fondamentale. 
Il sottomultiplo primo è un sessantesimo del grado, il sottomultiplo secondo 
è un sessantesimo del primo. 
Tutto ciò è piuttosto scomodo e noi lavoreremo solo in gradi espressi 
eventualmente con la virgola. Ma c'è un altro modo per esprimere l'ampiezza 
di un angolo: l'arco. Per capirlo osserviamo la figura:

\begin{wrapfigure}{l}{0.5\textwidth} 
 \vspace{-6pt}
  \begin{minipage}{.3\textwidth}
  \begin{center}
\begin{inaccessibleblock}[Una freccia curva in verso antiorario con un segno 
 più, una in verso orario con un segno meno.]
    % (c) 2012 Dimitrios Vrettos - d.vrettos@gmail.com

\begin{tikzpicture}[x=20mm,y=20mm, >=latex]

% Griglia
\draw[gray!50, very thin, step=1] (-1.2, -1.2) grid (1.2, 1.2);

% (c) 2014 Daniele Zambelli - daniele.zambelli@gmail.com

%%%
% Circonferenza goniometrica
%%%%
% 
% % Griglia
% \draw[gray!50, very thin, step=1] (-1.2, -1.2) grid (1.2, 1.2);

%Asse x
\draw [-{Stealth[length=2mm, open, round]}] (-1.3,0) -- (1.3,0) node [below]
      {$x$};
%Asse y
\draw [-{Stealth[length=2mm, open, round]}] (0, -1.3) -- (0, 1.3) node [left]  
      {$y$};
%Circonferenza
\draw [very thin] (0,0) circle (1);


\begin{scope}[->, very thick]
\draw [Maroon!50!black, rotate=65] (0,0)--(1,0 );
\draw [blue!50!black] (0, 0) (20:1.3) arc (20:70:1.3);
\draw [red!50!black] (0, 0) (-20:1.3) arc (-20:-70:1.3);
\draw [green!50!black] (.3, 0) arc(0:65:.3);
\draw [green!50!black] (1, 0) arc(0:65:1);
\end{scope}

\begin{scope}[black]
\draw (+1.2, +0.8) node {$+$};
\draw (+1.2, -0.8) node {$-$};
\draw (.35, .2) node {$\alpha$};
\draw (1.1, .1) node {$A$};
\draw (0, 0) (65:1.1) node {$B$};
\end{scope}

\end{tikzpicture}

    \caption{Rotazione di un vettore}
    \label{fig:trigo_rotazione_vettore}
\end{inaccessibleblock}
  \end{center}
  \end{minipage}
   \vspace{-12pt}
  \vspace{1pt}
\end{wrapfigure} 

In un riferimento cartesiano consideriamo un vettore di modulo~1 applicato
nell'origine degli assi. Il vettore è libero di ruotare e quindi il suo estremo 
libero descrive una circonferenza di centro~(0;~0) e raggio~1. 

Questa circonferenza è l'insieme di tutti i punti che hanno distanza 
dall'origine uguale a~1. Ricordandoci la formula della distanza tra due punti
nel piano cartesiano possiamo dire che le coordinate di ciascun punto della 
circonferenza soddisfano l'equazione $\sqrt{x^2 + y^2} = 1$.

\providecommand*\wideparen[1]{
  \setbox254\hbox{$#1$}
  \dimen0=\wd254\relax
  \setbox254\hbox{\rotatebox{-90}{(}}
  \vbox{
    \ialign{
     ##\crcr
     \resizebox{\dimen0}{!}{\box254}\crcr
     $\hfil\displaystyle{#1}\hfil$\crcr
    }
  }
}

La direzione del vettore può essere individuata o fornendo l'angolo $\alpha$ o 
fornendo la lunghezza dell'arco~$\wideparen{AB}$.

Ricordandoci la formula della circonferenza ($\mathcal{C}= 2 \pi r$) si può 
vedere facilmente che tra gli angoli e i corrispondenti archi valgono le 
corrispondenze riportate nella seguente tabella:

\begin{center}
\begin{tabular}{cccccccc}
angolo \quad & \quad 0 \quad & \quad 90 \quad & \quad 180 
\quad & \quad 270 \quad & \quad 360 \quad & \quad 
$\alpha$ \quad & \quad $\dfrac{\alpha}{\pi} \cdot 180$ \\

arco \quad & \quad 0 \quad & \quad $\dfrac{1}{2} \pi$ \quad & \quad $\pi$ 
\quad & \quad $\dfrac{3}{2} \pi$ \quad & \quad $2 \pi$ \quad &  \quad
$\dfrac{\alpha}{180} \cdot \pi$ \quad & \quad $\alpha$
\end{tabular}
\end{center}

\section{La circonferenza goniometrica}
\label{sec:gonio_circonferenza_goniometrica}

% \begin{wrapfigure}{l}{0.5\textwidth} 
%  \vspace{-6pt}
%   \begin{minipage}{.48\textwidth}
%   \begin{center}
% \begin{inaccessibleblock}[La circonferenza goniometrica.]
%     % (c) 2014 Daniele Zambelli - daniele.zambelli@gmail.com

\begin{tikzpicture}[scale=2.0, cap=round, >=latex]

% (c) 2014 Daniele Zambelli - daniele.zambelli@gmail.com

\newcommand{\circonferenzagoniometrica}[5]{%
% Colors
  \colorlet{anglecolor}{green!50!black}
  \colorlet{sincolor}{blue!50!black}
  \colorlet{tancolor}{orange!70!black}
  \colorlet{coscolor}{red!50!black}

% Local definitions
  \def \pangle{#1}
  \def \palfangle{\pangle / 2}
  \def \petichetta{#2}
  \def \psin{#3}
  \def \pcos{#4}
  \def \ptan{#5}
  
% Assi, circonferenza 
  % (c) 2014 Daniele Zambelli - daniele.zambelli@gmail.com

%%%
% Circonferenza goniometrica
%%%%
% 
% % Griglia
% \draw[gray!50, very thin, step=1] (-1.2, -1.2) grid (1.2, 1.2);

%Asse x
\draw [-{Stealth[length=2mm, open, round]}] (-1.3,0) -- (1.3,0) node [below]
      {$x$};
%Asse y
\draw [-{Stealth[length=2mm, open, round]}] (0, -1.3) -- (0, 1.3) node [left]  
      {$y$};
%Circonferenza
\draw [very thin] (0,0) circle (1);

% Tangente
  \draw[style=help lines] (1, -1.2) -- (1, 1.8);
% Punti legati al sistema di riferimento
  \coordinate (origin) at (0, 0);
  \draw (origin) node [below left] {$O$} [fill] circle(0.4pt);
  \draw (1, 0) node [below right] {$K$} [fill] circle(0.4pt);
% Parte che dipende dall'angolo
  \begin{scope}[very thick]
% Angolo e archi
    \begin{scope}[anglecolor]
      \draw [anglecolor, rotate=\pangle] (0, 0) -- (2.2, 0);
      \begin{scope}[->]
        \draw (0.3, 0) arc(0:\pangle:0.3);
        \draw (1, 0) arc(0:\pangle:1);
        \draw [-, anglecolor] (\palfangle:.45) node {\petichetta};
      \end{scope}
    \end{scope}
% Seno
    \draw [sincolor]
      (\pangle:1) node (p) [above] {$P$} circle(0.2pt)-- 
      node[left] {\psin} (p |- origin) 
      node (h) {}; 
     \draw (h) node [black, below] {$H$} [fill] circle(0.2pt); 
% Coseno
    \draw [coscolor] (h) -- 
      node [coscolor, below] {\pcos} (origin);
% Tangente
    \draw [tancolor] (1,0) --
      node [right=1pt,fill=white] {\ptan} 
      (intersection of 0,0 -- \pangle:1 and 1,0--1,1) [below right] node {$T$}
        circle(0.2pt);
  \end{scope}
}


\clip(-1.2, -1.3) rectangle (1.5, 1.4);

\circonferenzagoniometrica {40} {$\alpha$} {$\sin \alpha$}
                            {$\cos \alpha$} {$\tan \alpha$}

\end{tikzpicture}

% \begin{tikzpicture}[scale=2.7, cap=round, >=latex]
%   % Local definitions
%   \def\angle{40}
%   \def\alfangle{20}
%   \def\sinangle{0.6427876}
%   \def\cosangle{0.7660444}
% 
%   % Colors
%   \colorlet{anglecolor}{green!50!black}
%   \colorlet{sincolor}{blue!50!black}
%   \colorlet{tancolor}{orange!70!black}
%   \colorlet{coscolor}{red!50!black}
% 
% % (c) 2014 Daniele Zambelli - daniele.zambelli@gmail.com

%%%
% Circonferenza goniometrica
%%%%
% 
% % Griglia
% \draw[gray!50, very thin, step=1] (-1.2, -1.2) grid (1.2, 1.2);

%Asse x
\draw [-{Stealth[length=2mm, open, round]}] (-1.3,0) -- (1.3,0) node [below]
      {$x$};
%Asse y
\draw [-{Stealth[length=2mm, open, round]}] (0, -1.3) -- (0, 1.3) node [left]  
      {$y$};
%Circonferenza
\draw [very thin] (0,0) circle (1);

% 
% \draw (0, 0) node [below left] {$O$} [fill] circle(0.4pt);
% \draw (\cosangle, 0) node [below] {$H$} [fill] circle(0.4pt);
% \draw (1, 0) node [below right] {$K$} [fill] circle(0.4pt);
% 
% \node at (.38, .45) {$1$};
% 
% \draw[style=help lines] (1, -1.2) -- (1, 1.3);
% 
% \begin{scope}[very thick]
%  \begin{scope}[->, green!50!black]
%   \draw (0.3, 0) arc(0:\angle:0.3);
%   \draw (1.0, 0) arc(0:\angle:1.0);
%  \end{scope}
%  \draw (\alfangle:.2) node[anglecolor] {$\alpha$};
%  
%  \draw [sincolor]
%     (\angle:1) node [above] {$P$} circle(0.5pt)-- 
%                node[left=0pt,fill=white] {$\sin \alpha$} +(0, -\sinangle);
% 
%  \draw [coscolor]
%     (0,0) -- node[below=2pt,fill=white] {$\cos \alpha$} (\cosangle, 0);
% 
%   \draw [tancolor] (1,0) --
%     node [right=1pt,fill=white] {$\displaystyle \tan \alpha$} 
%      (intersection of 0,0 -- \angle:1 and 1,0--1,1) [below right] node {$T$}
%       circle(0.5pt); % coordinate (t);
% 
%  \draw [Maroon!50!black, rotate=\angle] (0, 0)--(1.8, 0);
%  
% \end{scope}
% 
% \end{tikzpicture}

%     \caption{Circonferenza goniometrica}
%     \label{fig:trigo_circ_gonio}
% \end{inaccessibleblock}
%   \end{center}
%   \end{minipage}
%    \vspace{-12pt}
% \end{wrapfigure} 

Se i problemi che incontreremo riguardassero solo angoli, allora potremmo
chiudere qui il capitolo, ma di solito i problemi della vita reale coinvolgono
angoli e segmenti sono perciò stati inventati degli \emph{strumenti} matematici 
che permettono di passare dagli angoli a determinati segmenti. Lo strumento di 
base che useremo per passare da angoli a segmenti è la \emph{Circonferenza 
goniometrica} vedi figura~\ref{fig:trigo_circ_gonio}).

\begin{definizione}
 La \emph{Circonferenza goniometrica} è una circonferenza di raggio~1 con 
 centro nell'origine di un piano cartesiano.
\end{definizione}

Ora consideriamo l'angolo $\alpha$ che ha il vertice nell'origine degli assi e 
un lato coincidente con l'asse~$x$. L'altro lato interseca la circonferenza in 
un punto $P$ e interseca la tangente alla circonferenza nel punto~$T$
(vedi figura~\ref{fig:trigo_circ_gonio}
Il segmento $OP$ è lungo ~1 essendo il raggio della circonferenza goniometrica.
Possiamo costruire un triangolo rettangolo che ha l'ipotenusa=$OP$ e i cateti
paralleli agli assi. Chiamiamo:

\begin{itemize*}
 \item $\sin \alpha$ il cateto opposto all'angolo~$\alpha$;
 \item $\cos \alpha$ il cateto adiacente all'angolo~$\alpha$;
 \item $\tan \alpha$ la distanza di~$T$ dall'asse~$x$;
 \item con $\alpha$ possiamo intendere sia l'angolo sia l'arco.
\end{itemize*}


% \begin{definizione}
% La \emph{componente orizzontale} $u_{x}$ del vettore unitario inclinato dell'angolo~${\hat{\alpha}}$ sull'asse~$x$, si chiama \emph{coseno dell'angolo}
% ${\hat{\alpha}}$ in simboli~$u_{x}=\cos \alpha$. Chiamiamo \emph{seno dell'angolo} ${\alpha}$ la \emph{componente verticale} $u_{y}$
% del vettore unitario inclinato dell'angolo~${\alpha}$ sull'asse~$x$ in simboli~$u_{y}=\sin \alpha$.
% Scriviamo~$\vec{u}=(\cos \alpha,\sin \alpha)$ o anche~$B=(\cos \alpha,\sin \alpha)$.
% \end{definizione}

Si può osservare che:

\begin{itemize*}
 \item $\cos \alpha$ è l'ascissa di~$P$ e~$\sin \alpha$ è 
  l'ordinata di~$P$: $P \left(\cos \alpha;~\sin \alpha \right)$;
 \item per il teorema di Pitagora:~$\sin^2 \alpha + \cos^2 \alpha = 1$;
 \item sia $\cos \alpha$ sia~$\sin \alpha$ hanno valori interni
  all'intervallo~$\left[-1;~+1 \right]$.
\end{itemize*}


\begin{figure}[h] 
 \vspace{-6pt}
  \begin{center}
\begin{inaccessibleblock}[La circonferenza goniometrica.]
    % (c) 2014 Daniele Zambelli - daniele.zambelli@gmail.com

\begin{tikzpicture}[scale=2.0, cap=round, >=latex]

% (c) 2014 Daniele Zambelli - daniele.zambelli@gmail.com

\newcommand{\circonferenzagoniometrica}[5]{%
% Colors
  \colorlet{anglecolor}{green!50!black}
  \colorlet{sincolor}{blue!50!black}
  \colorlet{tancolor}{orange!70!black}
  \colorlet{coscolor}{red!50!black}

% Local definitions
  \def \pangle{#1}
  \def \palfangle{\pangle / 2}
  \def \petichetta{#2}
  \def \psin{#3}
  \def \pcos{#4}
  \def \ptan{#5}
  
% Assi, circonferenza 
  % (c) 2014 Daniele Zambelli - daniele.zambelli@gmail.com

%%%
% Circonferenza goniometrica
%%%%
% 
% % Griglia
% \draw[gray!50, very thin, step=1] (-1.2, -1.2) grid (1.2, 1.2);

%Asse x
\draw [-{Stealth[length=2mm, open, round]}] (-1.3,0) -- (1.3,0) node [below]
      {$x$};
%Asse y
\draw [-{Stealth[length=2mm, open, round]}] (0, -1.3) -- (0, 1.3) node [left]  
      {$y$};
%Circonferenza
\draw [very thin] (0,0) circle (1);

% Tangente
  \draw[style=help lines] (1, -1.2) -- (1, 1.8);
% Punti legati al sistema di riferimento
  \coordinate (origin) at (0, 0);
  \draw (origin) node [below left] {$O$} [fill] circle(0.4pt);
  \draw (1, 0) node [below right] {$K$} [fill] circle(0.4pt);
% Parte che dipende dall'angolo
  \begin{scope}[very thick]
% Angolo e archi
    \begin{scope}[anglecolor]
      \draw [anglecolor, rotate=\pangle] (0, 0) -- (2.2, 0);
      \begin{scope}[->]
        \draw (0.3, 0) arc(0:\pangle:0.3);
        \draw (1, 0) arc(0:\pangle:1);
        \draw [-, anglecolor] (\palfangle:.45) node {\petichetta};
      \end{scope}
    \end{scope}
% Seno
    \draw [sincolor]
      (\pangle:1) node (p) [above] {$P$} circle(0.2pt)-- 
      node[left] {\psin} (p |- origin) 
      node (h) {}; 
     \draw (h) node [black, below] {$H$} [fill] circle(0.2pt); 
% Coseno
    \draw [coscolor] (h) -- 
      node [coscolor, below] {\pcos} (origin);
% Tangente
    \draw [tancolor] (1,0) --
      node [right=1pt,fill=white] {\ptan} 
      (intersection of 0,0 -- \pangle:1 and 1,0--1,1) [below right] node {$T$}
        circle(0.2pt);
  \end{scope}
}


\clip(-1.2, -1.3) rectangle (1.5, 1.4);

\circonferenzagoniometrica {40} {$\alpha$} {$\sin \alpha$}
                            {$\cos \alpha$} {$\tan \alpha$}

\end{tikzpicture}

% \begin{tikzpicture}[scale=2.7, cap=round, >=latex]
%   % Local definitions
%   \def\angle{40}
%   \def\alfangle{20}
%   \def\sinangle{0.6427876}
%   \def\cosangle{0.7660444}
% 
%   % Colors
%   \colorlet{anglecolor}{green!50!black}
%   \colorlet{sincolor}{blue!50!black}
%   \colorlet{tancolor}{orange!70!black}
%   \colorlet{coscolor}{red!50!black}
% 
% % (c) 2014 Daniele Zambelli - daniele.zambelli@gmail.com

%%%
% Circonferenza goniometrica
%%%%
% 
% % Griglia
% \draw[gray!50, very thin, step=1] (-1.2, -1.2) grid (1.2, 1.2);

%Asse x
\draw [-{Stealth[length=2mm, open, round]}] (-1.3,0) -- (1.3,0) node [below]
      {$x$};
%Asse y
\draw [-{Stealth[length=2mm, open, round]}] (0, -1.3) -- (0, 1.3) node [left]  
      {$y$};
%Circonferenza
\draw [very thin] (0,0) circle (1);

% 
% \draw (0, 0) node [below left] {$O$} [fill] circle(0.4pt);
% \draw (\cosangle, 0) node [below] {$H$} [fill] circle(0.4pt);
% \draw (1, 0) node [below right] {$K$} [fill] circle(0.4pt);
% 
% \node at (.38, .45) {$1$};
% 
% \draw[style=help lines] (1, -1.2) -- (1, 1.3);
% 
% \begin{scope}[very thick]
%  \begin{scope}[->, green!50!black]
%   \draw (0.3, 0) arc(0:\angle:0.3);
%   \draw (1.0, 0) arc(0:\angle:1.0);
%  \end{scope}
%  \draw (\alfangle:.2) node[anglecolor] {$\alpha$};
%  
%  \draw [sincolor]
%     (\angle:1) node [above] {$P$} circle(0.5pt)-- 
%                node[left=0pt,fill=white] {$\sin \alpha$} +(0, -\sinangle);
% 
%  \draw [coscolor]
%     (0,0) -- node[below=2pt,fill=white] {$\cos \alpha$} (\cosangle, 0);
% 
%   \draw [tancolor] (1,0) --
%     node [right=1pt,fill=white] {$\displaystyle \tan \alpha$} 
%      (intersection of 0,0 -- \angle:1 and 1,0--1,1) [below right] node {$T$}
%       circle(0.5pt); % coordinate (t);
% 
%  \draw [Maroon!50!black, rotate=\angle] (0, 0)--(1.8, 0);
%  
% \end{scope}
% 
% \end{tikzpicture}

    \caption{Circonferenza goniometrica}
    \label{fig:trigo_circ_gonio}
\end{inaccessibleblock}
  \end{center}
   \vspace{-24pt}
\end{figure} 

Possiamo riprendere la tabella precedente e completarla con i valori di 
\emph{seno}, \emph{coseno} e \emph{tangente}:

\begin{center}
\begin{tabular}{cccccccc}
angolo \quad & \quad 0 \grado \quad & \quad 90 \grado \quad & \quad 180 \grado 
\quad & \quad 270 \grado \quad & \quad 360 \grado \quad & \quad 
$\alpha$ \quad & \quad $\dfrac{\alpha}{\pi} \cdot 180$ \\

arco \quad & \quad 0 \quad & \quad $\dfrac{1}{2} \pi$ \quad & \quad $\pi$ 
\quad & \quad $\dfrac{3}{2} \pi$ \quad & \quad $2 \pi$ \quad &  \quad
$\dfrac{\alpha}{180} \cdot \pi$ \quad & \quad $\alpha$ \\ \\

% \vspace{12pt}

seno \quad & \quad ... \quad & \quad ...  \quad & \quad ...  
\quad & \quad ...  \quad & \quad ...  \quad & \quad \\ \\ 

% \vspace{12pt}

coseno \quad & \quad ...  \quad & \quad ...  \quad & \quad ...  
\quad & \quad ...  \quad & \quad ...  \quad & \\ \\
 
% \vspace{12pt}

tangente \quad & \quad ...  \quad & \quad ...  \quad & \quad ...  
\quad & \quad ...  \quad & \quad ...  \quad & \\

\end{tabular}
\end{center}

\section{Le funzioni circolari}
\label{sec:gonio_funzionicircolari}

La circonferenza goniometrica è uno strumento che permette di associare ad 
ogni valore dell'arco~$\alpha$ un ben preciso numero detto~\emph{seno},
cioè permette di definire una funzione tra i valori dell'angolo e i rispettivi
seni. Lo stesso vale per il coseno e per la tangente. 
Cioè la circonferenza goniometrica ci permette di creare tre funzioni nei 
numeri reali. Vediamo ora come disegnare il grafico della funzione seno.

\begin{procedura}
 Per disegnare le funzioni circolari:
%  \begin{enumerate}[a)]
 \begin{enumeratea}
  \item procurati 4 fogli di quaderno a quadretti da~5mm;
  \item incollali uno di seguito all'altro 
   facendo attenzione a far combaciare le linee dei quadretti;
  \item nel primo foglio disegna, con un compasso, una circonferenza di
   raggio~1dm;
  \item dividi questa circonferenza in~24 archi uguali 
   (puoi usare i quadretti per gli angoli di~0°, 45°, 90°, 135°, \dots 
   e il compasso per quelli di:~30°, 60°, \dots
   e ancora il compasso per:~15°, 75°, \dots);
  \item disegna un asse~$x$ che attraversa tutti i fogli in orizzontale;
  \item disegna un asse~$y$ verticale tangente alla circonferenza;
  \item sull'asse~$x$ segna una tacca ogni~6 quadretti 
   (sarebbe un po' di più ma questa approssimazione non dovrebbe deformare 
   troppo il grafico della funzione);
  \item scrivi~0 all'incrocio degli assi e poi 15°, 30°, 45°, 60°, 75°, \dots;
  \item in corrispondenza di ogni valore di arco, riporta il corrispondente
   valore del seno (l'ordinata;
  \item congiungi tutti i punti disegnati con una linea il più possibile 
   regolare.
 \end{enumeratea}
%  \end{enumerate}
\end{procedura}

\begin{figure}[h] 
 \vspace{-6pt}
  \begin{center}
\begin{inaccessibleblock}[Disegno della sinusoide a partire dalla
 circonferenza goniometrica.]
    % (c) 2014 Daniele Zambelli - daniele.zambelli@gmail.com

% \tiny
% \scriptsize
% \footnotesize

\begin{tikzpicture}[x=17mm,y=17mm, font=\small, cap=round, >=latex]

\begin{scope}[-{Stealth[length=2mm, open, round]}]
% Asse x
\draw (-1.1,0) -- (7.2, 0) node [below] {$x$};
% Asse y
\draw (1, -1.3) -- (1, 1.3) node [left] {$y$};
\end{scope}

% Circonferenza
\draw [thin] (0, 0) circle (1);

\begin{scope}[font=\tiny]
% Tacche asse x
\foreach \x/\xtext in {
    1/0, 1.5/30, 2/60, 2.5/90, 
    3/120, 3.5/150, 4/180, 4.5/210, 
    5/240, 5.5/270, 6/300, 6.5/330, 
    7/360}
\node[below] at(\x, 0) {$\xtext \grado$};
\end{scope}

% Tacche asse x e vettori
\begin{scope}[font=\tiny]
\foreach \x/\z in {
    1/0, 1.25/15, 1.5/30, 1.75/45, 
    2/60, 2.25/75, 2.5/90, 2.75/105, 
    3/120, 3.25/135, 3.5/150, 3.75/165, 
    4/180, 4.25/195, 4.5/210, 4.75/225, 
    5/240, 5.25/255, 5.5/270, 5.75/285, 
    6/300, 6.25/315, 6.5/330, 6.75/345, 
    7/360}
{\draw [black] (\x, -0.02) -- (\x, +0.05) node (a) {};
 \draw [->, Maroon, rotate=\z] (0, 0) -- (1, 0) node (b) {};
%  \coordinate (c) at (a |- b);
 \draw [thin, dotted, green!50!black] (b) -- (a |- b) -- (a);
 \draw [black] (a |- b) circle(1pt);
}
\end{scope}

\end{tikzpicture}
    \caption{Disegno della sinusoide}
    \label{fig:trigo_sinusoide}
\end{inaccessibleblock}
  \end{center}
   \vspace{-24pt}
\end{figure} 

In modo analogo puoi riportare sul piano i punti relativi al coseno e alla 
tangente. ottenendo il grafico delle tre funzioni circolari fondamentali.

\begin{figure}[!h] 
 \vspace{-6pt}
  \begin{center}
\begin{inaccessibleblock}[Sinusoide nel piano cartesiano.]
    % (c) 2014 Daniele Zambelli - daniele.zambelli@gmail.com

%%%
% La sinusoide nel piano cartesiano.
%%%%
%  
\begin{tikzpicture}[x=7mm, y=7mm, smooth, color=Blue!50!black]]

\tkzInit[xmin=-7.5,xmax=+7.5,ymin=-5.5,ymax=+5.5]

\clip (-7.3, -1.3) rectangle (7.7, 2);

% (c) 2014 Daniele Zambelli - daniele.zambelli@gmail.com

%%%
% Piano cartesiano: da (-7; -1.5) a (+7; +1.5)
%%%%

% Griglia
\draw[gray!50, very thin, step=1] (-7.2, -1.5) grid (7.2, 1.5);

%Assi
\begin{scope}[-{Stealth[length=2mm, open, round]}, black]
 \draw (-7.3,0) -- (7.5,0) node [below] {$x$};
 \draw (0, -1.5) -- (0, 1.7) node [left] {$y$};
\end{scope}


\tkzFct[domain=-7.2:+7.2, ultra thick]{sin(x)}

\end{tikzpicture}

    \caption{Sinusoide nel piano cartesiano}
    \label{fig:trigo_sin}
\end{inaccessibleblock}
  \end{center}
   \vspace{-24pt}
\end{figure} 
Puoi osservare che la funzione seno:
\begin{itemize*}
 \item è definita per ogni valore di x;
 \item è simmetrica rispetto \dots
 \item è contenuta nell'intervallo \dots
 \item si ripete sempre uguale, cioè è periodica con periodo \dots
 \item interseca l'asse delle ascisse con un angolo di \dots
\end{itemize*}
\begin{figure}[!h] 
 \vspace{-6pt}
  \begin{center}
  
\begin{inaccessibleblock}[Cosinusoide nel piano cartesiano.]
    % (c) 2014 Daniele Zambelli - daniele.zambelli@gmail.com

%%%
% La cosinusoide nel piano cartesiano.
%%%%
%  
\begin{tikzpicture}[x=7mm, y=7mm, smooth, color=Red!50!black]]

\tkzInit[xmin=-7.5,xmax=+7.5,ymin=-5.5,ymax=+5.5]

\clip (-7.3, -1.3) rectangle (7.7, 2);

% (c) 2014 Daniele Zambelli - daniele.zambelli@gmail.com

%%%
% Piano cartesiano: da (-7; -1.5) a (+7; +1.5)
%%%%

% Griglia
\draw[gray!50, very thin, step=1] (-7.2, -1.5) grid (7.2, 1.5);

%Assi
\begin{scope}[-{Stealth[length=2mm, open, round]}, black]
 \draw (-7.3,0) -- (7.5,0) node [below] {$x$};
 \draw (0, -1.5) -- (0, 1.7) node [left] {$y$};
\end{scope}


\tkzFct[domain=-7.2:+7.2, ultra thick]{cos(x)}

\end{tikzpicture}

    \caption{Cosinusoide nel piano cartesiano}
    \label{fig:trigo_sin}
\end{inaccessibleblock}
  \end{center}
   \vspace{-24pt}
\end{figure} 
Puoi osservare che la funzione coseno:
\begin{itemize*}
 \item è definita per ogni valore di x;
 \item è simmetrica rispetto \dots
 \item è contenuta nell'intervallo \dots
 \item si ripete sempre uguale, cioè è periodica con periodo \dots
 \item la funzione seno e la funzione coseno hanno la stessa forma,
  sono solo \dots
 \item interseca l'asse delle ascisse con un angolo di \dots
\end{itemize*}

\begin{figure}[!h] 
 \vspace{-6pt}
  \begin{center}
\begin{inaccessibleblock}[Tangentoide nel piano cartesiano.]
    % (c) 2014 Daniele Zambelli - daniele.zambelli@gmail.com

%%%
% La tangentoide nel piano cartesiano.
%%%%
%  
\begin{tikzpicture}[x=7mm, y=7mm, smooth, color=Orange!70!black]]

\tkzInit[xmin=-7.5,xmax=+7.5,ymin=-5.5,ymax=+5.5]

\clip (-7.3, -5.3) rectangle (7.7, 5.7);

% (c) 2014 Daniele Zambelli - daniele.zambelli@gmail.com

%%%
% Piano cartesiano: da (-7; -5) a (+7; +5)
%%%%

% Griglia
\draw[gray!50, very thin, step=1] (-7.2, -5.2) grid (7.2, 5.2);

%Assi
\begin{scope}[-{Stealth[length=2mm, open, round]}, black]
 \draw (-7.3,0) -- (7.5,0) node [below] {$x$};
 \draw (0, -5.3) -- (0, 5.5) node [left] {$y$};
\end{scope}


\tkzFct[domain=-7.2:-4.8, ultra thick]{tan(x)}

\tkzFct[domain=-4.6:-1.7, ultra thick]{tan(x)}

\tkzFct[domain=-1.5:+1.5, ultra thick]{tan(x)}

\tkzFct[domain=1.7:+4.6, ultra thick]{tan(x)}

\tkzFct[domain=+4.8:7.2, ultra thick]{tan(x)}

\end{tikzpicture}

    \caption{Tangentoide nel piano cartesiano}
    \label{fig:trigo_sin}
\end{inaccessibleblock}
  \end{center}
   \vspace{-24pt}
\end{figure} 
Puoi osservare che la funzione tangente:
\begin{itemize*}
 \item non è definita quando x vale \dots
 \item è simmetrica rispetto \dots
 \item è illimitata superiormente e inferiormente:
 \item si ripete sempre uguale, cioè è periodica con periodo \dots
 \item interseca l'asse delle ascisse con un angolo di \dots
\end{itemize*}

\section{Relazioni tra le funzioni circolari}
\label{sec:gonio_relazioni}

Come si può vedere dai grafici precedenti, le funzioni circolari (seno, coseno, 
tangente) sono legate tra di loro. Vediamo alcuni legami.

\subsection{Relazione fondamentale della goniometria}

La prima relazione che lega seno e coseno deriva dal teorema di Pitagora:

\[\left(\sin \alpha \right)^2 + \left(\cos \alpha \right)^2 = 1\]

Può essere scritta con lo stesso significato risparmiando qualche parentesi:

\[\sin^2 \alpha + \cos^2 \alpha = 1\]

Deriva dalla definizione di seno e coseno, vedi 
figura~\ref{fig:trigo_circ_gonio}.

Da questa si ricavano le formule:

\[\sin^2 \alpha = 1 - \cos^2 \alpha\]
\[\cos^2 \alpha = 1 - \sin^2 \alpha\]

\subsection{Tangente in funzione di seno e coseno}

Si può osservare che quando il seno vale~0 anche la tangente vale~0 e quando 
il coseno vale~0 la tangente non è definita. Questo può darci un indizio 
riguardo alla seguente relazione:

\[\tan \alpha = \frac{\sin \alpha}{\cos \alpha}\]

La si può dimostrare considerando che i triangoli $OHP$ e $OKT$ sono simili
e quindi i rapporti tra lati corrispondenti sono uguali:

\[OHP \sim OKT \Rightarrow \frac{HP}{OH} = \frac{KT}{OK}\]

\begin{minipage}{.3\textwidth}
e considerando che:
\begin{itemize*}
 \item $HP = \sin \alpha$;
 \item $OH = \cos \alpha$;
 \item $KT = \tan \alpha$;
 \item $OK = 1$;
\end{itemize*}
Operando le sostituzioni si ottiene la relazione precedente.
\end{minipage}
\begin{minipage}{.6\textwidth}
\begin{center}
 % (c) 2014 Daniele Zambelli - daniele.zambelli@gmail.com

\begin{tikzpicture}[scale=2.0, cap=round, >=latex]

% (c) 2014 Daniele Zambelli - daniele.zambelli@gmail.com

\newcommand{\circonferenzagoniometrica}[5]{%
% Colors
  \colorlet{anglecolor}{green!50!black}
  \colorlet{sincolor}{blue!50!black}
  \colorlet{tancolor}{orange!70!black}
  \colorlet{coscolor}{red!50!black}

% Local definitions
  \def \pangle{#1}
  \def \palfangle{\pangle / 2}
  \def \petichetta{#2}
  \def \psin{#3}
  \def \pcos{#4}
  \def \ptan{#5}
  
% Assi, circonferenza 
  % (c) 2014 Daniele Zambelli - daniele.zambelli@gmail.com

%%%
% Circonferenza goniometrica
%%%%
% 
% % Griglia
% \draw[gray!50, very thin, step=1] (-1.2, -1.2) grid (1.2, 1.2);

%Asse x
\draw [-{Stealth[length=2mm, open, round]}] (-1.3,0) -- (1.3,0) node [below]
      {$x$};
%Asse y
\draw [-{Stealth[length=2mm, open, round]}] (0, -1.3) -- (0, 1.3) node [left]  
      {$y$};
%Circonferenza
\draw [very thin] (0,0) circle (1);

% Tangente
  \draw[style=help lines] (1, -1.2) -- (1, 1.8);
% Punti legati al sistema di riferimento
  \coordinate (origin) at (0, 0);
  \draw (origin) node [below left] {$O$} [fill] circle(0.4pt);
  \draw (1, 0) node [below right] {$K$} [fill] circle(0.4pt);
% Parte che dipende dall'angolo
  \begin{scope}[very thick]
% Angolo e archi
    \begin{scope}[anglecolor]
      \draw [anglecolor, rotate=\pangle] (0, 0) -- (2.2, 0);
      \begin{scope}[->]
        \draw (0.3, 0) arc(0:\pangle:0.3);
        \draw (1, 0) arc(0:\pangle:1);
        \draw [-, anglecolor] (\palfangle:.45) node {\petichetta};
      \end{scope}
    \end{scope}
% Seno
    \draw [sincolor]
      (\pangle:1) node (p) [above] {$P$} circle(0.2pt)-- 
      node[left] {\psin} (p |- origin) 
      node (h) {}; 
     \draw (h) node [black, below] {$H$} [fill] circle(0.2pt); 
% Coseno
    \draw [coscolor] (h) -- 
      node [coscolor, below] {\pcos} (origin);
% Tangente
    \draw [tancolor] (1,0) --
      node [right=1pt,fill=white] {\ptan} 
      (intersection of 0,0 -- \pangle:1 and 1,0--1,1) [below right] node {$T$}
        circle(0.2pt);
  \end{scope}
}


\clip(-1.2, -1.3) rectangle (1.5, 1.4);

\circonferenzagoniometrica {40} {$\alpha$} {$\sin \alpha$}
                            {$\cos \alpha$} {$\tan \alpha$}

\end{tikzpicture}

% \begin{tikzpicture}[scale=2.7, cap=round, >=latex]
%   % Local definitions
%   \def\angle{40}
%   \def\alfangle{20}
%   \def\sinangle{0.6427876}
%   \def\cosangle{0.7660444}
% 
%   % Colors
%   \colorlet{anglecolor}{green!50!black}
%   \colorlet{sincolor}{blue!50!black}
%   \colorlet{tancolor}{orange!70!black}
%   \colorlet{coscolor}{red!50!black}
% 
% % (c) 2014 Daniele Zambelli - daniele.zambelli@gmail.com

%%%
% Circonferenza goniometrica
%%%%
% 
% % Griglia
% \draw[gray!50, very thin, step=1] (-1.2, -1.2) grid (1.2, 1.2);

%Asse x
\draw [-{Stealth[length=2mm, open, round]}] (-1.3,0) -- (1.3,0) node [below]
      {$x$};
%Asse y
\draw [-{Stealth[length=2mm, open, round]}] (0, -1.3) -- (0, 1.3) node [left]  
      {$y$};
%Circonferenza
\draw [very thin] (0,0) circle (1);

% 
% \draw (0, 0) node [below left] {$O$} [fill] circle(0.4pt);
% \draw (\cosangle, 0) node [below] {$H$} [fill] circle(0.4pt);
% \draw (1, 0) node [below right] {$K$} [fill] circle(0.4pt);
% 
% \node at (.38, .45) {$1$};
% 
% \draw[style=help lines] (1, -1.2) -- (1, 1.3);
% 
% \begin{scope}[very thick]
%  \begin{scope}[->, green!50!black]
%   \draw (0.3, 0) arc(0:\angle:0.3);
%   \draw (1.0, 0) arc(0:\angle:1.0);
%  \end{scope}
%  \draw (\alfangle:.2) node[anglecolor] {$\alpha$};
%  
%  \draw [sincolor]
%     (\angle:1) node [above] {$P$} circle(0.5pt)-- 
%                node[left=0pt,fill=white] {$\sin \alpha$} +(0, -\sinangle);
% 
%  \draw [coscolor]
%     (0,0) -- node[below=2pt,fill=white] {$\cos \alpha$} (\cosangle, 0);
% 
%   \draw [tancolor] (1,0) --
%     node [right=1pt,fill=white] {$\displaystyle \tan \alpha$} 
%      (intersection of 0,0 -- \angle:1 and 1,0--1,1) [below right] node {$T$}
%       circle(0.5pt); % coordinate (t);
% 
%  \draw [Maroon!50!black, rotate=\angle] (0, 0)--(1.8, 0);
%  
% \end{scope}
% 
% \end{tikzpicture}

\end{center}
\end{minipage}

\subsection{Traslazione di seno e coseno}

Osservando i grafici delle funzioni seno e coseno possiamo vedere che traslando
la funzione coseno verso destra di $\frac{\pi}{2}$ si ottiene il grafico della 
funzione seno:
\[\cos\left(\frac{\pi}{2}-\alpha\right)=\sin\alpha\]
\vspace{-6pt}
\begin{figure}[!h] 
%  \vspace{-6pt}
  \begin{center}
\begin{inaccessibleblock}[Sinusoide e cosinusoide nel piano cartesiano.]
    % (c) 2014 Daniele Zambelli - daniele.zambelli@gmail.com

%%%
% La cosinusoide nel piano cartesiano.
%%%%
%  
\disegno[7]{
  \rcom{-7}{+7}{-1}{+1}{gray!50, very thin, step=1}
  \tkzInit[xmin=-7.5,xmax=+7.5,ymin=-1.3,ymax=+1.3]
  \tkzFct[domain=-7.2:+7.2, ultra thick, color=blue!50!black]{sin(x)}
  \tkzFct[domain=-7.2:+7.2, ultra thick, color=red!50!black]{cos(x)}
}

% \begin{tikzpicture}[x=9mm, y=9mm, smooth]
% 
% \tkzInit[xmin=-7.5,xmax=+7.5,ymin=-5.5,ymax=+5.5]
% 
% \clip (-7.3, -1.3) rectangle (7.7, 2);
% 
% % (c) 2014 Daniele Zambelli - daniele.zambelli@gmail.com

%%%
% Piano cartesiano: da (-7; -1.5) a (+7; +1.5)
%%%%

% Griglia
\draw[gray!50, very thin, step=1] (-7.2, -1.5) grid (7.2, 1.5);

%Assi
\begin{scope}[-{Stealth[length=2mm, open, round]}, black]
 \draw (-7.3,0) -- (7.5,0) node [below] {$x$};
 \draw (0, -1.5) -- (0, 1.7) node [left] {$y$};
\end{scope}

% 
% \tkzFct[domain=-7.2:+7.2, ultra thick, color=blue!50!black]{sin(x)}
% \tkzFct[domain=-7.2:+7.2, ultra thick, color=Red!50!black]{cos(x)}
% 
% \end{tikzpicture}

    \caption{Sinusoide e cosinusoide nel piano cartesiano}
    \label{fig:trigo_sin_cos}
\end{inaccessibleblock}
  \end{center}
\vspace{-12pt}
\end{figure} 
\vspace{-6pt}
E viceversa traslando verso sinistra di $\frac{\pi}{2}$ il grafico della 
funzione seno si ottiene la funzione coseno:
\[\sin\left(\frac{\pi}{2}-\alpha\right)=\cos\alpha\]

\section{Angoli associati}
\label{sec:gonio_angoli_associati}

Osservando la circonferenza goniometrica possiamo vedere che ci sono angoli 
diversi che hanno lo stesso seno o lo stesso coseno. 
In certe condizioni, il seno di $\alpha$ è uguale al coseno di $\beta$. 
In altri casi i valori di queste funzioni sono opposti.
Si può osservare facilmente che: (\emph{completa tu la seguente tabella})

 \begin{figure}[!h] 
%  \vspace{-6pt}
\begin{minipage}{.3\textwidth}
  \begin{center}
\begin{inaccessibleblock}[Circonferenza goniometrica con evidenziati 
  gli angoli associati.]
    % (c) 2014 Daniele Zambelli - daniele.zambelli@gmail.com

\begin{tikzpicture}[scale=1.5, cap=round, >=latex]
% Local definitions
  \def\angle{20}
  \def\alfangle{\angle / 2}
% Colors
  \colorlet{anglecolor}{green!50!black}
  \colorlet{sincolor}{blue!50!black}
  \colorlet{tancolor}{orange!70!black}
  \colorlet{coscolor}{red!50!black}

% (c) 2014 Daniele Zambelli - daniele.zambelli@gmail.com

%%%
% Circonferenza goniometrica
%%%%
% 
% % Griglia
% \draw[gray!50, very thin, step=1] (-1.2, -1.2) grid (1.2, 1.2);

%Asse x
\draw [-{Stealth[length=2mm, open, round]}] (-1.3,0) -- (1.3,0) node [below]
      {$x$};
%Asse y
\draw [-{Stealth[length=2mm, open, round]}] (0, -1.3) -- (0, 1.3) node [left]  
      {$y$};
%Circonferenza
\draw [very thin] (0,0) circle (1);


% Disegno primo angolo
 
\coordinate (origin) at (0, 0);

\begin{scope}[->, anglecolor, very thick]
  \draw (0.5, 0) arc(0:\angle:0.5);
  \draw (1.0, 0) arc(0:\angle:1.0);
\end{scope}
 \draw (\alfangle:.6) node [anglecolor] {$\alpha$};

% Disegno angoli associati
 
\begin{scope}[very thick]
\draw [->, anglecolor, rotate=\angle] (0, 0) -- (1, 0) node (b) {};
\draw [sincolor] (b) -- (b |- origin);
\draw [coscolor] (origin) -- (b |- origin);
\begin{scope}[->, anglecolor]
\foreach \z in {
    90-\angle, 90+\angle, 180-\angle, 180+\angle, 270-\angle, 270+\angle,
    -\angle
    }
{\draw [rotate=\z] (0, 0) -- (1, 0) node (b) {};}
\end{scope}
\end{scope}

\end{tikzpicture}

    \caption{Angoli associati}
    \label{fig:trigo_angoli_associati}
\end{inaccessibleblock}
  \end{center}
\end{minipage}
\begin{minipage}{.7\textwidth}
\begin{center}
\begin{tabular}{rclclc}
$\sin \alpha$ & $=$ & $\cos \left(\frac{1}{2}\pi-\alpha \right)$ 
              & $=$ & $-\cos \left(\frac{1}{2}\pi+\alpha \right)$ & $=$\\
              & $=$ & $\sin \left(\pi-\alpha \right)$  
              & $=$ & $-\sin \left(\pi+\alpha \right)$ & $=$ \\
              & $=$ & $-\cos \left(\frac{3}{2}\pi-\alpha \right)$  
              & $=$ & $\cos \left(\frac{3}{2}\pi+\alpha \right)$ & $=$ \\
              & $=$ & $-\sin \left(-\alpha \right)$ \\
$\cos \alpha$ & $=$ & $\text{. . . . . .}
                       \left(\frac{1}{2}\pi-\alpha \right)$ 
              & $=$ & $\text{. . . . . .}
                       \left(\frac{1}{2}\pi+\alpha \right)$ & $=$\\
              & $=$ & $\text{. . . . . . .}
                       \left(\pi-\alpha \right)$  
              & $=$ & $\text{. . . . . . .}
                       \left(\pi+\alpha \right)$ & $=$ \\
              & $=$ & $\text{. . . . . .}
                       \left(\frac{3}{2}\pi-\alpha \right)$  
              & $=$ & $\text{. . . . . .}
                       \left(\frac{3}{2}\pi+\alpha \right)$ & $=$ \\
              & $=$ & $\text{. . . . . . . . .}
                       \left(-\alpha \right)$ \\
$\tan \alpha$ & $=$ & $\text{. . . . . .}
                       \left(\frac{1}{2}\pi-\alpha \right)$ 
              & $=$ & $\text{. . . . . .}
                       \left(\frac{1}{2}\pi+\alpha \right)$ & $=$\\
              & $=$ & $\text{. . . . . . .}
                       \left(\pi-\alpha \right)$  
              & $=$ & $\text{. . . . . . .}
                       \left(\pi+\alpha \right)$ & $=$ \\
              & $=$ & $\text{. . . . . .}
                       \left(\frac{3}{2}\pi-\alpha \right)$  
              & $=$ & $\text{. . . . . .}
                       \left(\frac{3}{2}\pi+\alpha \right)$ & $=$ \\
              & $=$ & $\text{. . . . . . . . .}
                       \left(-\alpha \right)$ 
\end{tabular}
\end{center}
\end{minipage}
\vspace{-18pt}
\end{figure} 

\section{Angoli particolari}
\label{sec:gonio_angoli_particolari}

Il modo più semplice per calcolare seno, coseno, tangente di un angolo è quello 
di usare una calcolatrice non le apposite funzioni. 
È importante controllare le impostazioni della calcolatrice:
\begin{itemize*}
 \item se gli angoli vengono misurati in gradi deve apparire la 
 scritta~``deg'';
 \item se gli angoli vengono misurati in radianti deve apparire la 
 scritta~``rad'';
\end{itemize*}

Alle volte abbiamo a che fare con angoli particolari di cui possiamo ricordarci 
il valore delle corrispondenti funzioni circolari. 
Prima di affrontarli, però dobbiamo ripassare alcuni risultati ottenuti 
applicando il teorema di Pitagora.

\newpage %-------------------------------------------------------

\subsection{Digressione pitagorica}

 \begin{figure}[!h]
 \begin{minipage}[t]{.45\textwidth}
\begin{inaccessibleblock}[Triangolo rettangolo isoscele.]
\centering
 % (c) 2014 Daniele Zambelli - daniele.zambelli@gmail.com

\begin{tikzpicture}[x=20mm,y=20mm, 
                    font=\small, cap=round, join=round, >=latex]

\colorlet{anglecolor}{green!50!black}

\newcommand{\quadrato}[4]{%
\coordinate (a) at (#1, 0);
\coordinate (b) at (#1+#2, 0);
\coordinate (c) at (#1+#2, #2);
\coordinate (d) at (#1, #2);

    \draw[thick, Maroon!50!black] (a)--(b)
          node [black, sloped, midway, below] {#3} -- (c) -- cycle 
          node [black, sloped, midway, above] {#4};
    \draw[thick, Maroon!50] (a)--(d)--(c);
    \draw [->, anglecolor, thick](#1+.3, 0) arc(0:45:0.3);
    \draw (#1+.45, .15) node [anglecolor] {$45 \text{°}$};
}

\quadrato{0}{1}{$l$}{$l \cdot \sqrt{2}$}; 

\quadrato{1.5}{1}{$d \cdot \frac{\sqrt{2}}{2}$}{$d$}; 

\end{tikzpicture}
 \caption{Triangolo rettangolo isoscele}
 \label{fig:trigo_triangolo_isoscele}
\end{inaccessibleblock}
 \end{minipage}
 \begin{minipage}[t]{.45\textwidth}
\begin{inaccessibleblock}[Triangolo equilatero con evidenziata un'altezza.]
\centering
 % (c) 2014 Daniele Zambelli - daniele.zambelli@gmail.com

\begin{tikzpicture}[x=15mm,y=15mm, 
                    font=\small, cap=round, join=round, >=latex]

\colorlet{anglecolor}{green!50!black}

\newcommand{\triequi}[2]{%
\coordinate (a) at (#1, 0);
\coordinate (m) at (#1+#2 / 2, 0);
\coordinate (b) at (#1+#2, 0);
\coordinate (c) at (#1+#2 / 2, #2 * 0.8660254);

    \draw[thick, Maroon!50!black] (a)--
          (m)
          node [black, sloped, midway, below] {$\frac{l}{2}$} -- 
          (c)
          node [black, midway, above] {$l \cdot \frac{\sqrt{3}}{2}$} -- 
          cycle 
          node [black, midway, above] {$l$};
    \draw[thick, Maroon!50] (m)--(b)--(c);
    \draw [->, anglecolor, thick](#1+.3, 0) arc(0:60:0.3);
    \draw (#1+.45, .15) node [anglecolor] {$60 \textdegree$};
}

\triequi{0}{2}; 
% 
\end{tikzpicture}


% 
% \coordinate (a) at (0,0);
% \coordinate (b) at (6,0);
% \node(c1) at (a)[circle through=(b)] {};
% \node(c2) at (b)[circle through=(a)] {};
% \coordinate  (c) at (intersection 2 of c1 and c2);
% 
% \draw[linea] (a)--(b)--(c) --cycle;
% 
% \node(A)[label=$A$] at (a) {$\bullet$};
% \node(B)[label=50:$B$] at (b)  {$\bullet$};
% \node(C)[label=$C$] at (c) {$\bullet$};

 \caption{Triangolo equilatero}
 \label{fig:trigo_equilatero}
\end{inaccessibleblock}
 \end{minipage}
\end{figure}
\vspace{-24pt}

\subsubsection{Triangolo rettangolo isoscele}

Consideriamo un triangolo rettangolo isoscele, possiamo vedere questo 
triangolo anche come la metà di un quadrato. 
Vedi figura \ref{fig:trigo_triangolo_isoscele}
Possiamo anche osservare che gli angoli acuti sono di~45° dato che sono 
congruenti tra di loro e la loro somma è di~90°.
Supponiamo di conoscere la lunghezza dei cateti (lati del quadrato) e di voler 
calcolare la lunghezza dell'ipotenusa (diagonale del quadrato).
Per il teorema di Pitagora:
% \vspace{-6pt}
\[ipotenusa = \sqrt{lato^2 +lato^2} = 
              \sqrt{2 \cdot lato^2} = lato \cdot \sqrt{2}\]
Con le regole del calcolo delle formule inverse possiamo trovare che:
% \vspace{-6pt}
\[lato = \frac{ipotenusa}{\sqrt{2}} \quad \text{ e, razionalizzando: } \quad 
  lato = ipotenusa \cdot \frac{\sqrt{2}}{2}\]
% \vspace{-6pt}
\subsubsection{Triangolo equilatero}

Non possiamo applicare il teorema di Pitagora al triangolo equilatero, 
ma se ne consideriamo la metà, otteniamo un triangolo rettangolo con un angolo 
di~60° e un angolo di~30°. In questo triangolo il cateto minore, quello opposto 
all'angolo di~30° è lungo la metà dell'ipotenusa e il cateto maggiore 
(l'altezza del triangolo equilatero) si ottiene dal teorema di Pitagora:

\[altezza = \sqrt{lato^2 - \left (\frac{lato}{2} \right )^2} = 
            \sqrt{lato^2 - \frac{lato^2}{4}}= \]
\[          =\sqrt{\frac{4 \cdot lato^2 - lato^2}{4}}=
            \sqrt{\frac{3 \cdot lato^2}{4}}=
            lato \cdot \frac{\sqrt{3}}{2} \]

Con le regole del calcolo delle formule inverse possiamo trovare che:

\[\frac{lato}{2} = altezza \cdot \frac{1}{\sqrt{3}} \quad 
\text{ e, razionalizzando: } \quad 
  \frac{lato}{2} = altezza \cdot \frac{\sqrt{3}}{3}\]

Osserviamo i tre casi seguenti che fanno riferimento ad angoli minori di un
angolo retto. Per quanto detto nel paragrafo precedente è facile estendere le
conclusioni agli altri angoli associati.

\subsection{Angolo di 30°}

L'angolo di~30° l'abbiamo trovato nello studio del triangolo equilatero. ora 
disegniamo una circonferenza goniometrica e un angolo di 30 gradi, mettiamo in 
evidenza seno coseno e tangente e ritorniamo al problema già risolto sul 
triangolo equilatero.

 \begin{figure}[!h]
 \begin{minipage}{.45\textwidth}
  \begin{center}
\begin{inaccessibleblock}[Circonferenza goniometrica con angolo di~30°.]
    % (c) 2014 Daniele Zambelli - daniele.zambelli@gmail.com

\begin{tikzpicture}[scale=2.0, cap=round, >=latex]

% (c) 2014 Daniele Zambelli - daniele.zambelli@gmail.com

\newcommand{\circonferenzagoniometrica}[5]{%
% Colors
  \colorlet{anglecolor}{green!50!black}
  \colorlet{sincolor}{blue!50!black}
  \colorlet{tancolor}{orange!70!black}
  \colorlet{coscolor}{red!50!black}

% Local definitions
  \def \pangle{#1}
  \def \palfangle{\pangle / 2}
  \def \petichetta{#2}
  \def \psin{#3}
  \def \pcos{#4}
  \def \ptan{#5}
  
% Assi, circonferenza 
  % (c) 2014 Daniele Zambelli - daniele.zambelli@gmail.com

%%%
% Circonferenza goniometrica
%%%%
% 
% % Griglia
% \draw[gray!50, very thin, step=1] (-1.2, -1.2) grid (1.2, 1.2);

%Asse x
\draw [-{Stealth[length=2mm, open, round]}] (-1.3,0) -- (1.3,0) node [below]
      {$x$};
%Asse y
\draw [-{Stealth[length=2mm, open, round]}] (0, -1.3) -- (0, 1.3) node [left]  
      {$y$};
%Circonferenza
\draw [very thin] (0,0) circle (1);

% Tangente
  \draw[style=help lines] (1, -1.2) -- (1, 1.8);
% Punti legati al sistema di riferimento
  \coordinate (origin) at (0, 0);
  \draw (origin) node [below left] {$O$} [fill] circle(0.4pt);
  \draw (1, 0) node [below right] {$K$} [fill] circle(0.4pt);
% Parte che dipende dall'angolo
  \begin{scope}[very thick]
% Angolo e archi
    \begin{scope}[anglecolor]
      \draw [anglecolor, rotate=\pangle] (0, 0) -- (2.2, 0);
      \begin{scope}[->]
        \draw (0.3, 0) arc(0:\pangle:0.3);
        \draw (1, 0) arc(0:\pangle:1);
        \draw [-, anglecolor] (\palfangle:.45) node {\petichetta};
      \end{scope}
    \end{scope}
% Seno
    \draw [sincolor]
      (\pangle:1) node (p) [above] {$P$} circle(0.2pt)-- 
      node[left] {\psin} (p |- origin) 
      node (h) {}; 
     \draw (h) node [black, below] {$H$} [fill] circle(0.2pt); 
% Coseno
    \draw [coscolor] (h) -- 
      node [coscolor, below] {\pcos} (origin);
% Tangente
    \draw [tancolor] (1,0) --
      node [right=1pt,fill=white] {\ptan} 
      (intersection of 0,0 -- \pangle:1 and 1,0--1,1) [below right] node {$T$}
        circle(0.2pt);
  \end{scope}
}


\clip(-.2, -.4) rectangle (1.4, 1.4);

\circonferenzagoniometrica {30} {$30$} {$\frac{1}{2}$}
                            {$\frac{\sqrt{3}}{2}$} {$\frac{\sqrt{3}}{3}$}

\end{tikzpicture}

    \caption{Angolo di 30°}
    \label{fig:trigo_angolo_30}
\end{inaccessibleblock}
  \end{center}
 \end{minipage}
 \begin{minipage}{.45\textwidth}
Consideriamo i due triangoli rettangoli $OHP$ e $OKT$ possiamo riconoscere che:
\begin{itemize*}
 \item il cateto $OH$ corrisponde all'altezza del triangolo equilatero; 
 \item il cateto $HP$ corrisponde a metà del lato del triangolo equilatero;
 \item il cateto $KT$ corrisponde a metà del lato del triangolo equilatero.
\end{itemize*}
 \end{minipage}
\end{figure}
\vspace{-12pt}

Tenendo presente che: $OP=1$ e $OK=1$ applichiamo a questi casi i risultati 
ottenuti sopra:
\begin{itemize*}
 \item il coseno dell'angolo di~30° è uguale a $\frac{\sqrt{3}}{2}$: 
  $\cos 30 = \frac{\sqrt{3}}{2}$; 
 \item il seno dell'angolo di~30° è uguale a $\frac{1}{2}$: 
  $\sin 30 = \frac{1}{2}$; 
 \item la tangente dell'angolo di~30° è uguale a $\frac{1}{2}$: 
  $\tan 30 = \frac{\sqrt{3}}{3}$.
\end{itemize*}

\subsection{Angolo di 45°}

L'angolo di~45° l'abbiamo trovato nello studio del quadrato. ora 
disegniamo una circonferenza goniometrica e un angolo di 45 gradi, mettiamo in 
evidenza seno coseno e tangente e ritorniamo al problema già risolto sul 
quadrato.

 \begin{figure}[!h]
 \begin{minipage}{.45\textwidth}
  \begin{center}
\begin{inaccessibleblock}[Circonferenza goniometrica con angolo di~45°.]
    % (c) 2014 Daniele Zambelli - daniele.zambelli@gmail.com

\begin{tikzpicture}[scale=2.0, cap=round, >=latex]

% (c) 2014 Daniele Zambelli - daniele.zambelli@gmail.com

\newcommand{\circonferenzagoniometrica}[5]{%
% Colors
  \colorlet{anglecolor}{green!50!black}
  \colorlet{sincolor}{blue!50!black}
  \colorlet{tancolor}{orange!70!black}
  \colorlet{coscolor}{red!50!black}

% Local definitions
  \def \pangle{#1}
  \def \palfangle{\pangle / 2}
  \def \petichetta{#2}
  \def \psin{#3}
  \def \pcos{#4}
  \def \ptan{#5}
  
% Assi, circonferenza 
  % (c) 2014 Daniele Zambelli - daniele.zambelli@gmail.com

%%%
% Circonferenza goniometrica
%%%%
% 
% % Griglia
% \draw[gray!50, very thin, step=1] (-1.2, -1.2) grid (1.2, 1.2);

%Asse x
\draw [-{Stealth[length=2mm, open, round]}] (-1.3,0) -- (1.3,0) node [below]
      {$x$};
%Asse y
\draw [-{Stealth[length=2mm, open, round]}] (0, -1.3) -- (0, 1.3) node [left]  
      {$y$};
%Circonferenza
\draw [very thin] (0,0) circle (1);

% Tangente
  \draw[style=help lines] (1, -1.2) -- (1, 1.8);
% Punti legati al sistema di riferimento
  \coordinate (origin) at (0, 0);
  \draw (origin) node [below left] {$O$} [fill] circle(0.4pt);
  \draw (1, 0) node [below right] {$K$} [fill] circle(0.4pt);
% Parte che dipende dall'angolo
  \begin{scope}[very thick]
% Angolo e archi
    \begin{scope}[anglecolor]
      \draw [anglecolor, rotate=\pangle] (0, 0) -- (2.2, 0);
      \begin{scope}[->]
        \draw (0.3, 0) arc(0:\pangle:0.3);
        \draw (1, 0) arc(0:\pangle:1);
        \draw [-, anglecolor] (\palfangle:.45) node {\petichetta};
      \end{scope}
    \end{scope}
% Seno
    \draw [sincolor]
      (\pangle:1) node (p) [above] {$P$} circle(0.2pt)-- 
      node[left] {\psin} (p |- origin) 
      node (h) {}; 
     \draw (h) node [black, below] {$H$} [fill] circle(0.2pt); 
% Coseno
    \draw [coscolor] (h) -- 
      node [coscolor, below] {\pcos} (origin);
% Tangente
    \draw [tancolor] (1,0) --
      node [right=1pt,fill=white] {\ptan} 
      (intersection of 0,0 -- \pangle:1 and 1,0--1,1) [below right] node {$T$}
        circle(0.2pt);
  \end{scope}
}


\clip(-.2, -.4) rectangle (1.4, 1.4);

\circonferenzagoniometrica {45} {$45$} {$\frac{\sqrt{2}}{2}$}
                            {$\frac{\sqrt{2}}{2}$} {$1$}

\end{tikzpicture}

    \caption{Angolo di 45°}
    \label{fig:trigo_angolo_45}
\end{inaccessibleblock}
  \end{center}
 \end{minipage}
 \begin{minipage}{.45\textwidth}
Consideriamo i due triangoli rettangoli $OHP$ e $OKT$ possiamo riconoscere che:
\begin{itemize*}
 \item il cateto $OH$ è congruente al cateto $HP$; 
 \item il cateto $OK$ è congruente al cateto $KT$.
\end{itemize*}
 \end{minipage}
\end{figure}
\vspace{-12pt}

Tenendo presente che: $OP=1$ e $OK=1$ applichiamo a questi casi i risultati 
ottenuti sopra:
\begin{itemize*}
 \item il coseno e il seno dell'angolo di~45° sono uguali;
 \item coseno e seno dell'angolo di~45° sono uguali a $\frac{\sqrt{2}}{2}$: 
  $\cos 45 = \sin 45 = \frac{\sqrt{2}}{2}$; 
 \item la tangente dell'angolo di~45° è uguale al raggio: 
  $\tan 45 = 1$.
\end{itemize*}

\subsection{Angolo di 60°}

L'angolo di~60° l'abbiamo trovato nello studio del triangolo equilatero. ora 
disegniamo una circonferenza goniometrica e un angolo di 30 gradi, mettiamo in 
evidenza seno coseno e tangente e ritorniamo al problema già risolto sul 
triangolo equilatero.

 \begin{figure}[!h]
 \begin{minipage}{.45\textwidth}
  \begin{center}
\begin{inaccessibleblock}[Circonferenza goniometrica con angolo di~60°.]
    % (c) 2014 Daniele Zambelli - daniele.zambelli@gmail.com

\begin{tikzpicture}[scale=2.0, cap=round, >=latex]

% (c) 2014 Daniele Zambelli - daniele.zambelli@gmail.com

\newcommand{\circonferenzagoniometrica}[5]{%
% Colors
  \colorlet{anglecolor}{green!50!black}
  \colorlet{sincolor}{blue!50!black}
  \colorlet{tancolor}{orange!70!black}
  \colorlet{coscolor}{red!50!black}

% Local definitions
  \def \pangle{#1}
  \def \palfangle{\pangle / 2}
  \def \petichetta{#2}
  \def \psin{#3}
  \def \pcos{#4}
  \def \ptan{#5}
  
% Assi, circonferenza 
  % (c) 2014 Daniele Zambelli - daniele.zambelli@gmail.com

%%%
% Circonferenza goniometrica
%%%%
% 
% % Griglia
% \draw[gray!50, very thin, step=1] (-1.2, -1.2) grid (1.2, 1.2);

%Asse x
\draw [-{Stealth[length=2mm, open, round]}] (-1.3,0) -- (1.3,0) node [below]
      {$x$};
%Asse y
\draw [-{Stealth[length=2mm, open, round]}] (0, -1.3) -- (0, 1.3) node [left]  
      {$y$};
%Circonferenza
\draw [very thin] (0,0) circle (1);

% Tangente
  \draw[style=help lines] (1, -1.2) -- (1, 1.8);
% Punti legati al sistema di riferimento
  \coordinate (origin) at (0, 0);
  \draw (origin) node [below left] {$O$} [fill] circle(0.4pt);
  \draw (1, 0) node [below right] {$K$} [fill] circle(0.4pt);
% Parte che dipende dall'angolo
  \begin{scope}[very thick]
% Angolo e archi
    \begin{scope}[anglecolor]
      \draw [anglecolor, rotate=\pangle] (0, 0) -- (2.2, 0);
      \begin{scope}[->]
        \draw (0.3, 0) arc(0:\pangle:0.3);
        \draw (1, 0) arc(0:\pangle:1);
        \draw [-, anglecolor] (\palfangle:.45) node {\petichetta};
      \end{scope}
    \end{scope}
% Seno
    \draw [sincolor]
      (\pangle:1) node (p) [above] {$P$} circle(0.2pt)-- 
      node[left] {\psin} (p |- origin) 
      node (h) {}; 
     \draw (h) node [black, below] {$H$} [fill] circle(0.2pt); 
% Coseno
    \draw [coscolor] (h) -- 
      node [coscolor, below] {\pcos} (origin);
% Tangente
    \draw [tancolor] (1,0) --
      node [right=1pt,fill=white] {\ptan} 
      (intersection of 0,0 -- \pangle:1 and 1,0--1,1) [below right] node {$T$}
        circle(0.2pt);
  \end{scope}
}


\clip(-.2, -.3) rectangle (1.4, 1.8);

\circonferenzagoniometrica {60} {$60$} {$\frac{\sqrt{3}}{2}$}
                            {$\frac{1}{2}$} {$\sqrt{3}$}

\end{tikzpicture}

    \caption{Angolo di 60°}
    \label{fig:trigo_angolo_60}
\end{inaccessibleblock}
  \end{center}
 \end{minipage}
 \begin{minipage}{.45\textwidth}
Consideriamo i due triangoli rettangoli $OHP$ e $OKT$ possiamo riconoscere che:
\begin{itemize*}
 \item il cateto $OH$ corrisponde a metà lato del triangolo equilatero; 
 \item il cateto $HP$ corrisponde all'altezza del triangolo equilatero;
 \item il cateto $KT$ corrisponde all'altezza del triangolo equilatero.
\end{itemize*}
 \end{minipage}
\vspace{-12pt}
\end{figure}

\vspace{-6pt}

Tenendo presente che: $OP=1$ e $OK=1$ applichiamo a questi casi i risultati 
ottenuti sopra:
\begin{itemize*}
 \item il coseno dell'angolo di~60° è uguale a $\frac{1}{2}$: 
  $\cos 60 = \frac{1}{2}$; 
 \item il seno dell'angolo di~60° è uguale a $\frac{\sqrt{3}}{2}$: 
  $\sin 60 = \frac{\sqrt{3}}{2}$; 
 \item la tangente dell'angolo di~60° è uguale a $\sqrt{3}$: 
  $\tan 60 = \sqrt{3}$.
\end{itemize*}

\vspace{-6pt}

\section{Formule goniometriche}
\label{sec:gonio_formule}

È evidente che con le funzioni goniometriche certe scorciatoie non sono 
possibili: il seno della somma di due angoli non può essere uguale alla 
somma dei seni:
\vspace{-6pt}
\[\sin \left( \alpha + \beta \right) \ne \sin \alpha + \sin \beta\]
Si può ad esempio considerare l'angolo di~90°: la somma di due angoli di~90° 
è un angoli di~180° il cui seno vale~0 mentre la somma dei seni di  due angoli 
di~90° vale~2:
\vspace{-6pt}
\[0 = \sin 180 = \sin \left( 90 + 90 \right) \ne \sin 90 + \sin 90 = 2 \]
Di seguito sono riportate, senza dimostrazione, alcune formule della 
goniometria, verificale con qualche esempio.
\vspace{-6pt}

\begin{itemize*}
 \item Formule di addizione
  \subitem $\sin(\alpha + \beta)=\sin\alpha \cos\beta + \cos\alpha \sin\beta$
  \subitem $\cos(\alpha + \beta)=\cos\alpha \cos\beta - \sin\alpha \sin\beta$
  \subitem $\tan(\alpha + \beta)=\frac {\tan\alpha + \tan\beta} 
                                        {1 - \tan\alpha \tan\beta}$
 \item Formule di sottrazione
  \subitem $\sin(\alpha - \beta)=\sin\alpha \cos\beta - \cos\alpha \, \sin\beta$
  \subitem $\cos(\alpha - \beta)=\cos\alpha \cos\beta + \sin\alpha \sin\beta$
  \subitem $\tan(\alpha - \beta)=\frac {\tan\alpha - \tan\beta} 
                                        {1 + \tan\alpha \tan\beta}$
 \item Formule di duplicazione
  \subitem $\sin(2\alpha)=2\sin\alpha \cos\alpha$
  \subitem $\cos(2\alpha)=\cos^2\alpha - \sin^2\alpha = 1 - 2\sin^{2}\alpha = 
            2\cos^{2}\alpha - 1$
  \subitem $\tan(2\alpha)=\frac{2\tan\alpha}{1 - \tan^2\alpha}$
 \item Formule di bisezione
  \subitem $\cos\left(\frac{\alpha} 2\right)=\pm\sqrt{\frac{1+\cos\alpha}{2} }$
  \subitem $\sin\left(\frac{\alpha} 2\right)=\pm\sqrt{\frac{1-\cos\alpha}{2} }$
  \subitem $\tan\left(\frac{\alpha} 2\right)=\pm\sqrt{\frac{1-\cos\alpha}
                                                           {1+\cos\alpha}}$
\end{itemize*}

\section{Equazioni goniometriche}
\label{sec:gonio_equazionigonio}

Risolvere un'equazione goniometrica vuol dire trovare gli angoli che rendono 
vera un'uguaglianza che contiene funzioni goniometriche. 
Alla realtà, e anche ai matematici piace inventare situazioni che richiedono, 
per essere studiate, di risolvere equazioni goniometriche complesse, ma noi ci 
accontenteremo di casi molto semplici.

\subsection{Equazioni goniometriche elementari}

Chiamiamo equazione goniometrica elementare una equazione che può essere 
scritta nella forma:
\[f(x)=k\]

dove $f(x)$ è una funzione goniometrica, ad esempio $\sin x$, $\cos x$, 
$\tan x$ e $k$ è una costante.

% \begin{exrig}
 \begin{esempio}
  $3(\sin x +2) - \sin x = 4 (\sin x +4)-2$
  
  Per prima cosa cerchiamo di scriverlo in forma normale:
  
  $3 \sin x +6 - \sin x = 4 \sin x +16-2$
  
  $3 \sin x - \sin x - 4 \sin x = +16-2-6 $
  
  $- 2 \sin x = +8 \quad \Rightarrow \quad \sin x = -4 $
  
  L'equazione di partenza non ha soluzioni, infatti $\sin x$ è sempre compreso 
  nell'intervallo $\left [-1;~+1 \right]$ e non può valere~$-4$ per alcun 
  valore di~$x$.
 \end{esempio}

 \begin{esempio}
  $(\sin x -2)^2 - 2 \sin x -4 = (\sin x +2)(\sin x -2)+1$
  
  Per prima cosa cerchiamo di scriverlo in forma normale:
  
  $(\sin^2 x -4 \sin x +4 - 2 \sin x -4 = \sin^2x -4 +1$
  
  $-6 \sin x = -3 \quad \Rightarrow \quad \sin x = \frac{1}{2}$
  
\begin{figure}[!h] 
 \vspace{-6pt}
  \begin{center}
\begin{inaccessibleblock}[Soluzione grafica dell'equazione: 
    $\sin x = \frac{1}{2}$.]
    % (c) 2014 Daniele Zambelli - daniele.zambelli@gmail.com

\begin{tikzpicture}[x=13mm,y=13mm, font=\small, cap=round, >=latex]


% (c) 2014 Daniele Zambelli - daniele.zambelli@gmail.com

% \begin{tikzpicture}[x=17mm,y=17mm, font=\small, cap=round, >=latex]

\colorlet{anglecolor}{green!50!black}
\colorlet{sincolor}{blue!50!black}
% \colorlet{coscolor}{red!50!black}
% \colorlet{tancolor}{orange!70!black}

\def \_deg2rad{0.01745329252}

\newcommand{\eqsin}[3]{% 
% Soluzione dell'equazione $\sin x = k$ 
% nella circonferenza goniometrica e nella sinusoide.
% Chiamata tipica: \eqsin{.5}{30}{150}

% Parametri
  \def \psin{#1}
  \def \padeg{#2}
  \def \pbdeg{#3}
  \def \parad{#2*\_deg2rad}
  \def \pbrad{#3*\_deg2rad}
% Assi
  \begin{scope}[-{Stealth[length=2mm, open, round]}]
    \draw (-2.1,0) -- (6.5, 0) node [below] {$x$}; % Asse x
    \draw (-1, -1.3) -- (-1, 1.3) node [left] {$y'$}; % Asse y'
    \draw (0, -1.3) -- (0, 1.3) node [left] {$\sin x$}; % Asse y
    \foreach \y in {-1, -.5, +.5, +1}{
      \draw [-] (-0.02, \y) -- (+0.05, \y);}
  \end{scope}
% Circonferenza
  \coordinate (a) at (-1, 0);
  \coordinate (b) at (0, 0);
  \node(c0) at (a)[draw, circle through=(b)] {};  
% Tacche con etichetta asse x
  \begin{scope}[font=\tiny]
  \foreach \x/\xtext in {
      0.0/0, 0.5236/30, 1.047/60, 1.571/90, 2.094/120, 2.618/150, 
      3.142/180, 3.665/210, 4.189/240, 4.712/270, 5.236/300, 5.76/330, 6.283/360}
  \node[below] at(\x, 0) {$\xtext \grado$};
  \end{scope}
% Tacche asse x
  \begin{scope}[font=\tiny]
  \foreach \x in { 
      0.0, 0.2618, 0.5236, 0.7854, 1.047, 1.309, 1.571, 1.833, 
      2.094, 2.356, 2.618, 2.88, 3.142, 3.403, 3.665, 3.927, 
      4.189, 4.451, 4.712, 4.974, 5.236, 5.498, 5.76, 6.021, 6.283}
  {\draw [black] (\x, -0.02) -- (\x, +0.05) node (a) {};}
  \end{scope}
%Sinusoide
  \tkzInit[xmin=0,xmax=+6.5,ymin=-1.1,ymax=+1.1]
  \tkzFct[domain=0:+6.5, thick]{sin(x)}
% Segmento
  \begin{scope}[sincolor, thick]
    \draw (-1, 0) -- (-1, \psin) [fill] circle(1.5pt);
    \draw (-2.1, \psin) coordinate (pt0) -- 
          (+6.5, \psin) coordinate (pt1);
  \end{scope}
% Intersezioni
  \begin{scope}[anglecolor]
%     \begin{scope}[->, fill]
    \draw [->] (-1, 0) --
      (intersection 1 of c0 and pt0--pt1) [fill] circle(1.5pt);
    \draw [->] (-1, 0) -- 
      (intersection 1 of c0 and pt1--pt0) [fill] circle(1.5pt);
%     \end{scope}
% Soluzioni
      \draw (\parad, 0) [fill] circle(1.5pt) -- 
            (\parad, \psin) [fill] circle(1.5pt);
      \draw (\pbrad, 0) [fill] circle(1.5pt) -- 
            (\pbrad, \psin) [fill] circle(1.5pt);
% Angoli
    \begin{scope}[anglecolor]
%       \draw [anglecolor, rotate=#2, xshift=-1cm] (0, 0) -- (2.2, 0); perché 
%       \draw [anglecolor, rotate=#3, xshift=-1cm] (0, 0) -- (2.2, 0); non va???
      \begin{scope}[->]
        \draw (-1+0.3, 0) arc(0:\padeg:0.3);
        \draw (-1+0.4, 0) arc(0:\pbdeg:0.4);
      \end{scope}
%         \draw (#2/2:.3.5) [xshift=-1] node {$\alpha_0$}; % perché 
%         \draw (#3/2:.4.5) [xshift=-1] node {$\alpha_1$}; % non va???
    \end{scope}
  \end{scope}

}


\eqsin{.5}{30}{150}


\end{tikzpicture}
    \caption{Soluzione grafica di: $\sin x = \frac{1}{2}$}
    \label{fig:trigo_equazione01}
\end{inaccessibleblock}
  \end{center}
\vspace{-18pt}
\end{figure} 

  Ora disegniamo la circonferenza goniometrica e anche la funzione~$\sin$ 
  e le intersechiamo con la retta~$y=\frac{1}{2}$.
  Scopriamo così che, tra~0° e~360° ci sono due angoli che rendono vera 
  l'equazione e sono:~30° e~150°. 
 \end{esempio}

 \begin{esempio}
  $2 ( -\cos x +1) + 3 (\cos x -2 \sqrt{2}) = 3 \cos x -\sqrt{2} (5 -\sqrt{2})$
  
  Per prima cosa la scriviamo in forma normale:
  
  $- 2 \cos x +2 + 3 \cos x -6 \sqrt{2} = 3 \cos x -5 \sqrt{2} +2$
  
  $-2 \cos x  = \sqrt{2} \quad \Rightarrow \quad \cos x  = -\frac{\sqrt{2}}{2}$
  
  Ora disegniamo la circonferenza goniometrica e anche la funzione~$\sin$ 
  e le intersechiamo con la retta~$y=\frac{1}{2}$.
  
\begin{figure}[!h] 
 \vspace{-6pt}
  \begin{center}
\begin{inaccessibleblock}[Soluzione grafica dell'equazione: 
    $\cos x = \frac{\sqrt{2}}{2}$.]
    % (c) 2014 Daniele Zambelli - daniele.zambelli@gmail.com

\begin{tikzpicture}[x=10mm,y=10mm, font=\small, cap=round, >=latex]
% (c) 2014 Daniele Zambelli - daniele.zambelli@gmail.com

% \begin{tikzpicture}[x=17mm,y=17mm, font=\small, cap=round, >=latex]

\colorlet{anglecolor}{green!30!black}
% \colorlet{sincolor}{blue!50!black}
\colorlet{coscolor}{red!50!black}
% \colorlet{tancolor}{orange!70!black}

\def \_deg2rad{0.01745329252}

\newcommand{\eqcos}[3]{%
% Soluzione dell'equazione $\cos x = k$ 
% nella circonferenza goniometrica e nella sinusoide.
% Chiamata tipica: \eqcos{-0.707106781187}{135}{225}

% Parametri
  \def \pcos{#1}
  \def \padeg{#2}
  \def \pbdeg{#3}
  \def \parad{#2*\_deg2rad}
  \def \pbrad{#3*\_deg2rad}
% Assi
  \begin{scope}[-{Stealth[length=2mm, open, round]}]
    \draw (-2.1,0) -- (6.5, 0) node [below] {$x$}; % Asse x
    \draw (-1, -1.3) -- (-1, 1.3) node [left] {$y'$}; % Asse y'
    \draw (0, -1.3) -- (0, 1.3) node [left] {$\cos x$}; % Asse y
    \foreach \y in {-1, -.5, +.5, +1}{
      \draw [-] (-0.02, \y) -- (+0.05, \y);}
  \end{scope}
% Circonferenza
  \coordinate (a) at (-1, 0);
  \coordinate (b) at (0, 0);
  \node(c0) at (a)[draw, circle through=(b)] {};  
  % Tacche con etichetta asse x
  \begin{scope}[font=\tiny]
  \foreach \x/\xtext in {
      0.0/0, 0.5236/30, 1.047/60, 1.571/90, 2.094/120, 2.618/150, 
      3.142/180, 3.665/210, 4.189/240, 4.712/270, 5.236/300, 5.76/330, 6.283/360}
  \node[below] at(\x, 0) {$\xtext \grado$};
  \end{scope}
% Tacche asse x
  \begin{scope}[font=\tiny]
  \foreach \x in { 
      0.0, 0.2618, 0.5236, 0.7854, 1.047, 1.309, 1.571, 1.833, 
      2.094, 2.356, 2.618, 2.88, 3.142, 3.403, 3.665, 3.927, 
      4.189, 4.451, 4.712, 4.974, 5.236, 5.498, 5.76, 6.021, 6.283}
  {\draw [black] (\x, -0.02) -- (\x, +0.05) node (a) {};}
  \end{scope}
%Cosinusoide
  \tkzInit[xmin=0,xmax=+6.5,ymin=-1.1,ymax=+1.1]
  \tkzFct[domain=0:+6.5, thick]{cos(x)}
% Segmenti
  \begin{scope}[coscolor, thick]
    \draw (-1, 0) -- (-1+\pcos, 0) [fill] circle(1.5pt);
    \draw (-1+\pcos, -1.2) coordinate (pt0) -- 
          (-1+\pcos, +1.2) coordinate (pt1);
    \draw (0, \pcos) -- (6.2, \pcos);
  \end{scope}
% Intersezioni
  \begin{scope}[anglecolor]
%     \begin{scope}[->, fill]
    \draw [->] (-1, 0) --
      (intersection 1 of c0 and pt0--pt1) [fill] circle(1.5pt);
    \draw [->] (-1, 0) -- 
      (intersection 1 of c0 and pt1--pt0) [fill] circle(1.5pt);
%     \end{scope}
    \draw (\parad, 0) [fill] circle(1.5pt) -- 
          (\parad, \pcos) [fill] circle(1.5pt);
    \draw (\pbrad, 0) [fill] circle(1.5pt) -- 
          (\pbrad, \pcos) [fill] circle(1.5pt);
% Angoli
    \begin{scope}[anglecolor]
%       \draw [anglecolor, rotate=#2, xshift=-1cm] (0, 0) -- (2.2, 0); perché 
%       \draw [anglecolor, rotate=#3, xshift=-1cm] (0, 0) -- (2.2, 0); non va???
      \begin{scope}[->]
        \draw (-1+0.3, 0) arc(0:\padeg:0.3);
        \draw (-1+0.4, 0) arc(0:\pbdeg:0.4);
      \end{scope}
%         \draw (#2/2:.3.5) [xshift=-1] node {$\alpha_0$}; % perché 
%         \draw (#3/2:.4.5) [xshift=-1] node {$\alpha_1$}; % non va???
    \end{scope}
  \end{scope}

}

\eqcos{-0.707106781187}{135}{225}
\end{tikzpicture}

    \caption{Soluzione grafica di: $\cos x = \frac{\sqrt{2}}{2}$}
    \label{fig:trigo_equazione02}
\end{inaccessibleblock}
  \end{center}
\vspace{-18pt}
\end{figure} 
  Scopriamo così che, tra~0° e~360° ci sono due angoli che rendono vera 
  l'equazione e sono:~45° e~315°~=~-45°. 
 \end{esempio}

 \begin{esempio}
  $\sin x (1 + \sqrt{3}) + \sqrt{3} = \sqrt{3}(\cos x + \sin x +1)$
  
  Per prima cosa la scriviamo in forma normale:
  
  $\sin x + \sqrt{3} \sin x + \sqrt{3} = 
   \sqrt{3} \cos x + \sqrt{3} \sin x + \sqrt{3}$
   
  $\sin x  = \sqrt{3} \cos x $
  
 Questa volta abbiamo ottenuto una equazione che non è elementare: 
  contiene sia $\sin x$ sia $\cos x$.
  A questo punto possiamo usare uno sporco trucco:
  
 \begin{figure}[!h]
 \begin{minipage}{.50\textwidth}
  \begin{center}
\begin{inaccessibleblock}[Soluzione grafica dell'equazione: 
    $\tan x = \sqrt{3}$.]
    % (c) 2014 Daniele Zambelli - daniele.zambelli@gmail.com

\begin{tikzpicture}[x=7mm,y=7mm, font=\small, cap=round, >=latex]


% (c) 2014 Daniele Zambelli - daniele.zambelli@gmail.com

% \begin{tikzpicture}[x=17mm,y=17mm, font=\small, cap=round, >=latex]

\colorlet{anglecolor}{green!50!black}
% \colorlet{sincolor}{blue!50!black}
% \colorlet{coscolor}{red!50!black}
\colorlet{tancolor}{orange!70!black}

\def \_deg2rad{0.01745329252}

\newcommand{\eqtan}[3]{%
% Soluzione dell'equazione $\tan x = k$ 
% nella circonferenza goniometrica e nella tangentoide.
% Chiamata tipica: \eqtan{1.73205080757}{60}{240}

% Parametri
  \def \ptan{#1}
  \def \padeg{#2}
  \def \pbdeg{#3}
  \def \parad{#2*\_deg2rad}
  \def \pbrad{#3*\_deg2rad}
% Assi
  \begin{scope}[-{Stealth[length=2mm, open, round]}]
    \draw (-2.1,0) -- (6.5, 0) node [below] {$x$}; % Asse x
    \draw (-1, -1.3) -- (-1, 1.3) node [left] {$y'$}; % Asse y'
    \draw (0, -4.3) -- (0, 4.3) node [left] {$\tan x$}; % Asse y
    \foreach \y in {-4, -3.5, ..., 4}{
      \draw [-] (-0.02, \y) -- (+0.05, \y);}
  \end{scope}
% Circonferenza
  \coordinate (a) at (-1, 0);
  \coordinate (b) at (0, 0);
  \node(c0) at (a)[draw, circle through=(b)] {};  
  % Tacche con etichetta asse x
  \begin{scope}[font=\tiny]
  \foreach \x/\xtext in {
      0.0/0, 0.5236/30, 1.047/60, 1.571/90, 2.094/120, 2.618/150, 
      3.142/180, 3.665/210, 4.189/240, 4.712/270, 5.236/300, 5.76/330, 6.283/360}
  \node[below] at(\x, 0) {$\xtext \grado$};
  \end{scope}
% Tacche asse x
  \begin{scope}[font=\tiny]
  \foreach \x in { 
      0.0, 0.2618, 0.5236, 0.7854, 1.047, 1.309, 1.571, 1.833, 
      2.094, 2.356, 2.618, 2.88, 3.142, 3.403, 3.665, 3.927, 
      4.189, 4.451, 4.712, 4.974, 5.236, 5.498, 5.76, 6.021, 6.283}
  {\draw [black] (\x, -0.02) -- (\x, +0.05) node (a) {};}
  \end{scope}
% Tangentoide
  \tkzInit[xmin=0,xmax=+6.5,ymin=-4.1,ymax=+4.1]
  \tkzFct[domain=0:+1.5, thick]{tan(x)}
  \tkzFct[domain=+1.7:4.6, thick]{tan(x)}
  \tkzFct[domain=4.8:+6.5, thick]{tan(x)}
% Segmento
  \begin{scope}[tancolor, thick]
    \draw (0, \ptan) [fill] circle(1.5pt) -- (6.2, \ptan);
%     \draw (-2.1, \ptan) coordinate (pt0) -- 
%           (+6.5, \ptan) coordinate (pt1);
  \end{scope}
% Intersezioni
  \begin{scope}[anglecolor]
% %     \begin{scope}[->, fill]
    \coordinate (pt0) at (0, \ptan);
    \coordinate (pt1) at (-2, -\ptan);
    \draw (pt0) -- (pt1);
    \draw (intersection 1 of c0 and pt0--pt1) [fill] circle(1.5pt);
    \draw (intersection 1 of c0 and pt1--pt0) [fill] circle(1.5pt);
    \draw (\parad, 0) [fill] circle(1.5pt) -- 
          (\parad, \ptan) [fill] circle(1.5pt);
    \draw (\pbrad, 0) [fill] circle(1.5pt) -- 
          (\pbrad, \ptan) [fill] circle(1.5pt);
% Angoli
    \begin{scope}[anglecolor]
%       \draw [anglecolor, rotate=#2, xshift=-1cm] (0, 0) -- (2.2, 0); perché 
%       \draw [anglecolor, rotate=#3, xshift=-1cm] (0, 0) -- (2.2, 0); non va???
      \begin{scope}[->]
        \draw (-1+0.3, 0) arc(0:\padeg:0.3);
        \draw (-1+0.4, 0) arc(0:\pbdeg:0.4);
      \end{scope}
%         \draw (#2/2:.3.5) [xshift=-1] node {$\alpha_0$}; % perché 
%         \draw (#3/2:.4.5) [xshift=-1] node {$\alpha_1$}; % non va???
    \end{scope}
  \end{scope}

}


\eqtan{1.73205080757}{60}{240}


\end{tikzpicture}
    \caption{Soluzione grafica di: $\tan x = \sqrt{3}$}
    \label{fig:trigo_equazione03}
\end{inaccessibleblock}
  \end{center}
 \end{minipage}
 \begin{minipage}{.48\textwidth}
  \begin{enumerate*}
   \item controlliamo che $x = 90$ e $x = 180$ non siano soluzioni 
    dell'equazione;
   \item se è vero il punto~1, dividiamo entrambi i membri per~$\cos x$.
  \end{enumerate*}
  Otteniamo:

  $\frac{\sin x}{\cos x} = \sqrt{3}$
  
  e ricordandoci della definizione di tangente:

  $\tan x = \sqrt{3}$
  
  Ora disegniamo la circonferenza goniometrica e anche la funzione~$\tan$ 
  e la intersechiamo con la retta~$y=\sqrt{3}$.
  
  Scopriamo così che, tra~0° e~360° ci sono due angoli che rendono vera 
  l'equazione e sono:~60° e 240°. 
 \end{minipage}
\end{figure}
 \end{esempio}

% \end{exrig}

 Normalmente, risolvendo un'equazione si incappa in valori che non 
 corrispondono ad angoli noti (0\grado, 30\grado, ...), come fare in questi 
 casi? Possiamo utilizzare degli strumenti che ci aiutino ad ottenere un 
 risultato approssimato. Il primo strumento lo abbiamo realizzato all'inizio 
 del capitolo. La circonferenza goniometrica che abbiamo disegnato per 
 costruire le funzioni goniometriche ci permette anche di seguire il 
 procedimento inverso e passare dal valore di una funzione all'angolo. 
 Vediamo un esempio.
 
 \begin{esempio}
  $6 ( \cos x -1) -3 ( \cos x +1) = -6 \cos x - \cos x$
  
  Eseguendo i calcoli si arriva all'equazione scritta in forma elementare:
  
  $6 \cos x -6 -3 \cos x -3 = -6 \cos x - \cos x$
  
  $6 \cos x -3 \cos x +6 \cos x + \cos x = +6 +3$
  
  $10 \cos x= 9 \Rightarrow \cos x = \frac{9}{10} = 0,9$


 \begin{figure}[!h]
 \vspace{-6pt}
 \begin{minipage}{.50\textwidth}
Nella circonferenza goniometrica individuiamo sull'asse~x (nel caso del 
coseno) il valore~$0,9$ e da questo risaliamo agli angoli che lo generano.
Tracciamo la retta~$x=.9$ (parallela all'asse~y) e la intersechiamo con
la circonferenza goniometrica. Poi tracciamo due semirette che uniscono 
l'origine degli assi con le intersezioni. Gli angoli compresi tra l'asse~x 
e queste due rette hanno il coseno cercato. Senz'altro l'angolo~$x_1$ è 
compreso tra~$15\grado$ e~$30\grado$.

Possiamo stimarlo con maggior precisione osservando che è più vicino a~$30$ 
e che si trova circa ad un terzo di distanza tra i due valori. 
Possiamo quindi valutare per gli angoli i valori:

$x_0 \approx -25\grado, x_1 \approx +25\grado$
 \end{minipage}
 \begin{minipage}{.45\textwidth}
  \begin{center}
\begin{inaccessibleblock}[Soluzione grafica approssimata dell'equazione: 
    $\cos x = 0,8$.]
    % (c) 2014 Daniele Zambelli - daniele.zambelli@gmail.com

\colorlet{anglecolor}{green!30!black}
% \colorlet{sincolor}{blue!50!black}
\colorlet{coscolor}{red!50!black}
% \colorlet{tancolor}{orange!70!black}

\def \_deg2rad{0.01745329252}

\newcommand{\circgoncos}[2]{%
% Circonferenza goniometrica con cos x = 0,9.

% Parametri
  \def \pcos{#1}
  \def \psin{#2}

\clip (-.08, -.65) rectangle (1.15, .65);

% Griglia
    \draw[gray!50, very thin, step=.1] (-0.15, -.63) grid (1.2, .63);
% Assi
  \begin{scope}[-{Stealth[length=2mm, open, round]}]
    \draw (-0.1, 0) -- (1.1, 0) node [below] {$x$}; % Asse x
    \draw (0, -.63) -- (0, .63) node [left] {$y$}; % Asse y
  \end{scope}
% Circonferenza
  \coordinate (centro) at (0, 0);
  \coordinate (p) at (1, 0);
  \draw (p) arc(0:55:1);  
  \draw (p) arc(0:-55:1); 
  \draw (\pcos, 0) [fill] circle(2pt) node [below left] {$\pcos$};
  \draw (\pcos, -.8) -- (\pcos, .8);
% Angoli
  \begin{scope}[anglecolor, very thick]
  \draw (\pcos, -\psin) [fill] circle(2pt);
  \draw (\pcos, \psin) [fill] circle(2pt);
    \draw (0, 0) -- (\pcos*2, \psin*2);
    \draw (0, 0) -- (\pcos*2, -\psin*2);
  \end{scope}
% Angoli di riferimento
  \foreach \z in {-75, -60, ..., +75}
 \draw [->, thin,  Maroon, rotate=\z] (0, 0) -- (1, 0) node [right] {$\z$};
}

\begin{tikzpicture}[x=45mm,y=45mm, font=\small, cap=round, >=latex]
\circgoncos{0.9}{0.435889894354}
\end{tikzpicture}

    \caption{Soluzione grafica di: $\cos x = 0,9$}
    \label{fig:trigo_equazione04}
\end{inaccessibleblock}
  \end{center}
 \end{minipage}
\vspace{-18pt}
\end{figure}
 
 \end{esempio}

A volte però vogliamo avere un risultato con una maggiore approssimazione di 
quanto possiamo ottenere a occhio. L'equazione:

$\cos x = 0,9$ è equivalente 
a~$x = \text{``funzione inversa del coseno'' } 0.9$

La funzione inversa del coseno si chiama $\arccos$ riceve come argomento una
lunghezza e dà come risultato un angolo. La seguente tabella riporta le 
funzioni goniometriche che abbiamo studiato, le loro inverse e il nome che 
viene dato loro, di solito, sulle calcolatrici:

\begin{center}
\begin{tabular}{ccc}
funzione & inversa & calcolatrice\\
$\sin$ & $\arcsin$ & $\sin^{-1}$\\
$\cos$ & $\arccos$ & $\cos^{-1}$\\
$\tan$ & $\arctan$ & $\tan^{-1}$
\end{tabular}
\end{center}

\begin{esempio}
 Riprendendo l'esempio precedente:~$\cos x = 0,9$:
 
 $x = \arccos 0,9 = 25,8419327632$ 
 
 Possiamo vedere così che nella stima fatta con la circonferenza goniometrica 
 avevamo commesso un errore minore di un grado. Non male!
\end{esempio}


\section{Disequazioni goniometriche}
\label{sec:gonio_disequazionigonio}

Possiamo ora affrontare un nuovo problema: la soluzione di disequazioni 
goniometriche elementari. La soluzione di questo problema si svolge in due 
tempi:
% \vspace{-4}
\begin{enumerate*}
 \item si risolve la disequazione nella variabile $\sin x$ 
  (o~$\cos x$ o~$\tan x$);
 \item in modo grafico si passa dalla funzione goniometrica alla variabile~$x$.
\end{enumerate*}

Anche in questo caso vediamo un esempio.

 \begin{esempio}
  $4 \sin^2 x - 2 (\sqrt{3} + 1) \sin x + \sqrt{3} \le 0$
  
  In un primo tempo possiamo considerare la disequazione come se la funzione 
  $\sin x$ fosse una semplice variabile.
  Per semplificarci la vita le cambiamo il nome:
  
  $\sin x = t$
  
  $4 t^2 - 2 (\sqrt{3} + 1) t + \sqrt{3} \le 0$
  
  L'equazione associata dà come soluzioni:
  
  $t_{1,~2} = 
   \frac{\sqrt{3} + 1 \mp \sqrt{(\sqrt{3} + 1)^2 - 4 \cdot \sqrt{3}}}{4} =  
   \frac{\sqrt{3} + 1 \mp \sqrt{3 + 2 \sqrt{3} +1 -  4 \sqrt{3}}}{4} = 
   = \dots = \frac{\sqrt{3} + 1 \mp (\sqrt{3} -1)}{4}$

  $t_{1} = \frac{\sqrt{3} + 1 - \sqrt{3} +1}{4} = 
           \frac{2}{4} = \frac{1}{2} \qquad
   t_{2} = \frac{\sqrt{3} + 1 + \sqrt{3} -1}{4} = 
           \frac{2 \sqrt{3}}{4} = \frac{\sqrt{3}}{2} $
  
  Abbiamo così trovato gli zeri della funzione: 
  $4 t^2 - 2 (\sqrt{3} - 1) t - \sqrt{3} \le 0$
  Ora dobbiamo trovare per quali valori di~$t$ questa funzione è $\le 0$.
  
  Essendo il grafico di questa funzione una parabola con la concavità verso 
  l'alto, il grafico sta sotto l'ascissa quando~t è compreso tra i due zeri:
  $f(t) \le 0 \text{ se } 
        t \in \left[\frac{1}{2}; \frac{\sqrt{3}}{2} \right]$ 
  
  E risostituendo la variabile~$t$ con~$\sin x$:
  $f(\sin x) \le 0 \text{ se } 
        \sin x \in \left[\frac{1}{2}; \frac{\sqrt{3}}{2} \right]$ 
        
%   Nel grafico della circonferenza goniometrica e della funzione seno 
  Evidenziamo questo intervallo sull'asse dei seni:
  
\begin{figure}[!h] 
 \vspace{-6pt}
  \begin{center}
\begin{inaccessibleblock}[Inizio della soluzione grafica della disequazione: 
    $4 \sin^2 x - 2 (\sqrt{3} + 2) \sin x + \sqrt{3} \le 0$.]
    % (c) 2014 Daniele Zambelli - daniele.zambelli@gmail.com

\begin{tikzpicture}[x=11mm,y=11mm, font=\small, cap=round, >=latex, smooth]


% (c) 2014 Daniele Zambelli - daniele.zambelli@gmail.com

% \begin{tikzpicture}[x=17mm,y=17mm, font=\small, cap=round, >=latex]

\colorlet{anglecolor}{green!50!black}
\colorlet{sincolor}{blue!50!black}
% \colorlet{coscolor}{red!50!black}
% \colorlet{tancolor}{orange!70!black}

\def \_deg2rad{0.01745329252}

\newcommand{\diseqsina}[2]{%
% Soluzione dell'equazione $\sin x = k$ 
% nella circonferenza goniometrica e nella sinusoide.
% Chiamata tipica: \eqsin{.5}{30}{150}

% Parametri
  \def \psina{#1}
  \def \psinb{#2}
% Assi
  \begin{scope}[-{Stealth[length=2mm, open, round]}]
    \draw (-2.1,0) -- (6.5, 0) node [below] {$x$}; % Asse x
    \draw (-1, -1.3) -- (-1, 1.3) node [left] {$y'$}; % Asse y'
    \draw (0, -1.3) -- (0, 1.3) node [left] {$\sin x$}; % Asse y
    \foreach \y in {-1, -.5, +.5, +1}{
      \draw [-] (-0.02, \y) -- (+0.05, \y);}
  \end{scope}
% Circonferenza
  \coordinate (a) at (-1, 0);
  \coordinate (b) at (0, 0);
  \node(c0) at (a)[draw, circle through=(b)] {};  
% Tacche con etichetta asse x
  \begin{scope}[font=\tiny]
  \foreach \x/\xtext in {
      0.0/0, 0.5236/30, 1.047/60, 1.571/90, 2.094/120, 2.618/150, 
      3.142/180, 3.665/210, 4.189/240, 4.712/270, 5.236/300, 5.76/330, 6.283/360}
  \node[below] at(\x, 0) {$\xtext \grado$};
  \end{scope}
% Tacche asse x
  \begin{scope}[font=\tiny]
  \foreach \x in { 
      0.0, 0.2618, 0.5236, 0.7854, 1.047, 1.309, 1.571, 1.833, 
      2.094, 2.356, 2.618, 2.88, 3.142, 3.403, 3.665, 3.927, 
      4.189, 4.451, 4.712, 4.974, 5.236, 5.498, 5.76, 6.021, 6.283}
  {\draw [black] (\x, -0.02) -- (\x, +0.05) node (a) {};}
  \end{scope}
%Sinusoide
  \tkzInit[xmin=0,xmax=+6.5,ymin=-1.1,ymax=+1.1]
  \tkzFct[domain=0:+6.5, thick]{sin(x)}
% Segmenti
  \begin{scope}[sincolor, thick]
  \foreach \x in {-1, 0} 
    {\draw [decorate, decoration=snake] (\x, \psina) -- (\x, \psinb);}
  \foreach \x in {-1, 0} 
    {\draw (\x, \psina) [fill] circle(2pt) {}; 
     \draw (\x, \psinb) [fill] circle(2pt) {};}
  \foreach \y in {\psina, \psinb} 
    {\draw (-2.1, \y) coordinate (pt0) -- (+6.5, \y) coordinate (pt1);}
  \end{scope}

}


\diseqsina{.5}{1.73205080757 / 2}


\end{tikzpicture}
    \caption{Inizio della soluzione grafica di: 
             $4 \sin^2 x - 2 (\sqrt{3} + 2) \sin x + \sqrt{3} \le 0$}
    \label{fig:trigo_disequ01a}
\end{inaccessibleblock}
  \end{center}
\vspace{-18pt}
\end{figure} 

  Ora dobbiamo trovare i valori di~x che danno un~$\sin x$ interno 
  all'intervallo evidenziato.
  
  La soluzione di~$\sin x = \frac{1}{2}$ \quad
  è~\quad $x_0 = 30 \grado \quad \vee \quad x_1 = 150 \grado$
  
  La soluzione di~$\sin x = -\frac{\sqrt{3}}{2}$ \quad
  è~\quad $x_0 = 60 \grado \quad \vee \quad x_1 = 120 \grado$
  
\begin{figure}[!h] 
 \vspace{-6pt}
  \begin{center}
\begin{inaccessibleblock}[Soluzione grafica della disequazione: 
    $4 \sin^2 x - 2 (\sqrt{3} + 2) \sin x + \sqrt{3} \le 0$.]
    % (c) 2014 Daniele Zambelli - daniele.zambelli@gmail.com

\begin{tikzpicture}[x=10mm,y=10mm, font=\small, cap=round, >=latex, smooth]
% (c) 2014 Daniele Zambelli - daniele.zambelli@gmail.com

% \begin{tikzpicture}[x=17mm,y=17mm, font=\small, cap=round, >=latex]

\colorlet{anglecolor}{green!30!black}
\colorlet{sincolor}{blue!50!black}
% \colorlet{coscolor}{red!50!black}
% \colorlet{tancolor}{orange!70!black}

\def \_deg2rad{0.01745329252}

\newcommand{\diseqsin}[6]{%
% Soluzione della disequazione $\sin x \in [a;~b]$ 
% nella circonferenza goniometrica e nella sinusoide.
% Chiamata tipica: \diseqsin{-.5}{1.73205080757 / 2}{0}{60}{120}{210}{330}{360}

% Parametri
  \def \psina{#1}
  \def \psinb{#2}
  \def \padeg{#3}
  \def \pbdeg{#4}
  \def \pcdeg{#5}
  \def \pddeg{#6}
  \def \parad{#3*\_deg2rad}
  \def \pbrad{#4*\_deg2rad}
  \def \pcrad{#5*\_deg2rad}
  \def \pdrad{#6*\_deg2rad}
% Assi
  \begin{scope}[-{Stealth[length=2mm, open, round]}]
    \draw (-2.1,0) -- (6.5, 0) node [below] {$x$}; % Asse x
    \draw (-1, -1.3) -- (-1, 1.3) node [left] {$y'$}; % Asse y'
    \draw (0, -1.3) -- (0, 1.3) node [left] {$\sin x$}; % Asse y
    \foreach \y in {-1, -.5, +.5, +1}{
      \draw [-] (-0.02, \y) -- (+0.05, \y);}
  \end{scope}
% Circonferenza
  \coordinate (a) at (-1, 0);
  \coordinate (b) at (0, 0);
  \node(c0) at (a)[draw, circle through=(b)] {};  
% Tacche con etichetta asse x
  \begin{scope}[font=\tiny]
  \foreach \x/\xtext in {
      0.0/0, 0.5236/30, 1.047/60, 1.571/90, 2.094/120, 2.618/150, 
      3.142/180, 3.665/210, 4.189/240, 4.712/270, 5.236/300, 5.76/330, 6.283/360}
  \node[below] at(\x, 0) {$\xtext \grado$};
  \end{scope}
% Tacche asse x
  \begin{scope}[font=\tiny]
  \foreach \x in { 
      0.0, 0.2618, 0.5236, 0.7854, 1.047, 1.309, 1.571, 1.833, 
      2.094, 2.356, 2.618, 2.88, 3.142, 3.403, 3.665, 3.927, 
      4.189, 4.451, 4.712, 4.974, 5.236, 5.498, 5.76, 6.021, 6.283}
  {\draw [black] (\x, -0.02) -- (\x, +0.05) node (a) {};}
  \end{scope}
%Sinusoide
  \tkzInit[xmin=0,xmax=+6.5,ymin=-1.1,ymax=+1.1]
  \tkzFct[domain=0:+6.5, thick]{sin(x)}
% Tratto di sinusoide evidenziata PECCATO CHE NON FUNZIONI!!!
%   \tkzFct[domain=\parad:\pbrad, thick, decoration=snake]{sin(x)}
% Segmenti
  \begin{scope}[sincolor, thick]
  \foreach \x in {-1, 0} 
    {\draw [decorate, decoration=snake] (\x, \psina) -- (\x, \psinb);}
  \foreach \x in {-1, 0} 
    {\draw (\x, \psina) [fill] circle(2pt) {}; 
     \draw (\x, \psinb) [fill] circle(2pt) {};}
  \draw (-2.1, \psina) coordinate (pt0a) -- (+6.5, \psina) coordinate (pt1a);
  \draw (-2.1, \psinb) coordinate (pt0b) -- (+6.5, \psinb) coordinate (pt1b);
  \end{scope}
% Intersezioni
  \coordinate (ca) at (intersection 1 of c0 and pt0a--pt1a);
  \coordinate (cb) at (intersection 1 of c0 and pt1a--pt0a);
  \coordinate (cc) at (intersection 1 of c0 and pt0b--pt1b);
  \coordinate (cd) at (intersection 1 of c0 and pt1b--pt0b);
  \begin{scope}[anglecolor]
  \foreach \p in {(ca), (cb), (cc), (cd)}
    {\draw (-1, 0) -- \p [fill] circle(1.5pt);}
% Soluzioni
    \draw (\parad, 0) [fill] circle(1.5pt) -- 
          (\parad, \psina) [fill] circle(1.5pt);
    \draw (\pbrad, 0) [fill] circle(1.5pt) -- 
          (\pbrad, \psinb) [fill] circle(1.5pt);
    \draw (\pcrad, 0) [fill] circle(1.5pt) -- 
          (\pcrad, \psinb) [fill] circle(1.5pt);
    \draw (\pdrad, 0) [fill] circle(1.5pt) -- 
          (\pdrad, \psina) [fill] circle(1.5pt);
    \draw [decorate, decoration=snake] (\parad, 0) -- (\pbrad, 0);
    \draw [decorate, decoration=snake] (\pcrad, 0) -- (\pdrad, 0);
    
    \draw (\parad, 0) [fill] circle(2pt); 
    \draw (\pbrad, 0) [fill] circle(2pt);
    \draw (\pcrad, 0) [fill] circle(2pt); 
    \draw (\pdrad, 0) [fill] circle(2pt);
% Angoli
    \draw [decorate, decoration=snake] (ca) arc(\padeg:\pbdeg:1);
    \draw [decorate, decoration=snake] (cd) arc(\pcdeg:\pddeg:1);
  \end{scope}

}

\diseqsin{.5}{1.73205080757 / 2}{30}{60}{120}{150}
\end{tikzpicture}

    \caption{Soluzione grafica di: 
             $4 \sin^2 x - 2 (\sqrt{3} + 2) \sin x + \sqrt{3} \le 0$}
    \label{fig:trigo_disequ01}
\end{inaccessibleblock}
  \end{center}
\vspace{-18pt}
\end{figure} 

Dopo averla trovata graficamente, possiamo riscrivere la soluzione della 
disequazione usando i predicati:\quad 
$30\grado \le x \le 60\grado \quad \vee \quad 
 120\grado \le x \le 150\grado$

e le parentesi:\quad 
$\left[30\grado ; 60\grado \right] \quad \cup \quad
 \left[120\grado ; 150\grado \right]$
  
 \end{esempio}
