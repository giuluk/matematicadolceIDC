% (c) 2015 Daniele Zambelli daniele.zambelli@gmail.com

\begin{comment}
 Cap10: es10.3 e 10.4 sono uguali, uno va tolto. Vanno anticipati il 10.6 e 
10.7 prima del 10.3 e 10.5 perché in questi ultimi ci sono anche angoli 
associati. Mancherebbero esercizi sui grafici delle funzioni goniometriche 
e 
quindi equazioni e disequazioni per via grafica ( senx=3-X, .. disegnando 
rette e parabole.)
\end{comment}


\section{Esercizi}

\subsection{Esercizi dei singoli paragrafi}

\subsubsection*{\numnameref{sec:gonio_angoli_archi}}

\begin{esercizio}\label{ese:gonio.1}
 Converti i seguenti angoli da gradi a radianti:
\begin{multicols}{3}
 \begin{enumeratea}
  \item  \(180\grado\)
   \hfill \(\quadra{1\pi}\)
  \item  \(-40\grado\)
   \hfill \(\quadra{-\dfrac{2}{9}\pi}\)
  \item  \(-60\grado\)
   \hfill \(\quadra{-\dfrac{1}{3}\pi}\)
  \item  \(270\grado\)
   \hfill \(\quadra{\dfrac{3}{2}\pi}\)
  \item  \(36\grado\)
   \hfill \(\quadra{\dfrac{1}{5}\pi}\)
  \item  \(-108\grado\)
   \hfill \(\quadra{-\dfrac{3}{5}\pi}\)
  \item  \(-270\grado\)
   \hfill \(\quadra{-\dfrac{3}{2}\pi}\)
  \item  \(0\grado\)
   \hfill \(\quadra{0\pi}\)
  \item  \(60\grado\)
   \hfill \(\quadra{\dfrac{1}{3}\pi}\)
  \item  \(252\grado\)
   \hfill \(\quadra{\dfrac{7}{5}\pi}\)
  \item  \(315\grado\)
   \hfill \(\quadra{\dfrac{7}{4}\pi}\)
  \item  \(72\grado\)
   \hfill \(\quadra{\dfrac{2}{5}\pi}\)
  \item  \(315\grado\)
   \hfill \(\quadra{\dfrac{7}{4}\pi}\)
  \item  \(-160\grado\)
   \hfill \(\quadra{-\dfrac{8}{9}\pi}\)
  \item  \(-180\grado\)
   \hfill \(\quadra{-1\pi}\)
  \item  \(-324\grado\)
   \hfill \(\quadra{-\dfrac{9}{5}\pi}\)
  \item  \(-315\grado\)
   \hfill \(\quadra{-\dfrac{7}{4}\pi}\)
  \item  \(-45\grado\)
   \hfill \(\quadra{-\dfrac{1}{4}\pi}\)
%   \item  \(340\grado\)
%    \hfill \(\quadra{\dfrac{17}{9}\pi}\)
%   \item  \(144\grado\)
%    \hfill \(\quadra{\dfrac{4}{5}\pi}\)
%   \item  \(320.0\grado\)
%    \hfill \(\quadra{\dfrac{16}{9}\pi}\)
 \end{enumeratea}
\end{multicols}
\end{esercizio}

\begin{esercizio}\label{ese:gonio.2}
 Converti i seguenti angoli da radianti a gradi:
\begin{multicols}{3}
 \begin{enumeratea}
  \item  \(1 \pi\)
   \hfill \(\quadra{180 \grado}\)
  \item  \(-\dfrac{3}{2} \pi\)
   \hfill \(\quadra{-270 \grado}\)
  \item  \(-\dfrac{4}{3} \pi\)
   \hfill \(\quadra{-240 \grado}\)
  \item  \(-\dfrac{1}{5} \pi\)
   \hfill \(\quadra{-36 \grado}\)
  \item  \(\dfrac{5}{3} \pi\)
   \hfill \(\quadra{300 \grado}\)
  \item  \(-\dfrac{3}{4} \pi\)
   \hfill \(\quadra{-135 \grado}\)
  \item  \(\dfrac{7}{5} \pi\)
   \hfill \(\quadra{252 \grado}\)
  \item  \(0 \pi\)
   \hfill \(\quadra{0 \grado}\)
  \item  \(-1 \pi\)
   \hfill \(\quadra{-180 \grado}\)
  \item  \(\dfrac{3}{2} \pi\)
   \hfill \(\quadra{270 \grado}\)
  \item  \(\dfrac{6}{5} \pi\)
   \hfill \(\quadra{216 \grado}\)
  \item  \(-\dfrac{1}{3} \pi\)
   \hfill \(\quadra{-60 \grado}\)
  \item  \(\dfrac{6}{5} \pi\)
   \hfill \(\quadra{216 \grado}\)
  \item  \(\dfrac{2}{3} \pi\)
   \hfill \(\quadra{120 \grado}\)
%   \item  \(-\dfrac{1}{6} \pi\)
%    \hfill \(\quadra{-30 \grado}\)
  \item  \(2 \pi\)
   \hfill \(\quadra{0 \grado}\)
  \item  \(-\dfrac{8}{5} \pi\)
   \hfill \(\quadra{-288 \grado}\)
  \item  \(\dfrac{4}{5} \pi\)
   \hfill \(\quadra{144 \grado}\)
  \item  \(\dfrac{3}{4} \pi\)
   \hfill \(\quadra{135 \grado}\)
%   \item  \(-\dfrac{1}{2} \pi\)
%    \hfill \(\quadra{-90 \grado}\)
%   \item  \(-\dfrac{11}{9} \pi\)
%    \hfill \(\quadra{-220 \grado}\)
%   \item  \(-\dfrac{4}{9} \pi\)
%    \hfill \(\quadra{-80 \grado}\)
%   \item  \(\dfrac{4}{3} \pi\)
%    \hfill \(\quadra{240 \grado}\)
 \end{enumeratea}
\end{multicols}
\end{esercizio}

\subsubsection*{\numnameref{sec:gonio_circonferenza_goniometrica}}

\begin{esercizio}\label{ese:}
 Usando la calcolatrice, calcola il valore delle seguenti espressioni:
 \begin{enumeratea}
  \item  \(\sin 150 \grado+ \sin 270 \grado+ \cos 180 \grado+ 
          \cos 240 \grado+ \sin 0 \grado+ \cos 60 \grado= \)
   \hfill \(\quadra{-\dfrac{3}{2}}\)
  \item  \(\sin 330 \grado+ \cos 270 \grado+ \cos 300 \grado+ 
          \cos 270 \grado+ \sin 90 \grado+ \cos 240 \grado= \)
   \hfill \(\quadra{\dfrac{1}{2}}\)
  \item  \(\cos 90 \grado + \cos 0 \grado + \sin 90 \grado + 
          \sin 270 \grado + \sin 210 \grado + \sin 180 \grado = \)
   \hfill \(\quadra{\dfrac{1}{2}}\)
  \item  \(\cos 120 \grado + \sin 270 \grado + \sin 90 \grado + 
          \sin 180 \grado + \sin 270 \grado + \cos 0 \grado = \)
   \hfill \(\quadra{-\dfrac{1}{2}}\)
  \item  \(\sin 270 \grado + \sin 180 \grado + \sin 210 \grado + 
          \cos 0 \grado + \cos 120 \grado + \sin 90 \grado = \)
   \hfill \(\quadra{0}\)
  \item  \(\sin 90 \grado + \cos 300 \grado + \sin 150 \grado + 
          \sin 270 \grado + \sin 90 \grado + \sin 0 \grado = \)
   \hfill \(\quadra{2}\)
  \item  \(\cos 300 \grado + \sin 210 \grado + \cos 180 \grado + 
          \cos 60 \grado + \cos 180 \grado + \sin 150 \grado = \)
   \hfill \(\quadra{-1}\)
  \item  \(\cos 120 \grado + \cos 60 \grado + \cos 270 \grado + 
          \sin 180 \grado + \cos 300 \grado + \cos 90 \grado = \)
   \hfill \(\quadra{\dfrac{1}{2}}\)
%   \item  \(\sin 210 \grado + \sin 330 \grado + \sin 180 \grado + 
%           \sin 180 \grado + \sin 330 \grado + \sin 150 \grado = \)
%    \hfill \(\quadra{-1}\)
%   \item  \(\sin 150 \grado + \sin 270 \grado + \cos 270 \grado + 
%           \cos 240 \grado + \cos 90 \grado + \cos 180 \grado = \)
%    \hfill \(\quadra{-2}\)
%   \item  \(\sin 90 \grado + \cos 0 \grado + \sin 210 \grado + 
%           \sin 210 \grado + \cos 0 \grado + \cos 240 \grado = \)
%    \hfill \(\quadra{\dfrac{3}{2}}\)
%   \item  \(\sin 270 \grado + \sin 330 \grado + \cos 120 \grado + 
%           \sin 210 \grado + \sin 330 \grado + \cos 0 \grado = \)
%    \hfill \(\quadra{-2}\)
%   \item  \(\sin 90 \grado + \cos 0 \grado + \sin 90 \grado + 
%           \cos 0 \grado + \sin 180 \grado + \cos 240 \grado = \)
%    \hfill \(\quadra{\dfrac{7}{2}}\)
%   \item  \(\sin 180 \grado + \sin 90 \grado + \cos 240 \grado + 
%           \cos 120 \grado + \sin 210 \grado + \sin 330 \grado = \)
%    \hfill \(\quadra{-1}\)
%   \item  \(\cos 240 \grado + \cos 90 \grado + \sin 150 \grado + 
%           \cos 90 \grado + \cos 180 \grado + \cos 180 \grado = \)
%    \hfill \(\quadra{-2}\)
%   \item  \(\sin 90 \grado + \cos 180 \grado + \sin 90 \grado + 
%           \sin 330 \grado + \cos 180 \grado + \sin 270 \grado = \)
%    \hfill \(\quadra{-\dfrac{3}{2}}\)
%   \item  \(\sin 270 \grado + \cos 300 \grado + \sin 90 \grado + 
%           \sin 270 \grado + \cos 180 \grado + \sin 90 \grado = \)
%    \hfill \(\quadra{-\dfrac{1}{2}}\)
%   \item  \(\sin 150 \grado + \sin 0 \grado + \sin 270 \grado + 
%           \cos 60 \grado + \sin 330 \grado + \sin 90 \grado = \)
%    \hfill \(\quadra{\dfrac{1}{2}}\)
%   \item  \(\sin 270 \grado + \sin 270 \grado + \sin 180 \grado + 
%           \sin 30 \grado + \cos 60 \grado + \sin 90 \grado = \)
%    \hfill \(\quadra{0}\)
%   \item  \(\cos 180 \grado + \cos 180 \grado + \sin 30 \grado + 
%           \cos 0 \grado + \cos 180 \grado + \cos 300 \grado = \)
%    \hfill \(\quadra{-1}\)
%   \item  \(\sin 210 \grado + \cos 300 \grado + \sin 330 \grado + 
%           \sin 90 \grado + \cos 60 \grado + \cos 300 \grado = \)
%    \hfill \(\quadra{\dfrac{3}{2}}\)
%   \item  \(\sin 210 \grado + \cos 180 \grado + \cos 180 \grado + 
%           \cos 240 \grado + \sin 30 \grado + \cos 180 \grado = \)
%    \hfill \(\quadra{-\dfrac{7}{2}}\)
%   \item  \(\sin 330 \grado + \cos 60 \grado + \sin 0 \grado + 
%           \cos 300 \grado + \cos 180 \grado + \sin 270 \grado = \)
%    \hfill \(\quadra{-\dfrac{3}{2}}\)
%   \item  \(\sin 30 \grado + \sin 180 \grado + \cos 270 \grado + 
%           \cos 0 \grado + \sin 180 \grado + \cos 0 \grado = \)
%    \hfill \(\quadra{\dfrac{5}{2}}\)
%   \item  \(\sin 150 \grado + \cos 0 \grado + \sin 150 \grado + 
%           \sin 0 \grado + \cos 0 \grado + \cos 90 \grado = \)
%    \hfill \(\quadra{3}\)
%   \item  \(\sin 180 \grado + \sin 330 \grado + \sin 270 \grado + 
%           \sin 270 \grado + \cos 270 \grado + \cos 270 \grado \)
%    \hfill \(\quadra{-\dfrac{5}{2}}\)
%   \item  \(\cos 60 \grado + \sin 180 \grado + \cos 90 \grado + 
%           \sin 90 \grado + \cos 180 \grado + \sin 0 \grado = \)
%    \hfill \(\quadra{\dfrac{1}{2}}\)
%   \item  \(\cos 0 \grado + \cos 60 \grado + \sin 270 \grado + 
%           \cos 0 \grado + \cos 90 \grado + \cos 0 \grado = \)
%    \hfill \(\quadra{\dfrac{5}{2}}\)
%   \item  \(\sin 270 \grado + \cos 300 \grado + \sin 270 \grado + 
%           \cos 0 \grado + \sin 90 \grado + \sin 180 \grado = \)
%    \hfill \(\quadra{\dfrac{1}{2}}\)
%   \item  \(\cos 0 \grado + \sin 270 \grado + \sin 270 \grado + 
%           \cos 60 \grado + \cos 180 \grado + \sin 270 \grado = \)
%    \hfill \(\quadra{-\dfrac{5}{2}}\)
 \end{enumeratea}
\end{esercizio}


\subsubsection*{\numnameref{sec:gonio_angoli_associati}}

\begin{esercizio}\label{ese:03.1}
Riduci al primo quadrante gli angoli delle seguenti funzioni:
\begin{multicols}{4}
 \begin{enumeratea}
  \item \(\sin 120 \grado\)
   \hfill
  \item \(\tan 290 \grado\)
   \hfill
  \item \(\cos 200 \grado\) 
   \hfill
  \item \(\sin 112 \grado\) 
   \hfill
  \item \(\cos 232 \grado\) 
   \hfill
  \item \(\sin 315 \grado\) 
   \hfill
  \item \(\tan 165 \grado\) 
   \hfill
  \item \(\sin 260 \grado\) 
   \hfill
  \item \(\tan \dfrac{4}{3}\pi\) 
   \hfill
  \item \(\sin \dfrac{5}{3}\pi\) 
   \hfill
  \item \(\cos \dfrac{10}{6}\pi\) 
   \hfill
  \item \(\tan \dfrac{5}{4}\pi\) 
   \hfill
%   \item \(\cos \dfrac{11}{3}\pi\) 
%    \hfill
%   \item \(\sin \dfrac{7}{4}\pi\) 
%    \hfill
 \end{enumeratea}
 \end{multicols}
\end{esercizio}

\subsubsection*{\numnameref{sec:gonio_angoli_particolari}}

\begin{esercizio}\label{ese:03.1}
Calcola il valore delle seguenti espressioni:
\begin{multicols}{2}
 \begin{enumeratea}
  \item  \(\dfrac{\sin 60 \grado -\sin 30 \grado}
                {\sin 60 \grado +\sin 30 \grado} \)
   \hfill \(\quadra{2-\sqrt{3}}\)
  \item  \(\sin 60 \grado \cos 30 \grado - \sin 30 \grado \cos 60 \grado \)
   \hfill \(\quadra{\dfrac{1}{2}}\)
  \item  \(2 \cos^2 \dfrac{\pi}{6} - 2 + 2 \sin^2 \dfrac{\pi}{6}\)
   \hfill \(\quadra{0}\)
  \item  \(\sin^2 \dfrac{\pi}{6} + \sin^2 \dfrac{\pi}{3} + 
          \tan^2 \dfrac{\pi}{4} - 4 \cos^2 \dfrac{\pi}{3}\)
   \hfill \(\quadra{0}\)
  \item  \(\cos^2 \dfrac{\pi}{3} + \sin^2 \dfrac{\pi}{6} - 
          \sin \dfrac{\pi}{4} \cos \dfrac{\pi}{4}\)
   \hfill \(\quadra{0}\)
  \item  \(\tan^2 \dfrac{\pi}{3} + 4 \cos^2 \dfrac{\pi}{4} - 
          \dfrac{3}{\sin^2 \dfrac{\pi}{3}}\)
   \hfill \(\quadra{9}\)
  \item  \(2 \sin \dfrac{\pi}{4} + \dfrac{1}{2 \sin^2 \dfrac{\pi}{4}}\)
   \hfill \(\quadra{\dfrac{3 \sqrt{2}}{2}}\)
  \item  \(\sin^2 \dfrac{\pi}{6} + 12 \cos^2 \dfrac{\pi}{4} - 
          \dfrac{4}{\cos \dfrac{\pi}{3}} + 2 \tan^2 \dfrac{\pi}{6}\)
   \hfill \(\quadra{0}\)
  \item  \(2 \sqrt{3} \sin \dfrac{\pi}{3} +4 \sin \dfrac{\pi}{6} - 
          \dfrac{1}{\cos \dfrac{\pi}{3}}\)
   \hfill \(\quadra{3}\)
  \item  \(\dfrac{2}{\sqrt{3}} \tan \dfrac{\pi}{3} -
          \dfrac{3}{4} \tan \dfrac{\pi}{4} + 
          \dfrac{1}{\cos \pi} \)
   \hfill \(\quadra{\dfrac{1}{4}}\)
 \end{enumeratea}
 \end{multicols}
\end{esercizio}

\begin{esercizio}\label{ese:03.1}
Calcola il valore delle seguenti espressioni:
 \begin{enumeratea}
  \item  \(4 \sin 60 + 2 \cos 30 + \dfrac{2}{\tan 45} - 
          \tan 60 - \dfrac{2}{\tan 30}\)
   \hfill \(\quadra{2}\)
  \item  \(\dfrac{2}{\sqrt{3}} \tan 60 - \dfrac{3}{4} \tan 45 + 
          \dfrac{1}{\cos 540} - \sin^2 30 \)
   \hfill \(\quadra{0}\)
  \item  \(\tonda{1 + \sin \dfrac{\pi}{6} + \cos \dfrac{\pi}{6}}^2 -
          2 \tonda{1 + \sin \dfrac{\pi}{6}} \tonda{1 + \cos \dfrac{13}{6}\pi}\)
   \hfill \(\quadra{0}\)
  \item  \(\sin \dfrac{\pi}{4} + \cos \dfrac{9}{4}\pi - 
          \tonda{1 + \tan \dfrac{5}{4}\pi} \cos \dfrac{\pi}{4}\)
   \hfill \(\quadra{0}\)
  \item  \(4 \sin 750 + 3 \cos 30 - \dfrac{1}{2} \tan 60 - 2 \tan 45\)
   \hfill \(\quadra{\sqrt{3}}\)
%   \item  \(\dfrac{\sqrt{3} \sin \dfrac{\pi}{3} - \sin \dfrac{\pi}{6}}
%                 {\sin \dfrac{\pi}{4} - 
%                  \tonda{1 + \sqrt{}2} \sin \dfrac{\pi}{6} +
%                  \sin \dfrac{\pi}{2}}\)
%    \hfill \(\quadra{2}\)
%   \item  \(\dfrac{\sin^2 \dfrac{\pi}{6} + \sin^2 \dfrac{\pi}{4} +
%                  \sin^2 \dfrac{\pi}{3}}
%                 {8 \dfrac{6 - 2 \sqrt{5}}{16} + \sqrt{5}}\)
%    \hfill \(\quadra{\dfrac{1}{2}}\)
 \end{enumeratea}
\end{esercizio}

\begin{esercizio}\label{ese:}
 Calcola il valore delle seguenti espressioni:
 \begin{enumeratea}
  \item  \(\dfrac{\sin 60 - \sin 30}{\sin 60 + \sin 30}\)
   \hfill \(\quadra{2 - \sqrt{3}}\)
  \item  \(\sin 60 - \cos 30 - \sin 30 - \cos 60\)
   \hfill \(\quadra{\dfrac{1}{2}}\)
  \item  \(2 \cos^2 \dfrac{\pi}{6} - 2 + 2 \sin^2 \dfrac{\pi}{6}\)
   \hfill \(\quadra{0}\)
  \item  \(\sin^2 \dfrac{\pi}{6} + \sin^2 \dfrac{\pi}{3} +
          \tan^2 \dfrac{\pi}{4} - 4 \cos \dfrac{\pi}{3}\)
   \hfill \(\quadra{0}\)
  \item  \(\tan^2 \dfrac{\pi}{3} + 4 \cos^2 \dfrac{\pi}{4} + 
          \dfrac{3}{\sin \dfrac{\pi}{3}}\)
   \hfill \(\quadra{}\)
 \end{enumeratea}
\end{esercizio}

\subsubsection*{\numnameref{sec:gonio_formule}}

\begin{esercizio}\label{ese:03.1}
Verifica le seguenti identità:
 \begin{enumeratea}
  \item \(\sin \beta \cos \tonda{\alpha - \beta} +
         \cos \beta \sin \tonda{\alpha - \beta} = \sin \alpha\)
  \item \(\tan \tonda{\alpha + \beta} + \tan \tonda{\alpha - \beta} =
         \dfrac{2 \tan \alpha \frac{1}{\cos \beta}}
               {1- \tan^2 \alpha \tan^2 \beta}\)
  \item \(\sin \tonda{\alpha - \beta} \cos \gamma +
         \sin \tonda{\beta - \gamma} \cos \alpha +
         \sin \tonda{\gamma - \alpha} \cos \beta = 0\)
  \item \(\sin^2 \alpha  - \sin^2 \beta = 
         \sin \tonda{\alpha + \beta} \sin \tonda{\alpha - \beta}\)
  \item \(\cos^2 \alpha  - \cos^2 \beta = 
         - \sin \tonda{\alpha + \beta} \sin \tonda{\alpha - \beta}\)
  \item \(\cos \alpha + \sin \alpha = 
  \sqrt{2} \cos \tonda{\dfrac{\pi}{4} - \alpha} = 
  \sqrt{2} \sin \tonda{\dfrac{\pi}{4} + \alpha}\)
  \item \(\tonda{\cos \alpha + \sin \alpha} 
         \cos \tonda{\dfrac{\pi}{4} + \alpha} =
         \dfrac{\sqrt{2}}{2} \tonda{\cos^ \alpha - \sin^ \alpha}\)
  \item \(\tan \alpha + \tan |beta =
         \dfrac{cos \tonda{\alpha + \beta}}{\cos \alpha \cos \beta}\)
%   \item \(\sin^2 \alpha - \sin^2 \beta =
%          \sin \tonda{\alpha + \beta} \sin \tonda{\alpha - \beta}\)
%   \item \(\cos^2 \alpha - \cos^2 \beta =
%          - \sin \tonda{\alpha + \beta} \sin \tonda{\alpha - \beta}\)
 \end{enumeratea}
\end{esercizio}

\subsubsection*{\numnameref{sec:gonio_equazionigonio}}

\begin{esercizio}\label{ese:03.1}
Verifica le seguenti identità:
 \begin{enumeratea}
  \item \(\dfrac{1}{\sin^2 \alpha \cos ^2 \alpha} = 
         \dfrac{1}{\cos ^2 \alpha} + \dfrac{1}{\sin^2 \alpha}\)
  \item \(\sin^4 \alpha + \cos ^4 \alpha = 1 - 2 \sin^2 \alpha \cos ^2 \alpha\)
  \item \(\tonda{1 + \sin \alpha + \cos \alpha}^2 =
         2 \tonda{1 + \cos \alpha} \tonda{1 + \sin \alpha}\)
  \item \(\dfrac{1}{\sin^2 \alpha \cos ^2 \alpha} = 
         \tan^2 \alpha + \dfrac{\cos^2 \alpha}{\sin^2 \alpha} + 2\)
  \item \(\dfrac{1}{1 - \sin \alpha} + \dfrac{1}{1 + \sin \alpha} =
         \dfrac{2}{\cos^2 \alpha}\)
  \item \(\dfrac{1}{1 + \sin^2 \alpha} + 
         \dfrac{\sin^2 \alpha}{1 + \sin^2 \alpha} = 1\)
  \item \(\tan \alpha =
         \dfrac{\sin^3 \alpha}{\cos \alpha - \cos^3 \alpha}\)
  \item \(\tonda{-\sin \alpha - \cos \alpha}^2 +
         \tonda{\sin \alpha - \cos \alpha}^2 = 2\)
%   \item \(\dfrac{2 \tonda{\tan \alpha + \sin \alpha}}
%                {1 + \cos \alpha} = 2 \tan \alpha\)
%   \item \(\dfrac{\sin^2 \alpha}{1 + \cos \alpha} = 1 - \cos \alpha\)
 \end{enumeratea}
\end{esercizio}

\begin{esercizio}\label{ese:03.1}
Risolvi le seguenti equazioni:
 \begin{enumeratea}
  \item \(\sin x = 0\)
   \hfill \(\quadra{x = k \pi}\)
  \item \(\cos x = 0\)
   \hfill \(\quadra{x = \dfrac{\pi}{2} + k \pi}\)
  \item \(\sin x = 1\)
   \hfill \(\quadra{x = \dfrac{\pi}{2} + k \pi}\)
  \item \(\cos x = 1\)
   \hfill \(\quadra{x = 2 k \pi}\)
  \item \(\tan x = 0\)
   \hfill \(\quadra{x = 0 + k \pi}\)
  \item \(\sin x = \dfrac{1}{2}\)
   \hfill \(\quadra{x = \dfrac{\pi}{6} + 2k \pi; \quad
                   x = \dfrac{5}{6}\pi + 2k \pi}\)
  \item \(\cos x = \dfrac{1}{2}\)
   \hfill \(\quadra{x = \dfrac{\pi}{3} + 2k \pi; \quad
                   x = -\dfrac{\pi}{3} + 2k \pi}\)
  \item \(\sin x = \dfrac{\sqrt{2}}{2}\)
   \hfill \(\quadra{x = \dfrac{\pi}{4} + 2k \pi; \quad
                   x = \dfrac{3}{4}\pi + 2k \pi}\)
  \item \(\cos x = \dfrac{\sqrt{2}}{2}\)
   \hfill \(\quadra{x = \dfrac{\pi}{4} + 2k \pi; \quad
                   x = -\dfrac{\pi}{4} + 2k \pi}\)
%   \item \(\sin x = \dfrac{\sqrt{3}}{2}\)
%    \hfill \(\quadra{x = \dfrac{\pi}{3} + 2k \pi; \quad
%                    x = \dfrac{2}{3}\pi + 2k \pi}\)
%   \item \(\cos x = \dfrac{\sqrt{3}}{2}\)
%    \hfill \(\quadra{x = \dfrac{\pi}{6} + 2k \pi; \quad
%                    x = -\dfrac{\pi}{6} + 2k \pi}\)
%   \item \(\tan x = \dfrac{\sqrt{3}}{3}\)
%    \hfill \(\quadra{x = \dfrac{\pi}{6} + k \pi}\)
%   \item \(\tan x = 1\)
%    \hfill \(\quadra{x = \dfrac{\pi}{4} + k \pi}\)
%   \item \(\tan x = \sqrt{3}\)
%    \hfill \(\quadra{x = \dfrac{\pi}{3} + k \pi}\)
 \end{enumeratea}
\end{esercizio}

% come quello sopra ma con valori negativi
\begin{esercizio}\label{ese:03.1}
Risolvi le seguenti equazioni:
 \begin{enumeratea}
  \item \(\sin x = -1\)
   \hfill \(\quadra{x = \dfrac{3}{2}\pi + k \pi}\)
  \item \(\cos x = -1\)
   \hfill \(\quadra{x = \pi + 2k \pi}\)
  \item \(\sin x = -\dfrac{1}{2}\)
   \hfill \(\quadra{x = \dfrac{7}{6}\pi + 2k \pi; \quad
                   x = \dfrac{11}{6}\pi + 2k \pi}\)
  \item \(\cos x = -\dfrac{1}{2}\)
   \hfill \(\quadra{x = \mp \dfrac{2}{3}\pi + 2k \pi}\)
  \item \(\sin x = -\dfrac{\sqrt{2}}{2}\)
   \hfill \(\quadra{x = \dfrac{5}{4}\pi + 2k \pi; \quad
                   x = \dfrac{7}{4}\pi + 2k \pi}\)
  \item \(\cos x = -\dfrac{\sqrt{2}}{2}\)
   \hfill \(\quadra{x = \mp \dfrac{3}{4}\pi + 2k \pi}\)
  \item \(\sin x = -\dfrac{\sqrt{3}}{2}\)
   \hfill \(\quadra{x = \dfrac{4}{3}\pi + 2k \pi; \quad
                   x = \dfrac{5}{3}\pi + 2k \pi}\)
  \item \(\cos x = -\dfrac{\sqrt{3}}{2}\)
   \hfill \(\quadra{x = \mp \dfrac{5}{6}\pi + 2k \pi}\)
  \item \(\tan x = -\dfrac{\sqrt{3}}{3}\)
   \hfill \(\quadra{x = -\dfrac{\pi}{6} + k \pi}\)
%   \item \(\tan x = -1\)
%    \hfill \(\quadra{x = -\dfrac{\pi}{4} + k \pi}\)
%   \item \(\tan x = -\sqrt{3}\)
%    \hfill \(\quadra{x = -\dfrac{\pi}{3} + k \pi}\)
 \end{enumeratea}
\end{esercizio}

\begin{esercizio}\label{ese:03.1}
Risolvi le seguenti equazioni:
 \begin{enumeratea}
  \item \(\sin 5x = 0\)
   \hfill \(\quadra{x = \dfrac{k}{5} \pi}\)
  \item \(\cos \dfrac{1}{4} x = \dfrac{1}{2}\)
   \hfill \(\quadra{x = \dfrac{4 \pi}{3} + 8k \pi; \quad
                   x = -\dfrac{4\pi}{3} + 8k \pi}\)
  \item \(\cos 3x = -\dfrac{\sqrt{2}}{2}\)
   \hfill \(\quadra{x = \mp \dfrac{\pi}{4} + \dfrac{2k}{3} \pi}\)
  \item \(\tan 2x = \sqrt{3}\)
   \hfill \(\quadra{x = \dfrac{\pi}{6} + \dfrac{\pi}{2}}\)
  \item \(\sin \dfrac{1}{2} x = -1\)
   \hfill \(\quadra{x = 3\pi + 2k \pi}\)
  \item \(\cos \dfrac{1}{2} x = -1\)
   \hfill \(\quadra{x = 2 \pi + 4k \pi}\)
%   \item \(\sin \dfrac{2}{3} x = -\dfrac{1}{2}\)
%    \hfill \(\quadra{x = \dfrac{7}{4}\pi + 3k \pi; \quad
%                    x = \dfrac{11}{4}\pi + 3k \pi}\)
%   \item \(\sin \dfrac{5}{6} x = -\dfrac{\sqrt{3}}{2}\)
%    \hfill \(\quadra{x = \dfrac{8}{5}\pi + 2k \pi; \quad
%                    x = 2 \pi + \dfrac{12}{5} k \pi}\)
  \item \(\cos \dfrac{7}{8}x = 0\)
   \hfill \(\quadra{x = \dfrac{4\pi}{7} + \dfrac{8}{7}k \pi}\)
  \item \(\cos \dfrac{6}{5}x = -\dfrac{\sqrt{3}}{2}\)
   \hfill \(\quadra{x = \mp \pi + \dfrac{5}{3} k \pi}\)
%   \item \(\cos 5x = -\dfrac{1}{2}\)
%    \hfill \(\quadra{x = \mp\dfrac{2}{15}\pi + \dfrac{2}{5} k \pi}\)
 \end{enumeratea}
\end{esercizio}

\begin{esercizio}\label{ese:03.1}
Risolvi le seguenti equazioni:
 \begin{enumeratea}
  \item \(\cos \tonda{2x + \dfrac{\pi}{6}} = -\dfrac{1}{2}\)
   \hfill \(\quadra{x = \dfrac{1}{4}\pi + k \pi; \quad
                   x = -\dfrac{5}{12}\pi + k \pi}\)
  \item \(\sin \tonda{3 x - \dfrac{3}{4} \pi}= \dfrac{\sqrt{2}}{2}\)
   \hfill \(\quadra{x = \dfrac{\pi}{3} + \dfrac{2}{3} k \pi; \quad
                   x = \dfrac{2}{6}\pi + \dfrac{2}{3} k \pi}\)
  \item \(\tan \tonda{\dfrac{2}{3} x + \dfrac{\pi}{2}} = -\dfrac{\sqrt{3}}{3}\)
   \hfill \(\quadra{x = \dfrac{\pi}{2} + \dfrac{3}{2} k \pi}\)
  \item \(\sin \tonda{2 x - \dfrac{\pi}{3}} = \sin \tonda{x + \dfrac{2}{3}\pi}\)
   \hfill \(\quadra{x = \tonda{2k +1} \pi; \quad
                   x = \dfrac{2}{9}\pi + \dfrac{2}{3} k \pi}\)
  \item \(\sin \tonda{3 x - \dfrac{\pi}{5}} = - \sin \tonda{x + \dfrac{\pi}{4}}\)
   \hfill \(\quadra{x = \dfrac{\pi}{80} + k \dfrac{\pi}{2}; \quad
                   x = \dfrac{29}{40}\pi + k \pi}\)
  \item \(\cos \tonda{-x + \dfrac{3}{4}\pi} = \cos \tonda{2 x - \dfrac{\pi}{5}}\)
   \hfill \(\quadra{x = \dfrac{19}{60}\pi - \dfrac{2}{3} k \pi; \quad
                   x = - \dfrac{11}{20}\pi + 2 k \pi}\)
  \item \(\sin x = \sin \tonda{x - 60 \grado}\)
   \hfill \(\quadra{x = 120 \grado + k 180 \grado}\)
%   \item \(\cos \tonda{2x - 50 \grado} = \cos \tonda{10 \grado - x}\)
%    \hfill \(\quadra{x = 20 \grado + k 120 \grado; \quad
%                    x = 40 \grado + k 360 \grado}\)
%   \item \(\cos \tonda{3x - 15 \grado} = - \cos \tonda{2 x - 5 \grado}\)
%    \hfill \(\quadra{x = 40 \grado + k 72 \grado; \quad
%                    x = -170 \grado + k 360 \grado}\)
%   \item \(\sin 7 x = \cos 5 x\)
%    \hfill \(\quadra{x = \dfrac{\pi}{24} + k \dfrac{\pi}{6}; \quad
%                    x = \dfrac{\pi}{4} + k \pi}\)
%   \item \(\sin \tonda{2x - 30 \grado} = \cos \tonda{- x + 15 \grado}\)
%    \hfill \(\quadra{x = 105 \grado + k 360 \grado; \quad
%                    x = 45 \grado + k 120 \grado}\)
 \end{enumeratea}
\end{esercizio}

% \newpage %-----------------------------------------

\begin{esercizio}\label{ese:03.1}
Risolvi le seguenti equazioni:
 \begin{enumeratea}
  \item \(2 \cos^2 x - 5 \cos x + 2 = 0\)
   \hfill \(\quadra{x = \mp \dfrac{\pi}{3} + 2 k \pi}\)
  \item \(2 \sin^2 x - 7 \sin x + 3 = 0\)
   \hfill \(\quadra{x = \dfrac{\pi}{6} + 2 k \pi; \quad
                   x = \dfrac{5}{6} \pi + 2 k \pi}\)
  \item \(6 \sin^2 x - 13 \sin x + 5 = 0\)
   \hfill \(\quadra{x = \dfrac{\pi}{6} + 2 k \pi; \quad
                   x = \dfrac{5}{6} \pi + 2 k \pi}\)
  \item \(2 \cos^2 x - 3 \cos x + 1 = 0\)
   \hfill \(\quadra{x = 2 k \pi; \quad 
                   x = \mp \dfrac{\pi}{3} + 2 k \pi}\)
  \item \(3 \tonda{1 - \cos x} = \sin^2 x\)
   \hfill \(\quadra{x = 2 k \pi}\)
  \item \(\tan^2 x + \tonda{\sqrt{3} - 1} \tan x - \sqrt{3} = 0\)
   \hfill \(\quadra{x = \dfrac{\pi}{3} + k \pi; \quad
                   x = \dfrac{3}{4} \pi + k \pi}\)
  \item \(\tan^2 x - \tonda{1 + \sqrt{3}} \tan x + \sqrt{3} = 0\)
   \hfill \(\quadra{x = \dfrac{\pi}{3} + k \pi; \quad
                   x = \dfrac{\pi}{4} + k \pi}\)
  \item \(2 \sin^2 x - 5 \cos x -4 = 0\)
   \hfill \(\quadra{x = \mp \dfrac{2}{3} \pi + 2 k \pi}\)
  \item \(3 \tan^2 x + 5 = \dfrac{7}{\cos x}\)
   \hfill \(\quadra{x = \mp 60 \grado + k 360 \grado}\)
%   \item \(\sin^2 x - \sin x = \cos^2 x + \cos x\)
%    \hfill \(\quadra{x = \dfrac{\pi}{2} + 2 k \pi; \quad
%                    x = \dfrac{3}{4} \pi + k \pi}\)
%   \item \(5 \sin^2 x - 2 \cos^2 x - 3 \sin x \cos x = 0\)
%    \hfill \(\quadra{x = \dfrac{\pi}{4} + k \pi; \quad
%                    x = \arctan \tonda{-\dfrac{2}{5}} + k \pi}\)
%   \item \(2 \sin^2 x + 4 \sin x \cos x - 4 \cos^2 x = 1\)
%    \hfill \(\quadra{x = \dfrac{\pi}{4} + k \pi; \quad
%                    x = \arctan \tonda{- 5} + k \pi}\)
 \end{enumeratea}
\end{esercizio}

\subsubsection*{\numnameref{sec:gonio_disequazionigonio}}

\begin{esercizio}\label{ese:03.1}
Risolvi le seguenti disequazioni:
 \begin{enumeratea}
  \item \(\sin x < \dfrac{1}{2}\)
   \hfill\(\quadra{2 k \pi < x < \dfrac{\pi}{6} + 2 k \pi \quad \vee \quad 
                  \dfrac{5}{6}\pi + 2 k \pi < x < 2 \pi \tonda{1 + k}}\)
  \item \(\cos x < \dfrac{\sqrt{3}}{2}\)
   \hfill\(\quadra{-\dfrac{\pi}{6} + 2 k \pi < x < \dfrac{\pi}{6} + 2 k \pi}\)
  \item \(\tan x > \sqrt{3}\)
   \hfill\(\quadra{\dfrac{\pi}{3} + k \pi < x < \dfrac{\pi}{2} + k \pi}\)
  \item \(\cos x > - \dfrac{1}{2}\)
   \hfill\(\quadra{-\dfrac{2}{3}\pi + 2 k \pi < x < \dfrac{2}{3} \pi + 2 k \pi}\)
  \item \(\sin x < - \dfrac{\sqrt{2}}{2}\)
   \hfill\(\quadra{\dfrac{5}{4}\pi + 2 k \pi < x < \dfrac{7}{4} \pi + 2 k \pi}\)
  \item \(\tan x > \dfrac{\sqrt{3}}{3}\)
   \hfill\(\quadra{k \pi < x < \dfrac{\pi}{2} + k \pi \quad \vee \quad 
                  \dfrac{5}{6}\pi + k \pi < x < k \tonda{\pi + 1}}\)
  \item \(\sin 3 x < \dfrac{\sqrt{3}}{2}\)
   \hfill\(\quadra{- \dfrac{4}{9} \pi +\dfrac{2}{3} k \pi < x < 
                    \dfrac{\pi}{9} + \dfrac{2}{3} k \dfrac{\pi}{2}}\)
  \item \(\cos 2 x \leqslant \dfrac{1}{2}\)
   \hfill\(\quadra{...}\)
  \item \(\sin 4 x \geqslant - \dfrac{\sqrt{2}}{2}\)
   \hfill\(\quadra{...}\)
  \item \(\tan 5 x \leqslant \dfrac{\sqrt{3}}{3}\)
   \hfill\(\quadra{...}\)
  \item \(\tan 3 x \leqslant -1 \)
   \hfill\(\quadra{...}\)
 \end{enumeratea}
\end{esercizio}
% 
% \begin{esercizio}
% \label{ese:D.19}
% testo esercizio
% \end{esercizio}
% 
% \begin{esercizio}\label{ese:03.1}
% Consegna:
%  \begin{enumeratea}
%   \item  
%  \end{enumeratea}
% \end{esercizio}
% 
% \subsection{Esercizi riepilogativi}
% 
% \begin{esercizio}
% \label{ese:D.19}
% testo esercizio
% \end{esercizio}
% 
% \begin{esercizio}\label{ese:03.1}
% Consegna:
%  \begin{enumerate}
%   \item  
%  \end{enumerate}
% \end{esercizio}
