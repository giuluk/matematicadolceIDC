% (c) 2015 Daniele Zambelli daniele.zambelli@gmail.com
% Luciana Formenti

% \section{TODO}

\section{Esercizi}

% \subsection{Esercizi dei singoli paragrafi}

% \subsubsection*{\numnameref{sec:01_}}

\begin{esercizio}\label{ese:03.1}
Data la funzione \quad \(d(p) = \dfrac{750 -5p}{25}\) \quad determina: 
 \begin{enumeratea}
  \item per quali valori di p può essere assunta come una funzione di 
domanda;
  \item dopo averne costruito il grafico, determina il valore massimo e 
minimo assunto dalla funzione;
  \item la quantità di bene domandata in corrispondenza del prezzo media 
aritmetica tra prezzo massimo e prezzo minimo;
  \item il coefficiente di elasticità per una variazione di prezzo da 
\(p_1 = 100 \stext{ a } p_2 = 125\) 
e la conseguente classificazione della domanda.
 \end{enumeratea}
\end{esercizio}

\begin{esercizio}
\label{ese:D.19}
La funzione di domanda e di offerta di un certo prodotto sono espresse 
dalle seguenti equazioni:
\[d(p) = -2p^2 +1200 \quad h(p) = -40 +22p\]
Dopo aver rappresentato le due curve, determina il prezzo di equilibrio e la 
quantità di merce domandata ed offerta a tale prezzo.
\end{esercizio}

\begin{esercizio}\label{ese:03.1}
Una funzione offerta è lineare. Sappiamo che:
\(h(4) = 8 \stext{ e } h(10) = 56\).
 \begin{enumeratea}
  \item Rappresenta la funzione e verifica che può essere assunta come 
funzione offerta.
  \item Determina la sua equazione.
  \item Determina il prezzo al di sotto del quale i produttori non sono 
disposti a immettere il bene sul mercato.
  \item Il prezzo massimo, supponendo una capacità produttiva massima di 
\(296\) unità.
 \end{enumeratea}
\end{esercizio}

\begin{esercizio}\label{ese:03.1}
Data la funzione \quad \(d(p) = 2500 -p^2\) \quad determina: 
 \begin{enumeratea}
  \item per quali valori di p può essere assunta come una funzione di 
domanda;
  \item dopo averne costruito il grafico, determina il valore massimo e 
minimo assunto dalla funzione;
  \item determina il prezzo limite a partire dal quale nessuno è disposto a 
comprare il bene;
  \item il coefficiente di elasticità per una variazione di prezzo da 
\(p_1 = 20 \stext{ a } p_2 = 30\) 
e la conseguente classificazione della domanda.
 \end{enumeratea}
\end{esercizio}

\begin{esercizio}
\label{ese:D.19}
La funzione di domanda e di offerta di un certo prodotto sono espresse 
dalle seguenti equazioni:
\[d(p) = -2p +650 \quad h(p) = -850 +4p\]
Dopo aver rappresentato le due funzioni, determina il prezzo di equilibrio e 
la quantità di merce domandata ed offerta a tale prezzo.
\end{esercizio}

\begin{esercizio}\label{ese:03.1}
Una funzione offerta è lineare. Sappiamo che:
\(h(10) = 40 \stext{ e } h(60) = 12\).
 \begin{enumeratea}
  \item Determina la sua equazione.
  \item Rappresenta la funzione e verifica che può essere assunta come 
funzione offerta.
 \end{enumeratea}
\end{esercizio}

\begin{esercizio}
\label{ese:D.19}
La domanda e l’offerta di un bene sono descritte dalle funzioni:
\[d(p) = -6p +a \quad h(p) = +4p^2 +b\]
determina \(a\) e \(b\), sapendo che il prezzo di equilibrio è \(p_e = 20\) e 
la corrispondente quantità domandata e offerta è uguale a \(200\) unità.
\end{esercizio}

\begin{comment}
\begin{esercizio}
\label{ese:D.19}
testo esercizio
\end{esercizio}

\begin{esercizio}\label{ese:03.1}
Consegna:
 \begin{enumeratea}
  \item  
 \end{enumeratea}
\end{esercizio}
\end{comment}

% \subsection{Esercizi riepilogativi}


