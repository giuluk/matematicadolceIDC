% (c) 2015 Daniele Zambelli daniele.zambelli@gmail.com
% Luciana Formenti

\input{\folder modellieconomici_grafici.tex}

\chapter{Modelli Economici}

\section{Economia}
\label{sec:modelli_economia}

\footnote{
La parte più discorsiva di questo capitolo è tratta da wikipedia:\\
\url{it.wikipedia.org/wiki/Economia} \quad e \quad
\url{it.wikipedia.org/wiki/Microeconomia}\\
a cui si rimanda per la bibliografia e per i collegamenti di approfondimento.
}Per ``economia'' – dal greco %οἶκος
(\emph{oikos}), ``casa'' inteso anche come ``beni di famiglia'', e %νόμος 
(\emph{nomos}), ``norma'' o ``legge'' – si intende sia l'organizzazione 
dell'uso di risorse scarse (limitate o finite) quando attuata al 
fine di soddisfare al meglio bisogni individuali o collettivi, sia 
un sistema di interazioni che garantisce un tale tipo di organizzazione, 
sistema detto anche \emph{sistema economico}\footnote{
Lionel Robbins, \emph{Essay on the Nature and Significance of Economic 
Science}, Macmillan, London, 1945 \\
\url{http://mises.org/books/robbinsessay2.pdf}}.

Per l'economista e politico francese Raymond Barre
\footnote{Raymond Barre, \emph{Economie politique}, Presses universitaires de 
France, 1959}: 
\begin{quote}
L'economia 
è la scienza della gestione delle risorse scarse. Essa prende in esame le 
forme assunte dal comportamento umano nella gestione di tali risorse; 
analizza e spiega le modalità secondo le quali un individuo o una società 
destinano mezzi limitati alla soddisfazione di esigenze molteplici ed 
illimitate.
\end{quote}

Per l'economista inglese Alfred Marshall:
\footnote{Alfred Marshall, \emph{Principi di Economia}, 1890}
\begin{quotation}
 L'economia è uno studio del genere umano negli affari ordinari della vita. 
\end{quotation}

I soggetti che creano tali sistemi di organizzazione possono essere persone, 
organizzazioni o istituzioni. Normalmente si considerano i soggetti (detti 
anche "agenti" o "attori" o "operatori" economici) come attivi nell'ambito di 
un dato territorio; peraltro si tiene conto anche delle interazioni con altri 
soggetti attivi fuori dal territorio.

\section{Il sistema economico}
\label{sec:modelli_sistemaeconomico}

Il sistema economico, secondo la visione dell'economia di mercato della 
moderna società occidentale, è la rete di interdipendenze ed interconnessioni 
tra operatori o soggetti economici che svolgono le attività di produzione, 
consumo, scambio, lavoro, risparmio e investimento per soddisfare i bisogni 
individuali e realizzare il massimo profitto, ottimizzando l'uso delle 
risorse, evitando sprechi e aumentando la produttività individuale anche 
diminuendo il costo del lavoro.

\pagebreak %-------------------------------------------------
% \newpage %-------------------------------------------------

\subsection{Componenti o sottosistemi}

I componenti o sottosistemi del sistema economico sono:

\begin{description} [noitemsep]
 \item [Sistema di produzione:] attraverso la produzione promuove e 
determina l'offerta di beni e servizi sotto 
continua spinta all'investimento per produrre innovazione 
(aziende e imprese).
 \item [Sistema dei consumatori:] promuove e determina 
attraverso il consumo la domanda e offerta di beni e 
servizi (es. famiglie e in parte anche imprese).
 \item [Sistema creditizio-finanziario:] da esso i 
precedenti sottosistemi afferiscono fondi di liquidità (capitali) e 
strumenti finanziari per 
promuovere e raggiungere i loro obiettivi (produzione e/o consumo) 
(banche e istituti di intermediazione finanziaria).
 \item [Mercato:] è l'ambiente di interazione dei precedenti sottosistemi 
dove avviene lo scambio di beni, servizi e denaro tipicamente regolati 
dalla legge della domanda e dell'offerta.
 \item [Stato:] alimenta il sistema economico attraverso la spesa 
pubblica (offerta di servizi pubblici a fronte di prelievo fiscale), 
regolandolo anche attraverso interventi mirati di politica economica 
(politica di bilancio e politica monetaria).
\end{description}

Il livello di sviluppo ed efficienza di tali sottosistemi e del relativo 
sistema economico riflette il livello di sviluppo della società stessa e 
varia in funzione delle epoche storiche o della parte del mondo o Stato 
considerato. 
Storicamente si passa da economie prettamente agricole ad economie 
agricole-industriali fino ad arrivare a economie 
agricole-industriali-terziarie, mentre attualmente e geograficamente si 
classifica l'efficienza dei sistemi economici con le denominazioni di primo 
mondo, secondo mondo, terzo mondo e quarto mondo. Il processo di 
globalizzazione sta gradualmente portando ad una progressiva 
omogeneizzazione dei vari sistemi economici mondiali grazie 
all'interdipendenza a livello internazionale dei vari mercati nazionali 
(internazionalizzazione).

\subsection{Operatori economici e loro funzioni}

% File:Trieste Assicurazioni Generali 04032007 01.jpg Le Assicurazioni 
% Generali a Trieste

Il sistema economico può definirsi, altresì, come l'ambiente o l'insieme delle 
attività promosse dagli operatori economici per le suddette finalità. Gli 
operatori economici svolgono una o più delle seguenti funzioni:
\begin{itemize} [nosep]
\item produzione di beni e servizi;
\item consumo di beni e servizi;
\item intermediazione finanziaria;
\item accumulazione di ricchezza;
\item redistribuzione del reddito e della ricchezza;
\item assicurazione.
\end{itemize}

% \newpage %-----------------------------------------------------
\subsection{I settori economici}

Le diverse attività di produzione di beni e servizi vengono ripartite nei 
settori economici:
\begin{description} [noitemsep]
\item [settore primario,] che comprende l'agricoltura, la selvicoltura, 
la pesca, lo sfruttamento delle cave e delle miniere;
\item [settore secondario,] che comprende l'industria in senso stretto, 
l'edilizia e l'artigianato;
\item [settore terziario,] che produce e fornisce servizi.
\end{description}

Vengono attualmente utilizzate, tuttavia, classificazioni più articolate.

\section{Studio dei sistemi economici}

% File:GDP PPP per capita 2007 IMF.pngStati per PIL (PPA) pro 
% Paesi in base al PIL (PPA) pro capite del 2007

L'Economia politica studia i sistemi economici per individuarne le leggi di 
funzionamento. L'economia politica in senso moderno nasce quando si afferma 
la separazione tra etica e politica e ci si pone espressamente il problema 
della potenza economica degli Stati. Per lungo tempo tale disciplina 
si è occupata prevalentemente di sistemi economici nazionali;
i suoi concetti e metodi si sono tuttavia 
progressivamente estesi allo studio sia di sistemi sociali di ogni genere 
(economia aziendale), sia di singoli settori economici (economia 
agraria, economia industriale ecc.).

La Statistica economica ha invece come obiettivo la misurazione degli 
aspetti quantitativi di un'economia, dalla misura di grandezze semplici e di 
loro aggregati, all'analisi della dinamica e alle previsioni economiche, alla 
stima e alla verifica di modelli di comportamenti economici. Ad esempio, lo 
stato di un'economia nazionale viene rilevato mediante la contabilità 
economica nazionale (in Europa si usa il sistema di conti detto Sec95).

La Storia economica tenta di ricostruire il funzionamento di sistemi 
economici del passato, avvalendosi sia dei concetti dell'economia politica 
che dei metodi della statistica economica.

A partire dalla conoscenza o analisi del sistema economico è possibile agire 
sul sistema economico stesso con misure o interventi di politica economica 
mirati a stimolarne la stabilità o la crescita economica.

La Filosofia dell'economia è una branca della filosofia che studia le 
questioni relative all'economia o, in alternativa, il settore dell'economia 
che si occupa delle proprie fondamenta e del proprio \emph{status} di scienza 
umana.

\section{Microeconomia}
\label{sec:modelli_microeconomia}

La \emph{microeconomia} è quella branca della teoria economica che studia il 
comportamento dei singoli agenti economici, o sistemi con un numero limitato 
di agenti, che operano in condizioni di\emph{scarsità di risorse}. Assieme 
alla \emph{macroeconomia}, che studia sistemi a livello aggregato, 
costituisce la macro-categoria in cui si possono raggruppare tutte le 
discipline legate all'economia politica.

\subsection{Differenze con la macroeconomia}

La macroeconomia si 
occupa delle grandezze economiche cosiddette ''aggregate``, come, per 
esempio, il livello e il tasso di crescita del prodotto nazionale, i tassi di 
interesse, la disoccupazione e l'inflazione, le quali dipendono in 
qualche modo dalla ''somma`` delle grandezze microeconomiche ovvero dai 
comportamenti microeconomici globali dei consumatori. La filosofia di fondo è 
dunque quella del riduzionismo classico: il sistema economico globale è 
descritto a partire dalla somma delle azioni o comportamenti dei singoli 
consumatori.

Il confine tra la microeconomia e la macroeconomia è diventato negli ultimi 
anni sempre meno netto. Il motivo principale è dovuto al fatto che anche la 
macroeconomia ha a che fare con l'analisi dei mercati. Per capire come 
funzionano, infatti, è necessario comprendere prima di tutto il comportamento 
dei singoli operatori che costituiscono questi mercati. Quindi i 
macroeconomisti sono diventati sempre più attenti ai 
fondamenti microeconomici dei fenomeni economici aggregati.

\subsection{L'uso e i limiti della teoria microeconomica}

Come ogni scienza, l'economia si occupa della \emph{spiegazione} e della 
\emph{previsione} dei fenomeni osservati. La spiegazione e la previsione sono 
fondate su \emph{teorie}, le quali servono a spiegare i fenomeni osservati, 
in termini di un insieme di regole e di ipotesi di base. 

Nessuna teoria è perfettamente corretta. Ognuna parte da assunzioni di base o 
da approssimazioni più o meno ragionevoli o realistiche della realtà. 
L'utilità e la validità di una teoria dipendono dalla capacità che essa ha di 
spiegare e prevedere l'insieme dei fenomeni reali che si vogliono studiare. 
Dato questo obiettivo, le teorie sono continuamente messe a confronto 
(testate) con le osservazioni della realtà; in seguito a questo confronto, 
esse sono spesso soggette a modifica e riformulazione, e a volte anche al 
rigetto. 
Il processo di verifica e riformulazione è di primaria importanza per lo 
sviluppo dell'economia come scienza. Per valutare una teoria, è importante 
tenere presente che essa è necessariamente imperfetta.

\subsection{Analisi positiva e analisi normativa}

Le teorie nascono per spiegare i fenomeni, vengono confrontate con 
l'osservazione e sono utilizzate per costruire modelli su cui basare le 
previsioni. L'uso della teoria economica per formulare previsioni è 
importante sia per i manager delle imprese sia per le politiche economiche 
pubbliche. 

La microeconomia dà risposta a diversi interrogativi siano essi di natura 
\emph{positiva} o di natura \emph{normativa}. Gli interrogativi di natura 
''positiva`` hanno a che fare con la spiegazione e la previsione, mentre le 
questioni di natura ''normativa`` riguardano ciò che dovrebbe essere.

A volte si vuole andare oltre la spiegazione e la previsione per porsi 
domande del tipo: <<Che cosa sarebbe meglio fare?>>. 
È questo il campo dell'analisi \emph{normativa}, anch'essa importante sia per 
i manager d'impresa sia per coloro che devono prendere decisioni di politica 
economica. 
L'analisi normativa non si occupa soltanto delle diverse opzioni di politica 
economica, ma riguarda anche l'implementazione delle politiche prescelte. 
Questa analisi è spesso accompagnata da giudizi di valore. Ogni volta che 
sono necessari giudizi di valore, la microeconomia non è in grado di dirci 
quale sia la soluzione migliore, ma può chiarire le varie scelte 
alternative e aiutare quindi a individuare i problemi e a prendere delle 
decisioni.

\section{Modelli domanda e offerta}
\label{sec:modelli_modelli_domanda_offerta}

Di seguito riportiamo alcuni modelli semplificati per analizzare la relazione 
tra domanda e offerta.

% \subsection{Funzione domanda}
% \label{subsec:modelli_domanda}

\subsection{Domanda}
\label{subsec:modelli_domanda}

In microeconomia per \emph{domanda} s'intende la quantità di consumo 
richiesta dal mercato e dai consumatori di un certo bene o servizio, dato 
un determinato prezzo. 
In ottica macroeconomica, per la scuola neoclassica l'insieme 
delle domande dei singoli consumatori costituisce la domanda collettiva o 
``domanda aggregata''.

Ci sono diversi fattori che influenzano la domanda:
\begin{enumerate} [noitemsep]
 \item Il prezzo del bene acquistato;
 \item Il prezzo dei beni complementari e succedanei;
 \item Il reddito del consumatore;
 \item Le aspettative soggettive dei consumatori;
 \item Il costo del denaro;
 \item L'elasticità o la rigidità della domanda;
 \item I bisogni del consumatore.
\end{enumerate}

In questo modello semplificato considereremo solo la dipendenza dal prezzo:
\[domanda = d(p)\]
Alcune caratteristiche:
\begin{enumerate} [nosep]
\item il prezzo è sempre maggiore di zero, chi produce lo fa per guadagnare;
\item la domanda non è negativa non si può vendere al produttore;
\item la domanda diminuisce all'aumentare del prezzo.
\end{enumerate}
\[\sistema{p > 0\\ d \geqslant 0\\ d(p_2) < d(p_1) \stext{ se } p_2 > p_1}\]
Per i punti 1. e 2. il modello sarà rappresentato nel primo quadrante del 
piano cartesiano.

Possiamo avere diversi modelli, matematicamente semplici, che rispettano 
queste caratteristiche; 
In tutti questi modelli:
\begin{description}

\item [Intersezione con l'asse \(d~~(p=0)\)]
Quando \(p = 0\) si ha il massimo di beni che possono essere venduti,
\(d(0)\) indica il mercato potenziale \(MP\).

\item  [Inclinazione del grafico \(d'(p)\)]
Maggiore è il valore assoluto dell'inclinazione (che è negativa), più 
rapidamente diminuisce la domanda all'aumentare del prezzo. 

\item  [Intersezione con l'asse \(p~~(d=0)\)]
Il prezzo che provoca una domanda uguale a zero può essere considerato il 
prezzo massimo a cui un bene può essere prodotto.
\end{description}

\subsubsection{Domanda: modello lineare}

La domanda potrebbe avere una relazione lineare con il prezzo:
\[d= -ap+b \qquad \text{con} \quad a \geqslant 0 \stext{ e } b > 0\]
Il coefficiente della variabile \(p\) è un numero negativo, la retta è, 
perciò, decrescente. 
Ma non possiamo parlare di coefficiente angolare, perché le 
grandezze \(d\) e \(p\) non sono confrontabili: una è la domanda, cioè il 
numero di beni o servizi richiesti e l'altro è il prezzo,  cioè il costo di 
un bene o un servizio.

\affiancati{.59}{.39}{
\begin{esempio}
\(d = -3p + 20\)

Possiamo individuare:

\begin{itemize}
\item Il mercato potenziale: 
\[\text{mp} = d(0) = b = 20\]
\item Il prezzo massimo si ottiene cercando l'intersezione del grafico con 
l'asse dei prezzi:
\[\sistema{d = 0\\ -30p +20 = 0} \sRarrow p = \dfrac{20}{30} \approx 0,67\]
\end{itemize}
\end{esempio}
}{
\scalebox{1}{\dommodlin}
}

\subsubsection{Domanda: modello quadratico}

In generale quando il prezzo è alto la domanda è bassa e la domanda continua 
ad aumentare man mano che il prezzo diminuisce, ma quando il prezzo si 
avvicina a zero normalmente la domanda si stabilizza attorno a un certo 
valore.

Il modello quadratico rappresenta questa situazione:
\[d= -ap^2+b \qquad \text{con} \quad a \geqslant 0 \stext{ e } b > 0\]
Il coefficiente della variabile \(p^2\) è un numero negativo, la parabola ha 
la concavità rivolta verso il basso. 

Il massimo della parabola, mercato potenziale, si ha quando il prezzo è 
uguale a zero.

\affiancati{.59}{.39}{
\begin{esempio}
\(d = -0,5p + 22\)

Possiamo individuare:

\begin{itemize}
\item Il mercato potenziale: 
\[\text{mp} = d(0) = b = 22\]
\item Il prezzo massimo si ottiene cercando l'intersezione del grafico con 
l'asse dei prezzi:
\[\sistema{d = 0\\ -0,5p^2 +22 = 0} \sRarrow 
  p = \mp \sqrt{\dfrac{22}{0,5}} \approx 6,63\]
\end{itemize}
\end{esempio}
}{
\scalebox{1}{\dommodquad}
}

\subsubsection{Domanda: modello esponenziale}

Alcuni beni sono necessari, ad esempio l'acqua o il cibo: per quanto aumenti 
il prezzo, la domanda non si ridurrà mai a zero.
Il modello esponenziale rappresenta questa situazione:
\[d= ae^{-bp} \qquad \text{con} \quad a \geqslant 0 \stext{ e } b > 0\]
L'esponente 
negativo, quindi la funzione è decrescente.
Il massimo si ha quando \(p = 0\): \(d(0) = ae^{-b\cdot 0} = ae^0 = a\) 
e quindi l'intersezione con l'asse della domanda è uguale ad \(a\).

\affiancati{.59}{.39}{
\begin{esempio}
\(d= 18e^{-7p}\)

Possiamo individuare:

\begin{itemize}
\item Il mercato potenziale: 
\[\text{mp} = d(0) = 18e^{-b\cdot 0} = 18e^0 = 18\]
\item Poiché la funzione esponenziale non interseca l'asse delle ascisse, la 
funzione non ha prezzo massimo.
\end{itemize}
\end{esempio}
}{
\dommodesp
}

\subsubsection{Domanda: modello iperbolico}

Altre situazioni si presentano con alcune caratteristiche del modello lineare 
e alcune del modello esponenziale. 
Esiste un prezzo massimo, ma la curva presenta una concavità verso l'alto:
\[d = \dfrac{a}{p+b}-c \qquad \text{con} \quad a \geqslant 0 \stext{ e } b 
> 0 \stext{ e } c \geqslant 0\]
Anche in questo caso, per valori positivi di \(p\) la funzione è decrescente.
Il massimo, e quindi il mercato potenziale, si ha quando 
\(p = 0\): \(d(0) = \dfrac{a}{0+b}-c = \dfrac{a}{b}-c\).

\affiancati{.59}{.39}{
\begin{esempio}
\(d = \dfrac{80}{p+3}-4\)

Possiamo individuare:

\begin{itemize}
\item Il mercato potenziale: 
\[\text{mp} = d(0) = \dfrac{80}{0+3}-4 = \dfrac{80}{3}-4 
= \dfrac{68}{3} \approx 22,67\]
\item Il prezzo massimo si ottiene cercando l'intersezione del grafico con 
l'asse dei prezzi:
\begin{align*}
&\sistema{d = 0\\ \dfrac{80}{p+3}-4 = 0} \sRarrow 
  \dfrac{80}{p+3} = 4 \sRarrow \\ 
& \sRarrow \dfrac{p+3}{80} = \dfrac{1}{4} \sRarrow 
  p+3 = \dfrac{80}{4} \sRarrow p = 17
\end{align*}
\end{itemize}
\end{esempio}
}{
\dommodiper
}

\subsection{Coefficiente di elasticità della domanda}
\label{subsec:modelli_elasticita_domanda}

In microeconomia siamo interessati a studiare quanto è sensibile la domanda 
rispetto alla variazione dei prezzi il valore che rispecchia questa 
sensibilità viene detto: \emph{elasticità della domanda}.

L' elasticità della domanda rispetto ai prezzi venne elaborata 
dall'economista Léon Walras, e indica l'attesa variazione percentuale della 
domanda di un dato prodotto/servizio (quantità venduta \(d\)) rispetto ad una 
variazione percentuale del prezzo \(p\) (elasticità incrociata).

La variazione di prezzo assoluta ci interessa poco, è abbastanza ovvio che 
l'aumento di prezzo di un euro è poco significativa nell'acquisto di 
un'automobile, è molto significativa nell'acquisto di un litro di latte.
Quindi siamo interessati al rapporto tra la variazione di prezzo o rispetto 
al prezzo e alla variazione di domanda rispetto alla domanda:
\begin{align*}
\text{variazione relativa del prezzo} &= 
\frac{\Delta p}{p} = \frac{p_1 - p_0}{p_0} \\
\text{variazione relativa della domanda} &= 
\frac{\Delta d}{d} = \frac{d_1 - d_0}{d_0}
\end{align*}

Queste formule danno le \emph{variazioni medie}, ma se vogliamo la 
\emph{variazione relativa in un punto}, dobbiamo considerare una differenza 
infinitesima tra \(p_0 \stext{e} p_1\). 
Indicheremo le variazioni infinitesime con:

\[vrp = \frac{\delta p}{p} \quad \text{ e } \quad vrd = \frac{\delta d}{d}\]

Altra osservazione: \(\Delta d\) non è una variazione qualunque, ma la 
variazione di domanda che dipende dalla variazione di prezzo: \(\Delta p\).

Chiamiamo elasticità della domanda, \(\epsilon_d\), il rapporto tra la 
variazione relativa della domanda e la variazione relativa del prezzo:
\[\epsilon_d = -\frac {\frac {\delta d} {d} } { \frac {\delta p} {p}} =
\frac {p \cdot \delta d}{d \cdot \delta p}\]

Ogni bene differisce dall'altro per quanto riguarda l'elasticità, ossia la 
sensibilità alle variazioni del prezzo. L'elasticità della domanda dipende da 
numerosi fattori economici, anche se tende ad essere più elevata per i beni 
di lusso, per i quali sono disponibili beni sostitutivi. 
Mentre i beni di prima necessità tendono ad avere una domanda meno sensibile 
alle variazioni di prezzo.

Vi sono diverse categorie di elasticità:

\begin{description}
 \item [domanda elastica rispetto al prezzo]
quando una certa variazione del prezzo genera una maggiore variazione della 
domanda: \(\epsilon_d > 1\).
 \item [domanda a elasticità unitaria]
quando una certa variazione del prezzo genera una uguale variazione della 
domanda: \(\epsilon_d = 1\).
 \item [domanda rigida rispetto al prezzo]
quando una certa variazione del prezzo genera una variazione della 
domanda inferiore: \(\epsilon_d < 1\).
 \item [domanda totalmente rigida rispetto al prezzo]
quando una certa variazione del prezzo non genera una variazione della 
domanda: \(\epsilon_d = 0\).
\end{description}

\subsection{Offerta}
\label{subsec:modelli_offerta}

In economia, per \emph{offerta} si intende la quantità di un certo bene o 
servizio che viene messa in vendita in un dato momento a un dato prezzo.

Si suppone che per ogni bene si possa tracciare una curva di offerta 
rappresentante le diverse quantità messe in vendita dalle imprese di un bene 
o servizio in corrispondenza di ciascun prezzo.

L'offerta viene influenzata da diversi fattori:
\begin{description} [noitemsep]
 \item [Costi di produzione:] la diminuzione dei salari percepiti dagli 
operai nel settore, abbassa i costi e incrementa l'offerta.
 \item [Tecnologia:] migliore tecnologia comporta un'iniziale spesa maggiore 
per la Ricerca e lo Sviluppo, ma in seguito riduce i costi di produzione e 
incrementa l'offerta.
 \item [Prezzi:] un aumento dei prezzi incentiva la produzione.
 \item [Politiche governative:] l'abolizione dei dazi doganali determina un 
aumento dell'offerta dei prodotti esportabili.
\end{description}

Anche l'offerta può essere modellizzata, in prima approssimazione, come una 
funzione del prezzo unitario: \(\text{offerta} = h(p)\); dove \(p\) 
rappresenta il prezzo unitario e \(h\) il numero di beni o servizi prodotti e 
offerti sul mercato.

È abbastanza intuitivo che, all'aumentare del prezzo di un bene o servizio, 
aumenterà anche il numero delle persone che si organizzano per fornirlo e 
quindi aumenterà anche l'offerta.
La funzione \(h(p)\) (offerta) è dunque una funzione crescente.

Alcune caratteristiche:
\begin{enumerate} [nosep]
\item Il prezzo non è negativo: nessun produttore paga gli acquirenti. 
\item La produzione ha un limite massimo dovuto a condizioni 
socio-economiche-ambientali: limite di produzione (\(LP\)).
\item All'aumentare dei prezzo unitario aumenta la produzione:
La funzione \(h(p)\) è crescente.
\end{enumerate}
\[\sistema{p \geqslant 0\\ 0 \leqslant h \leqslant LP\\ 
h(p_2) > h(p_1) \stext{ se } p_2 > p_1}\]
Anche in questo caso il modello sarà rappresentato nel primo quadrante del 
piano cartesiano.

Possiamo avere diversi modelli, matematicamente semplici, che rispettano 
queste caratteristiche.

In tutti questi modelli:
\begin{description}
\item [Intersezione con l'asse \(p (h=0)\)]
Si incomincia a produrre e quindi ad offrire un bene o un servizio, solo 
quando il prezzo unitario supera una certa soglia minima che può essere 
identificata con i costi di produzione (\(CP\)).
\item  [Inclinazione del grafico \(h'(p)\)]
Maggiore è l'inclinazione, più rapidamente aumenta l'offerta 
all'aumentare del prezzo. 
\item  [Intersezione con l'asse del limite di produzione]
Anche prezzi unitari più elevati non portano ad un aumento di offerta.
\end{description}

\subsubsection{Produzione: modello lineare}

L'offerta potrebbe avere una relazione lineare con il prezzo:
\[h= ap-b \qquad \text{con} \quad a > 0 \stext{ e } b > 0\]
Il coefficiente della variabile \(p\) è un numero positivo, la retta è, 
perciò, crescente. 
% Ma non possiamo parlare di coefficiente angolare, perché le 
% grandezze \(d\) e \(p\) non sono confrontabili: una è la domanda, cioè il 
% numero di beni o servizi richiesti e l'altro è il prezzo,  cioè il costo di 
% un bene o un servizio.

\affiancati{.59}{.39}{
\begin{esempio}
\(h = 2p -10 \stext{ e } LP = 20\)

Possiamo individuare:

\begin{itemize}
\item Il costo di produzione è: 
\[\sistema{h = 0\\ 2p -10 = 0} \sRarrow cp = \dfrac{10}{2} = 5\]
\item Il prezzo che porta alla saturazione della produzione si ottiene 
cercando l'intersezione del grafico con l'asse del limite di produzione:
\[\sistema{h = 20\\ 2p -10 = 20} \sRarrow sp = \dfrac{30}{2} = 15\]
\end{itemize}
\end{esempio}
}{
\prodmodlin
}

\subsubsection{Produzione: modello radice}

L'offerta potrebbe avere una crescita rapida una volta superati i costi di 
produzione per poi rallentare man mano che si avvicina al limite di produzione:
\[h= a \sqrt[n]{p-b} \qquad \text{con} \quad a > 0 \stext{ e } 0 < b < p\]

\affiancati{.59}{.39}{
\begin{esempio}
\(h = 5 \sqrt{p -3} \stext{ e } LP = 18\)

Possiamo individuare:

\begin{itemize}
\item Il costo di produzione è: 
\[\sistema{h = 0\\ 5 \sqrt{p -3} = 0} \sRarrow p = 3\]
\item Il prezzo che porta alla saturazione della produzione si ottiene 
cercando l'intersezione del grafico con l'asse del limite di produzione:
\begin{align*}
&\sistema{h = 18\\ 5 \sqrt{p -3} = 18} \sRarrow 
\sqrt{p -3} = \frac{18}{5} \sRarrow \\
&p -3 = \frac{324}{25} \sRarrow 
\sRarrow p = \frac{324}{25} = \frac{399}{25} \approx 16 
\end{align*}
\end{itemize}
\end{esempio}
}{
\prodmodrad
}

% \pagebreak %-------------------------------------------------------

\subsubsection{Produzione: modello potenza}

L'offerta potrebbe avere una crescita più lenta con prezzi bassi e sempre più 
rapida man mano che i prezzi aumentano:

\[h= a p^n-b \qquad \text{con} \quad a > 0 \stext{ e } b > 0 
\stext{ e } n > 0\]

\affiancati{.59}{.39}{
\begin{esempio}
\(h = 0,1p^2-1,5 \stext{ e } LP = 22\)

Possiamo individuare:

\begin{itemize}
\item Il costo di produzione è: 
\[\sistema{h = 0\\ 0,1p^2-1,5 = 0} \sRarrow p = \sqrt{15} \approx 3,87\]
\item Il prezzo che porta alla saturazione della produzione si ottiene 
cercando l'intersezione del grafico con l'asse del limite di produzione:
\begin{align*}
&\sistema{h = 22 \\ 0,1p^2-1,5 = 22} \sRarrow 
0,1p^2 = 23,5 \sRarrow \\
&p^2 = 235 \sRarrow 
\sRarrow p = \sqrt{235} \approx 15,33 
\end{align*}
\end{itemize}
\end{esempio}
}{
\prodmodpot
}

\pagebreak %-------------------------------------------------------

\subsection{Coefficiente di elasticità dell'offerta}
\label{subsec:modelli_elasticita_offerta}

Analogamente a quanto visto per la domanda, anche l'offerta ha un suo 
coefficiente di elasticità rispetto al prezzo.

Anche qui siamo interessati alla variazione relativa dell'offerta:
\begin{align*}
\text{variazione relativa dell'offerta} &= 
\frac{\Delta h}{h} = \frac{h_1 - h_0}{h_0}
\end{align*}

Queste formule danno le \emph{variazioni medie}, se vogliamo la variazione 
in un punto dobbiamo considerare una differenza infinitesima tra 
\(p_0 \stext{e} p_1\). Indicheremo le variazioni infinitesime con:

\[vrd = \frac{\delta h}{h}\]

Chiamiamo elasticità dell'offerta, \(\epsilon_h\), il rapporto tra la 
variazione relativa dell'offerta e la variazione relativa del prezzo:
\[\epsilon_h = -\frac {\frac {\delta h} {h} } { \frac {\delta p} {p}} =
\frac {p \cdot \delta h}{h \cdot \delta p}\]
Anche per l'elasticità dell'offerta possiamo distinguere alcuni casi:
\begin{description} [noitemsep]
 \item [offerta elastica rispetto al prezzo]
quando una certa variazione del prezzo genera una maggiore variazione della 
offerta: \(\epsilon_h > 1\).
 \item [offerta a elasticità unitaria]
quando una certa variazione del prezzo genera una uguale variazione della 
offerta: \(\epsilon_h = 1\).
 \item [offerta rigida rispetto al prezzo]
quando una certa variazione del prezzo genera una variazione della 
offerta inferiore: \(\epsilon_h < 1\).
 \item [offerta totalmente rigida rispetto al prezzo]
quando una certa variazione del prezzo non genera una variazione della 
offerta: \(\epsilon_h = 0\).
\end{description}

\section{Prezzo di equilibrio}
\label{sec:modelli_prezzo_equilibrio}

\subsection{Concorrenza perfetta e monopolio}

Diremo che siamo in regime di \emph{concorrenza perfetta} quando si 
verificano le seguenti caratteristiche:
\begin{description} [nosep]
 \item[Polverizzazione (atomizzazione) del mercato:] 
esistono molti piccoli produttori dello stesso bene.
 \item[Omogeneità del prodotto:] 
le imprese non hanno la possibilità di differenziare i propri prodotti; 
di conseguenza, il consumatore percepisce in maniera identica il valore dello 
stesso prodotto di due imprese distinte.
 \item[Assenza di barriere all'entrata:] 
le imprese che vogliono entrare nel mercato non incontrano alcun ostacolo.
\end{description}

\vspace{.5em}
Il \emph{monopolio}, invece, è una forma di mercato in cui una merce, di cui 
non esiste un sostituto equivalente, è prodotta da un'unica impresa.
Sono inoltre presenti delle barriere all'entrata, quindi non è possibile per 
le altre imprese entrare facilmente nel mercato. 
% L'impresa che detiene il monopolio viene detta price maker.


\subsection{Determinazione del prezzo di equilibrio}

Se siamo in presenza di concorrenza perfetta, per un certo bene, il prezzo 
dipende solo dalla domanda e dall'offerta del mercato.

Il prezzo \(p\) di un prodotto, che rende massimo il ricavo, è determinato 
dall'equilibrio tra le due curve della domanda \(d\) e dell'offerta \(h\). 

% In economia \emph{domanda e offerta} è un modello matematico di 
% determinazione del prezzo nell'ambito del sistema 
% matematico denominato \emph{mercato}.

Chiameremo \emph{prezzo di equilibrio} (\(p_e\)) il prezzo per il quale la 
domanda e l'offerta coincidono:
\[d(p_e) = h(p_e)\]
Per questo prezzo tutta la domanda di un bene o di un servizio è soddisfatta 
e tutta l'offerta è esaurita.

\affiancati{.59}{.39}{
Disegnando nello stesso piano i grafici della domanda e dell'offerta, il 
prezzo di equilibrio è l'ascissa dell'intersezione dei due grafici e si 
ottiene risolvendo il sistema:
\[\sistema{d = d(p)\\h = h(p)}\\d = h \sRarrow d(p) = h(p)\]
}{
\equilibrioa
}

\affiancati{.59}{.39}{
\begin{esempio}
trova il prezzo di equilibrio quando la funzione domanda è: \quad
\(d = -3p +4\) \quad 
e la funzione offerta è: \quad
\(h = +5p -1 \stext{ e } LP = 22\)

Il prezzo di equilibrio si ottiene risolvendo l'equazione: 
\begin{align*}
&\-3p_e +4 = +5p_e -1 \sRarrow 
-3p_e -5p_e = -4 -1 \sRarrow \\
&-8p_e = -3 \sRarrow 
p_e = +\frac{5}{8} =0,625 
\end{align*}
\end{esempio}
}{
\equilibriob
}

\subsubsection{Offerta variabile}

\affiancati{.59}{.39}{
Se la domanda è fissa e l'offerta è variabile 
\[h_2(x) < h_1(x) \sRarrow p_{e2} > p_{e1}\]
}{
\equilibrioc
}

\subsubsection{Domanda variabile}

\affiancati{.59}{.39}{
Se l'offerta è fissa e la domanda è variabile 
\[d_2(x) > d_1(x) \sRarrow p_{e2} > p_{e1}\]
}{
\equilibriod
}









































