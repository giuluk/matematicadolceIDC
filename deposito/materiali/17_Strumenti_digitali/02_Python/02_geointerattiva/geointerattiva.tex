% (c) 2014 Daniele Zambelli - daniele.zambelli@gmail.com

% \lstset{language=Python}

% per disegnare il simbolo >>> in 05_02_tartaruga.
%\newcommand{\tggg}[0]{\textgreater\textgreater\textgreater}
%

\chapter{Geometria interattiva}

\section{Introduzione}
\label{sec:introduzione}

\emph{Cos'è la geometria interattiva, i primi oggetti.}

La geometria interattiva permette di creare gli oggetti della geometria
euclidea:
\begin{itemize} [noitemsep]
\item punti;
\item rette;
\item circonferenze.
\end{itemize}

I punti possono essere:

\begin{itemize} [noitemsep]
\item liberi;
\item vincolati a una linea;
\item intersezioni di due linee.
\end{itemize}

Le rette possono essere anche:
\begin{itemize} [noitemsep]
\item semirette;
\item segmenti.
\end{itemize}

I punti base possono essere trascinati con il mouse quindi, se ho realizzato
una costruzione geometrica a partire da alcuni punti, quando muovo questi
punti tutta la costruzione si muove.

La Geometria interattiva mette in evidenza quali sono le caratteristiche
invarianti e quali quelle variabili di una certa costruzione.

Esistono molti programmi che permettono di operare con la geometria
interattiva, a questo indirizzo se ne possono trovare ben 36:

\url{http://en.wikipedia.org/wiki/List_of_interactive_geometry_software}

In questo testo propongo l'uso del linguaggio
Python con la libreria \texttt{pyig}.

È comunque possibile seguire il percorso proposto anche con un programma
\emph{punta e clicca} invece che con un linguaggio.

Per l'installazione di Python e della libreria Pygraph, che contiene anche
Pyig, vedi l'introduzione alla geometria della tartaruga nel volume
precedente.

\subsection{Installiamo un interprete}
\label{sec:05_02installazione}

\emph{Cosa installare per lavorare con la geometria interattiva.}

\subsubsection{Python}

Chi usa come sistema operativo Windows può installare \texttt{Python} 
a partire dal sito:

\url{www.python.org/downloads}

E installare la versione più recente della serie 3.x.x.

Chi utilizza altri sistemi operativi può installarlo partendo dal proprio
gestore di pacchetti installando \texttt{Python3} e anche \texttt{IDLE}.

\subsubsection{\texttt{pygraph}}

Si può scaricare l'intero pacchetto da:

\url{bitbucket.org/zambu/pygraph/downloads}

A questo punto bisogna fare a mano alcune operazioni che dipendono dal 
proprio sistema operativo:

\subsubsection*{Windows}

\begin{itemize} [noitemsep]
\item {} Scompattare il file scaricato.
\item {} Entrare nella cartella \texttt{pygraph}.
\item {} Selezionare il file \texttt{pygraph.pth} e la cartella 
  \texttt{pygraph} lì presenti.
\item {} Copiarli nella cartella 
\texttt{C:\textbackslash Python3x\textbackslash Lib\textbackslash site-package}
\end{itemize}

Dove ``Python3x'' potrebbe essere: ``Python34'', ``Python35'' ...

\subsubsection*{MacOSX}

\begin{itemize} [noitemsep]
\item {} Scompattare il file scaricato.
\item {} Entrare nella cartella \texttt{pygraph}.
\item {} Selezionare il file \texttt{pygraph.pth} e la cartella 
  \texttt{pygraph} lì presenti.
\item {} Copiarli nella cartella \texttt{HD/libreria/python/3.x/site-package}
\end{itemize}

Se in ``HD/libreria/python/'' non è presente la cartella 
``3.4/site-packages'', bisogna crearla.

\subsubsection*{GNU/Linux}

\begin{itemize} [noitemsep]
\item {} Scompattare il file scaricato.
\item {} Entrare nella directory \texttt{pygraph}.
\item {} Aprire un terminale in questa directory.
\item {} Copiare la cartella \texttt{pygraph} e il file \texttt{pygraph.pth}
  nella cartella
  
  \texttt{/usr/lib/python3/dist-packages/}
  
  Dato che in Linux, per modificare le directory di sistema bisogna essere 
  amministratori, il comando da dare assomiglierà a questo:
  
  \texttt{sudo cp -R python* /usr/lib/python3/dist-packages/}
\end{itemize}

A questo punto se tutto è andato bene dovremmo essere in grado di avviare 
Python-IDLE e dare il comando:

\texttt{import pyig as ig}

Se non succede nulla vuol dire che tutto è andato a buon fine, 
se invece appare una scritta rossa, bisogna leggere almeno l'ultima riga
e cercare di capire cosa non è andato bene. Magari ci si può far aiutare
da qualcuno esperto nell'installazione di programmi.

Se tutto è andato per il verso giusto possiamo procedere.

\textbf{Riassumendo}
\begin{itemize} [noitemsep]
\item La geometria interattiva permette di creare e di muovere gli oggetti 
della geometria euclidea.

\item Ci sono molti programmi che permettono di giocare con la geometria
interattiva, noi utilizzeremo il linguaggio Python con la libreria 
\texttt{pyig}.

\item Gli oggetti di base sono:
\begin{itemize} [noitemsep]
\item punti:
\begin{itemize} [noitemsep]
\item liberi,
\item vincolati,
\item intersezioni;
\end{itemize}

\item rette:
\begin{itemize} [noitemsep]
\item rette,
\item semirette,
\item segmenti;
\end{itemize}
\item circonferenze.
\end{itemize}
\end{itemize}

\textbf{Prova tu}
\begin{enumerate} [noitemsep]
\item Installa Python.
\item Installa la libreria pygraph.
\end{enumerate}


\section{Elementi fondamentali}
\label{sec:elementi-fondamentali}
\emph{Come creare un piano vuoto, dei punti, delle rette e altri
oggetti geometrici.}

La geometria interattiva permette di visualizzare facilmente elementi
varianti e invarianti di una certa costruzione geometrica.

\subsection{Strumenti}

In questo capitolo utilizzeremo i seguenti strumenti di Pyig:
\begin{itemize} [noitemsep]
\item \texttt{Point(x, y)} crea un punto con date coordinate.
\item \texttt{Line(p0, p1)} crea una retta passante per p0 e p1.
\item \texttt{Ray(p0, p1)} crea una semiretta con origine in p0 passante per p1.
\item \texttt{Segment(p0, p1)} crea un segmento di estremi p0 e p1.
\item \texttt{Circle(centro, punto)} crea una circonferenza dati il centro e 
un suo punto.

\end{itemize}


\subsection{Problema}

Crea un piano e disegna i quattro vertici di un quadrato, poi
disegna i quattro lati del quadrato.
Modifica poi la figura trascinando i punti base con il mouse.

\subsection{Soluzione guidata}

\begin{enumerate} [noitemsep]
\item Crea un nuovo programma e salvarlo con il nome: 
\texttt{gi01\_elementi.py}.
Per creare un nuovo programma:

\begin{enumerate} [noitemsep]
\item avvia IDLE (in Windows: menu-programmi-Python-IDLE);
\item crea un nuovo editor: menu-file-new window;
\item salvalo nella tua cartella con il nome desiderato: menu-file-save;
\end{enumerate}

\item Incomincia a scrivere il programma.
\begin{enumerate} [noitemsep]
\item Scrivi un'intestazione fatta da commenti che contenga le informazioni:

\begin{itemize} [noitemsep]
\item data,
\item nome,
\item titolo del programma;
\end{itemize}
\end{enumerate}

Esegui il programma in modo da controllare che non ci siano errori 
(\textless{}F5\textgreater{}).
\item Il programma vero e proprio è fatto da tre parti:

\begin{enumerate} [noitemsep]
\item la lettura delle librerie;
\item il programma principale;
\item l'attivazione della finestra grafica.
\end{enumerate}

A questo punto il programma assomiglierà a:

\begin{lstlisting}
# <data>
# <autore>
# Elementi di base della geometria

"""
Problema: Disegna i quattro vertici di un quadrato.
Disegna i quattro lati del quadrato.
"""

# lettura delle librerie

# programma principale

# attivazione della finestra grafica
\end{lstlisting}

\item Fin qui abbiamo scritto solo commenti, ora incominciamo a scrivere 
comandi:
\begin{enumerate} [noitemsep]
\item Leggo la libreria Pyig e le do un nome più breve, ``ig'':
\end{enumerate}
\begin{quote}
 import pyig
\end{quote}
\begin{enumerate} [noitemsep]
\item Il programma principale consiste, per ora, in una sola istruzione,
creo un ``InteractivePlane'' della libreria ``ig'' e associo questo
oggetto all'identificatore (=alla parola) ``ip'':

\end{enumerate}
\begin{quote}
ip = pyig.InteractivePlane()
\end{quote}
\begin{enumerate} [noitemsep]
\item Rendo attiva la finestra grafica:

\end{enumerate}
\begin{quote}
ip.mainloop()
\end{quote}

\item Aggiungi le istruzioni sotto ai commenti:

\begin{lstlisting}
# <data>
# <nome>
# Elementi di base della geometria

"""
Problema: Disegna i quattro vertici di un quadrato.
Disegna i quattro lati del quadrato.
"""

# lettura delle librerie
import pyig as ig

# programma principale
ip = ig.InteractivePlane()

# attivazione della finestra grafica
ip.mainloop()
\end{lstlisting}

\item Prova il programma premendo il tasto: 
\texttt{\textless{}Ctrl-F5\textgreater{}} o cliccando su
\texttt{menu-Run-Run module}.
Deve apparire una finestra grafica con un riferimento cartesiano e una
griglia di punti.
La finestra grafica è attiva, risponde al mouse e si può chiudere.
Se non avviene questo, probabilmente è apparso un messaggio di errore in
rosso nella shell di \texttt{IDLE},
leggi il messaggio, correggi l'errore e ritenta.

\end{enumerate}

Ora incominciamo ad aggiungere al programma principale le istruzioni per
risolvere il problema. Incominciamo creando un punto.
Aggiungiamo al programma principale il comando della libreria \texttt{pyig}  che
crea un punto associando l'oggetto appena creato all'identificatore ``p\_0'':

\begin{lstlisting}
p_0 = ig.Point(3, 4)
\end{lstlisting}

È possibile trascinare con il mouse il punto nel suo piano: il punto non cambia
cambiando la sua posizione.

In geometria un punto non dovrebbe avere altre caratteristiche oltre la propria
posizione, ma a noi fa comodo poter dare ai punti anche altre caratteristiche
come: uno spessore, un colore, un'etichetta:

\begin{lstlisting}
p_0 = ig.Point(3, 4, color='red', width=6, name='A')
\end{lstlisting}

In generale a tutti gli oggetti di pyig che possono essere visualizzati si
possono assegnare le seguenti caratteristiche:

\begin{itemize} [noitemsep]
\item colore: color=\textless{}una stringa di colore\textgreater{};
\item spessore: width=\textless{}un numero\textgreater{};
\item etichetta: name=\textless{}una stringa\textgreater{};
\item visibilità: visible=\textless{}True\textgreater{} o 
\textless{}False\textgreater{}.
\end{itemize}

La sintassi del costruttore dell'oggetto \texttt{Point} è:

\begin{lstlisting}
Point(<x>, <y>
      [, visible=True][, color='green'][, width=3][, name=''])
\end{lstlisting}

I primi due parametri, \texttt{x} e \texttt{y}, sono obbligatori, quelli messi 
tra parentesi quadre sono opzionali e, se non specificati, hanno il valore
riportato sopra.

Passiamo alla seconda richiesta del problema: disegnare una retta.Per poter
tracciare una retta abbiamo bisogno di due punti infatti due punti individuano
univocamente una retta (per due punti passa una e una sola retta).
Possiamo utilizzare i due punti già disegnati e creare la retta passante per
\texttt{p\_0} e \texttt{p\_1}:

\begin{lstlisting}
s_0 = ig.Segment(p_0, p_1)
\end{lstlisting}

Anche in questo caso possiamo modificare ``colore'' e ``spessore'' del 
segmento:

\begin{lstlisting}
s_0 = ig.Segment(p_0, p_1, color='pink', width=6)      # lati
\end{lstlisting}

La sintassi del costruttore dell'oggetto \texttt{Segment} è:

\begin{lstlisting}
Segment(<punto_0>, <punto_1>
      [, visible=True][, color='blue'][, width=3][, name=''])
\end{lstlisting}

È possibile trascinare con il mouse i punti base del segmento, ma il segmento
continuerà a passare per quei due punti.

Le sintassi dei costruttori di alcuni altri oggetti della geometria interattiva:

\begin{lstlisting}
Line(<punto_0>, <punto_1>
     [, visible=True][, color='blue'][, width=3][, name=''])
    
Ray(<punto_0>, <punto_1>
    [, visible=True][, color='blue'][, width=3][, name=''])

Circle(<centro>, <uunto>
       [, visible=True][, color='blue'][, width=3][, name=''])
\end{lstlisting}

Ora crea tu una retta, una semiretta e una circonferenza.

\textbf{Riassumendo}
\begin{itemize} [noitemsep]
\item Per lavorare con la geometria interattiva dobbiamo far caricare a Python
la relativa \emph{libreria} ad esempio con il comando:

\begin{lstlisting}
import pyig
\end{lstlisting}

\item Un programma è composto (per ora) dalle seguenti parti:

\begin{lstlisting}
<intestazione>
<lettura delle librerie>
<programma principale>
<attivazione della finestra grafica>
\end{lstlisting}

\item La sintassi dei costruttori degli oggetti base della geometria è:

\begin{lstlisting}
Point(<x>, <y>
      [, visible=True][, color='green'][, width=3][, name=''])

Line(<punto_0>, <punto_1>
      [, visible=True][, color='blue'][, width=3][, name=''])

Ray(<punto_0>, <punto_1>
    [, visible=True][, color='blue'][, width=3][, name=''])

Segment(<punto_0>, <punto_1>
        [, visible=True][, color='blue'][, width=3][, name=''])
    
Circle(<centro>, <punto>
        [, visible=True][, color='blue'][, width=3][, name=''])
\end{lstlisting}

\end{itemize}

\textbf{Prova tu}

\begin{enumerate} [noitemsep]
\item Crea un nuovo programma che disegni un segmento di colore viola,
con due estremi rosa, grandi a piacere.
\item Crea un programma che disegni un rettangolo. Muovendo i punti base
continua a rimanere un rettangolo?
\item Crea un programma che disegni un triangolo. Muovendo i punti base
continua a rimanere un triangolo?
\item Crea un programma che disegni un quadrilatero delimitato da semirette.
\item Crea un programma che disegni tre punti \texttt{A}, \texttt{B} e 
\texttt{C}, disegna poi le tre circonferenze:

\begin{itemize} [noitemsep]
\item di centro \texttt{A} e passante per \texttt{B};
\item di centro \texttt{B} e passante per \texttt{C};
\item di centro \texttt{C} e passante per \texttt{A};
\end{itemize}

\item Disegna due circonferenze concentriche. Muovendo i punti base, si 
mantiene la proprietà ``essere concentriche''?
\end{enumerate}


\section{Intersezioni}
\label{sec:intersezioni}

\emph{Come usare intersezioni tra oggetti.}

Oltre a quelli visti nel capitolo precedente, per poter realizzare costruzioni
geometriche abbiamo bisogno di poter creare l'intersezione tra due rette,
tra una retta e una circondìferenza o tra due circonferenze.


\subsection{Strumenti}

In questo capitolo utilizzeremo i seguenti strumenti di Pyig:

\begin{itemize} [noitemsep]
\item \texttt{Point(x, y)} crea un punto con date coordinate.
\item \texttt{Line(p0, p1)} crea una retta passante per p0 e p1.
\item \texttt{Segment(p0, p1)} crea un segmento di estremi p0 e p1.
\item \texttt{Intersection(oggetto\_0, oggetto\_1, {[}which{]})} crea un punto 
di intersezione tra due oggetti.
\end{itemize}


\subsection{Problema}

Crea un piano e inserisci:

\begin{itemize} [noitemsep]
\item due rette nel terzo e quarto quadrante
e il segmento che congiunge la loro intersezione con l'origine;
\item una retta e una circonferenza nel secondo quadrante 
e i segmenti che congiungono le loro intersezioni con l'origine;
\item due circonferenze nel primo quadrante 
e i segmenti che congiungono le loro intersezione con l'origine;
\end{itemize}


\subsection{Soluzione guidata}

\begin{enumerate} [noitemsep]

\item Crea un nuovo programma e salvarlo con il nome: 
\texttt{gi02\_intersezioni.py}.
Per creare un nuovo programma: vedi la soluzione guidata del capitolo
precedente.

\item Scrivi un'intestazione adeguata.

\item Scrivi lo scheletro del programma:

\begin{enumerate} [noitemsep]
\item la lettura delle librerie,
\item il programma principale,
\item l'attivazione della finestra grafica;
\end{enumerate}

e verifica che tutto funzioni.

\item Ora scriviamo dei commenti che indicano come intendiamo risolvere il
problema:

\begin{lstlisting}
# Creo le due rette
# Creo un punto nell'origine degli assi
# Creo l'intersezione tra le due rette
# Creo il segmento che congiunge l'origine all'intersezione
\end{lstlisting}

\item A questo punto il programma dovrebbe apparire circa così:

\begin{lstlisting}
# <data>
# <nome>
# Intersezioni

# lettura delle librerie
import pyig as ig

# programma principale
ip = ig.InteractivePlane()

# Creo le due rette
# Creo un punto nell'origine degli assi
# Creo l'intersezione tra le due rette
# Creo il segmento che congiunge l'origine all'intersezione


# attivazione della finestra grafica
ip.mainloop()
\end{lstlisting}

\item Ora iniziamo a popolare di istruzioni il programma principale creando le
due rette:

\begin{lstlisting}
r_0 = ig.Line(ig.Point(-6, -4, width=6), ig.Point(2, -6, width=6))
r_1 = ig.Line(ig.Point(-11, -9, width=6), ig.Point(-3, -8, width=6))
\end{lstlisting}

Eseguiamo il programma controllando che rispetti le specifiche.

\item Ora dobbiamo creare un segmento con un estremo nell'origine, quindi 
dobbiamo creare un punto nell'origine:

\begin{lstlisting}
origine = ig.Point(0, 0, visible=False)
\end{lstlisting}

e siccome vogliamo che nessuno possa muoverlo, lo facciamo invisibile.

\item L'altro estremo è l'intersezione delle due rette:

\begin{lstlisting}
i_0 = ig.Intersection(r_0, r_1, color='red')
\end{lstlisting}

\item Infine creiamo il segmento:

\begin{lstlisting}
s_0 = ig.Segment(origine, i_0, color='#505010')
\end{lstlisting}

\item A questo punto il programma dovrebbe apparire circa così:

\begin{lstlisting}
# <data>
# <nome>
# Intersezioni

# lettura delle librerie
import pyig as ig

# programma principale
ip = ig.InteractivePlane()

# Creo le due rette
r_0 = ig.Line(ig.Point(-6, -4, width=6), ig.Point(2, -6, width=6))
r_1 = ig.Line(ig.Point(-11, -9, width=6), ig.Point(-3, -8, width=6))
# Creo un punto nell'origine degli assi
origine = ig.Point(0, 0, visible=False)
# Creo l'intersezione tra le due rette
i_0 = ig.Intersection(r_0, r_1, color='red')
# Creo il segmento che congiunge l'origine all'intersezione
s_0 = ig.Segment(origine, i_0, color='#505010')

# attivazione della finestra grafica
ip.mainloop()
\end{lstlisting}

\item Proviamo il programma e controlliamo che rispetti le specifiche richieste
dal problema. Cosa succede quando muovo i punti base di una retta?

\item Se tutto funziona regolarmente possiamo passare alla seconda parte
del problema:

\begin{lstlisting}
# Creo una retta
# Creo una circonferenza
# Creo le intersezioni tra la retta e la circonferenza
# Creo i segmenti
\end{lstlisting}

\item Per quanto riguarda i primi due punti non dovrebbero esserci problemi,
per il terzo invece presenta una novità rispetto a quanto visto per
l'intersezione di due rette, infatti una retta interseca una circonferenza
in due punti (e non sempre) e noi dobbiamo indicare a Python quale delle
due intersezioni vogliamo:

\begin{lstlisting}
i_1 = ig.Intersection(r_2, c_0, -1, color='red')
i_2 = ig.Intersection(r_2, c_0, +1, color='red')
\end{lstlisting}

\item Dopo aver controllato che fin qui il programma funzioni, disegniamo i due
segmenti. la seconda parte dovrebbe assomigliare a questa:

\begin{lstlisting}
# Creo una retta
r_2 = ig.Line(ig.Point(-11, 9, width=6), ig.Point(-6, 1, width=6))
# Creo una circonferenza
c_0 = ig.Circle(ig.Point(-6, 7), ig.Point(-5, 2))
# Creo le intersezioni tra la retta e la circonferenza
i_1 = ig.Intersection(r_2, c_0, -1, color='red')
i_2 = ig.Intersection(r_2, c_0, +1, color='red')
# Creo i segmenti
s_1 = ig.Segment(origine, i_1, color='#10a010')
s_2 = ig.Segment(origine, i_2, color='#10a080')
\end{lstlisting}

\item Proviamo il programma e controlliamo che rispetti le specifiche richieste
dal problema. Cosa succede quando muovo i punti base della retta?

\item Completiamo il programma per risolvere anche la terza parte del problema.

\end{enumerate}

\textbf{Riassumendo}
\begin{itemize} [noitemsep]
\item pyig mette a disposizione un oggetto intersezione. L'intersezione tra 
rette non ha bisogno di ulteriori informazioni, quella tra una retta e una
circonferenza o tra due circonferenze richiede un ulteriore argomento:
con \texttt{-1} si indica un'intersezione, con \texttt{+1} si indica l'altra.
La sintassi del costruttore di un'intersezione è:

\begin{lstlisting}
Intersection(oggetto_0, oggetto_1 [, which] 
              [, visible=True] [, color='blue'] [, width=3] [, name=''])
\end{lstlisting}

\end{itemize}

\textbf{Prova tu}
\begin{enumerate} [noitemsep]
\item Disegna una circonferenza \texttt{c\_0} con il centro nell'origine,
una retta \texttt{r\_0} e un'altra circonferenza \texttt{c\_1}.
Disegna in modo evidente le intersezioni tra la retta \texttt{r\_0} e la
circonferenza \texttt{c\_0} e tra la circonferenza \texttt{c\_1} e la
circonferenza \texttt{c\_0}.

\item Disegna una circonferenza e una retta. Poi disegna un'intersezione tra
la retta e la circonferenza e assegna a questa intersezione il nome:
``Ciao''. Poi disegna una circonferenza che ha centro nell'intersezione
e passa per il punto (3; 1).

\item Disegna le intersezioni tra due circonferenze che hanno centro in un 
estremo di un segmento e passano per l'altro estremo del segmento.

\end{enumerate}


\section{Costruzioni geometriche}
\label{sec:costruzioni}

\emph{Come usare intersezioni tra oggetti.}

Lo strumento base della geometria greca era lo spago:

\begin{itemize} [noitemsep]
\item tenendo teso un pezzo di corda si poteva rappresentare un segmento
allungabile a piacere;
\item tenendo fisso un estremo e facendo ruotare l'altro, si poteva 
rappresentare una circonferenza.
\end{itemize}

Con questo strumento hanno costruito la geometria euclidea e risolto
innumerevoli problemi.

\subsection{Strumenti}

In questo capitolo utilizzeremo i seguenti strumenti di Pyig:

\begin{itemize} [noitemsep]
\item \texttt{Point(x, y)} crea un punto con date coordinate.
\item \texttt{Line(p0, p1)} crea una retta passante per p0 e p1.
\item \texttt{Segment(p0, p1)} crea un segmento di estremi p0 e p1.
\item \texttt{Intersection(oggetto\_0, oggetto\_1, {[}which{]})} crea un 
punto di intersezione tra due oggetti.
\item \texttt{Polygon((punto0, punto1, punto2, ...))} crea un poligono dati 
i vertici.
\end{itemize}

\subsection{Problema}

Crea un piano e disegna:
\begin{itemize} [noitemsep]
\item nel primo quadrante: 
 due punti e il triangolo equilatero costruito su quei due punti;
\item nel secondo quadrante: 
 un segmento e l'asse di quel segmento;
\item nel terzo quadrante: 
 un angolo e la bisettrice di quell'angolo;
\item nel quarto quadrante: 
 due punti e il quadrato costruito su quei due punti.
\end{itemize}

\subsection{Soluzione guidata}

\begin{enumerate} [noitemsep]
\item Crea un nuovo programma e salvarlo con il nome: 
 \texttt{gi03\_costruzioni.py}.
 Per creare un nuovo programma: vedi la soluzione guidata del primo capitolo.

\item Scrivi un'intestazione adeguata.

\item Scrivi lo scheletro del programma.

\item Ora scriviamo dei commenti che indicano come intendiamo risolvere il
problema:

\begin{lstlisting}
# Disegno tre punti disposti accuratamente
# Disegno il poligono che passa per i tre punti
\end{lstlisting}

\item Risolviamo la prima parte del problema ottenendo un programma principale
simile a questo:

\begin{lstlisting}
# programma principale
ip = ig.InteractivePlane()

# Disegno tre punti disposti accuratamente
p_0 = ig.Point(1, 2, width=6)
p_1 = ig.Point(11, 2, width=6)
p_2 = ig.Point(6, 10.66, width=6)
# Disegno il poligono che passa per i tre punti
triequi = ig.Polygon((p_0, p_1, p_2),
                    width=5, color='green', intcolor='gold')
\end{lstlisting}

Osservate che il costruttore di \texttt{Poligon} vuole un primo argomento 
formato da una sequenza di punti per cui i vertici del poligono devono essere
racchiusi tra parentesi.

\item Proviamo il programma e controlliamo che rispetti le specifiche richieste
dal problema. Cosa succede quando muovo i punti base?

Accidenti, il triangolo è equilatero all'inizio, ma non lo è più quando
sposto uno dei punti base.

Dobbiamo affrontare il problema in un altro modo. Il terzo vertice va
costruito partendo dai primi due:

\begin{lstlisting}
# Disegno due punti in una posizione qualunque
# Disegno le due circonferenze che hanno centro in un punto e 
# passano per l'altro
# Trovo un'intersezione delle due circonferenze
# Disegno il poligono che ha per vertici i due punti e l'intersezione
\end{lstlisting}

\item Dovremmo ottenere un programma che assomiglia a questo:

\begin{lstlisting}
# programma principale
ip = ig.InteractivePlane()

# Disegno due punti in una posizione qualunque
p_0 = ig.Point(1, 2, width=6)
p_1 = ig.Point(11, 2, width=6)
# Disegno le due circonferenze che hanno centro in un punto e 
# passano per l'altro
c_0 = ig.Circle(p_0, p_1, width=1)
c_1 = ig.Circle(p_1, p_0, width=1)
# Trovo un'intersezione delle due circonferenze
p_2 = ig.Intersection(c_0, c_1, +1, width=6)
# Disegno il poligono che ha per vertici i due punti e l'intersezione
triequi = ig.Polygon((p_0, p_1, p_2),
                     width=5, color='green', intcolor='gold')
\end{lstlisting}

Osservate che è buona norma tenere le linee di costruzione più sottili
rispetto alle altre, o addirittura renderle invisibili \texttt{visible=False}.

Si può osservare che questa volta i punti liberi sono solo due, il terzo
vertice è vincolato alla posizione di questi due dalla nostra costruzione.
Ora, se muoviamo i punti base il nostro triangolo cambia posizione e
dimensioni, ma resta sempre un triangolo equilatero come era richiesto dal
problema.

\item A questo punto cerca su un libro di disegno su internet come risolvere
gli altri tre problemi. Risolvili con riga e compasso, poi con \texttt{pyig}.
Di seguito riporto le tre tracce di soluzione.

\item Asse di un segmento:

\begin{lstlisting}
# Disegno due punti
# Disegno il segmento
# Disegno le due circ. che hanno centro in un estremo e passano per l'altro
# Chiamo i_0 e i_1 le due intersezioni delle circonferenze
# L'asse e' la retta passante per i_0 e i_1
\end{lstlisting}

\item Bisettrice di un angolo:

\begin{lstlisting}
# Disegno tre punti: p_0, vertice, p_1
# Disegno i due lati dell'angolo: r_0 e r_1
# Disegno una circ. che ha centro nel vertice e passa per p_0
# Chiamo i_1 l'intersezione della circonferenze con il lato r_1
# Disegno le circonferenze di centro p_0 e i_1 passanti per il vertice
# Chiamo i_2 l'intersezione delle due circonferenze
# La retta vertice - i_2 e' la bisettrice cercata
\end{lstlisting}

\item Quadrato dati due vertici consecutivi:

\begin{lstlisting}
# Disegno due punti: p_0, p_1
# c_0 e' la circ. che ha centro in p_0 e passa per p_1
# c_1 e' la circ. che ha centro in p_1 e passa per p_0
# i_0 e' l'intersezione di queste due circonferenze c_0 e c_1
# c_2 e' la circ. che ha centro in i_0 e passa per p_0
# i_1 e' l'intersezione delle circonferenze c_0 e c_2
# c_3 e' la circonferenza di centro i_1 passante per p_0
# i_2 e' l'intersezione delle circonferenze c_3 e c_2
# r_0 e' la retta passante per p_0 e i_2
# p_3 e' l'intersezione della retta r_0 con la circonferenza c_0
# c_4 e' la circonferenza di centro i_3 passante per p_0
# p_2 e' l'intersezione della circonferenza c_4 con la circonferenza c_1
# Il quadrato cercato e il poligono di vertici: (p_0, p_1, p_2, p_3)
\end{lstlisting}

Quando il programma e complicato, come in quest'ultimo caso è importante
eseguire il programma ogni volta che si aggiunge un'istruzione in modo
da individuare immediatamente eventuali errori sia sintattici sia logici.

\item Completiamo il programma per risolvere anche la terza parte del problema.

\end{enumerate}

\textbf{Riassumendo}
\begin{itemize} [noitemsep]
\item Polygon permette di disegnare un poligono data una sequenza di punti.
La sintassi del costruttore di \texttt{Polygon} è:

\begin{lstlisting}
Polygon(sequenza di punti [, intcolor=white]
        [, visible=True] [, color='blue'] [, width=3] [, name=''])
\end{lstlisting}

\item Per affrontare problemi complicati: prima pianifica la soluzione
descrivendola per mezzo di commenti, poi scrivi le istruzioni per risolvere
il problema eseguendo il programma ad ogni modifica.

\item Nei libri d disegno, o in internet, si possono trovare molte costruzioni
geometriche che si possono realizzare con rette, circonferenze e
intersezioni.

\end{itemize}

\textbf{Prova tu}

\begin{enumerate} [noitemsep]
\item Disegna un quadrato dati due vertici opposti.
\item Disegna un esagono regolare dati due vertici consecutivi.
\item Disegna un esagono regolare dato il centro e un vertice.
\item Disegna un pentagono regolare dati due vertici consecutivi.
\item Disegna un parallelogramma dati tre vertici consecutivi.
\end{enumerate}


\section{Strumenti di uso comune}
\label{sec:strumenti-di-uso-comune}
\emph{Quali altri oggetti abbiamo a disposizione.}

Nel paragrafo precedente abbiamo visto come realizzare oggetti nuovi come
assi, bisettrici, triangoli, quadrati, ...
Ma se ho bisogno di vari assi per realizzare una costruzione complessa,
non è comodo per ognuno di questi ripetere tutta la costruzione.
Alcuni oggetti di uso comune sono già prefabbricati e vengono messi a
disposizione dalla libreria pyig, basta chiamarli.

Nei prossimi paragrafi riporto quelli di uso più comune, l'elenco completo
si trova nel manuale di Pygraph che è stato scaricato assieme alle librerie.

\subsection{Lettura della libreria}

Nel seguito si dà per sottinteso che all'inizio del programma sia stata 
caricata la libreria con l'istruzione:

\begin{lstlisting}
import pyig as ig
\end{lstlisting}

\subsection{\texttt{InteractivePlane}}
\label{sub:geoint_interactiveplane} 
\texttt{InteractivePlane} Crea il piano interattivo nel fare questa operazione 
si possono decidere alcune caratteristiche.

\textbf{Sintassi}

\begin{lstlisting}
<nome_variabile> = InteractivePlane([<parametri>])
\end{lstlisting}

\textbf{Osservazioni}

Il costruttore presenta molti parametri tutti con un valore predefinito.
Nel momento in cui si crea un piano cartesiano si possono quindi decidere
le sue caratteristiche. Vediamole in dettaglio:

\begin{itemize} [noitemsep]
\item titolo della finestra, valore predefinito: ``Interactive geometry'';
\item dimensione, valori predefiniti: larghezza=600, altezza=600;
\item scala di rappresentazione, valori predefiniti: una unità = 20 pixel;
\item posizione dell'origine, valore predefinito: il centro della finestra;
\item rappresentazione degli assi cartesiani, valore predefinito: 
 \texttt{True};
\item rappresentazione di una griglia di punti, valore predefinito: 
 \texttt{True};
\item colore degli assi valore predefinito: `\#808080' (grigio).
\item colore della griglia valore predefinito: `\#808080'.
\item riferimento alla finestra che contiene il piano cartesiano,
valore predefinito: \texttt{None}.
\end{itemize}

Poiché tutti i parametri hanno un valore predefinito, possiamo creare un
oggetto della classe \texttt{InteractivePlane} senza specificare alcun 
argomento: verranno usati tutti i valori predefiniti. 
Oppure possiamo specificare per
nome gli argomenti che vogliamo siano diversi dal comportamento predefinito,
si vedano di seguito alcuni esempi.

\textbf{Esempio}

Si vedano tutti gli esempi seguenti.


\subsection{\texttt{Point}} 
\label{sub:geoint_point}
\textbf{Scopo}

Crea un \emph{punto libero} date le coordinate della sua posizione iniziale.

Questo oggetto è la base di ogni costruzione; dai punti liberi dipendono,
direttamente o indirettamente, gli altri oggetti grafici.

Quando il puntatore del mouse si sovrappone ad un punto libero questo cambia
colore.
Trascinando un punto libero, con il mouse, tutti gli oggetti che dipendono
da lui, verranno modificati.

\texttt{Point} essendo un oggetto che può essere trascinato con il mouse ha un
colore predefinito diverso da quello degli altri oggetti.

\textbf{Sintassi}

\begin{lstlisting}
Point(x, y[, visible][, color][, width][, name])
\end{lstlisting}

\emph{Nota}: 
Spesso nella pratica è necessario assegnare l'oggetto creato ad un
identificatore in modo da poter fare riferimento ad un oggetto nella
costruzione di altri oggetti:

\begin{lstlisting}
<identificatore> = Point(x, y[, visible][, color][, width][, name])
\end{lstlisting}

Si vedano gli esempi seguenti.

\textbf{Osservazioni}

\begin{itemize} [noitemsep]
\item \texttt{x} e \texttt{y} sono due numeri, \texttt{x} è l'ascissa e 
\texttt{y} l'ordinata del punto.
\item Per quanto riguarda i parametri non obbligatori si veda quanto scritto nel
paragrafo relativo agli attributi degli oggetti visibili.
\end{itemize}

\emph{Nota}: 
Nel resto del manuale riporterò solo gli argomenti obbligatori, è sottinteso
che tutti gli oggetti che possono essere visualizzati hanno anche i
parametri: \texttt{visible}, \texttt{color}, \texttt{width}, \texttt{name}.

\textbf{Esempio}

Funzione definita in N ad andamento casuale.

\begin{lstlisting}
import random
ip = ig.InteractivePlane('Point')
y = 0
for x in range(-14, 14):
    y += random.randrange(-1, 2)
    ig.Point(x, y, color='red')
\end{lstlisting}


\subsection{Attributi degli oggetti geometrici}
\label{sub:geoint_attributi}
\textbf{Scopo}

\texttt{Point}, come tutti gli oggetti geometrici ha degli attributi che 
possono essere determinati nel momento della creazione dell'oggetto stesso o in
seguito. Questi attributi definiscono alcune caratteristiche degli oggetti che
possono essere visualizzati.

\begin{itemize} [noitemsep]
\item \texttt{visible} stabilisce se l'oggetto sarà visibili o invisibile;
\item \texttt{color} imposta il colore dell'oggetto;
\item \texttt{width} imposta la larghezza dell'oggetto.
\item \texttt{name} imposta il nome dell'oggetto.
\end{itemize}

\textbf{Sintassi}

\begin{lstlisting}
<oggetto>.visible = v
<oggetto>.color = c
<oggetto>.width = w
<oggetto>.name = s
\end{lstlisting}

\textbf{Osservazioni}

\begin{itemize} [noitemsep]
\item \texttt{v} è un valore booleano, può essere True o False.
\item \texttt{w} è un numero che indica la larghezza in pixel.
\item \texttt{c} può essere:

\begin{itemize} [noitemsep]
\item una stringa nel formato: ``\#rrggbb'' dove rr, gg e bb sono numeri
esadecimali di due cifre che rappresentano rispettivamente le componenti
rossa, verde, e blu del colore;
\item Una stringa contenente il nome di un colore;
\item Una terna di numeri nell'intervallo 0-1 rappresentanti le componenti
rossa verde e blu.
\end{itemize}

\item \texttt{s} è una stringa
\end{itemize}

\textbf{Esempio}

Disegna tre punti: uno con i valori di default,
uno con colore dimensione e nome definiti quando viene creato,
uno con valori cambiati dopo essere stato creato.

\begin{lstlisting}
ip = ig.InteractivePlane('attributi')
a = ig.Point(-5, 3)
b = ig.Point(2, 3, color='indian red', width=8, name='B')
c = ig.Point(9, 3)
c.color = 'dark orange'
c.width = 8
c.name = 'C'
\end{lstlisting}


\subsection{Metodi degli oggetti geometrici}
\label{sub:geoint_metodi}

\textbf{Scopo}

Tutti gli oggetti geometrici hanno anche dei metodi che danno come risultato
alcune informazioni relative all'oggetto stesso.
\begin{itemize} [noitemsep]
\item \texttt{xcoord} l'ascissa;
\item \texttt{ycoord} l'ordinata;
\item \texttt{coords} le coordinate.
\end{itemize}

\textbf{Sintassi}

\begin{lstlisting}
<oggetto>.xcoord()
<oggetto>.ycoord()
<oggetto>.coords()
\end{lstlisting}

\textbf{Osservazioni}

Non richiedono argomenti e restituiscono un particolare oggetto che può essere
utilizzato all'interno di un testo variabile.

\textbf{Esempio}

Scrivi ascissa, ordinata e posizione di un punto.

\begin{lstlisting}
ip = ig.InteractivePlane('coords, xcoord, ycoord')
a = ig.Point(-5, 8, name='A')
ig.VarText(-5, -1, 'ascissa di A: {0}', a.xcoord())
ig.VarText(-5, -2, 'ordinata di A: {0}', a.ycoord())
ig.VarText(-5, -3, 'posizione di A: {0}', a.coords())
\end{lstlisting}


\subsection{\texttt{Segment}}
\label{sub:geoint_segment}

\textbf{Scopo}

Crea un segmento dati i due estremi, i due estremi sono \emph{punti}.

\textbf{Sintassi}

\begin{lstlisting}
<identificatore> = Segment(point0, point1)
\end{lstlisting}

\textbf{Osservazioni}

\texttt{point0} e \texttt{point1} sono due punti.

\textbf{Esempio}

Disegna un triangolo con i lati colorati in modo differente.

\begin{lstlisting}
ip = ig.InteractivePlane('Segment')
# creo i 3 vertici
v0 = ig.Point(-4, -3, width=5)
v1 = ig.Point( 5, -1, width=5)
v2 = ig.Point( 2,  6, width=5)
# creo i 3 lati
l0 = ig.Segment(v0, v1, color='steel blue')
l1 = ig.Segment(v1, v2, color='sea green')
l2 = ig.Segment(v2, v0, color='saddle brown')
\end{lstlisting}


\subsection{\texttt{length}}
\label{sub:geoint_length}
\textbf{Scopo}

È il metodo della classe \texttt{Segment} che restituisce un oggetto 
\texttt{data}
contenente la lunghezza del segmento stesso.

\textbf{Sintassi}

\begin{lstlisting}
<obj>.length()
\end{lstlisting}

\textbf{Osservazioni}

La lunghezza è la distanza tra \texttt{point0} e \texttt{point1}.

\textbf{Esempio}

Disegna un segmento e scrivi la sua lunghezza.

\begin{lstlisting}
ip = ig.InteractivePlane('length')
p0 = ig.Point(-4, 7, width=5, name='A')
p1 = ig.Point(8, 10, width=5, name='B')
seg = ig.Segment(p0, p1)
ig.VarText(-5, -5, 'lunghezza di AB = {0}', seg.length())
\end{lstlisting}


\subsection{\texttt{MidPoints}}
\label{sub:geoint_midpoints}
\textbf{Scopo}

Crea il punto medio tra due punti.

\textbf{Sintassi}

\begin{lstlisting}
MidPoints(point0, point1)
\end{lstlisting}

\textbf{Osservazioni}

\texttt{point0} e \texttt{point1} sono due punti.

\textbf{Esempio}

Punto medio tra due punti.

\begin{lstlisting}
    ip = ig.InteractivePlane('MidPoints')
    # creo due punti
    p0 = ig.Point(-2, -5)
    p1 = ig.Point(4, 7)
    # cambio i loro attributi
    p0.color = "#00a600"
    p0.width = 5
    p1.color = "#006a00"
    p1.width = 5
    # creao il punto medio tra p0 e p1
    m = ig.MidPoints(p0, p1, name='M')
    # cambio gli attributi di m
    m.color = "#f0f000"
    m.width = 10
\end{lstlisting}


\subsection{\texttt{MidPoint}}
\label{sub:geoint_midpoint}
\textbf{Scopo}

Crea il punto medio di un segmento

\textbf{Sintassi}

\begin{lstlisting}
MidPoint(segment)
\end{lstlisting}

\textbf{Osservazioni}

\texttt{segment} è un oggetto che ha un \texttt{point0} e un \texttt{point1}.

\textbf{Esempio}

Punto medio di un segmento.

\begin{lstlisting}
ip = ig.InteractivePlane('MidPoint')
# creo un segmento
s = ig.Segment(ig.Point(-2, -1, color="#a60000", width=5),
                ig.Point(5, 7, color="#6a0000", width=5), 
                color="#a0a0a0")
# creo il suo punto medio
ig.MidPoint(s, color="#6f6f00", width=10, name=\'M\')
\end{lstlisting}


\subsection{\texttt{Line}}
\label{sub:geoint_line}
\textbf{Scopo}

Crea una retta per due punti.

\textbf{Sintassi}

\begin{lstlisting}
Line(point0, point1)
\end{lstlisting}

\textbf{Osservazioni}

\texttt{point0} e \texttt{point1} sono, indovina un po', due punti.

Vedi anche i metodi delle classi \emph{linea} presentati nella classe 
\texttt{Segment}.

\textbf{Esempio}

Triangolo delimitato da rette.

\begin{lstlisting}
ip = ig.InteractivePlane('Line')
# creo i 3 punti
a = ig.Point(0, 0)
b = ig.Point(1, 5)
c = ig.Point(5, 1)
# creo i 3 lati
ig.Line(a, b, color="#dead34")
ig.Line(b, c, color="#dead34")
ig.Line(c, a, color="#dead34")
\end{lstlisting}


\subsection{\texttt{Ray}}
\label{sub:geoint_ray}
\textbf{Scopo}

Traccia una semiretta con l'origine in un punto e passante per un altro
punto.

\textbf{Sintassi}

\begin{lstlisting}
Ray(point0, point1)
\end{lstlisting}

\textbf{Osservazioni}

\texttt{point0} è l'origine della semiretta che passa per \texttt{point1}.

Vedi anche i metodi delle classi \emph{linea} presentati nella classe 
\texttt{Segment}.

\textbf{Esempio}

Triangolo delimitato da semirette.

\begin{lstlisting}
ip = ig.InteractivePlane('Ray')
# creo i 3 punti
a = ig.Point(0, 0)
b = ig.Point(1, 5)
c = ig.Point(5, 1)
# creo i 3 lati
ig.Ray(a, b, color="#de34ad")
ig.Ray(b, c, color="#de34ad")
ig.Ray(c, a, color="#de34ad")
\end{lstlisting}


\subsection{\texttt{Orthogonal}}
\label{sub:geoint_orthogonal}
\textbf{Scopo}

Crea la retta perpendicolare ad una retta data passante per un punto.

\textbf{Sintassi}

\begin{lstlisting}
Orthogonal(line, point)
\end{lstlisting}

\textbf{Osservazioni}

\texttt{line} è la retta alla quale si costruisce la perpendicolare passante 
per \texttt{point}.

Vedi anche i metodi delle classi \emph{linea} presentati nella classe 
\texttt{Segment}.

\textbf{Esempio}

Disegna la perpendicolare ad una retta data passante per un punto.

\begin{lstlisting}
ip = ig.InteractivePlane('Orthogonal')
retta = ig.Line(ig.Point(-4, -1, width=5), 
                ig.Point(6, 2, width=5), 
                width=3, color='DarkOrange1', name='r')
punto = ig.Point(-3, 5, width=5, name='P')
ig.Orthogonal(retta, punto)
\end{lstlisting}


\subsection{\texttt{Parallel}}
\label{sub:geoint_parallel}
\textbf{Scopo}

Crea la retta parallela ad una retta data passante per un punto.

\textbf{Sintassi}

\begin{lstlisting}
Parallel(line, point)
\end{lstlisting}

\textbf{Osservazioni}

\texttt{line} è la retta alla quale si costruisce la parallela passante per
\texttt{point}.

Vedi anche i metodi delle classi \emph{linea} presentati nella classe 
\texttt{Segment}.

\textbf{Esempio}

Disegna la parallela ad una retta data passante per un punto.

\begin{lstlisting}
ip = ig.InteractivePlane('Parallel')
retta = ig.Line(ig.Point(-4, -1, width=5), 
                ig.Point(6, 2, width=5), 
                width=3, color='DarkOrange1', name='r')
punto = ig.Point(-3, 5, width=5, name='P')
ig.Parallel(retta, punto)
\end{lstlisting}


\subsection{\texttt{Polygon}}
\label{sub:geoint_polygon}
\textbf{Scopo}

Crea un poligono data una sequenza di vertici.

\textbf{Sintassi}

\begin{lstlisting}
Polygon(points)
\end{lstlisting}

\textbf{Osservazioni}

\texttt{points} è una sequenza di punti, può essere una lista (delimitata da
parentesi quadre) o una tupla (delimitata da parentesi tonde).

\textbf{Esempio}

Disegna un poligono date le coordinate dei vertici.

\begin{lstlisting}
ip = ig.InteractivePlane('24: Polygon')
# Lista di coordinate
coords = ((-8, -3), (-6, -2), (-5, -2), (-4, 2), (-2, 3), (0, 4),
          (2, 3), (4, 2), (5, -2), (6, -2), (8, -3))
# Costruzione di una lista di punti partendo da una lista di coordinate:
# listcompreension
ip.defwidth = 5
points = [ig.Point(x, y) for x,y in coords]
ig.Polygon(points, color='HotPink3')
\end{lstlisting}


\subsection{\texttt{perimeter} e \texttt{surface}}
\label{sub:geoint_perimeter_e_surface}
\textbf{Scopo}

Sono metodi presenti in tutte le classi \emph{figura}, restituiscono
la lunghezza del contorno e l'area della superficie dell'oggetto.

\textbf{Sintassi}

\begin{lstlisting}
<figura>.perimeter()
<figura>.surface()
\end{lstlisting}

\textbf{Osservazioni}

Sono metodi degli oggetti che sono \emph{figure piane} e non richiede argomenti.

\textbf{Esempio}

Scrive alcune informazioni relative a un poligono.

\begin{lstlisting}
poli = ig.Polygon((ig.Point(-7, -3, width=5, name="A"),
                    ig.Point(5, -5, width=5, name="B"),
                    ig.Point(-3, 8, width=5, name="C")),
                    width=4, color='magenta', intcolor='olive drab')
ig.VarText(-3, -6, "perimetro={0}", poli.perimeter(), color=\'magenta\')
ig.VarText(-3, -7, "area={0}", poli.surface(), color=\'olive drab\')
\end{lstlisting}


\subsection{\texttt{Circle}}
\label{sub:geoint_circle}
\textbf{Scopo}

Circonferenza dato il centro e un punto o il centro e il raggio (un segmento).

\textbf{Sintassi}

\begin{lstlisting}
Circle(center, point)
Circle(center, segment)
\end{lstlisting}

\textbf{Osservazioni}

\texttt{center} è il centro della circonferenza passante per \texttt{point} o 
di raggio
segment.

Vedi anche i metodi delle classi \emph{figure piane} presentati nella classe
\texttt{Polygon}.

\textbf{Esempio}

Circonferenze con centro nell'origine.

\begin{lstlisting}
ip = ig.InteractivePlane('Circle(Point, Point)')
origine = ig.Point(0, 0, visible=False, name="O")
p0 = ig.Point(-7, -3, width=5, name="P")
ig.Circle(origine, p0, color="#c0c0de", width=4)
raggio = ig.Segment(ig.Point(-7, 9, width=5, name="A"),
                    ig.Point(-4, 9, width=5, name="B"))
ig.Circle(origine, raggio, color="#c0c0de", width=4)
\end{lstlisting}

\subsection{\texttt{Intersection}}
\label{sub:geoint_intersection}
\textbf{Scopo}

Crea il punto di intersezione tra due oggetti.

\textbf{Sintassi}

\begin{lstlisting}
Intersection(obj0, obj1)
Intersection(obj0, obj1, which)
\end{lstlisting}

\textbf{Osservazioni}
% 
\texttt{obj0} e \texttt{obj1} possono essere rette o circonferenze. Se uno dei 
due
oggetti è una circonferenza è necessario specificare quale delle due
intersezioni verrà restituita indicando come terzo parametro \texttt{+1} o 
\texttt{-1}.

\textbf{Esempio}

Disegna una circonferenza tangente a una retta.

\begin{lstlisting}
ip = ig.InteractivePlane('Intersection line line')
# Disegno retta e punto
retta = ig.Line(ig.Point(-4, -1, width=5), 
                ig.Point(6, 2, width=5), 
                width=3, color='DarkOrange1', name='r')
punto = ig.Point(-3, 5, width=5, name='P')
# trovo il punto di tangenza
perpendicolare = ig.Orthogonal(retta, punto, width=1)
p_tang = ig.Intersection(retta, perpendicolare, width=5)
# disegno la circonferenza
ig.Circle(punto, p_tang, width=4, color='IndianRed')
\end{lstlisting}

Disegna il simmetrico di un punto rispetto ad una retta.

\begin{lstlisting}
ip = ig.InteractivePlane('Intersection line circle')
# disegno l'asse di simmetria e il punto
asse = ig.Line(ig.Point(-4, -11, width=5), 
               ig.Point(-2, 12, width=5), 
               width=3, color='DarkOrange1', name='r')
punto = Point(-7, 3, width=5, name='P')
# disegno la perpendicolare all'asse passante per il punto
perp = ig.Orthogonal(asse, punto, width=1)
# trovo l'intersezione tra la perpendicolare e l'asse
piede = ig.Intersection(perp, asse)
# disegno la circonferenza di centro piede e passante per punto
circ = ig.Circle(piede, punto, width=1)
# trovo il simmetrico di punto rispetto a asse
ig.Intersection(perp, circ, -1, width=5, color=\'DebianRed\', name="P\'")
\end{lstlisting}

Disegna un triangolo equilatero.

\begin{lstlisting}
ip = ig.InteractivePlane('Intersection circle circle')
# Disegno i due primi vertici
v0 = ig.Point(-2, -1, width=5, name='A')
v1 = ig.Point(3, 2, width=5, name='B')
# Disegno le due circonferenze di centro p0 e p1 e passanti per p1 e p0
c0 = ig.Circle(v0, v1, width=1)
c1 = ig.Circle(v1, v0, width=1)
# terzo vertice: intersezione delle due circonferenze
v2 = ig.Intersection(c0, c1, 1, width=5, name='C')
# triangolo per i 3 punti
ig.Polygon((v0, v1, v2), width=4, color='DarkSeaGreen4')
\end{lstlisting}

\subsection{\texttt{Text}}
\label{sub:geoint_text}
\textbf{Scopo}

Crea un testo posizionato in un punto del piano.

\textbf{Sintassi}

\begin{lstlisting}
Text(x, y, text[, iplane=None])
\end{lstlisting}

\textbf{Osservazioni}

\begin{itemize} [noitemsep]
\item \texttt{x} e \texttt{y} sono due numeri interi o razionali relativi; 
\texttt{x} è l'ascissa e \texttt{y} l'ordinata del punto.
\item \texttt{text} è la stringa che verrà visualizzata.
\item Se sono presenti più piani interattivi, si può specificare l'argomento
\texttt{iplane} per indicare in quale di questi la scritta deve essere
visualizzata.
\end{itemize}

\textbf{Esempio}

Scrive un titolo in due finestre grafiche.

\begin{lstlisting}
ip0 = ig.InteractivePlane('Text pale green', w=400, h=200)
ip1 = ig.InteractivePlane('Text blue violet', w=400, h=200)
ig.Text(-2, 2, "Prove di testo blue violet", 
        color='blue violet', width=20)
ig.Text(-2, 2, "Prove di testo pale green", 
        color='pale green', width=20, iplane=ip0)
\end{lstlisting}


\subsection{\texttt{VarText}}
\label{sub:geoint_vartext}
\textbf{Scopo}

Crea un testo variabile. Il testo contiene dei ``segnaposto'' che verranno
sostituiti con i valori prodotti dai dati presenti nel parametro variables.

\textbf{Sintassi}

\begin{lstlisting}
VarText(x, y, text, variables[, iplane=None])
\end{lstlisting}

\textbf{Osservazioni}
\begin{itemize} [noitemsep]
\item \texttt{x} e \texttt{y} sono due numeri interi o razionali relativi 
\texttt{x} è l'ascissa
e \texttt{y} l'ordinata del punto.
\item \texttt{text} è la stringa che contiene la parte costante e i segnaposto.
\item In genere i \emph{segnaposto} saranno nella forma: ``\{0\}'' che indica a 
Python di
convertire in stringa il risultato prodotto dal dato.
\item \texttt{variables} è un dato o una tupla di dati.
\item Se sono presenti più piani interattivi, si può specificare l'argomento
\texttt{iplane} per indicare in quale di questi la scritta deve essere
visualizzata.
\end{itemize}

\textbf{Esempio}

Un testo che riporta la posizione dei un punto.

\begin{lstlisting}
ip = ig.InteractivePlane('VarText')
p0 = ig.Point(7, 3, color='green', width=10, name='A')
ig.VarText(-4, -3, "Posizione del punto A: ({0}; {1})",
           (p0.xcoord(), p0.ycoord()),
           color='green', width=10)
\end{lstlisting}


\subsection{\texttt{PointOn}}
\label{sub:geoint_pointon}
\textbf{Scopo}

Punto disegnato su un oggetto in una posizione fissa.

\textbf{Sintassi}

\begin{lstlisting}
PointOn(obj, parameter)
\end{lstlisting}

\textbf{Osservazioni}

L'oggetto deve essere una linea o una retta o una circonferenza,
\texttt{parameter} è un numero che individua una precisa posizione sull'oggetto.
Sia le rette sia le circonferenze hanno una loro metrica che è legata ai punti
base dell'oggetto. Su una retta una semiretta o un segmento \texttt{point0}
corrisponde al parametro 0 mentre \texttt{point1} corrisponde al parametro 1.
Nelle circoferenze il punto di base della circonferenza stessa corrisponde
al parametro 0 l'intera circonferenza vale 2.
Il punto creato con \texttt{PointOn} non può essere trascinato con il mouse.

\textbf{Esempio}

Disegna il simmetrico di un punto rispetto ad una retta.

\begin{lstlisting}
ip = ig.InteractivePlane('PointOn')
# disegno l'asse di simmetria e il punto
asse = ig.Line(ig.Point(-4, -11, width=5), 
               ig.Point(-2, 12, width=5), 
               width=3, color='DarkOrange1', name='r')
punto = ig.Point(-7, 3, width=5, name='P')
# disegno la perpendicolare all'asse passante per il punto
perp = ig.Orthogonal(asse, punto, width=1)
# trovo il simmetrico di punto rispetto a asse
ig.PointOn(perp, -1, width=5, color=\'DebianRed\', name="P\'")
ig.Text(-5, -6, """P\' e\' il simmetrico di P.""")
\end{lstlisting}


\subsection{\texttt{ConstrainedPoint}}
\label{sub:geoint_constrainedpoint}
\textbf{Scopo}

Punto legato ad un oggetto.

\textbf{Sintassi}

\begin{lstlisting}
ConstrainedPoint(obj, parameter)
\end{lstlisting}

\textbf{Osservazioni}

Per quanto riguarda \texttt{parameter}, valgono le osservazioni fatte per
\texttt{PoinOn}.
Questo punto però può essere trascinato con il mouse pur restando sempre
sull'oggetto. Dato che può essere trascinato con il mouse
ha un colore di default diverso da quello degli altri oggetti.

\textbf{Esempio}

Circonferenza e proiezioni sugli assi.

\begin{lstlisting}
ip = ig.InteractivePlane('ConstrainedPoint', sx=200)
# Circonferenza
origine = ig.Point(0, 0, visible=False)
unix = ig.Point(1, 0, visible=False)
uniy = ig.Point(0, 1, visible=False)
circ = ig.Circle(origine, unix, color="gray10")
# Punto sulla circonferenza
cursore = ig.ConstrainedPoint(circ, 0.25, color='magenta', width=20)
# assi
assex = Line(origine, unix, visible=False)
assey = Line(origine, uniy, visible=False)
# proiezioni
py = ig.Parallel(assey, cursore, visible=False)
hx = ig.Intersection(assex, py, color='red', width=8)
px = ig.Parallel(assex, cursore, visible=False)
hy = ig.Intersection(assey, px, color='blue', width=8)
\end{lstlisting}


\subsection{\texttt{parameter}}
\label{sub:geoint_parameter}
\textbf{Scopo}

I punti legati agli oggetti hanno un metodo che permette di ottenere il
parametro.

\textbf{Sintassi}

\begin{lstlisting}
<constrained point>.parameter()
\end{lstlisting}

\textbf{Osservazioni}

In \texttt{PointOn} il parametro è fissato nel momento della costruzione
dell'oggetto. In \texttt{ConstrainedPoint} il parametro può essere variato
trascinando il punto con il mouse.

\textbf{Esempio}

Scrivi i dati relativi a un punto collegato a un oggetto.

\begin{lstlisting}
ip = ig.InteractivePlane('parameter')
c0 = ig.Circle(ig.Point(-6, 6, width=6), ig.Point(-1, 5, width=6))
c1 = ig.Circle(ig.Point(6, 6, width=6), ig.Point(1, 5, width=6))
a = ig.PointOn(c0, 0.5, name='A')
b = ig.ConstrainedPoint(c1, 0.5, name='B')
ig.VarText(-5, -1, 'ascissa di A: {0}', a.xcoord())
ig.VarText(-5, -2, 'ordinata di A: {0}', a.ycoord())
ig.VarText(-5, -3, 'posizione di A: {0}', a.coords())
ig.VarText(-5, -4, 'parametro di A: {0}', a.parameter())
ig.VarText(5, -1, 'ascissa di B: {0}', b.xcoord())
ig.VarText(5, -2, 'ordinata di B: {0}', b.ycoord())
ig.VarText(5, -3, 'posizione di B: {0}', b.coords())
ig.VarText(5, -4, 'parametro di B: {0}', b.parameter())
\end{lstlisting}


\subsection{\texttt{Angle}}
\label{sub:geoint_angle}
\textbf{Scopo}

Angolo dati tre punti o due punti e un altro angolo.
Il secondo punto rappresenta il vertice. Il verso di costruzione dell'angolo
è quello antiorario.

\textbf{Sintassi}

\begin{lstlisting}
Angle(point0, vertex, point1[, sides])
Angle(point0, vertex, angle[, sides])
\end{lstlisting}

\textbf{Osservazioni}

L'argomento \texttt{sides} può valere:

\begin{itemize} [noitemsep]
\item \texttt{True} (o \texttt{(0, 1)}): vengono disegnati i lati;
\item \texttt{0}: viene disegnato il lato 0;
\item \texttt{1}: viene disegnato il lato 1;
\end{itemize}

\texttt{Angle} fornisce i seguenti metodi dal significato piuttosto evidente:

\begin{itemize} [noitemsep]
\item \texttt{extent}: ampiezza dell'angolo;
\item \texttt{bisector}: bisettrice.
\end{itemize}

\textbf{Esempio}

Disegna un angolo e un angolo con i lati.

\begin{lstlisting}
ip = ig.InteractivePlane('Angle(Point, Point, Point)')
ip.defwidth = 5
a = ig.Point(-2, 4, color="#40c040", name="A")
b = ig.Point(-5, -2, color="#40c040", name="B")
c = ig.Point(-8, 6, color="#40c040", name="C")
d = ig.Point(8, 6, color="#40c040", name="D")
e = ig.Point(5, -2, color="#40c040", name="E")
f = ig.Point(2, 4, color="#40c040", name="F")
# angolo senza i lati
ig.Angle(a, b, c, color="#40c040")
# angolo con i lati
ig.Angle(d, e, f, color="#c04040", sides=True)
\end{lstlisting}

Somma di due angoli.

\begin{lstlisting}
ip = ig.InteractivePlane('Angle(Point, Point, Angle)')
# i 2 angoli di partenza
a = ig.Angle(ig.Point(-3, 7, width=6),
             ig.Point(-7, 5, width=6),
             ig.Point(-6, 8, width=6),
             sides=(0, 1), color="#f09000", name=\'alfa\')
b = ig.Angle(ig.Point(9, 2, width=6),
             ig.Point(2, 3, width=6),
             ig.Point(6, 4, width=6),
             sides=(0, 1), color="#0090f0", name=\'beta\')
# Punti di base dell'angolo somma di a b
v = ig.Point(-11, -8, width=6)
p0 = ig.Point(3, -10, width=6)
# la somma degli angoli
b1 = ig.Angle(p0, v, b, (0, 1), color="#0090f0")
p1 = b1.point1()
a1 = ig.Angle(p1, v, a, sides=True, color="#f09000")
ig.Text(-4, -12, "Somma di due angoli")
\end{lstlisting}


\subsection{\texttt{Bisector}}
\label{sub:geoint_bisector}
\textbf{Scopo}

Retta bisettrice di un angolo.

\textbf{Sintassi}

\begin{lstlisting}
<Bisector>(<angle>)
\end{lstlisting}

\textbf{Osservazioni}

Vedi \texttt{Ray}.

\textbf{Esempio}

Disegna l'incentro di un triangolo.

\begin{lstlisting}
ip = ig.InteractivePlane('Bisector')
# I tre vertici del triangolo
a = ig.Point(-7, -3, color="#40c040", width=5, name="A")
b = ig.Point(5, -5, color="#40c040", width=5, name="B")
c = ig.Point(-3, 8, color="#40c040", width=5, name="C")
# Il triangolo
ig.Polygon((a, b, c))
# Due angoli del triangolo
cba = ig.Angle(c, b, a)
bac = ig.Angle(b, a, c)
# Le bisettrici dei due angoli
b1 = ig.Bisector(cba, color="#a0c040")
b2 = ig.Bisector(bac, color="#a0c040")
# L'incentro
ig.Intersection(b1, b2, color="#c040c0", width=5, name="I")
\end{lstlisting}


\subsection{\texttt{Calc}}
\label{sub:geoint_calc}
\textbf{Scopo}

Dato che contiene il risultato di un calcolo.

\textbf{Sintassi}

\begin{lstlisting}
Calc(function, variables)
\end{lstlisting}

\textbf{Osservazioni}
\begin{itemize} [noitemsep]
\item \texttt{function} è una funzione python, al momento del calcolo, alla
funzione vengono passati come argomenti il contenuto di \texttt{variables}.
\item \texttt{variables} è un oggetto \texttt{Data} o una tupla che contiene 
oggetti \texttt{Data}. 
Il risultato è memorizzato all'interno dell'oggetto \texttt{Calc} e può
essere visualizzato con \texttt{VarText} o utilizzato per definire la posizione
di un punto.
\end{itemize}

\textbf{Esempio}

Calcola il quadrato di un lato e la somma dei quadrati degli altri due di
un triangolo.

\begin{lstlisting}
ip = ig.InteractivePlane('Calc')
ig.Circle(ig.Point(2, 4), ig.Point(-3, 4), width=1)
ip.defwidth = 5
a = ig.Point(-3, 4, name="A")
b = ig.Point(7, 4, name="B")
c = ig.Point(-1, 8, name="C")
ab = ig.Segment(a, b, color="#40c040")
bc = ig.Segment(b, c, color="#c04040")
ca = ig.Segment(c, a, color="#c04040")
q1 = ig.Calc(lambda a: a*a, ab.length())
q2 = ig.Calc(lambda a, b: a*a+b*b, (bc.length(), ca.length()))
ig.VarText(-5, -5, "ab^2 = {0}", q1, color="#40c040")
ig.VarText(-5, -6, "bc^2 + ca^2 = {0}", q2, color="#c04040")
\end{lstlisting}

\textbf{Riassumendo}
\begin{itemize} [noitemsep]
\item In questo paragrafo sono stati presentati i seguenti oggetti.
\begin{itemize} [noitemsep]
\item \texttt{Angle}
Angolo dati tre punti o due punti e un angolo, il secondo punto
rappresenta il vertice.
Il verso di costruzione dell'angolo è quello antiorario.
\item \texttt{Bisector}
Retta bisettrice di un angolo.
\item \texttt{Circle}
Circonferenza dato il centro e un punto o il centro e un raggio
(un segmento).
\item \texttt{ConstrainedPoint}
Punto legato ad un oggetto.
\item \texttt{Calc}
Dato che contiene il risultato di un calcolo.
\item \texttt{InteractivePlane}
Crea il piano cartesiano e inizializza gli attributi del \emph{piano}.
\item \texttt{Intersection}
Crea il punto di intersezione tra due rette.
\item \texttt{Line}
Crea una retta per due punti.
\item \texttt{MidPoint}
Crea il punto medio di un segmento
\item \texttt{MidPoints}
Crea il punto medio tra due punti.
\item \texttt{Orthogonal}
Crea la retta perpendicolare ad una retta data passante per un punto.
\item \texttt{Parallel}
Crea la retta parallela ad una retta data passante per un punto.
\item \texttt{Point}
Crea un \emph{punto libero} date le coordinate della sua posizione iniziale.
\item \texttt{PointOn}
Punto disegnato su un oggetto in una posizione fissa.
\item \texttt{Polygon}
Crea un poligono data una sequenza di vertici.
\item \texttt{Ray}
Traccia una semiretta con l'origine in un punto e passante per un altro
punto.
\item \texttt{Segment}
Crea un segmento dati i due estremi, i due estremi sono \emph{punti}.
\item \texttt{Text}
Crea un testo posizionato in un punto del piano.
\item \texttt{VarText}
Crea un testo variabile. Il testo contiene dei ``segnaposto'' che verranno
sostituiti con i valori prodotti dai dati presenti nel parametro variables.
\end{itemize}

\item In questo paragrafo sono stati presentati i seguenti attributi.
\begin{itemize} [noitemsep]
\item \texttt{\textless{}oggetto\_visibile\textgreater{}.color}
Attributo degli oggetti geometrici: imposta il colore dell'oggetto;
\item \texttt{\textless{}oggetto\_visibile\textgreater{}.name}
Attributo degli oggetti geometrici: imposta il nome dell'oggetto.
\item \texttt{\textless{}oggetto\_visibile\textgreater{}.visible}
Attributo degli oggetti geometrici: stabilisce se l'oggetto sarà
visibili o invisibile;
\item \texttt{\textless{}oggetto\_visibile\textgreater{}.width}
Attributi degli oggetti geometrici: imposta la larghezza dell'oggetto.
\end{itemize}

\item In questo paragrafo sono stati presentati i seguenti metodi.
\begin{itemize} [noitemsep]
\item \texttt{\textless{}circonferenza\textgreater{}.radius}
Metodo delle classi \emph{circonferenza} che  restituisce un oggetto 
\texttt{data} che contiene la lunghezza del raggio della circonferenza.
\item \texttt{\textless{}figura\textgreater{}.perimeter}
Metodo delle classi \emph{figura} che  restituisce un oggetto \texttt{data}
contenete la lunghezza del contorno dell'oggetto.
\item \texttt{\textless{}figura\textgreater{}.surface}
Metodo delle classi \emph{figura} che  restituisce un oggetto \texttt{data}
contenete l'area della superficie dell'oggetto.
\item \texttt{\textless{}oggetto\_visibile\textgreater{}.coords}
Restituisce un dato che contiene le coordinate.
\item \texttt{\textless{}oggetto\_visibile\textgreater{}.xcoord}
Metodo degli oggetti visualizzabili: restituisce un dato che contiene
l'ascissa.
\item \texttt{\textless{}oggetto\_visibile\textgreater{}.ycoord}
Metodo degli oggetti visualizzabili: restituisce un dato che contiene
l'ordinata.
\item \texttt{\textless{}punto\_legato\textgreater{}.parameter}
Metodo dei punti legati agli oggetti che restituisce un oggetto \texttt{data}
contenete il parametro.
\item \texttt{\textless{}segmento\textgreater{}.length}
Metodo della classe \texttt{Segment} che restituisce un oggetto \texttt{data}
contenete la lunghezza del segmento stesso.
\end{itemize}
\end{itemize}

\textbf{Prova tu}
\begin{enumerate} [noitemsep]
\item Ricopia e modifica alcuni esempi del manuale.
\item Disegna un triangolo con evidenziati i punti medi dei lati.
\item Disegna un quadrato usando gli oggetti: \texttt{Orthogonal} e 
 \texttt{Parallel}.
\item Disegna un esagono regolare dato il centro e un vertice.
\item Disegna un poligono regolare dato il centro, un vertice e il numero di 
 lati.
\item Disegna un poligono regolare e tutte le sue diagonali.
\end{enumerate}
