% (c) 2015 Daniele Zambelli daniele.zambelli@gmail.com

\input{\folder esp_log_graf.tex}

\chapter[Esponenziali e logaritmi]{Esponenziali e logaritmi}

\section{Un problema}
\label{sec:esplog_problemi}

\emph{
Di ritorno da una viaggio nella foresta pluviale, mi sono portato come ricordo 
una piccola piantina che vive sulla superficie dell'acqua. 
Il primo giorno di giugno getto la piantina nello stagno vicino a casa, che ha 
una superficie di circa \(1\) km\(\,^2\).
Nei giorni seguenti vado a controllare lo stagno e non vedo più la pianta: il 
vento l'avrà spinta chissà dove! 
Questa specie ha la particolarità di duplicarsi ogni giorno e gli animali che 
vivono dalle nostre parti non la conoscono e non se ne cibano.
Parto per un altro viaggio, torno il trenta giugno e scopro che la piantina 
ha ricoperto tutto lo stagno.}

\begin{enumerate}
 \item
\emph{Se in 30 giorni la pianta ha ricoperto tutto lo stagno, in quanti giorni 
ne aveva coperto metà?}
 \item 
\emph{Quanta superficie era ricoperta il 20 giugno?}
 \item 
\emph{Quanta superficie era ricoperta il 10 giugno?}
\end{enumerate}

\noindent Vediamo come rispondere a queste tre domande:

\begin{enumerate}
 \item
L'idea è quella di non guardare il periodo complessivo 
ragionando in avanti. Con questa impostazione verrebbe spontaneo pensare che se 
in 30 giorni copre tutto lo stagno, metà stagno sarà coperto in 15 giorni. Purtroppo non 
è così.
Innanzitutto noi non abbiamo informazioni su quanto la pianta ricopre lo stagno il primo 
giorno, ma su quanta parte di stagno è ricoperta il trentesimo giorno.
Quindi dobbiamo partire dalla fine: se il primo luglio la superficie dello 
stagno è tutta coperta e ogni giorno la pianta raddoppia, il giorno precedente 
sarà ricoperta solo la metà dello stagno. Quindi per ricoprire metà dello stagno 
ha impiegato 29 giorni e per ricoprire l'altra metà impiega solo un giorno.

 \item
Per calcolare quale superficie è ricoperta dalla pianta il ventesimo giorno, 
possiamo costruire una tabella, sempre partendo dalla fine:

\begin{center}
\begin{tabular}{c|c|c|c|c|c|c|c}
giorno & 30 & 29 & 28 & 27 & 26 & 25 & \dots \\[6pt]
\hline &&&&&&\\ [-6pt]
superficie & 1 & \(\dfrac{1}{2}\) & 
 \(\dfrac{1}{4}\) & 
 \(\dfrac{1}{8}\) & 
 \(\dfrac{1}{16}\) & 
 \(\dfrac{1}{32}\) & \dots \\
\end{tabular}
\end{center}

Infatti ogni giorno la parte ricoperta è la metà del giorno successivo.
Completando la tabella puoi scoprire quale parte di stagno è ricoperta il 
ventesimo giorno. Ma esiste un modo più veloce per calcolarlo?
\'E sufficiente riscrivere la tabella nel seguente modo:

\begin{center}
\begin{tabular}{c|c|c|c|c|c|c|c}
giorno & 30 & 29 & 28 & 27 & 26 & 25 & \dots \\[6pt]
\hline &&&&&&\\ [-6pt]
superficie & 1 & \(\dfrac{1}{2^1}\) & 
 \(\dfrac{1}{2^2}\) & 
 \(\dfrac{1}{2^3}\) & 
 \(\dfrac{1}{2^4}\) & 
 \(\dfrac{1}{2^5}\) & \dots \\
\end{tabular}
\end{center}

Risulta definita automaticamente una formula che permette di calcolare la parte di stagno ricoperta
$n$ giorni prima dell'ultimo.
\begin{itemize}
 \item 3 giorni prima dell'ultimo giorno è ricoperto $\dfrac{1}{2^3}$ di stagno.
 \item 10 giorni prima dell'ultimo giorno (ovvero il $20^\circ$ giorno) è ricoperto $\dfrac{1}{2^{10}}$ di stagno.
 \item $n$ giorni prima dell'ultimo giorno è ricoperto $\dfrac{1}{2^{n}}$ di stagno.
\end{itemize}

\item Con quest'ultima formula è possibile calcolare immediatamente quanta 
superficie dello stagno è ricoperta il decimo giorno (ovvero 20 giorni prima della fine):
\[\text{superficie} = \dfrac{1}{2^{20}} \approx 
\text{ 1 milionesimo della superficie}\]
Ciò significa che nei primi 10 giorni arriva a coprire meno di un milionesimo 
della superficie dello stagno e poi in un giorno da metà lago lo copre tutto.
\end{enumerate}

\section{Esponenziali}
\label{sec:esplog_esponenziali}

Il fenomeno riportato nel problema precedente trova un proprio modello nelle 
funzioni esponenziali. Studiamone in dettaglio il comportamento.

\subsection{La successione delle potenze di~2}
\label{subsec:esplog_succpotdue}

Riprendiamo l'esempio precedente, ma poniamo come giorno zero quello in cui viene 
ricoperto l'intero stagno: al valore zero corrisponde la superficie uno, al 
valore uno la superficie doppia, \dots. Nei giorni precedenti: al valore meno uno 
corrisponde la superficie $1/2$, al valore meno due \dots.

Completando la tabella e riportando i valori nel grafico si ottiene:

\begin{figure}[h]
 \centering
 \begin{minipage}[]{.48\textwidth}
  \begin{center}
   \begin{tabular}{c|l}
    $n$   & $y=2^n$ \\
    \hline
    \dots & \dots \\
    $-4$ & $2^{-4} = 0,0625$ \\
    $-3$ & $2^{-3} = 0,125$ \\
    $-2$ & $2^{-2} = 0,25$ \\
    $-1$ & $2^{-1} = 0,5$ \\
    $0$ & $2^{0} = 1$ \\
    $+1$ & $2^{+1} = 2$ \\
    $+2$ & $2^{+2} = 4$ \\
    $+3$ & $2^{+3} = 8$ \\
    \dots & \dots \\
   \end{tabular}
  \end{center}
 \end{minipage}
\begin{minipage}[]{.48\textwidth}
\begin{center}
\begin{inaccessibleblock}[I punti della tabella precedente riportati nel piano 
cartesiano si dispongono lungo una curva che 
a sinistra si trovano appena sopra all'asse x, 
attraversano l'asse y nel punto~1, poi crescono molto rapidamente.]
  \puntia
  \caption{Grafico di $y = 2^n$} \label{fig:potdue0}
\end{inaccessibleblock}
\end{center}
\end{minipage}
\end{figure}

\begin{osservazione}
Possiamo fare alcune osservazioni su questa successione.

\begin{enumerate}
 \item 
  È sempre crescente: se \(n_1 > n_0\) allora \(2^{n_1} > 2^{n_0}\)
 \item 
  Verso sinistra i valori di \(2^n\) diventano sempre più piccoli, ma rimangono 
  sempre maggiori di zero. Quindi se \(n\) è un infinito negativo 
  allora \(2^n\) sarà un infinitesimo.
 \item 
  Verso destra cresce molto rapidamente. Quindi se \(n\) è un infinito positivo 
  allora \(2^n\) sarà anch'esso un infinito positivo.
 \item 
  \'E interessante studiare l'incremento della funzione \;$\Delta y_n = y_{n+1}-y_n$:
\begin{center}
\renewcommand\arraystretch{1.8}
\begin{tabular}{c|ccccccccc}
$n$ & -4    & -3    & -2   & -1  & 0 & +1 & +2 & +3  & \dots \\
\hline 
% \vspace*{4pt}
$y$ & 0,0625 & 0,125 & 0,25 & 0,5 & 1 & 2  & 4  & 8 & \dots\\
\hline
% \vspace*{4pt}
$\Delta y_n$ & 
  0,0625 & 0,125 & 0,25 & 0,5 & 1 & 2  & 4  & 8& \dots \\
\end{tabular}
\end{center}

  Possiamo osservare che l'incremento della successione in un punto $\Delta y_n$
  è uguale al valore della successione in quel punto $y_n$

\end{enumerate}
\end{osservazione}

\subsection{Le potenze di~2 con esponente reale}
\label{subsec:esplog_potdue}

Abbiamo visto quanto vale \(2^0\) e \(2^1\), ma se l'esponente è in mezzo 
tra zero e uno, cosa succede? Ingrandiamo la scala del grafico disegnato sopra 
e andiamo a studiare il comportamento della funzione reale: \(y=2^x\) quando 
\(x\) assume valori non interi:

\begin{figure}[h]
 \centering
 \begin{minipage}[]{.48\textwidth}
%  \vspace*{1cm}
  \begin{center}
  \renewcommand\arraystretch{1.3}
   \begin{tabular}{r|l}
    $n$   & $y=2^n$ \\
    \hline
    \dots & \dots \\
    $-1$ & $2^{-1} = 0,5$ \\
    $-0,75$ & $2^{-0,75} = 2^{\frac{-3}{4}} = \sqrt[4]{2^{-3}} = 0,5946$ \\
    $-0,5$ & $2^{-0,5} = 2^{\frac{-1}{2}} = \sqrt{2^{-1}} = 0,7071$ \\
    $-0,25$ & $2^{-0,25} = 2^{\frac{-1}{4}} = \sqrt[4]{2^{-1}} = 0,8409$ \\
    $0$ & $2^{0} = 1$ \\
    $0,25$ & $2^{0,25} = 2^{\frac{1}{4}} = \sqrt[4]{2} = 1,1892$ \\
    $+0,5$ & $2^{0,5} = 2^{\frac{1}{2}} = \sqrt{2} = 1,4142$ \\
    $0,75$ & $2^{0,75} = 2^{\frac{3}{4}} = \sqrt[4]{2^3} = 1,6818$ \\
    $+1$ & $2^{+1} = 2$ \\
    \dots & \dots \\
   \end{tabular}
%  \vspace*{1.8cm}
  \end{center}
 \end{minipage}
\begin{minipage}[]{.48\textwidth}
\begin{center}
\begin{inaccessibleblock}[Ingrandimento del grafico precedente con alcuni 
punti interpolati.]
  \vspace*{.8cm}
  \puntib
  \vspace*{.65cm}
  \caption{Grafico di $y=2^x$} \label{fig:potdue1}
\end{inaccessibleblock}
\end{center}
\end{minipage}
\end{figure}

Disegnando punti sempre più fitti si può pensare di disegnare i punti 
corrispondenti ad ogni valore \(x \in \R\) ed ottenere così il grafico della 
funzione reale \(y=2^x\).

\begin{definizione}[Funzioni esponenziali]{
Chiamiamo \emph{funzione esponenziale} una funzione in cui la variabile indipendente compare all'esponente.
}
\end{definizione}

Molti fenomeni hanno, per certi periodi, un andamento che può essere 
modellizzato da una funzione esponenziale, sono i fenomeni dove la crescita è 
proporzionale al valore in un dato momento. Alcuni esempi:

\begin{figure}[h]
 \centering
 \begin{minipage}[]{.46\textwidth}
%  \vspace*{1cm}
\begin{itemize}
\small 
 \item 
Il capitale che cresce con un certo tasso di interesse e che quindi ha una 
crescita proporzionale al valore del capitale stesso.
 \item 
La crescita di una popolazione, in condizioni favorevoli: maggiori sono gli individui fertili e più rapidamente cresce
la popolazione. Si può pensare in particolare alla diffusione di specie non autoctone in 
territori dove non trovano predatori.
 \item 
Crescita del numero di batteri o virus in un 
individuo contagiato da una malattia.
 \item 
La diffusione di un'epidemia.
 \item 
L'aumento di temperatura dovuto all'aumento di ``gas serra'' che porta allo 
scioglimento delle calotte polari con la diffusione di ulteriori quantità di 
``gas serra''.
\end{itemize}

\end{minipage} \quad
\begin{minipage}[]{.48\textwidth}
\begin{center}
\begin{inaccessibleblock}[Grafico della funzione esponenziale che 
a sinistra si trova appena sopra all'asse x, 
attraversa l'asse y nel punto~1, poi cresce molto rapidamente.]
  \graficoesponenziale
  \caption{Grafico della funzione \(y=2^x\).} \label{fig:funx2^x}
\end{inaccessibleblock}
\end{center}
\end{minipage}
\end{figure}

\subsection{Le funzioni esponenziali}
\label{subsec:esplog_fesponenziale}

È ora il momento di generalizzare la funzione. 
Studiamo come si comporta la funzione \(y=a^x\) per diversi valori di \(a\).
Cosa succede se la base della potenza è un numero diverso da 2? 
Come cambierà il suo grafico?\\[10pt]
\fbox{{$a<0$}}\; Proviamo a disegnare, per esempio, il grafico della funzione: \(y=(-2)^x\).\\[4pt]
Diamo ad \(x\) diversi valori, calcoliamo i corrispondenti valori di \(y\) e 
riportiamoli su un grafico (vedi Figura \ref{fig:potmenodue0}).
\begin{figure}[h]
 \centering
 \begin{minipage}[]{.48\textwidth}
 \vspace*{.6cm}
  \begin{center}
   \begin{tabular}{r|l}
    $x$   & $y=(-2)^x$ \\
    \hline
    \dots & \dots \\
    $-5$ & $(-2)^{-5} = -0,03125$ \\
    $-4$ & $(-2)^{-4} = +0,0625$ \\
    $-3$ & $(-2)^{-3} = -0,125$ \\
    $-2$ & $(-2)^{-2} = +0,25$ \\
    $-1$ & $(-2)^{-1} = -0,5$ \\
    $0$ & $(-2)^{0} = +1$ \\
    $+1$ & $(-2)^{+1} = -2$ \\
    $+2$ & $(-2)^{+2} = +4$ \\
    $+3$ & $(-2)^{+3} = -8$ \\
    \dots & \dots \\
   \end{tabular}
 \vspace*{.6cm} \label{tab:potmenodue0}
  \end{center}
 \end{minipage}
\begin{minipage}[]{.48\textwidth}
\begin{center}
\begin{inaccessibleblock}[I punti della tabella precedente riportati nel piano 
cartesiano si dispongono alcuni nel semipiano positivo e alcuni nel semipiano 
negativo.]
  \puntimenodue
  \caption{Grafico di $y=(-2)^n$} \label{fig:potmenodue0}
\end{inaccessibleblock}
\end{center}
\end{minipage}
\end{figure}
La successione risulta un po' strana, ma è accettabile. Cosa succede però se 
l'esponente è un numero con la virgola? Proviamo a far calcolare alla 
calcolatrice le seguenti espressioni:
\[\tonda{-2}^{1,5}= \dots \qquad \tonda{-2}^{2/7}= \dots \qquad 
 \tonda{-2}^{-1,5}= \dots \qquad \tonda{-2}^{3/8}= \dots \qquad 
\]
Molto probabilmente la calcolatrice si rifiuterà di calcolare queste 
espressioni perché il loro risultato non è un numero reale. 
Quanto visto per \(-2\) vale per qualunque base negativa.
Possiamo concludere che, se la base è negativa, possiamo calcolare la successione 
delle sue potenze, ma non possiamo calcolare i valori della funzione 
per qualunque \(x \in \R\). 
La funzione \(y=a^x \text{ con } a<0\) non è definita nei numeri reali 
(in \(\R\)).
\paragraph{\fbox{$a=0$}} Se $a=0$, otteniamo di nuovo una funzione alquanto strana. Infatti $y = 0^x$ vale
zero per qualunque $x>0$, mentre non esiste per valori $x<0$.\\[4pt]

D'ora in avanti considereremo esponenziali solamente con base $a>1$.

\paragraph{\fbox{$a>1$}}
Tenendo presente il valore delle potenze che abbiamo imparato, possiamo 
affermare che maggiore è la base e più ripido diventa il grafico della funzione 
sulla destra (per valori positivi) mentre a sinistra (per valori negativi) il 
grafico si appiattisce più rapidamente sull'asse \(x\).
\paragraph{\fbox{$a=1$}}
% Se la base è uguale a~1 la qualunque potenza sarà uguale a~1 quindi la funzione 
% diventa una costante:~\(y=1\). 
Se la base è proprio 1, la funzione esponenziale 
diventa molto particolare: infatti per ogni valore di $x$ la funzione \(y=1^x\) vale sempre 1.
\paragraph{\fbox{$0<a<1$}}
Calcolando alcuni valori di una funzione con la base minore di~1, 
ad esempio \(y=\tonda{\frac{1}{2}}^x\), possiamo osservare facilmente che il 
comportamento della funzione assomiglia molto a quello della funzione
\(y=2^x\), ma i valori ottenuti sono simmetrici rispetto all'asse \(y\). 
La funzione è decrescente, i valori verso sinistra crescono rapidamente, mentre 
verso destra si appiattiscono sull'asse \(x\) rimanendo comunque sempre 
positivi. In effetti si può notare che:
\[\tonda{\frac{1}{2}}^x = 2^{-x} \quad \text{e quindi, in generale: }\quad \tonda{\frac{1}{a}}^x = a^{-x}\]
Quindi una funzione esponenziale con base minore di 1 è equivalente alla rotazione intorno all'asse $x$ della
corrispondente funzione esponenziale con base maggiore di 1. Da notare inoltre che, se la base si avvicina
a 1, rimanendo maggiore di 1, la funzione a destra si 
appiattisce crescendo più lentamente e a sinistra si avvicina più 
lentamente all'asse \(x\). Se la base si avvicina a 0 o all'infinito, la funzione si avvicina agli assi.

\vspace{5pt}

\begin{figure}[h]
\begin{minipage}{.69\textwidth}
 \begin{inaccessibleblock}[Grafici di funzioni esponenziali con basi diverse.]
  \espdiversebasi
\end{inaccessibleblock}
\end{minipage}
\hspace{12pt}
\begin{minipage}{.20\textwidth}
 \begin{enumerate} [label=\alph*]
   \item :~$y=...$
   \item :~$y=...$
   \item :~$y=...$
   \item :~$y=...$
   \item :~$y=...$
   \item :~$y=...$
   \item :~$y=...$
   \item :~$y=...$
 \end{enumerate}
\end{minipage}
  \caption{Esponenziali con diverse basi.} \label{fig:diversebasi}
\end{figure}

Possiamo quindi riassumere quanto detto in poche semplici regole: 
\begin{enumerate*}
 \item La funzione esponenziale è sempre maggiore di zero qualunque sia la base 
e l'esponente.
 \item Se la base $a>1$ la funzione è strettamente crescente.
 \item Se la base $0<a<1$ la funzione è strettamente decrescente.
 \item Se $a=1$ la funzione è costante.
 \item Se il valore assoluto di $a$ è grande, il grafico si avvicina agli assi.
 \item Se il valore assoluto di $a$ è vicino a uno, il grafico si avvicina alla 
retta~\(y=1\).
 \item La concavità è sempre verso l'alto.
%  \item Nel grafico precedente ho evidenziato dei punti che dovrebbero aiutare 
% a scoprire la base della funzione esponenziale.
\end{enumerate*}

In figura \ref{fig:diversebasi} sono evidenziati dei punti che dovrebbero 
aiutare a scoprire la base della funzione esponenziale corrispondente.

\begin{osservazione}
La base dell'esponenziale può essere un numero qualunque: per esempio è possibile disegnare il grafico di $y=\pi^x$
come pure il grafico di $y = \sqrt{7}^{\,x}$. Una base che riveste particolare importanza nello studio di
problemi di matematica avanzata è il \emph{Numero di Nepero}, espresso solitamente col simbolo ``$e$''.
Si tratta di un numero reale che si calcola come:
\[e = \frac{1}{0!}+\frac{1}{1!}+\frac{1}{2!}+\frac{1}{3!}+\frac{1}{4!}+\frac{1}{5!}+\dots \approx 2,718281828459\]
Sulle calcolatrici solitamente è presente il tasto per calcolare la funzione $y = e^x$. Prova per esercizio a disegnarne il grafico
(poiché $e>1$ il grafico sarà crescente, \dots).
\end{osservazione}



\subsection{Equazioni esponenziali}
\label{subsec:esplog_equazioniesponenziali}

\begin{definizione}[Equazioni esponenziali]{
Chiamiamo \emph{equazione esponenziale} un'equazione nella quale l'incognita compare all'esponente.
}
\end{definizione}

Vediamo alcuni tipi di equazioni esponenziali che si possono risolvere 
abbastanza facilmente.

\subsubsection{Equazioni elementari}
\label{subsubsec:esplog_eq_elementari}

Sono equazioni di forma:  \;\; \fbox{\( a^x=b\)} \quad (con $a>0$)\\[4pt]
Per quanto abbiamo detto precedentemente, essendo \(a\) positiva, anche \(b\) 
dovrà essere un numero positivo perché nessuna potenza con base positiva può 
avere valore negativo.

Per risolvere equazioni di questo tipo dovremo utilizzare una delle 
operazioni inverse della potenza: il \emph{logaritmo} (ulteriori informazioni verranno
date nelle seguenti pagine). Si tratta di un'operazione che 
restituisce il valore dell'esponente, conoscendo la base e il valore della potenza. 
In pratica l'equazione precedente è automaticamente risolta col logaritmo:
\[ a^x=b \quad \Longrightarrow \quad x = \log_a b\]

\begin{esempio} \(5^x=10\)\\[4pt]
Usando il logaritmo si ottiene: \(x=\log_5 10 = 1,4306765580733933\).\\[2pt]
infatti se con la calcolatrice calcolate \(5^{1,4306765580733933}\) 
otterrete proprio~10.
\end{esempio}

Spesso gli esercizi proposti si possono risolvere con un semplice trucco che 
permette di evitare il calcolo del logaritmo. Il metodo nasce dalla seguente osservazione:

\begin{osservazione}
 Se due potenze sono uguali e hanno la stessa base, dovranno per 
forza avere anche lo stesso esponente (perché la funzione esponenziale è 
biunivoca). quindi: 
\[ a^x = a^y \quad \Longrightarrow \quad x=y\]
\end{osservazione}
\begin{esempio}
\(4^x=64\) \\[4pt]Dato che: \(64=4^3\) possiamo scrivere: \(4^x=4^3\) 
e quindi: \(x=3\).
\end{esempio}

\begin{esempio}
\(3^{2x+3}=81\)\\[4pt] Anche in questo caso possiamo riportarci al meccanismo 
utilizzato precedentemente: \(3^{2x+3}=3^4\), quindi 
\({2x+3}=4\). In questo modo abbiamo trasformato un'equazione esponenziale in 
una semplice equazione polinomiale.
\end{esempio}

\subsubsection{Applicazione delle proprietà delle potenze}
\label{subsubsec:esplog_eq_proprpot}

A volte per riportarci al caso elementare è necessario applicare le proprietà delle 
potenze. Rivediamole:

\begin{itemize} %[nosep]% [nosepitem]
 \item \fbox{\(a^m \cdot a^n = a^{m+n}\)}
 Il prodotto di due  potenze che hanno la stessa base è una potenza che ha per 
base la stessa base e per esponente la somma degli esponenti.
 \item \fbox{\(a^m \div a^n = a^{m-n}\)}
 Il quoziente di due  potenze che hanno la stessa base è una potenza che ha per 
base la stessa base e per esponente la differenza degli esponenti.
 \item \fbox{\(\tonda{a^m}^n = a^{m \cdot n}\)}
 La potenza di una potenza è una potenza che ha per base la stessa base e per 
esponente il prodotto degli esponenti.
 \item \fbox{\(a^n \cdot b^n = \tonda{a \cdot b}^n\)}
 Il prodotto di due  potenze che hanno lo stesso esponente è una potenza che ha 
per base il prodotto delle basi e per esponente lo stesso esponente.
 \item \fbox{\(a^n \div b^n = \tonda{a \div b}^n\)}
 Il quoziente di due  potenze che hanno lo stesso esponente è una potenza che 
ha per base il quoziente delle basi e per esponente lo stesso esponente.
\end{itemize}

\begin{esempio}\(7^{x^2} \div 7^{\,5} - \frac{49}{7^{6x}} = 0\)\\[4pt] Usando le proprietà delle potenze 
ci riportiamo ad una situazione nota. 
% \footnote{
% Un rapido richiamo alle cinque proprietà delle potenze:
% % \begin{multicols}{3}
% \begin{enumerate} [nosep]% [nosepitem]
%  \item \(a^m \cdot a^n = a^{m+n}\)
%  \item \(a^m \div a^n = a^{m-n}\)
%  \item \(\tonda{a^m}^n = a^{m \cdot n}\)
%  \item \(a^n \cdot b^n = \tonda{a \cdot b}^n\)
%  \item  \(a^n \div b^n = \tonda{a \div b}^n\)
% \end{enumerate}
% % \end{multicols}
% }
L'equazione precedente è equivalente a: 
\(7^{x^2-5} - 7^{2-6x} = 0\) che può essere riscritta come:
\(7^{x^2-5} = 7^{2-6x}\) e usando i metodi precedenti viene 
trasformata in un'equazione polinomiale facilmente risolvibile:
\[x^2-5 = 2-6x \quad\Rightarrow\quad \dots\quad \Rightarrow \quad x_1=-7 \quad;\quad  x_2=1\]
\end{esempio}

\begin{esempio}
\({5}^{3x} \div 5^2-{2}^{9x-6}=0\)\\[4pt]
Usando le proprietà delle potenze possiamo scrivere: \({5}^{3x-2}-{2}^{9x-6}=0\).\\
Ma a questo punto abbiamo due potenze con basi diverse, perciò i trucchi visti sopra 
non possiamo usarli! Dobbiamo escogitare qualcos'altro! Possiamo 
raccogliere 3 nell'esponente della seconda potenza: in questo modo otteniamo due 
potenze con basi diverse, ma con gli esponenti che si assomigliano:
\[{5}^{3x-2}-{2}^{3\tonda{3x-2}}=0 \quad \Rightarrow \quad {5}^{3x-2}-{8}^{3x-2}=0 \quad 
\Rightarrow \quad {5}^{3x-2}={8}^{3x-2}\]
Dividendo entrambi i membri per l'espressione che si trova a secondo membro:
\(\tonda{\frac{5}{8}}^{3x-2}=1\).\\
Ma ora 1 si può sostituire con \(\tonda{\frac{5}{8}}^{0}\) e quindi:
\(\tonda{\frac{5}{8}}^{3x-2}=\tonda{\frac{5}{8}}^{0} \quad\Rightarrow\quad 3x-2=0 \quad\dots\)
\end{esempio}

\subsubsection{Sostituzione di variabile}
\label{subsubsec:esplog_sostituzione}

A volte un'opportuna sostituzione di variabile rende l'equazione più semplice. 
Si può effettuare la sostituzione, risolvere l'equazione più semplice ottenuta e 
poi ripristinare la variabile originale nelle soluzioni trovate. Qualche esempio può aiutare a capire.

\begin{esempio} 
\(5^{2x} +6 \cdot 5^x -7=0\) \\[4pt]
\textbf{\underline{Sostituzione}}:
ponendo \(5^x=z\) l'equazione diventa: \(z^2 +6z -7=0\), ovvero un'equazione di secondo grado, facile da risolvere:
\(\tonda{z-1} \tonda{z+7}=0 \quad \Rightarrow \quad z_1=+1 \;,\; z_2=-7\) 
 \\[4pt]
\textbf{\underline{Risostituzione}}: al posto di \(z\) mettiamo le soluzioni trovate:

\(z = 1\quad \Rightarrow \quad 5^x=1 \quad \Rightarrow \quad 5^x=5^0\quad \Rightarrow \quad x=0\)

\(z = -7 \quad \Rightarrow \quad 5^x=-7 \quad \Rightarrow \quad \text{Non ha soluzioni reali}\)
\end{esempio}

\begin{esempio} 
\(2^{\frac{4}{3}x} +4 \cdot 2^{\frac{2}{3}x} -32=0\) \\[4pt]
\textbf{\underline{Sostituzione}}:
ponendo \(2^{\frac{2}{3}x}=z\) l'equazione diventa: \(z^2 +4z -32=0\), che può essere
risolta come sopra:
\(\tonda{z+8}\tonda{z-4}=0 \quad \Rightarrow \quad z_1=-8 \;,\; z_2=+4\) 
\\[4pt]
\textbf{\underline{Risostituzione}}: al posto di \(z\) sostituiamo le soluzioni trovate:

\(z = -8\quad \Rightarrow \quad 2^{\frac{2}{3}x}=-8 \quad \Rightarrow \quad \text{Non ha soluzioni reali}\)

\(z = 4\quad \Rightarrow \quad 2^{\frac{2}{3}x}=4 \quad \Rightarrow \quad 2^{\frac{2}{3}x}=2^2
\quad \Rightarrow \quad\dfrac{2}{3}x=2 \quad \Rightarrow \quad x=3\)
\end{esempio}

\subsection{Le disequazioni esponenziali}
\label{subsubsec:esplog_disequazioniesponenziali}

L'altro problema che ci resta da risolvere è la soluzione di disequazioni in 
cui la variabile indipendente si trova ad esponente. 
% Innanzitutto recuperiamo 
% il metodo seguito finora per risolvere le disequazioni:
% 
% \begin{enumerate} [noitemsep]
%  \item Studio del segno:
%  \begin{enumerate} [noitemsep]
%   \item Equazione Associata;
%   \item Funzione Associata;
%  \end{enumerate}
%  \item Individuazione dell'intervallo soluzione:
%  \begin{enumerate} [noitemsep]
%   \item modo Grafico;
%   \item con i Predicati;
%   \item con le Parentesi;
%  \end{enumerate} 
% \end{enumerate}
Si possono utilizzare due metodi equivalenti. Vediamo due esempi:

\begin{esempio}
 \(\tonda{\dfrac{1}{2}}^x \geqslant \dfrac{1}{64}\)
 
\begin{enumerate} [noitemsep]
 \item Studio del segno:
 \begin{itemize} [noitemsep]
  \item Equazione Associata: 
  \(\tonda{\dfrac{1}{2}}^x = \dfrac{1}{64} \quad\Rightarrow \quad\)
   \(\tonda{\dfrac{1}{2}}^x = \tonda{\dfrac{1}{2}}^6 \quad \Rightarrow \quad x= 6\)
 \end{itemize}
\begin{minipage}{.5\textwidth}
\begin{itemize} [noitemsep]
\item Funzione Associata: \(y=\tonda{\dfrac{1}{2}}^x -\dfrac{1}{64}\)
\end{itemize}
 
prima di \(x=6\) la funzione è positiva, dopo questo valore è negativa e non 
può ridiventare positiva dato che continua a calare.
\end{minipage}
 \begin{minipage}{.5\textwidth}
 \begin{inaccessibleblock}[Funzione esponenziale decrescente che taglia 
l'asse~\(x\) nel punto~6 con un segno più prima di questo valore e meno dopo.]
  \grafdiseq{.8}{6}{-4}
 \end{inaccessibleblock}
  \end{minipage}
  
  \vspace{15pt}
  
 \item Individuazione dell'intervallo soluzione:
 dato che la disequazione richiede che la funzione sia maggiore di zero, i 
valori che la rendono tale sono quelli positivi. Quindi la soluzione è:\\[7pt]
\begin{minipage}{.5\textwidth}
 \begin{enumerate} [noitemsep]
  \item modo Grafico (vedi figura a lato)
  \item con i Predicati: \(x \leqslant 6\);
  \item con le Parentesi \(\left]\infty;~6\right]\);
 \end{enumerate}
\end{minipage}
\begin{minipage}{.5\textwidth}
\begin{inaccessibleblock}[un asse delle~\(x\) con evidenziati i punti che nel 
grafico precedente erano segnati come positivi.]
\dissolincl{-6}{6}
 \end{inaccessibleblock}
\end{minipage}
 

\end{enumerate}

\end{esempio}

Prima di procedere facciamo un'osservazione importante: 

\begin{osservazione}
Se una funzione è crescente, da \(f(a)>f(b)\) consegue che \(a>b\).
Se una funzione è decrescente, da \(f(a)>f(b)\) consegue che \(a<b\).
\end{osservazione}
Applicando questa osservazione alle funzioni esponenziali possiamo 
concludere che:

\begin{itemize}
 \item Se $a>1$ allora: \quad $\begin{matrix}
 a^b>a^c \quad \Rightarrow \quad b>c \\
 a^b<a^c \quad \Rightarrow \quad b<c \\
\end{matrix}$
 \item Se $0<a<1$ allora: \quad $\begin{matrix}
 a^b>a^c \quad \Rightarrow \quad b<c \\
 a^b<a^c \quad \Rightarrow \quad b>c \\
\end{matrix}$
\end{itemize}
In pratica, passando dagli esponenziali agli esponenti, nella disequazione è necessario stare attenti
al valore della base: se la base è minore di 1, si cambia il verso alla disequazione. 
Vediamo subito come utilizzare questa osservazione per risolvere l'esercizio precedente:


\begin{esempio}
 \(\tonda{\dfrac{1}{2}}^x \geqslant \dfrac{1}{64}\)\\[4pt]
 Riscriviamo:  \(\tonda{\dfrac{1}{2}}^x \geqslant \dfrac{1}{64} \quad\Rightarrow\quad
    \tonda{\dfrac{1}{2}}^x \geqslant \tonda{\dfrac{1}{2}}^6 \quad\Rightarrow\quad 
    x \leqslant 6\)
\\[6pt]
Se mi ricordo di cambiar verso al predicato, quando necessario, questo metodo 
risulta decisamente più rapido.
\end{esempio}

\section{Logaritmi}
\label{sec:esplog_logaritmi}

Cerchiamo di studiare un pò meglio il \emph{logaritmo}, ovvero quella funzione inversa
dell'esponenziale che ci è servita per calcolare la soluzione nel caso di equazioni esponenziali elementari.

\subsection{Le operazioni inverse e la potenza}
\label{subsec:esplog_operazioni_inverse}

Fin dalle elementari abbiamo imparato che alcune operazioni sono tra loro 
inverse: se ad un numero ne aggiungo un altro e poi lo tolgo ritorno al numero 
di partenza. Quindi l'operazione inversa dell'addizione è la sottrazione:
\[7 + 5 = 12 \quad \sRarrow \quad 12 - 5 = 7 \sand 12 - 7 = 5\]
E fin qui è semplice, ma qual è l'operazione inversa della sottrazione?
Spontaneamente diremmo: l'addizione! Proviamo:
\[15 - 7 = 8 \quad \sRarrow \quad 8 + 7 = 15 \sand 8 + 15 \neq 7\]
C'è qualcosa che non va! Perché i conti tornino dobbiamo scrivere:
\[15 - 7 = 8 \quad \sRarrow \quad 8 + 7 = 15 \sand 15 - 8 = 7\]
L'addizione ha un'operazione inversa, la sottrazione 
ne ha due a seconda se dobbiamo trovare il primo operando (detto minuendo) o il 
secondo (detto sottraendo). Questo comportamento è dovuto al fatto che 
l'addizione è \emph{commutativa} mentre la sottrazione non lo è.

È possibile fare una discorso analogo per la moltiplicazione e per la 
divisione, ma qui vogliamo concentrarci sulla potenza. Se di una potenza 
conosciamo il risultato e l'esponente per calcolare la base possiamo utilizzare 
l'operazione di radice:
\[2^3 = 8 \sRarrow \sqrt[3]{8} = 2 \]
Ma se conosciamo il risultato e la base come facciamo a calcolare l'esponente?
La radice non ci serve in questo caso:
\[2^3 = 8 \sRarrow \sqrt[2]{8} = 3 \]
Il risultato è chiaramente sbagliato. Per risolvere questo problema dobbiamo 
utilizzare una nuova operazione: il \emph{logaritmo}.

\begin{definizione}[Logaritmo]
 Si dice \emph{logaritmo} di un numero in una determinata base l'esponente da 
dare alla base per ottenere il numero:
\[\log_a {b} = c \sLRarrow a^c = b\]
diciamo che \(a\) è la \emph{base} del logaritmo, mentre \(b\) è il suo argomento.
\end{definizione}

È importante tenere presente che il logaritmo è un esponente. 

\subsubsection{Proprietà dei logaritmi}
\label{subsubsec:esplog_proprieta_logaritmi}

Su questa nuova operazione possiamo fare alcune osservazioni:

\begin{itemize}
 \item La base, come per gli esponenziali, dev'essere un numero positivo e diverso da 1, altrimenti si ricade in casistiche strane o prive di utilità.
Quindi $0<a<1 \lor a>1$.
 \item L'argomento dovrà essere per forza un numero positivo, perché non esiste 
nessun esponente che, dato ad una base positiva, possa far ottenere un numero 
negativo. Quindi $b>0$
 \item Nel logaritmo, come nella potenza, non valgono né la proprietà 
associativa né la proprietà commutativa.
 \item Il logaritmo non ha un elemento neutro.
\end{itemize}

\noindent Ma allora nei logaritmi vale qualche proprietà? 
Dalle proprietà delle potenze viste dal punto di vista degli 
esponenti si ricavano tre proprietà dei logaritmi, a cui se ne aggiungono altre molto importanti:
% 
% \begin{enumerate} 
%  \item Da \(a^m \cdot a^n = a^{m+n}\) si ricava:
%  \[log_a {b} + log_a {c} = log_a \tonda{b \cdot c} \]
% \paragraph{Dimostrazione} 
% Poniamo \(a^m = b\) e \(a^n = c\) 
% allora: \(m = log_a {b}\) e \(n = log_a {c}\) possiamo allora costruire la 
% seguente catena di uguaglianze:
% \[log_a {b} + log_a {c} =
% m + n =  
% log_a {a^{m + n}} =
% log_a \tonda{{a^m \cdot a^n}}  =
% log_a \tonda{{b \cdot c}}\]
%  \item Da \(a^m \div a^n = a^{m-n}\) si ricava:
%  \[log_a {b} - log_a {c} = log_a \tonda{\frac{b}{c}} \]
% \paragraph{Dimostrazione} 
% Poniamo \(a^m = b\) e \(a^n = c\) 
% allora: . . . . . . . . . . . e . . . . . . . . . . . 
% possiamo allora costruire la seguente catena di uguaglianze:
% \[\dots\]
%  \item Da \(\tonda{a^m}^n = a^{mn}\) si ricava:
%  \[log_a {b} \cdot log_a {c} = log_a \tonda{{b}^{c}} \]
% \paragraph{Dimostrazione} Scrivila tu sul margine della pagina.
%  \item Da \(a^0 = 1\) si ricava:
%  \[log_a {1} = 0\]
%  \item Combinando le proprietà~4 e~2 si ottiene:
%  \[-log_a {b} = log_a {\frac{1}{b}}\]
%  \item Si può anche dimostrare che:
%  \[-log_a {b} = log_{\frac{1}{a}} b\]
%  \item Un'ultima importante proprietà che ci permette di convertire un 
% logaritmo da una base in un'altra è:
%  \[log_a {b} = \frac{log_c b}{log_c a}\]
% \end{enumerate}

\begin{enumerate} 
 \item Dalla definizione di logaritmo si ricava che:
\fbox{\(a^{log_a(b)} = b\)} e \fbox{\(log_a{a^b} = b\)}
 \item Da \quad \(a^m \cdot a^n = a^{m+n}\) \quad si ricava: \quad
 \fbox{\(log_a {b} + log_a {c} = log_a \tonda{b \cdot c} \)}
\paragraph{Dimostrazione} 
Poniamo \(a^m = b\) e \(a^n = c\) 
allora: \(m = log_a {b}\) e \(n = log_a {c}\) possiamo allora costruire la 
seguente catena di uguaglianze:
\[log_a {b} + log_a {c} =
m + n =  
log_a {a^{m + n}} =
log_a \tonda{{a^m \cdot a^n}}  =
log_a \tonda{{b \cdot c}}\]
 \item Da \quad \(a^m \div a^n = a^{m-n}\) \quad si ricava: \quad
 \fbox{\(log_a {b} - log_a {c} = log_a \tonda{\frac{b}{c}}\)}
\paragraph{Dimostrazione} Simile alla precedente.\\[4pt]
 \item Da \quad \(\tonda{a^m}^n = a^{mn}\) \quad si ricava: \quad
 \fbox{\(c \cdot log_a {b} = log_a {b^c} \)}
\paragraph{Dimostrazione} 
Poniamo \(a^m = b\)  
allora: \(m = log_a {b}\) possiamo allora costruire la 
seguente catena di uguaglianze:
\[c \cdot log_a {b} =
c \cdot m = 
log_a {a^{m \cdot c}} =
log_a {\tonda{a^{m}}^c} = 
log_a {b^c}\]
 \item Da \quad \(a^0 = 1\) \quad si ricava: \quad
 \fbox{\(log_a {1} = 0\)}
 \item Combinando le proprietà~4 e~2 si ottiene: \quad
 \fbox{\(log_a {\frac{1}{b}} =-log_a {b}\)}
 \item Si può anche dimostrare che: \quad
 \fbox{\(log_{\frac{1}{a}} b=-log_a {b}\)}
 \item Un'ultima importante proprietà che ci permette di convertire un 
logaritmo da una base in un'altra è: \quad
 \fbox{\(log_a {b} = \frac{log_c b}{log_c a}\)}
\end{enumerate}

\noindent
Dato che è sempre possibile cambiare base a un logaritmo, spesso si 
usano logaritmi in particolari basi. Quelle più diffuse e presenti in tutte le 
calcolatrici scientifiche sono: 
\begin{itemize}
 \item 
la base~10 che dà origine ai logaritmi decimali anche scritti: \(Log\);
 \item 
la base $e$ che dà origine ai logaritmi naturali anche scritti: \(\ln\).
\end{itemize}

\begin{esempio}
 Utilizzando la definizione di logaritmo verifica che: \(\log_2 32 = 5\)
\end{esempio}

\begin{esempio}
 Utilizzando la definizione di logaritmo verifica che: \(Log~1000 = 3\)
\end{esempio}

\begin{esempio}
 Usando la calcolatrice verifica che: \(Log~4 = 0,602059991\)
\end{esempio}

\begin{esempio}
 Usando la calcolatrice verifica che: \(\ln 4 = 1,386294361\)
\end{esempio}

\begin{esempio}
 Usando la calcolatrice verifica che: \(\ln 7 + \ln 4 = \ln 28\)
\end{esempio}

\begin{esempio}
 Usando la calcolatrice verifica che: \(Log~43 = \dfrac{\ln 43}{\ln 10}\)
\end{esempio}

\subsubsection{La funzione logaritmo}
\label{subsubsec:esplog_funzione_logaritmo}
\noindent
\begin{minipage}[]{.60\textwidth}
Prima di disegnare per punti la funzione logaritmo, riesaminiamo una 
trasformazione geometrica: la simmetria rispetto alla bisettrice del primo e terzo quadrante (ovvero
la retta $y=x$).\\[7pt]
Confrontando le coordinate di $A$ e $A'$, $B$ e $B'$, si può osservare che per 
passare da un punto al suo simmetrico basta semplicemente scambiare l'ascissa 
con l'ordinata, ovvero:
\[ x'=y \quad ; \quad  y'=x\]


\end{minipage} \hspace{.04\textwidth}
\begin{minipage}[]{.35\textwidth}
\begin{center}
\begin{inaccessibleblock}[Bisettrice del primo e terzo quadrante e 
alcuni punti simmetrici rispetto a questa retta]
  \simmetriayx
%   \caption{...e i corrispondenti punti.} \label{fig:potdue0}
\end{inaccessibleblock}
\end{center}
\end{minipage} 
% \vspace{12pt}

\begin{minipage}{.35\textwidth}
\begin{center}
\begin{inaccessibleblock}[Grafico di una funzione esponenziale e 
il suo simmetrico rispetto a y=x]
  \graficologaritmica
%   \caption{...e i corrispondenti punti.} \label{fig:potdue0}
\end{inaccessibleblock}
\end{center}
\end{minipage} \qquad
\begin{minipage}{.6\textwidth}
Ma se in una funzione esponenziale: \(y=a^x\) scambiamo $x$ e $y$ 
otteniamo: \(x=a^y\) da cui, esplicitando $y$, si ha: \(y=\log_a x\).\\[7pt] Si può quindi 
osservare che la funzione logaritmo è la funzione inversa della funzione 
esponenziale: il suo grafico si ottiene quindi applicando alla funzione 
esponenziale la simmetria di asse: \(y=x\).
\end{minipage}

\begin{minipage}{.52\textwidth}
 Possiamo fare alcune osservazioni:
\begin{enumerate}
 \item Il dominio è l'intervallo: \(\intervaa{0}{\infty}\)
 \item Se la base è maggiore di~1 la funzione è crescente, 
 se è compresa tra~0 e~1 la funzione è decrescente.
 \item Il grafico interseca l'asse~\(x\) nel punto $\punto{1}{0}$.
 \item Si può trovare la base del logaritmo individuando il punto di 
 ordinata~1 (o~\(-1\)).
 \item Quando x è un infinitesimo positivo, y è un infinito (negativo o 
positivo).
 \item Quando x è un infinito positivo, y è un infinito (positivo o negativo).
\end{enumerate}
\end{minipage} \qquad 
\begin{minipage}{.45\textwidth}
 \begin{inaccessibleblock}[Grafici di funzioni logaritmiche con basi diverse.]
  \logdiversebasi
%   \caption{Funzioni logaritmiche con diverse basi.} \label{fig:diversebasi}
\end{inaccessibleblock}
 % \caption{Funzioni logaritmiche con diverse basi.} \label{fig:log_diversebasi}
\end{minipage}

\vspace{.5cm}

Al variare della base cambia il grafico della funzione: 
se la base si avvicina a 1, la funzione si avvicina 
alla retta \(x=1\). 
Se la base si avvicina a 0 o all'infinito, la funzione si avvicina agli assi.

\subsection{Le equazioni logaritmiche}
\label{subsec:esplog_equazionilogaritmiche}

\begin{definizione}[Equazioni logaritmiche]{
Chiamiamo \emph{equazione logaritmica} un'equazione in cui l'incognita compare 
nell'argomento di un logaritmo.
}
\end{definizione}

Vediamo alcuni tipi di equazioni logaritmiche che si possono risolvere 
abbastanza facilmente.

\subsubsection{Equazioni logaritmiche elementari}
\label{subsubsec:esplog_eq_log_elementari}

\noindent Sono le equazioni nella forma: \quad \fbox{\(\log_a{f(x)} = b\)} \\[7pt]
Questo tipo di equazione può essere risolta tenendo semplicemente conto della definizione stessa di logaritmo:
\[\log_a{f(x)} = b \quad \Rightarrow \quad f(x)=a^b\]
\begin{esempio}
 \(\log_2{x} = 8\)\\[4pt] 
 Ricordando la definizione di logaritmo posso risolvere facilmente: \; \(x = 2^8 = 256\)
\end{esempio}
\begin{esempio}
 \(\log_{3,6}{x} = 4\)\\[4pt]
 Anche in questo caso è sufficiente calcolare: \quad \( x = 3,6^{\,4} = 167,9616\)
\end{esempio}

In realtà, come per le esponenziali, anche per le equazioni logaritmiche può essere usato qualche piccolo
accorgimento che permette di semplificare la struttura dell'equazione.
\[\log_a{f(x)} = \log_a{g(x)} \sRarrow f(x) = g(x)\]
Infatti se sono uguali le basi e i logaritmi, allora saranno equivalenti gli argomenti, 
dato che anche la funzione logaritmo è biunivoca.

\begin{esempio}
 Riprendiamo il primo esempio: \(\log_2{x} = 8\), possiamo osservare che 
 \(8 = \log_2{256}\) quindi, sostituendo, 
 otteniamo: \(\log_2{x} = \log_2{256}\). Ma se due logaritmi sono uguali e 
hanno la stessa base, i loro argomenti dovranno essere equivalenti, quindi:
\(x = 256\)
\end{esempio}

\begin{esempio}
 \(\log_{3}\tonda{2x-7} = 2 \)\\[4pt]
 Sfruttando la stessa proprietà: 
 \(\log_{3}\tonda{2x-7} = \log_{3}9 \sRarrow  
2x -7 = 9 \sRarrow x = 8\)
\\[4pt] 
Sostituendo nell'equazione di partenza l'incognita con il valore 8 
possiamo verificare che si tratta proprio della soluzione dell'equazione:
\(\log_{3}\tonda{2 \cdot 8-7} = \log_{3} 9 = 2\)
\end{esempio}

%\(\log_{}\tonda{} \)
\begin{esempio}
 \(\ln\tonda{5x +7} = \ln\tonda{9x +15}\)
 \\[4pt]
 Uguali i logaritmi, uguali le basi, quindi:
 \(5x +7 = 9x +15  \sRarrow x = -2\)
 \\[4pt]
Sostituendo l'incognita otteniamo:
 \(\ln\tonda{5 \cdot \tonda{-2} +7} = 
   \ln\tonda{9 \cdot \tonda{-2} +15} \sRarrow 
   \ln\tonda{-3} = \ln\tonda{-3}\)
   \\[4pt]
 Ma il logaritmo di un numero negativo non dà un risultato reale! Prova con la calcolatrice.
\end{esempio}

L'ultimo esempio ci mostra come le cose siano un po' più complicate: 
l'operazione di passaggio dall'uguaglianza dei logaritmi all'uguaglianza degli 
argomenti fa perdere delle informazioni (come quando in un'equazione fratta si 
eliminano i denominatori uguali). 
Quando eliminiamo i logaritmi scompare l'informazione che ``certe espressioni 
erano argomenti del logaritmo'' e quindi che l'espressione originaria ha valore 
reale solo se questa espressione è maggiore di zero. 
Scrivere questa condizione si traduce nel classico: ``porre le condizioni di esistenza''.
Vediamo allora come risolvere l'esercizio precedente senza perdere informazioni.

\begin{esempio}
\(\ln\tonda{5x +7} = \ln\tonda{9x +15}\)
 \\[4pt]
 Uguali i logaritmi, uguali le basi, quindi:
 \\[4pt]
\(\sistema{5x +7 >0 & \quad\text{(Condizione d'esistenza del $1^\circ$ logaritmo)}\\ 
           9x +15 > 0 & \quad \text{(Condizione d'esistenza del $2^\circ$ logaritmo)}\\
           5x +7 = 9x +15 & \quad \text{(Equazione ottenuta eliminando i logaritmi)}}\)
 \\[4pt]
\(\sistema{x > -\frac{7}{5} \\[3pt] x > -\frac{5}{3} \\[3pt] 5x - 9x = +15 -7} \quad \Longrightarrow \quad
\sistema{x > -\frac{7}{5} \\[3pt] x > -\frac{5}{3} \\[3pt] x = -2}  \quad \Longrightarrow \quad
\sistema{x > -\frac{7}{5} \\x = -2}\)
 \\[4pt]
L'ultimo passaggio è giustificato dal fatto che, se un numero è più grande di \(-\frac{7}{5} \approx −1,4\) 
sarà senz'altro più grande anche di~\(-\frac{5}{3} \approx −1,67\).
Ma ora questo sistema non ha soluzioni, perché se \(x=-2\), di certo 
non è più grande di \(-\frac{7}{5}\).
\end{esempio}


\subsubsection{Equazioni logaritmiche con l'uso delle proprietà}
\label{subsubsec:esplog_eq_log_proprieta}

\noindent Supponiamo di dover risolvere un'equazione di questo tipo:
\[\log_2\tonda{5x -7} + \log_2{2x}= \log_2\tonda{2x -4}\]
Non possiamo di certo far finta che non ci siano i logaritmi e scrivere:
\(5x -7 + 2x = 2x -4\)\\
Per convincersi dell'errore, è sufficiente guardare l'esempio seguente:
\[\log_2 32 -\log_2 4 = \log_2 8 \quad \nRightarrow\quad 32 -4 = 8 \quad \Rightarrow\quad 28 = 8\]
Per risolvere queste equazioni bisogna:
\begin{enumerate}
 \item ricorrere alle proprietà dei logaritmi presentate in precedenza;
 \item considerare la condizione di esistenza del logaritmo 
 (argomento maggiore di zero).
\end{enumerate}

\begin{esempio}
 \(\ln\tonda{3x - 1} + \ln\tonda{2x +2} = \ln 5 + \ln\tonda{x^2 +2}\)
 \\[4pt]
Ricordando la prima proprietà, l'equazione è equivalente a:
\[\cancel{\ln}\tonda{\tonda{3x - 1} \tonda{2x +2}}= \cancel{\ln}\tonda{5\tonda{x^2 +2}} \quad \Rightarrow\quad
x^2 +4x -12 = 0\]
  
Tenendo conto anche delle condizioni di esistenza dei vari logaritmi, 
l'equazione logaritmica precedente è equivalente a:
\\[4pt]
\(\sistema{
3x - 1 > 0 \\
2x +2 > 0 \\
5 > 0 \quad \text {(Sempre vera)}\\
x^2 +2 > 0 \quad \text {(Sempre vera)}\\
x^2 +4x -12 = 0} \quad \Rightarrow \quad \sistema{
x > \frac{1}{3} \\
x > -1 \\
\tonda{x +6} \tonda{x -2} = 0} \quad \Rightarrow \quad \sistema{
x > \frac{1}{3} \\[3pt]
x = -6 \lor x=2
}\)\\[4pt]
Abbiamo quindi trovato due soluzioni: in particolare, guardando la disequazione \(x>\frac{1}{3}\),
possiamo concludere che:
\[x_1 = -6  \text{ (Soluzione Non Accettabile)} \quad;\quad 
  x_2 = +2 \text{ (Soluzione Accettabile)}\]
\end{esempio}

\begin{esempio}
 \(\log_2{2x} + \log_2\tonda{5x -7} = \log_2\tonda{2x -4}\)
 \\[4pt]
Come sopra, usando le proprietà dei logaritmi:
\(\quad \log_2\tonda{2x \tonda{5x -7}}= \log_2\tonda{2x -4}\)
  \\[4pt]
Eliminando i logaritmi ed aggiungendo le condizioni d'esistenza:
\\[4pt]
\(\sistema{
2x > 0 \\
5x - 7 > 0 \\
2x - 4 > 0 \\
10x^2 - 14x - 2x + 4 = 0} \quad \Rightarrow \quad \sistema{
x > 0 \\
x > \frac{7}{5} \\
x > 2 \\
10x^2 - 16x + 4 = 0} \quad \Rightarrow \quad\sistema{
x > 2 \\
5x^2 - 8x + 2 = 0}\)
\\[6pt]
Risolviamo l'equazione di secondo grado: 
\(x_{1,2} = \frac{4 \pm \sqrt{16-10}}{5} = \frac{4 \pm \sqrt{6}}{5}\)
\\[7pt]
Tenendo conto della condizione $x > 2$ possiamo concludere che: \[x_1 = \frac{4 - \sqrt{6}}{5} \approx 0,310102
\quad\text{(Soluzione Non Accettabile)} \] 
\[x_2 = \frac{4 + \sqrt{6}}{5} \approx 1,289898
 \quad \text{(Soluzione Non Accettabile)}\]
\end{esempio}

% \end{comment}

\subsection{Le disequazioni logaritmiche}
\label{subsubsec:esplog_disequazionilogaritmiche}

Ricordiamo che, come la funzione esponenziale, anche quella logaritmica è 
crescente se la base è maggiore di 1 e decrescente se la base è compresa tra 
zero e uno.
Le disequazioni logaritmiche si possono quindi risolvere in modo analogo a 
quelle esponenziali.

\begin{esempio}
 \(\log_{\frac{2}{3}} \tonda{4x -6} \leqslant \log_{\frac{2}{3}} \tonda{x -3}\)
\\[4pt]
Eliminando i logaritmi, tenendo conto delle condizioni di esistenza, e cambiando verso alla disequazione, in quanto la
base dei logaritmi è minore di uno:
\\[4pt]
\(\sistema{
4x -6 > 0 \\
x -3 > 0 \\
4x -6 \geqslant x -3} \quad \Rightarrow \quad \sistema{
x > \frac{3}{2} \\
x > 3 \\
3x -3 \geqslant 0} \quad \Rightarrow \quad  \sistema{
x > 3 \\
x \geqslant 1}\quad \Rightarrow \quad x > 3\) 

\end{esempio}

\begin{esempio}
\(\ln \tonda{-7x+2} - \ln \tonda{x +1} \leqslant 0\)
 \\[4pt]
Spostando il secondo logaritmo a destra, la disequazione diventa:
\(\ln \tonda{-7x+2} \leqslant \ln \tonda{x +1}\)
\\[4pt]
Dato che la base è maggiore di zero, il sistema risolutivo è:
\\[4pt]
\(\sistema{
-7x+2 > 0 \\
x +1 > 0 \\
-7x+2 \leqslant x +1}\quad \Rightarrow \quad \sistema{
x < \frac{2}{7} \\
x > -1 \\
-8x \leqslant -1}\quad \Rightarrow \quad \sistema{
-1 < x < \frac{2}{7} \\
x \geqslant \frac{1}{8}}\)
\\[7pt]
\noindent
La soluzione grafica del sistema è:
\\[5pt]
\begin{minipage}{.5\textwidth}
\noindent
\begin{inaccessibleblock}[Soluzione grafica di un sistema di disequazioni.]
  \dissistemaa
\end{inaccessibleblock}
\end{minipage}\qquad
\begin{minipage}{.4\textwidth}
Per cui la soluzione, descritta per intervalli e con i predicati, è:
\[\intervca{\frac{1}{8}}{\frac{2}{7}} \qquad ; \qquad 
  \frac{1}{8} \leqslant x < \frac{2}{7}\]
\end{minipage}





\end{esempio}

% TODO Un esempio sensato in cui usare tutto l'ambaradan dei grafi!
% \begin{esempio}
%  \(\ln \tonda{-7x+2} - \ln \tonda{x +1} \leqslant 0\)
%  
% Tenendo presente che \(0 = ln 1\) e applicando la seconda proprietà 
% dei logaritmi, l'equazione precedente diventa:
% 
% \(\ln \dfrac{-7x+2}{x +1} \leqslant \ln 1\)
% 
% Dato che la base è maggiore di zero:
% 
% \(\sistema{
% -7x+2 > 0 \\
% x +1 > 0 \\
% \dfrac{-7x+2}{x +1} \leqslant 1}\)
% 
% E riducendo:
% 
% \(\sistema{
% x < \frac{2}{7} \\
% x > -1 \\
% \dfrac{-7x+2}{x +1} - \dfrac{x +1}{x +1} \leqslant 0}\)
% 
% \(\sistema{
% -1 < x < \frac{2}{7} \\
% \dfrac{-8x+1}{x +1} \leqslant 0}\)
% 
% \noindent
% \begin{minipage}{.48\textwidth}
% \vspace*{-12pt}
% Soluzione grafica dell'ultima disequazione:
% 
% \begin{inaccessibleblock}[Studio del segno e soluzione della disequazione
% (-8x + 1)/(x +1).]
%   \disfratta
% \end{inaccessibleblock}
% \end{minipage}
% \hspace{.04\textwidth}
% \begin{minipage}{.48\textwidth}
% Soluzione grafica del sistema:
% 
% \begin{inaccessibleblock}[Soluzione grafica di un sistema di disequazioni.]
%   \dissistema
% \end{inaccessibleblock}
% \end{minipage}
% 
% Per cui la soluzione della disequazione logaritmica è:
% 
% \[\intervca{\frac{1}{8}}{\frac{2}{7}} \qquad \text{ o } \qquad 
%   \frac{1}{8} \leqslant x < \frac{2}{7}\]
% 
% \end{esempio}







































