% (c) 2014 Daniele Zambelli - daniele.zambelli@gmail.com
% 
% Tutti i grafici per il capitolo relativo alle parabole
%

\newcommand{\puntia}{% 
    % Alcuni punti di 2^x.
    \disegno{
    \rcom{-5}{+5}{0}{9}{gray!50, very thin, step=1}
    \foreach \pi in {
    (-4, 0.0625), (-3, 0.125), (-2, 0.25), (-1, 0.5), 
    (0, 1), (1, 2), (2, 4), (3, 8)}
    \filldraw [brown!50!black] \pi circle (2pt);
    }
}

\newcommand{\puntib}{% 
    % Altri punti di una parabola.
    \begin{tikzpicture}[x=20mm, y=20mm, smooth]
    \rcom{-1}{+1}{0}{2}{gray!50, very thin, step=0.5}
    \foreach \pi in {
    (-1, 0.5), (0, 1), (1, 2), (-0.75,0.5946), (-0.25,0.8409), (-0.5, 0.7071), 
(0.5, 1.41421),
    (0.25,1.1892), (0.75,1.6818)
    }
    \filldraw [brown!50!black] \pi circle (2pt);
    \end{tikzpicture}
}

\newcommand{\graficoesponenziale}{% 
    % Grafico di un trinomio di secondo grado.
    \disegno{\graficoxy{-5}{+5}{0}{+11}{brown!50!black}{2**x}
%     \rcom{-5}{+5}{0}{11}{gray!50, very thin, step=1}
%     \tkzInit[xmin=-5,xmax=+5,ymin=-0.3,ymax=11]
%     \tkzFct[domain=-5:4, ultra thick, color=brown!50!black]{2**x}
    }
}

\newcommand{\puntimenodue}{% 
    % Alcuni punti di (-2)^x.
    \disegno{
    \rcom{-5}{+3}{-3}{5}{gray!50, very thin, step=1}
    \foreach \pi in {
    (-5, -0.03125), (-4, 0.0625), (-3, -0.125), (-2, 0.25), (-1, -0.5), 
    (0, 1), (1, -2), (2, 4)}
    \filldraw [brown!50!black] \pi circle (2pt);
    }
}

\newcommand{\espdiversebasi}{% 
    % Esponenziali con basi diverse.
    \disegno{
    \rcom{-10}{+10}{-1}{10}{gray!50, very thin, step=1}
    \begin{scope}[ultra thick, color=brown!50!black]
    \tkzInit[xmin=-10.3,xmax=+10.3,ymin=-0.3,ymax=+10.3]
    \tkzFct[domain=-10.3:+3]{3**x}
    \tkzFct[domain=-10.3:+4]{2**x}
    \tkzFct[domain=-10.3:+6]{(3./2)**x}
    \tkzFct[domain=-10.3:+10.3]{(5./4)**x}
    \foreach \pi in {(1, 3), (1, 2), (1, 3./2), (1, 5./4)}
      \filldraw \pi circle (1.2pt);
    \begin{scope}[color=Green!50!black]
    \tkzFct[domain=-3:+10.3]{(1./3)**x}
    \tkzFct[domain=-4:+10.3]{(1./2)**x}
    \tkzFct[domain=-6:+10.3]{(2./3)**x}
    \tkzFct[domain=-10.3:+10.3]{(4./5)**x}
    \foreach \pi in {(-1, 3), (-1, 2), (-1, 3./2), (-1, 5./4)}
      \filldraw \pi circle (1.2pt);
    \end{scope}
    \end{scope}
    \begin{scope}[color=black]
    \draw (-9.5, 7.5) node{a} (-5.2, 9.5) node{b} 
          (-2.8, 9.5) node{c} (-1.5, 9.5) node{d}; 
    \draw (9.5, 7.5) node{h} (5.2, 9.5) node{g} 
          (2.8, 9.5) node{f} (1.5, 9.5) node{e};
    \end{scope}
    }
}

\newcommand{\altrebasi}{% 
    % Esponenziali con altre basi.
    \disegno[4.5]{
    \rcom{-5}{+5}{-1}{10}{gray!50, very thin, step=1}
    \tkzInit[xmin=-5.3,xmax=+5.3,ymin=-1.3,ymax=+10.3]
    \tkzFct[domain=-5.3:+3, ultra thick, color=Red!30!black]{pi**x}
    \node at (3.5, 9) {$y=\pi^x$};
    \tkzFct[domain=-5.3:+5.3, ultra thick, color=Blue!30!black]
           {sqrt(2)**x}
    \node at (4, 2) {$y=\sqrt{2}^x$};
    \tkzFct[domain=-3:+5.3, ultra thick, color=Green!30!black]{(1./9)**x}
    \node at (-3, 8) {$y=\tonda{\frac{1}{9}}^x$};
    }
}

\newcommand{\pendenzae}{% 
    % Esponenziali con basi diverse.
    \disegno{
    \rcom{-5}{+5}{0}{10}{gray!50, very thin, step=1}
    \tkzInit[xmin=-5.3,xmax=+5.3,ymin=-0.3,ymax=+10.3]
    \tkzFct[domain=-5.3:+3, ultra thick, color=brown!50!black]{exp(x)}
    \begin{scope}[thick, color=Green!50!black]
    \tkzFct[domain=-5.3:+5.3, color=Green!50!black]{x+1}
    \tkzFct[domain=-5.3:+5.3, color=Green!50!black]{exp(1)*x}
    \tkzFct[domain=-5.3:+5.3, color=Green!50!black]{6*x - 6*log(6)+6}
    \filldraw (0, 1) circle (1.2pt) node [left] {$P_0$};
    \filldraw (1, 2.7818) circle (1.2pt) node [left] {$P_1$};
    \filldraw (1.792, 6) circle (1.2pt) node [left] {$P_2$};
    \end{scope}
    }
}

\newcommand{\areasottesa}{% 
    % Esponenziali con basi diverse.
    \disegno{
    \begin{scope}
      \clip (-5, 0) rectangle (2, 12);
      \fill [top color=green!40!black!30, bottom color=green!60!black!20] 
            (-5, 0) .. controls (-2, 0.1) and (-1, 0) .. 
            (0, 1) .. controls (0.5, 1.5) and (1.4, 2.5) .. (2, 7.39) --
            (2, 0) -- cycle;
    \end{scope}
 
    \rcom{-5}{+5}{0}{10}{gray!50, very thin, step=1}
    \tkzInit[xmin=-5.3,xmax=+5.3,ymin=-0.3,ymax=+10.3]
    \tkzFct[domain=-5.3:+3, ultra thick, color=brown!50!black]{exp(x)}
    \draw  (0, 7.389) node [left] {$e^2$} -- 
           (2, 7.389) -- (2, 0) node [below] {$2$};
    \node at (1, .5) {$A=e^2$};
    }
}

\newcommand{\grafdiseq}[3]{% 
    % Grafico di un trinomio di secondo grado.
    \def \base{#1}
    \def \lab{#2}
    \def \pospiu{#3}
    \disegno{\assex{-6}{+6}{0}
    \tkzInit[xmin=-6.3,xmax=+6.3,ymin=-1.3,ymax=+3.3]
    \tkzFct[domain=-6:+6, ultra thick, color=brown!50!black]{(\base)**x-1}
    \draw [thick, color=Blue!50!black] (0, -.2) -- (0, +.2);
    \node [above, color=Black] at (0, +.2) {\lab};
    \node [above, color=Black] at (-\pospiu, 0) {\(-\)};
    \node [above, color=Black] at (+\pospiu, 0) {\(+\)};
    }
}

\newcommand{\dissolincl}[2]{% 
    % Grafico di un trinomio di secondo grado.
    \def \finoa{#1}
    \def \lab{#2}
    \disegno{\coordinate (a) at (0, 0);
    \coordinate (b) at (\finoa, 0);
    \assex{-6}{+6}{0}
    \node [above, color=Black] at (0, +.2) {\lab};
    \draw [-, decorate, decoration=snake, blue, thick] (a) -- (b);
    \draw[fill, blue] (a) circle (3pt);
    }
}

\newcommand{\simmetriayx}{% 
    % Simmetria rispetto a y=x.
    \disegno{
    \rcom{-4}{+4}{-4}{+4}{gray!50, very thin, step=1}
    \tkzInit[xmin=-4,xmax=+4,ymin=-4,ymax=+4]
    \tkzFct[domain=-4:+4, ultra thick, color=red!50!black]{x}
    \foreach \pi/\n in {(-3, -2)/\(A\), (2, 3)/\(B\),(-1, 3)/\(C\)}
  \filldraw [brown!50!black] \pi circle (2pt) node [below left] {\n};
    \foreach \pi/\n in {(-2, -3)/\(A'\), (3, 2)/\(B'\), (3, -1)/\(C'\)}
    \filldraw [Blue!50!black] \pi circle (2pt) node [below right] {\n};
    }
}

\newcommand{\graficologaritmica}{% 
    % Grafico di un trinomio di secondo grado.
    \disegno{
    \rcom{-3}{+6}{-3}{+6}{gray!50, very thin, step=1}
    \tkzInit[xmin=-3,xmax=+6,ymin=-3,ymax=+6]
    \tkzFct[domain=-10:+10.3, ultra thick, color=red!50!black, dotted]{x}
    \tkzFct[domain=-4.3:+4, ultra thick, color=Blue!50]{2**x}
    \tkzFct[domain=0:+8.3, ultra thick, color=brown!50!black]{log(x)/log(2)}
    }
}

\newcommand{\logdiversebasi}{% 
    % Esponenziali con basi diverse.
    \disegno{
    \rcom{-1}{+10}{-7}{7}{gray!50, very thin, step=1}
    \begin{scope}[ultra thick, color=brown!50!black]
    \tkzInit[xmin=-1,xmax=+10,ymin=-7,ymax=+7]
    \tkzFct[domain=0:+10.3]{log(x)/log(3)}
    \tkzFct[domain=0:+10.3]{log(x)/log(2)}
    \tkzFct[domain=0:+10.3]{log(x)/log(3./2)}
    \tkzFct[domain=0:+10.3]{log(x)/log(5./4)}
    \foreach \pi in {(3, 1), (2, 1), (3./2, 1), (5./4, 1)}
      \filldraw \pi circle (1.2pt);
    \begin{scope}[color=Green!50!black]
    \tkzFct[domain=0:+10.3]{log(x)/log(1./3)}
    \tkzFct[domain=0:+10.3]{log(x)/log(1./2)}
    \tkzFct[domain=0:+10.3]{log(x)/log(2./3)}
    \tkzFct[domain=0:+10.3]{log(x)/log(4./5)}
    \foreach \pi in {(3, -1), (2, -1), (3./2, -1), (5./4, -1)}
      \filldraw \pi circle (1.2pt);
    \end{scope}
    \end{scope}
    \begin{scope}[color=black]
    \draw (4, 5.4) node{a} (7.5, 5.7) node{b} 
          (9.5, 2.8) node{c} (9.5, 1.5) node{d}; 
    \draw (9.5, -1.5) node{e} (9.5, -2.8) node{f} 
          (7.5, -5.7) node{g} (4, -5.4) node{h};
    \end{scope}
    }
}

\newcommand{\dissistemaa}{% 
    % Sistema di disequazioni -1 < x < 2/7 e x>= 1/8
  \disegno{
    \sistematreassitrepunti{5}{1.2}
                           {\(D_1\)}{\(D_2\)}{\(D_1 \wedge D_2\)}
                           {\(-1\)}{\(\frac{1}{8}\)}{\(\frac{2}{7}\)}
    \def \posx{5/2}
    \begin{scope}[blue,thick]
    \def \posy{0}
      \draw [-,decorate, decoration=snake] (-\posx, \posy) -- (+\posx, \posy);
      \draw[fill=white] (-\posx, \posy) circle (2pt);
      \draw[fill=white] (+\posx, \posy) circle (2pt);
    \def \posy{-1.2}
%       \draw [-,decorate, decoration=snake] (-5, \posy) -- (-\posx, \posy);
%       \draw[fill=white] (-\posx, \posy) circle (2pt);
      \draw [-,decorate, decoration=snake] (0, \posy) -- (+5, \posy);
      \draw[fill] (0, \posy) circle (2pt);
    \def \posy{-2.4}
      \draw [-,decorate, decoration=snake] (0, \posy) -- (+\posx, \posy);
      \draw[fill] (0, \posy) circle (2pt);
      \draw[fill=white] (+\posx, \posy) circle (2pt);
    \end{scope}

  }
}


% \newcommand{\disfratta}{% 
%   % Studio dei segni di una frazione.
%   \disegno{
%     \segnitreassiduepunti{5}{1.2}
%                          {\(8x-1\)}{\(x+1\)}{\(f(x)\)}
%                          {\(-1\)}{\(\frac{1}{8}\)}
%     \def \posx{3}
%     \def \posy{.5}
%     \draw (-\posx, \posy) node {\(+\)} (0, \posy) node {\(+\)} 
%     (+\posx, \posy) node {\(-\)};
%     \cerchietto{5/3}{\posy}
%     \def \posy{-.7}
%     \draw (-\posx, \posy) node {\(-\)} (0, \posy) node {\(+\)} 
%     (+\posx, \posy) node {\(+\)};
%     \crocetta{-5/3}{\posy}
%     \def \posy{-1.9}
%     \draw (-\posx, \posy) node {\(-\)} (0, \posy) node {\(+\)} 
%     (+\posx, \posy) node {\(+\)};
%     \crocetta{-5/3}{\posy}
%     \cerchietto{5/3}{\posy}
%     \def \posy{-2.4}
%     \coordinate (a) at (-5, \posy);
%     \coordinate (b) at (-5/3, \posy);
%     \coordinate (c) at (+5/3, \posy);
%     \coordinate (d) at (+5, \posy);
%     \begin{scope}[blue,thick]
%       \draw [-,decorate, decoration=snake] (a) -- (b);
%       \draw[fill=white] (b) circle (2pt);
%       \draw [-,decorate, decoration=snake] (c) -- (d);
%       \draw[fill] (c) circle (2pt);
%     \end{scope}
% 
%   }
% }
% 
% \newcommand{\dissistema}{% 
%     % Sistema di disequazioni -1 < x < 2/7 e ( x<-1 o x>= 1/8)
%   \disegno{
%     \sistematreassitrepunti{5}{1.2}
%                            {\(D_1\)}{\(D_2\)}{\(D_1 \wedge D_2\)}
%                            {\(-1\)}{\(\frac{1}{8}\)}{\(\frac{2}{7}\)}
%     \def \posx{5/2}
%     \begin{scope}[blue,thick]
%     \def \posy{0}
%       \draw [-,decorate, decoration=snake] (-\posx, \posy) -- (+\posx, \posy);
%       \draw[fill=white] (-\posx, \posy) circle (2pt);
%       \draw[fill=white] (+\posx, \posy) circle (2pt);
%     \def \posy{-1.2}
%       \draw [-,decorate, decoration=snake] (-5, \posy) -- (-\posx, \posy);
%       \draw[fill=white] (-\posx, \posy) circle (2pt);
%       \draw [-,decorate, decoration=snake] (0, \posy) -- (+5, \posy);
%       \draw[fill] (0, \posy) circle (2pt);
%     \def \posy{-2.4}
%       \draw [-,decorate, decoration=snake] (0, \posy) -- (+\posx, \posy);
%       \draw[fill] (0, \posy) circle (2pt);
%       \draw[fill=white] (+\posx, \posy) circle (2pt);
%     \end{scope}
% 
%   }
% }

