%%%%%%%%%%%%%%%%%%%%%%%%%%%%%%%%%%%%%%%%%%%%%%%%%%%%%%%%%%%%%%%%%%%%
%%        Matematica dolce
%%%%%%%%%%%%%%%%%%%%%%%%%%%%%%%%%%%%%%%%%%%%%%%%%%%%%%%%%%%%%%%%%%%%
%% Copyright 2016 Daniele Zambelli
%------------------------------
%% intestazioni.tex
%------------------------------
%
% This work may be distributed and/or modified under the
% conditions of the LaTeX Project Public License, either version 1.3
% of this license or (at your option) any later version.
% The latest version of this license is in
%   http://www.latex-project.org/lppl.txt
% and version 1.3 or later is part of all distributions of LaTeX
% version 2005/12/01 or later.
%
% This work has the LPPL maintenance status `maintained'.
% 
% The Current Maintainer of this work is 
% Dimitrios Vrettos - d.vrettos@gmail.com
%
% This work consists of the files:
%  -  algebra1_light_1ed.tex (this file)
%  -  Makefile
%  -  README
%  -  the part of code of the files under chap/, img/  and lbr/ directories
%%%%%%%%%%%%%%%%%%%%%%%%%%%%%%%%%%%%%%%%%%%%%%%%%%%%%%%%%%%%%%%%%%%%
\documentclass[10pt,a4paper,openright]{matc3mem}
% \documentclass[10pt,a4paper,openright,gray]{matc3mem}
%========================
%  Lingua & codifica
%========================
\usepackage[T1]{fontenc} 
\usepackage{textcomp} 	
\usepackage[utf8x]{inputenc}
\usepackage[polutonikogreek,italian]{babel}
\usepackage{quoting}
%========================
% Font
%========================
\renewcommand{\rmdefault}{ppl}
\usepackage{mathpazo}
\usepackage[scaled=.95]{helvet}
\usepackage{eulervm}
%========================
% Tipografia
%========================
\usepackage[bindingoffset=6mm]{geometry}
\usepackage{multicol}
\usepackage{multirow}
\usepackage{indentfirst}
\usepackage{emptypage}
\usepackage{nonumonpart}
\usepackage{enumitem} % usare questo anche per le liste compatte 
%                           \begin{enumerate} [nosep] o [noitemsep]
% \usepackage{paralist} % va in conflitto con il pacchetto precedente
\usepackage{tabto}
\usepackage{microtype}
%========================
% Tabelle aggiunto da claudio
%========================
\usepackage{threeparttable}
%========================
% Matematica I
%========================
%\usepackage{amsmath} % aggiunto da daniele
\usepackage{amssymb}
\usepackage{amsthm}
\usepackage{cancel}
\usepackage[]{units}
\usepackage[np,noaddmissingzero]{numprint}
%========================
% Grafica
%========================
\usepackage[pdftex]{graphicx}
\usepackage{rotating}
\usepackage{shadow}
\usepackage{fancybox}
\usepackage{empheq}
\usepackage{framed}
\usepackage{wrapfig}
%========================
% pgf & TikZ
%========================
\usepackage{tikz}
\usepackage{pgfplots}
\usepgfplotslibrary{patchplots}
\pgfplotsset{compat=1.8}
\usepackage{tkz-euclide}
\usepackage{tkz-fct}
\usepackage{circuitikz}
\usepackage{tikz-qtree} % aggiunto da daniele
\usetkzobj{all}
%========================
% Librerie TikZ
%========================
\usetikzlibrary{arrows,%
                arrows.meta,
                through,
		automata,%
		backgrounds,%
		calc,%
		decorations.markings,%
		decorations.shapes,%
		decorations.text,% 
		decorations.pathreplacing,%
		fit,%
		matrix,%
		mindmap,%
		patterns,%
		positioning,%
		intersections,%aggiunto da claudio
		shapes,%
		shapes.geometric}
%========================
% Simboli
%========================
\usepackage{marvosym}
\usepackage{eurosym}
\usepackage{pifont}
% \usepackage{hiero}
%========================
% Matematica II
%========================
\usepackage{matc3}
%========================
% Collegamenti
%========================
\usepackage[colorlinks, hypertexnames=false]{hyperref}
\usepackage{bookmark}
%========================
% Per il coding
%========================
\usepackage{listings} 
%========================
% Per la geometria
%========================
% \usepackage{textgreek}
%========================
% Personalizzazioni
%========================
% %==============================%
% VARIABILI MATEMATICA DOLCE 1 %
%==============================%

%%
%% Dati del libro
%%
\newcommand{\editore}{Matematicamente.it}
\newcommand{\serie}{Matematica $C^3$}
\newcommand{\titolo}{Matematica dolce 1}	
\newcommand{\pdftitolo}{Matematica C3 - Matematica dolce 1}
\newcommand{\docvers}{\texttt{5.0}}
\newcommand{\edizione}{ 2016}
\newcommand{\Edizione}{2016 Edizione}
\newcommand{\tipo}{ (versione completa a colori)}
\newcommand{\descr}{Testo per il primo biennio \protect\\ della Scuola 
                    Secondaria di $II$ grado}
\newcommand{\oggi}{27 giugno 2016}
\newcommand{\mese}{giugno}
\newcommand{\anno}{2016}
\newcommand{\mcisbn}{9788896354681}

%%
%% Nomi di autori, collaboratori, etc
%%
\newcommand{\coord}{Daniele~Zambelli}
\newcommand{\autori}{Claudio~Carboncini, Antonio~Bernardo, 
Ubaldo~Pernigo, Erasmo~Modica, Anna~Cristina~Mocchetti,
Germano~Pettarin, Francesco~Daddi, Angela~D'Amato, Nicola~Chiriano, 
Daniele~Zambelli, Maria~Antonietta~Pollini}
\newcommand{\colab}{Laura~Todisco, Michela~Todeschi,
Nicola~De~Rosa, Paolo~Baggiani, Luca~Tedesco, Vittorio~Patriarca, 
Francesco~Speciale,
Alessandro~Paolino, Luciano~Sarra, Maria~Rosaria~Agrello, 
Alberto~Giuseppe~Brudaglio, Lucia~Rapella,
Francesca~Lorenzoni, Sara~Gobbato, Mauro~Paladini, Anna~Maria~Cavallo, 
Elena~Stante,
Giuseppe~Pipino, Silvia~Monatti, Andrea~Celia, Gemma~Fiorito, Dorotea~Jacona, 
Simone~Rea, Nicoletta~Passera, Pierluigi~Cunti, Francesco~Camia, 
Anna~Rita~Lorenzo, Alessandro~Castelli, Piero~Sbardellati, Luca~Frangella, 
Raffaele~Santoro, Alessandra~Marrata, Mario~Bochicchio, Angela~Iaciofano, 
Luca~Pieressa, Giovanni~Quagnano,
Elisabetta~Campana, Luciana~Formenti}
\newcommand{\texcol}{Claudio Carboncini, Silvia Cibola, Tiziana Manca,
Daniele~Zambelli}
%%%%%%% EOF

% \graphicspath{{img/}}
\setsecnumdepth{subsection} 
\maxtocdepth{subsection}
\setlength{\cftpartnumwidth}{2.25em}
\setlength{\shadeboxsep}{5pt} 
\setlength{\shadeboxrule}{.4pt} 
\setlength{\shadedtextwidth}{\textwidth}
\addtolength{\shadedtextwidth}{-2\shadeboxsep}
\addtolength{\shadedtextwidth}{-2\shadeboxrule}
\setlength{\shadeleftshift}{0pt}
\setlength{\shaderightshift}{0pt}
\linespread{1.05}
\captionnamefont{\small\scshape}
\captiontitlefont{\small}
\newcommand{\mail}[1]{\href{mailto:#1}{\texttt{#1}}}
\definecolor{grigio80}{gray}{0.8}
\definecolor{grigio70}{gray}{0.7}
\hypersetup{%
  pdffitwindow=true,%
  linkcolor=RoyalBlue,%	
%   linkcolor=Black,%	
  linktocpage=true,%
  filecolor=black,%
  urlcolor=RoyalBlue,%
%   urlcolor=Black,%
  plainpages=false,%
  pdftitle={\pdftitolo, \edizione \tipo},%
  pdfauthor=Dimitrios Vrettos,%
  pdfdisplaydoctitle=true%
}
\bookmarksetup{startatroot}
%======================== o l'una o l'altra
% Personalizzazione per matematica dolce
% per: postulato, proposizione, parte, capitolo, numnameref, tggg
% inacessibleblock
% %%%%%%%%%%%%%
% Claudoio Carboncini
%%%

\theoremstyle{plain}
% le seguenti due righe ora danno un errore penso vadano semplicemente tolte:
% \newshadetheorem{postulato}{\thmcolor{Postulato}}[chapter]
% \newshadetheorem{proposizione}{\thmcolor{Proposizione}}[chapter]

%%%%%%%%%%%%%
% Daniele Zambelli
%%%

% per 05_02_tarta
\usepackage{listings}             % Include the listings-package
%
% per mantenere allineati i riferimenti presenti
% negli esercizi e nelle soluzioni.
\newcommand{\numnameref}[1]{\ref{#1} \nameref{#1}}
%
% per disegnare il simbolo >>>
% in 05_02_tartaruga.
\newcommand{\tggg}[0]{\textgreater\textgreater\textgreater}

% Inizializza la ``variabile globale'' folder
\newcommand{\folder}{./}

% Crea una nuova parte
\newcommand{\parte}[2]{
  \renewcommand{\folder}{#1}
  \graphicspath{\folder}
  \include{\folder #2}
}

% Crea un nuovo capitolo
\newcommand{\capitolo}[2]{
  \renewcommand{\folder}{#1}
  \graphicspath{{\folder}}
  \include{\folder #2}
  \newpage
  \include{\folder #2_ese}
  \cleardoublepage
}

% Per contrassegnare e sostituire i blocchi inacessibili ai ciechi.
\newenvironment{inaccessibleblock}[1][]{}{}

% % \usepackage{enumerate}
% \usepackage{fancyhdr} % per intestazioni e pie di pagina
% \setlength{\columnseprule}{1pt}
% \def\columnseprulecolor{\color{black}}
% % \usepackage{widetable} % Per tabelle con larghezza fissa

%--------------------------
% Nuovi comandi di testo:

%--------------------------
% Insiemi numerici:
\newcommand{\N}{\mathbb{N}}
\newcommand{\Z}{\mathbb{Z}}
\newcommand{\Q}{\mathbb{Q}}
\newcommand{\A}{\mathbb{A}}
\newcommand{\R}{\mathbb{R}}
\newcommand{\I}{\mathbb{R}^*}
\newcommand{\C}{\mathbb{C}}
\newcommand{\K}{\mathbb{K}}

%--------------------------
% varianti di lettere greche:
\renewcommand{\epsilon}{\varepsilon}
\renewcommand{\theta}{\vartheta}
\renewcommand{\rho}{\varrho}
\renewcommand{\phi}{\varphi}

%--------------------------
% Delimitatori e parentesi:
\newcommand{\tonda}[1]{\left( #1 \right)}
\newcommand{\quadra}[1]{\left[ #1 \right]}
\newcommand{\graffa}[1]{\left \{ #1 \right \}}
\newcommand{\abs}[1]{\left \lvert #1 \right \rvert}
\newcommand{\modulo}[1]{\left| #1 \right|}

%--------------------------
% Sisteni, vettori, matrici:
\newcommand{\sistema}[1]{\left\{\begin{array}{lcl}#1\end{array}\right.}
\newcommand{\fcerotto}[1]{\left\{\begin{array}{lclcl}#1\end{array}\right.}
\newcommand{\vettore}[1]{\left(\begin{array}{c}#1\end{array}\right)}
\newcommand{\matrice}[2]{\tonda{\begin{array}{#1}#2\end{array}}}

%--------------------------
% sen invece che sin:
\DeclareMathOperator{\sen}{sen}
\DeclareMathOperator{\st}{st}

%--------------------------
% Punto date le coordinate:
\newcommand{\punto}[2]{\tonda{#1;~#2}}

%--------------------------
% Elenco numerato. Esempio di chiamata: 
% \elenconumerato{{$3+2$, $4 \cdot 5$, {con, VIRGOLE, $5^3$}}{\vspace{1cm}}}
\newcommand{\elenconumerato}[2]{%
\begin{enumerate}
 \foreach \x in #1 {\item \x #2}
\end{enumerate}
}

%--------------------------
% Centra un elemento utile nelle tabelle centra anche verticalmente: 
\newcommand{\centra}[1]{\begin{center}#1\end{center}}

% \newcommand{\piedipagina}[3]{% scrive il piedipagina
%   \def \sinistra{#1}
%   \def \centro{#2}
%   \def \destra{#3}
%   \pagestyle{fancy}
%   \renewcommand{\headrulewidth}{0pt}
%   \renewcommand{\footrulewidth}{0pt}
%   \lfoot{\scriptsize \sinistra}
%   \cfoot{\scriptsize \centro}
%   \rfoot{\scriptsize \destra}
% }

%========================
% Caratteri sans serif
%========================
% \renewcommand{\familydefault}{\sfdefault}
