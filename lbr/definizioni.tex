%%%%%%%%%%%%%
% Claudoio Carboncini
%%%

\theoremstyle{plain}
% le seguenti due righe ora danno un errore penso vadano semplicemente tolte:
% \newshadetheorem{postulato}{\thmcolor{Postulato}}[chapter]
% \newshadetheorem{proposizione}{\thmcolor{Proposizione}}[chapter]

%%%%%%%%%%%%%
% Daniele Zambelli
%%%

% per 05_02_tarta
\usepackage{listings}             % Include the listings-package
%
% per mantenere allineati i riferimenti presenti
% negli esercizi e nelle soluzioni.
\newcommand{\numnameref}[1]{\ref{#1} \nameref{#1}}
%
% per disegnare il simbolo >>>
% in 05_02_tartaruga.
\newcommand{\tggg}[0]{\textgreater\textgreater\textgreater}

% Inizializza la ``variabile globale'' folder
\newcommand{\folder}{./}

% Crea una nuova parte
\newcommand{\parte}[2]{
  \renewcommand{\folder}{#1}
  \graphicspath{\folder}
  \include{\folder #2}
}

% Crea un nuovo capitolo
\newcommand{\capitolo}[2]{
  \renewcommand{\folder}{#1}
  \graphicspath{{\folder}}
  \include{\folder #2}
  \newpage
  \include{\folder #2_ese}
  \cleardoublepage
}

% Per contrassegnare e sostituire i blocchi inacessibili ai ciechi.
\newenvironment{inaccessibleblock}[1][]{}{}
